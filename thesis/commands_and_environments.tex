\newcommand{\todo}[2][]{\textcolor{red}{TODO\ifthenelse{\equal{#1}{}}{}{[#1]}: #2}}
\newcommand{\done}[2][]{\textcolor{green!50!black}{DONE\ifthenelse{\equal{#1}{}}{}{[#1]}: #2}}
\newcommand{\remark}[2][]{\textcolor{red!70!yellow}{REMARK\ifthenelse{\equal{#1}{}}{}{[#1]}: #2}}

\newtheorem{definition}{Definition}[chapter]
\newtheorem{theorem}[definition]{Theorem}
\newtheorem{lemma}[definition]{Lemma}
\newtheorem{corollary}[definition]{Corollary}
\newtheorem{conjecture}[definition]{Conjecture}

\newcommand{\E}[1]{\mathbb{E}\left[ #1 \right]}
\newcommand{\naturals}{\mathbb{N}}

\newcommand{\p}[1]{Pr\left[#1\right]}
\newcommand{\alltasks}{{\mathbb T}}
\newcommand{\neededfor}{\rightarrow}
\WithSuffix\newcommand\neededfor*{\stackrel{*}{\rightarrow}}

\tikzstyle{task_cross}=[
    {path picture={ 
        \draw[black]
        (path picture bounding box.south east) -- 
        (path picture bounding box.north west) 
        (path picture bounding box.south west) -- 
        (path picture bounding box.north east);
      }
    }
]

\tikzstyle{task_scheduled}=[fill=white, draw=black, task_cross]

% \getwidthofnode will measure the width of the node given as its second
% parameter and store it into the first parameter.
\makeatletter
\newcommand\getwidthofnode[2]{%
    \pgfextractx{#1}{\pgfpointanchor{#2}{east}}%
    \pgfextractx{\pgf@xa}{\pgfpointanchor{#2}{west}}% \pgf@xa is a length defined by PGF for temporary storage. No need to create a new temporary length.
    \addtolength{#1}{-\pgf@xa}%
}
\makeatother

% profiles and stuff
\newcommand{\profile}[1]{\left\llbracket #1 \right\rrbracket}
%\newcommand{\profileones}[1]{\mathbb{1}^#1}
\newcommand{\profilerepeat}[2]{(#1)^{#2}}
\newcommand{\profileones}[1]{\profilerepeat{1}{#1}}

% stuff to draw diagrams levelwise
\newcommand{\leveltop}{0}
\newcommand{\leveltopI}{0}
\newcommand{\leveltopII}{0}
\newcommand{\leveltopIII}{0}
\newcommand{\leveltopIIII}{0}
\newcommand{\leveltopIIIII}{0}
\newcommand{\leveltopIIIIII}{0}
\newcommand{\leveltopIIIIIII}{0}
\newcommand{\leveltopIIIIIIII}{0}
\newcommand{\leveltopIIIIIIIII}{0}
\newcommand{\leveltopIIIIIIIIII}{0}
\newcommand{\leveltopIIIIIIIIIII}{0}
\newcommand{\leveltopIIIIIIIIIIII}{0}
\newcommand{\leveltopIIIIIIIIIIIII}{0}
\newcommand{\leveltopIIIIIIIIIIIIII}{0}
\newcommand{\leveltopIIIIIIIIIIIIIII}{0}
\newcommand{\leveltopIIIIIIIIIIIIIIII}{0}
\newcommand{\leveltopIIIIIIIIIIIIIIIII}{0}
\newcommand{\leveltopIIIIIIIIIIIIIIIIII}{0}
\newcommand{\leveltopIIIIIIIIIIIIIIIIIII}{0}
\newcommand{\leveltopIIIIIIIIIIIIIIIIIIII}{0}
\newcommand{\leveltopIIIIIIIIIIIIIIIIIIIII}{0}
\newcommand{\leveltopIIIIIIIIIIIIIIIIIIIIII}{0}
\newcommand{\leveltopIIIIIIIIIIIIIIIIIIIIIII}{0}
\newcommand{\leveltopIIIIIIIIIIIIIIIIIIIIIIII}{0}
\newcommand{\leveltopIIIIIIIIIIIIIIIIIIIIIIIII}{0}
\newcommand{\leveltopIIIIIIIIIIIIIIIIIIIIIIIIII}{0}
\newcommand{\leveltopIIIIIIIIIIIIIIIIIIIIIIIIIII}{0}
\newcommand{\leveltopIIIIIIIIIIIIIIIIIIIIIIIIIIII}{0}
\newcommand{\leveltopIIIIIIIIIIIIIIIIIIIIIIIIIIIII}{0}

%%% Local Variables:
%%% TeX-master: "thesis.tex"
%%% End: 