%%% thesis.tex --- 

%% Author: philipp@pmpc
%% Version: $Id: thesis.tex,v 0.0 2013/04/08 12:19:13 philipp Exp$

%%\revision$Header: /home/philipp/Documents/Uni/masterarbeit/thesis/thesis.tex,v 0.0 2013/04/08 12:19:13 philipp Exp$

\documentclass[letter]{report}

\usepackage[english]{babel}
\usepackage{lmodern}
\usepackage[utf8]{inputenc}
\usepackage{hyperref}
\usepackage{amsmath}
\usepackage{amssymb}
\usepackage{amsthm}
\usepackage{graphicx}
\usepackage{tikz}
\usepackage{suffix}
\usepackage{stmaryrd}
\usepackage[left=2.5cm,right=2.5cm]{geometry}

\usetikzlibrary{shapes.multipart,chains}
\usetikzlibrary{positioning}
\usetikzlibrary{matrix}
\usetikzlibrary{automata}
\usetikzlibrary{external} 

\newcommand{\todo}[1]{\textcolor{red}{TODO: #1}}

\newtheorem{definition}{Definition}[chapter]
\newtheorem{theorem}{Theorem}[chapter]
\newtheorem{conjecture}{Conjecture}[chapter]

\newcommand{\p}[1]{Pr\left[#1\right]}
\newcommand{\alltasks}{{\mathbb T}}
\newcommand{\neededfor}{\rightarrow}
\WithSuffix\newcommand\neededfor*{\stackrel{*}{\rightarrow}}

\tikzstyle{task_cross}=[
    {path picture={ 
        \draw[black]
        (path picture bounding box.south east) -- 
        (path picture bounding box.north west) 
        (path picture bounding box.south west) -- 
        (path picture bounding box.north east);
      }
    }
]

% \getwidthofnode will measure the width of the node given as its second
% parameter and store it into the first parameter.
\makeatletter
\newcommand\getwidthofnode[2]{%
    \pgfextractx{#1}{\pgfpointanchor{#2}{east}}%
    \pgfextractx{\pgf@xa}{\pgfpointanchor{#2}{west}}% \pgf@xa is a length defined by PGF for temporary storage. No need to create a new temporary length.
    \addtolength{#1}{-\pgf@xa}%
}
\makeatother

\begin{document}

% stuff to draw diagrams levelwise
\newcommand{\leveltop}{0}
\newcommand{\leveltopI}{0}
\newcommand{\leveltopII}{0}
\newcommand{\leveltopIII}{0}
\newcommand{\leveltopIIII}{0}
\newcommand{\leveltopIIIII}{0}
\newcommand{\leveltopIIIIII}{0}
\newcommand{\leveltopIIIIIII}{0}
\newcommand{\leveltopIIIIIIII}{0}
\newcommand{\leveltopIIIIIIIII}{0}
\newcommand{\leveltopIIIIIIIIII}{0}
\newcommand{\leveltopIIIIIIIIIII}{0}
\newcommand{\leveltopIIIIIIIIIIII}{0}
\newcommand{\leveltopIIIIIIIIIIIII}{0}
\newcommand{\leveltopIIIIIIIIIIIIII}{0}
\newcommand{\leveltopIIIIIIIIIIIIIII}{0}
\newcommand{\leveltopIIIIIIIIIIIIIIII}{0}
\newcommand{\leveltopIIIIIIIIIIIIIIIII}{0}
\newcommand{\leveltopIIIIIIIIIIIIIIIIII}{0}
\newcommand{\leveltopIIIIIIIIIIIIIIIIIII}{0}
\newcommand{\leveltopIIIIIIIIIIIIIIIIIIII}{0}
\newcommand{\leveltopIIIIIIIIIIIIIIIIIIIII}{0}
\newcommand{\leveltopIIIIIIIIIIIIIIIIIIIIII}{0}
\newcommand{\leveltopIIIIIIIIIIIIIIIIIIIIIII}{0}
\newcommand{\leveltopIIIIIIIIIIIIIIIIIIIIIIII}{0}
\newcommand{\leveltopIIIIIIIIIIIIIIIIIIIIIIIII}{0}
\newcommand{\leveltopIIIIIIIIIIIIIIIIIIIIIIIIII}{0}
\newcommand{\leveltopIIIIIIIIIIIIIIIIIIIIIIIIIII}{0}
\newcommand{\leveltopIIIIIIIIIIIIIIIIIIIIIIIIIIII}{0}
\newcommand{\leveltopIIIIIIIIIIIIIIIIIIIIIIIIIIIII}{0}

\chapter{Theoretical foundations}
\label{chap:theoretical-foundations}

\section{Probability theory}
\label{sec:some-probability}

As we will see later, we will often be using probability distributions (in particular continuous probability distributions).

\subsection{Exponential distribution}
\label{sec:exponential-distribution}

The well-known exponential distribution is the central distribution we are dealing with in this text. All definitions and theorems within this subsection are along the lines of \cite{schickinger2001diskrete}.

\begin{definition}
  A continuous random variable is \emph{exponentially distributed} with parameter $\lambda$ if it has density 
  \begin{equation*}
    f(x) =
    \begin{cases}
      \lambda \cdot e^{-\lambda x} & \text{ if } x \geq 0 
      \\ 0 & \text{ otherwise}
    \end{cases}
    .
  \end{equation*}
\end{definition}

Note that the above definition also determines the distribution function $F$ of an exponentially distributed random variable as follows:

\begin{equation*}
  F(x) =
  \begin{cases}
    1-e^{\lambda x} & \text{ if } x \geq 0 \\
    0 & \text{ otherwise}
  \end{cases}
\end{equation*}

\begin{theorem}
  Let $X$ be an exponentially distributed random variable. Then, the expectancy of $X$ is $\E{X} = \frac 1 \lambda$.
\end{theorem}

\begin{proof}
  We can compute the expectancy for $X$ as follows:
  \begin{eqnarray*}
    \E{X} 
    &=& \int_{-\infty}^\infty x\cdot f(x)\, dx = \\
    &=& \int_{0}^\infty x\cdot \lambda e^{-\lambda x}\, dx \\
    &=& \left[ - \frac{e^{-\lambda x}\cdot \left( \lambda x + 1 \right)}{\lambda} \right]_{0}^\infty \\
    &=& \frac{1}{\lambda}
  \end{eqnarray*}
\end{proof}

\begin{theorem}[Scalability]
  Let $X$ be an exponentially distributed random variable with parameter $\lambda$ and let $a\in \mathbb{R}^+$. Then, the random variable $aX$ is exponentially distributed with parameter $\frac{\lambda}{a}$.
\end{theorem}

\begin{proof}
  We compute the probability that the random variable $aX$ is less than $x$:
  \begin{eqnarray*}
    \p{aX \leq x} 
    &=& \p{X \leq \frac{x}{a}} = \\
    &=& 1 - e^{-\frac{\lambda}{a} \cdot x}
  \end{eqnarray*}
  This result is equivalent to the density function of an exponentially distributed random variable with parameter $\frac{\lambda}{a}$.
\end{proof}

\begin{theorem}
  Let $X_1,\dots,X_n$ be exponentially-distributed random variables with respective parameters $\lambda_1,\dots,\lambda_n$. Then, the ranvom variable $Z:=\min_{i\in\left\{ 1,\dots,n \right\}} \left\{ X_i \right\}$ is exponentially distributed with parameter $\lambda=\lambda_1+\dots+\lambda_n$.
\end{theorem}

\begin{proof}
  We prove the claim by induction. 

  Suppose, we have two exponentially distributed random variables $X_1$ resp. $X_2$ with parameters $\lambda_1$ resp. $\lambda_2$. We then can compute
  \begin{align*}
    \p{min\left\{ X_1,X_2 \right\} \geq x} & = \p{X_1 \geq x \wedge X_2 \geq x} = \\ 
    & = \p{X_1 \geq x}\cdot\p{X_2 \geq x} = \\
    & = e^{-\lambda_1 x} \cdot e^{-\lambda_2 x} = \\
    & = e^{-\lambda_1 x - \lambda_2 x} = \\
    & = e^{-\left( \lambda_1+\lambda_2 \right) \cdot x},
  \end{align*}
  from which we can conclude that $\min\left\{ X_1,X_2 \right\}$ is exponentially distributed with parameter $\lambda_1 + \lambda_2$. By induction, we obtain our claim.
\end{proof}

\begin{definition}[Memorylessness]
  A random variable $X$ is called \emph{memoryless} if
  \begin{equation*}
    \p{X>t+s \mid X>s} = \p{X > t}
  \end{equation*}
\end{definition}

\begin{theorem}
  \label{sec:exponential-memoryless}
  Let $X$ be an exponentially distributed random variable with parameter $\lambda$. Then, $X$ is memoryless.
\end{theorem}

\begin{proof}
  We start by using the definition of conditional probability and rewrite until we arrive at our goal:
  \begin{eqnarray*}
    \p{X>t+s \mid X>s} &=& \frac{\p{X>t+s \wedge X>s}}{\p{X>s}} = \\
    &=& \frac{\p{X>t+s}}{\p{X>s}} = \\
    &=& \frac{e^{-\lambda \cdot(t+s)}}{e^{-\lambda s}} = \\
    &=& e^{-\lambda t} = \\
    &=& \p{X>t}
  \end{eqnarray*}
\end{proof}

This is a very advantageous property that can be exploited in our considerations to follow.

\emph{Remark:} It can even be shown that any memoryless continuous random variable is exponentially distributed, but theorem \ref{sec:exponential-memoryless} is sufficient for our needs.

\subsection{Uniform distribution}
\label{sec:uniform-distribution}

\begin{definition}
  A continuous random variable is \emph{uniformly distributed} over the interval $\left[ a,b \right]$ if it has density
  \begin{equation*}
    f(x) =
    \begin{cases}
      \frac{1}{b-a} & \text{ if } x\in\left[ a,b \right] \\
      0 & \text{ otherwise}
    \end{cases}.
  \end{equation*}
\end{definition}

The density of a uniform random variable is thus given by
\begin{equation*}
  F(x) = \begin{cases}
    0 & \text{ if } x<a \\
    \frac{x-a}{b-a} & \text{ if } x\in\left[ a,b \right] \\
    1 & \text{ if } x>b
  \end{cases}.
\end{equation*}


\subsection{General stuff}
\label{sec:probability-misc}

\begin{theorem}
  Let $X_1,\dots,X_n$ be independent, identically distributed, continuous random variables and let $i\in\left\{ 1,\dots,n \right\}$. Then 
  \begin{equation}
    \label{eq:probability-that-cont-random-variable-is-smallest-out-of-iid-is-one-over-n}
    \p{X_i = \min_{j\in\left\{ 1,\dots,n \right\}}\left\{ X_j \right\}} = \frac{1}{n}.
  \end{equation}
\end{theorem}

\begin{proof}
  It is clear that 
  \begin{equation*}
    \p{X_1 = \min_{j\in\left\{ 1,\dots,n \right\}}\left\{ X_j \right\}} = \p{X_2 = \min_{j\in\left\{ 1,\dots,n \right\}}\left\{ X_j \right\}} = \dots = \p{X_n = \min_{j\in\left\{ 1,\dots,n \right\}}\left\{ X_j \right\}},
  \end{equation*}
  because all random variables $X_1$ through $X_n$ behave the same. Thus we can deduce equation (\ref{eq:probability-that-cont-random-variable-is-smallest-out-of-iid-is-one-over-n}).
\end{proof}

\section{Graph theory}
\label{sec:foundations-graph-theory}

As we will see later, we will constantly deal with \emph{intrees}. In this section we develop simple notation for such trees. We assume that the reader is familiar with the concept of undirected trees and develop our notation on top of the one for undirected trees. For a more detailed introduction on what trees are, see e.g. \cite{diestel2005graph}.

\begin{definition}[Intree]
  Let $I'$ be a undirected tree. Let $v$ be a vertex within $I'$. Let $I$ be the directed version of $I'$ in such a way that all edges are directed towards vertex $v$. Then we call $I$ an intree or a rooted tree with root $v$.

  If there is an edge $(t_1, t_2)$ in $I$, we call $t_1$ a (direct) predecessor of $t_2$ and $t_2$ a (direct) successor of $t_1$. If there is a path from $t_1$ to $t_2$, we simply speak of predecessor and successor.
\end{definition}

Throughout this book, if not stated otherwise, we use the terms intree and tree interchangeably, because we are dealing only with intrees.

Those intrees can naturally be used to describe dependencies between different tasks, as long as each task is the requirement for at most one other tas. Each vertex in an intree then represents a task and if $t_1$ is a direct (direct) predecessor of $t_2$, this means that the task associated with $t_1$ must be executed before the task associated with $t_2$.

We can draw such intrees in a very straightforward manner: We draw the root at the bottom and its direct predecessors one level above. For each predecessor, we inductively repeat this procedure to obtain a ``top-to-bottom-description'' of the tree.

Figure \ref{fig:intree-example-task-names-directed-edges} shows an intree, where tasks 8 is a requirement for task 6, which itself is -- like task 7, 8 and (indirectly) task 10 -- a requirement for task 3. This figure also illustrates the fact that -- in an intree -- each task is a direct requirement for \emph{at most} one other task.

However, we are mostly interested in the \emph{structure} of the tree, which is why we most of the time omit the labellings of the vertices (i.e. we omit the task names) and rely on a \emph{unlabelled} representation as shown in figure \ref{fig:intree-example-structure-version}.

\begin{figure}[t]
  \centering
  \begin{subfigure}{.45\textwidth}
    \centering{}
    \begin{tikzpicture}[scale=.6, anchor=south]
      \node[circle, scale=0.9, draw] (tid0) at (3,1.5){0};
      \node[circle, scale=0.9, draw] (tid1) at (2.25,3){1};
      \node[circle, scale=0.9, draw] (tid2) at (1.5,4.5){3};
      \node[circle, scale=0.9, draw] (tid7) at (0.15,6){6};
      \node[circle, scale=0.9, draw] (tid9) at (0.15,7.5){9};
      \draw[<-, thick](tid7) -- (tid9);
      \node[circle, scale=0.9, draw] (tid10) at (1.5,6){7};
      \draw[<-, thick](tid2) -- (tid7);
      \draw[<-, thick](tid2) -- (tid10);
      \node[circle, scale=0.9, draw] (tid3) at (3.9,4.5){4};
      \node[circle, scale=0.9, draw] (tid5) at (2.85,6){8};
      \node[circle, scale=0.9, draw] (tid6) at (2.85,7.5){\small 10};
      \draw[<-, thick](tid5) -- (tid6);
      \draw[<-, thick](tid2) -- (tid5);
      \draw[<-, thick](tid1) -- (tid2);
      \draw[<-, thick](tid1) -- (tid3);
      \node[circle, scale=0.9, draw] (tid4) at (5.25,3){2};
      \node[circle, scale=0.9, draw] (tid8) at (5.25,4.5){5};
      \draw[<-, thick](tid4) -- (tid8);
      \draw[<-, thick](tid0) -- (tid1);
      \draw[<-, thick](tid0) -- (tid4);
    \end{tikzpicture}
    \caption{Labelled version with vertex labels and edges drawn as arrows.}
    \label{fig:intree-example-task-names-directed-edges}
  \end{subfigure}
  \quad
  \begin{subfigure}{.45\textwidth}
    \centering{}
    \begin{tikzpicture}[scale=.6, anchor=south]
      \node[circle, scale=0.9, fill] (tid0) at (3,1.5){};
      \node[circle, scale=0.9, fill] (tid1) at (2.25,3){};
      \node[circle, scale=0.9, fill] (tid2) at (1.5,4.5){};
      \node[circle, scale=0.9, fill] (tid7) at (0.15,6){};
      \node[circle, scale=0.9, fill] (tid9) at (0.15,7.5){};
      \draw[](tid7) -- (tid9);
      \node[circle, scale=0.9, fill] (tid10) at (1.5,6){};
      \draw[](tid2) -- (tid7);
      \draw[](tid2) -- (tid10);
      \node[circle, scale=0.9, fill] (tid3) at (3.9,4.5){};
      \node[circle, scale=0.9, fill] (tid5) at (2.85,6){};
      \node[circle, scale=0.9, fill] (tid6) at (2.85,7.5){};
      \draw[](tid5) -- (tid6);
      \draw[](tid2) -- (tid5);
      \draw[](tid1) -- (tid2);
      \draw[](tid1) -- (tid3);
      \node[circle, scale=0.9, fill] (tid4) at (5.25,3){};
      \node[circle, scale=0.9, fill] (tid8) at (5.25,4.5){};
      \draw[](tid4) -- (tid8);
      \draw[](tid0) -- (tid1);
      \draw[](tid0) -- (tid4);
      % level separators
      \draw[dashed] (0, 2.6) -- +(10, 0) node[below left, yshift=-.125cm]{Level 0};
      \draw[dashed] (0, 4.1) -- +(10, 0) node[below left, yshift=-.125cm]{Level 1};
      \draw[dashed] (0, 5.6) -- +(10, 0) node[below left, yshift=-.125cm]{Level 2};
      \draw[dashed] (0, 7.1) -- +(10, 0) node[below left, yshift=-.125cm]{Level 3};
      \draw[      ] (0, 8.6)    +(10, 0) node[below left, yshift=-.125cm]{Level 4};
    \end{tikzpicture}
    \caption{Unlabelled without arrows, edges are implicitly towards the root.}
    \label{fig:intree-example-structure-version}
  \end{subfigure}
  \caption{Graphical representation of an intree with 5 levels (numbered 0 to 4). All edges are implicitly directed towards the root, which is drawn at the bottom of the tree. Most of the time, the \emph{structure} of the tree is enough, so we will omit vertex names most of the time.}
  \label{fig:intrees-introductory-explanation}
\end{figure}

\begin{definition}[Level]
  Let $I$ be an intree. Let $v$ be a vertex within $I$. We define $level(v)$ be number of edges along the (unique) path from $v$ to the root.
\end{definition}

The concept of levels is illustrated in figure \ref{fig:intrees-introductory-explanation}.

\todo{Anzahl der rooted trees angeben.}

%%% Local Variables:
%%% TeX-master: "../thesis.tex"
%%% End: 

\chapter{First thoughts on implementation}
\label{chap:first-thoughts-on-implementation}

\section{Configuration DAG}
\label{sec:configuration-dag}

Initially we are dealing with an intree (i.e. each node has at most one successor) of tasks that have to be processed by a certain number of processors.

We will call the set of \emph{all} tasks $\alltasks$. If task $t_2$ can only be executed if $t_1$ already has been processed, we write $t_1 \neededfor t_2$. Moreover, we introduce a shorthand notation that allows us to ``chain'' several of these symbols: If there exist tasks $s_1,\dots,s_m$ ($m\in\mathbb N$), we write $t_1 \neededfor* t_2$ if we have $t_1 \neededfor s_1$ and $s_1 \neededfor s_2, s_2 \neededfor s_3, \dots, s_{m-1} \neededfor s_m$ and $s_m \neededfor t_2$ or if $t_1\neededfor t_2$ or if $t_1=t_2$.

\begin{definition}
  Let $\alltasks$ be a set of tasks, and $T \subseteq \alltasks$. We call $T$ an intree (of tasks) if there is one designated task $t_0\in\alltasks$ such that the following two conditions hold:
  \begin{eqnarray*}
    \forall  t \in T. & \quad t \neededfor* t_0 \\
    \forall  t \in T. & \quad t\neededfor s \Rightarrow s\in T
  \end{eqnarray*}
\end{definition}

\begin{definition}
  Let $T$ be an intree of tasks. Let $M\subseteq\alltasks$ be a set of tasks such that the following two conditions hold:
  \begin{itemize}
  \item $\forall t\in M.\, t \in T$
  \item $\forall t\in M.\, \nexists u \in T.\, u\neededfor t $
  \end{itemize}
  We then call the tuple $\left( T, M \right)$ a \emph{configuration}.
\end{definition}

\chapter{Drawing trees}
\label{chap:drawing-trees}

This is just a test chapter to see how we can draw the intrees and the snapshot DAGs.
\newsavebox{\nodebox}

\section{P2: A whole intree-DAG and its condensed counterpart}

\renewcommand{\leveltopI}{-15cm + \leveltop}
\renewcommand{\leveltopII}{-15cm + \leveltopI}
\renewcommand{\leveltopIII}{-15cm + \leveltopII}
\renewcommand{\leveltopIIII}{-15cm + \leveltopIII}
\renewcommand{\leveltopIIIII}{-15cm + \leveltopIIII}
\renewcommand{\leveltopIIIIII}{-15cm + \leveltopIIIII}
\renewcommand{\leveltopIIIIIII}{-15cm + \leveltopIIIIII}
\renewcommand{\leveltopIIIIIIII}{-15cm + \leveltopIIIIIII}
\renewcommand{\leveltopIIIIIIIII}{-15cm + \leveltopIIIIIIII}
\renewcommand{\leveltopIIIIIIIIII}{-15cm + \leveltopIIIIIIIII}
\renewcommand{\leveltopIIIIIIIIIII}{-15cm + \leveltopIIIIIIIIII}
\begin{tikzpicture}[scale=.2, anchor=south]
  \begin{scope}[yshift=\leveltopI cm]
    \matrix (line1) [column sep=1cm] {
      \node[draw=black, rectangle split,  rectangle split parts=3] (sn0x104f980){
        \begin{tikzpicture}[scale=.2]
          \node[circle, scale=0.75, fill] (tid0) at (3.75,1.5){};
          \node[circle, scale=0.75, fill] (tid1) at (2.25,3){};
          \node[circle, scale=0.75, fill] (tid3) at (0.75,4.5){};
          \node[circle, scale=0.75, fill] (tid7) at (0.75,6){};
          \draw[](tid3) -- (tid7);
          \node[circle, scale=0.75, fill] (tid4) at (2.25,4.5){};
          \node[circle, scale=0.75, fill] (tid5) at (3.75,4.5){};
          \draw[](tid1) -- (tid3);
          \draw[](tid1) -- (tid4);
          \draw[](tid1) -- (tid5);
          \node[circle, scale=0.75, fill] (tid2) at (6,3){};
          \node[circle, scale=0.75, fill] (tid6) at (6,4.5){};
          \node[circle, scale=0.75, fill] (tid8) at (5.25,6){};
          \node[circle, scale=0.75, fill, red] (tid10) at (5.25,7.5){};
          \draw[](tid8) -- (tid10);
          \node[circle, scale=0.75, fill, red] (tid9) at (6.75,6){};
          \draw[](tid6) -- (tid8);
          \draw[](tid6) -- (tid9);
          \draw[](tid2) -- (tid6);
          \draw[](tid0) -- (tid1);
          \draw[](tid0) -- (tid2);
        \end{tikzpicture}
        \nodepart{two}
        \footnotesize{6.82812}
        \nodepart{three}
        \footnotesize{$50\:25\:25$}
      };
      & 
      \\
    };
  \end{scope}
  \begin{scope}[yshift=\leveltopII cm]
    \matrix (line2) [column sep=1cm] {
      \node[draw=black, rectangle split,  rectangle split parts=3] (sn0x1050190){
        \begin{tikzpicture}[scale=.2]
          \node[circle, scale=0.75, fill] (tid0) at (3,1.5){};
          \node[circle, scale=0.75, fill] (tid1) at (2.25,3){};
          \node[circle, scale=0.75, fill] (tid3) at (0.75,4.5){};
          \node[circle, scale=0.75, fill, red] (tid7) at (0.75,6){};
          \draw[](tid3) -- (tid7);
          \node[circle, scale=0.75, fill] (tid4) at (2.25,4.5){};
          \node[circle, scale=0.75, fill] (tid5) at (3.75,4.5){};
          \draw[](tid1) -- (tid3);
          \draw[](tid1) -- (tid4);
          \draw[](tid1) -- (tid5);
          \node[circle, scale=0.75, fill] (tid2) at (5.25,3){};
          \node[circle, scale=0.75, fill] (tid6) at (5.25,4.5){};
          \node[circle, scale=0.75, fill] (tid8) at (5.25,6){};
          \node[circle, scale=0.75, fill, red] (tid9) at (5.25,7.5){};
          \draw[](tid8) -- (tid9);
          \draw[](tid6) -- (tid8);
          \draw[](tid2) -- (tid6);
          \draw[](tid0) -- (tid1);
          \draw[](tid0) -- (tid2);
        \end{tikzpicture}
        \nodepart{two}
        \footnotesize{6.35938}
        \nodepart{three}
        \footnotesize{$50\:50$}
      };
      & 
      \node[draw=black, rectangle split,  rectangle split parts=3] (sn0x104cb60){
        \begin{tikzpicture}[scale=.2]
          \node[circle, scale=0.75, fill] (tid0) at (3.75,1.5){};
          \node[circle, scale=0.75, fill] (tid1) at (2.25,3){};
          \node[circle, scale=0.75, fill] (tid3) at (0.75,4.5){};
          \node[circle, scale=0.75, fill, red] (tid7) at (0.75,6){};
          \draw[](tid3) -- (tid7);
          \node[circle, scale=0.75, fill] (tid4) at (2.25,4.5){};
          \node[circle, scale=0.75, fill] (tid5) at (3.75,4.5){};
          \draw[](tid1) -- (tid3);
          \draw[](tid1) -- (tid4);
          \draw[](tid1) -- (tid5);
          \node[circle, scale=0.75, fill] (tid2) at (6,3){};
          \node[circle, scale=0.75, fill] (tid6) at (6,4.5){};
          \node[circle, scale=0.75, fill, red] (tid8) at (5.25,6){};
          \node[circle, scale=0.75, fill] (tid9) at (6.75,6){};
          \draw[](tid6) -- (tid8);
          \draw[](tid6) -- (tid9);
          \draw[](tid2) -- (tid6);
          \draw[](tid0) -- (tid1);
          \draw[](tid0) -- (tid2);
        \end{tikzpicture}
        \nodepart{two}
        \footnotesize{6.29688}
        \nodepart{three}
        \footnotesize{$50\:50$}
      };
      & 
      \node[draw=black, rectangle split,  rectangle split parts=3] (sn0x104dd20){
        \begin{tikzpicture}[scale=.2]
          \node[circle, scale=0.75, fill] (tid0) at (3.75,1.5){};
          \node[circle, scale=0.75, fill] (tid1) at (2.25,3){};
          \node[circle, scale=0.75, fill] (tid3) at (0.75,4.5){};
          \node[circle, scale=0.75, fill] (tid7) at (0.75,6){};
          \draw[](tid3) -- (tid7);
          \node[circle, scale=0.75, fill] (tid4) at (2.25,4.5){};
          \node[circle, scale=0.75, fill] (tid5) at (3.75,4.5){};
          \draw[](tid1) -- (tid3);
          \draw[](tid1) -- (tid4);
          \draw[](tid1) -- (tid5);
          \node[circle, scale=0.75, fill] (tid2) at (6,3){};
          \node[circle, scale=0.75, fill] (tid6) at (6,4.5){};
          \node[circle, scale=0.75, fill, red] (tid8) at (5.25,6){};
          \node[circle, scale=0.75, fill, red] (tid9) at (6.75,6){};
          \draw[](tid6) -- (tid8);
          \draw[](tid6) -- (tid9);
          \draw[](tid2) -- (tid6);
          \draw[](tid0) -- (tid1);
          \draw[](tid0) -- (tid2);
        \end{tikzpicture}
        \nodepart{two}
        \footnotesize{6.29688}
        \nodepart{three}
        \footnotesize{$1$}
      };
      & 
      \\
    };
  \end{scope}
  \begin{scope}[yshift=\leveltopIII cm]
    \matrix (line3) [column sep=1cm] {
      \node[draw=black, rectangle split,  rectangle split parts=3] (sn0x10519d0){
        \begin{tikzpicture}[scale=.2]
          \node[circle, scale=0.75, fill] (tid0) at (3,1.5){};
          \node[circle, scale=0.75, fill] (tid1) at (2.25,3){};
          \node[circle, scale=0.75, fill, red] (tid3) at (0.75,4.5){};
          \node[circle, scale=0.75, fill] (tid4) at (2.25,4.5){};
          \node[circle, scale=0.75, fill] (tid5) at (3.75,4.5){};
          \draw[](tid1) -- (tid3);
          \draw[](tid1) -- (tid4);
          \draw[](tid1) -- (tid5);
          \node[circle, scale=0.75, fill] (tid2) at (5.25,3){};
          \node[circle, scale=0.75, fill] (tid6) at (5.25,4.5){};
          \node[circle, scale=0.75, fill] (tid7) at (5.25,6){};
          \node[circle, scale=0.75, fill, red] (tid8) at (5.25,7.5){};
          \draw[](tid7) -- (tid8);
          \draw[](tid6) -- (tid7);
          \draw[](tid2) -- (tid6);
          \draw[](tid0) -- (tid1);
          \draw[](tid0) -- (tid2);
        \end{tikzpicture}
        \nodepart{two}
        \footnotesize{5.92188}
        \nodepart{three}
        \footnotesize{$50\:50$}
      };
      & 
      \node[draw=black, rectangle split,  rectangle split parts=3] (sn0x104fbd0){
        \begin{tikzpicture}[scale=.2]
          \node[circle, scale=0.75, fill] (tid0) at (3,1.5){};
          \node[circle, scale=0.75, fill] (tid1) at (2.25,3){};
          \node[circle, scale=0.75, fill] (tid3) at (0.75,4.5){};
          \node[circle, scale=0.75, fill, red] (tid7) at (0.75,6){};
          \draw[](tid3) -- (tid7);
          \node[circle, scale=0.75, fill] (tid4) at (2.25,4.5){};
          \node[circle, scale=0.75, fill] (tid5) at (3.75,4.5){};
          \draw[](tid1) -- (tid3);
          \draw[](tid1) -- (tid4);
          \draw[](tid1) -- (tid5);
          \node[circle, scale=0.75, fill] (tid2) at (5.25,3){};
          \node[circle, scale=0.75, fill] (tid6) at (5.25,4.5){};
          \node[circle, scale=0.75, fill, red] (tid8) at (5.25,6){};
          \draw[](tid6) -- (tid8);
          \draw[](tid2) -- (tid6);
          \draw[](tid0) -- (tid1);
          \draw[](tid0) -- (tid2);
        \end{tikzpicture}
        \nodepart{two}
        \footnotesize{5.79688}
        \nodepart{three}
        \footnotesize{$50\:33\:17$}
      };
      & 
      \node[draw=black, rectangle split,  rectangle split parts=3] (sn0x105a080){
        \begin{tikzpicture}[scale=.2]
          \node[circle, scale=0.75, fill] (tid0) at (3.75,1.5){};
          \node[circle, scale=0.75, fill] (tid1) at (2.25,3){};
          \node[circle, scale=0.75, fill] (tid3) at (0.75,4.5){};
          \node[circle, scale=0.75, fill] (tid4) at (2.25,4.5){};
          \node[circle, scale=0.75, fill] (tid5) at (3.75,4.5){};
          \draw[](tid1) -- (tid3);
          \draw[](tid1) -- (tid4);
          \draw[](tid1) -- (tid5);
          \node[circle, scale=0.75, fill] (tid2) at (6,3){};
          \node[circle, scale=0.75, fill] (tid6) at (6,4.5){};
          \node[circle, scale=0.75, fill, red] (tid7) at (5.25,6){};
          \node[circle, scale=0.75, fill, red] (tid8) at (6.75,6){};
          \draw[](tid6) -- (tid7);
          \draw[](tid6) -- (tid8);
          \draw[](tid2) -- (tid6);
          \draw[](tid0) -- (tid1);
          \draw[](tid0) -- (tid2);
        \end{tikzpicture}
        \nodepart{two}
        \footnotesize{5.79688}
        \nodepart{three}
        \footnotesize{$1$}
      };
      & 
      \\
    };
  \end{scope}
  \begin{scope}[yshift=\leveltopIIII cm]
    \matrix (line4) [column sep=1cm] {
      \node[draw=black, rectangle split,  rectangle split parts=3] (sn0x1052250){
        \begin{tikzpicture}[scale=.2]
          \node[circle, scale=0.75, fill] (tid0) at (2.25,1.5){};
          \node[circle, scale=0.75, fill] (tid1) at (0.75,3){};
          \node[circle, scale=0.75, fill] (tid3) at (0.75,4.5){};
          \node[circle, scale=0.75, fill] (tid6) at (0.75,6){};
          \node[circle, scale=0.75, fill, red] (tid7) at (0.75,7.5){};
          \draw[](tid6) -- (tid7);
          \draw[](tid3) -- (tid6);
          \draw[](tid1) -- (tid3);
          \node[circle, scale=0.75, fill] (tid2) at (3,3){};
          \node[circle, scale=0.75, fill, red] (tid4) at (2.25,4.5){};
          \node[circle, scale=0.75, fill] (tid5) at (3.75,4.5){};
          \draw[](tid2) -- (tid4);
          \draw[](tid2) -- (tid5);
          \draw[](tid0) -- (tid1);
          \draw[](tid0) -- (tid2);
        \end{tikzpicture}
        \nodepart{two}
        \footnotesize{5.54688}
        \nodepart{three}
        \footnotesize{$50\:50$}
      };
      & 
      \node[draw=black, rectangle split,  rectangle split parts=3] (sn0x1052960){
        \begin{tikzpicture}[scale=.2]
          \node[circle, scale=0.75, fill] (tid0) at (3,1.5){};
          \node[circle, scale=0.75, fill] (tid1) at (2.25,3){};
          \node[circle, scale=0.75, fill, red] (tid3) at (0.75,4.5){};
          \node[circle, scale=0.75, fill] (tid4) at (2.25,4.5){};
          \node[circle, scale=0.75, fill] (tid5) at (3.75,4.5){};
          \draw[](tid1) -- (tid3);
          \draw[](tid1) -- (tid4);
          \draw[](tid1) -- (tid5);
          \node[circle, scale=0.75, fill] (tid2) at (5.25,3){};
          \node[circle, scale=0.75, fill] (tid6) at (5.25,4.5){};
          \node[circle, scale=0.75, fill, red] (tid7) at (5.25,6){};
          \draw[](tid6) -- (tid7);
          \draw[](tid2) -- (tid6);
          \draw[](tid0) -- (tid1);
          \draw[](tid0) -- (tid2);
        \end{tikzpicture}
        \nodepart{two}
        \footnotesize{5.29688}
        \nodepart{three}
        \footnotesize{$50\:33\:17$}
      };
      & 
      \node[draw=black, rectangle split,  rectangle split parts=3] (sn0x10581e0){
        \begin{tikzpicture}[scale=.2]
          \node[circle, scale=0.75, fill] (tid0) at (3,1.5){};
          \node[circle, scale=0.75, fill] (tid1) at (2.25,3){};
          \node[circle, scale=0.75, fill] (tid3) at (0.75,4.5){};
          \node[circle, scale=0.75, fill, red] (tid7) at (0.75,6){};
          \draw[](tid3) -- (tid7);
          \node[circle, scale=0.75, fill, red] (tid4) at (2.25,4.5){};
          \node[circle, scale=0.75, fill] (tid5) at (3.75,4.5){};
          \draw[](tid1) -- (tid3);
          \draw[](tid1) -- (tid4);
          \draw[](tid1) -- (tid5);
          \node[circle, scale=0.75, fill] (tid2) at (5.25,3){};
          \node[circle, scale=0.75, fill] (tid6) at (5.25,4.5){};
          \draw[](tid2) -- (tid6);
          \draw[](tid0) -- (tid1);
          \draw[](tid0) -- (tid2);
        \end{tikzpicture}
        \nodepart{two}
        \footnotesize{5.29688}
        \nodepart{three}
        \footnotesize{$33\:17\:25\:25$}
      };
      & 
      \node[draw=black, rectangle split,  rectangle split parts=3] (sn0x1058550){
        \begin{tikzpicture}[scale=.2]
          \node[circle, scale=0.75, fill] (tid0) at (3,1.5){};
          \node[circle, scale=0.75, fill] (tid1) at (2.25,3){};
          \node[circle, scale=0.75, fill] (tid3) at (0.75,4.5){};
          \node[circle, scale=0.75, fill, red] (tid7) at (0.75,6){};
          \draw[](tid3) -- (tid7);
          \node[circle, scale=0.75, fill] (tid4) at (2.25,4.5){};
          \node[circle, scale=0.75, fill] (tid5) at (3.75,4.5){};
          \draw[](tid1) -- (tid3);
          \draw[](tid1) -- (tid4);
          \draw[](tid1) -- (tid5);
          \node[circle, scale=0.75, fill] (tid2) at (5.25,3){};
          \node[circle, scale=0.75, fill, red] (tid6) at (5.25,4.5){};
          \draw[](tid2) -- (tid6);
          \draw[](tid0) -- (tid1);
          \draw[](tid0) -- (tid2);
        \end{tikzpicture}
        \nodepart{two}
        \footnotesize{5.29688}
        \nodepart{three}
        \footnotesize{$50\:50$}
      };
      & 
      \\
    };
  \end{scope}
  \begin{scope}[yshift=\leveltopIIIII cm]
    \matrix (line5) [column sep=1cm] {
      \node[draw=black, rectangle split,  rectangle split parts=3] (sn0x10525f0){
        \begin{tikzpicture}[scale=.2]
          \node[circle, scale=0.75, fill] (tid0) at (1.5,1.5){};
          \node[circle, scale=0.75, fill] (tid1) at (0.75,3){};
          \node[circle, scale=0.75, fill] (tid3) at (0.75,4.5){};
          \node[circle, scale=0.75, fill] (tid5) at (0.75,6){};
          \node[circle, scale=0.75, fill, red] (tid6) at (0.75,7.5){};
          \draw[](tid5) -- (tid6);
          \draw[](tid3) -- (tid5);
          \draw[](tid1) -- (tid3);
          \node[circle, scale=0.75, fill] (tid2) at (2.25,3){};
          \node[circle, scale=0.75, fill, red] (tid4) at (2.25,4.5){};
          \draw[](tid2) -- (tid4);
          \draw[](tid0) -- (tid1);
          \draw[](tid0) -- (tid2);
        \end{tikzpicture}
        \nodepart{two}
        \footnotesize{5.25}
        \nodepart{three}
        \footnotesize{$50\:50$}
      };
      & 
      \node[draw=black, rectangle split,  rectangle split parts=3] (sn0x1053850){
        \begin{tikzpicture}[scale=.2]
          \node[circle, scale=0.75, fill] (tid0) at (2.25,1.5){};
          \node[circle, scale=0.75, fill] (tid1) at (1.5,3){};
          \node[circle, scale=0.75, fill, red] (tid3) at (0.75,4.5){};
          \node[circle, scale=0.75, fill] (tid4) at (2.25,4.5){};
          \draw[](tid1) -- (tid3);
          \draw[](tid1) -- (tid4);
          \node[circle, scale=0.75, fill] (tid2) at (3.75,3){};
          \node[circle, scale=0.75, fill] (tid5) at (3.75,4.5){};
          \node[circle, scale=0.75, fill, red] (tid6) at (3.75,6){};
          \draw[](tid5) -- (tid6);
          \draw[](tid2) -- (tid5);
          \draw[](tid0) -- (tid1);
          \draw[](tid0) -- (tid2);
        \end{tikzpicture}
        \nodepart{two}
        \footnotesize{4.84375}
        \nodepart{three}
        \footnotesize{$50\:25\:25$}
      };
      & 
      \node[draw=black, rectangle split,  rectangle split parts=3] (sn0x1056b00){
        \begin{tikzpicture}[scale=.2]
          \node[circle, scale=0.75, fill] (tid0) at (3,1.5){};
          \node[circle, scale=0.75, fill] (tid1) at (2.25,3){};
          \node[circle, scale=0.75, fill, red] (tid3) at (0.75,4.5){};
          \node[circle, scale=0.75, fill, red] (tid4) at (2.25,4.5){};
          \node[circle, scale=0.75, fill] (tid5) at (3.75,4.5){};
          \draw[](tid1) -- (tid3);
          \draw[](tid1) -- (tid4);
          \draw[](tid1) -- (tid5);
          \node[circle, scale=0.75, fill] (tid2) at (5.25,3){};
          \node[circle, scale=0.75, fill] (tid6) at (5.25,4.5){};
          \draw[](tid2) -- (tid6);
          \draw[](tid0) -- (tid1);
          \draw[](tid0) -- (tid2);
        \end{tikzpicture}
        \nodepart{two}
        \footnotesize{4.75}
        \nodepart{three}
        \footnotesize{$50\:50$}
      };
      & 
      \node[draw=black, rectangle split,  rectangle split parts=3] (sn0x1056fb0){
        \begin{tikzpicture}[scale=.2]
          \node[circle, scale=0.75, fill] (tid0) at (3,1.5){};
          \node[circle, scale=0.75, fill] (tid1) at (2.25,3){};
          \node[circle, scale=0.75, fill, red] (tid3) at (0.75,4.5){};
          \node[circle, scale=0.75, fill] (tid4) at (2.25,4.5){};
          \node[circle, scale=0.75, fill] (tid5) at (3.75,4.5){};
          \draw[](tid1) -- (tid3);
          \draw[](tid1) -- (tid4);
          \draw[](tid1) -- (tid5);
          \node[circle, scale=0.75, fill] (tid2) at (5.25,3){};
          \node[circle, scale=0.75, fill, red] (tid6) at (5.25,4.5){};
          \draw[](tid2) -- (tid6);
          \draw[](tid0) -- (tid1);
          \draw[](tid0) -- (tid2);
        \end{tikzpicture}
        \nodepart{two}
        \footnotesize{4.75}
        \nodepart{three}
        \footnotesize{$50\:50$}
      };
      & 
      \node[draw=black, rectangle split,  rectangle split parts=3] (sn0x1058f50){
        \begin{tikzpicture}[scale=.2]
          \node[circle, scale=0.75, fill] (tid0) at (2.25,1.5){};
          \node[circle, scale=0.75, fill] (tid1) at (1.5,3){};
          \node[circle, scale=0.75, fill] (tid3) at (0.75,4.5){};
          \node[circle, scale=0.75, fill, red] (tid6) at (0.75,6){};
          \draw[](tid3) -- (tid6);
          \node[circle, scale=0.75, fill, red] (tid4) at (2.25,4.5){};
          \draw[](tid1) -- (tid3);
          \draw[](tid1) -- (tid4);
          \node[circle, scale=0.75, fill] (tid2) at (3.75,3){};
          \node[circle, scale=0.75, fill] (tid5) at (3.75,4.5){};
          \draw[](tid2) -- (tid5);
          \draw[](tid0) -- (tid1);
          \draw[](tid0) -- (tid2);
        \end{tikzpicture}
        \nodepart{two}
        \footnotesize{4.84375}
        \nodepart{three}
        \footnotesize{$50\:25\:25$}
      };
      & 
      \node[draw=black, rectangle split,  rectangle split parts=3] (sn0x1058a50){
        \begin{tikzpicture}[scale=.2]
          \node[circle, scale=0.75, fill] (tid0) at (2.25,1.5){};
          \node[circle, scale=0.75, fill] (tid1) at (1.5,3){};
          \node[circle, scale=0.75, fill] (tid3) at (0.75,4.5){};
          \node[circle, scale=0.75, fill, red] (tid6) at (0.75,6){};
          \draw[](tid3) -- (tid6);
          \node[circle, scale=0.75, fill] (tid4) at (2.25,4.5){};
          \draw[](tid1) -- (tid3);
          \draw[](tid1) -- (tid4);
          \node[circle, scale=0.75, fill] (tid2) at (3.75,3){};
          \node[circle, scale=0.75, fill, red] (tid5) at (3.75,4.5){};
          \draw[](tid2) -- (tid5);
          \draw[](tid0) -- (tid1);
          \draw[](tid0) -- (tid2);
        \end{tikzpicture}
        \nodepart{two}
        \footnotesize{4.84375}
        \nodepart{three}
        \footnotesize{$50\:50$}
      };
      & 
      \node[draw=black, rectangle split,  rectangle split parts=3] (sn0x10597a0){
        \begin{tikzpicture}[scale=.2]
          \node[circle, scale=0.75, fill] (tid0) at (3,1.5){};
          \node[circle, scale=0.75, fill] (tid1) at (2.25,3){};
          \node[circle, scale=0.75, fill] (tid3) at (0.75,4.5){};
          \node[circle, scale=0.75, fill, red] (tid6) at (0.75,6){};
          \draw[](tid3) -- (tid6);
          \node[circle, scale=0.75, fill, red] (tid4) at (2.25,4.5){};
          \node[circle, scale=0.75, fill] (tid5) at (3.75,4.5){};
          \draw[](tid1) -- (tid3);
          \draw[](tid1) -- (tid4);
          \draw[](tid1) -- (tid5);
          \node[circle, scale=0.75, fill] (tid2) at (5.25,3){};
          \draw[](tid0) -- (tid1);
          \draw[](tid0) -- (tid2);
        \end{tikzpicture}
        \nodepart{two}
        \footnotesize{4.84375}
        \nodepart{three}
        \footnotesize{$50\:50$}
      };
      & 
      \\
    };
  \end{scope}
  \begin{scope}[yshift=\leveltopIIIIII cm]
    \matrix (line6) [column sep=1cm] {
      \node[draw=black, rectangle split,  rectangle split parts=3] (sn0x1053920){
        \begin{tikzpicture}[scale=.2]
          \node[circle, scale=0.75, fill] (tid0) at (1.5,1.5){};
          \node[circle, scale=0.75, fill] (tid1) at (0.75,3){};
          \node[circle, scale=0.75, fill] (tid3) at (0.75,4.5){};
          \node[circle, scale=0.75, fill] (tid4) at (0.75,6){};
          \node[circle, scale=0.75, fill, red] (tid5) at (0.75,7.5){};
          \draw[](tid4) -- (tid5);
          \draw[](tid3) -- (tid4);
          \draw[](tid1) -- (tid3);
          \node[circle, scale=0.75, fill, red] (tid2) at (2.25,3){};
          \draw[](tid0) -- (tid1);
          \draw[](tid0) -- (tid2);
        \end{tikzpicture}
        \nodepart{two}
        \footnotesize{5.0625}
        \nodepart{three}
        \footnotesize{$50\:50$}
      };
      & 
      \node[draw=black, rectangle split,  rectangle split parts=3] (sn0x1053bc0){
        \begin{tikzpicture}[scale=.2]
          \node[circle, scale=0.75, fill] (tid0) at (1.5,1.5){};
          \node[circle, scale=0.75, fill] (tid1) at (0.75,3){};
          \node[circle, scale=0.75, fill] (tid3) at (0.75,4.5){};
          \node[circle, scale=0.75, fill, red] (tid5) at (0.75,6){};
          \draw[](tid3) -- (tid5);
          \draw[](tid1) -- (tid3);
          \node[circle, scale=0.75, fill] (tid2) at (2.25,3){};
          \node[circle, scale=0.75, fill, red] (tid4) at (2.25,4.5){};
          \draw[](tid2) -- (tid4);
          \draw[](tid0) -- (tid1);
          \draw[](tid0) -- (tid2);
        \end{tikzpicture}
        \nodepart{two}
        \footnotesize{4.4375}
        \nodepart{three}
        \footnotesize{$50\:50$}
      };
      & 
      \node[draw=black, rectangle split,  rectangle split parts=3] (sn0x1056090){
        \begin{tikzpicture}[scale=.2]
          \node[circle, scale=0.75, fill] (tid0) at (2.25,1.5){};
          \node[circle, scale=0.75, fill] (tid1) at (1.5,3){};
          \node[circle, scale=0.75, fill, red] (tid3) at (0.75,4.5){};
          \node[circle, scale=0.75, fill, red] (tid4) at (2.25,4.5){};
          \draw[](tid1) -- (tid3);
          \draw[](tid1) -- (tid4);
          \node[circle, scale=0.75, fill] (tid2) at (3.75,3){};
          \node[circle, scale=0.75, fill] (tid5) at (3.75,4.5){};
          \draw[](tid2) -- (tid5);
          \draw[](tid0) -- (tid1);
          \draw[](tid0) -- (tid2);
        \end{tikzpicture}
        \nodepart{two}
        \footnotesize{4.25}
        \nodepart{three}
        \footnotesize{$1$}
      };
      & 
      \node[draw=black, rectangle split,  rectangle split parts=3] (sn0x1056160){
        \begin{tikzpicture}[scale=.2]
          \node[circle, scale=0.75, fill] (tid0) at (2.25,1.5){};
          \node[circle, scale=0.75, fill] (tid1) at (1.5,3){};
          \node[circle, scale=0.75, fill, red] (tid3) at (0.75,4.5){};
          \node[circle, scale=0.75, fill] (tid4) at (2.25,4.5){};
          \draw[](tid1) -- (tid3);
          \draw[](tid1) -- (tid4);
          \node[circle, scale=0.75, fill] (tid2) at (3.75,3){};
          \node[circle, scale=0.75, fill, red] (tid5) at (3.75,4.5){};
          \draw[](tid2) -- (tid5);
          \draw[](tid0) -- (tid1);
          \draw[](tid0) -- (tid2);
        \end{tikzpicture}
        \nodepart{two}
        \footnotesize{4.25}
        \nodepart{three}
        \footnotesize{$50\:50$}
      };
      & 
      \node[draw=black, rectangle split,  rectangle split parts=3] (sn0x1057630){
        \begin{tikzpicture}[scale=.2]
          \node[circle, scale=0.75, fill] (tid0) at (3,1.5){};
          \node[circle, scale=0.75, fill] (tid1) at (2.25,3){};
          \node[circle, scale=0.75, fill, red] (tid3) at (0.75,4.5){};
          \node[circle, scale=0.75, fill, red] (tid4) at (2.25,4.5){};
          \node[circle, scale=0.75, fill] (tid5) at (3.75,4.5){};
          \draw[](tid1) -- (tid3);
          \draw[](tid1) -- (tid4);
          \draw[](tid1) -- (tid5);
          \node[circle, scale=0.75, fill] (tid2) at (5.25,3){};
          \draw[](tid0) -- (tid1);
          \draw[](tid0) -- (tid2);
        \end{tikzpicture}
        \nodepart{two}
        \footnotesize{4.25}
        \nodepart{three}
        \footnotesize{$1$}
      };
      & 
      \node[draw=black, rectangle split,  rectangle split parts=3] (sn0x1058b20){
        \begin{tikzpicture}[scale=.2]
          \node[circle, scale=0.75, fill] (tid0) at (2.25,1.5){};
          \node[circle, scale=0.75, fill] (tid1) at (1.5,3){};
          \node[circle, scale=0.75, fill] (tid3) at (0.75,4.5){};
          \node[circle, scale=0.75, fill, red] (tid5) at (0.75,6){};
          \draw[](tid3) -- (tid5);
          \node[circle, scale=0.75, fill, red] (tid4) at (2.25,4.5){};
          \draw[](tid1) -- (tid3);
          \draw[](tid1) -- (tid4);
          \node[circle, scale=0.75, fill] (tid2) at (3.75,3){};
          \draw[](tid0) -- (tid1);
          \draw[](tid0) -- (tid2);
        \end{tikzpicture}
        \nodepart{two}
        \footnotesize{4.4375}
        \nodepart{three}
        \footnotesize{$50\:50$}
      };
      & 
      \\
    };
  \end{scope}
  \begin{scope}[yshift=\leveltopIIIIIII cm]
    \matrix (line7) [column sep=1cm] {
      \node[draw=black, rectangle split,  rectangle split parts=3] (sn0x10540d0){
        \begin{tikzpicture}[scale=.2]
          \node[circle, scale=0.75, fill] (tid0) at (0.75,1.5){};
          \node[circle, scale=0.75, fill] (tid1) at (0.75,3){};
          \node[circle, scale=0.75, fill] (tid2) at (0.75,4.5){};
          \node[circle, scale=0.75, fill] (tid3) at (0.75,6){};
          \node[circle, scale=0.75, fill, red] (tid4) at (0.75,7.5){};
          \draw[](tid3) -- (tid4);
          \draw[](tid2) -- (tid3);
          \draw[](tid1) -- (tid2);
          \draw[](tid0) -- (tid1);
        \end{tikzpicture}
        \nodepart{two}
        \footnotesize{5}
        \nodepart{three}
        \footnotesize{$1$}
      };
      & 
      \node[draw=black, rectangle split,  rectangle split parts=3] (sn0x1054480){
        \begin{tikzpicture}[scale=.2]
          \node[circle, scale=0.75, fill] (tid0) at (1.5,1.5){};
          \node[circle, scale=0.75, fill] (tid1) at (0.75,3){};
          \node[circle, scale=0.75, fill] (tid3) at (0.75,4.5){};
          \node[circle, scale=0.75, fill, red] (tid4) at (0.75,6){};
          \draw[](tid3) -- (tid4);
          \draw[](tid1) -- (tid3);
          \node[circle, scale=0.75, fill, red] (tid2) at (2.25,3){};
          \draw[](tid0) -- (tid1);
          \draw[](tid0) -- (tid2);
        \end{tikzpicture}
        \nodepart{two}
        \footnotesize{4.125}
        \nodepart{three}
        \footnotesize{$50\:50$}
      };
      & 
      \node[draw=black, rectangle split,  rectangle split parts=3] (sn0x1055dd0){
        \begin{tikzpicture}[scale=.2]
          \node[circle, scale=0.75, fill] (tid0) at (1.5,1.5){};
          \node[circle, scale=0.75, fill] (tid1) at (0.75,3){};
          \node[circle, scale=0.75, fill, red] (tid3) at (0.75,4.5){};
          \draw[](tid1) -- (tid3);
          \node[circle, scale=0.75, fill] (tid2) at (2.25,3){};
          \node[circle, scale=0.75, fill, red] (tid4) at (2.25,4.5){};
          \draw[](tid2) -- (tid4);
          \draw[](tid0) -- (tid1);
          \draw[](tid0) -- (tid2);
        \end{tikzpicture}
        \nodepart{two}
        \footnotesize{3.75}
        \nodepart{three}
        \footnotesize{$1$}
      };
      & 
      \node[draw=black, rectangle split,  rectangle split parts=3] (sn0x10568c0){
        \begin{tikzpicture}[scale=.2]
          \node[circle, scale=0.75, fill] (tid0) at (2.25,1.5){};
          \node[circle, scale=0.75, fill] (tid1) at (1.5,3){};
          \node[circle, scale=0.75, fill, red] (tid3) at (0.75,4.5){};
          \node[circle, scale=0.75, fill, red] (tid4) at (2.25,4.5){};
          \draw[](tid1) -- (tid3);
          \draw[](tid1) -- (tid4);
          \node[circle, scale=0.75, fill] (tid2) at (3.75,3){};
          \draw[](tid0) -- (tid1);
          \draw[](tid0) -- (tid2);
        \end{tikzpicture}
        \nodepart{two}
        \footnotesize{3.75}
        \nodepart{three}
        \footnotesize{$1$}
      };
      & 
      \\
    };
  \end{scope}
  \begin{scope}[yshift=\leveltopIIIIIIII cm]
    \matrix (line8) [column sep=1cm] {
      \node[draw=black, rectangle split,  rectangle split parts=3] (sn0x1054550){
        \begin{tikzpicture}[scale=.2]
          \node[circle, scale=0.75, fill] (tid0) at (0.75,1.5){};
          \node[circle, scale=0.75, fill] (tid1) at (0.75,3){};
          \node[circle, scale=0.75, fill] (tid2) at (0.75,4.5){};
          \node[circle, scale=0.75, fill, red] (tid3) at (0.75,6){};
          \draw[](tid2) -- (tid3);
          \draw[](tid1) -- (tid2);
          \draw[](tid0) -- (tid1);
        \end{tikzpicture}
        \nodepart{two}
        \footnotesize{4}
        \nodepart{three}
        \footnotesize{$1$}
      };
      & 
      \node[draw=black, rectangle split,  rectangle split parts=3] (sn0x1055270){
        \begin{tikzpicture}[scale=.2]
          \node[circle, scale=0.75, fill] (tid0) at (1.5,1.5){};
          \node[circle, scale=0.75, fill] (tid1) at (0.75,3){};
          \node[circle, scale=0.75, fill, red] (tid3) at (0.75,4.5){};
          \draw[](tid1) -- (tid3);
          \node[circle, scale=0.75, fill, red] (tid2) at (2.25,3){};
          \draw[](tid0) -- (tid1);
          \draw[](tid0) -- (tid2);
        \end{tikzpicture}
        \nodepart{two}
        \footnotesize{3.25}
        \nodepart{three}
        \footnotesize{$50\:50$}
      };
      & 
      \\
    };
  \end{scope}
  \begin{scope}[yshift=\leveltopIIIIIIIII cm]
    \matrix (line9) [column sep=1cm] {
      \node[draw=black, rectangle split,  rectangle split parts=3] (sn0x1054a50){
        \begin{tikzpicture}[scale=.2]
          \node[circle, scale=0.75, fill] (tid0) at (0.75,1.5){};
          \node[circle, scale=0.75, fill] (tid1) at (0.75,3){};
          \node[circle, scale=0.75, fill, red] (tid2) at (0.75,4.5){};
          \draw[](tid1) -- (tid2);
          \draw[](tid0) -- (tid1);
        \end{tikzpicture}
        \nodepart{two}
        \footnotesize{3}
        \nodepart{three}
        \footnotesize{$1$}
      };
      & 
      \node[draw=black, rectangle split,  rectangle split parts=3] (sn0x1054cb0){
        \begin{tikzpicture}[scale=.2]
          \node[circle, scale=0.75, fill] (tid0) at (1.5,1.5){};
          \node[circle, scale=0.75, fill, red] (tid1) at (0.75,3){};
          \node[circle, scale=0.75, fill, red] (tid2) at (2.25,3){};
          \draw[](tid0) -- (tid1);
          \draw[](tid0) -- (tid2);
        \end{tikzpicture}
        \nodepart{two}
        \footnotesize{2.5}
        \nodepart{three}
        \footnotesize{$1$}
      };
      & 
      \\
    };
  \end{scope}
  \begin{scope}[yshift=\leveltopIIIIIIIIII cm]
    \matrix (line10) [column sep=1cm] {
      \node[draw=black, rectangle split,  rectangle split parts=3] (sn0x1054b20){
        \begin{tikzpicture}[scale=.2]
          \node[circle, scale=0.75, fill] (tid0) at (0.75,1.5){};
          \node[circle, scale=0.75, fill, red] (tid1) at (0.75,3){};
          \draw[](tid0) -- (tid1);
        \end{tikzpicture}
        \nodepart{two}
        \footnotesize{2}
        \nodepart{three}
        \footnotesize{$1$}
      };
      & 
      \\
    };
  \end{scope}
  \begin{scope}[yshift=\leveltopIIIIIIIIIII cm]
    \matrix (line11) [column sep=1cm] {
      \node[draw=black, rectangle split,  rectangle split parts=3] (sn0x10547e0){
        \begin{tikzpicture}[scale=.2]
          \node[circle, scale=0.75, fill, red] (tid0) at (0.75,1.5){};
        \end{tikzpicture}
        \nodepart{two}
        \footnotesize{1}
        \nodepart{three}
        \footnotesize{$$}
      };
      & 
      \\
    };
  \end{scope}
  \begin{scope}[yshift=\leveltopIIIIIIIIIIII cm]
    \matrix (line12) [column sep=1cm] {
      \\
    };
  \end{scope}
  \draw (sn0x104f980.south) -- (sn0x1050190.north);
  \draw (sn0x104f980.south) -- (sn0x104cb60.north);
  \draw (sn0x104f980.south) -- (sn0x104dd20.north);
  \draw (sn0x1050190.south) -- (sn0x10519d0.north);
  \draw (sn0x1050190.south) -- (sn0x104fbd0.north);
  \draw (sn0x104cb60.south) -- (sn0x105a080.north);
  \draw (sn0x104cb60.south) -- (sn0x104fbd0.north);
  \draw (sn0x104dd20.south) -- (sn0x104fbd0.north);
  \draw (sn0x10519d0.south) -- (sn0x1052250.north);
  \draw (sn0x10519d0.south) -- (sn0x1052960.north);
  \draw (sn0x104fbd0.south) -- (sn0x1052960.north);
  \draw (sn0x104fbd0.south) -- (sn0x10581e0.north);
  \draw (sn0x104fbd0.south) -- (sn0x1058550.north);
  \draw (sn0x105a080.south) -- (sn0x1052960.north);
  \draw (sn0x1052250.south) -- (sn0x10525f0.north);
  \draw (sn0x1052250.south) -- (sn0x1053850.north);
  \draw (sn0x1052960.south) -- (sn0x1053850.north);
  \draw (sn0x1052960.south) -- (sn0x1056b00.north);
  \draw (sn0x1052960.south) -- (sn0x1056fb0.north);
  \draw (sn0x10581e0.south) -- (sn0x1058f50.north);
  \draw (sn0x10581e0.south) -- (sn0x1058a50.north);
  \draw (sn0x10581e0.south) -- (sn0x1056b00.north);
  \draw (sn0x10581e0.south) -- (sn0x1056fb0.north);
  \draw (sn0x1058550.south) -- (sn0x10597a0.north);
  \draw (sn0x1058550.south) -- (sn0x1056fb0.north);
  \draw (sn0x10525f0.south) -- (sn0x1053920.north);
  \draw (sn0x10525f0.south) -- (sn0x1053bc0.north);
  \draw (sn0x1053850.south) -- (sn0x1053bc0.north);
  \draw (sn0x1053850.south) -- (sn0x1056090.north);
  \draw (sn0x1053850.south) -- (sn0x1056160.north);
  \draw (sn0x1056b00.south) -- (sn0x1056090.north);
  \draw (sn0x1056b00.south) -- (sn0x1056160.north);
  \draw (sn0x1056fb0.south) -- (sn0x1056160.north);
  \draw (sn0x1056fb0.south) -- (sn0x1057630.north);
  \draw (sn0x1058f50.south) -- (sn0x1053bc0.north);
  \draw (sn0x1058f50.south) -- (sn0x1056090.north);
  \draw (sn0x1058f50.south) -- (sn0x1056160.north);
  \draw (sn0x1058a50.south) -- (sn0x1058b20.north);
  \draw (sn0x1058a50.south) -- (sn0x1056160.north);
  \draw (sn0x10597a0.south) -- (sn0x1058b20.north);
  \draw (sn0x10597a0.south) -- (sn0x1057630.north);
  \draw (sn0x1053920.south) -- (sn0x10540d0.north);
  \draw (sn0x1053920.south) -- (sn0x1054480.north);
  \draw (sn0x1053bc0.south) -- (sn0x1054480.north);
  \draw (sn0x1053bc0.south) -- (sn0x1055dd0.north);
  \draw (sn0x1056090.south) -- (sn0x1055dd0.north);
  \draw (sn0x1056160.south) -- (sn0x1055dd0.north);
  \draw (sn0x1056160.south) -- (sn0x10568c0.north);
  \draw (sn0x1057630.south) -- (sn0x10568c0.north);
  \draw (sn0x1058b20.south) -- (sn0x1054480.north);
  \draw (sn0x1058b20.south) -- (sn0x10568c0.north);
  \draw (sn0x10540d0.south) -- (sn0x1054550.north);
  \draw (sn0x1054480.south) -- (sn0x1054550.north);
  \draw (sn0x1054480.south) -- (sn0x1055270.north);
  \draw (sn0x1055dd0.south) -- (sn0x1055270.north);
  \draw (sn0x10568c0.south) -- (sn0x1055270.north);
  \draw (sn0x1054550.south) -- (sn0x1054a50.north);
  \draw (sn0x1055270.south) -- (sn0x1054a50.north);
  \draw (sn0x1055270.south) -- (sn0x1054cb0.north);
  \draw (sn0x1054a50.south) -- (sn0x1054b20.north);
  \draw (sn0x1054cb0.south) -- (sn0x1054b20.north);
  \draw (sn0x1054b20.south) -- (sn0x10547e0.north);

  \newcommand{\nd}[4]{
    \node[draw=black, rectangle split, rectangle split parts=3] (n#1#2) {
      $#1/#2$
      \nodepart{two}
      #3
      \nodepart{three}
      #4
    };
  }

  \begin{scope}[yshift=\leveltopI, xshift=70cm, rectangle, draw=black,anchor=south]
    \matrix (test) [column sep=1cm] {
      \nd{5}{6}{6.82812}{50 50};
      \\
    };
  \end{scope}

  \begin{scope}[yshift=\leveltopII, xshift=70cm, rectangle, draw=black,anchor=south]
    \matrix (test) [column sep=1cm] {
      \nd{5}{5}{6.35938}{50 50};
      &
      \nd{4}{6}{6.29688}{1};
      \\
      };
    \end{scope}

    \begin{scope}[yshift=\leveltopIII, xshift=70cm, rectangle, draw=black,anchor=south]
      \matrix (test) [column sep=1cm] {
        \nd{5}{4}{5.92188}{50 50};
        &
        \nd{4}{5}{5.79688}{1};
        \\
      };
    \end{scope}

    \begin{scope}[yshift=\leveltopIIII, xshift=70cm, rectangle, draw=black,anchor=south]
      \matrix (test) [column sep=1cm] {
        \nd{5}{3}{5.54688}{50 50};
        &
        \nd{4}{4}{5.29688}{50 50};
        \\
      };
    \end{scope}

    \begin{scope}[yshift=\leveltopIIIII, xshift=70cm, rectangle, draw=black,anchor=south]
      \matrix (test) [column sep=1cm] {
        \nd{5}{2}{5.25}{50 50};
        &
        \nd{4}{3}{4.84375}{50 50};
        &
        \nd{3}{4}{5.29688}{1};
        \\
      };
    \end{scope}

    \begin{scope}[yshift=\leveltopIIIIII, xshift=70cm, rectangle, draw=black,anchor=south]
      \matrix (test) [column sep=1cm] {
        \nd{5}{1}{5.0625}{50 50};
        &
        \nd{4}{2}{4.4375}{50 50};
        &
        \nd{3}{3}{5.75}{1};
        \\
      };
    \end{scope}

    \begin{scope}[yshift=\leveltopIIIIIII, xshift=70cm, rectangle, draw=black,anchor=south]
      \matrix (test) [column sep=1cm] {
        \nd{5}{0}{5}{50 50};
        &
        \nd{4}{1}{4.125}{50 50};
        &
        \nd{3}{2}{5.25}{1};
        \\
      };
    \end{scope}
    
    \begin{scope}[yshift=\leveltopIIIIIIII, xshift=70cm, rectangle, draw=black,anchor=south]
      \matrix (test) [column sep=1cm] {
        \nd{4}{0}{4}{1};
        &
        \nd{3}{1}{3.25}{50 50};
        \\
      };
    \end{scope}

    \begin{scope}[yshift=\leveltopIIIIIIIII, xshift=70cm, rectangle, draw=black,anchor=south]
      \matrix (test) [column sep=1cm] {
        \nd{3}{0}{3}{1};
        &
        \nd{2}{1}{2.5}{1};
        \\
      };
    \end{scope}

    \begin{scope}[yshift=\leveltopIIIIIIIIII, xshift=70cm, rectangle, draw=black,anchor=south]
      \matrix (test) [column sep=1cm] {
        \nd{2}{0}{2}{1};
        \\
      };
    \end{scope}
    
    \begin{scope}[yshift=\leveltopIIIIIIIIIII, xshift=70cm, rectangle, draw=black,anchor=south]
      \matrix (test) [column sep=1cm] {
        \nd{1}{0}{1}{1};
        \\
      };
    \end{scope}

    \draw (n56.south) -- (n55.north);
    \draw (n56.south) -- (n46.north);
    \draw (n55.south) -- (n54.north);
    \draw (n55.south) -- (n45.north);
    \draw (n46.south) -- (n45.north);
    \draw (n54.south) -- (n53.north);
    \draw (n54.south) -- (n44.north);
    \draw (n45.south) -- (n44.north);
    \draw (n53.south) -- (n52.north);
    \draw (n53.south) -- (n43.north);
    \draw (n44.south) -- (n43.north);
    \draw (n44.south) -- (n34.north);
    \draw (n52.south) -- (n51.north);
    \draw (n52.south) -- (n42.north);
    \draw (n43.south) -- (n42.north);
    \draw (n43.south) -- (n33.north);
    \draw (n34.south) -- (n33.north);
    \draw (n51.south) -- (n50.north);
    \draw (n51.south) -- (n41.north);
    \draw (n42.south) -- (n41.north);
    \draw (n42.south) -- (n32.north);
    \draw (n33.south) -- (n32.north);
    \draw (n32.south) -- (n31.north);
    \draw (n50.south) -- (n40.north);
    \draw (n41.south) -- (n40.north);
    \draw (n41.south) -- (n31.north);
    \draw (n40.south) -- (n30.north);
    \draw (n31.south) -- (n30.north);
    \draw (n31.south) -- (n21.north);
    \draw (n30.south) -- (n20.north);
    \draw (n21.south) -- (n20.north);
    \draw (n20.south) -- (n10.north);

\end{tikzpicture}

%%% Local Variables:
%%% TeX-master: "thesis/thesis.tex"
%%% End: 
\renewcommand{\leveltopI}{-15cm + \leveltop}
\renewcommand{\leveltopII}{-15cm + \leveltopI}
\renewcommand{\leveltopIII}{-15cm + \leveltopII}
\renewcommand{\leveltopIIII}{-15cm + \leveltopIII}
\renewcommand{\leveltopIIIII}{-15cm + \leveltopIIII}
\renewcommand{\leveltopIIIIII}{-15cm + \leveltopIIIII}
\renewcommand{\leveltopIIIIIII}{-15cm + \leveltopIIIIII}
\renewcommand{\leveltopIIIIIIII}{-15cm + \leveltopIIIIIII}
\renewcommand{\leveltopIIIIIIIII}{-15cm + \leveltopIIIIIIII}
\renewcommand{\leveltopIIIIIIIIII}{-15cm + \leveltopIIIIIIIII}
\renewcommand{\leveltopIIIIIIIIIII}{-15cm + \leveltopIIIIIIIIII}
% \begin{tikzpicture}[scale=.2, anchor=south]
%   \begin{scope}[yshift=\leveltopI cm]
%     \matrix (line1) [column sep=1cm] {
%       \node[draw=black, rectangle split,  rectangle split parts=3] (sn0x1050af0){
%         \begin{tikzpicture}[scale=.2]
%           \node[circle, scale=0.75, fill] (tid0) at (3.75,1.5){};
%           \node[circle, scale=0.75, fill] (tid1) at (2.25,3){};
%           \node[circle, scale=0.75, fill] (tid3) at (0.75,4.5){};
%           \node[circle, scale=0.75, fill, red] (tid7) at (0.75,6){};
%           \draw[](tid3) -- (tid7);
%           \node[circle, scale=0.75, fill] (tid4) at (2.25,4.5){};
%           \node[circle, scale=0.75, fill] (tid5) at (3.75,4.5){};
%           \draw[](tid1) -- (tid3);
%           \draw[](tid1) -- (tid4);
%           \draw[](tid1) -- (tid5);
%           \node[circle, scale=0.75, fill] (tid2) at (6,3){};
%           \node[circle, scale=0.75, fill] (tid6) at (6,4.5){};
%           \node[circle, scale=0.75, fill] (tid8) at (5.25,6){};
%           \node[circle, scale=0.75, fill, red] (tid10) at (5.25,7.5){};
%           \draw[](tid8) -- (tid10);
%           \node[circle, scale=0.75, fill] (tid9) at (6.75,6){};
%           \draw[](tid6) -- (tid8);
%           \draw[](tid6) -- (tid9);
%           \draw[](tid2) -- (tid6);
%           \draw[](tid0) -- (tid1);
%           \draw[](tid0) -- (tid2);
%         \end{tikzpicture}
%         \nodepart{two}
%         \footnotesize{6.82812}
%         \nodepart{three}
%         \footnotesize{$50\:50$}
%       };
%       & 
%       \\
%     };
%   \end{scope}
%   \begin{scope}[yshift=\leveltopII cm]
%     \matrix (line2) [column sep=1cm] {
%       \node[draw=black, rectangle split,  rectangle split parts=3] (sn0x105a150){
%         \begin{tikzpicture}[scale=.2]
%           \node[circle, scale=0.75, fill] (tid0) at (3.75,1.5){};
%           \node[circle, scale=0.75, fill] (tid1) at (1.5,3){};
%           \node[circle, scale=0.75, fill] (tid3) at (1.5,4.5){};
%           \node[circle, scale=0.75, fill] (tid7) at (0.75,6){};
%           \node[circle, scale=0.75, fill, red] (tid9) at (0.75,7.5){};
%           \draw[](tid7) -- (tid9);
%           \node[circle, scale=0.75, fill, red] (tid8) at (2.25,6){};
%           \draw[](tid3) -- (tid7);
%           \draw[](tid3) -- (tid8);
%           \draw[](tid1) -- (tid3);
%           \node[circle, scale=0.75, fill] (tid2) at (5.25,3){};
%           \node[circle, scale=0.75, fill] (tid4) at (3.75,4.5){};
%           \node[circle, scale=0.75, fill] (tid5) at (5.25,4.5){};
%           \node[circle, scale=0.75, fill] (tid6) at (6.75,4.5){};
%           \draw[](tid2) -- (tid4);
%           \draw[](tid2) -- (tid5);
%           \draw[](tid2) -- (tid6);
%           \draw[](tid0) -- (tid1);
%           \draw[](tid0) -- (tid2);
%         \end{tikzpicture}
%         \nodepart{two}
%         \footnotesize{6.35938}
%         \nodepart{three}
%         \footnotesize{$50\:50$}
%       };
%       & 
%       \node[draw=black, rectangle split,  rectangle split parts=3] (sn0x104cb60){
%         \begin{tikzpicture}[scale=.2]
%           \node[circle, scale=0.75, fill] (tid0) at (3.75,1.5){};
%           \node[circle, scale=0.75, fill] (tid1) at (2.25,3){};
%           \node[circle, scale=0.75, fill] (tid3) at (0.75,4.5){};
%           \node[circle, scale=0.75, fill, red] (tid7) at (0.75,6){};
%           \draw[](tid3) -- (tid7);
%           \node[circle, scale=0.75, fill] (tid4) at (2.25,4.5){};
%           \node[circle, scale=0.75, fill] (tid5) at (3.75,4.5){};
%           \draw[](tid1) -- (tid3);
%           \draw[](tid1) -- (tid4);
%           \draw[](tid1) -- (tid5);
%           \node[circle, scale=0.75, fill] (tid2) at (6,3){};
%           \node[circle, scale=0.75, fill] (tid6) at (6,4.5){};
%           \node[circle, scale=0.75, fill, red] (tid8) at (5.25,6){};
%           \node[circle, scale=0.75, fill] (tid9) at (6.75,6){};
%           \draw[](tid6) -- (tid8);
%           \draw[](tid6) -- (tid9);
%           \draw[](tid2) -- (tid6);
%           \draw[](tid0) -- (tid1);
%           \draw[](tid0) -- (tid2);
%         \end{tikzpicture}
%         \nodepart{two}
%         \footnotesize{6.29688}
%         \nodepart{three}
%         \footnotesize{$50\:50$}
%       };
%       & 
%       \\
%     };
%   \end{scope}
%   \begin{scope}[yshift=\leveltopIII cm]
%     \matrix (line3) [column sep=1cm] {
%       \node[draw=black, rectangle split,  rectangle split parts=3] (sn0x10519d0){
%         \begin{tikzpicture}[scale=.2]
%           \node[circle, scale=0.75, fill] (tid0) at (3,1.5){};
%           \node[circle, scale=0.75, fill] (tid1) at (2.25,3){};
%           \node[circle, scale=0.75, fill, red] (tid3) at (0.75,4.5){};
%           \node[circle, scale=0.75, fill] (tid4) at (2.25,4.5){};
%           \node[circle, scale=0.75, fill] (tid5) at (3.75,4.5){};
%           \draw[](tid1) -- (tid3);
%           \draw[](tid1) -- (tid4);
%           \draw[](tid1) -- (tid5);
%           \node[circle, scale=0.75, fill] (tid2) at (5.25,3){};
%           \node[circle, scale=0.75, fill] (tid6) at (5.25,4.5){};
%           \node[circle, scale=0.75, fill] (tid7) at (5.25,6){};
%           \node[circle, scale=0.75, fill, red] (tid8) at (5.25,7.5){};
%           \draw[](tid7) -- (tid8);
%           \draw[](tid6) -- (tid7);
%           \draw[](tid2) -- (tid6);
%           \draw[](tid0) -- (tid1);
%           \draw[](tid0) -- (tid2);
%         \end{tikzpicture}
%         \nodepart{two}
%         \footnotesize{5.92188}
%         \nodepart{three}
%         \footnotesize{$50\:50$}
%       };
%       & 
%       \node[draw=black, rectangle split,  rectangle split parts=3] (sn0x105a080){
%         \begin{tikzpicture}[scale=.2]
%           \node[circle, scale=0.75, fill] (tid0) at (3.75,1.5){};
%           \node[circle, scale=0.75, fill] (tid1) at (2.25,3){};
%           \node[circle, scale=0.75, fill] (tid3) at (0.75,4.5){};
%           \node[circle, scale=0.75, fill] (tid4) at (2.25,4.5){};
%           \node[circle, scale=0.75, fill] (tid5) at (3.75,4.5){};
%           \draw[](tid1) -- (tid3);
%           \draw[](tid1) -- (tid4);
%           \draw[](tid1) -- (tid5);
%           \node[circle, scale=0.75, fill] (tid2) at (6,3){};
%           \node[circle, scale=0.75, fill] (tid6) at (6,4.5){};
%           \node[circle, scale=0.75, fill, red] (tid7) at (5.25,6){};
%           \node[circle, scale=0.75, fill, red] (tid8) at (6.75,6){};
%           \draw[](tid6) -- (tid7);
%           \draw[](tid6) -- (tid8);
%           \draw[](tid2) -- (tid6);
%           \draw[](tid0) -- (tid1);
%           \draw[](tid0) -- (tid2);
%         \end{tikzpicture}
%         \nodepart{two}
%         \footnotesize{5.79688}
%         \nodepart{three}
%         \footnotesize{$1$}
%       };
%       & 
%       \node[draw=black, rectangle split,  rectangle split parts=3] (sn0x104fbd0){
%         \begin{tikzpicture}[scale=.2]
%           \node[circle, scale=0.75, fill] (tid0) at (3,1.5){};
%           \node[circle, scale=0.75, fill] (tid1) at (2.25,3){};
%           \node[circle, scale=0.75, fill] (tid3) at (0.75,4.5){};
%           \node[circle, scale=0.75, fill, red] (tid7) at (0.75,6){};
%           \draw[](tid3) -- (tid7);
%           \node[circle, scale=0.75, fill] (tid4) at (2.25,4.5){};
%           \node[circle, scale=0.75, fill] (tid5) at (3.75,4.5){};
%           \draw[](tid1) -- (tid3);
%           \draw[](tid1) -- (tid4);
%           \draw[](tid1) -- (tid5);
%           \node[circle, scale=0.75, fill] (tid2) at (5.25,3){};
%           \node[circle, scale=0.75, fill] (tid6) at (5.25,4.5){};
%           \node[circle, scale=0.75, fill, red] (tid8) at (5.25,6){};
%           \draw[](tid6) -- (tid8);
%           \draw[](tid2) -- (tid6);
%           \draw[](tid0) -- (tid1);
%           \draw[](tid0) -- (tid2);
%         \end{tikzpicture}
%         \nodepart{two}
%         \footnotesize{5.79688}
%         \nodepart{three}
%         \footnotesize{$50\:33\:17$}
%       };
%       & 
%       \\
%     };
%   \end{scope}
%   \begin{scope}[yshift=\leveltopIIII cm]
%     \matrix (line4) [column sep=1cm] {
%       \node[draw=black, rectangle split,  rectangle split parts=3] (sn0x1052250){
%         \begin{tikzpicture}[scale=.2]
%           \node[circle, scale=0.75, fill] (tid0) at (2.25,1.5){};
%           \node[circle, scale=0.75, fill] (tid1) at (0.75,3){};
%           \node[circle, scale=0.75, fill] (tid3) at (0.75,4.5){};
%           \node[circle, scale=0.75, fill] (tid6) at (0.75,6){};
%           \node[circle, scale=0.75, fill, red] (tid7) at (0.75,7.5){};
%           \draw[](tid6) -- (tid7);
%           \draw[](tid3) -- (tid6);
%           \draw[](tid1) -- (tid3);
%           \node[circle, scale=0.75, fill] (tid2) at (3,3){};
%           \node[circle, scale=0.75, fill, red] (tid4) at (2.25,4.5){};
%           \node[circle, scale=0.75, fill] (tid5) at (3.75,4.5){};
%           \draw[](tid2) -- (tid4);
%           \draw[](tid2) -- (tid5);
%           \draw[](tid0) -- (tid1);
%           \draw[](tid0) -- (tid2);
%         \end{tikzpicture}
%         \nodepart{two}
%         \footnotesize{5.54688}
%         \nodepart{three}
%         \footnotesize{$50\:50$}
%       };
%       & 
%       \node[draw=black, rectangle split,  rectangle split parts=3] (sn0x1052960){
%         \begin{tikzpicture}[scale=.2]
%           \node[circle, scale=0.75, fill] (tid0) at (3,1.5){};
%           \node[circle, scale=0.75, fill] (tid1) at (2.25,3){};
%           \node[circle, scale=0.75, fill, red] (tid3) at (0.75,4.5){};
%           \node[circle, scale=0.75, fill] (tid4) at (2.25,4.5){};
%           \node[circle, scale=0.75, fill] (tid5) at (3.75,4.5){};
%           \draw[](tid1) -- (tid3);
%           \draw[](tid1) -- (tid4);
%           \draw[](tid1) -- (tid5);
%           \node[circle, scale=0.75, fill] (tid2) at (5.25,3){};
%           \node[circle, scale=0.75, fill] (tid6) at (5.25,4.5){};
%           \node[circle, scale=0.75, fill, red] (tid7) at (5.25,6){};
%           \draw[](tid6) -- (tid7);
%           \draw[](tid2) -- (tid6);
%           \draw[](tid0) -- (tid1);
%           \draw[](tid0) -- (tid2);
%         \end{tikzpicture}
%         \nodepart{two}
%         \footnotesize{5.29688}
%         \nodepart{three}
%         \footnotesize{$50\:33\:17$}
%       };
%       & 
%       \node[draw=black, rectangle split,  rectangle split parts=3] (sn0x10581e0){
%         \begin{tikzpicture}[scale=.2]
%           \node[circle, scale=0.75, fill] (tid0) at (3,1.5){};
%           \node[circle, scale=0.75, fill] (tid1) at (2.25,3){};
%           \node[circle, scale=0.75, fill] (tid3) at (0.75,4.5){};
%           \node[circle, scale=0.75, fill, red] (tid7) at (0.75,6){};
%           \draw[](tid3) -- (tid7);
%           \node[circle, scale=0.75, fill, red] (tid4) at (2.25,4.5){};
%           \node[circle, scale=0.75, fill] (tid5) at (3.75,4.5){};
%           \draw[](tid1) -- (tid3);
%           \draw[](tid1) -- (tid4);
%           \draw[](tid1) -- (tid5);
%           \node[circle, scale=0.75, fill] (tid2) at (5.25,3){};
%           \node[circle, scale=0.75, fill] (tid6) at (5.25,4.5){};
%           \draw[](tid2) -- (tid6);
%           \draw[](tid0) -- (tid1);
%           \draw[](tid0) -- (tid2);
%         \end{tikzpicture}
%         \nodepart{two}
%         \footnotesize{5.29688}
%         \nodepart{three}
%         \footnotesize{$33\:17\:25\:25$}
%       };
%       & 
%       \node[draw=black, rectangle split,  rectangle split parts=3] (sn0x1058550){
%         \begin{tikzpicture}[scale=.2]
%           \node[circle, scale=0.75, fill] (tid0) at (3,1.5){};
%           \node[circle, scale=0.75, fill] (tid1) at (2.25,3){};
%           \node[circle, scale=0.75, fill] (tid3) at (0.75,4.5){};
%           \node[circle, scale=0.75, fill, red] (tid7) at (0.75,6){};
%           \draw[](tid3) -- (tid7);
%           \node[circle, scale=0.75, fill] (tid4) at (2.25,4.5){};
%           \node[circle, scale=0.75, fill] (tid5) at (3.75,4.5){};
%           \draw[](tid1) -- (tid3);
%           \draw[](tid1) -- (tid4);
%           \draw[](tid1) -- (tid5);
%           \node[circle, scale=0.75, fill] (tid2) at (5.25,3){};
%           \node[circle, scale=0.75, fill, red] (tid6) at (5.25,4.5){};
%           \draw[](tid2) -- (tid6);
%           \draw[](tid0) -- (tid1);
%           \draw[](tid0) -- (tid2);
%         \end{tikzpicture}
%         \nodepart{two}
%         \footnotesize{5.29688}
%         \nodepart{three}
%         \footnotesize{$50\:50$}
%       };
%       & 
%       \\
%     };
%   \end{scope}
%   \begin{scope}[yshift=\leveltopIIIII cm]
%     \matrix (line5) [column sep=1cm] {
%       \node[draw=black, rectangle split,  rectangle split parts=3] (sn0x10525f0){
%         \begin{tikzpicture}[scale=.2]
%           \node[circle, scale=0.75, fill] (tid0) at (1.5,1.5){};
%           \node[circle, scale=0.75, fill] (tid1) at (0.75,3){};
%           \node[circle, scale=0.75, fill] (tid3) at (0.75,4.5){};
%           \node[circle, scale=0.75, fill] (tid5) at (0.75,6){};
%           \node[circle, scale=0.75, fill, red] (tid6) at (0.75,7.5){};
%           \draw[](tid5) -- (tid6);
%           \draw[](tid3) -- (tid5);
%           \draw[](tid1) -- (tid3);
%           \node[circle, scale=0.75, fill] (tid2) at (2.25,3){};
%           \node[circle, scale=0.75, fill, red] (tid4) at (2.25,4.5){};
%           \draw[](tid2) -- (tid4);
%           \draw[](tid0) -- (tid1);
%           \draw[](tid0) -- (tid2);
%         \end{tikzpicture}
%         \nodepart{two}
%         \footnotesize{5.25}
%         \nodepart{three}
%         \footnotesize{$50\:50$}
%       };
%       & 
%       \node[draw=black, rectangle split,  rectangle split parts=3] (sn0x1053850){
%         \begin{tikzpicture}[scale=.2]
%           \node[circle, scale=0.75, fill] (tid0) at (2.25,1.5){};
%           \node[circle, scale=0.75, fill] (tid1) at (1.5,3){};
%           \node[circle, scale=0.75, fill, red] (tid3) at (0.75,4.5){};
%           \node[circle, scale=0.75, fill] (tid4) at (2.25,4.5){};
%           \draw[](tid1) -- (tid3);
%           \draw[](tid1) -- (tid4);
%           \node[circle, scale=0.75, fill] (tid2) at (3.75,3){};
%           \node[circle, scale=0.75, fill] (tid5) at (3.75,4.5){};
%           \node[circle, scale=0.75, fill, red] (tid6) at (3.75,6){};
%           \draw[](tid5) -- (tid6);
%           \draw[](tid2) -- (tid5);
%           \draw[](tid0) -- (tid1);
%           \draw[](tid0) -- (tid2);
%         \end{tikzpicture}
%         \nodepart{two}
%         \footnotesize{4.84375}
%         \nodepart{three}
%         \footnotesize{$50\:25\:25$}
%       };
%       & 
%       \node[draw=black, rectangle split,  rectangle split parts=3] (sn0x1056b00){
%         \begin{tikzpicture}[scale=.2]
%           \node[circle, scale=0.75, fill] (tid0) at (3,1.5){};
%           \node[circle, scale=0.75, fill] (tid1) at (2.25,3){};
%           \node[circle, scale=0.75, fill, red] (tid3) at (0.75,4.5){};
%           \node[circle, scale=0.75, fill, red] (tid4) at (2.25,4.5){};
%           \node[circle, scale=0.75, fill] (tid5) at (3.75,4.5){};
%           \draw[](tid1) -- (tid3);
%           \draw[](tid1) -- (tid4);
%           \draw[](tid1) -- (tid5);
%           \node[circle, scale=0.75, fill] (tid2) at (5.25,3){};
%           \node[circle, scale=0.75, fill] (tid6) at (5.25,4.5){};
%           \draw[](tid2) -- (tid6);
%           \draw[](tid0) -- (tid1);
%           \draw[](tid0) -- (tid2);
%         \end{tikzpicture}
%         \nodepart{two}
%         \footnotesize{4.75}
%         \nodepart{three}
%         \footnotesize{$50\:50$}
%       };
%       & 
%       \node[draw=black, rectangle split,  rectangle split parts=3] (sn0x1056fb0){
%         \begin{tikzpicture}[scale=.2]
%           \node[circle, scale=0.75, fill] (tid0) at (3,1.5){};
%           \node[circle, scale=0.75, fill] (tid1) at (2.25,3){};
%           \node[circle, scale=0.75, fill, red] (tid3) at (0.75,4.5){};
%           \node[circle, scale=0.75, fill] (tid4) at (2.25,4.5){};
%           \node[circle, scale=0.75, fill] (tid5) at (3.75,4.5){};
%           \draw[](tid1) -- (tid3);
%           \draw[](tid1) -- (tid4);
%           \draw[](tid1) -- (tid5);
%           \node[circle, scale=0.75, fill] (tid2) at (5.25,3){};
%           \node[circle, scale=0.75, fill, red] (tid6) at (5.25,4.5){};
%           \draw[](tid2) -- (tid6);
%           \draw[](tid0) -- (tid1);
%           \draw[](tid0) -- (tid2);
%         \end{tikzpicture}
%         \nodepart{two}
%         \footnotesize{4.75}
%         \nodepart{three}
%         \footnotesize{$50\:50$}
%       };
%       & 
%       \node[draw=black, rectangle split,  rectangle split parts=3] (sn0x1058f50){
%         \begin{tikzpicture}[scale=.2]
%           \node[circle, scale=0.75, fill] (tid0) at (2.25,1.5){};
%           \node[circle, scale=0.75, fill] (tid1) at (1.5,3){};
%           \node[circle, scale=0.75, fill] (tid3) at (0.75,4.5){};
%           \node[circle, scale=0.75, fill, red] (tid6) at (0.75,6){};
%           \draw[](tid3) -- (tid6);
%           \node[circle, scale=0.75, fill, red] (tid4) at (2.25,4.5){};
%           \draw[](tid1) -- (tid3);
%           \draw[](tid1) -- (tid4);
%           \node[circle, scale=0.75, fill] (tid2) at (3.75,3){};
%           \node[circle, scale=0.75, fill] (tid5) at (3.75,4.5){};
%           \draw[](tid2) -- (tid5);
%           \draw[](tid0) -- (tid1);
%           \draw[](tid0) -- (tid2);
%         \end{tikzpicture}
%         \nodepart{two}
%         \footnotesize{4.84375}
%         \nodepart{three}
%         \footnotesize{$50\:25\:25$}
%       };
%       & 
%       \node[draw=black, rectangle split,  rectangle split parts=3] (sn0x1058a50){
%         \begin{tikzpicture}[scale=.2]
%           \node[circle, scale=0.75, fill] (tid0) at (2.25,1.5){};
%           \node[circle, scale=0.75, fill] (tid1) at (1.5,3){};
%           \node[circle, scale=0.75, fill] (tid3) at (0.75,4.5){};
%           \node[circle, scale=0.75, fill, red] (tid6) at (0.75,6){};
%           \draw[](tid3) -- (tid6);
%           \node[circle, scale=0.75, fill] (tid4) at (2.25,4.5){};
%           \draw[](tid1) -- (tid3);
%           \draw[](tid1) -- (tid4);
%           \node[circle, scale=0.75, fill] (tid2) at (3.75,3){};
%           \node[circle, scale=0.75, fill, red] (tid5) at (3.75,4.5){};
%           \draw[](tid2) -- (tid5);
%           \draw[](tid0) -- (tid1);
%           \draw[](tid0) -- (tid2);
%         \end{tikzpicture}
%         \nodepart{two}
%         \footnotesize{4.84375}
%         \nodepart{three}
%         \footnotesize{$50\:50$}
%       };
%       & 
%       \node[draw=black, rectangle split,  rectangle split parts=3] (sn0x10597a0){
%         \begin{tikzpicture}[scale=.2]
%           \node[circle, scale=0.75, fill] (tid0) at (3,1.5){};
%           \node[circle, scale=0.75, fill] (tid1) at (2.25,3){};
%           \node[circle, scale=0.75, fill] (tid3) at (0.75,4.5){};
%           \node[circle, scale=0.75, fill, red] (tid6) at (0.75,6){};
%           \draw[](tid3) -- (tid6);
%           \node[circle, scale=0.75, fill, red] (tid4) at (2.25,4.5){};
%           \node[circle, scale=0.75, fill] (tid5) at (3.75,4.5){};
%           \draw[](tid1) -- (tid3);
%           \draw[](tid1) -- (tid4);
%           \draw[](tid1) -- (tid5);
%           \node[circle, scale=0.75, fill] (tid2) at (5.25,3){};
%           \draw[](tid0) -- (tid1);
%           \draw[](tid0) -- (tid2);
%         \end{tikzpicture}
%         \nodepart{two}
%         \footnotesize{4.84375}
%         \nodepart{three}
%         \footnotesize{$50\:50$}
%       };
%       & 
%       \\
%     };
%   \end{scope}
%   \begin{scope}[yshift=\leveltopIIIIII cm]
%     \matrix (line6) [column sep=1cm] {
%       \node[draw=black, rectangle split,  rectangle split parts=3] (sn0x1053920){
%         \begin{tikzpicture}[scale=.2]
%           \node[circle, scale=0.75, fill] (tid0) at (1.5,1.5){};
%           \node[circle, scale=0.75, fill] (tid1) at (0.75,3){};
%           \node[circle, scale=0.75, fill] (tid3) at (0.75,4.5){};
%           \node[circle, scale=0.75, fill] (tid4) at (0.75,6){};
%           \node[circle, scale=0.75, fill, red] (tid5) at (0.75,7.5){};
%           \draw[](tid4) -- (tid5);
%           \draw[](tid3) -- (tid4);
%           \draw[](tid1) -- (tid3);
%           \node[circle, scale=0.75, fill, red] (tid2) at (2.25,3){};
%           \draw[](tid0) -- (tid1);
%           \draw[](tid0) -- (tid2);
%         \end{tikzpicture}
%         \nodepart{two}
%         \footnotesize{5.0625}
%         \nodepart{three}
%         \footnotesize{$50\:50$}
%       };
%       & 
%       \node[draw=black, rectangle split,  rectangle split parts=3] (sn0x1053bc0){
%         \begin{tikzpicture}[scale=.2]
%           \node[circle, scale=0.75, fill] (tid0) at (1.5,1.5){};
%           \node[circle, scale=0.75, fill] (tid1) at (0.75,3){};
%           \node[circle, scale=0.75, fill] (tid3) at (0.75,4.5){};
%           \node[circle, scale=0.75, fill, red] (tid5) at (0.75,6){};
%           \draw[](tid3) -- (tid5);
%           \draw[](tid1) -- (tid3);
%           \node[circle, scale=0.75, fill] (tid2) at (2.25,3){};
%           \node[circle, scale=0.75, fill, red] (tid4) at (2.25,4.5){};
%           \draw[](tid2) -- (tid4);
%           \draw[](tid0) -- (tid1);
%           \draw[](tid0) -- (tid2);
%         \end{tikzpicture}
%         \nodepart{two}
%         \footnotesize{4.4375}
%         \nodepart{three}
%         \footnotesize{$50\:50$}
%       };
%       & 
%       \node[draw=black, rectangle split,  rectangle split parts=3] (sn0x1056090){
%         \begin{tikzpicture}[scale=.2]
%           \node[circle, scale=0.75, fill] (tid0) at (2.25,1.5){};
%           \node[circle, scale=0.75, fill] (tid1) at (1.5,3){};
%           \node[circle, scale=0.75, fill, red] (tid3) at (0.75,4.5){};
%           \node[circle, scale=0.75, fill, red] (tid4) at (2.25,4.5){};
%           \draw[](tid1) -- (tid3);
%           \draw[](tid1) -- (tid4);
%           \node[circle, scale=0.75, fill] (tid2) at (3.75,3){};
%           \node[circle, scale=0.75, fill] (tid5) at (3.75,4.5){};
%           \draw[](tid2) -- (tid5);
%           \draw[](tid0) -- (tid1);
%           \draw[](tid0) -- (tid2);
%         \end{tikzpicture}
%         \nodepart{two}
%         \footnotesize{4.25}
%         \nodepart{three}
%         \footnotesize{$1$}
%       };
%       & 
%       \node[draw=black, rectangle split,  rectangle split parts=3] (sn0x1056160){
%         \begin{tikzpicture}[scale=.2]
%           \node[circle, scale=0.75, fill] (tid0) at (2.25,1.5){};
%           \node[circle, scale=0.75, fill] (tid1) at (1.5,3){};
%           \node[circle, scale=0.75, fill, red] (tid3) at (0.75,4.5){};
%           \node[circle, scale=0.75, fill] (tid4) at (2.25,4.5){};
%           \draw[](tid1) -- (tid3);
%           \draw[](tid1) -- (tid4);
%           \node[circle, scale=0.75, fill] (tid2) at (3.75,3){};
%           \node[circle, scale=0.75, fill, red] (tid5) at (3.75,4.5){};
%           \draw[](tid2) -- (tid5);
%           \draw[](tid0) -- (tid1);
%           \draw[](tid0) -- (tid2);
%         \end{tikzpicture}
%         \nodepart{two}
%         \footnotesize{4.25}
%         \nodepart{three}
%         \footnotesize{$50\:50$}
%       };
%       & 
%       \node[draw=black, rectangle split,  rectangle split parts=3] (sn0x1057630){
%         \begin{tikzpicture}[scale=.2]
%           \node[circle, scale=0.75, fill] (tid0) at (3,1.5){};
%           \node[circle, scale=0.75, fill] (tid1) at (2.25,3){};
%           \node[circle, scale=0.75, fill, red] (tid3) at (0.75,4.5){};
%           \node[circle, scale=0.75, fill, red] (tid4) at (2.25,4.5){};
%           \node[circle, scale=0.75, fill] (tid5) at (3.75,4.5){};
%           \draw[](tid1) -- (tid3);
%           \draw[](tid1) -- (tid4);
%           \draw[](tid1) -- (tid5);
%           \node[circle, scale=0.75, fill] (tid2) at (5.25,3){};
%           \draw[](tid0) -- (tid1);
%           \draw[](tid0) -- (tid2);
%         \end{tikzpicture}
%         \nodepart{two}
%         \footnotesize{4.25}
%         \nodepart{three}
%         \footnotesize{$1$}
%       };
%       & 
%       \node[draw=black, rectangle split,  rectangle split parts=3] (sn0x1058b20){
%         \begin{tikzpicture}[scale=.2]
%           \node[circle, scale=0.75, fill] (tid0) at (2.25,1.5){};
%           \node[circle, scale=0.75, fill] (tid1) at (1.5,3){};
%           \node[circle, scale=0.75, fill] (tid3) at (0.75,4.5){};
%           \node[circle, scale=0.75, fill, red] (tid5) at (0.75,6){};
%           \draw[](tid3) -- (tid5);
%           \node[circle, scale=0.75, fill, red] (tid4) at (2.25,4.5){};
%           \draw[](tid1) -- (tid3);
%           \draw[](tid1) -- (tid4);
%           \node[circle, scale=0.75, fill] (tid2) at (3.75,3){};
%           \draw[](tid0) -- (tid1);
%           \draw[](tid0) -- (tid2);
%         \end{tikzpicture}
%         \nodepart{two}
%         \footnotesize{4.4375}
%         \nodepart{three}
%         \footnotesize{$50\:50$}
%       };
%       & 
%       \\
%     };
%   \end{scope}
%   \begin{scope}[yshift=\leveltopIIIIIII cm]
%     \matrix (line7) [column sep=1cm] {
%       \node[draw=black, rectangle split,  rectangle split parts=3] (sn0x10540d0){
%         \begin{tikzpicture}[scale=.2]
%           \node[circle, scale=0.75, fill] (tid0) at (0.75,1.5){};
%           \node[circle, scale=0.75, fill] (tid1) at (0.75,3){};
%           \node[circle, scale=0.75, fill] (tid2) at (0.75,4.5){};
%           \node[circle, scale=0.75, fill] (tid3) at (0.75,6){};
%           \node[circle, scale=0.75, fill, red] (tid4) at (0.75,7.5){};
%           \draw[](tid3) -- (tid4);
%           \draw[](tid2) -- (tid3);
%           \draw[](tid1) -- (tid2);
%           \draw[](tid0) -- (tid1);
%         \end{tikzpicture}
%         \nodepart{two}
%         \footnotesize{5}
%         \nodepart{three}
%         \footnotesize{$1$}
%       };
%       & 
%       \node[draw=black, rectangle split,  rectangle split parts=3] (sn0x1054480){
%         \begin{tikzpicture}[scale=.2]
%           \node[circle, scale=0.75, fill] (tid0) at (1.5,1.5){};
%           \node[circle, scale=0.75, fill] (tid1) at (0.75,3){};
%           \node[circle, scale=0.75, fill] (tid3) at (0.75,4.5){};
%           \node[circle, scale=0.75, fill, red] (tid4) at (0.75,6){};
%           \draw[](tid3) -- (tid4);
%           \draw[](tid1) -- (tid3);
%           \node[circle, scale=0.75, fill, red] (tid2) at (2.25,3){};
%           \draw[](tid0) -- (tid1);
%           \draw[](tid0) -- (tid2);
%         \end{tikzpicture}
%         \nodepart{two}
%         \footnotesize{4.125}
%         \nodepart{three}
%         \footnotesize{$50\:50$}
%       };
%       & 
%       \node[draw=black, rectangle split,  rectangle split parts=3] (sn0x1055dd0){
%         \begin{tikzpicture}[scale=.2]
%           \node[circle, scale=0.75, fill] (tid0) at (1.5,1.5){};
%           \node[circle, scale=0.75, fill] (tid1) at (0.75,3){};
%           \node[circle, scale=0.75, fill, red] (tid3) at (0.75,4.5){};
%           \draw[](tid1) -- (tid3);
%           \node[circle, scale=0.75, fill] (tid2) at (2.25,3){};
%           \node[circle, scale=0.75, fill, red] (tid4) at (2.25,4.5){};
%           \draw[](tid2) -- (tid4);
%           \draw[](tid0) -- (tid1);
%           \draw[](tid0) -- (tid2);
%         \end{tikzpicture}
%         \nodepart{two}
%         \footnotesize{3.75}
%         \nodepart{three}
%         \footnotesize{$1$}
%       };
%       & 
%       \node[draw=black, rectangle split,  rectangle split parts=3] (sn0x10568c0){
%         \begin{tikzpicture}[scale=.2]
%           \node[circle, scale=0.75, fill] (tid0) at (2.25,1.5){};
%           \node[circle, scale=0.75, fill] (tid1) at (1.5,3){};
%           \node[circle, scale=0.75, fill, red] (tid3) at (0.75,4.5){};
%           \node[circle, scale=0.75, fill, red] (tid4) at (2.25,4.5){};
%           \draw[](tid1) -- (tid3);
%           \draw[](tid1) -- (tid4);
%           \node[circle, scale=0.75, fill] (tid2) at (3.75,3){};
%           \draw[](tid0) -- (tid1);
%           \draw[](tid0) -- (tid2);
%         \end{tikzpicture}
%         \nodepart{two}
%         \footnotesize{3.75}
%         \nodepart{three}
%         \footnotesize{$1$}
%       };
%       & 
%       \\
%     };
%   \end{scope}
%   \begin{scope}[yshift=\leveltopIIIIIIII cm]
%     \matrix (line8) [column sep=1cm] {
%       \node[draw=black, rectangle split,  rectangle split parts=3] (sn0x1054550){
%         \begin{tikzpicture}[scale=.2]
%           \node[circle, scale=0.75, fill] (tid0) at (0.75,1.5){};
%           \node[circle, scale=0.75, fill] (tid1) at (0.75,3){};
%           \node[circle, scale=0.75, fill] (tid2) at (0.75,4.5){};
%           \node[circle, scale=0.75, fill, red] (tid3) at (0.75,6){};
%           \draw[](tid2) -- (tid3);
%           \draw[](tid1) -- (tid2);
%           \draw[](tid0) -- (tid1);
%         \end{tikzpicture}
%         \nodepart{two}
%         \footnotesize{4}
%         \nodepart{three}
%         \footnotesize{$1$}
%       };
%       & 
%       \node[draw=black, rectangle split,  rectangle split parts=3] (sn0x1055270){
%         \begin{tikzpicture}[scale=.2]
%           \node[circle, scale=0.75, fill] (tid0) at (1.5,1.5){};
%           \node[circle, scale=0.75, fill] (tid1) at (0.75,3){};
%           \node[circle, scale=0.75, fill, red] (tid3) at (0.75,4.5){};
%           \draw[](tid1) -- (tid3);
%           \node[circle, scale=0.75, fill, red] (tid2) at (2.25,3){};
%           \draw[](tid0) -- (tid1);
%           \draw[](tid0) -- (tid2);
%         \end{tikzpicture}
%         \nodepart{two}
%         \footnotesize{3.25}
%         \nodepart{three}
%         \footnotesize{$50\:50$}
%       };
%       & 
%       \\
%     };
%   \end{scope}
%   \begin{scope}[yshift=\leveltopIIIIIIIII cm]
%     \matrix (line9) [column sep=1cm] {
%       \node[draw=black, rectangle split,  rectangle split parts=3] (sn0x1054a50){
%         \begin{tikzpicture}[scale=.2]
%           \node[circle, scale=0.75, fill] (tid0) at (0.75,1.5){};
%           \node[circle, scale=0.75, fill] (tid1) at (0.75,3){};
%           \node[circle, scale=0.75, fill, red] (tid2) at (0.75,4.5){};
%           \draw[](tid1) -- (tid2);
%           \draw[](tid0) -- (tid1);
%         \end{tikzpicture}
%         \nodepart{two}
%         \footnotesize{3}
%         \nodepart{three}
%         \footnotesize{$1$}
%       };
%       & 
%       \node[draw=black, rectangle split,  rectangle split parts=3] (sn0x1054cb0){
%         \begin{tikzpicture}[scale=.2]
%           \node[circle, scale=0.75, fill] (tid0) at (1.5,1.5){};
%           \node[circle, scale=0.75, fill, red] (tid1) at (0.75,3){};
%           \node[circle, scale=0.75, fill, red] (tid2) at (2.25,3){};
%           \draw[](tid0) -- (tid1);
%           \draw[](tid0) -- (tid2);
%         \end{tikzpicture}
%         \nodepart{two}
%         \footnotesize{2.5}
%         \nodepart{three}
%         \footnotesize{$1$}
%       };
%       & 
%       \\
%     };
%   \end{scope}
%   \begin{scope}[yshift=\leveltopIIIIIIIIII cm]
%     \matrix (line10) [column sep=1cm] {
%       \node[draw=black, rectangle split,  rectangle split parts=3] (sn0x1054b20){
%         \begin{tikzpicture}[scale=.2]
%           \node[circle, scale=0.75, fill] (tid0) at (0.75,1.5){};
%           \node[circle, scale=0.75, fill, red] (tid1) at (0.75,3){};
%           \draw[](tid0) -- (tid1);
%         \end{tikzpicture}
%         \nodepart{two}
%         \footnotesize{2}
%         \nodepart{three}
%         \footnotesize{$1$}
%       };
%       & 
%       \\
%     };
%   \end{scope}
%   \begin{scope}[yshift=\leveltopIIIIIIIIIII cm]
%     \matrix (line11) [column sep=1cm] {
%       \node[draw=black, rectangle split,  rectangle split parts=3] (sn0x10547e0){
%         \begin{tikzpicture}[scale=.2]
%           \node[circle, scale=0.75, fill, red] (tid0) at (0.75,1.5){};
%         \end{tikzpicture}
%         \nodepart{two}
%         \footnotesize{1}
%         \nodepart{three}
%         \footnotesize{$$}
%       };
%       & 
%       \\
%     };
%   \end{scope}
%   \begin{scope}[yshift=\leveltopIIIIIIIIIIII cm]
%     \matrix (line12) [column sep=1cm] {
%       \\
%     };
%   \end{scope}
%   \draw (sn0x1050af0.south) -- (sn0x105a150.north);
%   \draw (sn0x1050af0.south) -- (sn0x104cb60.north);
%   \draw (sn0x105a150.south) -- (sn0x10519d0.north);
%   \draw (sn0x105a150.south) -- (sn0x105a080.north);
%   \draw (sn0x104cb60.south) -- (sn0x105a080.north);
%   \draw (sn0x104cb60.south) -- (sn0x104fbd0.north);
%   \draw (sn0x10519d0.south) -- (sn0x1052250.north);
%   \draw (sn0x10519d0.south) -- (sn0x1052960.north);
%   \draw (sn0x105a080.south) -- (sn0x1052960.north);
%   \draw (sn0x104fbd0.south) -- (sn0x1052960.north);
%   \draw (sn0x104fbd0.south) -- (sn0x10581e0.north);
%   \draw (sn0x104fbd0.south) -- (sn0x1058550.north);
%   \draw (sn0x1052250.south) -- (sn0x10525f0.north);
%   \draw (sn0x1052250.south) -- (sn0x1053850.north);
%   \draw (sn0x1052960.south) -- (sn0x1053850.north);
%   \draw (sn0x1052960.south) -- (sn0x1056b00.north);
%   \draw (sn0x1052960.south) -- (sn0x1056fb0.north);
%   \draw (sn0x10581e0.south) -- (sn0x1058f50.north);
%   \draw (sn0x10581e0.south) -- (sn0x1058a50.north);
%   \draw (sn0x10581e0.south) -- (sn0x1056b00.north);
%   \draw (sn0x10581e0.south) -- (sn0x1056fb0.north);
%   \draw (sn0x1058550.south) -- (sn0x10597a0.north);
%   \draw (sn0x1058550.south) -- (sn0x1056fb0.north);
%   \draw (sn0x10525f0.south) -- (sn0x1053920.north);
%   \draw (sn0x10525f0.south) -- (sn0x1053bc0.north);
%   \draw (sn0x1053850.south) -- (sn0x1053bc0.north);
%   \draw (sn0x1053850.south) -- (sn0x1056090.north);
%   \draw (sn0x1053850.south) -- (sn0x1056160.north);
%   \draw (sn0x1056b00.south) -- (sn0x1056090.north);
%   \draw (sn0x1056b00.south) -- (sn0x1056160.north);
%   \draw (sn0x1056fb0.south) -- (sn0x1056160.north);
%   \draw (sn0x1056fb0.south) -- (sn0x1057630.north);
%   \draw (sn0x1058f50.south) -- (sn0x1053bc0.north);
%   \draw (sn0x1058f50.south) -- (sn0x1056090.north);
%   \draw (sn0x1058f50.south) -- (sn0x1056160.north);
%   \draw (sn0x1058a50.south) -- (sn0x1058b20.north);
%   \draw (sn0x1058a50.south) -- (sn0x1056160.north);
%   \draw (sn0x10597a0.south) -- (sn0x1058b20.north);
%   \draw (sn0x10597a0.south) -- (sn0x1057630.north);
%   \draw (sn0x1053920.south) -- (sn0x10540d0.north);
%   \draw (sn0x1053920.south) -- (sn0x1054480.north);
%   \draw (sn0x1053bc0.south) -- (sn0x1054480.north);
%   \draw (sn0x1053bc0.south) -- (sn0x1055dd0.north);
%   \draw (sn0x1056090.south) -- (sn0x1055dd0.north);
%   \draw (sn0x1056160.south) -- (sn0x1055dd0.north);
%   \draw (sn0x1056160.south) -- (sn0x10568c0.north);
%   \draw (sn0x1057630.south) -- (sn0x10568c0.north);
%   \draw (sn0x1058b20.south) -- (sn0x1054480.north);
%   \draw (sn0x1058b20.south) -- (sn0x10568c0.north);
%   \draw (sn0x10540d0.south) -- (sn0x1054550.north);
%   \draw (sn0x1054480.south) -- (sn0x1054550.north);
%   \draw (sn0x1054480.south) -- (sn0x1055270.north);
%   \draw (sn0x1055dd0.south) -- (sn0x1055270.north);
%   \draw (sn0x10568c0.south) -- (sn0x1055270.north);
%   \draw (sn0x1054550.south) -- (sn0x1054a50.north);
%   \draw (sn0x1055270.south) -- (sn0x1054a50.north);
%   \draw (sn0x1055270.south) -- (sn0x1054cb0.north);
%   \draw (sn0x1054a50.south) -- (sn0x1054b20.north);
%   \draw (sn0x1054cb0.south) -- (sn0x1054b20.north);
%   \draw (sn0x1054b20.south) -- (sn0x10547e0.north);
% \end{tikzpicture}

%%% Local Variables:
%%% TeX-master: "thesis/thesis.tex"
%%% End: 


%\begin{tikzpicture}[scale=.2]
  \begin{scope}
    \node[draw=black] (sn0x115b5e0W9) at (9, -10) {\begin{tikzpicture}[scale=.2]
        \node[circle,scale=0.75,fill]{}[grow=up]
        child{node[circle,scale=0.75,fill]{}[grow=up]
          child{node[circle,scale=0.75,fill]{}[grow=up]
            child{node[circle,scale=0.75,fill]{}[grow=up]
              child{node[circle,scale=0.75,fill,red]{}[grow=up]
              }
            }
            child{node[circle,scale=0.75,fill,red]{}[grow=up]
            }
          }
        }
        child{node[circle,scale=0.75,fill]{}[grow=up]
          child{node[circle,scale=0.75,fill]{}[grow=up]
            child{node[circle,scale=0.75,fill]{}[grow=up]
            }
          }
          child{node[circle,scale=0.75,fill]{}[grow=up]
          }
          child{node[circle,scale=0.75,fill]{}[grow=up]
          }
        }
        ;
      \end{tikzpicture}
    };
    \node[draw=black] (sn0x115c080W9) at (9, -20) {\begin{tikzpicture}[scale=.2]
        \node[circle,scale=0.75,fill]{}[grow=up]
        child{node[circle,scale=0.75,fill]{}[grow=up]
          child{node[circle,scale=0.75,fill]{}[grow=up]
            child{node[circle,scale=0.75,fill,red]{}[grow=up]
            }
          }
          child{node[circle,scale=0.75,fill]{}[grow=up]
          }
          child{node[circle,scale=0.75,fill]{}[grow=up]
          }
        }
        child{node[circle,scale=0.75,fill]{}[grow=up]
          child{node[circle,scale=0.75,fill]{}[grow=up]
            child{node[circle,scale=0.75,fill]{}[grow=up]
              child{node[circle,scale=0.75,fill,red]{}[grow=up]
              }
            }
          }
        }
        ;
      \end{tikzpicture}
    };
    \node[draw=black] (sn0x115e1f0W9) at (9, -30) {\begin{tikzpicture}[scale=.2]
        \node[circle,scale=0.75,fill]{}[grow=up]
        child{node[circle,scale=0.75,fill]{}[grow=up]
          child{node[circle,scale=0.75,fill]{}[grow=up]
            child{node[circle,scale=0.75,fill]{}[grow=up]
              child{node[circle,scale=0.75,fill,red]{}[grow=up]
              }
            }
          }
        }
        child{node[circle,scale=0.75,fill]{}[grow=up]
          child{node[circle,scale=0.75,fill,red]{}[grow=up]
          }
          child{node[circle,scale=0.75,fill]{}[grow=up]
          }
          child{node[circle,scale=0.75,fill]{}[grow=up]
          }
        }
        ;
      \end{tikzpicture}
    };
    \node[draw=black] (sn0x115e9e0W9) at (9, -40) {\begin{tikzpicture}[scale=.2]
        \node[circle,scale=0.75,fill]{}[grow=up]
        child{node[circle,scale=0.75,fill]{}[grow=up]
          child{node[circle,scale=0.75,fill]{}[grow=up]
            child{node[circle,scale=0.75,fill]{}[grow=up]
              child{node[circle,scale=0.75,fill,red]{}[grow=up]
              }
            }
          }
        }
        child{node[circle,scale=0.75,fill]{}[grow=up]
          child{node[circle,scale=0.75,fill,red]{}[grow=up]
          }
          child{node[circle,scale=0.75,fill]{}[grow=up]
          }
        }
        ;
      \end{tikzpicture}
    };
    \node[draw=black] (sn0x115f370W9) at (9, -50) {\begin{tikzpicture}[scale=.2]
        \node[circle,scale=0.75,fill]{}[grow=up]
        child{node[circle,scale=0.75,fill]{}[grow=up]
          child{node[circle,scale=0.75,fill]{}[grow=up]
            child{node[circle,scale=0.75,fill]{}[grow=up]
              child{node[circle,scale=0.75,fill,red]{}[grow=up]
              }
            }
          }
        }
        child{node[circle,scale=0.75,fill]{}[grow=up]
          child{node[circle,scale=0.75,fill,red]{}[grow=up]
          }
        }
        ;
      \end{tikzpicture}
    };
    \node[draw=black] (sn0x115f500W9) at (9, -60) {\begin{tikzpicture}[scale=.2]
        \node[circle,scale=0.75,fill]{}[grow=up]
        child{node[circle,scale=0.75,fill]{}[grow=up]
          child{node[circle,scale=0.75,fill]{}[grow=up]
            child{node[circle,scale=0.75,fill]{}[grow=up]
              child{node[circle,scale=0.75,fill,red]{}[grow=up]
              }
            }
          }
        }
        child{node[circle,scale=0.75,fill,red]{}[grow=up]
        }
        ;
      \end{tikzpicture}
    };
    \node[draw=black] (sn0x1160260W9) at (9, -70) {\begin{tikzpicture}[scale=.2]
        \node[circle,scale=0.75,fill]{}[grow=up]
        child{node[circle,scale=0.75,fill]{}[grow=up]
          child{node[circle,scale=0.75,fill]{}[grow=up]
            child{node[circle,scale=0.75,fill]{}[grow=up]
              child{node[circle,scale=0.75,fill,red]{}[grow=up]
              }
            }
          }
        }
        ;
      \end{tikzpicture}
    };
    \node[draw=black] (sn0x11603f0W9) at (9, -80) {\begin{tikzpicture}[scale=.2]
        \node[circle,scale=0.75,fill]{}[grow=up]
        child{node[circle,scale=0.75,fill]{}[grow=up]
          child{node[circle,scale=0.75,fill]{}[grow=up]
            child{node[circle,scale=0.75,fill,red]{}[grow=up]
            }
          }
        }
        ;
      \end{tikzpicture}
    };
    \node[draw=black] (sn0x1160880W9) at (9, -90) {\begin{tikzpicture}[scale=.2]
        \node[circle,scale=0.75,fill]{}[grow=up]
        child{node[circle,scale=0.75,fill]{}[grow=up]
          child{node[circle,scale=0.75,fill,red]{}[grow=up]
          }
        }
        ;
      \end{tikzpicture}
    };
    \node[draw=black] (sn0x1160990W9) at (9, -100) {\begin{tikzpicture}[scale=.2]
        \node[circle,scale=0.75,fill]{}[grow=up]
        child{node[circle,scale=0.75,fill,red]{}[grow=up]
        }
        ;
      \end{tikzpicture}
    };
    \node[draw=black] (sn0x1160ac0W9) at (9, -110) {\begin{tikzpicture}[scale=.2]
        \node[circle,scale=0.75,fill]{}[grow=up]
        ;
      \end{tikzpicture}
    };
    \draw (sn0x1160990W9.south) -- (sn0x1160ac0W9.north);
    \draw (sn0x1160880W9.south) -- (sn0x1160990W9.north);
    \draw (sn0x11603f0W9.south) -- (sn0x1160880W9.north);
    \draw (sn0x1160260W9.south) -- (sn0x11603f0W9.north);
    \node[draw=black] (sn0x1160630W18) at (18, -70) {\begin{tikzpicture}[scale=.2]
        \node[circle,scale=0.75,fill]{}[grow=up]
        child{node[circle,scale=0.75,fill]{}[grow=up]
          child{node[circle,scale=0.75,fill]{}[grow=up]
            child{node[circle,scale=0.75,fill,red]{}[grow=up]
            }
          }
        }
        child{node[circle,scale=0.75,fill,red]{}[grow=up]
        }
        ;
      \end{tikzpicture}
    };
    \node[draw=black] (sn0x1160f10W18) at (18, -80) {\begin{tikzpicture}[scale=.2]
        \node[circle,scale=0.75,fill]{}[grow=up]
        child{node[circle,scale=0.75,fill]{}[grow=up]
          child{node[circle,scale=0.75,fill,red]{}[grow=up]
          }
        }
        child{node[circle,scale=0.75,fill,red]{}[grow=up]
        }
        ;
      \end{tikzpicture}
    };
    \node[draw=black] (sn0x1161350W18) at (18, -90) {\begin{tikzpicture}[scale=.2]
        \node[circle,scale=0.75,fill]{}[grow=up]
        child{node[circle,scale=0.75,fill,red]{}[grow=up]
        }
        child{node[circle,scale=0.75,fill,red]{}[grow=up]
        }
        ;
      \end{tikzpicture}
    };
    \draw (sn0x1161350W18.south) -- (sn0x1160990W9.north);
    \draw (sn0x1160f10W18.south) -- (sn0x1160880W9.north);
    \draw (sn0x1160f10W18.south) -- (sn0x1161350W18.north);
    \draw (sn0x1160630W18.south) -- (sn0x11603f0W9.north);
    \draw (sn0x1160630W18.south) -- (sn0x1160f10W18.north);
    \draw (sn0x115f500W9.south) -- (sn0x1160260W9.north);
    \draw (sn0x115f500W9.south) -- (sn0x1160630W18.north);
    \node[draw=black] (sn0x115f7f0W18) at (18, -60) {\begin{tikzpicture}[scale=.2]
        \node[circle,scale=0.75,fill]{}[grow=up]
        child{node[circle,scale=0.75,fill]{}[grow=up]
          child{node[circle,scale=0.75,fill]{}[grow=up]
            child{node[circle,scale=0.75,fill,red]{}[grow=up]
            }
          }
        }
        child{node[circle,scale=0.75,fill]{}[grow=up]
          child{node[circle,scale=0.75,fill,red]{}[grow=up]
          }
        }
        ;
      \end{tikzpicture}
    };
    \node[draw=black] (sn0x1161980W27) at (27, -70) {\begin{tikzpicture}[scale=.2]
        \node[circle,scale=0.75,fill]{}[grow=up]
        child{node[circle,scale=0.75,fill]{}[grow=up]
          child{node[circle,scale=0.75,fill,red]{}[grow=up]
          }
        }
        child{node[circle,scale=0.75,fill]{}[grow=up]
          child{node[circle,scale=0.75,fill,red]{}[grow=up]
          }
        }
        ;
      \end{tikzpicture}
    };
    \draw (sn0x1161980W27.south) -- (sn0x1160f10W18.north);
    \draw (sn0x115f7f0W18.south) -- (sn0x1160630W18.north);
    \draw (sn0x115f7f0W18.south) -- (sn0x1161980W27.north);
    \draw (sn0x115f370W9.south) -- (sn0x115f500W9.north);
    \draw (sn0x115f370W9.south) -- (sn0x115f7f0W18.north);
    \node[draw=black] (sn0x115f660W18) at (18, -50) {\begin{tikzpicture}[scale=.2]
        \node[circle,scale=0.75,fill]{}[grow=up]
        child{node[circle,scale=0.75,fill]{}[grow=up]
          child{node[circle,scale=0.75,fill]{}[grow=up]
            child{node[circle,scale=0.75,fill,red]{}[grow=up]
            }
          }
        }
        child{node[circle,scale=0.75,fill]{}[grow=up]
          child{node[circle,scale=0.75,fill,red]{}[grow=up]
          }
          child{node[circle,scale=0.75,fill]{}[grow=up]
          }
        }
        ;
      \end{tikzpicture}
    };
    \node[draw=black] (sn0x1161800W27) at (27, -60) {\begin{tikzpicture}[scale=.2]
        \node[circle,scale=0.75,fill]{}[grow=up]
        child{node[circle,scale=0.75,fill]{}[grow=up]
          child{node[circle,scale=0.75,fill,red]{}[grow=up]
          }
          child{node[circle,scale=0.75,fill]{}[grow=up]
          }
        }
        child{node[circle,scale=0.75,fill]{}[grow=up]
          child{node[circle,scale=0.75,fill,red]{}[grow=up]
          }
        }
        ;
      \end{tikzpicture}
    };
    \node[draw=black] (sn0x11624b0W36) at (36, -70) {\begin{tikzpicture}[scale=.2]
        \node[circle,scale=0.75,fill]{}[grow=up]
        child{node[circle,scale=0.75,fill]{}[grow=up]
          child{node[circle,scale=0.75,fill,red]{}[grow=up]
          }
          child{node[circle,scale=0.75,fill,red]{}[grow=up]
          }
        }
        child{node[circle,scale=0.75,fill]{}[grow=up]
        }
        ;
      \end{tikzpicture}
    };
    \draw (sn0x11624b0W36.south) -- (sn0x1160f10W18.north);
    \draw (sn0x1161800W27.south) -- (sn0x1161980W27.north);
    \draw (sn0x1161800W27.south) -- (sn0x11624b0W36.north);
    \node[draw=black] (sn0x1162160W36) at (36, -60) {\begin{tikzpicture}[scale=.2]
        \node[circle,scale=0.75,fill]{}[grow=up]
        child{node[circle,scale=0.75,fill]{}[grow=up]
          child{node[circle,scale=0.75,fill,red]{}[grow=up]
          }
          child{node[circle,scale=0.75,fill,red]{}[grow=up]
          }
        }
        child{node[circle,scale=0.75,fill]{}[grow=up]
          child{node[circle,scale=0.75,fill]{}[grow=up]
          }
        }
        ;
      \end{tikzpicture}
    };
    \draw (sn0x1162160W36.south) -- (sn0x1161980W27.north);
    \draw (sn0x115f660W18.south) -- (sn0x115f7f0W18.north);
    \draw (sn0x115f660W18.south) -- (sn0x1161800W27.north);
    \draw (sn0x115f660W18.south) -- (sn0x1162160W36.north);
    \draw (sn0x115e9e0W9.south) -- (sn0x115f370W9.north);
    \draw (sn0x115e9e0W9.south) -- (sn0x115f660W18.north);
    \node[draw=black] (sn0x115eed0W18) at (18, -40) {\begin{tikzpicture}[scale=.2]
        \node[circle,scale=0.75,fill]{}[grow=up]
        child{node[circle,scale=0.75,fill]{}[grow=up]
          child{node[circle,scale=0.75,fill,red]{}[grow=up]
          }
          child{node[circle,scale=0.75,fill]{}[grow=up]
          }
          child{node[circle,scale=0.75,fill]{}[grow=up]
          }
        }
        child{node[circle,scale=0.75,fill]{}[grow=up]
          child{node[circle,scale=0.75,fill]{}[grow=up]
            child{node[circle,scale=0.75,fill,red]{}[grow=up]
            }
          }
        }
        ;
      \end{tikzpicture}
    };
    \node[draw=black] (sn0x1162a20W27) at (27, -50) {\begin{tikzpicture}[scale=.2]
        \node[circle,scale=0.75,fill]{}[grow=up]
        child{node[circle,scale=0.75,fill]{}[grow=up]
          child{node[circle,scale=0.75,fill,red]{}[grow=up]
          }
          child{node[circle,scale=0.75,fill]{}[grow=up]
          }
          child{node[circle,scale=0.75,fill]{}[grow=up]
          }
        }
        child{node[circle,scale=0.75,fill]{}[grow=up]
          child{node[circle,scale=0.75,fill,red]{}[grow=up]
          }
        }
        ;
      \end{tikzpicture}
    };
    \node[draw=black] (sn0x11632c0W45) at (45, -60) {\begin{tikzpicture}[scale=.2]
        \node[circle,scale=0.75,fill]{}[grow=up]
        child{node[circle,scale=0.75,fill]{}[grow=up]
          child{node[circle,scale=0.75,fill,red]{}[grow=up]
          }
          child{node[circle,scale=0.75,fill,red]{}[grow=up]
          }
          child{node[circle,scale=0.75,fill]{}[grow=up]
          }
        }
        child{node[circle,scale=0.75,fill]{}[grow=up]
        }
        ;
      \end{tikzpicture}
    };
    \draw (sn0x11632c0W45.south) -- (sn0x11624b0W36.north);
    \draw (sn0x1162a20W27.south) -- (sn0x1161800W27.north);
    \draw (sn0x1162a20W27.south) -- (sn0x11632c0W45.north);
    \node[draw=black] (sn0x1163090W36) at (36, -50) {\begin{tikzpicture}[scale=.2]
        \node[circle,scale=0.75,fill]{}[grow=up]
        child{node[circle,scale=0.75,fill]{}[grow=up]
          child{node[circle,scale=0.75,fill,red]{}[grow=up]
          }
          child{node[circle,scale=0.75,fill,red]{}[grow=up]
          }
          child{node[circle,scale=0.75,fill]{}[grow=up]
          }
        }
        child{node[circle,scale=0.75,fill]{}[grow=up]
          child{node[circle,scale=0.75,fill]{}[grow=up]
          }
        }
        ;
      \end{tikzpicture}
    };
    \draw (sn0x1163090W36.south) -- (sn0x1162160W36.north);
    \draw (sn0x1163090W36.south) -- (sn0x1161800W27.north);
    \draw (sn0x115eed0W18.south) -- (sn0x115f660W18.north);
    \draw (sn0x115eed0W18.south) -- (sn0x1162a20W27.north);
    \draw (sn0x115eed0W18.south) -- (sn0x1163090W36.north);
    \draw (sn0x115e1f0W9.south) -- (sn0x115e9e0W9.north);
    \draw (sn0x115e1f0W9.south) -- (sn0x115eed0W18.north);
    \node[draw=black] (sn0x115e5a0W18) at (18, -30) {\begin{tikzpicture}[scale=.2]
        \node[circle,scale=0.75,fill]{}[grow=up]
        child{node[circle,scale=0.75,fill]{}[grow=up]
          child{node[circle,scale=0.75,fill]{}[grow=up]
            child{node[circle,scale=0.75,fill,red]{}[grow=up]
            }
          }
          child{node[circle,scale=0.75,fill]{}[grow=up]
          }
          child{node[circle,scale=0.75,fill]{}[grow=up]
          }
        }
        child{node[circle,scale=0.75,fill]{}[grow=up]
          child{node[circle,scale=0.75,fill]{}[grow=up]
            child{node[circle,scale=0.75,fill,red]{}[grow=up]
            }
          }
        }
        ;
      \end{tikzpicture}
    };
    \node[draw=black] (sn0x1163900W27) at (27, -40) {\begin{tikzpicture}[scale=.2]
        \node[circle,scale=0.75,fill]{}[grow=up]
        child{node[circle,scale=0.75,fill]{}[grow=up]
          child{node[circle,scale=0.75,fill]{}[grow=up]
            child{node[circle,scale=0.75,fill,red]{}[grow=up]
            }
          }
          child{node[circle,scale=0.75,fill]{}[grow=up]
          }
          child{node[circle,scale=0.75,fill]{}[grow=up]
          }
        }
        child{node[circle,scale=0.75,fill]{}[grow=up]
          child{node[circle,scale=0.75,fill,red]{}[grow=up]
          }
        }
        ;
      \end{tikzpicture}
    };
    \node[draw=black] (sn0x1164010W45) at (45, -50) {\begin{tikzpicture}[scale=.2]
        \node[circle,scale=0.75,fill]{}[grow=up]
        child{node[circle,scale=0.75,fill]{}[grow=up]
          child{node[circle,scale=0.75,fill]{}[grow=up]
            child{node[circle,scale=0.75,fill,red]{}[grow=up]
            }
          }
          child{node[circle,scale=0.75,fill,red]{}[grow=up]
          }
          child{node[circle,scale=0.75,fill]{}[grow=up]
          }
        }
        child{node[circle,scale=0.75,fill]{}[grow=up]
        }
        ;
      \end{tikzpicture}
    };
    \node[draw=black] (sn0x11648d0W54) at (54, -60) {\begin{tikzpicture}[scale=.2]
        \node[circle,scale=0.75,fill]{}[grow=up]
        child{node[circle,scale=0.75,fill]{}[grow=up]
          child{node[circle,scale=0.75,fill]{}[grow=up]
            child{node[circle,scale=0.75,fill,red]{}[grow=up]
            }
          }
          child{node[circle,scale=0.75,fill,red]{}[grow=up]
          }
        }
        child{node[circle,scale=0.75,fill]{}[grow=up]
        }
        ;
      \end{tikzpicture}
    };
    \draw (sn0x11648d0W54.south) -- (sn0x1160630W18.north);
    \draw (sn0x11648d0W54.south) -- (sn0x11624b0W36.north);
    \draw (sn0x1164010W45.south) -- (sn0x11648d0W54.north);
    \draw (sn0x1164010W45.south) -- (sn0x11632c0W45.north);
    \draw (sn0x1163900W27.south) -- (sn0x1164010W45.north);
    \draw (sn0x1163900W27.south) -- (sn0x1162a20W27.north);
    \node[draw=black] (sn0x1163b50W36) at (36, -40) {\begin{tikzpicture}[scale=.2]
        \node[circle,scale=0.75,fill]{}[grow=up]
        child{node[circle,scale=0.75,fill]{}[grow=up]
          child{node[circle,scale=0.75,fill]{}[grow=up]
            child{node[circle,scale=0.75,fill,red]{}[grow=up]
            }
          }
          child{node[circle,scale=0.75,fill,red]{}[grow=up]
          }
          child{node[circle,scale=0.75,fill]{}[grow=up]
          }
        }
        child{node[circle,scale=0.75,fill]{}[grow=up]
          child{node[circle,scale=0.75,fill]{}[grow=up]
          }
        }
        ;
      \end{tikzpicture}
    };
    \node[draw=black] (sn0x11641c0W54) at (54, -50) {\begin{tikzpicture}[scale=.2]
        \node[circle,scale=0.75,fill]{}[grow=up]
        child{node[circle,scale=0.75,fill]{}[grow=up]
          child{node[circle,scale=0.75,fill]{}[grow=up]
            child{node[circle,scale=0.75,fill,red]{}[grow=up]
            }
          }
          child{node[circle,scale=0.75,fill,red]{}[grow=up]
          }
        }
        child{node[circle,scale=0.75,fill]{}[grow=up]
          child{node[circle,scale=0.75,fill]{}[grow=up]
          }
        }
        ;
      \end{tikzpicture}
    };
    \draw (sn0x11641c0W54.south) -- (sn0x115f7f0W18.north);
    \draw (sn0x11641c0W54.south) -- (sn0x1162160W36.north);
    \draw (sn0x11641c0W54.south) -- (sn0x1161800W27.north);
    \node[draw=black] (sn0x11650c0W63) at (63, -50) {\begin{tikzpicture}[scale=.2]
        \node[circle,scale=0.75,fill]{}[grow=up]
        child{node[circle,scale=0.75,fill]{}[grow=up]
          child{node[circle,scale=0.75,fill]{}[grow=up]
            child{node[circle,scale=0.75,fill,red]{}[grow=up]
            }
          }
          child{node[circle,scale=0.75,fill]{}[grow=up]
          }
        }
        child{node[circle,scale=0.75,fill]{}[grow=up]
          child{node[circle,scale=0.75,fill,red]{}[grow=up]
          }
        }
        ;
      \end{tikzpicture}
    };
    \draw (sn0x11650c0W63.south) -- (sn0x11648d0W54.north);
    \draw (sn0x11650c0W63.south) -- (sn0x1161800W27.north);
    \draw (sn0x1163b50W36.south) -- (sn0x11641c0W54.north);
    \draw (sn0x1163b50W36.south) -- (sn0x11650c0W63.north);
    \draw (sn0x1163b50W36.south) -- (sn0x1163090W36.north);
    \draw (sn0x1163b50W36.south) -- (sn0x1162a20W27.north);
    \draw (sn0x115e5a0W18.south) -- (sn0x1163900W27.north);
    \draw (sn0x115e5a0W18.south) -- (sn0x1163b50W36.north);
    \draw (sn0x115e5a0W18.south) -- (sn0x115eed0W18.north);
    \draw (sn0x115c080W9.south) -- (sn0x115e1f0W9.north);
    \draw (sn0x115c080W9.south) -- (sn0x115e5a0W18.north);
    \node[draw=black] (sn0x115c150W18) at (18, -20) {\begin{tikzpicture}[scale=.2]
        \node[circle,scale=0.75,fill]{}[grow=up]
        child{node[circle,scale=0.75,fill]{}[grow=up]
          child{node[circle,scale=0.75,fill]{}[grow=up]
            child{node[circle,scale=0.75,fill]{}[grow=up]
            }
          }
          child{node[circle,scale=0.75,fill]{}[grow=up]
          }
          child{node[circle,scale=0.75,fill]{}[grow=up]
          }
        }
        child{node[circle,scale=0.75,fill]{}[grow=up]
          child{node[circle,scale=0.75,fill]{}[grow=up]
            child{node[circle,scale=0.75,fill,red]{}[grow=up]
            }
            child{node[circle,scale=0.75,fill,red]{}[grow=up]
            }
          }
        }
        ;
      \end{tikzpicture}
    };
    \draw (sn0x115c150W18.south) -- (sn0x115e5a0W18.north);
    \node[draw=black] (sn0x115dad0W27) at (27, -20) {\begin{tikzpicture}[scale=.2]
        \node[circle,scale=0.75,fill]{}[grow=up]
        child{node[circle,scale=0.75,fill]{}[grow=up]
          child{node[circle,scale=0.75,fill]{}[grow=up]
            child{node[circle,scale=0.75,fill,red]{}[grow=up]
            }
          }
          child{node[circle,scale=0.75,fill]{}[grow=up]
          }
          child{node[circle,scale=0.75,fill]{}[grow=up]
          }
        }
        child{node[circle,scale=0.75,fill]{}[grow=up]
          child{node[circle,scale=0.75,fill]{}[grow=up]
            child{node[circle,scale=0.75,fill,red]{}[grow=up]
            }
            child{node[circle,scale=0.75,fill]{}[grow=up]
            }
          }
        }
        ;
      \end{tikzpicture}
    };
    \node[draw=black] (sn0x1165740W27) at (27, -30) {\begin{tikzpicture}[scale=.2]
        \node[circle,scale=0.75,fill]{}[grow=up]
        child{node[circle,scale=0.75,fill]{}[grow=up]
          child{node[circle,scale=0.75,fill]{}[grow=up]
            child{node[circle,scale=0.75,fill,red]{}[grow=up]
            }
            child{node[circle,scale=0.75,fill,red]{}[grow=up]
            }
          }
        }
        child{node[circle,scale=0.75,fill]{}[grow=up]
          child{node[circle,scale=0.75,fill]{}[grow=up]
          }
          child{node[circle,scale=0.75,fill]{}[grow=up]
          }
          child{node[circle,scale=0.75,fill]{}[grow=up]
          }
        }
        ;
      \end{tikzpicture}
    };
    \draw (sn0x1165740W27.south) -- (sn0x115eed0W18.north);
    \draw (sn0x115dad0W27.south) -- (sn0x115e5a0W18.north);
    \draw (sn0x115dad0W27.south) -- (sn0x1165740W27.north);
    \draw (sn0x115b5e0W9.south) -- (sn0x115c080W9.north);
    \draw (sn0x115b5e0W9.south) -- (sn0x115c150W18.north);
    \draw (sn0x115b5e0W9.south) -- (sn0x115dad0W27.north);
  \end{scope}
  \newcommand{\shortnode}[4]{
    \node[draw=black] (s#1#2) at (#3, #4){$#1/#2$};
  }
  \begin{scope}[xshift=70cm,xscale=0.5]
    \shortnode{5}{6}{0}{-10};

    \shortnode{5}{5}{0}{-20};
    \shortnode{4}{6}{10}{-20};
    \draw[->] (s56) -- (s55);
    \draw[->] (s56) -- (s46);

    \shortnode{5}{4}{0}{-30};
    \shortnode{4}{5}{10}{-30};
    \draw[->] (s55) -- (s54);
    \draw[->] (s55) -- (s45);
    \draw[->] (s46) -- (s45);

    \shortnode{5}{3}{0}{-40};
    \shortnode{4}{4}{10}{-40};
    \draw[->] (s54) -- (s53);
    \draw[->] (s54) -- (s44);
    \draw[->] (s45) -- (s44);

    \shortnode{5}{2}{0}{-50};
    \shortnode{4}{3}{10}{-50};
    \shortnode{3}{4}{20}{-50};
    \draw[->] (s53) -- (s52);
    \draw[->] (s53) -- (s43);
    \draw[->] (s44) -- (s43);
    \draw[->] (s44) -- (s34);
    
    \shortnode{5}{1}{0}{-60};
    \shortnode{4}{2}{10}{-60};
    \shortnode{3}{3}{20}{-60};
    \draw[->] (s52) -- (s51);
    \draw[->] (s52) -- (s42);
    \draw[->] (s43) -- (s42);
    \draw[->] (s43) -- (s33);
    \draw[->] (s34) -- (s33);

    \shortnode{5}{0}{0}{-70};
    \shortnode{4}{1}{10}{-70};
    \shortnode{3}{2}{20}{-70};
    \draw[->] (s51) -- (s50);
    \draw[->] (s51) -- (s41);
    \draw[->] (s42) -- (s41);
    \draw[->] (s42) -- (s32);
    \draw[->] (s33) -- (s32);

    \shortnode{4}{0}{0}{-80};
    \shortnode{3}{1}{10}{-80};
    \draw[->] (s50) -- (s40);
    \draw[->] (s41) -- (s40);
    \draw[->] (s41) -- (s31);
    \draw[->] (s32) -- (s31);

    \shortnode{3}{0}{0}{-90};
    \shortnode{2}{1}{10}{-90};
    \draw[->] (s40) -- (s30);
    \draw[->] (s31) -- (s30);
    \draw[->] (s31) -- (s21);

    \shortnode{2}{0}{0}{-100};
    \draw[->] (s30) -- (s20);
    \draw[->] (s21) -- (s20);

    \shortnode{1}{0}{0}{-110};
    \draw[->] (s20) -- (s10);
  \end{scope}
\end{tikzpicture}

%%% Local Variables:
%%% TeX-master: "../thesis.tex"
%%% End:

\section{The current TikZ-file}

\renewcommand{\leveltopI}{-15cm + \leveltop}
\renewcommand{\leveltopII}{-15cm + \leveltopI}
\renewcommand{\leveltopIII}{-15cm + \leveltopII}
\renewcommand{\leveltopIIII}{-15cm + \leveltopIII}
\renewcommand{\leveltopIIIII}{-15cm + \leveltopIIII}
\renewcommand{\leveltopIIIIII}{-15cm + \leveltopIIIII}
\renewcommand{\leveltopIIIIIII}{-15cm + \leveltopIIIIII}
\renewcommand{\leveltopIIIIIIII}{-15cm + \leveltopIIIIIII}
\renewcommand{\leveltopIIIIIIIII}{-15cm + \leveltopIIIIIIII}
\renewcommand{\leveltopIIIIIIIIII}{-15cm + \leveltopIIIIIIIII}
\renewcommand{\leveltopIIIIIIIIIII}{-15cm + \leveltopIIIIIIIIII}
\begin{tikzpicture}[scale=.2, anchor=south]
\begin{scope}[yshift=\leveltopI cm]
\matrix (line1)[column sep=0.5cm] {
\node[draw=black, rectangle split,  rectangle split parts=4] (sn0xef11d0){
\footnotesize{100}
\nodepart{two}
\begin{tikzpicture}[scale=.2]
\node[circle, scale=0.75, fill] (tid0) at (3,1.5){};
\node[circle, scale=0.75, fill] (tid1) at (3,3){};
\node[circle, scale=0.75, fill] (tid2) at (0.75,4.5){};
\node[circle, scale=0.75, fill] (tid5) at (0.75,6){};
\node[circle, scale=0.75, fill] (tid9) at (0.75,7.5){};
\node[circle, scale=0.75, fill] (tid10) at (0.75,9){};
\draw[](tid9) -- (tid10);
\draw[](tid5) -- (tid9);
\draw[](tid2) -- (tid5);
\node[circle, scale=0.75, fill] (tid3) at (3,4.5){};
\node[circle, scale=0.75, fill, task_scheduled] (tid6) at (2.25,6){};
\node[circle, scale=0.75, fill, task_scheduled] (tid7) at (3.75,6){};
\draw[](tid3) -- (tid6);
\draw[](tid3) -- (tid7);
\node[circle, scale=0.75, fill] (tid4) at (5.25,4.5){};
\node[circle, scale=0.75, fill] (tid8) at (5.25,6){};
\draw[](tid4) -- (tid8);
\draw[](tid1) -- (tid2);
\draw[](tid1) -- (tid3);
\draw[](tid1) -- (tid4);
\draw[](tid0) -- (tid1);
\end{tikzpicture}
\nodepart{three}
\footnotesize{7.5791}
\nodepart{four}
\footnotesize{$50\:50$}
};
 & 
\\
};
\end{scope}
\begin{scope}[yshift=\leveltopII cm]
\matrix (line2)[column sep=0.5cm] {
\node[draw=black, rectangle split,  rectangle split parts=4] (sn0xef2a50){
\footnotesize{50}
\nodepart{two}
\begin{tikzpicture}[scale=.2]
\node[circle, scale=0.75, fill] (tid0) at (2.25,1.5){};
\node[circle, scale=0.75, fill] (tid1) at (2.25,3){};
\node[circle, scale=0.75, fill] (tid2) at (0.75,4.5){};
\node[circle, scale=0.75, fill] (tid5) at (0.75,6){};
\node[circle, scale=0.75, fill] (tid8) at (0.75,7.5){};
\node[circle, scale=0.75, fill] (tid9) at (0.75,9){};
\draw[](tid8) -- (tid9);
\draw[](tid5) -- (tid8);
\draw[](tid2) -- (tid5);
\node[circle, scale=0.75, fill] (tid3) at (2.25,4.5){};
\node[circle, scale=0.75, fill, task_scheduled] (tid6) at (2.25,6){};
\draw[](tid3) -- (tid6);
\node[circle, scale=0.75, fill] (tid4) at (3.75,4.5){};
\node[circle, scale=0.75, fill, task_scheduled] (tid7) at (3.75,6){};
\draw[](tid4) -- (tid7);
\draw[](tid1) -- (tid2);
\draw[](tid1) -- (tid3);
\draw[](tid1) -- (tid4);
\draw[](tid0) -- (tid1);
\end{tikzpicture}
\nodepart{three}
\footnotesize{7.21191}
\nodepart{four}
\footnotesize{$50\:50$}
};
 & 
\node[draw=black, rectangle split,  rectangle split parts=4] (sn0xef2540){
\footnotesize{50}
\nodepart{two}
\begin{tikzpicture}[scale=.2]
\node[circle, scale=0.75, fill] (tid0) at (2.25,1.5){};
\node[circle, scale=0.75, fill] (tid1) at (2.25,3){};
\node[circle, scale=0.75, fill] (tid2) at (0.75,4.5){};
\node[circle, scale=0.75, fill] (tid5) at (0.75,6){};
\node[circle, scale=0.75, fill] (tid8) at (0.75,7.5){};
\node[circle, scale=0.75, fill, task_scheduled] (tid9) at (0.75,9){};
\draw[](tid8) -- (tid9);
\draw[](tid5) -- (tid8);
\draw[](tid2) -- (tid5);
\node[circle, scale=0.75, fill] (tid3) at (2.25,4.5){};
\node[circle, scale=0.75, fill, task_scheduled] (tid6) at (2.25,6){};
\draw[](tid3) -- (tid6);
\node[circle, scale=0.75, fill] (tid4) at (3.75,4.5){};
\node[circle, scale=0.75, fill] (tid7) at (3.75,6){};
\draw[](tid4) -- (tid7);
\draw[](tid1) -- (tid2);
\draw[](tid1) -- (tid3);
\draw[](tid1) -- (tid4);
\draw[](tid0) -- (tid1);
\end{tikzpicture}
\nodepart{three}
\footnotesize{6.94629}
\nodepart{four}
\footnotesize{$25\:25\:25\:25$}
};
 & 
\\
};
\end{scope}
\begin{scope}[yshift=\leveltopIII cm]
\matrix (line3)[column sep=0.5cm] {
\node[draw=black, rectangle split,  rectangle split parts=4] (sn0xef2f20){
\footnotesize{25}
\nodepart{two}
\begin{tikzpicture}[scale=.2]
\node[circle, scale=0.75, fill] (tid0) at (2.25,1.5){};
\node[circle, scale=0.75, fill] (tid1) at (2.25,3){};
\node[circle, scale=0.75, fill] (tid2) at (0.75,4.5){};
\node[circle, scale=0.75, fill] (tid5) at (0.75,6){};
\node[circle, scale=0.75, fill] (tid7) at (0.75,7.5){};
\node[circle, scale=0.75, fill] (tid8) at (0.75,9){};
\draw[](tid7) -- (tid8);
\draw[](tid5) -- (tid7);
\draw[](tid2) -- (tid5);
\node[circle, scale=0.75, fill] (tid3) at (2.25,4.5){};
\node[circle, scale=0.75, fill, task_scheduled] (tid6) at (2.25,6){};
\draw[](tid3) -- (tid6);
\node[circle, scale=0.75, fill, task_scheduled] (tid4) at (3.75,4.5){};
\draw[](tid1) -- (tid2);
\draw[](tid1) -- (tid3);
\draw[](tid1) -- (tid4);
\draw[](tid0) -- (tid1);
\end{tikzpicture}
\nodepart{three}
\footnotesize{6.83789}
\nodepart{four}
\footnotesize{$50\:25\:25$}
};
 & 
\node[draw=black, rectangle split,  rectangle split parts=4] (sn0xef3e70){
\footnotesize{37.5}
\nodepart{two}
\begin{tikzpicture}[scale=.2]
\node[circle, scale=0.75, fill] (tid0) at (2.25,1.5){};
\node[circle, scale=0.75, fill] (tid1) at (2.25,3){};
\node[circle, scale=0.75, fill] (tid2) at (0.75,4.5){};
\node[circle, scale=0.75, fill] (tid5) at (0.75,6){};
\node[circle, scale=0.75, fill] (tid7) at (0.75,7.5){};
\node[circle, scale=0.75, fill, task_scheduled] (tid8) at (0.75,9){};
\draw[](tid7) -- (tid8);
\draw[](tid5) -- (tid7);
\draw[](tid2) -- (tid5);
\node[circle, scale=0.75, fill] (tid3) at (2.25,4.5){};
\node[circle, scale=0.75, fill, task_scheduled] (tid6) at (2.25,6){};
\draw[](tid3) -- (tid6);
\node[circle, scale=0.75, fill] (tid4) at (3.75,4.5){};
\draw[](tid1) -- (tid2);
\draw[](tid1) -- (tid3);
\draw[](tid1) -- (tid4);
\draw[](tid0) -- (tid1);
\end{tikzpicture}
\nodepart{three}
\footnotesize{6.58594}
\nodepart{four}
\footnotesize{$50\:25\:25$}
};
 & 
\node[draw=black, rectangle split,  rectangle split parts=4] (sn0xefa920){
\footnotesize{12.5}
\nodepart{two}
\begin{tikzpicture}[scale=.2]
\node[circle, scale=0.75, fill] (tid0) at (2.25,1.5){};
\node[circle, scale=0.75, fill] (tid1) at (2.25,3){};
\node[circle, scale=0.75, fill] (tid2) at (0.75,4.5){};
\node[circle, scale=0.75, fill] (tid5) at (0.75,6){};
\node[circle, scale=0.75, fill] (tid7) at (0.75,7.5){};
\node[circle, scale=0.75, fill, task_scheduled] (tid8) at (0.75,9){};
\draw[](tid7) -- (tid8);
\draw[](tid5) -- (tid7);
\draw[](tid2) -- (tid5);
\node[circle, scale=0.75, fill] (tid3) at (2.25,4.5){};
\node[circle, scale=0.75, fill] (tid6) at (2.25,6){};
\draw[](tid3) -- (tid6);
\node[circle, scale=0.75, fill, task_scheduled] (tid4) at (3.75,4.5){};
\draw[](tid1) -- (tid2);
\draw[](tid1) -- (tid3);
\draw[](tid1) -- (tid4);
\draw[](tid0) -- (tid1);
\end{tikzpicture}
\nodepart{three}
\footnotesize{6.57617}
\nodepart{four}
\footnotesize{$50\:25\:25$}
};
 & 
\node[draw=black, rectangle split,  rectangle split parts=4] (sn0xefabd0){
\footnotesize{12.5}
\nodepart{two}
\begin{tikzpicture}[scale=.2]
\node[circle, scale=0.75, fill] (tid0) at (2.25,1.5){};
\node[circle, scale=0.75, fill] (tid1) at (2.25,3){};
\node[circle, scale=0.75, fill] (tid2) at (0.75,4.5){};
\node[circle, scale=0.75, fill] (tid5) at (0.75,6){};
\node[circle, scale=0.75, fill] (tid8) at (0.75,7.5){};
\draw[](tid5) -- (tid8);
\draw[](tid2) -- (tid5);
\node[circle, scale=0.75, fill] (tid3) at (2.25,4.5){};
\node[circle, scale=0.75, fill, task_scheduled] (tid6) at (2.25,6){};
\draw[](tid3) -- (tid6);
\node[circle, scale=0.75, fill] (tid4) at (3.75,4.5){};
\node[circle, scale=0.75, fill, task_scheduled] (tid7) at (3.75,6){};
\draw[](tid4) -- (tid7);
\draw[](tid1) -- (tid2);
\draw[](tid1) -- (tid3);
\draw[](tid1) -- (tid4);
\draw[](tid0) -- (tid1);
\end{tikzpicture}
\nodepart{three}
\footnotesize{6.38281}
\nodepart{four}
\footnotesize{$50\:50$}
};
 & 
\node[draw=black, rectangle split,  rectangle split parts=4] (sn0xefb500){
\footnotesize{12.5}
\nodepart{two}
\begin{tikzpicture}[scale=.2]
\node[circle, scale=0.75, fill] (tid0) at (2.25,1.5){};
\node[circle, scale=0.75, fill] (tid1) at (2.25,3){};
\node[circle, scale=0.75, fill] (tid2) at (0.75,4.5){};
\node[circle, scale=0.75, fill] (tid5) at (0.75,6){};
\node[circle, scale=0.75, fill, task_scheduled] (tid8) at (0.75,7.5){};
\draw[](tid5) -- (tid8);
\draw[](tid2) -- (tid5);
\node[circle, scale=0.75, fill] (tid3) at (2.25,4.5){};
\node[circle, scale=0.75, fill, task_scheduled] (tid6) at (2.25,6){};
\draw[](tid3) -- (tid6);
\node[circle, scale=0.75, fill] (tid4) at (3.75,4.5){};
\node[circle, scale=0.75, fill] (tid7) at (3.75,6){};
\draw[](tid4) -- (tid7);
\draw[](tid1) -- (tid2);
\draw[](tid1) -- (tid3);
\draw[](tid1) -- (tid4);
\draw[](tid0) -- (tid1);
\end{tikzpicture}
\nodepart{three}
\footnotesize{6.24023}
\nodepart{four}
\footnotesize{$25\:25\:50$}
};
 & 
\\
};
\end{scope}
\begin{scope}[yshift=\leveltopIIII cm]
\matrix (line4)[column sep=0.5cm] {
\node[draw=black, rectangle split,  rectangle split parts=4] (sn0xef43a0){
\footnotesize{18.75}
\nodepart{two}
\begin{tikzpicture}[scale=.2]
\node[circle, scale=0.75, fill] (tid0) at (1.5,1.5){};
\node[circle, scale=0.75, fill] (tid1) at (1.5,3){};
\node[circle, scale=0.75, fill] (tid2) at (0.75,4.5){};
\node[circle, scale=0.75, fill] (tid4) at (0.75,6){};
\node[circle, scale=0.75, fill] (tid6) at (0.75,7.5){};
\node[circle, scale=0.75, fill, task_scheduled] (tid7) at (0.75,9){};
\draw[](tid6) -- (tid7);
\draw[](tid4) -- (tid6);
\draw[](tid2) -- (tid4);
\node[circle, scale=0.75, fill] (tid3) at (2.25,4.5){};
\node[circle, scale=0.75, fill, task_scheduled] (tid5) at (2.25,6){};
\draw[](tid3) -- (tid5);
\draw[](tid1) -- (tid2);
\draw[](tid1) -- (tid3);
\draw[](tid0) -- (tid1);
\end{tikzpicture}
\nodepart{three}
\footnotesize{6.25}
\nodepart{four}
\footnotesize{$50\:50$}
};
 & 
\node[draw=black, rectangle split,  rectangle split parts=4] (sn0xef4560){
\footnotesize{6.25}
\nodepart{two}
\begin{tikzpicture}[scale=.2]
\node[circle, scale=0.75, fill] (tid0) at (2.25,1.5){};
\node[circle, scale=0.75, fill] (tid1) at (2.25,3){};
\node[circle, scale=0.75, fill] (tid2) at (0.75,4.5){};
\node[circle, scale=0.75, fill] (tid5) at (0.75,6){};
\node[circle, scale=0.75, fill] (tid6) at (0.75,7.5){};
\node[circle, scale=0.75, fill] (tid7) at (0.75,9){};
\draw[](tid6) -- (tid7);
\draw[](tid5) -- (tid6);
\draw[](tid2) -- (tid5);
\node[circle, scale=0.75, fill, task_scheduled] (tid3) at (2.25,4.5){};
\node[circle, scale=0.75, fill, task_scheduled] (tid4) at (3.75,4.5){};
\draw[](tid1) -- (tid2);
\draw[](tid1) -- (tid3);
\draw[](tid1) -- (tid4);
\draw[](tid0) -- (tid1);
\end{tikzpicture}
\nodepart{three}
\footnotesize{6.5625}
\nodepart{four}
\footnotesize{$1$}
};
 & 
\node[draw=black, rectangle split,  rectangle split parts=4] (sn0xef4790){
\footnotesize{25}
\nodepart{two}
\begin{tikzpicture}[scale=.2]
\node[circle, scale=0.75, fill] (tid0) at (2.25,1.5){};
\node[circle, scale=0.75, fill] (tid1) at (2.25,3){};
\node[circle, scale=0.75, fill] (tid2) at (0.75,4.5){};
\node[circle, scale=0.75, fill] (tid5) at (0.75,6){};
\node[circle, scale=0.75, fill] (tid6) at (0.75,7.5){};
\node[circle, scale=0.75, fill, task_scheduled] (tid7) at (0.75,9){};
\draw[](tid6) -- (tid7);
\draw[](tid5) -- (tid6);
\draw[](tid2) -- (tid5);
\node[circle, scale=0.75, fill, task_scheduled] (tid3) at (2.25,4.5){};
\node[circle, scale=0.75, fill] (tid4) at (3.75,4.5){};
\draw[](tid1) -- (tid2);
\draw[](tid1) -- (tid3);
\draw[](tid1) -- (tid4);
\draw[](tid0) -- (tid1);
\end{tikzpicture}
\nodepart{three}
\footnotesize{6.28906}
\nodepart{four}
\footnotesize{$50\:25\:25$}
};
 & 
\node[draw=black, rectangle split,  rectangle split parts=4] (sn0xef8300){
\footnotesize{18.75}
\nodepart{two}
\begin{tikzpicture}[scale=.2]
\node[circle, scale=0.75, fill] (tid0) at (2.25,1.5){};
\node[circle, scale=0.75, fill] (tid1) at (2.25,3){};
\node[circle, scale=0.75, fill] (tid2) at (0.75,4.5){};
\node[circle, scale=0.75, fill] (tid5) at (0.75,6){};
\node[circle, scale=0.75, fill] (tid7) at (0.75,7.5){};
\draw[](tid5) -- (tid7);
\draw[](tid2) -- (tid5);
\node[circle, scale=0.75, fill] (tid3) at (2.25,4.5){};
\node[circle, scale=0.75, fill, task_scheduled] (tid6) at (2.25,6){};
\draw[](tid3) -- (tid6);
\node[circle, scale=0.75, fill, task_scheduled] (tid4) at (3.75,4.5){};
\draw[](tid1) -- (tid2);
\draw[](tid1) -- (tid3);
\draw[](tid1) -- (tid4);
\draw[](tid0) -- (tid1);
\end{tikzpicture}
\nodepart{three}
\footnotesize{5.97656}
\nodepart{four}
\footnotesize{$50\:25\:25$}
};
 & 
\node[draw=black, rectangle split,  rectangle split parts=4] (sn0xef90f0){
\footnotesize{18.75}
\nodepart{two}
\begin{tikzpicture}[scale=.2]
\node[circle, scale=0.75, fill] (tid0) at (2.25,1.5){};
\node[circle, scale=0.75, fill] (tid1) at (2.25,3){};
\node[circle, scale=0.75, fill] (tid2) at (0.75,4.5){};
\node[circle, scale=0.75, fill] (tid5) at (0.75,6){};
\node[circle, scale=0.75, fill, task_scheduled] (tid7) at (0.75,7.5){};
\draw[](tid5) -- (tid7);
\draw[](tid2) -- (tid5);
\node[circle, scale=0.75, fill] (tid3) at (2.25,4.5){};
\node[circle, scale=0.75, fill, task_scheduled] (tid6) at (2.25,6){};
\draw[](tid3) -- (tid6);
\node[circle, scale=0.75, fill] (tid4) at (3.75,4.5){};
\draw[](tid1) -- (tid2);
\draw[](tid1) -- (tid3);
\draw[](tid1) -- (tid4);
\draw[](tid0) -- (tid1);
\end{tikzpicture}
\nodepart{three}
\footnotesize{5.78906}
\nodepart{four}
\footnotesize{$50\:25\:25$}
};
 & 
\node[draw=black, rectangle split,  rectangle split parts=4] (sn0xefbc60){
\footnotesize{6.25}
\nodepart{two}
\begin{tikzpicture}[scale=.2]
\node[circle, scale=0.75, fill] (tid0) at (2.25,1.5){};
\node[circle, scale=0.75, fill] (tid1) at (2.25,3){};
\node[circle, scale=0.75, fill] (tid2) at (0.75,4.5){};
\node[circle, scale=0.75, fill] (tid5) at (0.75,6){};
\node[circle, scale=0.75, fill, task_scheduled] (tid7) at (0.75,7.5){};
\draw[](tid5) -- (tid7);
\draw[](tid2) -- (tid5);
\node[circle, scale=0.75, fill] (tid3) at (2.25,4.5){};
\node[circle, scale=0.75, fill] (tid6) at (2.25,6){};
\draw[](tid3) -- (tid6);
\node[circle, scale=0.75, fill, task_scheduled] (tid4) at (3.75,4.5){};
\draw[](tid1) -- (tid2);
\draw[](tid1) -- (tid3);
\draw[](tid1) -- (tid4);
\draw[](tid0) -- (tid1);
\end{tikzpicture}
\nodepart{three}
\footnotesize{5.82812}
\nodepart{four}
\footnotesize{$50\:50$}
};
 & 
\node[draw=black, rectangle split,  rectangle split parts=4] (sn0xefc0b0){
\footnotesize{6.25}
\nodepart{two}
\begin{tikzpicture}[scale=.2]
\node[circle, scale=0.75, fill] (tid0) at (2.25,1.5){};
\node[circle, scale=0.75, fill] (tid1) at (2.25,3){};
\node[circle, scale=0.75, fill] (tid2) at (0.75,4.5){};
\node[circle, scale=0.75, fill, task_scheduled] (tid5) at (0.75,6){};
\draw[](tid2) -- (tid5);
\node[circle, scale=0.75, fill] (tid3) at (2.25,4.5){};
\node[circle, scale=0.75, fill, task_scheduled] (tid6) at (2.25,6){};
\draw[](tid3) -- (tid6);
\node[circle, scale=0.75, fill] (tid4) at (3.75,4.5){};
\node[circle, scale=0.75, fill] (tid7) at (3.75,6){};
\draw[](tid4) -- (tid7);
\draw[](tid1) -- (tid2);
\draw[](tid1) -- (tid3);
\draw[](tid1) -- (tid4);
\draw[](tid0) -- (tid1);
\end{tikzpicture}
\nodepart{three}
\footnotesize{5.67188}
\nodepart{four}
\footnotesize{$50\:50$}
};
 & 
\\
};
\end{scope}
\begin{scope}[yshift=\leveltopIIIII cm]
\matrix (line5)[column sep=0.5cm] {
\node[draw=black, rectangle split,  rectangle split parts=4] (sn0xef48f0){
\footnotesize{28.125}
\nodepart{two}
\begin{tikzpicture}[scale=.2]
\node[circle, scale=0.75, fill] (tid0) at (1.5,1.5){};
\node[circle, scale=0.75, fill] (tid1) at (1.5,3){};
\node[circle, scale=0.75, fill] (tid2) at (0.75,4.5){};
\node[circle, scale=0.75, fill] (tid4) at (0.75,6){};
\node[circle, scale=0.75, fill] (tid5) at (0.75,7.5){};
\node[circle, scale=0.75, fill, task_scheduled] (tid6) at (0.75,9){};
\draw[](tid5) -- (tid6);
\draw[](tid4) -- (tid5);
\draw[](tid2) -- (tid4);
\node[circle, scale=0.75, fill, task_scheduled] (tid3) at (2.25,4.5){};
\draw[](tid1) -- (tid2);
\draw[](tid1) -- (tid3);
\draw[](tid0) -- (tid1);
\end{tikzpicture}
\nodepart{three}
\footnotesize{6.0625}
\nodepart{four}
\footnotesize{$50\:50$}
};
 & 
\node[draw=black, rectangle split,  rectangle split parts=4] (sn0xef5110){
\footnotesize{21.875}
\nodepart{two}
\begin{tikzpicture}[scale=.2]
\node[circle, scale=0.75, fill] (tid0) at (1.5,1.5){};
\node[circle, scale=0.75, fill] (tid1) at (1.5,3){};
\node[circle, scale=0.75, fill] (tid2) at (0.75,4.5){};
\node[circle, scale=0.75, fill] (tid4) at (0.75,6){};
\node[circle, scale=0.75, fill, task_scheduled] (tid6) at (0.75,7.5){};
\draw[](tid4) -- (tid6);
\draw[](tid2) -- (tid4);
\node[circle, scale=0.75, fill] (tid3) at (2.25,4.5){};
\node[circle, scale=0.75, fill, task_scheduled] (tid5) at (2.25,6){};
\draw[](tid3) -- (tid5);
\draw[](tid1) -- (tid2);
\draw[](tid1) -- (tid3);
\draw[](tid0) -- (tid1);
\end{tikzpicture}
\nodepart{three}
\footnotesize{5.4375}
\nodepart{four}
\footnotesize{$50\:50$}
};
 & 
\node[draw=black, rectangle split,  rectangle split parts=4] (sn0xef7880){
\footnotesize{10.9375}
\nodepart{two}
\begin{tikzpicture}[scale=.2]
\node[circle, scale=0.75, fill] (tid0) at (2.25,1.5){};
\node[circle, scale=0.75, fill] (tid1) at (2.25,3){};
\node[circle, scale=0.75, fill] (tid2) at (0.75,4.5){};
\node[circle, scale=0.75, fill] (tid5) at (0.75,6){};
\node[circle, scale=0.75, fill] (tid6) at (0.75,7.5){};
\draw[](tid5) -- (tid6);
\draw[](tid2) -- (tid5);
\node[circle, scale=0.75, fill, task_scheduled] (tid3) at (2.25,4.5){};
\node[circle, scale=0.75, fill, task_scheduled] (tid4) at (3.75,4.5){};
\draw[](tid1) -- (tid2);
\draw[](tid1) -- (tid3);
\draw[](tid1) -- (tid4);
\draw[](tid0) -- (tid1);
\end{tikzpicture}
\nodepart{three}
\footnotesize{5.625}
\nodepart{four}
\footnotesize{$1$}
};
 & 
\node[draw=black, rectangle split,  rectangle split parts=4] (sn0xef7c10){
\footnotesize{20.3125}
\nodepart{two}
\begin{tikzpicture}[scale=.2]
\node[circle, scale=0.75, fill] (tid0) at (2.25,1.5){};
\node[circle, scale=0.75, fill] (tid1) at (2.25,3){};
\node[circle, scale=0.75, fill] (tid2) at (0.75,4.5){};
\node[circle, scale=0.75, fill] (tid5) at (0.75,6){};
\node[circle, scale=0.75, fill, task_scheduled] (tid6) at (0.75,7.5){};
\draw[](tid5) -- (tid6);
\draw[](tid2) -- (tid5);
\node[circle, scale=0.75, fill, task_scheduled] (tid3) at (2.25,4.5){};
\node[circle, scale=0.75, fill] (tid4) at (3.75,4.5){};
\draw[](tid1) -- (tid2);
\draw[](tid1) -- (tid3);
\draw[](tid1) -- (tid4);
\draw[](tid0) -- (tid1);
\end{tikzpicture}
\nodepart{three}
\footnotesize{5.40625}
\nodepart{four}
\footnotesize{$50\:25\:25$}
};
 & 
\node[draw=black, rectangle split,  rectangle split parts=4] (sn0xef9e10){
\footnotesize{10.9375}
\nodepart{two}
\begin{tikzpicture}[scale=.2]
\node[circle, scale=0.75, fill] (tid0) at (2.25,1.5){};
\node[circle, scale=0.75, fill] (tid1) at (2.25,3){};
\node[circle, scale=0.75, fill] (tid2) at (0.75,4.5){};
\node[circle, scale=0.75, fill, task_scheduled] (tid5) at (0.75,6){};
\draw[](tid2) -- (tid5);
\node[circle, scale=0.75, fill] (tid3) at (2.25,4.5){};
\node[circle, scale=0.75, fill] (tid6) at (2.25,6){};
\draw[](tid3) -- (tid6);
\node[circle, scale=0.75, fill, task_scheduled] (tid4) at (3.75,4.5){};
\draw[](tid1) -- (tid2);
\draw[](tid1) -- (tid3);
\draw[](tid1) -- (tid4);
\draw[](tid0) -- (tid1);
\end{tikzpicture}
\nodepart{three}
\footnotesize{5.21875}
\nodepart{four}
\footnotesize{$50\:25\:25$}
};
 & 
\node[draw=black, rectangle split,  rectangle split parts=4] (sn0xefa190){
\footnotesize{7.8125}
\nodepart{two}
\begin{tikzpicture}[scale=.2]
\node[circle, scale=0.75, fill] (tid0) at (2.25,1.5){};
\node[circle, scale=0.75, fill] (tid1) at (2.25,3){};
\node[circle, scale=0.75, fill] (tid2) at (0.75,4.5){};
\node[circle, scale=0.75, fill, task_scheduled] (tid5) at (0.75,6){};
\draw[](tid2) -- (tid5);
\node[circle, scale=0.75, fill] (tid3) at (2.25,4.5){};
\node[circle, scale=0.75, fill, task_scheduled] (tid6) at (2.25,6){};
\draw[](tid3) -- (tid6);
\node[circle, scale=0.75, fill] (tid4) at (3.75,4.5){};
\draw[](tid1) -- (tid2);
\draw[](tid1) -- (tid3);
\draw[](tid1) -- (tid4);
\draw[](tid0) -- (tid1);
\end{tikzpicture}
\nodepart{three}
\footnotesize{5.125}
\nodepart{four}
\footnotesize{$1$}
};
 & 
\\
};
\end{scope}
\begin{scope}[yshift=\leveltopIIIIII cm]
\matrix (line6)[column sep=0.5cm] {
\node[draw=black, rectangle split,  rectangle split parts=4] (sn0xef5310){
\footnotesize{14.0625}
\nodepart{two}
\begin{tikzpicture}[scale=.2]
\node[circle, scale=0.75, fill] (tid0) at (0.75,1.5){};
\node[circle, scale=0.75, fill] (tid1) at (0.75,3){};
\node[circle, scale=0.75, fill] (tid2) at (0.75,4.5){};
\node[circle, scale=0.75, fill] (tid3) at (0.75,6){};
\node[circle, scale=0.75, fill] (tid4) at (0.75,7.5){};
\node[circle, scale=0.75, fill, task_scheduled] (tid5) at (0.75,9){};
\draw[](tid4) -- (tid5);
\draw[](tid3) -- (tid4);
\draw[](tid2) -- (tid3);
\draw[](tid1) -- (tid2);
\draw[](tid0) -- (tid1);
\end{tikzpicture}
\nodepart{three}
\footnotesize{6}
\nodepart{four}
\footnotesize{$1$}
};
 & 
\node[draw=black, rectangle split,  rectangle split parts=4] (sn0xef5750){
\footnotesize{46.0938}
\nodepart{two}
\begin{tikzpicture}[scale=.2]
\node[circle, scale=0.75, fill] (tid0) at (1.5,1.5){};
\node[circle, scale=0.75, fill] (tid1) at (1.5,3){};
\node[circle, scale=0.75, fill] (tid2) at (0.75,4.5){};
\node[circle, scale=0.75, fill] (tid4) at (0.75,6){};
\node[circle, scale=0.75, fill, task_scheduled] (tid5) at (0.75,7.5){};
\draw[](tid4) -- (tid5);
\draw[](tid2) -- (tid4);
\node[circle, scale=0.75, fill, task_scheduled] (tid3) at (2.25,4.5){};
\draw[](tid1) -- (tid2);
\draw[](tid1) -- (tid3);
\draw[](tid0) -- (tid1);
\end{tikzpicture}
\nodepart{three}
\footnotesize{5.125}
\nodepart{four}
\footnotesize{$50\:50$}
};
 & 
\node[draw=black, rectangle split,  rectangle split parts=4] (sn0xef7040){
\footnotesize{16.4062}
\nodepart{two}
\begin{tikzpicture}[scale=.2]
\node[circle, scale=0.75, fill] (tid0) at (1.5,1.5){};
\node[circle, scale=0.75, fill] (tid1) at (1.5,3){};
\node[circle, scale=0.75, fill] (tid2) at (0.75,4.5){};
\node[circle, scale=0.75, fill, task_scheduled] (tid4) at (0.75,6){};
\draw[](tid2) -- (tid4);
\node[circle, scale=0.75, fill] (tid3) at (2.25,4.5){};
\node[circle, scale=0.75, fill, task_scheduled] (tid5) at (2.25,6){};
\draw[](tid3) -- (tid5);
\draw[](tid1) -- (tid2);
\draw[](tid1) -- (tid3);
\draw[](tid0) -- (tid1);
\end{tikzpicture}
\nodepart{three}
\footnotesize{4.75}
\nodepart{four}
\footnotesize{$1$}
};
 & 
\node[draw=black, rectangle split,  rectangle split parts=4] (sn0xef7d10){
\footnotesize{7.8125}
\nodepart{two}
\begin{tikzpicture}[scale=.2]
\node[circle, scale=0.75, fill] (tid0) at (2.25,1.5){};
\node[circle, scale=0.75, fill] (tid1) at (2.25,3){};
\node[circle, scale=0.75, fill] (tid2) at (0.75,4.5){};
\node[circle, scale=0.75, fill] (tid5) at (0.75,6){};
\draw[](tid2) -- (tid5);
\node[circle, scale=0.75, fill, task_scheduled] (tid3) at (2.25,4.5){};
\node[circle, scale=0.75, fill, task_scheduled] (tid4) at (3.75,4.5){};
\draw[](tid1) -- (tid2);
\draw[](tid1) -- (tid3);
\draw[](tid1) -- (tid4);
\draw[](tid0) -- (tid1);
\end{tikzpicture}
\nodepart{three}
\footnotesize{4.75}
\nodepart{four}
\footnotesize{$1$}
};
 & 
\node[draw=black, rectangle split,  rectangle split parts=4] (sn0xef8590){
\footnotesize{15.625}
\nodepart{two}
\begin{tikzpicture}[scale=.2]
\node[circle, scale=0.75, fill] (tid0) at (2.25,1.5){};
\node[circle, scale=0.75, fill] (tid1) at (2.25,3){};
\node[circle, scale=0.75, fill] (tid2) at (0.75,4.5){};
\node[circle, scale=0.75, fill, task_scheduled] (tid5) at (0.75,6){};
\draw[](tid2) -- (tid5);
\node[circle, scale=0.75, fill, task_scheduled] (tid3) at (2.25,4.5){};
\node[circle, scale=0.75, fill] (tid4) at (3.75,4.5){};
\draw[](tid1) -- (tid2);
\draw[](tid1) -- (tid3);
\draw[](tid1) -- (tid4);
\draw[](tid0) -- (tid1);
\end{tikzpicture}
\nodepart{three}
\footnotesize{4.625}
\nodepart{four}
\footnotesize{$50\:50$}
};
 & 
\\
};
\end{scope}
\begin{scope}[yshift=\leveltopIIIIIII cm]
\matrix (line7)[column sep=0.5cm] {
\node[draw=black, rectangle split,  rectangle split parts=4] (sn0xef5850){
\footnotesize{37.1094}
\nodepart{two}
\begin{tikzpicture}[scale=.2]
\node[circle, scale=0.75, fill] (tid0) at (0.75,1.5){};
\node[circle, scale=0.75, fill] (tid1) at (0.75,3){};
\node[circle, scale=0.75, fill] (tid2) at (0.75,4.5){};
\node[circle, scale=0.75, fill] (tid3) at (0.75,6){};
\node[circle, scale=0.75, fill, task_scheduled] (tid4) at (0.75,7.5){};
\draw[](tid3) -- (tid4);
\draw[](tid2) -- (tid3);
\draw[](tid1) -- (tid2);
\draw[](tid0) -- (tid1);
\end{tikzpicture}
\nodepart{three}
\footnotesize{5}
\nodepart{four}
\footnotesize{$1$}
};
 & 
\node[draw=black, rectangle split,  rectangle split parts=4] (sn0xef64d0){
\footnotesize{55.0781}
\nodepart{two}
\begin{tikzpicture}[scale=.2]
\node[circle, scale=0.75, fill] (tid0) at (1.5,1.5){};
\node[circle, scale=0.75, fill] (tid1) at (1.5,3){};
\node[circle, scale=0.75, fill] (tid2) at (0.75,4.5){};
\node[circle, scale=0.75, fill, task_scheduled] (tid4) at (0.75,6){};
\draw[](tid2) -- (tid4);
\node[circle, scale=0.75, fill, task_scheduled] (tid3) at (2.25,4.5){};
\draw[](tid1) -- (tid2);
\draw[](tid1) -- (tid3);
\draw[](tid0) -- (tid1);
\end{tikzpicture}
\nodepart{three}
\footnotesize{4.25}
\nodepart{four}
\footnotesize{$50\:50$}
};
 & 
\node[draw=black, rectangle split,  rectangle split parts=4] (sn0xef8750){
\footnotesize{7.8125}
\nodepart{two}
\begin{tikzpicture}[scale=.2]
\node[circle, scale=0.75, fill] (tid0) at (2.25,1.5){};
\node[circle, scale=0.75, fill] (tid1) at (2.25,3){};
\node[circle, scale=0.75, fill, task_scheduled] (tid2) at (0.75,4.5){};
\node[circle, scale=0.75, fill, task_scheduled] (tid3) at (2.25,4.5){};
\node[circle, scale=0.75, fill] (tid4) at (3.75,4.5){};
\draw[](tid1) -- (tid2);
\draw[](tid1) -- (tid3);
\draw[](tid1) -- (tid4);
\draw[](tid0) -- (tid1);
\end{tikzpicture}
\nodepart{three}
\footnotesize{4}
\nodepart{four}
\footnotesize{$1$}
};
 & 
\\
};
\end{scope}
\begin{scope}[yshift=\leveltopIIIIIIII cm]
\matrix (line8)[column sep=0.5cm] {
\node[draw=black, rectangle split,  rectangle split parts=4] (sn0xef5c90){
\footnotesize{64.6484}
\nodepart{two}
\begin{tikzpicture}[scale=.2]
\node[circle, scale=0.75, fill] (tid0) at (0.75,1.5){};
\node[circle, scale=0.75, fill] (tid1) at (0.75,3){};
\node[circle, scale=0.75, fill] (tid2) at (0.75,4.5){};
\node[circle, scale=0.75, fill, task_scheduled] (tid3) at (0.75,6){};
\draw[](tid2) -- (tid3);
\draw[](tid1) -- (tid2);
\draw[](tid0) -- (tid1);
\end{tikzpicture}
\nodepart{three}
\footnotesize{4}
\nodepart{four}
\footnotesize{$1$}
};
 & 
\node[draw=black, rectangle split,  rectangle split parts=4] (sn0xef6940){
\footnotesize{35.3516}
\nodepart{two}
\begin{tikzpicture}[scale=.2]
\node[circle, scale=0.75, fill] (tid0) at (1.5,1.5){};
\node[circle, scale=0.75, fill] (tid1) at (1.5,3){};
\node[circle, scale=0.75, fill, task_scheduled] (tid2) at (0.75,4.5){};
\node[circle, scale=0.75, fill, task_scheduled] (tid3) at (2.25,4.5){};
\draw[](tid1) -- (tid2);
\draw[](tid1) -- (tid3);
\draw[](tid0) -- (tid1);
\end{tikzpicture}
\nodepart{three}
\footnotesize{3.5}
\nodepart{four}
\footnotesize{$1$}
};
 & 
\\
};
\end{scope}
\begin{scope}[yshift=\leveltopIIIIIIIII cm]
\matrix (line9)[column sep=0.5cm] {
\node[draw=black, rectangle split,  rectangle split parts=4] (sn0xef5950){
\footnotesize{100}
\nodepart{two}
\begin{tikzpicture}[scale=.2]
\node[circle, scale=0.75, fill] (tid0) at (0.75,1.5){};
\node[circle, scale=0.75, fill] (tid1) at (0.75,3){};
\node[circle, scale=0.75, fill, task_scheduled] (tid2) at (0.75,4.5){};
\draw[](tid1) -- (tid2);
\draw[](tid0) -- (tid1);
\end{tikzpicture}
\nodepart{three}
\footnotesize{3}
\nodepart{four}
\footnotesize{$1$}
};
 & 
\\
};
\end{scope}
\begin{scope}[yshift=\leveltopIIIIIIIIII cm]
\matrix (line10)[column sep=0.5cm] {
\node[draw=black, rectangle split,  rectangle split parts=4] (sn0xef5b30){
\footnotesize{100}
\nodepart{two}
\begin{tikzpicture}[scale=.2]
\node[circle, scale=0.75, fill] (tid0) at (0.75,1.5){};
\node[circle, scale=0.75, fill, task_scheduled] (tid1) at (0.75,3){};
\draw[](tid0) -- (tid1);
\end{tikzpicture}
\nodepart{three}
\footnotesize{2}
\nodepart{four}
\footnotesize{$1$}
};
 & 
\\
};
\end{scope}
\draw (sn0xef11d0.south) -- (sn0xef2a50.north);
\draw (sn0xef11d0.south) -- (sn0xef2540.north);
\draw (sn0xef2a50.south) -- (sn0xef2f20.north);
\draw (sn0xef2a50.south) -- (sn0xef3e70.north);
\draw (sn0xef2540.south) -- (sn0xefa920.north);
\draw (sn0xef2540.south) -- (sn0xef3e70.north);
\draw (sn0xef2540.south) -- (sn0xefabd0.north);
\draw (sn0xef2540.south) -- (sn0xefb500.north);
\draw (sn0xef2f20.south) -- (sn0xef43a0.north);
\draw (sn0xef2f20.south) -- (sn0xef4560.north);
\draw (sn0xef2f20.south) -- (sn0xef4790.north);
\draw (sn0xef3e70.south) -- (sn0xef4790.north);
\draw (sn0xef3e70.south) -- (sn0xef8300.north);
\draw (sn0xef3e70.south) -- (sn0xef90f0.north);
\draw (sn0xefa920.south) -- (sn0xef43a0.north);
\draw (sn0xefa920.south) -- (sn0xef8300.north);
\draw (sn0xefa920.south) -- (sn0xefbc60.north);
\draw (sn0xefabd0.south) -- (sn0xef8300.north);
\draw (sn0xefabd0.south) -- (sn0xef90f0.north);
\draw (sn0xefb500.south) -- (sn0xefbc60.north);
\draw (sn0xefb500.south) -- (sn0xef90f0.north);
\draw (sn0xefb500.south) -- (sn0xefc0b0.north);
\draw (sn0xef43a0.south) -- (sn0xef48f0.north);
\draw (sn0xef43a0.south) -- (sn0xef5110.north);
\draw (sn0xef4560.south) -- (sn0xef48f0.north);
\draw (sn0xef4790.south) -- (sn0xef48f0.north);
\draw (sn0xef4790.south) -- (sn0xef7880.north);
\draw (sn0xef4790.south) -- (sn0xef7c10.north);
\draw (sn0xef8300.south) -- (sn0xef5110.north);
\draw (sn0xef8300.south) -- (sn0xef7880.north);
\draw (sn0xef8300.south) -- (sn0xef7c10.north);
\draw (sn0xef90f0.south) -- (sn0xef7c10.north);
\draw (sn0xef90f0.south) -- (sn0xef9e10.north);
\draw (sn0xef90f0.south) -- (sn0xefa190.north);
\draw (sn0xefbc60.south) -- (sn0xef5110.north);
\draw (sn0xefbc60.south) -- (sn0xef9e10.north);
\draw (sn0xefc0b0.south) -- (sn0xef9e10.north);
\draw (sn0xefc0b0.south) -- (sn0xefa190.north);
\draw (sn0xef48f0.south) -- (sn0xef5310.north);
\draw (sn0xef48f0.south) -- (sn0xef5750.north);
\draw (sn0xef5110.south) -- (sn0xef5750.north);
\draw (sn0xef5110.south) -- (sn0xef7040.north);
\draw (sn0xef7880.south) -- (sn0xef5750.north);
\draw (sn0xef7c10.south) -- (sn0xef5750.north);
\draw (sn0xef7c10.south) -- (sn0xef7d10.north);
\draw (sn0xef7c10.south) -- (sn0xef8590.north);
\draw (sn0xef9e10.south) -- (sn0xef7040.north);
\draw (sn0xef9e10.south) -- (sn0xef7d10.north);
\draw (sn0xef9e10.south) -- (sn0xef8590.north);
\draw (sn0xefa190.south) -- (sn0xef8590.north);
\draw (sn0xef5310.south) -- (sn0xef5850.north);
\draw (sn0xef5750.south) -- (sn0xef5850.north);
\draw (sn0xef5750.south) -- (sn0xef64d0.north);
\draw (sn0xef7040.south) -- (sn0xef64d0.north);
\draw (sn0xef7d10.south) -- (sn0xef64d0.north);
\draw (sn0xef8590.south) -- (sn0xef64d0.north);
\draw (sn0xef8590.south) -- (sn0xef8750.north);
\draw (sn0xef5850.south) -- (sn0xef5c90.north);
\draw (sn0xef64d0.south) -- (sn0xef5c90.north);
\draw (sn0xef64d0.south) -- (sn0xef6940.north);
\draw (sn0xef8750.south) -- (sn0xef6940.north);
\draw (sn0xef5c90.south) -- (sn0xef5950.north);
\draw (sn0xef6940.south) -- (sn0xef5950.north);
\draw (sn0xef5950.south) -- (sn0xef5b30.north);
\end{tikzpicture}

%%% Local Variables:
%%% TeX-master: "thesis/thesis.tex"
%%% End: 
\renewcommand{\leveltopI}{-15cm + \leveltop}
\renewcommand{\leveltopII}{-15cm + \leveltopI}
\renewcommand{\leveltopIII}{-15cm + \leveltopII}
\renewcommand{\leveltopIIII}{-15cm + \leveltopIII}
\renewcommand{\leveltopIIIII}{-15cm + \leveltopIIII}
\renewcommand{\leveltopIIIIII}{-15cm + \leveltopIIIII}
\renewcommand{\leveltopIIIIIII}{-15cm + \leveltopIIIIII}
\renewcommand{\leveltopIIIIIIII}{-15cm + \leveltopIIIIIII}
\renewcommand{\leveltopIIIIIIIII}{-15cm + \leveltopIIIIIIII}
\renewcommand{\leveltopIIIIIIIIII}{-15cm + \leveltopIIIIIIIII}
\renewcommand{\leveltopIIIIIIIIIII}{-15cm + \leveltopIIIIIIIIII}
\begin{tikzpicture}[scale=.2, anchor=south]
\begin{scope}[yshift=\leveltopI cm]
\matrix (line1)[column sep=0.5cm] {
\node[draw=black, rectangle split,  rectangle split parts=4] (sn0xef1700){
\footnotesize{100}
\nodepart{two}
\begin{tikzpicture}[scale=.2]
\node[circle, scale=0.75, fill] (tid0) at (3,1.5){};
\node[circle, scale=0.75, fill] (tid1) at (3,3){};
\node[circle, scale=0.75, fill] (tid2) at (0.75,4.5){};
\node[circle, scale=0.75, fill] (tid5) at (0.75,6){};
\node[circle, scale=0.75, fill] (tid9) at (0.75,7.5){};
\node[circle, scale=0.75, fill] (tid10) at (0.75,9){};
\draw[](tid9) -- (tid10);
\draw[](tid5) -- (tid9);
\draw[](tid2) -- (tid5);
\node[circle, scale=0.75, fill] (tid3) at (3,4.5){};
\node[circle, scale=0.75, fill, task_scheduled] (tid6) at (2.25,6){};
\node[circle, scale=0.75, fill] (tid7) at (3.75,6){};
\draw[](tid3) -- (tid6);
\draw[](tid3) -- (tid7);
\node[circle, scale=0.75, fill] (tid4) at (5.25,4.5){};
\node[circle, scale=0.75, fill, task_scheduled] (tid8) at (5.25,6){};
\draw[](tid4) -- (tid8);
\draw[](tid1) -- (tid2);
\draw[](tid1) -- (tid3);
\draw[](tid1) -- (tid4);
\draw[](tid0) -- (tid1);
\end{tikzpicture}
\nodepart{three}
\footnotesize{7.59942}
\nodepart{four}
\footnotesize{$17\:17\:17\:25\:25$}
};
 & 
\\
};
\end{scope}
\begin{scope}[yshift=\leveltopII cm]
\matrix (line2)[column sep=0.5cm] {
\node[draw=black, rectangle split,  rectangle split parts=4] (sn0xefc570){
\footnotesize{16.6667}
\nodepart{two}
\begin{tikzpicture}[scale=.2]
\node[circle, scale=0.75, fill] (tid0) at (3,1.5){};
\node[circle, scale=0.75, fill] (tid1) at (3,3){};
\node[circle, scale=0.75, fill] (tid2) at (0.75,4.5){};
\node[circle, scale=0.75, fill] (tid5) at (0.75,6){};
\node[circle, scale=0.75, fill] (tid8) at (0.75,7.5){};
\node[circle, scale=0.75, fill] (tid9) at (0.75,9){};
\draw[](tid8) -- (tid9);
\draw[](tid5) -- (tid8);
\draw[](tid2) -- (tid5);
\node[circle, scale=0.75, fill] (tid3) at (3,4.5){};
\node[circle, scale=0.75, fill, task_scheduled] (tid6) at (2.25,6){};
\node[circle, scale=0.75, fill] (tid7) at (3.75,6){};
\draw[](tid3) -- (tid6);
\draw[](tid3) -- (tid7);
\node[circle, scale=0.75, fill, task_scheduled] (tid4) at (5.25,4.5){};
\draw[](tid1) -- (tid2);
\draw[](tid1) -- (tid3);
\draw[](tid1) -- (tid4);
\draw[](tid0) -- (tid1);
\end{tikzpicture}
\nodepart{three}
\footnotesize{7.18408}
\nodepart{four}
\footnotesize{$25\:25\:25\:25$}
};
 & 
\node[draw=black, rectangle split,  rectangle split parts=4] (sn0xefc970){
\footnotesize{16.6667}
\nodepart{two}
\begin{tikzpicture}[scale=.2]
\node[circle, scale=0.75, fill] (tid0) at (3,1.5){};
\node[circle, scale=0.75, fill] (tid1) at (3,3){};
\node[circle, scale=0.75, fill] (tid2) at (0.75,4.5){};
\node[circle, scale=0.75, fill] (tid5) at (0.75,6){};
\node[circle, scale=0.75, fill] (tid8) at (0.75,7.5){};
\node[circle, scale=0.75, fill] (tid9) at (0.75,9){};
\draw[](tid8) -- (tid9);
\draw[](tid5) -- (tid8);
\draw[](tid2) -- (tid5);
\node[circle, scale=0.75, fill] (tid3) at (3,4.5){};
\node[circle, scale=0.75, fill, task_scheduled] (tid6) at (2.25,6){};
\node[circle, scale=0.75, fill, task_scheduled] (tid7) at (3.75,6){};
\draw[](tid3) -- (tid6);
\draw[](tid3) -- (tid7);
\node[circle, scale=0.75, fill] (tid4) at (5.25,4.5){};
\draw[](tid1) -- (tid2);
\draw[](tid1) -- (tid3);
\draw[](tid1) -- (tid4);
\draw[](tid0) -- (tid1);
\end{tikzpicture}
\nodepart{three}
\footnotesize{7.21191}
\nodepart{four}
\footnotesize{$50\:50$}
};
 & 
\node[draw=black, rectangle split,  rectangle split parts=4] (sn0xefcc90){
\footnotesize{16.6667}
\nodepart{two}
\begin{tikzpicture}[scale=.2]
\node[circle, scale=0.75, fill] (tid0) at (3,1.5){};
\node[circle, scale=0.75, fill] (tid1) at (3,3){};
\node[circle, scale=0.75, fill] (tid2) at (0.75,4.5){};
\node[circle, scale=0.75, fill] (tid5) at (0.75,6){};
\node[circle, scale=0.75, fill] (tid8) at (0.75,7.5){};
\node[circle, scale=0.75, fill, task_scheduled] (tid9) at (0.75,9){};
\draw[](tid8) -- (tid9);
\draw[](tid5) -- (tid8);
\draw[](tid2) -- (tid5);
\node[circle, scale=0.75, fill] (tid3) at (3,4.5){};
\node[circle, scale=0.75, fill, task_scheduled] (tid6) at (2.25,6){};
\node[circle, scale=0.75, fill] (tid7) at (3.75,6){};
\draw[](tid3) -- (tid6);
\draw[](tid3) -- (tid7);
\node[circle, scale=0.75, fill] (tid4) at (5.25,4.5){};
\draw[](tid1) -- (tid2);
\draw[](tid1) -- (tid3);
\draw[](tid1) -- (tid4);
\draw[](tid0) -- (tid1);
\end{tikzpicture}
\nodepart{three}
\footnotesize{6.96323}
\nodepart{four}
\footnotesize{$25\:25\:17\:17\:17$}
};
 & 
\node[draw=black, rectangle split,  rectangle split parts=4] (sn0xef2a50){
\footnotesize{25}
\nodepart{two}
\begin{tikzpicture}[scale=.2]
\node[circle, scale=0.75, fill] (tid0) at (2.25,1.5){};
\node[circle, scale=0.75, fill] (tid1) at (2.25,3){};
\node[circle, scale=0.75, fill] (tid2) at (0.75,4.5){};
\node[circle, scale=0.75, fill] (tid5) at (0.75,6){};
\node[circle, scale=0.75, fill] (tid8) at (0.75,7.5){};
\node[circle, scale=0.75, fill] (tid9) at (0.75,9){};
\draw[](tid8) -- (tid9);
\draw[](tid5) -- (tid8);
\draw[](tid2) -- (tid5);
\node[circle, scale=0.75, fill] (tid3) at (2.25,4.5){};
\node[circle, scale=0.75, fill, task_scheduled] (tid6) at (2.25,6){};
\draw[](tid3) -- (tid6);
\node[circle, scale=0.75, fill] (tid4) at (3.75,4.5){};
\node[circle, scale=0.75, fill, task_scheduled] (tid7) at (3.75,6){};
\draw[](tid4) -- (tid7);
\draw[](tid1) -- (tid2);
\draw[](tid1) -- (tid3);
\draw[](tid1) -- (tid4);
\draw[](tid0) -- (tid1);
\end{tikzpicture}
\nodepart{three}
\footnotesize{7.21191}
\nodepart{four}
\footnotesize{$50\:50$}
};
 & 
\node[draw=black, rectangle split,  rectangle split parts=4] (sn0xef2540){
\footnotesize{25}
\nodepart{two}
\begin{tikzpicture}[scale=.2]
\node[circle, scale=0.75, fill] (tid0) at (2.25,1.5){};
\node[circle, scale=0.75, fill] (tid1) at (2.25,3){};
\node[circle, scale=0.75, fill] (tid2) at (0.75,4.5){};
\node[circle, scale=0.75, fill] (tid5) at (0.75,6){};
\node[circle, scale=0.75, fill] (tid8) at (0.75,7.5){};
\node[circle, scale=0.75, fill, task_scheduled] (tid9) at (0.75,9){};
\draw[](tid8) -- (tid9);
\draw[](tid5) -- (tid8);
\draw[](tid2) -- (tid5);
\node[circle, scale=0.75, fill] (tid3) at (2.25,4.5){};
\node[circle, scale=0.75, fill, task_scheduled] (tid6) at (2.25,6){};
\draw[](tid3) -- (tid6);
\node[circle, scale=0.75, fill] (tid4) at (3.75,4.5){};
\node[circle, scale=0.75, fill] (tid7) at (3.75,6){};
\draw[](tid4) -- (tid7);
\draw[](tid1) -- (tid2);
\draw[](tid1) -- (tid3);
\draw[](tid1) -- (tid4);
\draw[](tid0) -- (tid1);
\end{tikzpicture}
\nodepart{three}
\footnotesize{6.94629}
\nodepart{four}
\footnotesize{$25\:25\:25\:25$}
};
 & 
\\
};
\end{scope}
\begin{scope}[yshift=\leveltopIII cm]
\matrix (line3)[column sep=0.5cm] {
\node[draw=black, rectangle split,  rectangle split parts=4] (sn0xefd2a0){
\footnotesize{4.16667}
\nodepart{two}
\begin{tikzpicture}[scale=.2]
\node[circle, scale=0.75, fill] (tid0) at (2.25,1.5){};
\node[circle, scale=0.75, fill] (tid1) at (2.25,3){};
\node[circle, scale=0.75, fill] (tid2) at (0.75,4.5){};
\node[circle, scale=0.75, fill] (tid4) at (0.75,6){};
\node[circle, scale=0.75, fill] (tid7) at (0.75,7.5){};
\node[circle, scale=0.75, fill] (tid8) at (0.75,9){};
\draw[](tid7) -- (tid8);
\draw[](tid4) -- (tid7);
\draw[](tid2) -- (tid4);
\node[circle, scale=0.75, fill] (tid3) at (3,4.5){};
\node[circle, scale=0.75, fill, task_scheduled] (tid5) at (2.25,6){};
\node[circle, scale=0.75, fill, task_scheduled] (tid6) at (3.75,6){};
\draw[](tid3) -- (tid5);
\draw[](tid3) -- (tid6);
\draw[](tid1) -- (tid2);
\draw[](tid1) -- (tid3);
\draw[](tid0) -- (tid1);
\end{tikzpicture}
\nodepart{three}
\footnotesize{6.75}
\nodepart{four}
\footnotesize{$1$}
};
 & 
\node[draw=black, rectangle split,  rectangle split parts=4] (sn0xefe2f0){
\footnotesize{4.16667}
\nodepart{two}
\begin{tikzpicture}[scale=.2]
\node[circle, scale=0.75, fill] (tid0) at (2.25,1.5){};
\node[circle, scale=0.75, fill] (tid1) at (2.25,3){};
\node[circle, scale=0.75, fill] (tid2) at (0.75,4.5){};
\node[circle, scale=0.75, fill] (tid4) at (0.75,6){};
\node[circle, scale=0.75, fill] (tid7) at (0.75,7.5){};
\node[circle, scale=0.75, fill, task_scheduled] (tid8) at (0.75,9){};
\draw[](tid7) -- (tid8);
\draw[](tid4) -- (tid7);
\draw[](tid2) -- (tid4);
\node[circle, scale=0.75, fill] (tid3) at (3,4.5){};
\node[circle, scale=0.75, fill, task_scheduled] (tid5) at (2.25,6){};
\node[circle, scale=0.75, fill] (tid6) at (3.75,6){};
\draw[](tid3) -- (tid5);
\draw[](tid3) -- (tid6);
\draw[](tid1) -- (tid2);
\draw[](tid1) -- (tid3);
\draw[](tid0) -- (tid1);
\end{tikzpicture}
\nodepart{three}
\footnotesize{6.57227}
\nodepart{four}
\footnotesize{$50\:25\:25$}
};
 & 
\node[draw=black, rectangle split,  rectangle split parts=4] (sn0xef2f20){
\footnotesize{25}
\nodepart{two}
\begin{tikzpicture}[scale=.2]
\node[circle, scale=0.75, fill] (tid0) at (2.25,1.5){};
\node[circle, scale=0.75, fill] (tid1) at (2.25,3){};
\node[circle, scale=0.75, fill] (tid2) at (0.75,4.5){};
\node[circle, scale=0.75, fill] (tid5) at (0.75,6){};
\node[circle, scale=0.75, fill] (tid7) at (0.75,7.5){};
\node[circle, scale=0.75, fill] (tid8) at (0.75,9){};
\draw[](tid7) -- (tid8);
\draw[](tid5) -- (tid7);
\draw[](tid2) -- (tid5);
\node[circle, scale=0.75, fill] (tid3) at (2.25,4.5){};
\node[circle, scale=0.75, fill, task_scheduled] (tid6) at (2.25,6){};
\draw[](tid3) -- (tid6);
\node[circle, scale=0.75, fill, task_scheduled] (tid4) at (3.75,4.5){};
\draw[](tid1) -- (tid2);
\draw[](tid1) -- (tid3);
\draw[](tid1) -- (tid4);
\draw[](tid0) -- (tid1);
\end{tikzpicture}
\nodepart{three}
\footnotesize{6.83789}
\nodepart{four}
\footnotesize{$50\:25\:25$}
};
 & 
\node[draw=black, rectangle split,  rectangle split parts=4] (sn0xef3e70){
\footnotesize{31.25}
\nodepart{two}
\begin{tikzpicture}[scale=.2]
\node[circle, scale=0.75, fill] (tid0) at (2.25,1.5){};
\node[circle, scale=0.75, fill] (tid1) at (2.25,3){};
\node[circle, scale=0.75, fill] (tid2) at (0.75,4.5){};
\node[circle, scale=0.75, fill] (tid5) at (0.75,6){};
\node[circle, scale=0.75, fill] (tid7) at (0.75,7.5){};
\node[circle, scale=0.75, fill, task_scheduled] (tid8) at (0.75,9){};
\draw[](tid7) -- (tid8);
\draw[](tid5) -- (tid7);
\draw[](tid2) -- (tid5);
\node[circle, scale=0.75, fill] (tid3) at (2.25,4.5){};
\node[circle, scale=0.75, fill, task_scheduled] (tid6) at (2.25,6){};
\draw[](tid3) -- (tid6);
\node[circle, scale=0.75, fill] (tid4) at (3.75,4.5){};
\draw[](tid1) -- (tid2);
\draw[](tid1) -- (tid3);
\draw[](tid1) -- (tid4);
\draw[](tid0) -- (tid1);
\end{tikzpicture}
\nodepart{three}
\footnotesize{6.58594}
\nodepart{four}
\footnotesize{$50\:25\:25$}
};
 & 
\node[draw=black, rectangle split,  rectangle split parts=4] (sn0xefa920){
\footnotesize{14.5833}
\nodepart{two}
\begin{tikzpicture}[scale=.2]
\node[circle, scale=0.75, fill] (tid0) at (2.25,1.5){};
\node[circle, scale=0.75, fill] (tid1) at (2.25,3){};
\node[circle, scale=0.75, fill] (tid2) at (0.75,4.5){};
\node[circle, scale=0.75, fill] (tid5) at (0.75,6){};
\node[circle, scale=0.75, fill] (tid7) at (0.75,7.5){};
\node[circle, scale=0.75, fill, task_scheduled] (tid8) at (0.75,9){};
\draw[](tid7) -- (tid8);
\draw[](tid5) -- (tid7);
\draw[](tid2) -- (tid5);
\node[circle, scale=0.75, fill] (tid3) at (2.25,4.5){};
\node[circle, scale=0.75, fill] (tid6) at (2.25,6){};
\draw[](tid3) -- (tid6);
\node[circle, scale=0.75, fill, task_scheduled] (tid4) at (3.75,4.5){};
\draw[](tid1) -- (tid2);
\draw[](tid1) -- (tid3);
\draw[](tid1) -- (tid4);
\draw[](tid0) -- (tid1);
\end{tikzpicture}
\nodepart{three}
\footnotesize{6.57617}
\nodepart{four}
\footnotesize{$50\:25\:25$}
};
 & 
\node[draw=black, rectangle split,  rectangle split parts=4] (sn0xefabd0){
\footnotesize{6.25}
\nodepart{two}
\begin{tikzpicture}[scale=.2]
\node[circle, scale=0.75, fill] (tid0) at (2.25,1.5){};
\node[circle, scale=0.75, fill] (tid1) at (2.25,3){};
\node[circle, scale=0.75, fill] (tid2) at (0.75,4.5){};
\node[circle, scale=0.75, fill] (tid5) at (0.75,6){};
\node[circle, scale=0.75, fill] (tid8) at (0.75,7.5){};
\draw[](tid5) -- (tid8);
\draw[](tid2) -- (tid5);
\node[circle, scale=0.75, fill] (tid3) at (2.25,4.5){};
\node[circle, scale=0.75, fill, task_scheduled] (tid6) at (2.25,6){};
\draw[](tid3) -- (tid6);
\node[circle, scale=0.75, fill] (tid4) at (3.75,4.5){};
\node[circle, scale=0.75, fill, task_scheduled] (tid7) at (3.75,6){};
\draw[](tid4) -- (tid7);
\draw[](tid1) -- (tid2);
\draw[](tid1) -- (tid3);
\draw[](tid1) -- (tid4);
\draw[](tid0) -- (tid1);
\end{tikzpicture}
\nodepart{three}
\footnotesize{6.38281}
\nodepart{four}
\footnotesize{$50\:50$}
};
 & 
\node[draw=black, rectangle split,  rectangle split parts=4] (sn0xefb500){
\footnotesize{6.25}
\nodepart{two}
\begin{tikzpicture}[scale=.2]
\node[circle, scale=0.75, fill] (tid0) at (2.25,1.5){};
\node[circle, scale=0.75, fill] (tid1) at (2.25,3){};
\node[circle, scale=0.75, fill] (tid2) at (0.75,4.5){};
\node[circle, scale=0.75, fill] (tid5) at (0.75,6){};
\node[circle, scale=0.75, fill, task_scheduled] (tid8) at (0.75,7.5){};
\draw[](tid5) -- (tid8);
\draw[](tid2) -- (tid5);
\node[circle, scale=0.75, fill] (tid3) at (2.25,4.5){};
\node[circle, scale=0.75, fill, task_scheduled] (tid6) at (2.25,6){};
\draw[](tid3) -- (tid6);
\node[circle, scale=0.75, fill] (tid4) at (3.75,4.5){};
\node[circle, scale=0.75, fill] (tid7) at (3.75,6){};
\draw[](tid4) -- (tid7);
\draw[](tid1) -- (tid2);
\draw[](tid1) -- (tid3);
\draw[](tid1) -- (tid4);
\draw[](tid0) -- (tid1);
\end{tikzpicture}
\nodepart{three}
\footnotesize{6.24023}
\nodepart{four}
\footnotesize{$25\:25\:50$}
};
 & 
\node[draw=black, rectangle split,  rectangle split parts=4] (sn0xeffff0){
\footnotesize{2.77778}
\nodepart{two}
\begin{tikzpicture}[scale=.2]
\node[circle, scale=0.75, fill] (tid0) at (3,1.5){};
\node[circle, scale=0.75, fill] (tid1) at (3,3){};
\node[circle, scale=0.75, fill] (tid2) at (1.5,4.5){};
\node[circle, scale=0.75, fill, task_scheduled] (tid5) at (0.75,6){};
\node[circle, scale=0.75, fill] (tid6) at (2.25,6){};
\draw[](tid2) -- (tid5);
\draw[](tid2) -- (tid6);
\node[circle, scale=0.75, fill] (tid3) at (3.75,4.5){};
\node[circle, scale=0.75, fill] (tid7) at (3.75,6){};
\node[circle, scale=0.75, fill] (tid8) at (3.75,7.5){};
\draw[](tid7) -- (tid8);
\draw[](tid3) -- (tid7);
\node[circle, scale=0.75, fill, task_scheduled] (tid4) at (5.25,4.5){};
\draw[](tid1) -- (tid2);
\draw[](tid1) -- (tid3);
\draw[](tid1) -- (tid4);
\draw[](tid0) -- (tid1);
\end{tikzpicture}
\nodepart{three}
\footnotesize{6.39844}
\nodepart{four}
\footnotesize{$25\:25\:25\:25$}
};
 & 
\node[draw=black, rectangle split,  rectangle split parts=4] (sn0xf00d20){
\footnotesize{2.77778}
\nodepart{two}
\begin{tikzpicture}[scale=.2]
\node[circle, scale=0.75, fill] (tid0) at (3,1.5){};
\node[circle, scale=0.75, fill] (tid1) at (3,3){};
\node[circle, scale=0.75, fill] (tid2) at (1.5,4.5){};
\node[circle, scale=0.75, fill, task_scheduled] (tid5) at (0.75,6){};
\node[circle, scale=0.75, fill, task_scheduled] (tid6) at (2.25,6){};
\draw[](tid2) -- (tid5);
\draw[](tid2) -- (tid6);
\node[circle, scale=0.75, fill] (tid3) at (3.75,4.5){};
\node[circle, scale=0.75, fill] (tid7) at (3.75,6){};
\node[circle, scale=0.75, fill] (tid8) at (3.75,7.5){};
\draw[](tid7) -- (tid8);
\draw[](tid3) -- (tid7);
\node[circle, scale=0.75, fill] (tid4) at (5.25,4.5){};
\draw[](tid1) -- (tid2);
\draw[](tid1) -- (tid3);
\draw[](tid1) -- (tid4);
\draw[](tid0) -- (tid1);
\end{tikzpicture}
\nodepart{three}
\footnotesize{6.38281}
\nodepart{four}
\footnotesize{$50\:50$}
};
 & 
\node[draw=black, rectangle split,  rectangle split parts=4] (sn0xf01470){
\footnotesize{2.77778}
\nodepart{two}
\begin{tikzpicture}[scale=.2]
\node[circle, scale=0.75, fill] (tid0) at (3,1.5){};
\node[circle, scale=0.75, fill] (tid1) at (3,3){};
\node[circle, scale=0.75, fill] (tid2) at (1.5,4.5){};
\node[circle, scale=0.75, fill, task_scheduled] (tid5) at (0.75,6){};
\node[circle, scale=0.75, fill] (tid6) at (2.25,6){};
\draw[](tid2) -- (tid5);
\draw[](tid2) -- (tid6);
\node[circle, scale=0.75, fill] (tid3) at (3.75,4.5){};
\node[circle, scale=0.75, fill] (tid7) at (3.75,6){};
\node[circle, scale=0.75, fill, task_scheduled] (tid8) at (3.75,7.5){};
\draw[](tid7) -- (tid8);
\draw[](tid3) -- (tid7);
\node[circle, scale=0.75, fill] (tid4) at (5.25,4.5){};
\draw[](tid1) -- (tid2);
\draw[](tid1) -- (tid3);
\draw[](tid1) -- (tid4);
\draw[](tid0) -- (tid1);
\end{tikzpicture}
\nodepart{three}
\footnotesize{6.25499}
\nodepart{four}
\footnotesize{$25\:25\:17\:17\:17$}
};
 & 
\\
};
\end{scope}
\begin{scope}[yshift=\leveltopIIII cm]
\matrix (line4)[column sep=0.5cm] {
\node[draw=black, rectangle split,  rectangle split parts=4] (sn0xef43a0){
\footnotesize{26.0417}
\nodepart{two}
\begin{tikzpicture}[scale=.2]
\node[circle, scale=0.75, fill] (tid0) at (1.5,1.5){};
\node[circle, scale=0.75, fill] (tid1) at (1.5,3){};
\node[circle, scale=0.75, fill] (tid2) at (0.75,4.5){};
\node[circle, scale=0.75, fill] (tid4) at (0.75,6){};
\node[circle, scale=0.75, fill] (tid6) at (0.75,7.5){};
\node[circle, scale=0.75, fill, task_scheduled] (tid7) at (0.75,9){};
\draw[](tid6) -- (tid7);
\draw[](tid4) -- (tid6);
\draw[](tid2) -- (tid4);
\node[circle, scale=0.75, fill] (tid3) at (2.25,4.5){};
\node[circle, scale=0.75, fill, task_scheduled] (tid5) at (2.25,6){};
\draw[](tid3) -- (tid5);
\draw[](tid1) -- (tid2);
\draw[](tid1) -- (tid3);
\draw[](tid0) -- (tid1);
\end{tikzpicture}
\nodepart{three}
\footnotesize{6.25}
\nodepart{four}
\footnotesize{$50\:50$}
};
 & 
\node[draw=black, rectangle split,  rectangle split parts=4] (sn0xef4560){
\footnotesize{6.25}
\nodepart{two}
\begin{tikzpicture}[scale=.2]
\node[circle, scale=0.75, fill] (tid0) at (2.25,1.5){};
\node[circle, scale=0.75, fill] (tid1) at (2.25,3){};
\node[circle, scale=0.75, fill] (tid2) at (0.75,4.5){};
\node[circle, scale=0.75, fill] (tid5) at (0.75,6){};
\node[circle, scale=0.75, fill] (tid6) at (0.75,7.5){};
\node[circle, scale=0.75, fill] (tid7) at (0.75,9){};
\draw[](tid6) -- (tid7);
\draw[](tid5) -- (tid6);
\draw[](tid2) -- (tid5);
\node[circle, scale=0.75, fill, task_scheduled] (tid3) at (2.25,4.5){};
\node[circle, scale=0.75, fill, task_scheduled] (tid4) at (3.75,4.5){};
\draw[](tid1) -- (tid2);
\draw[](tid1) -- (tid3);
\draw[](tid1) -- (tid4);
\draw[](tid0) -- (tid1);
\end{tikzpicture}
\nodepart{three}
\footnotesize{6.5625}
\nodepart{four}
\footnotesize{$1$}
};
 & 
\node[draw=black, rectangle split,  rectangle split parts=4] (sn0xef4790){
\footnotesize{21.875}
\nodepart{two}
\begin{tikzpicture}[scale=.2]
\node[circle, scale=0.75, fill] (tid0) at (2.25,1.5){};
\node[circle, scale=0.75, fill] (tid1) at (2.25,3){};
\node[circle, scale=0.75, fill] (tid2) at (0.75,4.5){};
\node[circle, scale=0.75, fill] (tid5) at (0.75,6){};
\node[circle, scale=0.75, fill] (tid6) at (0.75,7.5){};
\node[circle, scale=0.75, fill, task_scheduled] (tid7) at (0.75,9){};
\draw[](tid6) -- (tid7);
\draw[](tid5) -- (tid6);
\draw[](tid2) -- (tid5);
\node[circle, scale=0.75, fill, task_scheduled] (tid3) at (2.25,4.5){};
\node[circle, scale=0.75, fill] (tid4) at (3.75,4.5){};
\draw[](tid1) -- (tid2);
\draw[](tid1) -- (tid3);
\draw[](tid1) -- (tid4);
\draw[](tid0) -- (tid1);
\end{tikzpicture}
\nodepart{three}
\footnotesize{6.28906}
\nodepart{four}
\footnotesize{$50\:25\:25$}
};
 & 
\node[draw=black, rectangle split,  rectangle split parts=4] (sn0xef8300){
\footnotesize{16.6667}
\nodepart{two}
\begin{tikzpicture}[scale=.2]
\node[circle, scale=0.75, fill] (tid0) at (2.25,1.5){};
\node[circle, scale=0.75, fill] (tid1) at (2.25,3){};
\node[circle, scale=0.75, fill] (tid2) at (0.75,4.5){};
\node[circle, scale=0.75, fill] (tid5) at (0.75,6){};
\node[circle, scale=0.75, fill] (tid7) at (0.75,7.5){};
\draw[](tid5) -- (tid7);
\draw[](tid2) -- (tid5);
\node[circle, scale=0.75, fill] (tid3) at (2.25,4.5){};
\node[circle, scale=0.75, fill, task_scheduled] (tid6) at (2.25,6){};
\draw[](tid3) -- (tid6);
\node[circle, scale=0.75, fill, task_scheduled] (tid4) at (3.75,4.5){};
\draw[](tid1) -- (tid2);
\draw[](tid1) -- (tid3);
\draw[](tid1) -- (tid4);
\draw[](tid0) -- (tid1);
\end{tikzpicture}
\nodepart{three}
\footnotesize{5.97656}
\nodepart{four}
\footnotesize{$50\:25\:25$}
};
 & 
\node[draw=black, rectangle split,  rectangle split parts=4] (sn0xef90f0){
\footnotesize{14.5833}
\nodepart{two}
\begin{tikzpicture}[scale=.2]
\node[circle, scale=0.75, fill] (tid0) at (2.25,1.5){};
\node[circle, scale=0.75, fill] (tid1) at (2.25,3){};
\node[circle, scale=0.75, fill] (tid2) at (0.75,4.5){};
\node[circle, scale=0.75, fill] (tid5) at (0.75,6){};
\node[circle, scale=0.75, fill, task_scheduled] (tid7) at (0.75,7.5){};
\draw[](tid5) -- (tid7);
\draw[](tid2) -- (tid5);
\node[circle, scale=0.75, fill] (tid3) at (2.25,4.5){};
\node[circle, scale=0.75, fill, task_scheduled] (tid6) at (2.25,6){};
\draw[](tid3) -- (tid6);
\node[circle, scale=0.75, fill] (tid4) at (3.75,4.5){};
\draw[](tid1) -- (tid2);
\draw[](tid1) -- (tid3);
\draw[](tid1) -- (tid4);
\draw[](tid0) -- (tid1);
\end{tikzpicture}
\nodepart{three}
\footnotesize{5.78906}
\nodepart{four}
\footnotesize{$50\:25\:25$}
};
 & 
\node[draw=black, rectangle split,  rectangle split parts=4] (sn0xefbc60){
\footnotesize{6.59722}
\nodepart{two}
\begin{tikzpicture}[scale=.2]
\node[circle, scale=0.75, fill] (tid0) at (2.25,1.5){};
\node[circle, scale=0.75, fill] (tid1) at (2.25,3){};
\node[circle, scale=0.75, fill] (tid2) at (0.75,4.5){};
\node[circle, scale=0.75, fill] (tid5) at (0.75,6){};
\node[circle, scale=0.75, fill, task_scheduled] (tid7) at (0.75,7.5){};
\draw[](tid5) -- (tid7);
\draw[](tid2) -- (tid5);
\node[circle, scale=0.75, fill] (tid3) at (2.25,4.5){};
\node[circle, scale=0.75, fill] (tid6) at (2.25,6){};
\draw[](tid3) -- (tid6);
\node[circle, scale=0.75, fill, task_scheduled] (tid4) at (3.75,4.5){};
\draw[](tid1) -- (tid2);
\draw[](tid1) -- (tid3);
\draw[](tid1) -- (tid4);
\draw[](tid0) -- (tid1);
\end{tikzpicture}
\nodepart{three}
\footnotesize{5.82812}
\nodepart{four}
\footnotesize{$50\:50$}
};
 & 
\node[draw=black, rectangle split,  rectangle split parts=4] (sn0xefe720){
\footnotesize{1.73611}
\nodepart{two}
\begin{tikzpicture}[scale=.2]
\node[circle, scale=0.75, fill] (tid0) at (2.25,1.5){};
\node[circle, scale=0.75, fill] (tid1) at (2.25,3){};
\node[circle, scale=0.75, fill] (tid2) at (1.5,4.5){};
\node[circle, scale=0.75, fill, task_scheduled] (tid4) at (0.75,6){};
\node[circle, scale=0.75, fill, task_scheduled] (tid5) at (2.25,6){};
\draw[](tid2) -- (tid4);
\draw[](tid2) -- (tid5);
\node[circle, scale=0.75, fill] (tid3) at (3.75,4.5){};
\node[circle, scale=0.75, fill] (tid6) at (3.75,6){};
\node[circle, scale=0.75, fill] (tid7) at (3.75,7.5){};
\draw[](tid6) -- (tid7);
\draw[](tid3) -- (tid6);
\draw[](tid1) -- (tid2);
\draw[](tid1) -- (tid3);
\draw[](tid0) -- (tid1);
\end{tikzpicture}
\nodepart{three}
\footnotesize{5.9375}
\nodepart{four}
\footnotesize{$1$}
};
 & 
\node[draw=black, rectangle split,  rectangle split parts=4] (sn0xefea20){
\footnotesize{1.73611}
\nodepart{two}
\begin{tikzpicture}[scale=.2]
\node[circle, scale=0.75, fill] (tid0) at (2.25,1.5){};
\node[circle, scale=0.75, fill] (tid1) at (2.25,3){};
\node[circle, scale=0.75, fill] (tid2) at (1.5,4.5){};
\node[circle, scale=0.75, fill, task_scheduled] (tid4) at (0.75,6){};
\node[circle, scale=0.75, fill] (tid5) at (2.25,6){};
\draw[](tid2) -- (tid4);
\draw[](tid2) -- (tid5);
\node[circle, scale=0.75, fill] (tid3) at (3.75,4.5){};
\node[circle, scale=0.75, fill] (tid6) at (3.75,6){};
\node[circle, scale=0.75, fill, task_scheduled] (tid7) at (3.75,7.5){};
\draw[](tid6) -- (tid7);
\draw[](tid3) -- (tid6);
\draw[](tid1) -- (tid2);
\draw[](tid1) -- (tid3);
\draw[](tid0) -- (tid1);
\end{tikzpicture}
\nodepart{three}
\footnotesize{5.85156}
\nodepart{four}
\footnotesize{$50\:25\:25$}
};
 & 
\node[draw=black, rectangle split,  rectangle split parts=4] (sn0xf01940){
\footnotesize{0.462963}
\nodepart{two}
\begin{tikzpicture}[scale=.2]
\node[circle, scale=0.75, fill] (tid0) at (3,1.5){};
\node[circle, scale=0.75, fill] (tid1) at (3,3){};
\node[circle, scale=0.75, fill] (tid2) at (1.5,4.5){};
\node[circle, scale=0.75, fill, task_scheduled] (tid5) at (0.75,6){};
\node[circle, scale=0.75, fill] (tid6) at (2.25,6){};
\draw[](tid2) -- (tid5);
\draw[](tid2) -- (tid6);
\node[circle, scale=0.75, fill] (tid3) at (3.75,4.5){};
\node[circle, scale=0.75, fill] (tid7) at (3.75,6){};
\draw[](tid3) -- (tid7);
\node[circle, scale=0.75, fill, task_scheduled] (tid4) at (5.25,4.5){};
\draw[](tid1) -- (tid2);
\draw[](tid1) -- (tid3);
\draw[](tid1) -- (tid4);
\draw[](tid0) -- (tid1);
\end{tikzpicture}
\nodepart{three}
\footnotesize{5.74219}
\nodepart{four}
\footnotesize{$25\:25\:50$}
};
 & 
\node[draw=black, rectangle split,  rectangle split parts=4] (sn0xf01e70){
\footnotesize{0.462963}
\nodepart{two}
\begin{tikzpicture}[scale=.2]
\node[circle, scale=0.75, fill] (tid0) at (3,1.5){};
\node[circle, scale=0.75, fill] (tid1) at (3,3){};
\node[circle, scale=0.75, fill] (tid2) at (1.5,4.5){};
\node[circle, scale=0.75, fill, task_scheduled] (tid5) at (0.75,6){};
\node[circle, scale=0.75, fill, task_scheduled] (tid6) at (2.25,6){};
\draw[](tid2) -- (tid5);
\draw[](tid2) -- (tid6);
\node[circle, scale=0.75, fill] (tid3) at (3.75,4.5){};
\node[circle, scale=0.75, fill] (tid7) at (3.75,6){};
\draw[](tid3) -- (tid7);
\node[circle, scale=0.75, fill] (tid4) at (5.25,4.5){};
\draw[](tid1) -- (tid2);
\draw[](tid1) -- (tid3);
\draw[](tid1) -- (tid4);
\draw[](tid0) -- (tid1);
\end{tikzpicture}
\nodepart{three}
\footnotesize{5.67188}
\nodepart{four}
\footnotesize{$50\:50$}
};
 & 
\node[draw=black, rectangle split,  rectangle split parts=4] (sn0xf02290){
\footnotesize{0.462963}
\nodepart{two}
\begin{tikzpicture}[scale=.2]
\node[circle, scale=0.75, fill] (tid0) at (3,1.5){};
\node[circle, scale=0.75, fill] (tid1) at (3,3){};
\node[circle, scale=0.75, fill] (tid2) at (1.5,4.5){};
\node[circle, scale=0.75, fill, task_scheduled] (tid5) at (0.75,6){};
\node[circle, scale=0.75, fill] (tid6) at (2.25,6){};
\draw[](tid2) -- (tid5);
\draw[](tid2) -- (tid6);
\node[circle, scale=0.75, fill] (tid3) at (3.75,4.5){};
\node[circle, scale=0.75, fill, task_scheduled] (tid7) at (3.75,6){};
\draw[](tid3) -- (tid7);
\node[circle, scale=0.75, fill] (tid4) at (5.25,4.5){};
\draw[](tid1) -- (tid2);
\draw[](tid1) -- (tid3);
\draw[](tid1) -- (tid4);
\draw[](tid0) -- (tid1);
\end{tikzpicture}
\nodepart{three}
\footnotesize{5.6901}
\nodepart{four}
\footnotesize{$33\:17\:25\:25$}
};
 & 
\node[draw=black, rectangle split,  rectangle split parts=4] (sn0xefc0b0){
\footnotesize{3.125}
\nodepart{two}
\begin{tikzpicture}[scale=.2]
\node[circle, scale=0.75, fill] (tid0) at (2.25,1.5){};
\node[circle, scale=0.75, fill] (tid1) at (2.25,3){};
\node[circle, scale=0.75, fill] (tid2) at (0.75,4.5){};
\node[circle, scale=0.75, fill, task_scheduled] (tid5) at (0.75,6){};
\draw[](tid2) -- (tid5);
\node[circle, scale=0.75, fill] (tid3) at (2.25,4.5){};
\node[circle, scale=0.75, fill, task_scheduled] (tid6) at (2.25,6){};
\draw[](tid3) -- (tid6);
\node[circle, scale=0.75, fill] (tid4) at (3.75,4.5){};
\node[circle, scale=0.75, fill] (tid7) at (3.75,6){};
\draw[](tid4) -- (tid7);
\draw[](tid1) -- (tid2);
\draw[](tid1) -- (tid3);
\draw[](tid1) -- (tid4);
\draw[](tid0) -- (tid1);
\end{tikzpicture}
\nodepart{three}
\footnotesize{5.67188}
\nodepart{four}
\footnotesize{$50\:50$}
};
 & 
\\
};
\end{scope}
\begin{scope}[yshift=\leveltopIIIII cm]
\matrix (line5)[column sep=0.5cm] {
\node[draw=black, rectangle split,  rectangle split parts=4] (sn0xef48f0){
\footnotesize{30.2083}
\nodepart{two}
\begin{tikzpicture}[scale=.2]
\node[circle, scale=0.75, fill] (tid0) at (1.5,1.5){};
\node[circle, scale=0.75, fill] (tid1) at (1.5,3){};
\node[circle, scale=0.75, fill] (tid2) at (0.75,4.5){};
\node[circle, scale=0.75, fill] (tid4) at (0.75,6){};
\node[circle, scale=0.75, fill] (tid5) at (0.75,7.5){};
\node[circle, scale=0.75, fill, task_scheduled] (tid6) at (0.75,9){};
\draw[](tid5) -- (tid6);
\draw[](tid4) -- (tid5);
\draw[](tid2) -- (tid4);
\node[circle, scale=0.75, fill, task_scheduled] (tid3) at (2.25,4.5){};
\draw[](tid1) -- (tid2);
\draw[](tid1) -- (tid3);
\draw[](tid0) -- (tid1);
\end{tikzpicture}
\nodepart{three}
\footnotesize{6.0625}
\nodepart{four}
\footnotesize{$50\:50$}
};
 & 
\node[draw=black, rectangle split,  rectangle split parts=4] (sn0xef5110){
\footnotesize{27.2569}
\nodepart{two}
\begin{tikzpicture}[scale=.2]
\node[circle, scale=0.75, fill] (tid0) at (1.5,1.5){};
\node[circle, scale=0.75, fill] (tid1) at (1.5,3){};
\node[circle, scale=0.75, fill] (tid2) at (0.75,4.5){};
\node[circle, scale=0.75, fill] (tid4) at (0.75,6){};
\node[circle, scale=0.75, fill, task_scheduled] (tid6) at (0.75,7.5){};
\draw[](tid4) -- (tid6);
\draw[](tid2) -- (tid4);
\node[circle, scale=0.75, fill] (tid3) at (2.25,4.5){};
\node[circle, scale=0.75, fill, task_scheduled] (tid5) at (2.25,6){};
\draw[](tid3) -- (tid5);
\draw[](tid1) -- (tid2);
\draw[](tid1) -- (tid3);
\draw[](tid0) -- (tid1);
\end{tikzpicture}
\nodepart{three}
\footnotesize{5.4375}
\nodepart{four}
\footnotesize{$50\:50$}
};
 & 
\node[draw=black, rectangle split,  rectangle split parts=4] (sn0xef7880){
\footnotesize{9.63542}
\nodepart{two}
\begin{tikzpicture}[scale=.2]
\node[circle, scale=0.75, fill] (tid0) at (2.25,1.5){};
\node[circle, scale=0.75, fill] (tid1) at (2.25,3){};
\node[circle, scale=0.75, fill] (tid2) at (0.75,4.5){};
\node[circle, scale=0.75, fill] (tid5) at (0.75,6){};
\node[circle, scale=0.75, fill] (tid6) at (0.75,7.5){};
\draw[](tid5) -- (tid6);
\draw[](tid2) -- (tid5);
\node[circle, scale=0.75, fill, task_scheduled] (tid3) at (2.25,4.5){};
\node[circle, scale=0.75, fill, task_scheduled] (tid4) at (3.75,4.5){};
\draw[](tid1) -- (tid2);
\draw[](tid1) -- (tid3);
\draw[](tid1) -- (tid4);
\draw[](tid0) -- (tid1);
\end{tikzpicture}
\nodepart{three}
\footnotesize{5.625}
\nodepart{four}
\footnotesize{$1$}
};
 & 
\node[draw=black, rectangle split,  rectangle split parts=4] (sn0xef7c10){
\footnotesize{16.9271}
\nodepart{two}
\begin{tikzpicture}[scale=.2]
\node[circle, scale=0.75, fill] (tid0) at (2.25,1.5){};
\node[circle, scale=0.75, fill] (tid1) at (2.25,3){};
\node[circle, scale=0.75, fill] (tid2) at (0.75,4.5){};
\node[circle, scale=0.75, fill] (tid5) at (0.75,6){};
\node[circle, scale=0.75, fill, task_scheduled] (tid6) at (0.75,7.5){};
\draw[](tid5) -- (tid6);
\draw[](tid2) -- (tid5);
\node[circle, scale=0.75, fill, task_scheduled] (tid3) at (2.25,4.5){};
\node[circle, scale=0.75, fill] (tid4) at (3.75,4.5){};
\draw[](tid1) -- (tid2);
\draw[](tid1) -- (tid3);
\draw[](tid1) -- (tid4);
\draw[](tid0) -- (tid1);
\end{tikzpicture}
\nodepart{three}
\footnotesize{5.40625}
\nodepart{four}
\footnotesize{$50\:25\:25$}
};
 & 
\node[draw=black, rectangle split,  rectangle split parts=4] (sn0xefeea0){
\footnotesize{0.549769}
\nodepart{two}
\begin{tikzpicture}[scale=.2]
\node[circle, scale=0.75, fill] (tid0) at (2.25,1.5){};
\node[circle, scale=0.75, fill] (tid1) at (2.25,3){};
\node[circle, scale=0.75, fill] (tid2) at (1.5,4.5){};
\node[circle, scale=0.75, fill, task_scheduled] (tid4) at (0.75,6){};
\node[circle, scale=0.75, fill, task_scheduled] (tid5) at (2.25,6){};
\draw[](tid2) -- (tid4);
\draw[](tid2) -- (tid5);
\node[circle, scale=0.75, fill] (tid3) at (3.75,4.5){};
\node[circle, scale=0.75, fill] (tid6) at (3.75,6){};
\draw[](tid3) -- (tid6);
\draw[](tid1) -- (tid2);
\draw[](tid1) -- (tid3);
\draw[](tid0) -- (tid1);
\end{tikzpicture}
\nodepart{three}
\footnotesize{5.25}
\nodepart{four}
\footnotesize{$1$}
};
 & 
\node[draw=black, rectangle split,  rectangle split parts=4] (sn0xeff210){
\footnotesize{0.549769}
\nodepart{two}
\begin{tikzpicture}[scale=.2]
\node[circle, scale=0.75, fill] (tid0) at (2.25,1.5){};
\node[circle, scale=0.75, fill] (tid1) at (2.25,3){};
\node[circle, scale=0.75, fill] (tid2) at (1.5,4.5){};
\node[circle, scale=0.75, fill, task_scheduled] (tid4) at (0.75,6){};
\node[circle, scale=0.75, fill] (tid5) at (2.25,6){};
\draw[](tid2) -- (tid4);
\draw[](tid2) -- (tid5);
\node[circle, scale=0.75, fill] (tid3) at (3.75,4.5){};
\node[circle, scale=0.75, fill, task_scheduled] (tid6) at (3.75,6){};
\draw[](tid3) -- (tid6);
\draw[](tid1) -- (tid2);
\draw[](tid1) -- (tid3);
\draw[](tid0) -- (tid1);
\end{tikzpicture}
\nodepart{three}
\footnotesize{5.28125}
\nodepart{four}
\footnotesize{$25\:25\:50$}
};
 & 
\node[draw=black, rectangle split,  rectangle split parts=4] (sn0xf026b0){
\footnotesize{0.154321}
\nodepart{two}
\begin{tikzpicture}[scale=.2]
\node[circle, scale=0.75, fill] (tid0) at (3,1.5){};
\node[circle, scale=0.75, fill] (tid1) at (3,3){};
\node[circle, scale=0.75, fill] (tid2) at (1.5,4.5){};
\node[circle, scale=0.75, fill, task_scheduled] (tid5) at (0.75,6){};
\node[circle, scale=0.75, fill] (tid6) at (2.25,6){};
\draw[](tid2) -- (tid5);
\draw[](tid2) -- (tid6);
\node[circle, scale=0.75, fill, task_scheduled] (tid3) at (3.75,4.5){};
\node[circle, scale=0.75, fill] (tid4) at (5.25,4.5){};
\draw[](tid1) -- (tid2);
\draw[](tid1) -- (tid3);
\draw[](tid1) -- (tid4);
\draw[](tid0) -- (tid1);
\end{tikzpicture}
\nodepart{three}
\footnotesize{5.25}
\nodepart{four}
\footnotesize{$25\:25\:25\:25$}
};
 & 
\node[draw=black, rectangle split,  rectangle split parts=4] (sn0xf02bb0){
\footnotesize{0.0771605}
\nodepart{two}
\begin{tikzpicture}[scale=.2]
\node[circle, scale=0.75, fill] (tid0) at (3,1.5){};
\node[circle, scale=0.75, fill] (tid1) at (3,3){};
\node[circle, scale=0.75, fill] (tid2) at (1.5,4.5){};
\node[circle, scale=0.75, fill, task_scheduled] (tid5) at (0.75,6){};
\node[circle, scale=0.75, fill, task_scheduled] (tid6) at (2.25,6){};
\draw[](tid2) -- (tid5);
\draw[](tid2) -- (tid6);
\node[circle, scale=0.75, fill] (tid3) at (3.75,4.5){};
\node[circle, scale=0.75, fill] (tid4) at (5.25,4.5){};
\draw[](tid1) -- (tid2);
\draw[](tid1) -- (tid3);
\draw[](tid1) -- (tid4);
\draw[](tid0) -- (tid1);
\end{tikzpicture}
\nodepart{three}
\footnotesize{5.125}
\nodepart{four}
\footnotesize{$1$}
};
 & 
\node[draw=black, rectangle split,  rectangle split parts=4] (sn0xef9e10){
\footnotesize{9.08565}
\nodepart{two}
\begin{tikzpicture}[scale=.2]
\node[circle, scale=0.75, fill] (tid0) at (2.25,1.5){};
\node[circle, scale=0.75, fill] (tid1) at (2.25,3){};
\node[circle, scale=0.75, fill] (tid2) at (0.75,4.5){};
\node[circle, scale=0.75, fill, task_scheduled] (tid5) at (0.75,6){};
\draw[](tid2) -- (tid5);
\node[circle, scale=0.75, fill] (tid3) at (2.25,4.5){};
\node[circle, scale=0.75, fill] (tid6) at (2.25,6){};
\draw[](tid3) -- (tid6);
\node[circle, scale=0.75, fill, task_scheduled] (tid4) at (3.75,4.5){};
\draw[](tid1) -- (tid2);
\draw[](tid1) -- (tid3);
\draw[](tid1) -- (tid4);
\draw[](tid0) -- (tid1);
\end{tikzpicture}
\nodepart{three}
\footnotesize{5.21875}
\nodepart{four}
\footnotesize{$50\:25\:25$}
};
 & 
\node[draw=black, rectangle split,  rectangle split parts=4] (sn0xefa190){
\footnotesize{5.55556}
\nodepart{two}
\begin{tikzpicture}[scale=.2]
\node[circle, scale=0.75, fill] (tid0) at (2.25,1.5){};
\node[circle, scale=0.75, fill] (tid1) at (2.25,3){};
\node[circle, scale=0.75, fill] (tid2) at (0.75,4.5){};
\node[circle, scale=0.75, fill, task_scheduled] (tid5) at (0.75,6){};
\draw[](tid2) -- (tid5);
\node[circle, scale=0.75, fill] (tid3) at (2.25,4.5){};
\node[circle, scale=0.75, fill, task_scheduled] (tid6) at (2.25,6){};
\draw[](tid3) -- (tid6);
\node[circle, scale=0.75, fill] (tid4) at (3.75,4.5){};
\draw[](tid1) -- (tid2);
\draw[](tid1) -- (tid3);
\draw[](tid1) -- (tid4);
\draw[](tid0) -- (tid1);
\end{tikzpicture}
\nodepart{three}
\footnotesize{5.125}
\nodepart{four}
\footnotesize{$1$}
};
 & 
\\
};
\end{scope}
\begin{scope}[yshift=\leveltopIIIIII cm]
\matrix (line6)[column sep=0.5cm] {
\node[draw=black, rectangle split,  rectangle split parts=4] (sn0xef5310){
\footnotesize{15.1042}
\nodepart{two}
\begin{tikzpicture}[scale=.2]
\node[circle, scale=0.75, fill] (tid0) at (0.75,1.5){};
\node[circle, scale=0.75, fill] (tid1) at (0.75,3){};
\node[circle, scale=0.75, fill] (tid2) at (0.75,4.5){};
\node[circle, scale=0.75, fill] (tid3) at (0.75,6){};
\node[circle, scale=0.75, fill] (tid4) at (0.75,7.5){};
\node[circle, scale=0.75, fill, task_scheduled] (tid5) at (0.75,9){};
\draw[](tid4) -- (tid5);
\draw[](tid3) -- (tid4);
\draw[](tid2) -- (tid3);
\draw[](tid1) -- (tid2);
\draw[](tid0) -- (tid1);
\end{tikzpicture}
\nodepart{three}
\footnotesize{6}
\nodepart{four}
\footnotesize{$1$}
};
 & 
\node[draw=black, rectangle split,  rectangle split parts=4] (sn0xef5750){
\footnotesize{46.8316}
\nodepart{two}
\begin{tikzpicture}[scale=.2]
\node[circle, scale=0.75, fill] (tid0) at (1.5,1.5){};
\node[circle, scale=0.75, fill] (tid1) at (1.5,3){};
\node[circle, scale=0.75, fill] (tid2) at (0.75,4.5){};
\node[circle, scale=0.75, fill] (tid4) at (0.75,6){};
\node[circle, scale=0.75, fill, task_scheduled] (tid5) at (0.75,7.5){};
\draw[](tid4) -- (tid5);
\draw[](tid2) -- (tid4);
\node[circle, scale=0.75, fill, task_scheduled] (tid3) at (2.25,4.5){};
\draw[](tid1) -- (tid2);
\draw[](tid1) -- (tid3);
\draw[](tid0) -- (tid1);
\end{tikzpicture}
\nodepart{three}
\footnotesize{5.125}
\nodepart{four}
\footnotesize{$50\:50$}
};
 & 
\node[draw=black, rectangle split,  rectangle split parts=4] (sn0xeff630){
\footnotesize{0.176022}
\nodepart{two}
\begin{tikzpicture}[scale=.2]
\node[circle, scale=0.75, fill] (tid0) at (2.25,1.5){};
\node[circle, scale=0.75, fill] (tid1) at (2.25,3){};
\node[circle, scale=0.75, fill] (tid2) at (1.5,4.5){};
\node[circle, scale=0.75, fill, task_scheduled] (tid4) at (0.75,6){};
\node[circle, scale=0.75, fill] (tid5) at (2.25,6){};
\draw[](tid2) -- (tid4);
\draw[](tid2) -- (tid5);
\node[circle, scale=0.75, fill, task_scheduled] (tid3) at (3.75,4.5){};
\draw[](tid1) -- (tid2);
\draw[](tid1) -- (tid3);
\draw[](tid0) -- (tid1);
\end{tikzpicture}
\nodepart{three}
\footnotesize{4.875}
\nodepart{four}
\footnotesize{$50\:50$}
};
 & 
\node[draw=black, rectangle split,  rectangle split parts=4] (sn0xeff8f0){
\footnotesize{0.176022}
\nodepart{two}
\begin{tikzpicture}[scale=.2]
\node[circle, scale=0.75, fill] (tid0) at (2.25,1.5){};
\node[circle, scale=0.75, fill] (tid1) at (2.25,3){};
\node[circle, scale=0.75, fill] (tid2) at (1.5,4.5){};
\node[circle, scale=0.75, fill, task_scheduled] (tid4) at (0.75,6){};
\node[circle, scale=0.75, fill, task_scheduled] (tid5) at (2.25,6){};
\draw[](tid2) -- (tid4);
\draw[](tid2) -- (tid5);
\node[circle, scale=0.75, fill] (tid3) at (3.75,4.5){};
\draw[](tid1) -- (tid2);
\draw[](tid1) -- (tid3);
\draw[](tid0) -- (tid1);
\end{tikzpicture}
\nodepart{three}
\footnotesize{4.75}
\nodepart{four}
\footnotesize{$1$}
};
 & 
\node[draw=black, rectangle split,  rectangle split parts=4] (sn0xef7040){
\footnotesize{18.9959}
\nodepart{two}
\begin{tikzpicture}[scale=.2]
\node[circle, scale=0.75, fill] (tid0) at (1.5,1.5){};
\node[circle, scale=0.75, fill] (tid1) at (1.5,3){};
\node[circle, scale=0.75, fill] (tid2) at (0.75,4.5){};
\node[circle, scale=0.75, fill, task_scheduled] (tid4) at (0.75,6){};
\draw[](tid2) -- (tid4);
\node[circle, scale=0.75, fill] (tid3) at (2.25,4.5){};
\node[circle, scale=0.75, fill, task_scheduled] (tid5) at (2.25,6){};
\draw[](tid3) -- (tid5);
\draw[](tid1) -- (tid2);
\draw[](tid1) -- (tid3);
\draw[](tid0) -- (tid1);
\end{tikzpicture}
\nodepart{three}
\footnotesize{4.75}
\nodepart{four}
\footnotesize{$1$}
};
 & 
\node[draw=black, rectangle split,  rectangle split parts=4] (sn0xef7d10){
\footnotesize{6.54176}
\nodepart{two}
\begin{tikzpicture}[scale=.2]
\node[circle, scale=0.75, fill] (tid0) at (2.25,1.5){};
\node[circle, scale=0.75, fill] (tid1) at (2.25,3){};
\node[circle, scale=0.75, fill] (tid2) at (0.75,4.5){};
\node[circle, scale=0.75, fill] (tid5) at (0.75,6){};
\draw[](tid2) -- (tid5);
\node[circle, scale=0.75, fill, task_scheduled] (tid3) at (2.25,4.5){};
\node[circle, scale=0.75, fill, task_scheduled] (tid4) at (3.75,4.5){};
\draw[](tid1) -- (tid2);
\draw[](tid1) -- (tid3);
\draw[](tid1) -- (tid4);
\draw[](tid0) -- (tid1);
\end{tikzpicture}
\nodepart{three}
\footnotesize{4.75}
\nodepart{four}
\footnotesize{$1$}
};
 & 
\node[draw=black, rectangle split,  rectangle split parts=4] (sn0xef8590){
\footnotesize{12.1745}
\nodepart{two}
\begin{tikzpicture}[scale=.2]
\node[circle, scale=0.75, fill] (tid0) at (2.25,1.5){};
\node[circle, scale=0.75, fill] (tid1) at (2.25,3){};
\node[circle, scale=0.75, fill] (tid2) at (0.75,4.5){};
\node[circle, scale=0.75, fill, task_scheduled] (tid5) at (0.75,6){};
\draw[](tid2) -- (tid5);
\node[circle, scale=0.75, fill, task_scheduled] (tid3) at (2.25,4.5){};
\node[circle, scale=0.75, fill] (tid4) at (3.75,4.5){};
\draw[](tid1) -- (tid2);
\draw[](tid1) -- (tid3);
\draw[](tid1) -- (tid4);
\draw[](tid0) -- (tid1);
\end{tikzpicture}
\nodepart{three}
\footnotesize{4.625}
\nodepart{four}
\footnotesize{$50\:50$}
};
 & 
\\
};
\end{scope}
\begin{scope}[yshift=\leveltopIIIIIII cm]
\matrix (line7)[column sep=0.5cm] {
\node[draw=black, rectangle split,  rectangle split parts=4] (sn0xef5850){
\footnotesize{38.52}
\nodepart{two}
\begin{tikzpicture}[scale=.2]
\node[circle, scale=0.75, fill] (tid0) at (0.75,1.5){};
\node[circle, scale=0.75, fill] (tid1) at (0.75,3){};
\node[circle, scale=0.75, fill] (tid2) at (0.75,4.5){};
\node[circle, scale=0.75, fill] (tid3) at (0.75,6){};
\node[circle, scale=0.75, fill, task_scheduled] (tid4) at (0.75,7.5){};
\draw[](tid3) -- (tid4);
\draw[](tid2) -- (tid3);
\draw[](tid1) -- (tid2);
\draw[](tid0) -- (tid1);
\end{tikzpicture}
\nodepart{three}
\footnotesize{5}
\nodepart{four}
\footnotesize{$1$}
};
 & 
\node[draw=black, rectangle split,  rectangle split parts=4] (sn0xeffc80){
\footnotesize{0.0880112}
\nodepart{two}
\begin{tikzpicture}[scale=.2]
\node[circle, scale=0.75, fill] (tid0) at (1.5,1.5){};
\node[circle, scale=0.75, fill] (tid1) at (1.5,3){};
\node[circle, scale=0.75, fill] (tid2) at (1.5,4.5){};
\node[circle, scale=0.75, fill, task_scheduled] (tid3) at (0.75,6){};
\node[circle, scale=0.75, fill, task_scheduled] (tid4) at (2.25,6){};
\draw[](tid2) -- (tid3);
\draw[](tid2) -- (tid4);
\draw[](tid1) -- (tid2);
\draw[](tid0) -- (tid1);
\end{tikzpicture}
\nodepart{three}
\footnotesize{4.5}
\nodepart{four}
\footnotesize{$1$}
};
 & 
\node[draw=black, rectangle split,  rectangle split parts=4] (sn0xef64d0){
\footnotesize{55.3048}
\nodepart{two}
\begin{tikzpicture}[scale=.2]
\node[circle, scale=0.75, fill] (tid0) at (1.5,1.5){};
\node[circle, scale=0.75, fill] (tid1) at (1.5,3){};
\node[circle, scale=0.75, fill] (tid2) at (0.75,4.5){};
\node[circle, scale=0.75, fill, task_scheduled] (tid4) at (0.75,6){};
\draw[](tid2) -- (tid4);
\node[circle, scale=0.75, fill, task_scheduled] (tid3) at (2.25,4.5){};
\draw[](tid1) -- (tid2);
\draw[](tid1) -- (tid3);
\draw[](tid0) -- (tid1);
\end{tikzpicture}
\nodepart{three}
\footnotesize{4.25}
\nodepart{four}
\footnotesize{$50\:50$}
};
 & 
\node[draw=black, rectangle split,  rectangle split parts=4] (sn0xef8750){
\footnotesize{6.08724}
\nodepart{two}
\begin{tikzpicture}[scale=.2]
\node[circle, scale=0.75, fill] (tid0) at (2.25,1.5){};
\node[circle, scale=0.75, fill] (tid1) at (2.25,3){};
\node[circle, scale=0.75, fill, task_scheduled] (tid2) at (0.75,4.5){};
\node[circle, scale=0.75, fill, task_scheduled] (tid3) at (2.25,4.5){};
\node[circle, scale=0.75, fill] (tid4) at (3.75,4.5){};
\draw[](tid1) -- (tid2);
\draw[](tid1) -- (tid3);
\draw[](tid1) -- (tid4);
\draw[](tid0) -- (tid1);
\end{tikzpicture}
\nodepart{three}
\footnotesize{4}
\nodepart{four}
\footnotesize{$1$}
};
 & 
\\
};
\end{scope}
\begin{scope}[yshift=\leveltopIIIIIIII cm]
\matrix (line8)[column sep=0.5cm] {
\node[draw=black, rectangle split,  rectangle split parts=4] (sn0xef5c90){
\footnotesize{66.2604}
\nodepart{two}
\begin{tikzpicture}[scale=.2]
\node[circle, scale=0.75, fill] (tid0) at (0.75,1.5){};
\node[circle, scale=0.75, fill] (tid1) at (0.75,3){};
\node[circle, scale=0.75, fill] (tid2) at (0.75,4.5){};
\node[circle, scale=0.75, fill, task_scheduled] (tid3) at (0.75,6){};
\draw[](tid2) -- (tid3);
\draw[](tid1) -- (tid2);
\draw[](tid0) -- (tid1);
\end{tikzpicture}
\nodepart{three}
\footnotesize{4}
\nodepart{four}
\footnotesize{$1$}
};
 & 
\node[draw=black, rectangle split,  rectangle split parts=4] (sn0xef6940){
\footnotesize{33.7396}
\nodepart{two}
\begin{tikzpicture}[scale=.2]
\node[circle, scale=0.75, fill] (tid0) at (1.5,1.5){};
\node[circle, scale=0.75, fill] (tid1) at (1.5,3){};
\node[circle, scale=0.75, fill, task_scheduled] (tid2) at (0.75,4.5){};
\node[circle, scale=0.75, fill, task_scheduled] (tid3) at (2.25,4.5){};
\draw[](tid1) -- (tid2);
\draw[](tid1) -- (tid3);
\draw[](tid0) -- (tid1);
\end{tikzpicture}
\nodepart{three}
\footnotesize{3.5}
\nodepart{four}
\footnotesize{$1$}
};
 & 
\\
};
\end{scope}
\begin{scope}[yshift=\leveltopIIIIIIIII cm]
\matrix (line9)[column sep=0.5cm] {
\node[draw=black, rectangle split,  rectangle split parts=4] (sn0xef5950){
\footnotesize{100}
\nodepart{two}
\begin{tikzpicture}[scale=.2]
\node[circle, scale=0.75, fill] (tid0) at (0.75,1.5){};
\node[circle, scale=0.75, fill] (tid1) at (0.75,3){};
\node[circle, scale=0.75, fill, task_scheduled] (tid2) at (0.75,4.5){};
\draw[](tid1) -- (tid2);
\draw[](tid0) -- (tid1);
\end{tikzpicture}
\nodepart{three}
\footnotesize{3}
\nodepart{four}
\footnotesize{$1$}
};
 & 
\\
};
\end{scope}
\begin{scope}[yshift=\leveltopIIIIIIIIII cm]
\matrix (line10)[column sep=0.5cm] {
\node[draw=black, rectangle split,  rectangle split parts=4] (sn0xef5b30){
\footnotesize{100}
\nodepart{two}
\begin{tikzpicture}[scale=.2]
\node[circle, scale=0.75, fill] (tid0) at (0.75,1.5){};
\node[circle, scale=0.75, fill, task_scheduled] (tid1) at (0.75,3){};
\draw[](tid0) -- (tid1);
\end{tikzpicture}
\nodepart{three}
\footnotesize{2}
\nodepart{four}
\footnotesize{$1$}
};
 & 
\\
};
\end{scope}
\draw (sn0xef1700.south) -- (sn0xef2a50.north);
\draw (sn0xef1700.south) -- (sn0xef2540.north);
\draw (sn0xef1700.south) -- (sn0xefc570.north);
\draw (sn0xef1700.south) -- (sn0xefc970.north);
\draw (sn0xef1700.south) -- (sn0xefcc90.north);
\draw (sn0xefc570.south) -- (sn0xefd2a0.north);
\draw (sn0xefc570.south) -- (sn0xefe2f0.north);
\draw (sn0xefc570.south) -- (sn0xef2f20.north);
\draw (sn0xefc570.south) -- (sn0xefa920.north);
\draw (sn0xefc970.south) -- (sn0xef2f20.north);
\draw (sn0xefc970.south) -- (sn0xef3e70.north);
\draw (sn0xefcc90.south) -- (sn0xefa920.north);
\draw (sn0xefcc90.south) -- (sn0xef3e70.north);
\draw (sn0xefcc90.south) -- (sn0xeffff0.north);
\draw (sn0xefcc90.south) -- (sn0xf00d20.north);
\draw (sn0xefcc90.south) -- (sn0xf01470.north);
\draw (sn0xef2a50.south) -- (sn0xef2f20.north);
\draw (sn0xef2a50.south) -- (sn0xef3e70.north);
\draw (sn0xef2540.south) -- (sn0xefa920.north);
\draw (sn0xef2540.south) -- (sn0xef3e70.north);
\draw (sn0xef2540.south) -- (sn0xefabd0.north);
\draw (sn0xef2540.south) -- (sn0xefb500.north);
\draw (sn0xefd2a0.south) -- (sn0xef43a0.north);
\draw (sn0xefe2f0.south) -- (sn0xef43a0.north);
\draw (sn0xefe2f0.south) -- (sn0xefe720.north);
\draw (sn0xefe2f0.south) -- (sn0xefea20.north);
\draw (sn0xef2f20.south) -- (sn0xef43a0.north);
\draw (sn0xef2f20.south) -- (sn0xef4560.north);
\draw (sn0xef2f20.south) -- (sn0xef4790.north);
\draw (sn0xef3e70.south) -- (sn0xef4790.north);
\draw (sn0xef3e70.south) -- (sn0xef8300.north);
\draw (sn0xef3e70.south) -- (sn0xef90f0.north);
\draw (sn0xefa920.south) -- (sn0xef43a0.north);
\draw (sn0xefa920.south) -- (sn0xef8300.north);
\draw (sn0xefa920.south) -- (sn0xefbc60.north);
\draw (sn0xefabd0.south) -- (sn0xef8300.north);
\draw (sn0xefabd0.south) -- (sn0xef90f0.north);
\draw (sn0xefb500.south) -- (sn0xefbc60.north);
\draw (sn0xefb500.south) -- (sn0xef90f0.north);
\draw (sn0xefb500.south) -- (sn0xefc0b0.north);
\draw (sn0xeffff0.south) -- (sn0xefe720.north);
\draw (sn0xeffff0.south) -- (sn0xefea20.north);
\draw (sn0xeffff0.south) -- (sn0xef8300.north);
\draw (sn0xeffff0.south) -- (sn0xefbc60.north);
\draw (sn0xf00d20.south) -- (sn0xef8300.north);
\draw (sn0xf00d20.south) -- (sn0xef90f0.north);
\draw (sn0xf01470.south) -- (sn0xefbc60.north);
\draw (sn0xf01470.south) -- (sn0xef90f0.north);
\draw (sn0xf01470.south) -- (sn0xf01940.north);
\draw (sn0xf01470.south) -- (sn0xf01e70.north);
\draw (sn0xf01470.south) -- (sn0xf02290.north);
\draw (sn0xef43a0.south) -- (sn0xef48f0.north);
\draw (sn0xef43a0.south) -- (sn0xef5110.north);
\draw (sn0xef4560.south) -- (sn0xef48f0.north);
\draw (sn0xef4790.south) -- (sn0xef48f0.north);
\draw (sn0xef4790.south) -- (sn0xef7880.north);
\draw (sn0xef4790.south) -- (sn0xef7c10.north);
\draw (sn0xef8300.south) -- (sn0xef5110.north);
\draw (sn0xef8300.south) -- (sn0xef7880.north);
\draw (sn0xef8300.south) -- (sn0xef7c10.north);
\draw (sn0xef90f0.south) -- (sn0xef7c10.north);
\draw (sn0xef90f0.south) -- (sn0xef9e10.north);
\draw (sn0xef90f0.south) -- (sn0xefa190.north);
\draw (sn0xefbc60.south) -- (sn0xef5110.north);
\draw (sn0xefbc60.south) -- (sn0xef9e10.north);
\draw (sn0xefe720.south) -- (sn0xef5110.north);
\draw (sn0xefea20.south) -- (sn0xef5110.north);
\draw (sn0xefea20.south) -- (sn0xefeea0.north);
\draw (sn0xefea20.south) -- (sn0xeff210.north);
\draw (sn0xf01940.south) -- (sn0xefeea0.north);
\draw (sn0xf01940.south) -- (sn0xeff210.north);
\draw (sn0xf01940.south) -- (sn0xef9e10.north);
\draw (sn0xf01e70.south) -- (sn0xef9e10.north);
\draw (sn0xf01e70.south) -- (sn0xefa190.north);
\draw (sn0xf02290.south) -- (sn0xef9e10.north);
\draw (sn0xf02290.south) -- (sn0xefa190.north);
\draw (sn0xf02290.south) -- (sn0xf026b0.north);
\draw (sn0xf02290.south) -- (sn0xf02bb0.north);
\draw (sn0xefc0b0.south) -- (sn0xef9e10.north);
\draw (sn0xefc0b0.south) -- (sn0xefa190.north);
\draw (sn0xef48f0.south) -- (sn0xef5310.north);
\draw (sn0xef48f0.south) -- (sn0xef5750.north);
\draw (sn0xef5110.south) -- (sn0xef5750.north);
\draw (sn0xef5110.south) -- (sn0xef7040.north);
\draw (sn0xef7880.south) -- (sn0xef5750.north);
\draw (sn0xef7c10.south) -- (sn0xef5750.north);
\draw (sn0xef7c10.south) -- (sn0xef7d10.north);
\draw (sn0xef7c10.south) -- (sn0xef8590.north);
\draw (sn0xefeea0.south) -- (sn0xef7040.north);
\draw (sn0xeff210.south) -- (sn0xef7040.north);
\draw (sn0xeff210.south) -- (sn0xeff630.north);
\draw (sn0xeff210.south) -- (sn0xeff8f0.north);
\draw (sn0xf026b0.south) -- (sn0xeff630.north);
\draw (sn0xf026b0.south) -- (sn0xeff8f0.north);
\draw (sn0xf026b0.south) -- (sn0xef7d10.north);
\draw (sn0xf026b0.south) -- (sn0xef8590.north);
\draw (sn0xf02bb0.south) -- (sn0xef8590.north);
\draw (sn0xef9e10.south) -- (sn0xef7040.north);
\draw (sn0xef9e10.south) -- (sn0xef7d10.north);
\draw (sn0xef9e10.south) -- (sn0xef8590.north);
\draw (sn0xefa190.south) -- (sn0xef8590.north);
\draw (sn0xef5310.south) -- (sn0xef5850.north);
\draw (sn0xef5750.south) -- (sn0xef5850.north);
\draw (sn0xef5750.south) -- (sn0xef64d0.north);
\draw (sn0xeff630.south) -- (sn0xeffc80.north);
\draw (sn0xeff630.south) -- (sn0xef64d0.north);
\draw (sn0xeff8f0.south) -- (sn0xef64d0.north);
\draw (sn0xef7040.south) -- (sn0xef64d0.north);
\draw (sn0xef7d10.south) -- (sn0xef64d0.north);
\draw (sn0xef8590.south) -- (sn0xef64d0.north);
\draw (sn0xef8590.south) -- (sn0xef8750.north);
\draw (sn0xef5850.south) -- (sn0xef5c90.north);
\draw (sn0xeffc80.south) -- (sn0xef5c90.north);
\draw (sn0xef64d0.south) -- (sn0xef5c90.north);
\draw (sn0xef64d0.south) -- (sn0xef6940.north);
\draw (sn0xef8750.south) -- (sn0xef6940.north);
\draw (sn0xef5c90.south) -- (sn0xef5950.north);
\draw (sn0xef6940.south) -- (sn0xef5950.north);
\draw (sn0xef5950.south) -- (sn0xef5b30.north);
\end{tikzpicture}

%%% Local Variables:
%%% TeX-master: "thesis/thesis.tex"
%%% End: 
\renewcommand{\leveltopI}{-15cm + \leveltop}
\renewcommand{\leveltopII}{-15cm + \leveltopI}
\renewcommand{\leveltopIII}{-15cm + \leveltopII}
\renewcommand{\leveltopIIII}{-15cm + \leveltopIII}
\renewcommand{\leveltopIIIII}{-15cm + \leveltopIIII}
\renewcommand{\leveltopIIIIII}{-15cm + \leveltopIIIII}
\renewcommand{\leveltopIIIIIII}{-15cm + \leveltopIIIIII}
\renewcommand{\leveltopIIIIIIII}{-15cm + \leveltopIIIIIII}
\renewcommand{\leveltopIIIIIIIII}{-15cm + \leveltopIIIIIIII}
\renewcommand{\leveltopIIIIIIIIII}{-15cm + \leveltopIIIIIIIII}
\renewcommand{\leveltopIIIIIIIIIII}{-15cm + \leveltopIIIIIIIIII}
\begin{tikzpicture}[scale=.2, anchor=south]
\begin{scope}[yshift=\leveltopI cm]
\matrix (line1)[column sep=0.5cm] {
\node[draw=black, rectangle split,  rectangle split parts=4] (sn0xef1ca0){
\footnotesize{100}
\nodepart{two}
\begin{tikzpicture}[scale=.2]
\node[circle, scale=0.75, fill] (tid0) at (3,1.5){};
\node[circle, scale=0.75, fill] (tid1) at (3,3){};
\node[circle, scale=0.75, fill] (tid2) at (0.75,4.5){};
\node[circle, scale=0.75, fill] (tid5) at (0.75,6){};
\node[circle, scale=0.75, fill] (tid9) at (0.75,7.5){};
\node[circle, scale=0.75, fill, task_scheduled] (tid10) at (0.75,9){};
\draw[](tid9) -- (tid10);
\draw[](tid5) -- (tid9);
\draw[](tid2) -- (tid5);
\node[circle, scale=0.75, fill] (tid3) at (3,4.5){};
\node[circle, scale=0.75, fill, task_scheduled] (tid6) at (2.25,6){};
\node[circle, scale=0.75, fill] (tid7) at (3.75,6){};
\draw[](tid3) -- (tid6);
\draw[](tid3) -- (tid7);
\node[circle, scale=0.75, fill] (tid4) at (5.25,4.5){};
\node[circle, scale=0.75, fill] (tid8) at (5.25,6){};
\draw[](tid4) -- (tid8);
\draw[](tid1) -- (tid2);
\draw[](tid1) -- (tid3);
\draw[](tid1) -- (tid4);
\draw[](tid0) -- (tid1);
\end{tikzpicture}
\nodepart{three}
\footnotesize{7.36497}
\nodepart{four}
\footnotesize{$50\:17\:17\:17$}
};
 & 
\\
};
\end{scope}
\begin{scope}[yshift=\leveltopII cm]
\matrix (line2)[column sep=0.5cm] {
\node[draw=black, rectangle split,  rectangle split parts=4] (sn0xef2540){
\footnotesize{50}
\nodepart{two}
\begin{tikzpicture}[scale=.2]
\node[circle, scale=0.75, fill] (tid0) at (2.25,1.5){};
\node[circle, scale=0.75, fill] (tid1) at (2.25,3){};
\node[circle, scale=0.75, fill] (tid2) at (0.75,4.5){};
\node[circle, scale=0.75, fill] (tid5) at (0.75,6){};
\node[circle, scale=0.75, fill] (tid8) at (0.75,7.5){};
\node[circle, scale=0.75, fill, task_scheduled] (tid9) at (0.75,9){};
\draw[](tid8) -- (tid9);
\draw[](tid5) -- (tid8);
\draw[](tid2) -- (tid5);
\node[circle, scale=0.75, fill] (tid3) at (2.25,4.5){};
\node[circle, scale=0.75, fill, task_scheduled] (tid6) at (2.25,6){};
\draw[](tid3) -- (tid6);
\node[circle, scale=0.75, fill] (tid4) at (3.75,4.5){};
\node[circle, scale=0.75, fill] (tid7) at (3.75,6){};
\draw[](tid4) -- (tid7);
\draw[](tid1) -- (tid2);
\draw[](tid1) -- (tid3);
\draw[](tid1) -- (tid4);
\draw[](tid0) -- (tid1);
\end{tikzpicture}
\nodepart{three}
\footnotesize{6.94629}
\nodepart{four}
\footnotesize{$25\:25\:25\:25$}
};
 & 
\node[draw=black, rectangle split,  rectangle split parts=4] (sn0xf02cb0){
\footnotesize{16.6667}
\nodepart{two}
\begin{tikzpicture}[scale=.2]
\node[circle, scale=0.75, fill] (tid0) at (3,1.5){};
\node[circle, scale=0.75, fill] (tid1) at (3,3){};
\node[circle, scale=0.75, fill] (tid2) at (1.5,4.5){};
\node[circle, scale=0.75, fill, task_scheduled] (tid5) at (0.75,6){};
\node[circle, scale=0.75, fill, task_scheduled] (tid6) at (2.25,6){};
\draw[](tid2) -- (tid5);
\draw[](tid2) -- (tid6);
\node[circle, scale=0.75, fill] (tid3) at (3.75,4.5){};
\node[circle, scale=0.75, fill] (tid7) at (3.75,6){};
\node[circle, scale=0.75, fill] (tid9) at (3.75,7.5){};
\draw[](tid7) -- (tid9);
\draw[](tid3) -- (tid7);
\node[circle, scale=0.75, fill] (tid4) at (5.25,4.5){};
\node[circle, scale=0.75, fill] (tid8) at (5.25,6){};
\draw[](tid4) -- (tid8);
\draw[](tid1) -- (tid2);
\draw[](tid1) -- (tid3);
\draw[](tid1) -- (tid4);
\draw[](tid0) -- (tid1);
\end{tikzpicture}
\nodepart{three}
\footnotesize{6.81152}
\nodepart{four}
\footnotesize{$50\:50$}
};
 & 
\node[draw=black, rectangle split,  rectangle split parts=4] (sn0xf03560){
\footnotesize{16.6667}
\nodepart{two}
\begin{tikzpicture}[scale=.2]
\node[circle, scale=0.75, fill] (tid0) at (3,1.5){};
\node[circle, scale=0.75, fill] (tid1) at (3,3){};
\node[circle, scale=0.75, fill] (tid2) at (1.5,4.5){};
\node[circle, scale=0.75, fill, task_scheduled] (tid5) at (0.75,6){};
\node[circle, scale=0.75, fill] (tid6) at (2.25,6){};
\draw[](tid2) -- (tid5);
\draw[](tid2) -- (tid6);
\node[circle, scale=0.75, fill] (tid3) at (3.75,4.5){};
\node[circle, scale=0.75, fill] (tid7) at (3.75,6){};
\node[circle, scale=0.75, fill] (tid9) at (3.75,7.5){};
\draw[](tid7) -- (tid9);
\draw[](tid3) -- (tid7);
\node[circle, scale=0.75, fill] (tid4) at (5.25,4.5){};
\node[circle, scale=0.75, fill, task_scheduled] (tid8) at (5.25,6){};
\draw[](tid4) -- (tid8);
\draw[](tid1) -- (tid2);
\draw[](tid1) -- (tid3);
\draw[](tid1) -- (tid4);
\draw[](tid0) -- (tid1);
\end{tikzpicture}
\nodepart{three}
\footnotesize{6.82847}
\nodepart{four}
\footnotesize{$25\:25\:17\:17\:17$}
};
 & 
\node[draw=black, rectangle split,  rectangle split parts=4] (sn0xf038a0){
\footnotesize{16.6667}
\nodepart{two}
\begin{tikzpicture}[scale=.2]
\node[circle, scale=0.75, fill] (tid0) at (3,1.5){};
\node[circle, scale=0.75, fill] (tid1) at (3,3){};
\node[circle, scale=0.75, fill] (tid2) at (1.5,4.5){};
\node[circle, scale=0.75, fill, task_scheduled] (tid5) at (0.75,6){};
\node[circle, scale=0.75, fill] (tid6) at (2.25,6){};
\draw[](tid2) -- (tid5);
\draw[](tid2) -- (tid6);
\node[circle, scale=0.75, fill] (tid3) at (3.75,4.5){};
\node[circle, scale=0.75, fill] (tid7) at (3.75,6){};
\node[circle, scale=0.75, fill, task_scheduled] (tid9) at (3.75,7.5){};
\draw[](tid7) -- (tid9);
\draw[](tid3) -- (tid7);
\node[circle, scale=0.75, fill] (tid4) at (5.25,4.5){};
\node[circle, scale=0.75, fill] (tid8) at (5.25,6){};
\draw[](tid4) -- (tid8);
\draw[](tid1) -- (tid2);
\draw[](tid1) -- (tid3);
\draw[](tid1) -- (tid4);
\draw[](tid0) -- (tid1);
\end{tikzpicture}
\nodepart{three}
\footnotesize{6.71097}
\nodepart{four}
\footnotesize{$50\:17\:33$}
};
 & 
\\
};
\end{scope}
\begin{scope}[yshift=\leveltopIII cm]
\matrix (line3)[column sep=0.5cm] {
\node[draw=black, rectangle split,  rectangle split parts=4] (sn0xefa920){
\footnotesize{12.5}
\nodepart{two}
\begin{tikzpicture}[scale=.2]
\node[circle, scale=0.75, fill] (tid0) at (2.25,1.5){};
\node[circle, scale=0.75, fill] (tid1) at (2.25,3){};
\node[circle, scale=0.75, fill] (tid2) at (0.75,4.5){};
\node[circle, scale=0.75, fill] (tid5) at (0.75,6){};
\node[circle, scale=0.75, fill] (tid7) at (0.75,7.5){};
\node[circle, scale=0.75, fill, task_scheduled] (tid8) at (0.75,9){};
\draw[](tid7) -- (tid8);
\draw[](tid5) -- (tid7);
\draw[](tid2) -- (tid5);
\node[circle, scale=0.75, fill] (tid3) at (2.25,4.5){};
\node[circle, scale=0.75, fill] (tid6) at (2.25,6){};
\draw[](tid3) -- (tid6);
\node[circle, scale=0.75, fill, task_scheduled] (tid4) at (3.75,4.5){};
\draw[](tid1) -- (tid2);
\draw[](tid1) -- (tid3);
\draw[](tid1) -- (tid4);
\draw[](tid0) -- (tid1);
\end{tikzpicture}
\nodepart{three}
\footnotesize{6.57617}
\nodepart{four}
\footnotesize{$50\:25\:25$}
};
 & 
\node[draw=black, rectangle split,  rectangle split parts=4] (sn0xef3e70){
\footnotesize{12.5}
\nodepart{two}
\begin{tikzpicture}[scale=.2]
\node[circle, scale=0.75, fill] (tid0) at (2.25,1.5){};
\node[circle, scale=0.75, fill] (tid1) at (2.25,3){};
\node[circle, scale=0.75, fill] (tid2) at (0.75,4.5){};
\node[circle, scale=0.75, fill] (tid5) at (0.75,6){};
\node[circle, scale=0.75, fill] (tid7) at (0.75,7.5){};
\node[circle, scale=0.75, fill, task_scheduled] (tid8) at (0.75,9){};
\draw[](tid7) -- (tid8);
\draw[](tid5) -- (tid7);
\draw[](tid2) -- (tid5);
\node[circle, scale=0.75, fill] (tid3) at (2.25,4.5){};
\node[circle, scale=0.75, fill, task_scheduled] (tid6) at (2.25,6){};
\draw[](tid3) -- (tid6);
\node[circle, scale=0.75, fill] (tid4) at (3.75,4.5){};
\draw[](tid1) -- (tid2);
\draw[](tid1) -- (tid3);
\draw[](tid1) -- (tid4);
\draw[](tid0) -- (tid1);
\end{tikzpicture}
\nodepart{three}
\footnotesize{6.58594}
\nodepart{four}
\footnotesize{$50\:25\:25$}
};
 & 
\node[draw=black, rectangle split,  rectangle split parts=4] (sn0xefabd0){
\footnotesize{25}
\nodepart{two}
\begin{tikzpicture}[scale=.2]
\node[circle, scale=0.75, fill] (tid0) at (2.25,1.5){};
\node[circle, scale=0.75, fill] (tid1) at (2.25,3){};
\node[circle, scale=0.75, fill] (tid2) at (0.75,4.5){};
\node[circle, scale=0.75, fill] (tid5) at (0.75,6){};
\node[circle, scale=0.75, fill] (tid8) at (0.75,7.5){};
\draw[](tid5) -- (tid8);
\draw[](tid2) -- (tid5);
\node[circle, scale=0.75, fill] (tid3) at (2.25,4.5){};
\node[circle, scale=0.75, fill, task_scheduled] (tid6) at (2.25,6){};
\draw[](tid3) -- (tid6);
\node[circle, scale=0.75, fill] (tid4) at (3.75,4.5){};
\node[circle, scale=0.75, fill, task_scheduled] (tid7) at (3.75,6){};
\draw[](tid4) -- (tid7);
\draw[](tid1) -- (tid2);
\draw[](tid1) -- (tid3);
\draw[](tid1) -- (tid4);
\draw[](tid0) -- (tid1);
\end{tikzpicture}
\nodepart{three}
\footnotesize{6.38281}
\nodepart{four}
\footnotesize{$50\:50$}
};
 & 
\node[draw=black, rectangle split,  rectangle split parts=4] (sn0xefb500){
\footnotesize{33.3333}
\nodepart{two}
\begin{tikzpicture}[scale=.2]
\node[circle, scale=0.75, fill] (tid0) at (2.25,1.5){};
\node[circle, scale=0.75, fill] (tid1) at (2.25,3){};
\node[circle, scale=0.75, fill] (tid2) at (0.75,4.5){};
\node[circle, scale=0.75, fill] (tid5) at (0.75,6){};
\node[circle, scale=0.75, fill, task_scheduled] (tid8) at (0.75,7.5){};
\draw[](tid5) -- (tid8);
\draw[](tid2) -- (tid5);
\node[circle, scale=0.75, fill] (tid3) at (2.25,4.5){};
\node[circle, scale=0.75, fill, task_scheduled] (tid6) at (2.25,6){};
\draw[](tid3) -- (tid6);
\node[circle, scale=0.75, fill] (tid4) at (3.75,4.5){};
\node[circle, scale=0.75, fill] (tid7) at (3.75,6){};
\draw[](tid4) -- (tid7);
\draw[](tid1) -- (tid2);
\draw[](tid1) -- (tid3);
\draw[](tid1) -- (tid4);
\draw[](tid0) -- (tid1);
\end{tikzpicture}
\nodepart{three}
\footnotesize{6.24023}
\nodepart{four}
\footnotesize{$25\:25\:50$}
};
 & 
\node[draw=black, rectangle split,  rectangle split parts=4] (sn0xeffff0){
\footnotesize{2.77778}
\nodepart{two}
\begin{tikzpicture}[scale=.2]
\node[circle, scale=0.75, fill] (tid0) at (3,1.5){};
\node[circle, scale=0.75, fill] (tid1) at (3,3){};
\node[circle, scale=0.75, fill] (tid2) at (1.5,4.5){};
\node[circle, scale=0.75, fill, task_scheduled] (tid5) at (0.75,6){};
\node[circle, scale=0.75, fill] (tid6) at (2.25,6){};
\draw[](tid2) -- (tid5);
\draw[](tid2) -- (tid6);
\node[circle, scale=0.75, fill] (tid3) at (3.75,4.5){};
\node[circle, scale=0.75, fill] (tid7) at (3.75,6){};
\node[circle, scale=0.75, fill] (tid8) at (3.75,7.5){};
\draw[](tid7) -- (tid8);
\draw[](tid3) -- (tid7);
\node[circle, scale=0.75, fill, task_scheduled] (tid4) at (5.25,4.5){};
\draw[](tid1) -- (tid2);
\draw[](tid1) -- (tid3);
\draw[](tid1) -- (tid4);
\draw[](tid0) -- (tid1);
\end{tikzpicture}
\nodepart{three}
\footnotesize{6.39844}
\nodepart{four}
\footnotesize{$25\:25\:25\:25$}
};
 & 
\node[draw=black, rectangle split,  rectangle split parts=4] (sn0xf00d20){
\footnotesize{2.77778}
\nodepart{two}
\begin{tikzpicture}[scale=.2]
\node[circle, scale=0.75, fill] (tid0) at (3,1.5){};
\node[circle, scale=0.75, fill] (tid1) at (3,3){};
\node[circle, scale=0.75, fill] (tid2) at (1.5,4.5){};
\node[circle, scale=0.75, fill, task_scheduled] (tid5) at (0.75,6){};
\node[circle, scale=0.75, fill, task_scheduled] (tid6) at (2.25,6){};
\draw[](tid2) -- (tid5);
\draw[](tid2) -- (tid6);
\node[circle, scale=0.75, fill] (tid3) at (3.75,4.5){};
\node[circle, scale=0.75, fill] (tid7) at (3.75,6){};
\node[circle, scale=0.75, fill] (tid8) at (3.75,7.5){};
\draw[](tid7) -- (tid8);
\draw[](tid3) -- (tid7);
\node[circle, scale=0.75, fill] (tid4) at (5.25,4.5){};
\draw[](tid1) -- (tid2);
\draw[](tid1) -- (tid3);
\draw[](tid1) -- (tid4);
\draw[](tid0) -- (tid1);
\end{tikzpicture}
\nodepart{three}
\footnotesize{6.38281}
\nodepart{four}
\footnotesize{$50\:50$}
};
 & 
\node[draw=black, rectangle split,  rectangle split parts=4] (sn0xf01470){
\footnotesize{2.77778}
\nodepart{two}
\begin{tikzpicture}[scale=.2]
\node[circle, scale=0.75, fill] (tid0) at (3,1.5){};
\node[circle, scale=0.75, fill] (tid1) at (3,3){};
\node[circle, scale=0.75, fill] (tid2) at (1.5,4.5){};
\node[circle, scale=0.75, fill, task_scheduled] (tid5) at (0.75,6){};
\node[circle, scale=0.75, fill] (tid6) at (2.25,6){};
\draw[](tid2) -- (tid5);
\draw[](tid2) -- (tid6);
\node[circle, scale=0.75, fill] (tid3) at (3.75,4.5){};
\node[circle, scale=0.75, fill] (tid7) at (3.75,6){};
\node[circle, scale=0.75, fill, task_scheduled] (tid8) at (3.75,7.5){};
\draw[](tid7) -- (tid8);
\draw[](tid3) -- (tid7);
\node[circle, scale=0.75, fill] (tid4) at (5.25,4.5){};
\draw[](tid1) -- (tid2);
\draw[](tid1) -- (tid3);
\draw[](tid1) -- (tid4);
\draw[](tid0) -- (tid1);
\end{tikzpicture}
\nodepart{three}
\footnotesize{6.25499}
\nodepart{four}
\footnotesize{$25\:25\:17\:17\:17$}
};
 & 
\node[draw=black, rectangle split,  rectangle split parts=4] (sn0xf04590){
\footnotesize{2.77778}
\nodepart{two}
\begin{tikzpicture}[scale=.2]
\node[circle, scale=0.75, fill] (tid0) at (3,1.5){};
\node[circle, scale=0.75, fill] (tid1) at (3,3){};
\node[circle, scale=0.75, fill] (tid2) at (1.5,4.5){};
\node[circle, scale=0.75, fill, task_scheduled] (tid5) at (0.75,6){};
\node[circle, scale=0.75, fill, task_scheduled] (tid6) at (2.25,6){};
\draw[](tid2) -- (tid5);
\draw[](tid2) -- (tid6);
\node[circle, scale=0.75, fill] (tid3) at (3.75,4.5){};
\node[circle, scale=0.75, fill] (tid7) at (3.75,6){};
\draw[](tid3) -- (tid7);
\node[circle, scale=0.75, fill] (tid4) at (5.25,4.5){};
\node[circle, scale=0.75, fill] (tid8) at (5.25,6){};
\draw[](tid4) -- (tid8);
\draw[](tid1) -- (tid2);
\draw[](tid1) -- (tid3);
\draw[](tid1) -- (tid4);
\draw[](tid0) -- (tid1);
\end{tikzpicture}
\nodepart{three}
\footnotesize{6.17188}
\nodepart{four}
\footnotesize{$1$}
};
 & 
\node[draw=black, rectangle split,  rectangle split parts=4] (sn0xf04f90){
\footnotesize{5.55556}
\nodepart{two}
\begin{tikzpicture}[scale=.2]
\node[circle, scale=0.75, fill] (tid0) at (3,1.5){};
\node[circle, scale=0.75, fill] (tid1) at (3,3){};
\node[circle, scale=0.75, fill] (tid2) at (1.5,4.5){};
\node[circle, scale=0.75, fill, task_scheduled] (tid5) at (0.75,6){};
\node[circle, scale=0.75, fill] (tid6) at (2.25,6){};
\draw[](tid2) -- (tid5);
\draw[](tid2) -- (tid6);
\node[circle, scale=0.75, fill] (tid3) at (3.75,4.5){};
\node[circle, scale=0.75, fill, task_scheduled] (tid7) at (3.75,6){};
\draw[](tid3) -- (tid7);
\node[circle, scale=0.75, fill] (tid4) at (5.25,4.5){};
\node[circle, scale=0.75, fill] (tid8) at (5.25,6){};
\draw[](tid4) -- (tid8);
\draw[](tid1) -- (tid2);
\draw[](tid1) -- (tid3);
\draw[](tid1) -- (tid4);
\draw[](tid0) -- (tid1);
\end{tikzpicture}
\nodepart{three}
\footnotesize{6.18663}
\nodepart{four}
\footnotesize{$17\:17\:17\:50$}
};
 & 
\\
};
\end{scope}
\begin{scope}[yshift=\leveltopIIII cm]
\matrix (line4)[column sep=0.5cm] {
\node[draw=black, rectangle split,  rectangle split parts=4] (sn0xef43a0){
\footnotesize{6.25}
\nodepart{two}
\begin{tikzpicture}[scale=.2]
\node[circle, scale=0.75, fill] (tid0) at (1.5,1.5){};
\node[circle, scale=0.75, fill] (tid1) at (1.5,3){};
\node[circle, scale=0.75, fill] (tid2) at (0.75,4.5){};
\node[circle, scale=0.75, fill] (tid4) at (0.75,6){};
\node[circle, scale=0.75, fill] (tid6) at (0.75,7.5){};
\node[circle, scale=0.75, fill, task_scheduled] (tid7) at (0.75,9){};
\draw[](tid6) -- (tid7);
\draw[](tid4) -- (tid6);
\draw[](tid2) -- (tid4);
\node[circle, scale=0.75, fill] (tid3) at (2.25,4.5){};
\node[circle, scale=0.75, fill, task_scheduled] (tid5) at (2.25,6){};
\draw[](tid3) -- (tid5);
\draw[](tid1) -- (tid2);
\draw[](tid1) -- (tid3);
\draw[](tid0) -- (tid1);
\end{tikzpicture}
\nodepart{three}
\footnotesize{6.25}
\nodepart{four}
\footnotesize{$50\:50$}
};
 & 
\node[draw=black, rectangle split,  rectangle split parts=4] (sn0xef4790){
\footnotesize{6.25}
\nodepart{two}
\begin{tikzpicture}[scale=.2]
\node[circle, scale=0.75, fill] (tid0) at (2.25,1.5){};
\node[circle, scale=0.75, fill] (tid1) at (2.25,3){};
\node[circle, scale=0.75, fill] (tid2) at (0.75,4.5){};
\node[circle, scale=0.75, fill] (tid5) at (0.75,6){};
\node[circle, scale=0.75, fill] (tid6) at (0.75,7.5){};
\node[circle, scale=0.75, fill, task_scheduled] (tid7) at (0.75,9){};
\draw[](tid6) -- (tid7);
\draw[](tid5) -- (tid6);
\draw[](tid2) -- (tid5);
\node[circle, scale=0.75, fill, task_scheduled] (tid3) at (2.25,4.5){};
\node[circle, scale=0.75, fill] (tid4) at (3.75,4.5){};
\draw[](tid1) -- (tid2);
\draw[](tid1) -- (tid3);
\draw[](tid1) -- (tid4);
\draw[](tid0) -- (tid1);
\end{tikzpicture}
\nodepart{three}
\footnotesize{6.28906}
\nodepart{four}
\footnotesize{$50\:25\:25$}
};
 & 
\node[draw=black, rectangle split,  rectangle split parts=4] (sn0xef8300){
\footnotesize{20.8333}
\nodepart{two}
\begin{tikzpicture}[scale=.2]
\node[circle, scale=0.75, fill] (tid0) at (2.25,1.5){};
\node[circle, scale=0.75, fill] (tid1) at (2.25,3){};
\node[circle, scale=0.75, fill] (tid2) at (0.75,4.5){};
\node[circle, scale=0.75, fill] (tid5) at (0.75,6){};
\node[circle, scale=0.75, fill] (tid7) at (0.75,7.5){};
\draw[](tid5) -- (tid7);
\draw[](tid2) -- (tid5);
\node[circle, scale=0.75, fill] (tid3) at (2.25,4.5){};
\node[circle, scale=0.75, fill, task_scheduled] (tid6) at (2.25,6){};
\draw[](tid3) -- (tid6);
\node[circle, scale=0.75, fill, task_scheduled] (tid4) at (3.75,4.5){};
\draw[](tid1) -- (tid2);
\draw[](tid1) -- (tid3);
\draw[](tid1) -- (tid4);
\draw[](tid0) -- (tid1);
\end{tikzpicture}
\nodepart{three}
\footnotesize{5.97656}
\nodepart{four}
\footnotesize{$50\:25\:25$}
};
 & 
\node[draw=black, rectangle split,  rectangle split parts=4] (sn0xefbc60){
\footnotesize{12.8472}
\nodepart{two}
\begin{tikzpicture}[scale=.2]
\node[circle, scale=0.75, fill] (tid0) at (2.25,1.5){};
\node[circle, scale=0.75, fill] (tid1) at (2.25,3){};
\node[circle, scale=0.75, fill] (tid2) at (0.75,4.5){};
\node[circle, scale=0.75, fill] (tid5) at (0.75,6){};
\node[circle, scale=0.75, fill, task_scheduled] (tid7) at (0.75,7.5){};
\draw[](tid5) -- (tid7);
\draw[](tid2) -- (tid5);
\node[circle, scale=0.75, fill] (tid3) at (2.25,4.5){};
\node[circle, scale=0.75, fill] (tid6) at (2.25,6){};
\draw[](tid3) -- (tid6);
\node[circle, scale=0.75, fill, task_scheduled] (tid4) at (3.75,4.5){};
\draw[](tid1) -- (tid2);
\draw[](tid1) -- (tid3);
\draw[](tid1) -- (tid4);
\draw[](tid0) -- (tid1);
\end{tikzpicture}
\nodepart{three}
\footnotesize{5.82812}
\nodepart{four}
\footnotesize{$50\:50$}
};
 & 
\node[draw=black, rectangle split,  rectangle split parts=4] (sn0xef90f0){
\footnotesize{26.0417}
\nodepart{two}
\begin{tikzpicture}[scale=.2]
\node[circle, scale=0.75, fill] (tid0) at (2.25,1.5){};
\node[circle, scale=0.75, fill] (tid1) at (2.25,3){};
\node[circle, scale=0.75, fill] (tid2) at (0.75,4.5){};
\node[circle, scale=0.75, fill] (tid5) at (0.75,6){};
\node[circle, scale=0.75, fill, task_scheduled] (tid7) at (0.75,7.5){};
\draw[](tid5) -- (tid7);
\draw[](tid2) -- (tid5);
\node[circle, scale=0.75, fill] (tid3) at (2.25,4.5){};
\node[circle, scale=0.75, fill, task_scheduled] (tid6) at (2.25,6){};
\draw[](tid3) -- (tid6);
\node[circle, scale=0.75, fill] (tid4) at (3.75,4.5){};
\draw[](tid1) -- (tid2);
\draw[](tid1) -- (tid3);
\draw[](tid1) -- (tid4);
\draw[](tid0) -- (tid1);
\end{tikzpicture}
\nodepart{three}
\footnotesize{5.78906}
\nodepart{four}
\footnotesize{$50\:25\:25$}
};
 & 
\node[draw=black, rectangle split,  rectangle split parts=4] (sn0xefe720){
\footnotesize{0.694444}
\nodepart{two}
\begin{tikzpicture}[scale=.2]
\node[circle, scale=0.75, fill] (tid0) at (2.25,1.5){};
\node[circle, scale=0.75, fill] (tid1) at (2.25,3){};
\node[circle, scale=0.75, fill] (tid2) at (1.5,4.5){};
\node[circle, scale=0.75, fill, task_scheduled] (tid4) at (0.75,6){};
\node[circle, scale=0.75, fill, task_scheduled] (tid5) at (2.25,6){};
\draw[](tid2) -- (tid4);
\draw[](tid2) -- (tid5);
\node[circle, scale=0.75, fill] (tid3) at (3.75,4.5){};
\node[circle, scale=0.75, fill] (tid6) at (3.75,6){};
\node[circle, scale=0.75, fill] (tid7) at (3.75,7.5){};
\draw[](tid6) -- (tid7);
\draw[](tid3) -- (tid6);
\draw[](tid1) -- (tid2);
\draw[](tid1) -- (tid3);
\draw[](tid0) -- (tid1);
\end{tikzpicture}
\nodepart{three}
\footnotesize{5.9375}
\nodepart{four}
\footnotesize{$1$}
};
 & 
\node[draw=black, rectangle split,  rectangle split parts=4] (sn0xefea20){
\footnotesize{0.694444}
\nodepart{two}
\begin{tikzpicture}[scale=.2]
\node[circle, scale=0.75, fill] (tid0) at (2.25,1.5){};
\node[circle, scale=0.75, fill] (tid1) at (2.25,3){};
\node[circle, scale=0.75, fill] (tid2) at (1.5,4.5){};
\node[circle, scale=0.75, fill, task_scheduled] (tid4) at (0.75,6){};
\node[circle, scale=0.75, fill] (tid5) at (2.25,6){};
\draw[](tid2) -- (tid4);
\draw[](tid2) -- (tid5);
\node[circle, scale=0.75, fill] (tid3) at (3.75,4.5){};
\node[circle, scale=0.75, fill] (tid6) at (3.75,6){};
\node[circle, scale=0.75, fill, task_scheduled] (tid7) at (3.75,7.5){};
\draw[](tid6) -- (tid7);
\draw[](tid3) -- (tid6);
\draw[](tid1) -- (tid2);
\draw[](tid1) -- (tid3);
\draw[](tid0) -- (tid1);
\end{tikzpicture}
\nodepart{three}
\footnotesize{5.85156}
\nodepart{four}
\footnotesize{$50\:25\:25$}
};
 & 
\node[draw=black, rectangle split,  rectangle split parts=4] (sn0xf01940){
\footnotesize{1.38889}
\nodepart{two}
\begin{tikzpicture}[scale=.2]
\node[circle, scale=0.75, fill] (tid0) at (3,1.5){};
\node[circle, scale=0.75, fill] (tid1) at (3,3){};
\node[circle, scale=0.75, fill] (tid2) at (1.5,4.5){};
\node[circle, scale=0.75, fill, task_scheduled] (tid5) at (0.75,6){};
\node[circle, scale=0.75, fill] (tid6) at (2.25,6){};
\draw[](tid2) -- (tid5);
\draw[](tid2) -- (tid6);
\node[circle, scale=0.75, fill] (tid3) at (3.75,4.5){};
\node[circle, scale=0.75, fill] (tid7) at (3.75,6){};
\draw[](tid3) -- (tid7);
\node[circle, scale=0.75, fill, task_scheduled] (tid4) at (5.25,4.5){};
\draw[](tid1) -- (tid2);
\draw[](tid1) -- (tid3);
\draw[](tid1) -- (tid4);
\draw[](tid0) -- (tid1);
\end{tikzpicture}
\nodepart{three}
\footnotesize{5.74219}
\nodepart{four}
\footnotesize{$25\:25\:50$}
};
 & 
\node[draw=black, rectangle split,  rectangle split parts=4] (sn0xf01e70){
\footnotesize{1.38889}
\nodepart{two}
\begin{tikzpicture}[scale=.2]
\node[circle, scale=0.75, fill] (tid0) at (3,1.5){};
\node[circle, scale=0.75, fill] (tid1) at (3,3){};
\node[circle, scale=0.75, fill] (tid2) at (1.5,4.5){};
\node[circle, scale=0.75, fill, task_scheduled] (tid5) at (0.75,6){};
\node[circle, scale=0.75, fill, task_scheduled] (tid6) at (2.25,6){};
\draw[](tid2) -- (tid5);
\draw[](tid2) -- (tid6);
\node[circle, scale=0.75, fill] (tid3) at (3.75,4.5){};
\node[circle, scale=0.75, fill] (tid7) at (3.75,6){};
\draw[](tid3) -- (tid7);
\node[circle, scale=0.75, fill] (tid4) at (5.25,4.5){};
\draw[](tid1) -- (tid2);
\draw[](tid1) -- (tid3);
\draw[](tid1) -- (tid4);
\draw[](tid0) -- (tid1);
\end{tikzpicture}
\nodepart{three}
\footnotesize{5.67188}
\nodepart{four}
\footnotesize{$50\:50$}
};
 & 
\node[draw=black, rectangle split,  rectangle split parts=4] (sn0xf02290){
\footnotesize{1.38889}
\nodepart{two}
\begin{tikzpicture}[scale=.2]
\node[circle, scale=0.75, fill] (tid0) at (3,1.5){};
\node[circle, scale=0.75, fill] (tid1) at (3,3){};
\node[circle, scale=0.75, fill] (tid2) at (1.5,4.5){};
\node[circle, scale=0.75, fill, task_scheduled] (tid5) at (0.75,6){};
\node[circle, scale=0.75, fill] (tid6) at (2.25,6){};
\draw[](tid2) -- (tid5);
\draw[](tid2) -- (tid6);
\node[circle, scale=0.75, fill] (tid3) at (3.75,4.5){};
\node[circle, scale=0.75, fill, task_scheduled] (tid7) at (3.75,6){};
\draw[](tid3) -- (tid7);
\node[circle, scale=0.75, fill] (tid4) at (5.25,4.5){};
\draw[](tid1) -- (tid2);
\draw[](tid1) -- (tid3);
\draw[](tid1) -- (tid4);
\draw[](tid0) -- (tid1);
\end{tikzpicture}
\nodepart{three}
\footnotesize{5.6901}
\nodepart{four}
\footnotesize{$33\:17\:25\:25$}
};
 & 
\node[draw=black, rectangle split,  rectangle split parts=4] (sn0xefc0b0){
\footnotesize{22.2222}
\nodepart{two}
\begin{tikzpicture}[scale=.2]
\node[circle, scale=0.75, fill] (tid0) at (2.25,1.5){};
\node[circle, scale=0.75, fill] (tid1) at (2.25,3){};
\node[circle, scale=0.75, fill] (tid2) at (0.75,4.5){};
\node[circle, scale=0.75, fill, task_scheduled] (tid5) at (0.75,6){};
\draw[](tid2) -- (tid5);
\node[circle, scale=0.75, fill] (tid3) at (2.25,4.5){};
\node[circle, scale=0.75, fill, task_scheduled] (tid6) at (2.25,6){};
\draw[](tid3) -- (tid6);
\node[circle, scale=0.75, fill] (tid4) at (3.75,4.5){};
\node[circle, scale=0.75, fill] (tid7) at (3.75,6){};
\draw[](tid4) -- (tid7);
\draw[](tid1) -- (tid2);
\draw[](tid1) -- (tid3);
\draw[](tid1) -- (tid4);
\draw[](tid0) -- (tid1);
\end{tikzpicture}
\nodepart{three}
\footnotesize{5.67188}
\nodepart{four}
\footnotesize{$50\:50$}
};
 & 
\\
};
\end{scope}
\begin{scope}[yshift=\leveltopIIIII cm]
\matrix (line5)[column sep=0.5cm] {
\node[draw=black, rectangle split,  rectangle split parts=4] (sn0xef48f0){
\footnotesize{6.25}
\nodepart{two}
\begin{tikzpicture}[scale=.2]
\node[circle, scale=0.75, fill] (tid0) at (1.5,1.5){};
\node[circle, scale=0.75, fill] (tid1) at (1.5,3){};
\node[circle, scale=0.75, fill] (tid2) at (0.75,4.5){};
\node[circle, scale=0.75, fill] (tid4) at (0.75,6){};
\node[circle, scale=0.75, fill] (tid5) at (0.75,7.5){};
\node[circle, scale=0.75, fill, task_scheduled] (tid6) at (0.75,9){};
\draw[](tid5) -- (tid6);
\draw[](tid4) -- (tid5);
\draw[](tid2) -- (tid4);
\node[circle, scale=0.75, fill, task_scheduled] (tid3) at (2.25,4.5){};
\draw[](tid1) -- (tid2);
\draw[](tid1) -- (tid3);
\draw[](tid0) -- (tid1);
\end{tikzpicture}
\nodepart{three}
\footnotesize{6.0625}
\nodepart{four}
\footnotesize{$50\:50$}
};
 & 
\node[draw=black, rectangle split,  rectangle split parts=4] (sn0xef5110){
\footnotesize{21.0069}
\nodepart{two}
\begin{tikzpicture}[scale=.2]
\node[circle, scale=0.75, fill] (tid0) at (1.5,1.5){};
\node[circle, scale=0.75, fill] (tid1) at (1.5,3){};
\node[circle, scale=0.75, fill] (tid2) at (0.75,4.5){};
\node[circle, scale=0.75, fill] (tid4) at (0.75,6){};
\node[circle, scale=0.75, fill, task_scheduled] (tid6) at (0.75,7.5){};
\draw[](tid4) -- (tid6);
\draw[](tid2) -- (tid4);
\node[circle, scale=0.75, fill] (tid3) at (2.25,4.5){};
\node[circle, scale=0.75, fill, task_scheduled] (tid5) at (2.25,6){};
\draw[](tid3) -- (tid5);
\draw[](tid1) -- (tid2);
\draw[](tid1) -- (tid3);
\draw[](tid0) -- (tid1);
\end{tikzpicture}
\nodepart{three}
\footnotesize{5.4375}
\nodepart{four}
\footnotesize{$50\:50$}
};
 & 
\node[draw=black, rectangle split,  rectangle split parts=4] (sn0xef7880){
\footnotesize{6.77083}
\nodepart{two}
\begin{tikzpicture}[scale=.2]
\node[circle, scale=0.75, fill] (tid0) at (2.25,1.5){};
\node[circle, scale=0.75, fill] (tid1) at (2.25,3){};
\node[circle, scale=0.75, fill] (tid2) at (0.75,4.5){};
\node[circle, scale=0.75, fill] (tid5) at (0.75,6){};
\node[circle, scale=0.75, fill] (tid6) at (0.75,7.5){};
\draw[](tid5) -- (tid6);
\draw[](tid2) -- (tid5);
\node[circle, scale=0.75, fill, task_scheduled] (tid3) at (2.25,4.5){};
\node[circle, scale=0.75, fill, task_scheduled] (tid4) at (3.75,4.5){};
\draw[](tid1) -- (tid2);
\draw[](tid1) -- (tid3);
\draw[](tid1) -- (tid4);
\draw[](tid0) -- (tid1);
\end{tikzpicture}
\nodepart{three}
\footnotesize{5.625}
\nodepart{four}
\footnotesize{$1$}
};
 & 
\node[draw=black, rectangle split,  rectangle split parts=4] (sn0xef7c10){
\footnotesize{19.7917}
\nodepart{two}
\begin{tikzpicture}[scale=.2]
\node[circle, scale=0.75, fill] (tid0) at (2.25,1.5){};
\node[circle, scale=0.75, fill] (tid1) at (2.25,3){};
\node[circle, scale=0.75, fill] (tid2) at (0.75,4.5){};
\node[circle, scale=0.75, fill] (tid5) at (0.75,6){};
\node[circle, scale=0.75, fill, task_scheduled] (tid6) at (0.75,7.5){};
\draw[](tid5) -- (tid6);
\draw[](tid2) -- (tid5);
\node[circle, scale=0.75, fill, task_scheduled] (tid3) at (2.25,4.5){};
\node[circle, scale=0.75, fill] (tid4) at (3.75,4.5){};
\draw[](tid1) -- (tid2);
\draw[](tid1) -- (tid3);
\draw[](tid1) -- (tid4);
\draw[](tid0) -- (tid1);
\end{tikzpicture}
\nodepart{three}
\footnotesize{5.40625}
\nodepart{four}
\footnotesize{$50\:25\:25$}
};
 & 
\node[draw=black, rectangle split,  rectangle split parts=4] (sn0xefeea0){
\footnotesize{0.520833}
\nodepart{two}
\begin{tikzpicture}[scale=.2]
\node[circle, scale=0.75, fill] (tid0) at (2.25,1.5){};
\node[circle, scale=0.75, fill] (tid1) at (2.25,3){};
\node[circle, scale=0.75, fill] (tid2) at (1.5,4.5){};
\node[circle, scale=0.75, fill, task_scheduled] (tid4) at (0.75,6){};
\node[circle, scale=0.75, fill, task_scheduled] (tid5) at (2.25,6){};
\draw[](tid2) -- (tid4);
\draw[](tid2) -- (tid5);
\node[circle, scale=0.75, fill] (tid3) at (3.75,4.5){};
\node[circle, scale=0.75, fill] (tid6) at (3.75,6){};
\draw[](tid3) -- (tid6);
\draw[](tid1) -- (tid2);
\draw[](tid1) -- (tid3);
\draw[](tid0) -- (tid1);
\end{tikzpicture}
\nodepart{three}
\footnotesize{5.25}
\nodepart{four}
\footnotesize{$1$}
};
 & 
\node[draw=black, rectangle split,  rectangle split parts=4] (sn0xeff210){
\footnotesize{0.520833}
\nodepart{two}
\begin{tikzpicture}[scale=.2]
\node[circle, scale=0.75, fill] (tid0) at (2.25,1.5){};
\node[circle, scale=0.75, fill] (tid1) at (2.25,3){};
\node[circle, scale=0.75, fill] (tid2) at (1.5,4.5){};
\node[circle, scale=0.75, fill, task_scheduled] (tid4) at (0.75,6){};
\node[circle, scale=0.75, fill] (tid5) at (2.25,6){};
\draw[](tid2) -- (tid4);
\draw[](tid2) -- (tid5);
\node[circle, scale=0.75, fill] (tid3) at (3.75,4.5){};
\node[circle, scale=0.75, fill, task_scheduled] (tid6) at (3.75,6){};
\draw[](tid3) -- (tid6);
\draw[](tid1) -- (tid2);
\draw[](tid1) -- (tid3);
\draw[](tid0) -- (tid1);
\end{tikzpicture}
\nodepart{three}
\footnotesize{5.28125}
\nodepart{four}
\footnotesize{$25\:25\:50$}
};
 & 
\node[draw=black, rectangle split,  rectangle split parts=4] (sn0xf026b0){
\footnotesize{0.462963}
\nodepart{two}
\begin{tikzpicture}[scale=.2]
\node[circle, scale=0.75, fill] (tid0) at (3,1.5){};
\node[circle, scale=0.75, fill] (tid1) at (3,3){};
\node[circle, scale=0.75, fill] (tid2) at (1.5,4.5){};
\node[circle, scale=0.75, fill, task_scheduled] (tid5) at (0.75,6){};
\node[circle, scale=0.75, fill] (tid6) at (2.25,6){};
\draw[](tid2) -- (tid5);
\draw[](tid2) -- (tid6);
\node[circle, scale=0.75, fill, task_scheduled] (tid3) at (3.75,4.5){};
\node[circle, scale=0.75, fill] (tid4) at (5.25,4.5){};
\draw[](tid1) -- (tid2);
\draw[](tid1) -- (tid3);
\draw[](tid1) -- (tid4);
\draw[](tid0) -- (tid1);
\end{tikzpicture}
\nodepart{three}
\footnotesize{5.25}
\nodepart{four}
\footnotesize{$25\:25\:25\:25$}
};
 & 
\node[draw=black, rectangle split,  rectangle split parts=4] (sn0xf02bb0){
\footnotesize{0.231482}
\nodepart{two}
\begin{tikzpicture}[scale=.2]
\node[circle, scale=0.75, fill] (tid0) at (3,1.5){};
\node[circle, scale=0.75, fill] (tid1) at (3,3){};
\node[circle, scale=0.75, fill] (tid2) at (1.5,4.5){};
\node[circle, scale=0.75, fill, task_scheduled] (tid5) at (0.75,6){};
\node[circle, scale=0.75, fill, task_scheduled] (tid6) at (2.25,6){};
\draw[](tid2) -- (tid5);
\draw[](tid2) -- (tid6);
\node[circle, scale=0.75, fill] (tid3) at (3.75,4.5){};
\node[circle, scale=0.75, fill] (tid4) at (5.25,4.5){};
\draw[](tid1) -- (tid2);
\draw[](tid1) -- (tid3);
\draw[](tid1) -- (tid4);
\draw[](tid0) -- (tid1);
\end{tikzpicture}
\nodepart{three}
\footnotesize{5.125}
\nodepart{four}
\footnotesize{$1$}
};
 & 
\node[draw=black, rectangle split,  rectangle split parts=4] (sn0xef9e10){
\footnotesize{25.7812}
\nodepart{two}
\begin{tikzpicture}[scale=.2]
\node[circle, scale=0.75, fill] (tid0) at (2.25,1.5){};
\node[circle, scale=0.75, fill] (tid1) at (2.25,3){};
\node[circle, scale=0.75, fill] (tid2) at (0.75,4.5){};
\node[circle, scale=0.75, fill, task_scheduled] (tid5) at (0.75,6){};
\draw[](tid2) -- (tid5);
\node[circle, scale=0.75, fill] (tid3) at (2.25,4.5){};
\node[circle, scale=0.75, fill] (tid6) at (2.25,6){};
\draw[](tid3) -- (tid6);
\node[circle, scale=0.75, fill, task_scheduled] (tid4) at (3.75,4.5){};
\draw[](tid1) -- (tid2);
\draw[](tid1) -- (tid3);
\draw[](tid1) -- (tid4);
\draw[](tid0) -- (tid1);
\end{tikzpicture}
\nodepart{three}
\footnotesize{5.21875}
\nodepart{four}
\footnotesize{$50\:25\:25$}
};
 & 
\node[draw=black, rectangle split,  rectangle split parts=4] (sn0xefa190){
\footnotesize{18.6632}
\nodepart{two}
\begin{tikzpicture}[scale=.2]
\node[circle, scale=0.75, fill] (tid0) at (2.25,1.5){};
\node[circle, scale=0.75, fill] (tid1) at (2.25,3){};
\node[circle, scale=0.75, fill] (tid2) at (0.75,4.5){};
\node[circle, scale=0.75, fill, task_scheduled] (tid5) at (0.75,6){};
\draw[](tid2) -- (tid5);
\node[circle, scale=0.75, fill] (tid3) at (2.25,4.5){};
\node[circle, scale=0.75, fill, task_scheduled] (tid6) at (2.25,6){};
\draw[](tid3) -- (tid6);
\node[circle, scale=0.75, fill] (tid4) at (3.75,4.5){};
\draw[](tid1) -- (tid2);
\draw[](tid1) -- (tid3);
\draw[](tid1) -- (tid4);
\draw[](tid0) -- (tid1);
\end{tikzpicture}
\nodepart{three}
\footnotesize{5.125}
\nodepart{four}
\footnotesize{$1$}
};
 & 
\\
};
\end{scope}
\begin{scope}[yshift=\leveltopIIIIII cm]
\matrix (line6)[column sep=0.5cm] {
\node[draw=black, rectangle split,  rectangle split parts=4] (sn0xef5310){
\footnotesize{3.125}
\nodepart{two}
\begin{tikzpicture}[scale=.2]
\node[circle, scale=0.75, fill] (tid0) at (0.75,1.5){};
\node[circle, scale=0.75, fill] (tid1) at (0.75,3){};
\node[circle, scale=0.75, fill] (tid2) at (0.75,4.5){};
\node[circle, scale=0.75, fill] (tid3) at (0.75,6){};
\node[circle, scale=0.75, fill] (tid4) at (0.75,7.5){};
\node[circle, scale=0.75, fill, task_scheduled] (tid5) at (0.75,9){};
\draw[](tid4) -- (tid5);
\draw[](tid3) -- (tid4);
\draw[](tid2) -- (tid3);
\draw[](tid1) -- (tid2);
\draw[](tid0) -- (tid1);
\end{tikzpicture}
\nodepart{three}
\footnotesize{6}
\nodepart{four}
\footnotesize{$1$}
};
 & 
\node[draw=black, rectangle split,  rectangle split parts=4] (sn0xef5750){
\footnotesize{30.2951}
\nodepart{two}
\begin{tikzpicture}[scale=.2]
\node[circle, scale=0.75, fill] (tid0) at (1.5,1.5){};
\node[circle, scale=0.75, fill] (tid1) at (1.5,3){};
\node[circle, scale=0.75, fill] (tid2) at (0.75,4.5){};
\node[circle, scale=0.75, fill] (tid4) at (0.75,6){};
\node[circle, scale=0.75, fill, task_scheduled] (tid5) at (0.75,7.5){};
\draw[](tid4) -- (tid5);
\draw[](tid2) -- (tid4);
\node[circle, scale=0.75, fill, task_scheduled] (tid3) at (2.25,4.5){};
\draw[](tid1) -- (tid2);
\draw[](tid1) -- (tid3);
\draw[](tid0) -- (tid1);
\end{tikzpicture}
\nodepart{three}
\footnotesize{5.125}
\nodepart{four}
\footnotesize{$50\:50$}
};
 & 
\node[draw=black, rectangle split,  rectangle split parts=4] (sn0xeff630){
\footnotesize{0.245949}
\nodepart{two}
\begin{tikzpicture}[scale=.2]
\node[circle, scale=0.75, fill] (tid0) at (2.25,1.5){};
\node[circle, scale=0.75, fill] (tid1) at (2.25,3){};
\node[circle, scale=0.75, fill] (tid2) at (1.5,4.5){};
\node[circle, scale=0.75, fill, task_scheduled] (tid4) at (0.75,6){};
\node[circle, scale=0.75, fill] (tid5) at (2.25,6){};
\draw[](tid2) -- (tid4);
\draw[](tid2) -- (tid5);
\node[circle, scale=0.75, fill, task_scheduled] (tid3) at (3.75,4.5){};
\draw[](tid1) -- (tid2);
\draw[](tid1) -- (tid3);
\draw[](tid0) -- (tid1);
\end{tikzpicture}
\nodepart{three}
\footnotesize{4.875}
\nodepart{four}
\footnotesize{$50\:50$}
};
 & 
\node[draw=black, rectangle split,  rectangle split parts=4] (sn0xeff8f0){
\footnotesize{0.245949}
\nodepart{two}
\begin{tikzpicture}[scale=.2]
\node[circle, scale=0.75, fill] (tid0) at (2.25,1.5){};
\node[circle, scale=0.75, fill] (tid1) at (2.25,3){};
\node[circle, scale=0.75, fill] (tid2) at (1.5,4.5){};
\node[circle, scale=0.75, fill, task_scheduled] (tid4) at (0.75,6){};
\node[circle, scale=0.75, fill, task_scheduled] (tid5) at (2.25,6){};
\draw[](tid2) -- (tid4);
\draw[](tid2) -- (tid5);
\node[circle, scale=0.75, fill] (tid3) at (3.75,4.5){};
\draw[](tid1) -- (tid2);
\draw[](tid1) -- (tid3);
\draw[](tid0) -- (tid1);
\end{tikzpicture}
\nodepart{three}
\footnotesize{4.75}
\nodepart{four}
\footnotesize{$1$}
};
 & 
\node[draw=black, rectangle split,  rectangle split parts=4] (sn0xef7040){
\footnotesize{24.1753}
\nodepart{two}
\begin{tikzpicture}[scale=.2]
\node[circle, scale=0.75, fill] (tid0) at (1.5,1.5){};
\node[circle, scale=0.75, fill] (tid1) at (1.5,3){};
\node[circle, scale=0.75, fill] (tid2) at (0.75,4.5){};
\node[circle, scale=0.75, fill, task_scheduled] (tid4) at (0.75,6){};
\draw[](tid2) -- (tid4);
\node[circle, scale=0.75, fill] (tid3) at (2.25,4.5){};
\node[circle, scale=0.75, fill, task_scheduled] (tid5) at (2.25,6){};
\draw[](tid3) -- (tid5);
\draw[](tid1) -- (tid2);
\draw[](tid1) -- (tid3);
\draw[](tid0) -- (tid1);
\end{tikzpicture}
\nodepart{three}
\footnotesize{4.75}
\nodepart{four}
\footnotesize{$1$}
};
 & 
\node[draw=black, rectangle split,  rectangle split parts=4] (sn0xef7d10){
\footnotesize{11.509}
\nodepart{two}
\begin{tikzpicture}[scale=.2]
\node[circle, scale=0.75, fill] (tid0) at (2.25,1.5){};
\node[circle, scale=0.75, fill] (tid1) at (2.25,3){};
\node[circle, scale=0.75, fill] (tid2) at (0.75,4.5){};
\node[circle, scale=0.75, fill] (tid5) at (0.75,6){};
\draw[](tid2) -- (tid5);
\node[circle, scale=0.75, fill, task_scheduled] (tid3) at (2.25,4.5){};
\node[circle, scale=0.75, fill, task_scheduled] (tid4) at (3.75,4.5){};
\draw[](tid1) -- (tid2);
\draw[](tid1) -- (tid3);
\draw[](tid1) -- (tid4);
\draw[](tid0) -- (tid1);
\end{tikzpicture}
\nodepart{three}
\footnotesize{4.75}
\nodepart{four}
\footnotesize{$1$}
};
 & 
\node[draw=black, rectangle split,  rectangle split parts=4] (sn0xef8590){
\footnotesize{30.4036}
\nodepart{two}
\begin{tikzpicture}[scale=.2]
\node[circle, scale=0.75, fill] (tid0) at (2.25,1.5){};
\node[circle, scale=0.75, fill] (tid1) at (2.25,3){};
\node[circle, scale=0.75, fill] (tid2) at (0.75,4.5){};
\node[circle, scale=0.75, fill, task_scheduled] (tid5) at (0.75,6){};
\draw[](tid2) -- (tid5);
\node[circle, scale=0.75, fill, task_scheduled] (tid3) at (2.25,4.5){};
\node[circle, scale=0.75, fill] (tid4) at (3.75,4.5){};
\draw[](tid1) -- (tid2);
\draw[](tid1) -- (tid3);
\draw[](tid1) -- (tid4);
\draw[](tid0) -- (tid1);
\end{tikzpicture}
\nodepart{three}
\footnotesize{4.625}
\nodepart{four}
\footnotesize{$50\:50$}
};
 & 
\\
};
\end{scope}
\begin{scope}[yshift=\leveltopIIIIIII cm]
\matrix (line7)[column sep=0.5cm] {
\node[draw=black, rectangle split,  rectangle split parts=4] (sn0xef5850){
\footnotesize{18.2726}
\nodepart{two}
\begin{tikzpicture}[scale=.2]
\node[circle, scale=0.75, fill] (tid0) at (0.75,1.5){};
\node[circle, scale=0.75, fill] (tid1) at (0.75,3){};
\node[circle, scale=0.75, fill] (tid2) at (0.75,4.5){};
\node[circle, scale=0.75, fill] (tid3) at (0.75,6){};
\node[circle, scale=0.75, fill, task_scheduled] (tid4) at (0.75,7.5){};
\draw[](tid3) -- (tid4);
\draw[](tid2) -- (tid3);
\draw[](tid1) -- (tid2);
\draw[](tid0) -- (tid1);
\end{tikzpicture}
\nodepart{three}
\footnotesize{5}
\nodepart{four}
\footnotesize{$1$}
};
 & 
\node[draw=black, rectangle split,  rectangle split parts=4] (sn0xeffc80){
\footnotesize{0.122975}
\nodepart{two}
\begin{tikzpicture}[scale=.2]
\node[circle, scale=0.75, fill] (tid0) at (1.5,1.5){};
\node[circle, scale=0.75, fill] (tid1) at (1.5,3){};
\node[circle, scale=0.75, fill] (tid2) at (1.5,4.5){};
\node[circle, scale=0.75, fill, task_scheduled] (tid3) at (0.75,6){};
\node[circle, scale=0.75, fill, task_scheduled] (tid4) at (2.25,6){};
\draw[](tid2) -- (tid3);
\draw[](tid2) -- (tid4);
\draw[](tid1) -- (tid2);
\draw[](tid0) -- (tid1);
\end{tikzpicture}
\nodepart{three}
\footnotesize{4.5}
\nodepart{four}
\footnotesize{$1$}
};
 & 
\node[draw=black, rectangle split,  rectangle split parts=4] (sn0xef64d0){
\footnotesize{66.4026}
\nodepart{two}
\begin{tikzpicture}[scale=.2]
\node[circle, scale=0.75, fill] (tid0) at (1.5,1.5){};
\node[circle, scale=0.75, fill] (tid1) at (1.5,3){};
\node[circle, scale=0.75, fill] (tid2) at (0.75,4.5){};
\node[circle, scale=0.75, fill, task_scheduled] (tid4) at (0.75,6){};
\draw[](tid2) -- (tid4);
\node[circle, scale=0.75, fill, task_scheduled] (tid3) at (2.25,4.5){};
\draw[](tid1) -- (tid2);
\draw[](tid1) -- (tid3);
\draw[](tid0) -- (tid1);
\end{tikzpicture}
\nodepart{three}
\footnotesize{4.25}
\nodepart{four}
\footnotesize{$50\:50$}
};
 & 
\node[draw=black, rectangle split,  rectangle split parts=4] (sn0xef8750){
\footnotesize{15.2018}
\nodepart{two}
\begin{tikzpicture}[scale=.2]
\node[circle, scale=0.75, fill] (tid0) at (2.25,1.5){};
\node[circle, scale=0.75, fill] (tid1) at (2.25,3){};
\node[circle, scale=0.75, fill, task_scheduled] (tid2) at (0.75,4.5){};
\node[circle, scale=0.75, fill, task_scheduled] (tid3) at (2.25,4.5){};
\node[circle, scale=0.75, fill] (tid4) at (3.75,4.5){};
\draw[](tid1) -- (tid2);
\draw[](tid1) -- (tid3);
\draw[](tid1) -- (tid4);
\draw[](tid0) -- (tid1);
\end{tikzpicture}
\nodepart{three}
\footnotesize{4}
\nodepart{four}
\footnotesize{$1$}
};
 & 
\\
};
\end{scope}
\begin{scope}[yshift=\leveltopIIIIIIII cm]
\matrix (line8)[column sep=0.5cm] {
\node[draw=black, rectangle split,  rectangle split parts=4] (sn0xef5c90){
\footnotesize{51.5969}
\nodepart{two}
\begin{tikzpicture}[scale=.2]
\node[circle, scale=0.75, fill] (tid0) at (0.75,1.5){};
\node[circle, scale=0.75, fill] (tid1) at (0.75,3){};
\node[circle, scale=0.75, fill] (tid2) at (0.75,4.5){};
\node[circle, scale=0.75, fill, task_scheduled] (tid3) at (0.75,6){};
\draw[](tid2) -- (tid3);
\draw[](tid1) -- (tid2);
\draw[](tid0) -- (tid1);
\end{tikzpicture}
\nodepart{three}
\footnotesize{4}
\nodepart{four}
\footnotesize{$1$}
};
 & 
\node[draw=black, rectangle split,  rectangle split parts=4] (sn0xef6940){
\footnotesize{48.4031}
\nodepart{two}
\begin{tikzpicture}[scale=.2]
\node[circle, scale=0.75, fill] (tid0) at (1.5,1.5){};
\node[circle, scale=0.75, fill] (tid1) at (1.5,3){};
\node[circle, scale=0.75, fill, task_scheduled] (tid2) at (0.75,4.5){};
\node[circle, scale=0.75, fill, task_scheduled] (tid3) at (2.25,4.5){};
\draw[](tid1) -- (tid2);
\draw[](tid1) -- (tid3);
\draw[](tid0) -- (tid1);
\end{tikzpicture}
\nodepart{three}
\footnotesize{3.5}
\nodepart{four}
\footnotesize{$1$}
};
 & 
\\
};
\end{scope}
\begin{scope}[yshift=\leveltopIIIIIIIII cm]
\matrix (line9)[column sep=0.5cm] {
\node[draw=black, rectangle split,  rectangle split parts=4] (sn0xef5950){
\footnotesize{100}
\nodepart{two}
\begin{tikzpicture}[scale=.2]
\node[circle, scale=0.75, fill] (tid0) at (0.75,1.5){};
\node[circle, scale=0.75, fill] (tid1) at (0.75,3){};
\node[circle, scale=0.75, fill, task_scheduled] (tid2) at (0.75,4.5){};
\draw[](tid1) -- (tid2);
\draw[](tid0) -- (tid1);
\end{tikzpicture}
\nodepart{three}
\footnotesize{3}
\nodepart{four}
\footnotesize{$1$}
};
 & 
\\
};
\end{scope}
\begin{scope}[yshift=\leveltopIIIIIIIIII cm]
\matrix (line10)[column sep=0.5cm] {
\node[draw=black, rectangle split,  rectangle split parts=4] (sn0xef5b30){
\footnotesize{100}
\nodepart{two}
\begin{tikzpicture}[scale=.2]
\node[circle, scale=0.75, fill] (tid0) at (0.75,1.5){};
\node[circle, scale=0.75, fill, task_scheduled] (tid1) at (0.75,3){};
\draw[](tid0) -- (tid1);
\end{tikzpicture}
\nodepart{three}
\footnotesize{2}
\nodepart{four}
\footnotesize{$1$}
};
 & 
\\
};
\end{scope}
\draw (sn0xef1ca0.south) -- (sn0xef2540.north);
\draw (sn0xef1ca0.south) -- (sn0xf02cb0.north);
\draw (sn0xef1ca0.south) -- (sn0xf03560.north);
\draw (sn0xef1ca0.south) -- (sn0xf038a0.north);
\draw (sn0xef2540.south) -- (sn0xefa920.north);
\draw (sn0xef2540.south) -- (sn0xef3e70.north);
\draw (sn0xef2540.south) -- (sn0xefabd0.north);
\draw (sn0xef2540.south) -- (sn0xefb500.north);
\draw (sn0xf02cb0.south) -- (sn0xefabd0.north);
\draw (sn0xf02cb0.south) -- (sn0xefb500.north);
\draw (sn0xf03560.south) -- (sn0xefabd0.north);
\draw (sn0xf03560.south) -- (sn0xefb500.north);
\draw (sn0xf03560.south) -- (sn0xeffff0.north);
\draw (sn0xf03560.south) -- (sn0xf00d20.north);
\draw (sn0xf03560.south) -- (sn0xf01470.north);
\draw (sn0xf038a0.south) -- (sn0xefb500.north);
\draw (sn0xf038a0.south) -- (sn0xf04590.north);
\draw (sn0xf038a0.south) -- (sn0xf04f90.north);
\draw (sn0xefa920.south) -- (sn0xef43a0.north);
\draw (sn0xefa920.south) -- (sn0xef8300.north);
\draw (sn0xefa920.south) -- (sn0xefbc60.north);
\draw (sn0xef3e70.south) -- (sn0xef4790.north);
\draw (sn0xef3e70.south) -- (sn0xef8300.north);
\draw (sn0xef3e70.south) -- (sn0xef90f0.north);
\draw (sn0xefabd0.south) -- (sn0xef8300.north);
\draw (sn0xefabd0.south) -- (sn0xef90f0.north);
\draw (sn0xefb500.south) -- (sn0xefbc60.north);
\draw (sn0xefb500.south) -- (sn0xef90f0.north);
\draw (sn0xefb500.south) -- (sn0xefc0b0.north);
\draw (sn0xeffff0.south) -- (sn0xefe720.north);
\draw (sn0xeffff0.south) -- (sn0xefea20.north);
\draw (sn0xeffff0.south) -- (sn0xef8300.north);
\draw (sn0xeffff0.south) -- (sn0xefbc60.north);
\draw (sn0xf00d20.south) -- (sn0xef8300.north);
\draw (sn0xf00d20.south) -- (sn0xef90f0.north);
\draw (sn0xf01470.south) -- (sn0xefbc60.north);
\draw (sn0xf01470.south) -- (sn0xef90f0.north);
\draw (sn0xf01470.south) -- (sn0xf01940.north);
\draw (sn0xf01470.south) -- (sn0xf01e70.north);
\draw (sn0xf01470.south) -- (sn0xf02290.north);
\draw (sn0xf04590.south) -- (sn0xefc0b0.north);
\draw (sn0xf04f90.south) -- (sn0xefc0b0.north);
\draw (sn0xf04f90.south) -- (sn0xf01940.north);
\draw (sn0xf04f90.south) -- (sn0xf01e70.north);
\draw (sn0xf04f90.south) -- (sn0xf02290.north);
\draw (sn0xef43a0.south) -- (sn0xef48f0.north);
\draw (sn0xef43a0.south) -- (sn0xef5110.north);
\draw (sn0xef4790.south) -- (sn0xef48f0.north);
\draw (sn0xef4790.south) -- (sn0xef7880.north);
\draw (sn0xef4790.south) -- (sn0xef7c10.north);
\draw (sn0xef8300.south) -- (sn0xef5110.north);
\draw (sn0xef8300.south) -- (sn0xef7880.north);
\draw (sn0xef8300.south) -- (sn0xef7c10.north);
\draw (sn0xefbc60.south) -- (sn0xef5110.north);
\draw (sn0xefbc60.south) -- (sn0xef9e10.north);
\draw (sn0xef90f0.south) -- (sn0xef7c10.north);
\draw (sn0xef90f0.south) -- (sn0xef9e10.north);
\draw (sn0xef90f0.south) -- (sn0xefa190.north);
\draw (sn0xefe720.south) -- (sn0xef5110.north);
\draw (sn0xefea20.south) -- (sn0xef5110.north);
\draw (sn0xefea20.south) -- (sn0xefeea0.north);
\draw (sn0xefea20.south) -- (sn0xeff210.north);
\draw (sn0xf01940.south) -- (sn0xefeea0.north);
\draw (sn0xf01940.south) -- (sn0xeff210.north);
\draw (sn0xf01940.south) -- (sn0xef9e10.north);
\draw (sn0xf01e70.south) -- (sn0xef9e10.north);
\draw (sn0xf01e70.south) -- (sn0xefa190.north);
\draw (sn0xf02290.south) -- (sn0xef9e10.north);
\draw (sn0xf02290.south) -- (sn0xefa190.north);
\draw (sn0xf02290.south) -- (sn0xf026b0.north);
\draw (sn0xf02290.south) -- (sn0xf02bb0.north);
\draw (sn0xefc0b0.south) -- (sn0xef9e10.north);
\draw (sn0xefc0b0.south) -- (sn0xefa190.north);
\draw (sn0xef48f0.south) -- (sn0xef5310.north);
\draw (sn0xef48f0.south) -- (sn0xef5750.north);
\draw (sn0xef5110.south) -- (sn0xef5750.north);
\draw (sn0xef5110.south) -- (sn0xef7040.north);
\draw (sn0xef7880.south) -- (sn0xef5750.north);
\draw (sn0xef7c10.south) -- (sn0xef5750.north);
\draw (sn0xef7c10.south) -- (sn0xef7d10.north);
\draw (sn0xef7c10.south) -- (sn0xef8590.north);
\draw (sn0xefeea0.south) -- (sn0xef7040.north);
\draw (sn0xeff210.south) -- (sn0xef7040.north);
\draw (sn0xeff210.south) -- (sn0xeff630.north);
\draw (sn0xeff210.south) -- (sn0xeff8f0.north);
\draw (sn0xf026b0.south) -- (sn0xeff630.north);
\draw (sn0xf026b0.south) -- (sn0xeff8f0.north);
\draw (sn0xf026b0.south) -- (sn0xef7d10.north);
\draw (sn0xf026b0.south) -- (sn0xef8590.north);
\draw (sn0xf02bb0.south) -- (sn0xef8590.north);
\draw (sn0xef9e10.south) -- (sn0xef7040.north);
\draw (sn0xef9e10.south) -- (sn0xef7d10.north);
\draw (sn0xef9e10.south) -- (sn0xef8590.north);
\draw (sn0xefa190.south) -- (sn0xef8590.north);
\draw (sn0xef5310.south) -- (sn0xef5850.north);
\draw (sn0xef5750.south) -- (sn0xef5850.north);
\draw (sn0xef5750.south) -- (sn0xef64d0.north);
\draw (sn0xeff630.south) -- (sn0xeffc80.north);
\draw (sn0xeff630.south) -- (sn0xef64d0.north);
\draw (sn0xeff8f0.south) -- (sn0xef64d0.north);
\draw (sn0xef7040.south) -- (sn0xef64d0.north);
\draw (sn0xef7d10.south) -- (sn0xef64d0.north);
\draw (sn0xef8590.south) -- (sn0xef64d0.north);
\draw (sn0xef8590.south) -- (sn0xef8750.north);
\draw (sn0xef5850.south) -- (sn0xef5c90.north);
\draw (sn0xeffc80.south) -- (sn0xef5c90.north);
\draw (sn0xef64d0.south) -- (sn0xef5c90.north);
\draw (sn0xef64d0.south) -- (sn0xef6940.north);
\draw (sn0xef8750.south) -- (sn0xef6940.north);
\draw (sn0xef5c90.south) -- (sn0xef5950.north);
\draw (sn0xef6940.south) -- (sn0xef5950.north);
\draw (sn0xef5950.south) -- (sn0xef5b30.north);
\end{tikzpicture}

%%% Local Variables:
%%% TeX-master: "thesis/thesis.tex"
%%% End: 
\renewcommand{\leveltopI}{-15cm + \leveltop}
\renewcommand{\leveltopII}{-15cm + \leveltopI}
\renewcommand{\leveltopIII}{-15cm + \leveltopII}
\renewcommand{\leveltopIIII}{-15cm + \leveltopIII}
\renewcommand{\leveltopIIIII}{-15cm + \leveltopIIII}
\renewcommand{\leveltopIIIIII}{-15cm + \leveltopIIIII}
\renewcommand{\leveltopIIIIIII}{-15cm + \leveltopIIIIII}
\renewcommand{\leveltopIIIIIIII}{-15cm + \leveltopIIIIIII}
\renewcommand{\leveltopIIIIIIIII}{-15cm + \leveltopIIIIIIII}
\renewcommand{\leveltopIIIIIIIIII}{-15cm + \leveltopIIIIIIIII}
\renewcommand{\leveltopIIIIIIIIIII}{-15cm + \leveltopIIIIIIIIII}
\begin{tikzpicture}[scale=.2, anchor=south]
\begin{scope}[yshift=\leveltopI cm]
\matrix (line1)[column sep=0.5cm] {
\node[draw=black, rectangle split,  rectangle split parts=4] (sn0xef2440){
\footnotesize{100}
\nodepart{two}
\begin{tikzpicture}[scale=.2]
\node[circle, scale=0.75, fill] (tid0) at (3,1.5){};
\node[circle, scale=0.75, fill] (tid1) at (3,3){};
\node[circle, scale=0.75, fill] (tid2) at (0.75,4.5){};
\node[circle, scale=0.75, fill] (tid5) at (0.75,6){};
\node[circle, scale=0.75, fill] (tid9) at (0.75,7.5){};
\node[circle, scale=0.75, fill, task_scheduled] (tid10) at (0.75,9){};
\draw[](tid9) -- (tid10);
\draw[](tid5) -- (tid9);
\draw[](tid2) -- (tid5);
\node[circle, scale=0.75, fill] (tid3) at (3,4.5){};
\node[circle, scale=0.75, fill] (tid6) at (2.25,6){};
\node[circle, scale=0.75, fill] (tid7) at (3.75,6){};
\draw[](tid3) -- (tid6);
\draw[](tid3) -- (tid7);
\node[circle, scale=0.75, fill] (tid4) at (5.25,4.5){};
\node[circle, scale=0.75, fill, task_scheduled] (tid8) at (5.25,6){};
\draw[](tid4) -- (tid8);
\draw[](tid1) -- (tid2);
\draw[](tid1) -- (tid3);
\draw[](tid1) -- (tid4);
\draw[](tid0) -- (tid1);
\end{tikzpicture}
\nodepart{three}
\footnotesize{7.38134}
\nodepart{four}
\footnotesize{$17\:33\:33\:17$}
};
 & 
\\
};
\end{scope}
\begin{scope}[yshift=\leveltopII cm]
\matrix (line2)[column sep=0.5cm] {
\node[draw=black, rectangle split,  rectangle split parts=4] (sn0xf051d0){
\footnotesize{16.6667}
\nodepart{two}
\begin{tikzpicture}[scale=.2]
\node[circle, scale=0.75, fill] (tid0) at (3,1.5){};
\node[circle, scale=0.75, fill] (tid1) at (3,3){};
\node[circle, scale=0.75, fill] (tid2) at (0.75,4.5){};
\node[circle, scale=0.75, fill] (tid5) at (0.75,6){};
\node[circle, scale=0.75, fill] (tid8) at (0.75,7.5){};
\node[circle, scale=0.75, fill, task_scheduled] (tid9) at (0.75,9){};
\draw[](tid8) -- (tid9);
\draw[](tid5) -- (tid8);
\draw[](tid2) -- (tid5);
\node[circle, scale=0.75, fill] (tid3) at (3,4.5){};
\node[circle, scale=0.75, fill] (tid6) at (2.25,6){};
\node[circle, scale=0.75, fill] (tid7) at (3.75,6){};
\draw[](tid3) -- (tid6);
\draw[](tid3) -- (tid7);
\node[circle, scale=0.75, fill, task_scheduled] (tid4) at (5.25,4.5){};
\draw[](tid1) -- (tid2);
\draw[](tid1) -- (tid3);
\draw[](tid1) -- (tid4);
\draw[](tid0) -- (tid1);
\end{tikzpicture}
\nodepart{three}
\footnotesize{6.96965}
\nodepart{four}
\footnotesize{$50\:33\:17$}
};
 & 
\node[draw=black, rectangle split,  rectangle split parts=4] (sn0xefcc90){
\footnotesize{33.3333}
\nodepart{two}
\begin{tikzpicture}[scale=.2]
\node[circle, scale=0.75, fill] (tid0) at (3,1.5){};
\node[circle, scale=0.75, fill] (tid1) at (3,3){};
\node[circle, scale=0.75, fill] (tid2) at (0.75,4.5){};
\node[circle, scale=0.75, fill] (tid5) at (0.75,6){};
\node[circle, scale=0.75, fill] (tid8) at (0.75,7.5){};
\node[circle, scale=0.75, fill, task_scheduled] (tid9) at (0.75,9){};
\draw[](tid8) -- (tid9);
\draw[](tid5) -- (tid8);
\draw[](tid2) -- (tid5);
\node[circle, scale=0.75, fill] (tid3) at (3,4.5){};
\node[circle, scale=0.75, fill, task_scheduled] (tid6) at (2.25,6){};
\node[circle, scale=0.75, fill] (tid7) at (3.75,6){};
\draw[](tid3) -- (tid6);
\draw[](tid3) -- (tid7);
\node[circle, scale=0.75, fill] (tid4) at (5.25,4.5){};
\draw[](tid1) -- (tid2);
\draw[](tid1) -- (tid3);
\draw[](tid1) -- (tid4);
\draw[](tid0) -- (tid1);
\end{tikzpicture}
\nodepart{three}
\footnotesize{6.96323}
\nodepart{four}
\footnotesize{$25\:25\:17\:17\:17$}
};
 & 
\node[draw=black, rectangle split,  rectangle split parts=4] (sn0xf03560){
\footnotesize{33.3333}
\nodepart{two}
\begin{tikzpicture}[scale=.2]
\node[circle, scale=0.75, fill] (tid0) at (3,1.5){};
\node[circle, scale=0.75, fill] (tid1) at (3,3){};
\node[circle, scale=0.75, fill] (tid2) at (1.5,4.5){};
\node[circle, scale=0.75, fill, task_scheduled] (tid5) at (0.75,6){};
\node[circle, scale=0.75, fill] (tid6) at (2.25,6){};
\draw[](tid2) -- (tid5);
\draw[](tid2) -- (tid6);
\node[circle, scale=0.75, fill] (tid3) at (3.75,4.5){};
\node[circle, scale=0.75, fill] (tid7) at (3.75,6){};
\node[circle, scale=0.75, fill] (tid9) at (3.75,7.5){};
\draw[](tid7) -- (tid9);
\draw[](tid3) -- (tid7);
\node[circle, scale=0.75, fill] (tid4) at (5.25,4.5){};
\node[circle, scale=0.75, fill, task_scheduled] (tid8) at (5.25,6){};
\draw[](tid4) -- (tid8);
\draw[](tid1) -- (tid2);
\draw[](tid1) -- (tid3);
\draw[](tid1) -- (tid4);
\draw[](tid0) -- (tid1);
\end{tikzpicture}
\nodepart{three}
\footnotesize{6.82847}
\nodepart{four}
\footnotesize{$25\:25\:17\:17\:17$}
};
 & 
\node[draw=black, rectangle split,  rectangle split parts=4] (sn0xf05e10){
\footnotesize{16.6667}
\nodepart{two}
\begin{tikzpicture}[scale=.2]
\node[circle, scale=0.75, fill] (tid0) at (3,1.5){};
\node[circle, scale=0.75, fill] (tid1) at (3,3){};
\node[circle, scale=0.75, fill] (tid2) at (1.5,4.5){};
\node[circle, scale=0.75, fill] (tid5) at (0.75,6){};
\node[circle, scale=0.75, fill] (tid6) at (2.25,6){};
\draw[](tid2) -- (tid5);
\draw[](tid2) -- (tid6);
\node[circle, scale=0.75, fill] (tid3) at (3.75,4.5){};
\node[circle, scale=0.75, fill] (tid7) at (3.75,6){};
\node[circle, scale=0.75, fill, task_scheduled] (tid9) at (3.75,7.5){};
\draw[](tid7) -- (tid9);
\draw[](tid3) -- (tid7);
\node[circle, scale=0.75, fill] (tid4) at (5.25,4.5){};
\node[circle, scale=0.75, fill, task_scheduled] (tid8) at (5.25,6){};
\draw[](tid4) -- (tid8);
\draw[](tid1) -- (tid2);
\draw[](tid1) -- (tid3);
\draw[](tid1) -- (tid4);
\draw[](tid0) -- (tid1);
\end{tikzpicture}
\nodepart{three}
\footnotesize{6.73495}
\nodepart{four}
\footnotesize{$17\:33\:33\:17$}
};
 & 
\\
};
\end{scope}
\begin{scope}[yshift=\leveltopIII cm]
\matrix (line3)[column sep=0.5cm] {
\node[draw=black, rectangle split,  rectangle split parts=4] (sn0xefe2f0){
\footnotesize{8.33333}
\nodepart{two}
\begin{tikzpicture}[scale=.2]
\node[circle, scale=0.75, fill] (tid0) at (2.25,1.5){};
\node[circle, scale=0.75, fill] (tid1) at (2.25,3){};
\node[circle, scale=0.75, fill] (tid2) at (0.75,4.5){};
\node[circle, scale=0.75, fill] (tid4) at (0.75,6){};
\node[circle, scale=0.75, fill] (tid7) at (0.75,7.5){};
\node[circle, scale=0.75, fill, task_scheduled] (tid8) at (0.75,9){};
\draw[](tid7) -- (tid8);
\draw[](tid4) -- (tid7);
\draw[](tid2) -- (tid4);
\node[circle, scale=0.75, fill] (tid3) at (3,4.5){};
\node[circle, scale=0.75, fill, task_scheduled] (tid5) at (2.25,6){};
\node[circle, scale=0.75, fill] (tid6) at (3.75,6){};
\draw[](tid3) -- (tid5);
\draw[](tid3) -- (tid6);
\draw[](tid1) -- (tid2);
\draw[](tid1) -- (tid3);
\draw[](tid0) -- (tid1);
\end{tikzpicture}
\nodepart{three}
\footnotesize{6.57227}
\nodepart{four}
\footnotesize{$50\:25\:25$}
};
 & 
\node[draw=black, rectangle split,  rectangle split parts=4] (sn0xefa920){
\footnotesize{8.33333}
\nodepart{two}
\begin{tikzpicture}[scale=.2]
\node[circle, scale=0.75, fill] (tid0) at (2.25,1.5){};
\node[circle, scale=0.75, fill] (tid1) at (2.25,3){};
\node[circle, scale=0.75, fill] (tid2) at (0.75,4.5){};
\node[circle, scale=0.75, fill] (tid5) at (0.75,6){};
\node[circle, scale=0.75, fill] (tid7) at (0.75,7.5){};
\node[circle, scale=0.75, fill, task_scheduled] (tid8) at (0.75,9){};
\draw[](tid7) -- (tid8);
\draw[](tid5) -- (tid7);
\draw[](tid2) -- (tid5);
\node[circle, scale=0.75, fill] (tid3) at (2.25,4.5){};
\node[circle, scale=0.75, fill] (tid6) at (2.25,6){};
\draw[](tid3) -- (tid6);
\node[circle, scale=0.75, fill, task_scheduled] (tid4) at (3.75,4.5){};
\draw[](tid1) -- (tid2);
\draw[](tid1) -- (tid3);
\draw[](tid1) -- (tid4);
\draw[](tid0) -- (tid1);
\end{tikzpicture}
\nodepart{three}
\footnotesize{6.57617}
\nodepart{four}
\footnotesize{$50\:25\:25$}
};
 & 
\node[draw=black, rectangle split,  rectangle split parts=4] (sn0xef3e70){
\footnotesize{8.33333}
\nodepart{two}
\begin{tikzpicture}[scale=.2]
\node[circle, scale=0.75, fill] (tid0) at (2.25,1.5){};
\node[circle, scale=0.75, fill] (tid1) at (2.25,3){};
\node[circle, scale=0.75, fill] (tid2) at (0.75,4.5){};
\node[circle, scale=0.75, fill] (tid5) at (0.75,6){};
\node[circle, scale=0.75, fill] (tid7) at (0.75,7.5){};
\node[circle, scale=0.75, fill, task_scheduled] (tid8) at (0.75,9){};
\draw[](tid7) -- (tid8);
\draw[](tid5) -- (tid7);
\draw[](tid2) -- (tid5);
\node[circle, scale=0.75, fill] (tid3) at (2.25,4.5){};
\node[circle, scale=0.75, fill, task_scheduled] (tid6) at (2.25,6){};
\draw[](tid3) -- (tid6);
\node[circle, scale=0.75, fill] (tid4) at (3.75,4.5){};
\draw[](tid1) -- (tid2);
\draw[](tid1) -- (tid3);
\draw[](tid1) -- (tid4);
\draw[](tid0) -- (tid1);
\end{tikzpicture}
\nodepart{three}
\footnotesize{6.58594}
\nodepart{four}
\footnotesize{$50\:25\:25$}
};
 & 
\node[draw=black, rectangle split,  rectangle split parts=4] (sn0xefabd0){
\footnotesize{8.33333}
\nodepart{two}
\begin{tikzpicture}[scale=.2]
\node[circle, scale=0.75, fill] (tid0) at (2.25,1.5){};
\node[circle, scale=0.75, fill] (tid1) at (2.25,3){};
\node[circle, scale=0.75, fill] (tid2) at (0.75,4.5){};
\node[circle, scale=0.75, fill] (tid5) at (0.75,6){};
\node[circle, scale=0.75, fill] (tid8) at (0.75,7.5){};
\draw[](tid5) -- (tid8);
\draw[](tid2) -- (tid5);
\node[circle, scale=0.75, fill] (tid3) at (2.25,4.5){};
\node[circle, scale=0.75, fill, task_scheduled] (tid6) at (2.25,6){};
\draw[](tid3) -- (tid6);
\node[circle, scale=0.75, fill] (tid4) at (3.75,4.5){};
\node[circle, scale=0.75, fill, task_scheduled] (tid7) at (3.75,6){};
\draw[](tid4) -- (tid7);
\draw[](tid1) -- (tid2);
\draw[](tid1) -- (tid3);
\draw[](tid1) -- (tid4);
\draw[](tid0) -- (tid1);
\end{tikzpicture}
\nodepart{three}
\footnotesize{6.38281}
\nodepart{four}
\footnotesize{$50\:50$}
};
 & 
\node[draw=black, rectangle split,  rectangle split parts=4] (sn0xefb500){
\footnotesize{8.33333}
\nodepart{two}
\begin{tikzpicture}[scale=.2]
\node[circle, scale=0.75, fill] (tid0) at (2.25,1.5){};
\node[circle, scale=0.75, fill] (tid1) at (2.25,3){};
\node[circle, scale=0.75, fill] (tid2) at (0.75,4.5){};
\node[circle, scale=0.75, fill] (tid5) at (0.75,6){};
\node[circle, scale=0.75, fill, task_scheduled] (tid8) at (0.75,7.5){};
\draw[](tid5) -- (tid8);
\draw[](tid2) -- (tid5);
\node[circle, scale=0.75, fill] (tid3) at (2.25,4.5){};
\node[circle, scale=0.75, fill, task_scheduled] (tid6) at (2.25,6){};
\draw[](tid3) -- (tid6);
\node[circle, scale=0.75, fill] (tid4) at (3.75,4.5){};
\node[circle, scale=0.75, fill] (tid7) at (3.75,6){};
\draw[](tid4) -- (tid7);
\draw[](tid1) -- (tid2);
\draw[](tid1) -- (tid3);
\draw[](tid1) -- (tid4);
\draw[](tid0) -- (tid1);
\end{tikzpicture}
\nodepart{three}
\footnotesize{6.24023}
\nodepart{four}
\footnotesize{$25\:25\:50$}
};
 & 
\node[draw=black, rectangle split,  rectangle split parts=4] (sn0xeffff0){
\footnotesize{16.6667}
\nodepart{two}
\begin{tikzpicture}[scale=.2]
\node[circle, scale=0.75, fill] (tid0) at (3,1.5){};
\node[circle, scale=0.75, fill] (tid1) at (3,3){};
\node[circle, scale=0.75, fill] (tid2) at (1.5,4.5){};
\node[circle, scale=0.75, fill, task_scheduled] (tid5) at (0.75,6){};
\node[circle, scale=0.75, fill] (tid6) at (2.25,6){};
\draw[](tid2) -- (tid5);
\draw[](tid2) -- (tid6);
\node[circle, scale=0.75, fill] (tid3) at (3.75,4.5){};
\node[circle, scale=0.75, fill] (tid7) at (3.75,6){};
\node[circle, scale=0.75, fill] (tid8) at (3.75,7.5){};
\draw[](tid7) -- (tid8);
\draw[](tid3) -- (tid7);
\node[circle, scale=0.75, fill, task_scheduled] (tid4) at (5.25,4.5){};
\draw[](tid1) -- (tid2);
\draw[](tid1) -- (tid3);
\draw[](tid1) -- (tid4);
\draw[](tid0) -- (tid1);
\end{tikzpicture}
\nodepart{three}
\footnotesize{6.39844}
\nodepart{four}
\footnotesize{$25\:25\:25\:25$}
};
 & 
\node[draw=black, rectangle split,  rectangle split parts=4] (sn0xf05830){
\footnotesize{5.55556}
\nodepart{two}
\begin{tikzpicture}[scale=.2]
\node[circle, scale=0.75, fill] (tid0) at (3,1.5){};
\node[circle, scale=0.75, fill] (tid1) at (3,3){};
\node[circle, scale=0.75, fill] (tid2) at (1.5,4.5){};
\node[circle, scale=0.75, fill] (tid5) at (0.75,6){};
\node[circle, scale=0.75, fill] (tid6) at (2.25,6){};
\draw[](tid2) -- (tid5);
\draw[](tid2) -- (tid6);
\node[circle, scale=0.75, fill] (tid3) at (3.75,4.5){};
\node[circle, scale=0.75, fill] (tid7) at (3.75,6){};
\node[circle, scale=0.75, fill, task_scheduled] (tid8) at (3.75,7.5){};
\draw[](tid7) -- (tid8);
\draw[](tid3) -- (tid7);
\node[circle, scale=0.75, fill, task_scheduled] (tid4) at (5.25,4.5){};
\draw[](tid1) -- (tid2);
\draw[](tid1) -- (tid3);
\draw[](tid1) -- (tid4);
\draw[](tid0) -- (tid1);
\end{tikzpicture}
\nodepart{three}
\footnotesize{6.30425}
\nodepart{four}
\footnotesize{$50\:33\:17$}
};
 & 
\node[draw=black, rectangle split,  rectangle split parts=4] (sn0xf00d20){
\footnotesize{11.1111}
\nodepart{two}
\begin{tikzpicture}[scale=.2]
\node[circle, scale=0.75, fill] (tid0) at (3,1.5){};
\node[circle, scale=0.75, fill] (tid1) at (3,3){};
\node[circle, scale=0.75, fill] (tid2) at (1.5,4.5){};
\node[circle, scale=0.75, fill, task_scheduled] (tid5) at (0.75,6){};
\node[circle, scale=0.75, fill, task_scheduled] (tid6) at (2.25,6){};
\draw[](tid2) -- (tid5);
\draw[](tid2) -- (tid6);
\node[circle, scale=0.75, fill] (tid3) at (3.75,4.5){};
\node[circle, scale=0.75, fill] (tid7) at (3.75,6){};
\node[circle, scale=0.75, fill] (tid8) at (3.75,7.5){};
\draw[](tid7) -- (tid8);
\draw[](tid3) -- (tid7);
\node[circle, scale=0.75, fill] (tid4) at (5.25,4.5){};
\draw[](tid1) -- (tid2);
\draw[](tid1) -- (tid3);
\draw[](tid1) -- (tid4);
\draw[](tid0) -- (tid1);
\end{tikzpicture}
\nodepart{three}
\footnotesize{6.38281}
\nodepart{four}
\footnotesize{$50\:50$}
};
 & 
\node[draw=black, rectangle split,  rectangle split parts=4] (sn0xf01470){
\footnotesize{16.6667}
\nodepart{two}
\begin{tikzpicture}[scale=.2]
\node[circle, scale=0.75, fill] (tid0) at (3,1.5){};
\node[circle, scale=0.75, fill] (tid1) at (3,3){};
\node[circle, scale=0.75, fill] (tid2) at (1.5,4.5){};
\node[circle, scale=0.75, fill, task_scheduled] (tid5) at (0.75,6){};
\node[circle, scale=0.75, fill] (tid6) at (2.25,6){};
\draw[](tid2) -- (tid5);
\draw[](tid2) -- (tid6);
\node[circle, scale=0.75, fill] (tid3) at (3.75,4.5){};
\node[circle, scale=0.75, fill] (tid7) at (3.75,6){};
\node[circle, scale=0.75, fill, task_scheduled] (tid8) at (3.75,7.5){};
\draw[](tid7) -- (tid8);
\draw[](tid3) -- (tid7);
\node[circle, scale=0.75, fill] (tid4) at (5.25,4.5){};
\draw[](tid1) -- (tid2);
\draw[](tid1) -- (tid3);
\draw[](tid1) -- (tid4);
\draw[](tid0) -- (tid1);
\end{tikzpicture}
\nodepart{three}
\footnotesize{6.25499}
\nodepart{four}
\footnotesize{$25\:25\:17\:17\:17$}
};
 & 
\node[draw=black, rectangle split,  rectangle split parts=4] (sn0xf04f90){
\footnotesize{5.55556}
\nodepart{two}
\begin{tikzpicture}[scale=.2]
\node[circle, scale=0.75, fill] (tid0) at (3,1.5){};
\node[circle, scale=0.75, fill] (tid1) at (3,3){};
\node[circle, scale=0.75, fill] (tid2) at (1.5,4.5){};
\node[circle, scale=0.75, fill, task_scheduled] (tid5) at (0.75,6){};
\node[circle, scale=0.75, fill] (tid6) at (2.25,6){};
\draw[](tid2) -- (tid5);
\draw[](tid2) -- (tid6);
\node[circle, scale=0.75, fill] (tid3) at (3.75,4.5){};
\node[circle, scale=0.75, fill, task_scheduled] (tid7) at (3.75,6){};
\draw[](tid3) -- (tid7);
\node[circle, scale=0.75, fill] (tid4) at (5.25,4.5){};
\node[circle, scale=0.75, fill] (tid8) at (5.25,6){};
\draw[](tid4) -- (tid8);
\draw[](tid1) -- (tid2);
\draw[](tid1) -- (tid3);
\draw[](tid1) -- (tid4);
\draw[](tid0) -- (tid1);
\end{tikzpicture}
\nodepart{three}
\footnotesize{6.18663}
\nodepart{four}
\footnotesize{$17\:17\:17\:50$}
};
 & 
\node[draw=black, rectangle split,  rectangle split parts=4] (sn0xf06500){
\footnotesize{2.77778}
\nodepart{two}
\begin{tikzpicture}[scale=.2]
\node[circle, scale=0.75, fill] (tid0) at (3,1.5){};
\node[circle, scale=0.75, fill] (tid1) at (3,3){};
\node[circle, scale=0.75, fill] (tid2) at (1.5,4.5){};
\node[circle, scale=0.75, fill] (tid5) at (0.75,6){};
\node[circle, scale=0.75, fill] (tid6) at (2.25,6){};
\draw[](tid2) -- (tid5);
\draw[](tid2) -- (tid6);
\node[circle, scale=0.75, fill] (tid3) at (3.75,4.5){};
\node[circle, scale=0.75, fill, task_scheduled] (tid7) at (3.75,6){};
\draw[](tid3) -- (tid7);
\node[circle, scale=0.75, fill] (tid4) at (5.25,4.5){};
\node[circle, scale=0.75, fill, task_scheduled] (tid8) at (5.25,6){};
\draw[](tid4) -- (tid8);
\draw[](tid1) -- (tid2);
\draw[](tid1) -- (tid3);
\draw[](tid1) -- (tid4);
\draw[](tid0) -- (tid1);
\end{tikzpicture}
\nodepart{three}
\footnotesize{6.22222}
\nodepart{four}
\footnotesize{$33\:67$}
};
 & 
\\
};
\end{scope}
\begin{scope}[yshift=\leveltopIIII cm]
\matrix (line4)[column sep=0.5cm] {
\node[draw=black, rectangle split,  rectangle split parts=4] (sn0xef43a0){
\footnotesize{8.33333}
\nodepart{two}
\begin{tikzpicture}[scale=.2]
\node[circle, scale=0.75, fill] (tid0) at (1.5,1.5){};
\node[circle, scale=0.75, fill] (tid1) at (1.5,3){};
\node[circle, scale=0.75, fill] (tid2) at (0.75,4.5){};
\node[circle, scale=0.75, fill] (tid4) at (0.75,6){};
\node[circle, scale=0.75, fill] (tid6) at (0.75,7.5){};
\node[circle, scale=0.75, fill, task_scheduled] (tid7) at (0.75,9){};
\draw[](tid6) -- (tid7);
\draw[](tid4) -- (tid6);
\draw[](tid2) -- (tid4);
\node[circle, scale=0.75, fill] (tid3) at (2.25,4.5){};
\node[circle, scale=0.75, fill, task_scheduled] (tid5) at (2.25,6){};
\draw[](tid3) -- (tid5);
\draw[](tid1) -- (tid2);
\draw[](tid1) -- (tid3);
\draw[](tid0) -- (tid1);
\end{tikzpicture}
\nodepart{three}
\footnotesize{6.25}
\nodepart{four}
\footnotesize{$50\:50$}
};
 & 
\node[draw=black, rectangle split,  rectangle split parts=4] (sn0xef4790){
\footnotesize{4.16667}
\nodepart{two}
\begin{tikzpicture}[scale=.2]
\node[circle, scale=0.75, fill] (tid0) at (2.25,1.5){};
\node[circle, scale=0.75, fill] (tid1) at (2.25,3){};
\node[circle, scale=0.75, fill] (tid2) at (0.75,4.5){};
\node[circle, scale=0.75, fill] (tid5) at (0.75,6){};
\node[circle, scale=0.75, fill] (tid6) at (0.75,7.5){};
\node[circle, scale=0.75, fill, task_scheduled] (tid7) at (0.75,9){};
\draw[](tid6) -- (tid7);
\draw[](tid5) -- (tid6);
\draw[](tid2) -- (tid5);
\node[circle, scale=0.75, fill, task_scheduled] (tid3) at (2.25,4.5){};
\node[circle, scale=0.75, fill] (tid4) at (3.75,4.5){};
\draw[](tid1) -- (tid2);
\draw[](tid1) -- (tid3);
\draw[](tid1) -- (tid4);
\draw[](tid0) -- (tid1);
\end{tikzpicture}
\nodepart{three}
\footnotesize{6.28906}
\nodepart{four}
\footnotesize{$50\:25\:25$}
};
 & 
\node[draw=black, rectangle split,  rectangle split parts=4] (sn0xef8300){
\footnotesize{18.0556}
\nodepart{two}
\begin{tikzpicture}[scale=.2]
\node[circle, scale=0.75, fill] (tid0) at (2.25,1.5){};
\node[circle, scale=0.75, fill] (tid1) at (2.25,3){};
\node[circle, scale=0.75, fill] (tid2) at (0.75,4.5){};
\node[circle, scale=0.75, fill] (tid5) at (0.75,6){};
\node[circle, scale=0.75, fill] (tid7) at (0.75,7.5){};
\draw[](tid5) -- (tid7);
\draw[](tid2) -- (tid5);
\node[circle, scale=0.75, fill] (tid3) at (2.25,4.5){};
\node[circle, scale=0.75, fill, task_scheduled] (tid6) at (2.25,6){};
\draw[](tid3) -- (tid6);
\node[circle, scale=0.75, fill, task_scheduled] (tid4) at (3.75,4.5){};
\draw[](tid1) -- (tid2);
\draw[](tid1) -- (tid3);
\draw[](tid1) -- (tid4);
\draw[](tid0) -- (tid1);
\end{tikzpicture}
\nodepart{three}
\footnotesize{5.97656}
\nodepart{four}
\footnotesize{$50\:25\:25$}
};
 & 
\node[draw=black, rectangle split,  rectangle split parts=4] (sn0xefbc60){
\footnotesize{12.5}
\nodepart{two}
\begin{tikzpicture}[scale=.2]
\node[circle, scale=0.75, fill] (tid0) at (2.25,1.5){};
\node[circle, scale=0.75, fill] (tid1) at (2.25,3){};
\node[circle, scale=0.75, fill] (tid2) at (0.75,4.5){};
\node[circle, scale=0.75, fill] (tid5) at (0.75,6){};
\node[circle, scale=0.75, fill, task_scheduled] (tid7) at (0.75,7.5){};
\draw[](tid5) -- (tid7);
\draw[](tid2) -- (tid5);
\node[circle, scale=0.75, fill] (tid3) at (2.25,4.5){};
\node[circle, scale=0.75, fill] (tid6) at (2.25,6){};
\draw[](tid3) -- (tid6);
\node[circle, scale=0.75, fill, task_scheduled] (tid4) at (3.75,4.5){};
\draw[](tid1) -- (tid2);
\draw[](tid1) -- (tid3);
\draw[](tid1) -- (tid4);
\draw[](tid0) -- (tid1);
\end{tikzpicture}
\nodepart{three}
\footnotesize{5.82812}
\nodepart{four}
\footnotesize{$50\:50$}
};
 & 
\node[draw=black, rectangle split,  rectangle split parts=4] (sn0xef90f0){
\footnotesize{18.0556}
\nodepart{two}
\begin{tikzpicture}[scale=.2]
\node[circle, scale=0.75, fill] (tid0) at (2.25,1.5){};
\node[circle, scale=0.75, fill] (tid1) at (2.25,3){};
\node[circle, scale=0.75, fill] (tid2) at (0.75,4.5){};
\node[circle, scale=0.75, fill] (tid5) at (0.75,6){};
\node[circle, scale=0.75, fill, task_scheduled] (tid7) at (0.75,7.5){};
\draw[](tid5) -- (tid7);
\draw[](tid2) -- (tid5);
\node[circle, scale=0.75, fill] (tid3) at (2.25,4.5){};
\node[circle, scale=0.75, fill, task_scheduled] (tid6) at (2.25,6){};
\draw[](tid3) -- (tid6);
\node[circle, scale=0.75, fill] (tid4) at (3.75,4.5){};
\draw[](tid1) -- (tid2);
\draw[](tid1) -- (tid3);
\draw[](tid1) -- (tid4);
\draw[](tid0) -- (tid1);
\end{tikzpicture}
\nodepart{three}
\footnotesize{5.78906}
\nodepart{four}
\footnotesize{$50\:25\:25$}
};
 & 
\node[draw=black, rectangle split,  rectangle split parts=4] (sn0xefe720){
\footnotesize{6.25}
\nodepart{two}
\begin{tikzpicture}[scale=.2]
\node[circle, scale=0.75, fill] (tid0) at (2.25,1.5){};
\node[circle, scale=0.75, fill] (tid1) at (2.25,3){};
\node[circle, scale=0.75, fill] (tid2) at (1.5,4.5){};
\node[circle, scale=0.75, fill, task_scheduled] (tid4) at (0.75,6){};
\node[circle, scale=0.75, fill, task_scheduled] (tid5) at (2.25,6){};
\draw[](tid2) -- (tid4);
\draw[](tid2) -- (tid5);
\node[circle, scale=0.75, fill] (tid3) at (3.75,4.5){};
\node[circle, scale=0.75, fill] (tid6) at (3.75,6){};
\node[circle, scale=0.75, fill] (tid7) at (3.75,7.5){};
\draw[](tid6) -- (tid7);
\draw[](tid3) -- (tid6);
\draw[](tid1) -- (tid2);
\draw[](tid1) -- (tid3);
\draw[](tid0) -- (tid1);
\end{tikzpicture}
\nodepart{three}
\footnotesize{5.9375}
\nodepart{four}
\footnotesize{$1$}
};
 & 
\node[draw=black, rectangle split,  rectangle split parts=4] (sn0xefea20){
\footnotesize{9.02778}
\nodepart{two}
\begin{tikzpicture}[scale=.2]
\node[circle, scale=0.75, fill] (tid0) at (2.25,1.5){};
\node[circle, scale=0.75, fill] (tid1) at (2.25,3){};
\node[circle, scale=0.75, fill] (tid2) at (1.5,4.5){};
\node[circle, scale=0.75, fill, task_scheduled] (tid4) at (0.75,6){};
\node[circle, scale=0.75, fill] (tid5) at (2.25,6){};
\draw[](tid2) -- (tid4);
\draw[](tid2) -- (tid5);
\node[circle, scale=0.75, fill] (tid3) at (3.75,4.5){};
\node[circle, scale=0.75, fill] (tid6) at (3.75,6){};
\node[circle, scale=0.75, fill, task_scheduled] (tid7) at (3.75,7.5){};
\draw[](tid6) -- (tid7);
\draw[](tid3) -- (tid6);
\draw[](tid1) -- (tid2);
\draw[](tid1) -- (tid3);
\draw[](tid0) -- (tid1);
\end{tikzpicture}
\nodepart{three}
\footnotesize{5.85156}
\nodepart{four}
\footnotesize{$50\:25\:25$}
};
 & 
\node[draw=black, rectangle split,  rectangle split parts=4] (sn0xf01940){
\footnotesize{5.55556}
\nodepart{two}
\begin{tikzpicture}[scale=.2]
\node[circle, scale=0.75, fill] (tid0) at (3,1.5){};
\node[circle, scale=0.75, fill] (tid1) at (3,3){};
\node[circle, scale=0.75, fill] (tid2) at (1.5,4.5){};
\node[circle, scale=0.75, fill, task_scheduled] (tid5) at (0.75,6){};
\node[circle, scale=0.75, fill] (tid6) at (2.25,6){};
\draw[](tid2) -- (tid5);
\draw[](tid2) -- (tid6);
\node[circle, scale=0.75, fill] (tid3) at (3.75,4.5){};
\node[circle, scale=0.75, fill] (tid7) at (3.75,6){};
\draw[](tid3) -- (tid7);
\node[circle, scale=0.75, fill, task_scheduled] (tid4) at (5.25,4.5){};
\draw[](tid1) -- (tid2);
\draw[](tid1) -- (tid3);
\draw[](tid1) -- (tid4);
\draw[](tid0) -- (tid1);
\end{tikzpicture}
\nodepart{three}
\footnotesize{5.74219}
\nodepart{four}
\footnotesize{$25\:25\:50$}
};
 & 
\node[draw=black, rectangle split,  rectangle split parts=4] (sn0xf069c0){
\footnotesize{1.85185}
\nodepart{two}
\begin{tikzpicture}[scale=.2]
\node[circle, scale=0.75, fill] (tid0) at (3,1.5){};
\node[circle, scale=0.75, fill] (tid1) at (3,3){};
\node[circle, scale=0.75, fill] (tid2) at (1.5,4.5){};
\node[circle, scale=0.75, fill] (tid5) at (0.75,6){};
\node[circle, scale=0.75, fill] (tid6) at (2.25,6){};
\draw[](tid2) -- (tid5);
\draw[](tid2) -- (tid6);
\node[circle, scale=0.75, fill] (tid3) at (3.75,4.5){};
\node[circle, scale=0.75, fill, task_scheduled] (tid7) at (3.75,6){};
\draw[](tid3) -- (tid7);
\node[circle, scale=0.75, fill, task_scheduled] (tid4) at (5.25,4.5){};
\draw[](tid1) -- (tid2);
\draw[](tid1) -- (tid3);
\draw[](tid1) -- (tid4);
\draw[](tid0) -- (tid1);
\end{tikzpicture}
\nodepart{three}
\footnotesize{5.78646}
\nodepart{four}
\footnotesize{$50\:17\:33$}
};
 & 
\node[draw=black, rectangle split,  rectangle split parts=4] (sn0xf01e70){
\footnotesize{3.7037}
\nodepart{two}
\begin{tikzpicture}[scale=.2]
\node[circle, scale=0.75, fill] (tid0) at (3,1.5){};
\node[circle, scale=0.75, fill] (tid1) at (3,3){};
\node[circle, scale=0.75, fill] (tid2) at (1.5,4.5){};
\node[circle, scale=0.75, fill, task_scheduled] (tid5) at (0.75,6){};
\node[circle, scale=0.75, fill, task_scheduled] (tid6) at (2.25,6){};
\draw[](tid2) -- (tid5);
\draw[](tid2) -- (tid6);
\node[circle, scale=0.75, fill] (tid3) at (3.75,4.5){};
\node[circle, scale=0.75, fill] (tid7) at (3.75,6){};
\draw[](tid3) -- (tid7);
\node[circle, scale=0.75, fill] (tid4) at (5.25,4.5){};
\draw[](tid1) -- (tid2);
\draw[](tid1) -- (tid3);
\draw[](tid1) -- (tid4);
\draw[](tid0) -- (tid1);
\end{tikzpicture}
\nodepart{three}
\footnotesize{5.67188}
\nodepart{four}
\footnotesize{$50\:50$}
};
 & 
\node[draw=black, rectangle split,  rectangle split parts=4] (sn0xf02290){
\footnotesize{5.55556}
\nodepart{two}
\begin{tikzpicture}[scale=.2]
\node[circle, scale=0.75, fill] (tid0) at (3,1.5){};
\node[circle, scale=0.75, fill] (tid1) at (3,3){};
\node[circle, scale=0.75, fill] (tid2) at (1.5,4.5){};
\node[circle, scale=0.75, fill, task_scheduled] (tid5) at (0.75,6){};
\node[circle, scale=0.75, fill] (tid6) at (2.25,6){};
\draw[](tid2) -- (tid5);
\draw[](tid2) -- (tid6);
\node[circle, scale=0.75, fill] (tid3) at (3.75,4.5){};
\node[circle, scale=0.75, fill, task_scheduled] (tid7) at (3.75,6){};
\draw[](tid3) -- (tid7);
\node[circle, scale=0.75, fill] (tid4) at (5.25,4.5){};
\draw[](tid1) -- (tid2);
\draw[](tid1) -- (tid3);
\draw[](tid1) -- (tid4);
\draw[](tid0) -- (tid1);
\end{tikzpicture}
\nodepart{three}
\footnotesize{5.6901}
\nodepart{four}
\footnotesize{$33\:17\:25\:25$}
};
 & 
\node[draw=black, rectangle split,  rectangle split parts=4] (sn0xefc0b0){
\footnotesize{6.94444}
\nodepart{two}
\begin{tikzpicture}[scale=.2]
\node[circle, scale=0.75, fill] (tid0) at (2.25,1.5){};
\node[circle, scale=0.75, fill] (tid1) at (2.25,3){};
\node[circle, scale=0.75, fill] (tid2) at (0.75,4.5){};
\node[circle, scale=0.75, fill, task_scheduled] (tid5) at (0.75,6){};
\draw[](tid2) -- (tid5);
\node[circle, scale=0.75, fill] (tid3) at (2.25,4.5){};
\node[circle, scale=0.75, fill, task_scheduled] (tid6) at (2.25,6){};
\draw[](tid3) -- (tid6);
\node[circle, scale=0.75, fill] (tid4) at (3.75,4.5){};
\node[circle, scale=0.75, fill] (tid7) at (3.75,6){};
\draw[](tid4) -- (tid7);
\draw[](tid1) -- (tid2);
\draw[](tid1) -- (tid3);
\draw[](tid1) -- (tid4);
\draw[](tid0) -- (tid1);
\end{tikzpicture}
\nodepart{three}
\footnotesize{5.67188}
\nodepart{four}
\footnotesize{$50\:50$}
};
 & 
\\
};
\end{scope}
\begin{scope}[yshift=\leveltopIIIII cm]
\matrix (line5)[column sep=0.5cm] {
\node[draw=black, rectangle split,  rectangle split parts=4] (sn0xef48f0){
\footnotesize{6.25}
\nodepart{two}
\begin{tikzpicture}[scale=.2]
\node[circle, scale=0.75, fill] (tid0) at (1.5,1.5){};
\node[circle, scale=0.75, fill] (tid1) at (1.5,3){};
\node[circle, scale=0.75, fill] (tid2) at (0.75,4.5){};
\node[circle, scale=0.75, fill] (tid4) at (0.75,6){};
\node[circle, scale=0.75, fill] (tid5) at (0.75,7.5){};
\node[circle, scale=0.75, fill, task_scheduled] (tid6) at (0.75,9){};
\draw[](tid5) -- (tid6);
\draw[](tid4) -- (tid5);
\draw[](tid2) -- (tid4);
\node[circle, scale=0.75, fill, task_scheduled] (tid3) at (2.25,4.5){};
\draw[](tid1) -- (tid2);
\draw[](tid1) -- (tid3);
\draw[](tid0) -- (tid1);
\end{tikzpicture}
\nodepart{three}
\footnotesize{6.0625}
\nodepart{four}
\footnotesize{$50\:50$}
};
 & 
\node[draw=black, rectangle split,  rectangle split parts=4] (sn0xef5110){
\footnotesize{30.2083}
\nodepart{two}
\begin{tikzpicture}[scale=.2]
\node[circle, scale=0.75, fill] (tid0) at (1.5,1.5){};
\node[circle, scale=0.75, fill] (tid1) at (1.5,3){};
\node[circle, scale=0.75, fill] (tid2) at (0.75,4.5){};
\node[circle, scale=0.75, fill] (tid4) at (0.75,6){};
\node[circle, scale=0.75, fill, task_scheduled] (tid6) at (0.75,7.5){};
\draw[](tid4) -- (tid6);
\draw[](tid2) -- (tid4);
\node[circle, scale=0.75, fill] (tid3) at (2.25,4.5){};
\node[circle, scale=0.75, fill, task_scheduled] (tid5) at (2.25,6){};
\draw[](tid3) -- (tid5);
\draw[](tid1) -- (tid2);
\draw[](tid1) -- (tid3);
\draw[](tid0) -- (tid1);
\end{tikzpicture}
\nodepart{three}
\footnotesize{5.4375}
\nodepart{four}
\footnotesize{$50\:50$}
};
 & 
\node[draw=black, rectangle split,  rectangle split parts=4] (sn0xef7880){
\footnotesize{5.55556}
\nodepart{two}
\begin{tikzpicture}[scale=.2]
\node[circle, scale=0.75, fill] (tid0) at (2.25,1.5){};
\node[circle, scale=0.75, fill] (tid1) at (2.25,3){};
\node[circle, scale=0.75, fill] (tid2) at (0.75,4.5){};
\node[circle, scale=0.75, fill] (tid5) at (0.75,6){};
\node[circle, scale=0.75, fill] (tid6) at (0.75,7.5){};
\draw[](tid5) -- (tid6);
\draw[](tid2) -- (tid5);
\node[circle, scale=0.75, fill, task_scheduled] (tid3) at (2.25,4.5){};
\node[circle, scale=0.75, fill, task_scheduled] (tid4) at (3.75,4.5){};
\draw[](tid1) -- (tid2);
\draw[](tid1) -- (tid3);
\draw[](tid1) -- (tid4);
\draw[](tid0) -- (tid1);
\end{tikzpicture}
\nodepart{three}
\footnotesize{5.625}
\nodepart{four}
\footnotesize{$1$}
};
 & 
\node[draw=black, rectangle split,  rectangle split parts=4] (sn0xef7c10){
\footnotesize{14.5833}
\nodepart{two}
\begin{tikzpicture}[scale=.2]
\node[circle, scale=0.75, fill] (tid0) at (2.25,1.5){};
\node[circle, scale=0.75, fill] (tid1) at (2.25,3){};
\node[circle, scale=0.75, fill] (tid2) at (0.75,4.5){};
\node[circle, scale=0.75, fill] (tid5) at (0.75,6){};
\node[circle, scale=0.75, fill, task_scheduled] (tid6) at (0.75,7.5){};
\draw[](tid5) -- (tid6);
\draw[](tid2) -- (tid5);
\node[circle, scale=0.75, fill, task_scheduled] (tid3) at (2.25,4.5){};
\node[circle, scale=0.75, fill] (tid4) at (3.75,4.5){};
\draw[](tid1) -- (tid2);
\draw[](tid1) -- (tid3);
\draw[](tid1) -- (tid4);
\draw[](tid0) -- (tid1);
\end{tikzpicture}
\nodepart{three}
\footnotesize{5.40625}
\nodepart{four}
\footnotesize{$50\:25\:25$}
};
 & 
\node[draw=black, rectangle split,  rectangle split parts=4] (sn0xefeea0){
\footnotesize{3.64583}
\nodepart{two}
\begin{tikzpicture}[scale=.2]
\node[circle, scale=0.75, fill] (tid0) at (2.25,1.5){};
\node[circle, scale=0.75, fill] (tid1) at (2.25,3){};
\node[circle, scale=0.75, fill] (tid2) at (1.5,4.5){};
\node[circle, scale=0.75, fill, task_scheduled] (tid4) at (0.75,6){};
\node[circle, scale=0.75, fill, task_scheduled] (tid5) at (2.25,6){};
\draw[](tid2) -- (tid4);
\draw[](tid2) -- (tid5);
\node[circle, scale=0.75, fill] (tid3) at (3.75,4.5){};
\node[circle, scale=0.75, fill] (tid6) at (3.75,6){};
\draw[](tid3) -- (tid6);
\draw[](tid1) -- (tid2);
\draw[](tid1) -- (tid3);
\draw[](tid0) -- (tid1);
\end{tikzpicture}
\nodepart{three}
\footnotesize{5.25}
\nodepart{four}
\footnotesize{$1$}
};
 & 
\node[draw=black, rectangle split,  rectangle split parts=4] (sn0xeff210){
\footnotesize{4.57176}
\nodepart{two}
\begin{tikzpicture}[scale=.2]
\node[circle, scale=0.75, fill] (tid0) at (2.25,1.5){};
\node[circle, scale=0.75, fill] (tid1) at (2.25,3){};
\node[circle, scale=0.75, fill] (tid2) at (1.5,4.5){};
\node[circle, scale=0.75, fill, task_scheduled] (tid4) at (0.75,6){};
\node[circle, scale=0.75, fill] (tid5) at (2.25,6){};
\draw[](tid2) -- (tid4);
\draw[](tid2) -- (tid5);
\node[circle, scale=0.75, fill] (tid3) at (3.75,4.5){};
\node[circle, scale=0.75, fill, task_scheduled] (tid6) at (3.75,6){};
\draw[](tid3) -- (tid6);
\draw[](tid1) -- (tid2);
\draw[](tid1) -- (tid3);
\draw[](tid0) -- (tid1);
\end{tikzpicture}
\nodepart{three}
\footnotesize{5.28125}
\nodepart{four}
\footnotesize{$25\:25\:50$}
};
 & 
\node[draw=black, rectangle split,  rectangle split parts=4] (sn0xf06c40){
\footnotesize{0.308642}
\nodepart{two}
\begin{tikzpicture}[scale=.2]
\node[circle, scale=0.75, fill] (tid0) at (3,1.5){};
\node[circle, scale=0.75, fill] (tid1) at (3,3){};
\node[circle, scale=0.75, fill] (tid2) at (1.5,4.5){};
\node[circle, scale=0.75, fill] (tid5) at (0.75,6){};
\node[circle, scale=0.75, fill] (tid6) at (2.25,6){};
\draw[](tid2) -- (tid5);
\draw[](tid2) -- (tid6);
\node[circle, scale=0.75, fill, task_scheduled] (tid3) at (3.75,4.5){};
\node[circle, scale=0.75, fill, task_scheduled] (tid4) at (5.25,4.5){};
\draw[](tid1) -- (tid2);
\draw[](tid1) -- (tid3);
\draw[](tid1) -- (tid4);
\draw[](tid0) -- (tid1);
\end{tikzpicture}
\nodepart{three}
\footnotesize{5.375}
\nodepart{four}
\footnotesize{$1$}
};
 & 
\node[draw=black, rectangle split,  rectangle split parts=4] (sn0xf026b0){
\footnotesize{2.46914}
\nodepart{two}
\begin{tikzpicture}[scale=.2]
\node[circle, scale=0.75, fill] (tid0) at (3,1.5){};
\node[circle, scale=0.75, fill] (tid1) at (3,3){};
\node[circle, scale=0.75, fill] (tid2) at (1.5,4.5){};
\node[circle, scale=0.75, fill, task_scheduled] (tid5) at (0.75,6){};
\node[circle, scale=0.75, fill] (tid6) at (2.25,6){};
\draw[](tid2) -- (tid5);
\draw[](tid2) -- (tid6);
\node[circle, scale=0.75, fill, task_scheduled] (tid3) at (3.75,4.5){};
\node[circle, scale=0.75, fill] (tid4) at (5.25,4.5){};
\draw[](tid1) -- (tid2);
\draw[](tid1) -- (tid3);
\draw[](tid1) -- (tid4);
\draw[](tid0) -- (tid1);
\end{tikzpicture}
\nodepart{three}
\footnotesize{5.25}
\nodepart{four}
\footnotesize{$25\:25\:25\:25$}
};
 & 
\node[draw=black, rectangle split,  rectangle split parts=4] (sn0xf02bb0){
\footnotesize{0.925926}
\nodepart{two}
\begin{tikzpicture}[scale=.2]
\node[circle, scale=0.75, fill] (tid0) at (3,1.5){};
\node[circle, scale=0.75, fill] (tid1) at (3,3){};
\node[circle, scale=0.75, fill] (tid2) at (1.5,4.5){};
\node[circle, scale=0.75, fill, task_scheduled] (tid5) at (0.75,6){};
\node[circle, scale=0.75, fill, task_scheduled] (tid6) at (2.25,6){};
\draw[](tid2) -- (tid5);
\draw[](tid2) -- (tid6);
\node[circle, scale=0.75, fill] (tid3) at (3.75,4.5){};
\node[circle, scale=0.75, fill] (tid4) at (5.25,4.5){};
\draw[](tid1) -- (tid2);
\draw[](tid1) -- (tid3);
\draw[](tid1) -- (tid4);
\draw[](tid0) -- (tid1);
\end{tikzpicture}
\nodepart{three}
\footnotesize{5.125}
\nodepart{four}
\footnotesize{$1$}
};
 & 
\node[draw=black, rectangle split,  rectangle split parts=4] (sn0xef9e10){
\footnotesize{20.2546}
\nodepart{two}
\begin{tikzpicture}[scale=.2]
\node[circle, scale=0.75, fill] (tid0) at (2.25,1.5){};
\node[circle, scale=0.75, fill] (tid1) at (2.25,3){};
\node[circle, scale=0.75, fill] (tid2) at (0.75,4.5){};
\node[circle, scale=0.75, fill, task_scheduled] (tid5) at (0.75,6){};
\draw[](tid2) -- (tid5);
\node[circle, scale=0.75, fill] (tid3) at (2.25,4.5){};
\node[circle, scale=0.75, fill] (tid6) at (2.25,6){};
\draw[](tid3) -- (tid6);
\node[circle, scale=0.75, fill, task_scheduled] (tid4) at (3.75,4.5){};
\draw[](tid1) -- (tid2);
\draw[](tid1) -- (tid3);
\draw[](tid1) -- (tid4);
\draw[](tid0) -- (tid1);
\end{tikzpicture}
\nodepart{three}
\footnotesize{5.21875}
\nodepart{four}
\footnotesize{$50\:25\:25$}
};
 & 
\node[draw=black, rectangle split,  rectangle split parts=4] (sn0xefa190){
\footnotesize{11.2269}
\nodepart{two}
\begin{tikzpicture}[scale=.2]
\node[circle, scale=0.75, fill] (tid0) at (2.25,1.5){};
\node[circle, scale=0.75, fill] (tid1) at (2.25,3){};
\node[circle, scale=0.75, fill] (tid2) at (0.75,4.5){};
\node[circle, scale=0.75, fill, task_scheduled] (tid5) at (0.75,6){};
\draw[](tid2) -- (tid5);
\node[circle, scale=0.75, fill] (tid3) at (2.25,4.5){};
\node[circle, scale=0.75, fill, task_scheduled] (tid6) at (2.25,6){};
\draw[](tid3) -- (tid6);
\node[circle, scale=0.75, fill] (tid4) at (3.75,4.5){};
\draw[](tid1) -- (tid2);
\draw[](tid1) -- (tid3);
\draw[](tid1) -- (tid4);
\draw[](tid0) -- (tid1);
\end{tikzpicture}
\nodepart{three}
\footnotesize{5.125}
\nodepart{four}
\footnotesize{$1$}
};
 & 
\\
};
\end{scope}
\begin{scope}[yshift=\leveltopIIIIII cm]
\matrix (line6)[column sep=0.5cm] {
\node[draw=black, rectangle split,  rectangle split parts=4] (sn0xef5310){
\footnotesize{3.125}
\nodepart{two}
\begin{tikzpicture}[scale=.2]
\node[circle, scale=0.75, fill] (tid0) at (0.75,1.5){};
\node[circle, scale=0.75, fill] (tid1) at (0.75,3){};
\node[circle, scale=0.75, fill] (tid2) at (0.75,4.5){};
\node[circle, scale=0.75, fill] (tid3) at (0.75,6){};
\node[circle, scale=0.75, fill] (tid4) at (0.75,7.5){};
\node[circle, scale=0.75, fill, task_scheduled] (tid5) at (0.75,9){};
\draw[](tid4) -- (tid5);
\draw[](tid3) -- (tid4);
\draw[](tid2) -- (tid3);
\draw[](tid1) -- (tid2);
\draw[](tid0) -- (tid1);
\end{tikzpicture}
\nodepart{three}
\footnotesize{6}
\nodepart{four}
\footnotesize{$1$}
};
 & 
\node[draw=black, rectangle split,  rectangle split parts=4] (sn0xef5750){
\footnotesize{31.0764}
\nodepart{two}
\begin{tikzpicture}[scale=.2]
\node[circle, scale=0.75, fill] (tid0) at (1.5,1.5){};
\node[circle, scale=0.75, fill] (tid1) at (1.5,3){};
\node[circle, scale=0.75, fill] (tid2) at (0.75,4.5){};
\node[circle, scale=0.75, fill] (tid4) at (0.75,6){};
\node[circle, scale=0.75, fill, task_scheduled] (tid5) at (0.75,7.5){};
\draw[](tid4) -- (tid5);
\draw[](tid2) -- (tid4);
\node[circle, scale=0.75, fill, task_scheduled] (tid3) at (2.25,4.5){};
\draw[](tid1) -- (tid2);
\draw[](tid1) -- (tid3);
\draw[](tid0) -- (tid1);
\end{tikzpicture}
\nodepart{three}
\footnotesize{5.125}
\nodepart{four}
\footnotesize{$50\:50$}
};
 & 
\node[draw=black, rectangle split,  rectangle split parts=4] (sn0xeff630){
\footnotesize{2.06887}
\nodepart{two}
\begin{tikzpicture}[scale=.2]
\node[circle, scale=0.75, fill] (tid0) at (2.25,1.5){};
\node[circle, scale=0.75, fill] (tid1) at (2.25,3){};
\node[circle, scale=0.75, fill] (tid2) at (1.5,4.5){};
\node[circle, scale=0.75, fill, task_scheduled] (tid4) at (0.75,6){};
\node[circle, scale=0.75, fill] (tid5) at (2.25,6){};
\draw[](tid2) -- (tid4);
\draw[](tid2) -- (tid5);
\node[circle, scale=0.75, fill, task_scheduled] (tid3) at (3.75,4.5){};
\draw[](tid1) -- (tid2);
\draw[](tid1) -- (tid3);
\draw[](tid0) -- (tid1);
\end{tikzpicture}
\nodepart{three}
\footnotesize{4.875}
\nodepart{four}
\footnotesize{$50\:50$}
};
 & 
\node[draw=black, rectangle split,  rectangle split parts=4] (sn0xeff8f0){
\footnotesize{1.76022}
\nodepart{two}
\begin{tikzpicture}[scale=.2]
\node[circle, scale=0.75, fill] (tid0) at (2.25,1.5){};
\node[circle, scale=0.75, fill] (tid1) at (2.25,3){};
\node[circle, scale=0.75, fill] (tid2) at (1.5,4.5){};
\node[circle, scale=0.75, fill, task_scheduled] (tid4) at (0.75,6){};
\node[circle, scale=0.75, fill, task_scheduled] (tid5) at (2.25,6){};
\draw[](tid2) -- (tid4);
\draw[](tid2) -- (tid5);
\node[circle, scale=0.75, fill] (tid3) at (3.75,4.5){};
\draw[](tid1) -- (tid2);
\draw[](tid1) -- (tid3);
\draw[](tid0) -- (tid1);
\end{tikzpicture}
\nodepart{three}
\footnotesize{4.75}
\nodepart{four}
\footnotesize{$1$}
};
 & 
\node[draw=black, rectangle split,  rectangle split parts=4] (sn0xef7040){
\footnotesize{31.1632}
\nodepart{two}
\begin{tikzpicture}[scale=.2]
\node[circle, scale=0.75, fill] (tid0) at (1.5,1.5){};
\node[circle, scale=0.75, fill] (tid1) at (1.5,3){};
\node[circle, scale=0.75, fill] (tid2) at (0.75,4.5){};
\node[circle, scale=0.75, fill, task_scheduled] (tid4) at (0.75,6){};
\draw[](tid2) -- (tid4);
\node[circle, scale=0.75, fill] (tid3) at (2.25,4.5){};
\node[circle, scale=0.75, fill, task_scheduled] (tid5) at (2.25,6){};
\draw[](tid3) -- (tid5);
\draw[](tid1) -- (tid2);
\draw[](tid1) -- (tid3);
\draw[](tid0) -- (tid1);
\end{tikzpicture}
\nodepart{three}
\footnotesize{4.75}
\nodepart{four}
\footnotesize{$1$}
};
 & 
\node[draw=black, rectangle split,  rectangle split parts=4] (sn0xef7d10){
\footnotesize{9.32678}
\nodepart{two}
\begin{tikzpicture}[scale=.2]
\node[circle, scale=0.75, fill] (tid0) at (2.25,1.5){};
\node[circle, scale=0.75, fill] (tid1) at (2.25,3){};
\node[circle, scale=0.75, fill] (tid2) at (0.75,4.5){};
\node[circle, scale=0.75, fill] (tid5) at (0.75,6){};
\draw[](tid2) -- (tid5);
\node[circle, scale=0.75, fill, task_scheduled] (tid3) at (2.25,4.5){};
\node[circle, scale=0.75, fill, task_scheduled] (tid4) at (3.75,4.5){};
\draw[](tid1) -- (tid2);
\draw[](tid1) -- (tid3);
\draw[](tid1) -- (tid4);
\draw[](tid0) -- (tid1);
\end{tikzpicture}
\nodepart{three}
\footnotesize{4.75}
\nodepart{four}
\footnotesize{$1$}
};
 & 
\node[draw=black, rectangle split,  rectangle split parts=4] (sn0xef8590){
\footnotesize{21.4796}
\nodepart{two}
\begin{tikzpicture}[scale=.2]
\node[circle, scale=0.75, fill] (tid0) at (2.25,1.5){};
\node[circle, scale=0.75, fill] (tid1) at (2.25,3){};
\node[circle, scale=0.75, fill] (tid2) at (0.75,4.5){};
\node[circle, scale=0.75, fill, task_scheduled] (tid5) at (0.75,6){};
\draw[](tid2) -- (tid5);
\node[circle, scale=0.75, fill, task_scheduled] (tid3) at (2.25,4.5){};
\node[circle, scale=0.75, fill] (tid4) at (3.75,4.5){};
\draw[](tid1) -- (tid2);
\draw[](tid1) -- (tid3);
\draw[](tid1) -- (tid4);
\draw[](tid0) -- (tid1);
\end{tikzpicture}
\nodepart{three}
\footnotesize{4.625}
\nodepart{four}
\footnotesize{$50\:50$}
};
 & 
\\
};
\end{scope}
\begin{scope}[yshift=\leveltopIIIIIII cm]
\matrix (line7)[column sep=0.5cm] {
\node[draw=black, rectangle split,  rectangle split parts=4] (sn0xef5850){
\footnotesize{18.6632}
\nodepart{two}
\begin{tikzpicture}[scale=.2]
\node[circle, scale=0.75, fill] (tid0) at (0.75,1.5){};
\node[circle, scale=0.75, fill] (tid1) at (0.75,3){};
\node[circle, scale=0.75, fill] (tid2) at (0.75,4.5){};
\node[circle, scale=0.75, fill] (tid3) at (0.75,6){};
\node[circle, scale=0.75, fill, task_scheduled] (tid4) at (0.75,7.5){};
\draw[](tid3) -- (tid4);
\draw[](tid2) -- (tid3);
\draw[](tid1) -- (tid2);
\draw[](tid0) -- (tid1);
\end{tikzpicture}
\nodepart{three}
\footnotesize{5}
\nodepart{four}
\footnotesize{$1$}
};
 & 
\node[draw=black, rectangle split,  rectangle split parts=4] (sn0xeffc80){
\footnotesize{1.03443}
\nodepart{two}
\begin{tikzpicture}[scale=.2]
\node[circle, scale=0.75, fill] (tid0) at (1.5,1.5){};
\node[circle, scale=0.75, fill] (tid1) at (1.5,3){};
\node[circle, scale=0.75, fill] (tid2) at (1.5,4.5){};
\node[circle, scale=0.75, fill, task_scheduled] (tid3) at (0.75,6){};
\node[circle, scale=0.75, fill, task_scheduled] (tid4) at (2.25,6){};
\draw[](tid2) -- (tid3);
\draw[](tid2) -- (tid4);
\draw[](tid1) -- (tid2);
\draw[](tid0) -- (tid1);
\end{tikzpicture}
\nodepart{three}
\footnotesize{4.5}
\nodepart{four}
\footnotesize{$1$}
};
 & 
\node[draw=black, rectangle split,  rectangle split parts=4] (sn0xef64d0){
\footnotesize{69.5626}
\nodepart{two}
\begin{tikzpicture}[scale=.2]
\node[circle, scale=0.75, fill] (tid0) at (1.5,1.5){};
\node[circle, scale=0.75, fill] (tid1) at (1.5,3){};
\node[circle, scale=0.75, fill] (tid2) at (0.75,4.5){};
\node[circle, scale=0.75, fill, task_scheduled] (tid4) at (0.75,6){};
\draw[](tid2) -- (tid4);
\node[circle, scale=0.75, fill, task_scheduled] (tid3) at (2.25,4.5){};
\draw[](tid1) -- (tid2);
\draw[](tid1) -- (tid3);
\draw[](tid0) -- (tid1);
\end{tikzpicture}
\nodepart{three}
\footnotesize{4.25}
\nodepart{four}
\footnotesize{$50\:50$}
};
 & 
\node[draw=black, rectangle split,  rectangle split parts=4] (sn0xef8750){
\footnotesize{10.7398}
\nodepart{two}
\begin{tikzpicture}[scale=.2]
\node[circle, scale=0.75, fill] (tid0) at (2.25,1.5){};
\node[circle, scale=0.75, fill] (tid1) at (2.25,3){};
\node[circle, scale=0.75, fill, task_scheduled] (tid2) at (0.75,4.5){};
\node[circle, scale=0.75, fill, task_scheduled] (tid3) at (2.25,4.5){};
\node[circle, scale=0.75, fill] (tid4) at (3.75,4.5){};
\draw[](tid1) -- (tid2);
\draw[](tid1) -- (tid3);
\draw[](tid1) -- (tid4);
\draw[](tid0) -- (tid1);
\end{tikzpicture}
\nodepart{three}
\footnotesize{4}
\nodepart{four}
\footnotesize{$1$}
};
 & 
\\
};
\end{scope}
\begin{scope}[yshift=\leveltopIIIIIIII cm]
\matrix (line8)[column sep=0.5cm] {
\node[draw=black, rectangle split,  rectangle split parts=4] (sn0xef5c90){
\footnotesize{54.4789}
\nodepart{two}
\begin{tikzpicture}[scale=.2]
\node[circle, scale=0.75, fill] (tid0) at (0.75,1.5){};
\node[circle, scale=0.75, fill] (tid1) at (0.75,3){};
\node[circle, scale=0.75, fill] (tid2) at (0.75,4.5){};
\node[circle, scale=0.75, fill, task_scheduled] (tid3) at (0.75,6){};
\draw[](tid2) -- (tid3);
\draw[](tid1) -- (tid2);
\draw[](tid0) -- (tid1);
\end{tikzpicture}
\nodepart{three}
\footnotesize{4}
\nodepart{four}
\footnotesize{$1$}
};
 & 
\node[draw=black, rectangle split,  rectangle split parts=4] (sn0xef6940){
\footnotesize{45.5211}
\nodepart{two}
\begin{tikzpicture}[scale=.2]
\node[circle, scale=0.75, fill] (tid0) at (1.5,1.5){};
\node[circle, scale=0.75, fill] (tid1) at (1.5,3){};
\node[circle, scale=0.75, fill, task_scheduled] (tid2) at (0.75,4.5){};
\node[circle, scale=0.75, fill, task_scheduled] (tid3) at (2.25,4.5){};
\draw[](tid1) -- (tid2);
\draw[](tid1) -- (tid3);
\draw[](tid0) -- (tid1);
\end{tikzpicture}
\nodepart{three}
\footnotesize{3.5}
\nodepart{four}
\footnotesize{$1$}
};
 & 
\\
};
\end{scope}
\begin{scope}[yshift=\leveltopIIIIIIIII cm]
\matrix (line9)[column sep=0.5cm] {
\node[draw=black, rectangle split,  rectangle split parts=4] (sn0xef5950){
\footnotesize{100}
\nodepart{two}
\begin{tikzpicture}[scale=.2]
\node[circle, scale=0.75, fill] (tid0) at (0.75,1.5){};
\node[circle, scale=0.75, fill] (tid1) at (0.75,3){};
\node[circle, scale=0.75, fill, task_scheduled] (tid2) at (0.75,4.5){};
\draw[](tid1) -- (tid2);
\draw[](tid0) -- (tid1);
\end{tikzpicture}
\nodepart{three}
\footnotesize{3}
\nodepart{four}
\footnotesize{$1$}
};
 & 
\\
};
\end{scope}
\begin{scope}[yshift=\leveltopIIIIIIIIII cm]
\matrix (line10)[column sep=0.5cm] {
\node[draw=black, rectangle split,  rectangle split parts=4] (sn0xef5b30){
\footnotesize{100}
\nodepart{two}
\begin{tikzpicture}[scale=.2]
\node[circle, scale=0.75, fill] (tid0) at (0.75,1.5){};
\node[circle, scale=0.75, fill, task_scheduled] (tid1) at (0.75,3){};
\draw[](tid0) -- (tid1);
\end{tikzpicture}
\nodepart{three}
\footnotesize{2}
\nodepart{four}
\footnotesize{$1$}
};
 & 
\\
};
\end{scope}
\draw (sn0xef2440.south) -- (sn0xf051d0.north);
\draw (sn0xef2440.south) -- (sn0xefcc90.north);
\draw (sn0xef2440.south) -- (sn0xf03560.north);
\draw (sn0xef2440.south) -- (sn0xf05e10.north);
\draw (sn0xf051d0.south) -- (sn0xefe2f0.north);
\draw (sn0xf051d0.south) -- (sn0xeffff0.north);
\draw (sn0xf051d0.south) -- (sn0xf05830.north);
\draw (sn0xefcc90.south) -- (sn0xefa920.north);
\draw (sn0xefcc90.south) -- (sn0xef3e70.north);
\draw (sn0xefcc90.south) -- (sn0xeffff0.north);
\draw (sn0xefcc90.south) -- (sn0xf00d20.north);
\draw (sn0xefcc90.south) -- (sn0xf01470.north);
\draw (sn0xf03560.south) -- (sn0xefabd0.north);
\draw (sn0xf03560.south) -- (sn0xefb500.north);
\draw (sn0xf03560.south) -- (sn0xeffff0.north);
\draw (sn0xf03560.south) -- (sn0xf00d20.north);
\draw (sn0xf03560.south) -- (sn0xf01470.north);
\draw (sn0xf05e10.south) -- (sn0xf05830.north);
\draw (sn0xf05e10.south) -- (sn0xf01470.north);
\draw (sn0xf05e10.south) -- (sn0xf04f90.north);
\draw (sn0xf05e10.south) -- (sn0xf06500.north);
\draw (sn0xefe2f0.south) -- (sn0xef43a0.north);
\draw (sn0xefe2f0.south) -- (sn0xefe720.north);
\draw (sn0xefe2f0.south) -- (sn0xefea20.north);
\draw (sn0xefa920.south) -- (sn0xef43a0.north);
\draw (sn0xefa920.south) -- (sn0xef8300.north);
\draw (sn0xefa920.south) -- (sn0xefbc60.north);
\draw (sn0xef3e70.south) -- (sn0xef4790.north);
\draw (sn0xef3e70.south) -- (sn0xef8300.north);
\draw (sn0xef3e70.south) -- (sn0xef90f0.north);
\draw (sn0xefabd0.south) -- (sn0xef8300.north);
\draw (sn0xefabd0.south) -- (sn0xef90f0.north);
\draw (sn0xefb500.south) -- (sn0xefbc60.north);
\draw (sn0xefb500.south) -- (sn0xef90f0.north);
\draw (sn0xefb500.south) -- (sn0xefc0b0.north);
\draw (sn0xeffff0.south) -- (sn0xefe720.north);
\draw (sn0xeffff0.south) -- (sn0xefea20.north);
\draw (sn0xeffff0.south) -- (sn0xef8300.north);
\draw (sn0xeffff0.south) -- (sn0xefbc60.north);
\draw (sn0xf05830.south) -- (sn0xefea20.north);
\draw (sn0xf05830.south) -- (sn0xf01940.north);
\draw (sn0xf05830.south) -- (sn0xf069c0.north);
\draw (sn0xf00d20.south) -- (sn0xef8300.north);
\draw (sn0xf00d20.south) -- (sn0xef90f0.north);
\draw (sn0xf01470.south) -- (sn0xefbc60.north);
\draw (sn0xf01470.south) -- (sn0xef90f0.north);
\draw (sn0xf01470.south) -- (sn0xf01940.north);
\draw (sn0xf01470.south) -- (sn0xf01e70.north);
\draw (sn0xf01470.south) -- (sn0xf02290.north);
\draw (sn0xf04f90.south) -- (sn0xefc0b0.north);
\draw (sn0xf04f90.south) -- (sn0xf01940.north);
\draw (sn0xf04f90.south) -- (sn0xf01e70.north);
\draw (sn0xf04f90.south) -- (sn0xf02290.north);
\draw (sn0xf06500.south) -- (sn0xf069c0.north);
\draw (sn0xf06500.south) -- (sn0xf02290.north);
\draw (sn0xef43a0.south) -- (sn0xef48f0.north);
\draw (sn0xef43a0.south) -- (sn0xef5110.north);
\draw (sn0xef4790.south) -- (sn0xef48f0.north);
\draw (sn0xef4790.south) -- (sn0xef7880.north);
\draw (sn0xef4790.south) -- (sn0xef7c10.north);
\draw (sn0xef8300.south) -- (sn0xef5110.north);
\draw (sn0xef8300.south) -- (sn0xef7880.north);
\draw (sn0xef8300.south) -- (sn0xef7c10.north);
\draw (sn0xefbc60.south) -- (sn0xef5110.north);
\draw (sn0xefbc60.south) -- (sn0xef9e10.north);
\draw (sn0xef90f0.south) -- (sn0xef7c10.north);
\draw (sn0xef90f0.south) -- (sn0xef9e10.north);
\draw (sn0xef90f0.south) -- (sn0xefa190.north);
\draw (sn0xefe720.south) -- (sn0xef5110.north);
\draw (sn0xefea20.south) -- (sn0xef5110.north);
\draw (sn0xefea20.south) -- (sn0xefeea0.north);
\draw (sn0xefea20.south) -- (sn0xeff210.north);
\draw (sn0xf01940.south) -- (sn0xefeea0.north);
\draw (sn0xf01940.south) -- (sn0xeff210.north);
\draw (sn0xf01940.south) -- (sn0xef9e10.north);
\draw (sn0xf069c0.south) -- (sn0xeff210.north);
\draw (sn0xf069c0.south) -- (sn0xf06c40.north);
\draw (sn0xf069c0.south) -- (sn0xf026b0.north);
\draw (sn0xf01e70.south) -- (sn0xef9e10.north);
\draw (sn0xf01e70.south) -- (sn0xefa190.north);
\draw (sn0xf02290.south) -- (sn0xef9e10.north);
\draw (sn0xf02290.south) -- (sn0xefa190.north);
\draw (sn0xf02290.south) -- (sn0xf026b0.north);
\draw (sn0xf02290.south) -- (sn0xf02bb0.north);
\draw (sn0xefc0b0.south) -- (sn0xef9e10.north);
\draw (sn0xefc0b0.south) -- (sn0xefa190.north);
\draw (sn0xef48f0.south) -- (sn0xef5310.north);
\draw (sn0xef48f0.south) -- (sn0xef5750.north);
\draw (sn0xef5110.south) -- (sn0xef5750.north);
\draw (sn0xef5110.south) -- (sn0xef7040.north);
\draw (sn0xef7880.south) -- (sn0xef5750.north);
\draw (sn0xef7c10.south) -- (sn0xef5750.north);
\draw (sn0xef7c10.south) -- (sn0xef7d10.north);
\draw (sn0xef7c10.south) -- (sn0xef8590.north);
\draw (sn0xefeea0.south) -- (sn0xef7040.north);
\draw (sn0xeff210.south) -- (sn0xef7040.north);
\draw (sn0xeff210.south) -- (sn0xeff630.north);
\draw (sn0xeff210.south) -- (sn0xeff8f0.north);
\draw (sn0xf06c40.south) -- (sn0xeff630.north);
\draw (sn0xf026b0.south) -- (sn0xeff630.north);
\draw (sn0xf026b0.south) -- (sn0xeff8f0.north);
\draw (sn0xf026b0.south) -- (sn0xef7d10.north);
\draw (sn0xf026b0.south) -- (sn0xef8590.north);
\draw (sn0xf02bb0.south) -- (sn0xef8590.north);
\draw (sn0xef9e10.south) -- (sn0xef7040.north);
\draw (sn0xef9e10.south) -- (sn0xef7d10.north);
\draw (sn0xef9e10.south) -- (sn0xef8590.north);
\draw (sn0xefa190.south) -- (sn0xef8590.north);
\draw (sn0xef5310.south) -- (sn0xef5850.north);
\draw (sn0xef5750.south) -- (sn0xef5850.north);
\draw (sn0xef5750.south) -- (sn0xef64d0.north);
\draw (sn0xeff630.south) -- (sn0xeffc80.north);
\draw (sn0xeff630.south) -- (sn0xef64d0.north);
\draw (sn0xeff8f0.south) -- (sn0xef64d0.north);
\draw (sn0xef7040.south) -- (sn0xef64d0.north);
\draw (sn0xef7d10.south) -- (sn0xef64d0.north);
\draw (sn0xef8590.south) -- (sn0xef64d0.north);
\draw (sn0xef8590.south) -- (sn0xef8750.north);
\draw (sn0xef5850.south) -- (sn0xef5c90.north);
\draw (sn0xeffc80.south) -- (sn0xef5c90.north);
\draw (sn0xef64d0.south) -- (sn0xef5c90.north);
\draw (sn0xef64d0.south) -- (sn0xef6940.north);
\draw (sn0xef8750.south) -- (sn0xef6940.north);
\draw (sn0xef5c90.south) -- (sn0xef5950.north);
\draw (sn0xef6940.south) -- (sn0xef5950.north);
\draw (sn0xef5950.south) -- (sn0xef5b30.north);
\end{tikzpicture}

%%% Local Variables:
%%% TeX-master: "thesis/thesis.tex"
%%% End: 


\chapter{Two Processors}
\label{chap:p2}

\section{Profiles}
\label{sec:p2-profiles}

If one considers the problem of scheduling an intree onto two processors, it becomes clear that HLF is optimal (\todo{Proof.}). \todo{Is the following correct:} Moreover, we can conclude that we can compute the optimal expected finish time in polynomial time.

This section shows how the original problem of an intree DAG can be mapped onto another, more compact structure.

\subsection{Profiles of Intrees}
\label{sec:p2-simple-method-runtime-profiles-for-intrees}

If we consider the trees in figure \ref{fig:p2-four-intrees-with-same-profile-6-3-1}, we can compute that for two processors HLF always yields an expected run time of $\frac{49}{8}$ for each of them, which is optimal:

\begin{figure}[ht]
  \centering
  \begin{tikzpicture}[scale=.2]
\node[circle, scale=0.75, fill] (tid0) at (4.5,1.5){};
\node[circle, scale=0.75, fill] (tid1) at (2.25,3){};
\node[circle, scale=0.75, fill] (tid4) at (0.75,4.5){};
\node[circle, scale=0.75, fill] (tid5) at (2.25,4.5){};
\node[circle, scale=0.75, fill] (tid6) at (3.75,4.5){};
\draw[](tid1) -- (tid4);
\draw[](tid1) -- (tid5);
\draw[](tid1) -- (tid6);
\node[circle, scale=0.75, fill] (tid2) at (6,3){};
\node[circle, scale=0.75, fill] (tid7) at (5.25,4.5){};
\node[circle, scale=0.75, fill] (tid8) at (6.75,4.5){};
\draw[](tid2) -- (tid7);
\draw[](tid2) -- (tid8);
\node[circle, scale=0.75, fill] (tid3) at (8.25,3){};
\node[circle, scale=0.75, fill] (tid9) at (8.25,4.5){};
\draw[](tid3) -- (tid9);
\draw[](tid0) -- (tid1);
\draw[](tid0) -- (tid2);
\draw[](tid0) -- (tid3);
\end{tikzpicture}
%%% Local Variables:
%%% TeX-master: "thesis/thesis.tex"
%%% End: \hspace{0.5cm}
  \begin{tikzpicture}[scale=.2]
\node[circle, scale=0.75, fill] (tid0) at (5.25,1.5){};
\node[circle, scale=0.75, fill] (tid1) at (3,3){};
\node[circle, scale=0.75, fill] (tid4) at (0.75,4.5){};
\node[circle, scale=0.75, fill] (tid5) at (2.25,4.5){};
\node[circle, scale=0.75, fill] (tid6) at (3.75,4.5){};
\node[circle, scale=0.75, fill] (tid7) at (5.25,4.5){};
\draw[](tid1) -- (tid4);
\draw[](tid1) -- (tid5);
\draw[](tid1) -- (tid6);
\draw[](tid1) -- (tid7);
\node[circle, scale=0.75, fill] (tid2) at (7.5,3){};
\node[circle, scale=0.75, fill] (tid8) at (6.75,4.5){};
\node[circle, scale=0.75, fill] (tid9) at (8.25,4.5){};
\draw[](tid2) -- (tid8);
\draw[](tid2) -- (tid9);
\node[circle, scale=0.75, fill] (tid3) at (9.75,3){};
\draw[](tid0) -- (tid1);
\draw[](tid0) -- (tid2);
\draw[](tid0) -- (tid3);
\end{tikzpicture}
%%% Local Variables:
%%% TeX-master: "thesis/thesis.tex"
%%% End: \hspace{0.5cm}
  \begin{tikzpicture}[scale=.2]
\node[circle, scale=0.75, fill] (tid0) at (6,1.5){};
\node[circle, scale=0.75, fill] (tid1) at (4.5,3){};
\node[circle, scale=0.75, fill, red] (tid4) at (0.75,4.5){};
\node[circle, scale=0.75, fill, red] (tid5) at (2.25,4.5){};
\node[circle, scale=0.75, fill] (tid6) at (3.75,4.5){};
\node[circle, scale=0.75, fill] (tid7) at (5.25,4.5){};
\node[circle, scale=0.75, fill] (tid8) at (6.75,4.5){};
\node[circle, scale=0.75, fill] (tid9) at (8.25,4.5){};
\draw[](tid1) -- (tid4);
\draw[](tid1) -- (tid5);
\draw[](tid1) -- (tid6);
\draw[](tid1) -- (tid7);
\draw[](tid1) -- (tid8);
\draw[](tid1) -- (tid9);
\node[circle, scale=0.75, fill] (tid2) at (9.75,3){};
\node[circle, scale=0.75, fill] (tid3) at (11.25,3){};
\draw[](tid0) -- (tid1);
\draw[](tid0) -- (tid2);
\draw[](tid0) -- (tid3);
\end{tikzpicture}
%%% Local Variables:
%%% TeX-master: "thesis/thesis.tex"
%%% End: \hspace{0.5cm}
  \begin{tikzpicture}[scale=.2]
\node[circle, scale=0.75, fill] (tid0) at (5.25,1.5){};
\node[circle, scale=0.75, fill] (tid1) at (2.25,3){};
\node[circle, scale=0.75, fill, task_scheduled] (tid4) at (0.75,4.5){};
\node[circle, scale=0.75, fill] (tid5) at (2.25,4.5){};
\node[circle, scale=0.75, fill] (tid6) at (3.75,4.5){};
\draw[](tid1) -- (tid4);
\draw[](tid1) -- (tid5);
\draw[](tid1) -- (tid6);
\node[circle, scale=0.75, fill] (tid2) at (6.75,3){};
\node[circle, scale=0.75, fill, task_scheduled] (tid7) at (5.25,4.5){};
\node[circle, scale=0.75, fill] (tid8) at (6.75,4.5){};
\node[circle, scale=0.75, fill] (tid9) at (8.25,4.5){};
\draw[](tid2) -- (tid7);
\draw[](tid2) -- (tid8);
\draw[](tid2) -- (tid9);
\node[circle, scale=0.75, fill] (tid3) at (9.75,3){};
\draw[](tid0) -- (tid1);
\draw[](tid0) -- (tid2);
\draw[](tid0) -- (tid3);
\end{tikzpicture}
%%% Local Variables:
%%% TeX-master: "thesis/thesis.tex"
%%% End: 
  \caption{Four intrees with the same profile ($\profile{6,3,1}$). All of them have expected run time of $49/8$ if scheduled with HLF on two processors.}
  \label{fig:p2-four-intrees-with-same-profile-6-3-1}
\end{figure}

The intrees in figure \ref{fig:p2-four-intrees-with-same-profile-6-3-1} have the following in common: At each level, they have the same amount of tasks (six tasks at the topmost level, three in the middle one and one at the bottom level).

We can use the number of tasks per level as a (non-bijective) ``encoding'' of intrees. For now, we call this encoding a \emph{profile} of the intree. The above intrees would all be encoded as a profile containing the numbers 6, 3 and 1 in that order. We denote the profile by $\profile{6, 3, 1}$.

Note that not all sequences of numbers can be used as profiles. In particular, the last number in a profile is (w.l.o.g.) 1 (since we have only one task as the root of the tree)\footnote{This, of course, introduces some overhead in notation, but we leave it as it is since it is easier to read this way.}. Moreover, it can not be the case that there is a zero in a profile (since this would imply that there is \emph{no task} on one specific level in the intree).

Moreover, we introduce a abbreviating notation for profiles.

\begin{definition}[Compact notation of profiles]
  For a profile $p$, we introduce a shorthand notation that groups successive ones. That is, instead of writing $j$ consecutive ones, we simply write $\profileones{j}$.
\end{definition}

As a simple example, we rewrite $\profile{2,1,1,1,5,2,1,1,1,1,1,2,1}$ as 
$\profile{2,\profileones{3},5,2,\profileones{5},2,\profileones{1}}$.

\subsection{Profiles and HLF}
\label{sec:p2-simple-method-runtime-profiles-hlf}

For two processors and HLF-scheduling, we can easily conclude the successors of a profile. Let us first of all consider some examples here: If we have the profile $\profile{5,4,2,1}$, then two of the five topmost tasks \emph{have to be scheduled} (since we are using HLF). If one of these two topmost tasks is finished, we reach $\profile{4,4,2,1}$ (see figure \ref{fig:p2-profiles-successors-of-5421-always-same} for reference).

\begin{figure}[ht]
  \centering
  \renewcommand{\leveltopI}{-10cm + \leveltop}
\renewcommand{\leveltopII}{-10cm + \leveltopI}
\renewcommand{\leveltopIII}{-10cm + \leveltopII}
\renewcommand{\leveltopIIII}{-10cm + \leveltopIII}
\renewcommand{\leveltopIIIII}{-10cm + \leveltopIIII}
\renewcommand{\leveltopIIIIII}{-10cm + \leveltopIIIII}
\renewcommand{\leveltopIIIIIII}{-10cm + \leveltopIIIIII}
\renewcommand{\leveltopIIIIIIII}{-10cm + \leveltopIIIIIII}
\renewcommand{\leveltopIIIIIIIII}{-10cm + \leveltopIIIIIIII}
\renewcommand{\leveltopIIIIIIIIII}{-10cm + \leveltopIIIIIIIII}
\renewcommand{\leveltopIIIIIIIIIII}{-10cm + \leveltopIIIIIIIIII}
\renewcommand{\leveltopIIIIIIIIIIII}{-10cm + \leveltopIIIIIIIIIII}
\renewcommand{\leveltopI}{-10cm + \leveltop}
\renewcommand{\leveltopII}{-10cm + \leveltopI}
\renewcommand{\leveltopIII}{-10cm + \leveltopII}
\renewcommand{\leveltopIIII}{-10cm + \leveltopIII}
\renewcommand{\leveltopIIIII}{-10cm + \leveltopIIII}
\renewcommand{\leveltopIIIIII}{-10cm + \leveltopIIIII}
\renewcommand{\leveltopIIIIIII}{-10cm + \leveltopIIIIII}
\renewcommand{\leveltopIIIIIIII}{-10cm + \leveltopIIIIIII}
\renewcommand{\leveltopIIIIIIIII}{-10cm + \leveltopIIIIIIII}
\renewcommand{\leveltopIIIIIIIIII}{-10cm + \leveltopIIIIIIIII}
\renewcommand{\leveltopIIIIIIIIIII}{-10cm + \leveltopIIIIIIIIII}
\renewcommand{\leveltopIIIIIIIIIIII}{-10cm + \leveltopIIIIIIIIIII}
\begin{tikzpicture}[scale=.2, anchor=south]
\begin{scope}[yshift=\leveltopI cm]
\matrix (line1)[column sep=0.1cm] {
\node[draw=black, rectangle split,  rectangle split parts=1] (sn0x9b5d5c8){
\begin{tikzpicture}[scale=.2]
\node[circle, scale=0.75, fill] (tid0) at (4.5,1.5){};
\node[circle, scale=0.75, fill] (tid1) at (3,3){};
\node[circle, scale=0.75, fill] (tid3) at (1.5,4.5){};
\node[circle, scale=0.75, fill, task_scheduled] (tid7) at (0.75,6){};
\node[circle, scale=0.75, fill, task_scheduled] (tid8) at (2.25,6){};
\draw[](tid3) -- (tid7);
\draw[](tid3) -- (tid8);
\node[circle, scale=0.75, fill] (tid4) at (3.75,4.5){};
\node[circle, scale=0.75, fill] (tid9) at (3.75,6){};
\draw[](tid4) -- (tid9);
\node[circle, scale=0.75, fill] (tid5) at (5.25,4.5){};
\draw[](tid1) -- (tid3);
\draw[](tid1) -- (tid4);
\draw[](tid1) -- (tid5);
\node[circle, scale=0.75, fill] (tid2) at (7.5,3){};
\node[circle, scale=0.75, fill] (tid6) at (7.5,4.5){};
\node[circle, scale=0.75, fill] (tid10) at (6.75,6){};
\node[circle, scale=0.75, fill] (tid11) at (8.25,6){};
\draw[](tid6) -- (tid10);
\draw[](tid6) -- (tid11);
\draw[](tid2) -- (tid6);
\draw[](tid0) -- (tid1);
\draw[](tid0) -- (tid2);
\end{tikzpicture}
};
 & 
\\
};
\end{scope}
\begin{scope}[yshift=\leveltopII cm]
\matrix (line2)[column sep=0.1cm] {
\node[draw=black, rectangle split,  rectangle split parts=1] (sn0x9b5de08){
\begin{tikzpicture}[scale=.2]
\node[circle, scale=0.75, fill] (tid0) at (3.75,1.5){};
\node[circle, scale=0.75, fill] (tid1) at (2.25,3){};
\node[circle, scale=0.75, fill] (tid3) at (0.75,4.5){};
\node[circle, scale=0.75, fill, task_scheduled] (tid7) at (0.75,6){};
\draw[](tid3) -- (tid7);
\node[circle, scale=0.75, fill] (tid4) at (2.25,4.5){};
\node[circle, scale=0.75, fill, task_scheduled] (tid8) at (2.25,6){};
\draw[](tid4) -- (tid8);
\node[circle, scale=0.75, fill] (tid5) at (3.75,4.5){};
\draw[](tid1) -- (tid3);
\draw[](tid1) -- (tid4);
\draw[](tid1) -- (tid5);
\node[circle, scale=0.75, fill] (tid2) at (6,3){};
\node[circle, scale=0.75, fill] (tid6) at (6,4.5){};
\node[circle, scale=0.75, fill] (tid9) at (5.25,6){};
\node[circle, scale=0.75, fill] (tid10) at (6.75,6){};
\draw[](tid6) -- (tid9);
\draw[](tid6) -- (tid10);
\draw[](tid2) -- (tid6);
\draw[](tid0) -- (tid1);
\draw[](tid0) -- (tid2);
\end{tikzpicture}
};
 & 
\node[draw=black, rectangle split,  rectangle split parts=1] (sn0x9b58978){
\begin{tikzpicture}[scale=.2]
\node[circle, scale=0.75, fill] (tid0) at (3.75,1.5){};
\node[circle, scale=0.75, fill] (tid1) at (2.25,3){};
\node[circle, scale=0.75, fill] (tid3) at (0.75,4.5){};
\node[circle, scale=0.75, fill, task_scheduled] (tid7) at (0.75,6){};
\draw[](tid3) -- (tid7);
\node[circle, scale=0.75, fill] (tid4) at (2.25,4.5){};
\node[circle, scale=0.75, fill] (tid8) at (2.25,6){};
\draw[](tid4) -- (tid8);
\node[circle, scale=0.75, fill] (tid5) at (3.75,4.5){};
\draw[](tid1) -- (tid3);
\draw[](tid1) -- (tid4);
\draw[](tid1) -- (tid5);
\node[circle, scale=0.75, fill] (tid2) at (6,3){};
\node[circle, scale=0.75, fill] (tid6) at (6,4.5){};
\node[circle, scale=0.75, fill, task_scheduled] (tid9) at (5.25,6){};
\node[circle, scale=0.75, fill] (tid10) at (6.75,6){};
\draw[](tid6) -- (tid9);
\draw[](tid6) -- (tid10);
\draw[](tid2) -- (tid6);
\draw[](tid0) -- (tid1);
\draw[](tid0) -- (tid2);
\end{tikzpicture}
};
 & 
\\
};
\end{scope}
\draw (sn0x9b5d5c8.south) -- (sn0x9b5de08.north);
\draw (sn0x9b5d5c8.south) -- (sn0x9b58978.north);
\end{tikzpicture}
\renewcommand{\leveltopI}{-10cm + \leveltop}
\renewcommand{\leveltopII}{-10cm + \leveltopI}
\renewcommand{\leveltopIII}{-10cm + \leveltopII}
\renewcommand{\leveltopIIII}{-10cm + \leveltopIII}
\renewcommand{\leveltopIIIII}{-10cm + \leveltopIIII}
\renewcommand{\leveltopIIIIII}{-10cm + \leveltopIIIII}
\renewcommand{\leveltopIIIIIII}{-10cm + \leveltopIIIIII}
\renewcommand{\leveltopIIIIIIII}{-10cm + \leveltopIIIIIII}
\renewcommand{\leveltopIIIIIIIII}{-10cm + \leveltopIIIIIIII}
\renewcommand{\leveltopIIIIIIIIII}{-10cm + \leveltopIIIIIIIII}
\renewcommand{\leveltopIIIIIIIIIII}{-10cm + \leveltopIIIIIIIIII}
\renewcommand{\leveltopIIIIIIIIIIII}{-10cm + \leveltopIIIIIIIIIII}
\begin{tikzpicture}[scale=.2, anchor=south]
\begin{scope}[yshift=\leveltopI cm]
\matrix (line1)[column sep=0.1cm] {
\node[draw=black, rectangle split,  rectangle split parts=1] (sn0x9b5ed70){
\begin{tikzpicture}[scale=.2]
\node[circle, scale=0.75, fill] (tid0) at (4.5,1.5){};
\node[circle, scale=0.75, fill] (tid1) at (3,3){};
\node[circle, scale=0.75, fill] (tid3) at (1.5,4.5){};
\node[circle, scale=0.75, fill, task_scheduled] (tid7) at (0.75,6){};
\node[circle, scale=0.75, fill] (tid8) at (2.25,6){};
\draw[](tid3) -- (tid7);
\draw[](tid3) -- (tid8);
\node[circle, scale=0.75, fill] (tid4) at (3.75,4.5){};
\node[circle, scale=0.75, fill, task_scheduled] (tid9) at (3.75,6){};
\draw[](tid4) -- (tid9);
\node[circle, scale=0.75, fill] (tid5) at (5.25,4.5){};
\draw[](tid1) -- (tid3);
\draw[](tid1) -- (tid4);
\draw[](tid1) -- (tid5);
\node[circle, scale=0.75, fill] (tid2) at (7.5,3){};
\node[circle, scale=0.75, fill] (tid6) at (7.5,4.5){};
\node[circle, scale=0.75, fill] (tid10) at (6.75,6){};
\node[circle, scale=0.75, fill] (tid11) at (8.25,6){};
\draw[](tid6) -- (tid10);
\draw[](tid6) -- (tid11);
\draw[](tid2) -- (tid6);
\draw[](tid0) -- (tid1);
\draw[](tid0) -- (tid2);
\end{tikzpicture}
};
 & 
\\
};
\end{scope}
\begin{scope}[yshift=\leveltopII cm]
\matrix (line2)[column sep=0.1cm] {
\node[draw=black, rectangle split,  rectangle split parts=1] (sn0x9b5e2b0){
\begin{tikzpicture}[scale=.2]
\node[circle, scale=0.75, fill] (tid0) at (4.5,1.5){};
\node[circle, scale=0.75, fill] (tid1) at (3,3){};
\node[circle, scale=0.75, fill] (tid3) at (1.5,4.5){};
\node[circle, scale=0.75, fill, task_scheduled] (tid7) at (0.75,6){};
\node[circle, scale=0.75, fill, task_scheduled] (tid8) at (2.25,6){};
\draw[](tid3) -- (tid7);
\draw[](tid3) -- (tid8);
\node[circle, scale=0.75, fill] (tid4) at (3.75,4.5){};
\node[circle, scale=0.75, fill] (tid5) at (5.25,4.5){};
\draw[](tid1) -- (tid3);
\draw[](tid1) -- (tid4);
\draw[](tid1) -- (tid5);
\node[circle, scale=0.75, fill] (tid2) at (7.5,3){};
\node[circle, scale=0.75, fill] (tid6) at (7.5,4.5){};
\node[circle, scale=0.75, fill] (tid9) at (6.75,6){};
\node[circle, scale=0.75, fill] (tid10) at (8.25,6){};
\draw[](tid6) -- (tid9);
\draw[](tid6) -- (tid10);
\draw[](tid2) -- (tid6);
\draw[](tid0) -- (tid1);
\draw[](tid0) -- (tid2);
\end{tikzpicture}
};
 & 
\node[draw=black, rectangle split,  rectangle split parts=1] (sn0x9b5ebe0){
\begin{tikzpicture}[scale=.2]
\node[circle, scale=0.75, fill] (tid0) at (4.5,1.5){};
\node[circle, scale=0.75, fill] (tid1) at (3,3){};
\node[circle, scale=0.75, fill] (tid3) at (1.5,4.5){};
\node[circle, scale=0.75, fill, task_scheduled] (tid7) at (0.75,6){};
\node[circle, scale=0.75, fill] (tid8) at (2.25,6){};
\draw[](tid3) -- (tid7);
\draw[](tid3) -- (tid8);
\node[circle, scale=0.75, fill] (tid4) at (3.75,4.5){};
\node[circle, scale=0.75, fill] (tid5) at (5.25,4.5){};
\draw[](tid1) -- (tid3);
\draw[](tid1) -- (tid4);
\draw[](tid1) -- (tid5);
\node[circle, scale=0.75, fill] (tid2) at (7.5,3){};
\node[circle, scale=0.75, fill] (tid6) at (7.5,4.5){};
\node[circle, scale=0.75, fill, task_scheduled] (tid9) at (6.75,6){};
\node[circle, scale=0.75, fill] (tid10) at (8.25,6){};
\draw[](tid6) -- (tid9);
\draw[](tid6) -- (tid10);
\draw[](tid2) -- (tid6);
\draw[](tid0) -- (tid1);
\draw[](tid0) -- (tid2);
\end{tikzpicture}
};
 & 
\node[draw=black, rectangle split,  rectangle split parts=1] (sn0x9b5de08){
\begin{tikzpicture}[scale=.2]
\node[circle, scale=0.75, fill] (tid0) at (3.75,1.5){};
\node[circle, scale=0.75, fill] (tid1) at (2.25,3){};
\node[circle, scale=0.75, fill] (tid3) at (0.75,4.5){};
\node[circle, scale=0.75, fill, task_scheduled] (tid7) at (0.75,6){};
\draw[](tid3) -- (tid7);
\node[circle, scale=0.75, fill] (tid4) at (2.25,4.5){};
\node[circle, scale=0.75, fill, task_scheduled] (tid8) at (2.25,6){};
\draw[](tid4) -- (tid8);
\node[circle, scale=0.75, fill] (tid5) at (3.75,4.5){};
\draw[](tid1) -- (tid3);
\draw[](tid1) -- (tid4);
\draw[](tid1) -- (tid5);
\node[circle, scale=0.75, fill] (tid2) at (6,3){};
\node[circle, scale=0.75, fill] (tid6) at (6,4.5){};
\node[circle, scale=0.75, fill] (tid9) at (5.25,6){};
\node[circle, scale=0.75, fill] (tid10) at (6.75,6){};
\draw[](tid6) -- (tid9);
\draw[](tid6) -- (tid10);
\draw[](tid2) -- (tid6);
\draw[](tid0) -- (tid1);
\draw[](tid0) -- (tid2);
\end{tikzpicture}
};
 & 
\node[draw=black, rectangle split,  rectangle split parts=1] (sn0x9b58978){
\begin{tikzpicture}[scale=.2]
\node[circle, scale=0.75, fill] (tid0) at (3.75,1.5){};
\node[circle, scale=0.75, fill] (tid1) at (2.25,3){};
\node[circle, scale=0.75, fill] (tid3) at (0.75,4.5){};
\node[circle, scale=0.75, fill, task_scheduled] (tid7) at (0.75,6){};
\draw[](tid3) -- (tid7);
\node[circle, scale=0.75, fill] (tid4) at (2.25,4.5){};
\node[circle, scale=0.75, fill] (tid8) at (2.25,6){};
\draw[](tid4) -- (tid8);
\node[circle, scale=0.75, fill] (tid5) at (3.75,4.5){};
\draw[](tid1) -- (tid3);
\draw[](tid1) -- (tid4);
\draw[](tid1) -- (tid5);
\node[circle, scale=0.75, fill] (tid2) at (6,3){};
\node[circle, scale=0.75, fill] (tid6) at (6,4.5){};
\node[circle, scale=0.75, fill, task_scheduled] (tid9) at (5.25,6){};
\node[circle, scale=0.75, fill] (tid10) at (6.75,6){};
\draw[](tid6) -- (tid9);
\draw[](tid6) -- (tid10);
\draw[](tid2) -- (tid6);
\draw[](tid0) -- (tid1);
\draw[](tid0) -- (tid2);
\end{tikzpicture}
};
 & 
\\
};
\end{scope}
\draw (sn0x9b5ed70.south) -- (sn0x9b5e2b0.north);
\draw (sn0x9b5ed70.south) -- (sn0x9b5ebe0.north);
\draw (sn0x9b5ed70.south) -- (sn0x9b5de08.north);
\draw (sn0x9b5ed70.south) -- (sn0x9b58978.north);
\end{tikzpicture}
%%% Local Variables:
%%% TeX-master: "thesis/thesis.tex"
%%% End: 

  \caption{Intree with profile $\profile{5,4,2,1}$. \emph{All} possible HLF-successors of the original intree have profile $\profile{4,4,2,1}$.}
  \label{fig:p2-profiles-successors-of-5421-always-same}
\end{figure}

\todo{More figures.}

Another interesting case is $\profile{1,5,2,1}$, where the (single) topmost task and one of the five tasks on the second level are scheduled. If the topmost task is finished (which happens with probability $\frac{1}{2}$), we reach $\profile{5,2,1}$. If the scheduled task on the second level finishes first, we reach $\profile{1,4,2,1}$.

The last example we want to give here is $\profile{1,1,1,3,1}$. In this case, the single topmost task and one of the three tasks of the second lowest level have to be scheduled. If the topmost task finishes first (which happens with probability $\frac{1}{2}$), the resulting profile will be $\profile{1,1,3,1}$ (where again the topmost task and one of the tree tasks in the second lowest level are scheduled). If the other scheduled task finishes first, we reach $\profile{1,1,1,2,1}$, where the single topmost task and one of the remaining \emph{two} tasks on the second lowest level are scheduled.

\section{Expected runtime for two processor HLF using profiles}
\label{sec:p2-profiles-hlf-exp-runtime}

We now use the profile notation to denote the expected run time (i.e. we say $\E{\profile{6,3,1}} = \frac{49}{8}$ --- see figure \ref{fig:p2-four-intrees-with-same-profile-6-3-1}).

\subsection{A recursive definition}
\label{sec:p2-profile-exp-run-time-rec-def}

Exploiting profile notation, we can define the following recursive formula that can be used to compute the optimal expected run time:

\begin{equation}
  \label{eq:p2-profile-optimal-exp-run-time}
  \E{\profile{n_1, \dots, n_r}} =
  \begin{cases}
    r, & \text{ if } n_1 = n_2 = \dots = n_r = 1 \\
    \frac{n_1-1}{2} + \E{\profile{1, n_2, n_3, \dots, n_r}} , & \text{ if } n_r\geq 2 \\
    \frac{1}{2} + \frac{1}{2} \cdot \left( \E{\profile{n_2, \dots, n_r}} + \E{SUC(\profile{n_1,\dots,n_r})} \right) ,& \text{ otherwise }
  \end{cases},
\end{equation}
where $SUC(\profile{n_1,\dots,n_r}) = \profile{n_1, n_2, n_3,\dots,n_{j-1},n_j-1,n_{j+1},\dots,n_r}$ such that $j$ is the minimum index such that $n_j>1$.

If we consider the second case of equation (\ref{eq:p2-profile-optimal-exp-run-time}), we see that we can simplify it to the following:

\begin{equation}
  \label{eq:p2-profile-optimal-exp-run-time-def-simplified}
  \E{\profile{n_1, \dots, n_r}} =
  \begin{cases}
    r, & \text{ if } n_1 = n_2 = \dots = n_r = 1 \\
    \frac{n_1}{2} + \frac{1}{2} \cdot \left( \E{\profile{n_2, \dots, n_r}} + \E{SUC(\profile{1,n_2,\dots,n_r})} \right) ,& \text{ otherwise }
  \end{cases},
\end{equation}
with $SUC$ as defined before.

\subsection{Solving the recurrence for special cases}
\label{sec:p2-profile-exp-runtime-closed-form-spec-cases}

Unfortunately, the recurrence relation in equation (\ref{eq:p2-profile-optimal-exp-run-time-def-simplified}) does not significantly simplify the original problem. However, we were able to deduce a closed form that can be used for special cases.

\begin{theorem}
  \label{theo:simple-profiles-exp-runtime-for-p2-hlf}
  Let $\profile{n_1,\profileones{j-2},n_j,\profileones{r-j}}$ be a profile 
  %where 
  %$\left|\left\{ i \in \left\{ 2,3,\dots,r \right\} \mid n_i > 1 \right\}\right| \leq 1$ 
  (i.e. at most the first and one other entry of the profile are different from 1).
  Then it holds that
  \begin{equation*}
    \E{\profile{n_1,\profileones{j-2},n_j,\profileones{r-j}}} = 
    r + \frac{A_0(n_1-2)}{2^{n_1-1}} + \frac{A_{j-1}(n_j-2)}{2^{n_j+j-2}},
  \end{equation*}
  % Then we can compute 
  % \begin{equation*}
  %   \profile{n_1,n_2,\dots,n_r} = 
  %   r + \sum_{i=1}^r \left( \frac{A_{i-1}(n_i - 2)}{2^{n_i+i-2}} \right),
  % \end{equation*}
  where $A_i$ is inductively defined as follows:
  \begin{align*}
    A_0(n) & = (n+1) \cdot 2^n \\
    A_{i+1}(n) & = \sum_{k=0}^n A_{i}(k)
  \end{align*}
\end{theorem}

\begin{table}
  \centering
  \begin{tabular}[ht]{ccccccccl}
    $n$ & -1 & 0 & 1 & 2 & 3 & 4 & 5 & Closed term \\
    \hline
    $A_0(n)$ & 0 & 1 & 4 & 12 & 32 & 80 & 192 & 
    $(n+1)\cdot 2^{n}$ \\
    $A_1(n)$ & 0 & 1 & 5 & 17 & 49 & 129 & 321 & 
    $n\cdot 2^{n+1} + 1$ \\
    $A_2(n)$ & 0 & 1 & 6 & 23 & 72 & 201& 522 & 
    $(n-1)\cdot 2^{n+2}+n+5$ \\
    $A_3(n)$ & 0 & 1 & 7 & 30 & 102 & 303 & 825 & 
    $(n-2)\cdot 2^{n+3}+(n^2+11 n+34)/2$ \\
    $A_4(n)$ & 0 & 1 & 8 & 38 & 140 & 443 & 1268 &
    $(n-3)\cdot2^{n+4}+\binom{n+3}{3}+4\cdot\left(\binom{n+1}{2}+4 n+12\right)$ \\
  \end{tabular}
  \caption{Example values for $A_i(n)$. \todo{OEIS zitieren.}}
  \label{tab:example-values-an-p2-profile}
\end{table}

Before we prove theorem \ref{theo:simple-profiles-exp-runtime-for-p2-hlf}, let us have a look at table \ref{tab:example-values-an-p2-profile} showing values for $A_i(n)$ for small values of $i$ and $n$.
%$(i,n) \in \left( \left\{ 0,1,\dots,4 \right\} \times \left\{ -1,0,1,2,\dots,5 \right\} \right)$. 

From this table and by looking at the definition of $A_i(n)$ we can deduce a simple lemma that will later be useful.

Note that there are closed expressions for $A_i(n)$ for $i\leq 5$ (and possibly for higher values of $i$, as well). However, these formulae are quite complex (also see table \ref{tab:example-values-an-p2-profile}) and we were not able to deduce a \emph{simple} pattern according to which $A_i(n)$ can be constructed\todo[Hr. Mayr]{Sagen Ihnen die Polynombestandteile in Tabelle \ref{tab:example-values-an-p2-profile} etwas?}. It seems that $A_i(n)$ involves the term $(n+1-i)\cdot 2^{n+i}$ in some way, and the remaining term seems to be a polynomial in $n$.

\begin{lemma}
  \label{lemma:p2-hlf-profiles-an-simple-recurrence}
  Let $A_i(n)$ be as defined in theorem \ref{theo:simple-profiles-exp-runtime-for-p2-hlf}. Then, we have
  \begin{equation*}
    A_{j-1}(n) + A_{j}(n-1) = A_{j}(n).
  \end{equation*}
\end{lemma}

\begin{proof}
  Proof is trivial by definition of $A_{j}(n) = \sum_{k=0}^{n} A_{j-1}(k) = A_{j-1}(n) + \sum_{k=0}^{n-1} A_{j-1}(k) = A_{j-1}(n) + A_{j}(n-1)$.
\end{proof}

We can now proof theorem \ref{theo:simple-profiles-exp-runtime-for-p2-hlf}.

\begin{proof}[Proof of theorem \ref{theo:simple-profiles-exp-runtime-for-p2-hlf}]
  We prove theorem \ref{theo:simple-profiles-exp-runtime-for-p2-hlf} by complete induction. The base case $\E{\profile{\profileones{r}}} = r$ is clear because in this case we can always schedule exactly one task. Since there are $r$ tasks in total, this results in an expected run time of $r$ (each task is expected to be exponentially distributed with expectation 1).

  We now consider the special cases where \emph{all elements but one} in the profile are 1. That is, we consider the profile, whose elements are all 1, exctept the element at position $j$, which will be $n$. That is, we examine
  \begin{equation*}
    \profile{\profileones{j-1},n,\profileones{r-j}}.
  \end{equation*}
  We can rewrite this using the definition and afterwards apply the induction hypothesis:

  \begin{eqnarray*}
    \E{\profile{\profileones{j-1},n,\profileones{r-j}}}
    & = & 
    \frac{1}{2} + \frac{1}{2} \cdot 
    \left( 
      \E{\profile{\profileones{j-2},n,\profileones{r-j}}} + 
      \E{\profile{\profileones{j-1},n-1,\profileones{r-j}}}
    \right) = \\
    & = & 
    \frac{1}{2} + \frac{1}{2} \cdot 
    \left( 
      (r-1) + \frac{A_{j-2}(n-2)}{2^{n+(j-1)-2}} +
      r + \frac{A_{j-1}(n-3)}{2^{(n-1)+j-2}}
    \right)
  \end{eqnarray*}
  We now apply lemma Lemma \ref{lemma:p2-hlf-profiles-an-simple-recurrence} and obtain
  \begin{eqnarray*}
    \profile{\profileones{j-1},n,\profileones{r-j}}
    & = & 
    \frac{1}{2} + \frac{1}{2} \cdot 
    \left( 
      (r-1)+r + 
      \frac{A_{j-2}(n-2) + A_{j-1}(n-3)}{2^{n+j-3}}
    \right) = \\
    & = &
    \frac{1}{2} + \frac{1}{2} \cdot 
    \left( 
      2r - 1
      \frac{A_{j-1}(n-2)}{2^{n+j-3}}
    \right) = \\
    & = &
    \frac{1}{2} + 
    r - \frac{1}{2}
    \frac{A_{j-1}(n-2)}{2^{n+j-2}} = \\
    & = &
    r + \frac{A_{j-1}(n-2)}{2^{n+j-2}}
  \end{eqnarray*}

  We conclude the proof by deriving the expected run time for $\profile{m,\profileones{j-2},n,\profileones{r-j}}$. We do this by applying the definition and thereby reducing the problem to $\profile{\profileones{j-1},n,\profileones{r-j}}$:

  \begin{eqnarray*}
    \E{\profile{m,\profileones{j-2},n,\profileones{r-j}}}
    & = & 
    \frac{m-1}{2} + \E{\profile{\profileones{j-1},n,\profileones{r-j}}} = \\
    & = &
    \frac{m-1}{2} + r + \frac{A_{j-1}(n-2)}{2^{n+j-2}} = \\
    & = &
    \frac{(m-1)\cdot 2^{m-2}}{2^{m-1}} + r + \frac{A_{j-1}(n-2)}{2^{n+j-2}} = \\
    & = &
    \frac{A_0(m-2)}{2^{m-1}} + r + \frac{A_{j-1}(n-2)}{2^{n+j-2}}
  \end{eqnarray*}
  
  This concludes the proof.
\end{proof}

Moritz Maaß has shown another property in \cite{MoritzMaasDiploma}, that we are able to generalize.

\begin{theorem}[Intrees with exactly two leaves and intrees with same profiles]
  Let $l, k\in\naturals$, $a\in\naturals_0$ and $\profile{\profilerepeat{1}{l-k}, \profilerepeat{2}{k}, \profilerepeat{1}{a+1}}$ be a profile. Then, it holds that
  \begin{eqnarray*}
    \E{\profile{\profilerepeat{1}{l-k}, \profilerepeat{2}{k}, \profilerepeat{1}{a+1}}}
    = & &
    % the following is an incorrect simplification of Maple
    % a
    % + 2 
    % - \frac{\binom{l+k-1}{k} + \binom{l+k-1}{l}}{2^{l+k}}
    % +  \sum_{i=1}^k \sum_{j=1}^l \left( \frac{1}{2} \right)^{k-i+l-j+1}\cdot \binom{k-i+l-j}{l-j}
    \sum_{i=1}^k \left(\frac{1}{2}\right)^{l+i-1} \cdot \binom{l+i-2}{i-1} \cdot \left( k-i+2 \right) \\
    & + & \sum_{j=1}^l \left(\frac{1}{2}\right)^{k+j-1} \cdot \binom{k+j-2}{j-1} \cdot \left( l-j+2 \right) \\
    & + & \sum_{i=1}^k \sum_{j=1}^l \left( \frac{1}{2}^{k-i+l-j+1}\cdot\binom{ki+l-j}{l-j} \right) \\
    & + & a
    .
  \end{eqnarray*}
\end{theorem}

\begin{proof}
  \todo{Proof.}
\end{proof}

Even if we were not able to deduce a more general formula that holds if more entries in the profile differ from 1, this might serve as a starting point for a more advanced proof.

\todo{Tabelle, die einige Beispiele von Profiles zeigt und deren Erwartungswerte.}

\section{Profile DAGs}
\label{sec:p2-profile-dags}

As seen in the previous chapter, a closed formula for $\E{\profile{n_1,\dots,n_r}}$ seems to be quite complex. This is why we may compute the expected runtime just with the recursive approach given in equation (\ref{eq:p2-profile-optimal-exp-run-time-def-simplified}).

Of course, it is an interesting question how complex this computation is. Therefore, we consider the \emph{profile DAG}. The profile DAG is -- intuitively -- a coarsening of the original snapshot DAG. It is created the following way: We ``merge'' snapshots having the same profile, thereby decreasing the number of vertices in the DAG. Figure \ref{fig:p2-profile-dag-example-000111223-hlfdet} shows a snapshot DAG and its corresponding profile DAG.

\begin{figure}[t]
  \centering
  \renewcommand{\leveltopI}{-11cm + \leveltop}
\renewcommand{\leveltopII}{-11cm + \leveltopI}
\renewcommand{\leveltopIII}{-11cm + \leveltopII}
\renewcommand{\leveltopIIII}{-10cm + \leveltopIII}
\renewcommand{\leveltopIIIII}{-10cm + \leveltopIIII}
\renewcommand{\leveltopIIIIII}{-10cm + \leveltopIIIII}
\renewcommand{\leveltopIIIIIII}{-10cm + \leveltopIIIIII}
\renewcommand{\leveltopIIIIIIII}{-9cm + \leveltopIIIIIII}
\renewcommand{\leveltopIIIIIIIII}{-8cm + \leveltopIIIIIIII}
\renewcommand{\leveltopIIIIIIIIII}{-7cm + \leveltopIIIIIIIII}
\begin{tikzpicture}[scale=.2, anchor=south, rotate=90]
\begin{scope}[yshift=\leveltopI cm, anchor = center]
\matrix (line1)[row sep=0.5cm] {
\node[draw=black, rectangle split,  rectangle split parts=2] (sn0x1c22990){
\nodepart{one}
\begin{tikzpicture}[scale=.2]
\node[circle, scale=0.75, fill] (tid0) at (4.5,1.5){};
\node[circle, scale=0.75, fill] (tid1) at (2.25,3){};
\node[circle, scale=0.75, fill] (tid4) at (0.75,4.5){};
\node[circle, scale=0.75, fill] (tid5) at (2.25,4.5){};
\node[circle, scale=0.75, fill] (tid6) at (3.75,4.5){};
\draw[](tid1) -- (tid4);
\draw[](tid1) -- (tid5);
\draw[](tid1) -- (tid6);
\node[circle, scale=0.75, fill] (tid2) at (6,3){};
\node[circle, scale=0.75, fill, task_scheduled] (tid7) at (5.25,4.5){};
\node[circle, scale=0.75, fill] (tid8) at (6.75,4.5){};
\draw[](tid2) -- (tid7);
\draw[](tid2) -- (tid8);
\node[circle, scale=0.75, fill] (tid3) at (8.25,3){};
\node[circle, scale=0.75, fill, task_scheduled] (tid9) at (8.25,4.5){};
\draw[](tid3) -- (tid9);
\draw[](tid0) -- (tid1);
\draw[](tid0) -- (tid2);
\draw[](tid0) -- (tid3);
\end{tikzpicture}
\nodepart{two}
\footnotesize{6.125}
};
 \\ 
\\
};
\end{scope}
\begin{scope}[yshift=\leveltopII cm, anchor = center]
\matrix (line2)[row sep=0.5cm] {
\node[draw=black, rectangle split,  rectangle split parts=2] (sn0x1c29d30){
\nodepart{one}
\begin{tikzpicture}[scale=.2]
\node[circle, scale=0.75, fill] (tid0) at (4.5,1.5){};
\node[circle, scale=0.75, fill] (tid1) at (2.25,3){};
\node[circle, scale=0.75, fill, task_scheduled] (tid4) at (0.75,4.5){};
\node[circle, scale=0.75, fill] (tid5) at (2.25,4.5){};
\node[circle, scale=0.75, fill] (tid6) at (3.75,4.5){};
\draw[](tid1) -- (tid4);
\draw[](tid1) -- (tid5);
\draw[](tid1) -- (tid6);
\node[circle, scale=0.75, fill] (tid2) at (6,3){};
\node[circle, scale=0.75, fill, task_scheduled] (tid7) at (5.25,4.5){};
\node[circle, scale=0.75, fill] (tid8) at (6.75,4.5){};
\draw[](tid2) -- (tid7);
\draw[](tid2) -- (tid8);
\node[circle, scale=0.75, fill] (tid3) at (8.25,3){};
\draw[](tid0) -- (tid1);
\draw[](tid0) -- (tid2);
\draw[](tid0) -- (tid3);
\end{tikzpicture}
\nodepart{two}
\footnotesize{5.625}
};
 \\ 
\node[draw=black, rectangle split,  rectangle split parts=2] (sn0x1c28c10){
\nodepart{one}
\begin{tikzpicture}[scale=.2]
\node[circle, scale=0.75, fill] (tid0) at (3.75,1.5){};
\node[circle, scale=0.75, fill] (tid1) at (2.25,3){};
\node[circle, scale=0.75, fill, task_scheduled] (tid4) at (0.75,4.5){};
\node[circle, scale=0.75, fill] (tid5) at (2.25,4.5){};
\node[circle, scale=0.75, fill] (tid6) at (3.75,4.5){};
\draw[](tid1) -- (tid4);
\draw[](tid1) -- (tid5);
\draw[](tid1) -- (tid6);
\node[circle, scale=0.75, fill] (tid2) at (5.25,3){};
\node[circle, scale=0.75, fill, task_scheduled] (tid7) at (5.25,4.5){};
\draw[](tid2) -- (tid7);
\node[circle, scale=0.75, fill] (tid3) at (6.75,3){};
\node[circle, scale=0.75, fill] (tid8) at (6.75,4.5){};
\draw[](tid3) -- (tid8);
\draw[](tid0) -- (tid1);
\draw[](tid0) -- (tid2);
\draw[](tid0) -- (tid3);
\end{tikzpicture}
\nodepart{two}
\footnotesize{5.625}
};
 \\ 
\\
};
\end{scope}
\begin{scope}[yshift=\leveltopIII cm, anchor = center]
\matrix (line3)[row sep=0.5cm] {
\node[draw=black, rectangle split,  rectangle split parts=2] (sn0x1c29050){
\nodepart{one}
\begin{tikzpicture}[scale=.2]
\node[circle, scale=0.75, fill] (tid0) at (3.75,1.5){};
\node[circle, scale=0.75, fill] (tid1) at (2.25,3){};
\node[circle, scale=0.75, fill, task_scheduled] (tid4) at (0.75,4.5){};
\node[circle, scale=0.75, fill, task_scheduled] (tid5) at (2.25,4.5){};
\node[circle, scale=0.75, fill] (tid6) at (3.75,4.5){};
\draw[](tid1) -- (tid4);
\draw[](tid1) -- (tid5);
\draw[](tid1) -- (tid6);
\node[circle, scale=0.75, fill] (tid2) at (5.25,3){};
\node[circle, scale=0.75, fill] (tid7) at (5.25,4.5){};
\draw[](tid2) -- (tid7);
\node[circle, scale=0.75, fill] (tid3) at (6.75,3){};
\draw[](tid0) -- (tid1);
\draw[](tid0) -- (tid2);
\draw[](tid0) -- (tid3);
\end{tikzpicture}
\nodepart{two}
\footnotesize{5.125}
};
 \\ 
\node[draw=black, rectangle split,  rectangle split parts=2] (sn0x1c29fa0){
\nodepart{one}
\begin{tikzpicture}[scale=.2]
\node[circle, scale=0.75, fill] (tid0) at (3.75,1.5){};
\node[circle, scale=0.75, fill] (tid1) at (1.5,3){};
\node[circle, scale=0.75, fill, task_scheduled] (tid4) at (0.75,4.5){};
\node[circle, scale=0.75, fill] (tid5) at (2.25,4.5){};
\draw[](tid1) -- (tid4);
\draw[](tid1) -- (tid5);
\node[circle, scale=0.75, fill] (tid2) at (4.5,3){};
\node[circle, scale=0.75, fill, task_scheduled] (tid6) at (3.75,4.5){};
\node[circle, scale=0.75, fill] (tid7) at (5.25,4.5){};
\draw[](tid2) -- (tid6);
\draw[](tid2) -- (tid7);
\node[circle, scale=0.75, fill] (tid3) at (6.75,3){};
\draw[](tid0) -- (tid1);
\draw[](tid0) -- (tid2);
\draw[](tid0) -- (tid3);
\end{tikzpicture}
\nodepart{two}
\footnotesize{5.125}
};
 \\ 
\node[draw=black, rectangle split,  rectangle split parts=2] (sn0x1c23690){
\nodepart{one}
\begin{tikzpicture}[scale=.2]
\node[circle, scale=0.75, fill] (tid0) at (3,1.5){};
\node[circle, scale=0.75, fill] (tid1) at (1.5,3){};
\node[circle, scale=0.75, fill, task_scheduled] (tid4) at (0.75,4.5){};
\node[circle, scale=0.75, fill] (tid5) at (2.25,4.5){};
\draw[](tid1) -- (tid4);
\draw[](tid1) -- (tid5);
\node[circle, scale=0.75, fill] (tid2) at (3.75,3){};
\node[circle, scale=0.75, fill, task_scheduled] (tid6) at (3.75,4.5){};
\draw[](tid2) -- (tid6);
\node[circle, scale=0.75, fill] (tid3) at (5.25,3){};
\node[circle, scale=0.75, fill] (tid7) at (5.25,4.5){};
\draw[](tid3) -- (tid7);
\draw[](tid0) -- (tid1);
\draw[](tid0) -- (tid2);
\draw[](tid0) -- (tid3);
\end{tikzpicture}
\nodepart{two}
\footnotesize{5.125}
};
 \\ 
\\
};
\end{scope}
\begin{scope}[yshift=\leveltopIIII cm, anchor = center]
\matrix (line4)[row sep=0.5cm] {
\node[draw=black, rectangle split,  rectangle split parts=2] (sn0x1c23270){
\nodepart{one}
\begin{tikzpicture}[scale=.2]
\node[circle, scale=0.75, fill] (tid0) at (3,1.5){};
\node[circle, scale=0.75, fill] (tid1) at (1.5,3){};
\node[circle, scale=0.75, fill, task_scheduled] (tid4) at (0.75,4.5){};
\node[circle, scale=0.75, fill, task_scheduled] (tid5) at (2.25,4.5){};
\draw[](tid1) -- (tid4);
\draw[](tid1) -- (tid5);
\node[circle, scale=0.75, fill] (tid2) at (3.75,3){};
\node[circle, scale=0.75, fill] (tid6) at (3.75,4.5){};
\draw[](tid2) -- (tid6);
\node[circle, scale=0.75, fill] (tid3) at (5.25,3){};
\draw[](tid0) -- (tid1);
\draw[](tid0) -- (tid2);
\draw[](tid0) -- (tid3);
\end{tikzpicture}
\nodepart{two}
\footnotesize{4.652}
};
 \\ 
\node[draw=black, rectangle split,  rectangle split parts=2] (sn0x1c28250){
\nodepart{one}
\begin{tikzpicture}[scale=.2]
\node[circle, scale=0.75, fill] (tid0) at (3,1.5){};
\node[circle, scale=0.75, fill] (tid1) at (1.5,3){};
\node[circle, scale=0.75, fill, task_scheduled] (tid4) at (0.75,4.5){};
\node[circle, scale=0.75, fill] (tid5) at (2.25,4.5){};
\draw[](tid1) -- (tid4);
\draw[](tid1) -- (tid5);
\node[circle, scale=0.75, fill] (tid2) at (3.75,3){};
\node[circle, scale=0.75, fill, task_scheduled] (tid6) at (3.75,4.5){};
\draw[](tid2) -- (tid6);
\node[circle, scale=0.75, fill] (tid3) at (5.25,3){};
\draw[](tid0) -- (tid1);
\draw[](tid0) -- (tid2);
\draw[](tid0) -- (tid3);
\end{tikzpicture}
\nodepart{two}
\footnotesize{4.652}
};
 \\ 
\node[draw=black, rectangle split,  rectangle split parts=2] (sn0x1c23860){
\nodepart{one}
\begin{tikzpicture}[scale=.2]
\node[circle, scale=0.75, fill] (tid0) at (2.25,1.5){};
\node[circle, scale=0.75, fill] (tid1) at (0.75,3){};
\node[circle, scale=0.75, fill, task_scheduled] (tid4) at (0.75,4.5){};
\draw[](tid1) -- (tid4);
\node[circle, scale=0.75, fill] (tid2) at (2.25,3){};
\node[circle, scale=0.75, fill, task_scheduled] (tid5) at (2.25,4.5){};
\draw[](tid2) -- (tid5);
\node[circle, scale=0.75, fill] (tid3) at (3.75,3){};
\node[circle, scale=0.75, fill] (tid6) at (3.75,4.5){};
\draw[](tid3) -- (tid6);
\draw[](tid0) -- (tid1);
\draw[](tid0) -- (tid2);
\draw[](tid0) -- (tid3);
\end{tikzpicture}
\nodepart{two}
\footnotesize{4.652}
};
 \\ 
\\
};
\end{scope}
\begin{scope}[yshift=\leveltopIIIII cm, anchor = center]
\matrix (line5)[row sep=0.5cm] {
\node[draw=black, rectangle split,  rectangle split parts=2] (sn0x1c285b0){
\nodepart{one}
\begin{tikzpicture}[scale=.2]
\node[circle, scale=0.75, fill] (tid0) at (3,1.5){};
\node[circle, scale=0.75, fill] (tid1) at (1.5,3){};
\node[circle, scale=0.75, fill, task_scheduled] (tid4) at (0.75,4.5){};
\node[circle, scale=0.75, fill, task_scheduled] (tid5) at (2.25,4.5){};
\draw[](tid1) -- (tid4);
\draw[](tid1) -- (tid5);
\node[circle, scale=0.75, fill] (tid2) at (3.75,3){};
\node[circle, scale=0.75, fill] (tid3) at (5.25,3){};
\draw[](tid0) -- (tid1);
\draw[](tid0) -- (tid2);
\draw[](tid0) -- (tid3);
\end{tikzpicture}
\nodepart{two}
\footnotesize{4.125}
};
 \\ 
\node[draw=black, rectangle split,  rectangle split parts=2] (sn0x1c23d40){
\nodepart{one}
\begin{tikzpicture}[scale=.2]
\node[circle, scale=0.75, fill] (tid0) at (2.25,1.5){};
\node[circle, scale=0.75, fill] (tid1) at (0.75,3){};
\node[circle, scale=0.75, fill, task_scheduled] (tid4) at (0.75,4.5){};
\draw[](tid1) -- (tid4);
\node[circle, scale=0.75, fill] (tid2) at (2.25,3){};
\node[circle, scale=0.75, fill, task_scheduled] (tid5) at (2.25,4.5){};
\draw[](tid2) -- (tid5);
\node[circle, scale=0.75, fill] (tid3) at (3.75,3){};
\draw[](tid0) -- (tid1);
\draw[](tid0) -- (tid2);
\draw[](tid0) -- (tid3);
\end{tikzpicture}
\nodepart{two}
\footnotesize{4.125}
};
 \\ 
\\
};
\end{scope}
\begin{scope}[yshift=\leveltopIIIIII cm, anchor = center]
\matrix (line6)[row sep=0.5cm] {
\node[draw=black, rectangle split,  rectangle split parts=2] (sn0x1c23fc0){
\nodepart{one}
\begin{tikzpicture}[scale=.2]
\node[circle, scale=0.75, fill] (tid0) at (2.25,1.5){};
\node[circle, scale=0.75, fill] (tid1) at (0.75,3){};
\node[circle, scale=0.75, fill, task_scheduled] (tid4) at (0.75,4.5){};
\draw[](tid1) -- (tid4);
\node[circle, scale=0.75, fill, task_scheduled] (tid2) at (2.25,3){};
\node[circle, scale=0.75, fill] (tid3) at (3.75,3){};
\draw[](tid0) -- (tid1);
\draw[](tid0) -- (tid2);
\draw[](tid0) -- (tid3);
\end{tikzpicture}
\nodepart{two}
\footnotesize{3.625}
};
 \\ 
\\
};
\end{scope}
\begin{scope}[yshift=\leveltopIIIIIII cm, anchor = center]
\matrix (line7)[row sep=0.5cm] {
\node[draw=black, rectangle split,  rectangle split parts=2] (sn0x1c240d0){
\nodepart{one}
\begin{tikzpicture}[scale=.2]
\node[circle, scale=0.75, fill] (tid0) at (1.5,1.5){};
\node[circle, scale=0.75, fill] (tid1) at (0.75,3){};
\node[circle, scale=0.75, fill, task_scheduled] (tid3) at (0.75,4.5){};
\draw[](tid1) -- (tid3);
\node[circle, scale=0.75, fill, task_scheduled] (tid2) at (2.25,3){};
\draw[](tid0) -- (tid1);
\draw[](tid0) -- (tid2);
\end{tikzpicture}
\nodepart{two}
\footnotesize{3.25}
};
 \\ 
\node[draw=black, rectangle split,  rectangle split parts=2] (sn0x1c241e0){
\nodepart{one}
\begin{tikzpicture}[scale=.2]
\node[circle, scale=0.75, fill] (tid0) at (2.25,1.5){};
\node[circle, scale=0.75, fill, task_scheduled] (tid1) at (0.75,3){};
\node[circle, scale=0.75, fill, task_scheduled] (tid2) at (2.25,3){};
\node[circle, scale=0.75, fill] (tid3) at (3.75,3){};
\draw[](tid0) -- (tid1);
\draw[](tid0) -- (tid2);
\draw[](tid0) -- (tid3);
\end{tikzpicture}
\nodepart{two}
\footnotesize{3}
};
 \\ 
\\
};
\end{scope}
\begin{scope}[yshift=\leveltopIIIIIIII cm, anchor = center]
\matrix (line8)[row sep=0.5cm] {
\node[draw=black, rectangle split,  rectangle split parts=2] (sn0x1c242f0){
\nodepart{one}
\begin{tikzpicture}[scale=.2]
\node[circle, scale=0.75, fill] (tid0) at (0.75,1.5){};
\node[circle, scale=0.75, fill] (tid1) at (0.75,3){};
\node[circle, scale=0.75, fill, task_scheduled] (tid2) at (0.75,4.5){};
\draw[](tid1) -- (tid2);
\draw[](tid0) -- (tid1);
\end{tikzpicture}
\nodepart{two}
\footnotesize{3}
};
 \\ 
\node[draw=black, rectangle split,  rectangle split parts=2] (sn0x1c245b0){
\nodepart{one}
\begin{tikzpicture}[scale=.2]
\node[circle, scale=0.75, fill] (tid0) at (1.5,1.5){};
\node[circle, scale=0.75, fill, task_scheduled] (tid1) at (0.75,3){};
\node[circle, scale=0.75, fill, task_scheduled] (tid2) at (2.25,3){};
\draw[](tid0) -- (tid1);
\draw[](tid0) -- (tid2);
\end{tikzpicture}
\nodepart{two}
\footnotesize{2.5}
};
 \\ 
\\
};
\end{scope}
\draw (sn0x1c22990.east) -- (sn0x1c28c10.west);
\draw (sn0x1c22990.east) -- (sn0x1c29d30.west);
\draw (sn0x1c29d30.east) -- (sn0x1c29fa0.west);
\draw (sn0x1c29d30.east) -- (sn0x1c29050.west);
\draw (sn0x1c28c10.east) -- (sn0x1c23690.west);
\draw (sn0x1c28c10.east) -- (sn0x1c29050.west);
\draw (sn0x1c29050.east) -- (sn0x1c23270.west);
\draw (sn0x1c29fa0.east) -- (sn0x1c28250.west);
\draw (sn0x1c29fa0.east) -- (sn0x1c23270.west);
\draw (sn0x1c23690.east) -- (sn0x1c23860.west);
\draw (sn0x1c23690.east) -- (sn0x1c23270.west);
\draw (sn0x1c23270.east) -- (sn0x1c23d40.west);
\draw (sn0x1c28250.east) -- (sn0x1c23d40.west);
\draw (sn0x1c28250.east) -- (sn0x1c285b0.west);
\draw (sn0x1c23860.east) -- (sn0x1c23d40.west);
\draw (sn0x1c285b0.east) -- (sn0x1c23fc0.west);
\draw (sn0x1c23d40.east) -- (sn0x1c23fc0.west);
\draw (sn0x1c23fc0.east) -- (sn0x1c240d0.west);
\draw (sn0x1c23fc0.east) -- (sn0x1c241e0.west);
\draw (sn0x1c240d0.east) -- (sn0x1c242f0.west);
\draw (sn0x1c240d0.east) -- (sn0x1c245b0.west);
\draw (sn0x1c241e0.east) -- (sn0x1c245b0.west);
\end{tikzpicture}
%% profile
\begin{tikzpicture}[scale=.2, anchor=south, rotate=90]
\begin{scope}[yshift=\leveltopI cm, anchor = center]
\matrix (line1)[row sep=0.1cm] {
\node[draw=black, rectangle split,  rectangle split parts=2] (sn0x1c22990){
\nodepart{one}
\begin{tikzpicture}[scale=.2]
\node[rectangle, scale=0.75] at (0, 0) {$\profile{6, 3, 1}$};
\end{tikzpicture}
\nodepart{two}
\footnotesize{6.125}
};
 \\ 
\\
};
\end{scope}
\begin{scope}[yshift=\leveltopII cm, anchor = center]
\matrix (line2)[row sep=0.1cm] {
\node[draw=black, rectangle split,  rectangle split parts=2] (sn0x1c28c10){
\nodepart{one}
\begin{tikzpicture}[scale=.2]
\node[rectangle, scale=0.75] at (0, 0) {$\profile{5, 3, 1}$};
\end{tikzpicture}
\nodepart{two}
\footnotesize{5.625}
};
 \\ 
\\
};
\end{scope}
\begin{scope}[yshift=\leveltopIII cm, anchor = center]
\matrix (line3)[row sep=0.1cm] {
\node[draw=black, rectangle split,  rectangle split parts=2] (sn0x1c29fa0){
\nodepart{one}
\begin{tikzpicture}[scale=.2]
\node[rectangle, scale=0.75] at (0, 0) {$\profile{4, 3, 1}$};
\end{tikzpicture}
\nodepart{two}
\footnotesize{5.125}
};
 \\ 
\\
};
\end{scope}
\begin{scope}[yshift=\leveltopIIII cm, anchor = center]
\matrix (line4)[row sep=0.1cm] {
\node[draw=black, rectangle split,  rectangle split parts=2] (sn0x1c28250){
\nodepart{one}
\begin{tikzpicture}[scale=.2]
\node[rectangle, scale=0.75] at (0, 0) {$\profile{3, 3, 1}$};
\end{tikzpicture}
\nodepart{two}
\footnotesize{4.625}
};
 \\ 
\\
};
\end{scope}
\begin{scope}[yshift=\leveltopIIIII cm, anchor = center]
\matrix (line5)[row sep=0.1cm] {
\node[draw=black, rectangle split,  rectangle split parts=2] (sn0x1c23d40){
\nodepart{one}
\begin{tikzpicture}[scale=.2]
\node[rectangle, scale=0.75] at (0, 0) {$\profile{2, 3, 1}$};
\end{tikzpicture}
\nodepart{two}
\footnotesize{4.125}
};
 \\ 
\\
};
\end{scope}
\begin{scope}[yshift=\leveltopIIIIII cm, anchor = center]
\matrix (line6)[row sep=0.1cm] {
\node[draw=black, rectangle split,  rectangle split parts=2] (sn0x1c23fc0){
\nodepart{one}
\begin{tikzpicture}[scale=.2]
\node[rectangle, scale=0.75] at (0, 0) {$\profile{1, 3, 1}$};
\end{tikzpicture}
\nodepart{two}
\footnotesize{3.625}
};
 \\ 
\\
};
\end{scope}
\begin{scope}[yshift=\leveltopIIIIIII cm, anchor = center]
\matrix (line7)[row sep=0.1cm] {
\node[draw=black, rectangle split,  rectangle split parts=2] (sn0x1c240d0){
\nodepart{one}
\begin{tikzpicture}[scale=.2]
\node[rectangle, scale=0.75] at (0, 0) {$\profile{1, 2, 1}$};
\end{tikzpicture}
\nodepart{two}
\footnotesize{3.25}
};
 \\ 
\node[draw=black, rectangle split,  rectangle split parts=2] (sn0x1c241e0){
\nodepart{one}
\begin{tikzpicture}[scale=.2]
\node[rectangle, scale=0.75] at (0, 0) {$\profile{3, 1}$};
\end{tikzpicture}
\nodepart{two}
\footnotesize{3}
};
 \\ 
\\
};
\end{scope}
\begin{scope}[yshift=\leveltopIIIIIIII cm, anchor = center]
\matrix (line8)[row sep=0.1cm] {
\node[draw=black, rectangle split,  rectangle split parts=2] (sn0x1c242f0){
\nodepart{one}
\begin{tikzpicture}[scale=.2]
\node[rectangle, scale=0.75] at (0, 0) {$\profile{1, 1, 1}$};
\end{tikzpicture}
\nodepart{two}
\footnotesize{3}
};
 \\ 
\node[draw=black, rectangle split,  rectangle split parts=2] (sn0x1c245b0){
\nodepart{one}
\begin{tikzpicture}[scale=.2]
\node[rectangle, scale=0.75] at (0, 0) {$\profile{2, 1}$};
\end{tikzpicture}
\nodepart{two}
\footnotesize{2.5}
};
 \\ 
\\
};
\end{scope}
\draw (sn0x1c22990.east) -- (sn0x1c28c10.west);
\draw (sn0x1c28c10.east) -- (sn0x1c29fa0.west);
\draw (sn0x1c29fa0.east) -- (sn0x1c28250.west);
\draw (sn0x1c28250.east) -- (sn0x1c23d40.west);
\draw (sn0x1c23d40.east) -- (sn0x1c23fc0.west);
\draw (sn0x1c23fc0.east) -- (sn0x1c240d0.west);
\draw (sn0x1c23fc0.east) -- (sn0x1c241e0.west);
\draw (sn0x1c240d0.east) -- (sn0x1c242f0.west);
\draw (sn0x1c240d0.east) -- (sn0x1c245b0.west);
\draw (sn0x1c241e0.east) -- (sn0x1c245b0.west);
\end{tikzpicture}
%%% Local Variables:
%%% TeX-master: "../thesis.tex"
%%% End: 
  \caption{A snapshot DAG (HLF, two processors) and its corresponding profile DAG containing much less nodes than the original snaphsot DAG. \emph{Remark:} Snapshots with same profiles have the same expected run time and can thus be merged. }
  \label{fig:p2-profile-dag-example-000111223-hlfdet}
\end{figure}

To be more mathematical: If the snapshot DAG is $(V_s, E_s)$, then the profile DAG can be expressed as $(V_p, E_p)$. These are defined as follows:
\begin{eqnarray*}
  V_p &=& \left\{ \text{PROFILE}(v) \mid v \in V_s \right\} \\
  E_p &=& \left\{ \left(\text{PROFILE}(v_1), \text{PROFILE}(v_2) \right) \mid (v_1, v_2)\in E_s\right\}
\end{eqnarray*}

The function $\text{PROFILE}$ takes a snapshot as input and returns the profile corresponding the the snapshot's intree. Note that $\text{PROFILE}$ then is a homomorphism between the snapthot and the profile DAG.\todo{Stimmt das?}

If we compute the expected run time via the profiles (as equation (\ref{eq:p2-profile-optimal-exp-run-time-def-simplified}) suggests), we only need to compute the profiles arising in the profile DAG. We can cache intermediate results, so we do not need to compute the runtime of any profile twice.

Thus, it is an important question how big these profile DAGs can get. To tackle this question, we examine for a profile $P=\profile{n_1,\dots,n_r}$ how many profiles of a certain length exist as successors of this profile.

We inspect the following example: Consider the profile $\profile{4,3,5,1}$. Its successing profiles of length 4 are
\begin{equation*}
  \profile{3, 3, 5, 1},
  \profile{2, 3, 5, 1},
  \profile{1, 3, 5, 1},
  \profile{1, 2, 5, 1},
  \profile{1, 1, 5, 1},
  \profile{1, 1, 4, 1},
  \profile{1, 1, 3, 1},
  \profile{1, 1, 2, 1} \text{ and }
  \profile{1, 1, 1, 1}.
\end{equation*}

We recognize that the first item in the succesing profiles has to be at most 4 (since the \emph{original} first entry was exactly 4). Moreover, the second entry can only be less then 3 (\emph{original} second entry: 3) if the first entry is 1. Similarily, the third entry in a successing profile can only be less then 5 (original third entry: 5) if the first and the second entry are 1.

More general: In a successing profile, the entry at a certain position can only be less than the original entry at this position if all entries \emph{up to that position} are already 1.

\begin{definition}[Successing profiles]
Let $P=\profile{n_1,\dots,n_r}$ be a profile of lenght $r$. The set of successing profiles of length $j$ of original profile $P$ is
\begin{equation*}
  S^p_{j}
  :=
  \left\{ 
    \profile{m_{r-j+1},\dots,m_{r}}
    \mid
    \exists p \in \left\{ r-j+1,\dots,r \right\}.\,
    \bigwedge_{i=1}^{p-1} m_i = 1 \ \wedge m_p < n_p
  \right\}.
\end{equation*}

The set of \emph{all} successing profiles clearly is then
\begin{equation*}
  S^P
  :=
  \bigcup_{j=1}^r S^P_j.
\end{equation*}
\end{definition}

The size of the profile DAG (starting at a certain profile $P$) is then clearly denoted by the size of $S^P$.

\begin{theorem}[Size of profile DAG]
  Let $P=\profile{n_1,\dots,n_r}$ be a profile. Then, 
  \begin{equation*}
    \left| S^P \right| = 
    \sum _{j=1}^{r} j \cdot n_j 
    -0.5r^2 + 1.5 r.
  \end{equation*}
\end{theorem}

\begin{proof}
  If $P=\profile{n_1,\dots,n_r}$ is a profile, then the size of the set $S^P_j$ is
  \begin{equation*}
    1+ \sum_{i=j}^{r-1} (n_i-1).
  \end{equation*}
  
  In a successing profile, the entry at position $p$ can take exactly the values $1, 2, 3, \dots, n_p-1$ (where $n_p$ is th entry at positon $p$ in profile $P$). This yields the term $(n_i-1)$. Moreover, we stated that the entry at position $p$ can only be smaller if \emph{all previous entries} are 1. Thus, we can simply sum um the terms for the different positions, yielding the above sum.

  We now sum up over the different \emph{lengths} (ranging from 1 to $r$) of the successor profiles, and afterwards apply well-known summation rules and formulae for triangular numbers:
  \begin{eqnarray*}
    \left| S^P \right| & = & \sum_{j=1}^{r} \left( 1+ \sum_{i=j}^{r-1} (n_i-1) \right) \\
    &=& \sum_{j=1}^{r} 1 + \sum_{j=1}^{r} \left( \sum_{i=j}^{r-1} n_i \right) - \sum_{j=1}^{r}\sum_{i=j}^{r-1} 1 \\
    &=&r - \sum_{j=1}^{r}(r-j) + \sum_{j=1}^{r} \left( \sum_{i=j}^{r-1} n_i \right) \\
    &=&r - \frac{r^2-r}{2} + \sum_{j=2}^{r-1} \left( \sum_{i=j}^{r-1} n_i \right) \\
    &=&\sum _{j=1}^{r} \left( \sum _{i=j}^{r-1}n_{{i}} \right) + -0.5r^2 + 1.5 r
  \end{eqnarray*}

  It remains to be shown that $\sum _{j=1}^{r} \left( \sum _{i=j}^{r-1}n_{{i}} \right) = \sum_{j=1}^{r}(j-1)\cdot n_j$. This can be seen by considering the following:
  \begin{eqnarray*}
    \setlength{\arraycolsep}{2pt}
    \begin{array}{cccccccccccccccc}
      \displaystyle
      \sum_{j=2}^{r-1} \left( \sum _{i=j}^{r-1}n_{{i}} \right) &=&
         n_1 & + & n_2 & + & n_3 & + & n_4 & + & \dots & + & n_{r-2} & + & n_{r-1}  & + \\
      & &    &   & n_2 & + & n_3 & + & n_4 & + & \dots & + & n_{r-2} & + & n_{r-1}  & + \\
      & &    &   &     &   & n_3 & + & n_4 & + & \dots & + & n_{r-2} & + & n_{r-1}  & + \\
      & &    &   &     &   &     & + & n_4 & + & \dots & + & n_{r-2} & + & n_{r-1}  & + \\
      & &    &   &     &   &     &  &  &  & \ddots &  &  &  & \vdots  & + \\
      & &    &   &     &   &     &   &     &   &   &   &  n_{r-2}  & +  & n_{r-1}  & +  \\
      & &    &   &     &   &     &   &     &   &   &   &    &   & n_{r-1}  &   \\
      \\
      &=& n_1 &+& 2 n_2 & + & 3 n_3 &+& 4 n_4 &+& \dots & + & (r-2)\cdot n_{r-2} &+& (r-1)\cdot n_{r-1} & \\
    \end{array}
  \end{eqnarray*}
  This shows that $\left|S^P\right| = \sum _{j=1}^{r-1} j \cdot n_j + 
    -0.5r^2 + 1.5 r$.
\end{proof}

%%% Local Variables:
%%% TeX-master: "../thesis.tex"
%%% End: 

\chapter{Three Processors}
\label{chap:p3}

\section{Some proofs that HLF is not optimal for P3}

This section showcases some situations, where HLF is not optimal. Figure \ref{fig:hlf-001112} an intree, where HLF can choose at some points, and different choices result in different runtimes. 

Figures \ref{fig:hlf-vs-opt-0012346688}, \ref{fig:hlf-vs-opt-0012446788} and \ref{fig:hlf-vs-opt-00123455799} show two examples that prove that HLF is not optimal for three processors.

\begin{figure}[ht]
  \centering
  \renewcommand{\leveltopI}{-15cm + \leveltop}
\renewcommand{\leveltopII}{-15cm + \leveltopI}
\renewcommand{\leveltopIII}{-15cm + \leveltopII}
\renewcommand{\leveltopIIII}{-15cm + \leveltopIII}
\renewcommand{\leveltopIIIII}{-15cm + \leveltopIIII}
\renewcommand{\leveltopIIIIII}{-15cm + \leveltopIIIII}
\renewcommand{\leveltopIIIIIII}{-15cm + \leveltopIIIIII}
\begin{tikzpicture}[scale=.2, anchor=south]
\begin{scope}[yshift=\leveltopI cm]
\matrix (line1)[column sep=0.5cm] {
\node[draw=black, rectangle split,  rectangle split parts=4] (sn0x8968ef8){
\footnotesize{100}
\nodepart{two}
\begin{tikzpicture}[scale=.2]
\node[circle, scale=0.75, fill] (tid0) at (3,1.5){};
\node[circle, scale=0.75, fill] (tid1) at (2.25,3){};
\node[circle, scale=0.75, fill, red] (tid3) at (0.75,4.5){};
\node[circle, scale=0.75, fill, red] (tid4) at (2.25,4.5){};
\node[circle, scale=0.75, fill, red] (tid5) at (3.75,4.5){};
\draw[](tid1) -- (tid3);
\draw[](tid1) -- (tid4);
\draw[](tid1) -- (tid5);
\node[circle, scale=0.75, fill] (tid2) at (5.25,3){};
\node[circle, scale=0.75, fill] (tid6) at (5.25,4.5){};
\draw[](tid2) -- (tid6);
\draw[](tid0) -- (tid1);
\draw[](tid0) -- (tid2);
\end{tikzpicture}
\nodepart{three}
\footnotesize{4.38889}
\nodepart{four}
\footnotesize{$1$}
};
 & 
\\
};
\end{scope}
\begin{scope}[yshift=\leveltopII cm]
\matrix (line2)[column sep=0.5cm] {
\node[draw=black, rectangle split,  rectangle split parts=4] (sn0x8969f98){
\footnotesize{100}
\nodepart{two}
\begin{tikzpicture}[scale=.2]
\node[circle, scale=0.75, fill] (tid0) at (2.25,1.5){};
\node[circle, scale=0.75, fill] (tid1) at (1.5,3){};
\node[circle, scale=0.75, fill, red] (tid3) at (0.75,4.5){};
\node[circle, scale=0.75, fill, red] (tid4) at (2.25,4.5){};
\draw[](tid1) -- (tid3);
\draw[](tid1) -- (tid4);
\node[circle, scale=0.75, fill] (tid2) at (3.75,3){};
\node[circle, scale=0.75, fill, red] (tid5) at (3.75,4.5){};
\draw[](tid2) -- (tid5);
\draw[](tid0) -- (tid1);
\draw[](tid0) -- (tid2);
\end{tikzpicture}
\nodepart{three}
\footnotesize{4.05556}
\nodepart{four}
\footnotesize{$67\:33$}
};
 & 
\\
};
\end{scope}
\begin{scope}[yshift=\leveltopIII cm]
\matrix (line3)[column sep=0.5cm] {
\node[draw=black, rectangle split,  rectangle split parts=4] (sn0x8969bb0){
\footnotesize{66.6667}
\nodepart{two}
\begin{tikzpicture}[scale=.2]
\node[circle, scale=0.75, fill] (tid0) at (1.5,1.5){};
\node[circle, scale=0.75, fill] (tid1) at (0.75,3){};
\node[circle, scale=0.75, fill, red] (tid3) at (0.75,4.5){};
\draw[](tid1) -- (tid3);
\node[circle, scale=0.75, fill] (tid2) at (2.25,3){};
\node[circle, scale=0.75, fill, red] (tid4) at (2.25,4.5){};
\draw[](tid2) -- (tid4);
\draw[](tid0) -- (tid1);
\draw[](tid0) -- (tid2);
\end{tikzpicture}
\nodepart{three}
\footnotesize{3.75}
\nodepart{four}
\footnotesize{$1$}
};
 & 
\node[draw=black, rectangle split,  rectangle split parts=4] (sn0x896a3b8){
\footnotesize{33.3333}
\nodepart{two}
\begin{tikzpicture}[scale=.2]
\node[circle, scale=0.75, fill] (tid0) at (2.25,1.5){};
\node[circle, scale=0.75, fill] (tid1) at (1.5,3){};
\node[circle, scale=0.75, fill, red] (tid3) at (0.75,4.5){};
\node[circle, scale=0.75, fill, red] (tid4) at (2.25,4.5){};
\draw[](tid1) -- (tid3);
\draw[](tid1) -- (tid4);
\node[circle, scale=0.75, fill, red] (tid2) at (3.75,3){};
\draw[](tid0) -- (tid1);
\draw[](tid0) -- (tid2);
\end{tikzpicture}
\nodepart{three}
\footnotesize{3.66667}
\nodepart{four}
\footnotesize{$67\:33$}
};
 & 
\\
};
\end{scope}
\draw (sn0x8968ef8.south) -- (sn0x8969f98.north);
\draw (sn0x8969f98.south) -- (sn0x8969bb0.north);
\draw (sn0x8969f98.south) -- (sn0x896a3b8.north);
\end{tikzpicture}
\renewcommand{\leveltopI}{-15cm + \leveltop}
\renewcommand{\leveltopII}{-15cm + \leveltopI}
\renewcommand{\leveltopIII}{-15cm + \leveltopII}
\renewcommand{\leveltopIIII}{-15cm + \leveltopIII}
\renewcommand{\leveltopIIIII}{-15cm + \leveltopIIII}
\renewcommand{\leveltopIIIIII}{-15cm + \leveltopIIIII}
\renewcommand{\leveltopIIIIIII}{-15cm + \leveltopIIIIII}
\begin{tikzpicture}[scale=.2, anchor=south]
\begin{scope}[yshift=\leveltopI cm]
\matrix (line1)[column sep=0.5cm] {
\node[draw=black, rectangle split,  rectangle split parts=4] (sn0x8969e68){
\footnotesize{100}
\nodepart{two}
\begin{tikzpicture}[scale=.2]
\node[circle, scale=0.75, fill] (tid0) at (3,1.5){};
\node[circle, scale=0.75, fill] (tid1) at (2.25,3){};
\node[circle, scale=0.75, fill, red] (tid3) at (0.75,4.5){};
\node[circle, scale=0.75, fill, red] (tid4) at (2.25,4.5){};
\node[circle, scale=0.75, fill] (tid5) at (3.75,4.5){};
\draw[](tid1) -- (tid3);
\draw[](tid1) -- (tid4);
\draw[](tid1) -- (tid5);
\node[circle, scale=0.75, fill] (tid2) at (5.25,3){};
\node[circle, scale=0.75, fill, red] (tid6) at (5.25,4.5){};
\draw[](tid2) -- (tid6);
\draw[](tid0) -- (tid1);
\draw[](tid0) -- (tid2);
\end{tikzpicture}
\nodepart{three}
\footnotesize{4.37037}
\nodepart{four}
\footnotesize{$67\:33$}
};
 & 
\\
};
\end{scope}
\begin{scope}[yshift=\leveltopII cm]
\matrix (line2)[column sep=0.5cm] {
\node[draw=black, rectangle split,  rectangle split parts=4] (sn0x8969f98){
\footnotesize{66.6667}
\nodepart{two}
\begin{tikzpicture}[scale=.2]
\node[circle, scale=0.75, fill] (tid0) at (2.25,1.5){};
\node[circle, scale=0.75, fill] (tid1) at (1.5,3){};
\node[circle, scale=0.75, fill, red] (tid3) at (0.75,4.5){};
\node[circle, scale=0.75, fill, red] (tid4) at (2.25,4.5){};
\draw[](tid1) -- (tid3);
\draw[](tid1) -- (tid4);
\node[circle, scale=0.75, fill] (tid2) at (3.75,3){};
\node[circle, scale=0.75, fill, red] (tid5) at (3.75,4.5){};
\draw[](tid2) -- (tid5);
\draw[](tid0) -- (tid1);
\draw[](tid0) -- (tid2);
\end{tikzpicture}
\nodepart{three}
\footnotesize{4.05556}
\nodepart{four}
\footnotesize{$67\:33$}
};
 & 
\node[draw=black, rectangle split,  rectangle split parts=4] (sn0x896a940){
\footnotesize{33.3333}
\nodepart{two}
\begin{tikzpicture}[scale=.2]
\node[circle, scale=0.75, fill] (tid0) at (3,1.5){};
\node[circle, scale=0.75, fill] (tid1) at (2.25,3){};
\node[circle, scale=0.75, fill, red] (tid3) at (0.75,4.5){};
\node[circle, scale=0.75, fill, red] (tid4) at (2.25,4.5){};
\node[circle, scale=0.75, fill, red] (tid5) at (3.75,4.5){};
\draw[](tid1) -- (tid3);
\draw[](tid1) -- (tid4);
\draw[](tid1) -- (tid5);
\node[circle, scale=0.75, fill] (tid2) at (5.25,3){};
\draw[](tid0) -- (tid1);
\draw[](tid0) -- (tid2);
\end{tikzpicture}
\nodepart{three}
\footnotesize{4}
\nodepart{four}
\footnotesize{$1$}
};
 & 
\\
};
\end{scope}
\begin{scope}[yshift=\leveltopIII cm]
\matrix (line3)[column sep=0.5cm] {
\node[draw=black, rectangle split,  rectangle split parts=4] (sn0x8969bb0){
\footnotesize{44.4444}
\nodepart{two}
\begin{tikzpicture}[scale=.2]
\node[circle, scale=0.75, fill] (tid0) at (1.5,1.5){};
\node[circle, scale=0.75, fill] (tid1) at (0.75,3){};
\node[circle, scale=0.75, fill, red] (tid3) at (0.75,4.5){};
\draw[](tid1) -- (tid3);
\node[circle, scale=0.75, fill] (tid2) at (2.25,3){};
\node[circle, scale=0.75, fill, red] (tid4) at (2.25,4.5){};
\draw[](tid2) -- (tid4);
\draw[](tid0) -- (tid1);
\draw[](tid0) -- (tid2);
\end{tikzpicture}
\nodepart{three}
\footnotesize{3.75}
\nodepart{four}
\footnotesize{$1$}
};
 & 
\node[draw=black, rectangle split,  rectangle split parts=4] (sn0x896a3b8){
\footnotesize{55.5556}
\nodepart{two}
\begin{tikzpicture}[scale=.2]
\node[circle, scale=0.75, fill] (tid0) at (2.25,1.5){};
\node[circle, scale=0.75, fill] (tid1) at (1.5,3){};
\node[circle, scale=0.75, fill, red] (tid3) at (0.75,4.5){};
\node[circle, scale=0.75, fill, red] (tid4) at (2.25,4.5){};
\draw[](tid1) -- (tid3);
\draw[](tid1) -- (tid4);
\node[circle, scale=0.75, fill, red] (tid2) at (3.75,3){};
\draw[](tid0) -- (tid1);
\draw[](tid0) -- (tid2);
\end{tikzpicture}
\nodepart{three}
\footnotesize{3.66667}
\nodepart{four}
\footnotesize{$67\:33$}
};
 & 
\\
};
\end{scope}
\draw (sn0x8969e68.south) -- (sn0x8969f98.north);
\draw (sn0x8969e68.south) -- (sn0x896a940.north);
\draw (sn0x8969f98.south) -- (sn0x8969bb0.north);
\draw (sn0x8969f98.south) -- (sn0x896a3b8.north);
\draw (sn0x896a940.south) -- (sn0x896a3b8.north);
\end{tikzpicture}
%%% Local Variables:
%%% TeX-master: "thesis/thesis.tex"
%%% End: 
  \caption{HLF on 001112. Different runs of HLF do not necessarily produce the same result.}
  \label{fig:hlf-001112}
\end{figure}

\begin{figure}[ht]
  \centering
  \renewcommand{\leveltopI}{-15cm + \leveltop}
\renewcommand{\leveltopII}{-15cm + \leveltopI}
\renewcommand{\leveltopIII}{-15cm + \leveltopII}
\renewcommand{\leveltopIIII}{-15cm + \leveltopIII}
\renewcommand{\leveltopIIIII}{-15cm + \leveltopIIII}
\renewcommand{\leveltopIIIIII}{-15cm + \leveltopIIIII}
\renewcommand{\leveltopIIIIIII}{-15cm + \leveltopIIIIII}
\renewcommand{\leveltopIIIIIIII}{-15cm + \leveltopIIIIIII}
\renewcommand{\leveltopIIIIIIIII}{-15cm + \leveltopIIIIIIII}
\renewcommand{\leveltopIIIIIIIIII}{-15cm + \leveltopIIIIIIIII}
\renewcommand{\leveltopIIIIIIIIIII}{-15cm + \leveltopIIIIIIIIII}
\begin{tikzpicture}[scale=.2, anchor=south]
\begin{scope}[yshift=\leveltopI cm]
\matrix (line1) [column sep=1cm] {
\node[draw=black, rectangle split,  rectangle split parts=3] (sn0x14639f0){
\begin{tikzpicture}[scale=.2]
\node[circle, scale=0.75, fill] (tid0) at (3,1.5){};
\node[circle, scale=0.75, fill] (tid1) at (2.25,3){};
\node[circle, scale=0.75, fill] (tid3) at (2.25,4.5){};
\node[circle, scale=0.75, fill] (tid5) at (2.25,6){};
\node[circle, scale=0.75, fill] (tid7) at (1.5,7.5){};
\node[circle, scale=0.75, fill, red] (tid9) at (0.75,9){};
\node[circle, scale=0.75, fill, red] (tid10) at (2.25,9){};
\draw[](tid7) -- (tid9);
\draw[](tid7) -- (tid10);
\node[circle, scale=0.75, fill, red] (tid8) at (3.75,7.5){};
\draw[](tid5) -- (tid7);
\draw[](tid5) -- (tid8);
\draw[](tid3) -- (tid5);
\draw[](tid1) -- (tid3);
\node[circle, scale=0.75, fill] (tid2) at (5.25,3){};
\node[circle, scale=0.75, fill] (tid4) at (5.25,4.5){};
\node[circle, scale=0.75, fill] (tid6) at (5.25,6){};
\draw[](tid4) -- (tid6);
\draw[](tid2) -- (tid4);
\draw[](tid0) -- (tid1);
\draw[](tid0) -- (tid2);
\end{tikzpicture}
\nodepart{two}
\footnotesize{6.96798}
\nodepart{three}
\footnotesize{$33\:67$}
};
 & 
\\
};
\end{scope}
\begin{scope}[yshift=\leveltopII cm]
\matrix (line2) [column sep=1cm] {
\node[draw=black, rectangle split,  rectangle split parts=3] (sn0x1460d30){
\begin{tikzpicture}[scale=.2]
\node[circle, scale=0.75, fill] (tid0) at (2.25,1.5){};
\node[circle, scale=0.75, fill] (tid1) at (1.5,3){};
\node[circle, scale=0.75, fill] (tid3) at (1.5,4.5){};
\node[circle, scale=0.75, fill] (tid5) at (1.5,6){};
\node[circle, scale=0.75, fill] (tid7) at (1.5,7.5){};
\node[circle, scale=0.75, fill, red] (tid8) at (0.75,9){};
\node[circle, scale=0.75, fill, red] (tid9) at (2.25,9){};
\draw[](tid7) -- (tid8);
\draw[](tid7) -- (tid9);
\draw[](tid5) -- (tid7);
\draw[](tid3) -- (tid5);
\draw[](tid1) -- (tid3);
\node[circle, scale=0.75, fill] (tid2) at (3.75,3){};
\node[circle, scale=0.75, fill] (tid4) at (3.75,4.5){};
\node[circle, scale=0.75, fill, red] (tid6) at (3.75,6){};
\draw[](tid4) -- (tid6);
\draw[](tid2) -- (tid4);
\draw[](tid0) -- (tid1);
\draw[](tid0) -- (tid2);
\end{tikzpicture}
\nodepart{two}
\footnotesize{6.77836}
\nodepart{three}
\footnotesize{$33\:67$}
};
 & 
\node[draw=black, rectangle split,  rectangle split parts=3] (sn0x14633d0){
\begin{tikzpicture}[scale=.2]
\node[circle, scale=0.75, fill] (tid0) at (2.25,1.5){};
\node[circle, scale=0.75, fill] (tid1) at (1.5,3){};
\node[circle, scale=0.75, fill] (tid3) at (1.5,4.5){};
\node[circle, scale=0.75, fill] (tid5) at (1.5,6){};
\node[circle, scale=0.75, fill] (tid7) at (0.75,7.5){};
\node[circle, scale=0.75, fill, red] (tid9) at (0.75,9){};
\draw[](tid7) -- (tid9);
\node[circle, scale=0.75, fill, red] (tid8) at (2.25,7.5){};
\draw[](tid5) -- (tid7);
\draw[](tid5) -- (tid8);
\draw[](tid3) -- (tid5);
\draw[](tid1) -- (tid3);
\node[circle, scale=0.75, fill] (tid2) at (3.75,3){};
\node[circle, scale=0.75, fill] (tid4) at (3.75,4.5){};
\node[circle, scale=0.75, fill, red] (tid6) at (3.75,6){};
\draw[](tid4) -- (tid6);
\draw[](tid2) -- (tid4);
\draw[](tid0) -- (tid1);
\draw[](tid0) -- (tid2);
\end{tikzpicture}
\nodepart{two}
\footnotesize{6.56279}
\nodepart{three}
\footnotesize{$33\:33\:33$}
};
 & 
\\
};
\end{scope}
\begin{scope}[yshift=\leveltopIII cm]
\matrix (line3) [column sep=1cm] {
\node[draw=black, rectangle split,  rectangle split parts=3] (sn0x145fad0){
\begin{tikzpicture}[scale=.2]
\node[circle, scale=0.75, fill] (tid0) at (2.25,1.5){};
\node[circle, scale=0.75, fill] (tid1) at (1.5,3){};
\node[circle, scale=0.75, fill] (tid3) at (1.5,4.5){};
\node[circle, scale=0.75, fill] (tid5) at (1.5,6){};
\node[circle, scale=0.75, fill] (tid6) at (1.5,7.5){};
\node[circle, scale=0.75, fill, red] (tid7) at (0.75,9){};
\node[circle, scale=0.75, fill, red] (tid8) at (2.25,9){};
\draw[](tid6) -- (tid7);
\draw[](tid6) -- (tid8);
\draw[](tid5) -- (tid6);
\draw[](tid3) -- (tid5);
\draw[](tid1) -- (tid3);
\node[circle, scale=0.75, fill] (tid2) at (3.75,3){};
\node[circle, scale=0.75, fill, red] (tid4) at (3.75,4.5){};
\draw[](tid2) -- (tid4);
\draw[](tid0) -- (tid1);
\draw[](tid0) -- (tid2);
\end{tikzpicture}
\nodepart{two}
\footnotesize{6.60069}
\nodepart{three}
\footnotesize{$33\:67$}
};
 & 
\node[draw=black, rectangle split,  rectangle split parts=3] (sn0x1460050){
\begin{tikzpicture}[scale=.2]
\node[circle, scale=0.75, fill] (tid0) at (1.5,1.5){};
\node[circle, scale=0.75, fill] (tid1) at (0.75,3){};
\node[circle, scale=0.75, fill] (tid3) at (0.75,4.5){};
\node[circle, scale=0.75, fill] (tid5) at (0.75,6){};
\node[circle, scale=0.75, fill] (tid7) at (0.75,7.5){};
\node[circle, scale=0.75, fill, red] (tid8) at (0.75,9){};
\draw[](tid7) -- (tid8);
\draw[](tid5) -- (tid7);
\draw[](tid3) -- (tid5);
\draw[](tid1) -- (tid3);
\node[circle, scale=0.75, fill] (tid2) at (2.25,3){};
\node[circle, scale=0.75, fill] (tid4) at (2.25,4.5){};
\node[circle, scale=0.75, fill, red] (tid6) at (2.25,6){};
\draw[](tid4) -- (tid6);
\draw[](tid2) -- (tid4);
\draw[](tid0) -- (tid1);
\draw[](tid0) -- (tid2);
\end{tikzpicture}
\nodepart{two}
\footnotesize{6.36719}
\nodepart{three}
\footnotesize{$50\:50$}
};
 & 
\node[draw=black, rectangle split,  rectangle split parts=3] (sn0x1462fe0){
\begin{tikzpicture}[scale=.2]
\node[circle, scale=0.75, fill] (tid0) at (2.25,1.5){};
\node[circle, scale=0.75, fill] (tid1) at (1.5,3){};
\node[circle, scale=0.75, fill] (tid3) at (1.5,4.5){};
\node[circle, scale=0.75, fill] (tid5) at (1.5,6){};
\node[circle, scale=0.75, fill] (tid6) at (0.75,7.5){};
\node[circle, scale=0.75, fill, red] (tid8) at (0.75,9){};
\draw[](tid6) -- (tid8);
\node[circle, scale=0.75, fill, red] (tid7) at (2.25,7.5){};
\draw[](tid5) -- (tid6);
\draw[](tid5) -- (tid7);
\draw[](tid3) -- (tid5);
\draw[](tid1) -- (tid3);
\node[circle, scale=0.75, fill] (tid2) at (3.75,3){};
\node[circle, scale=0.75, fill, red] (tid4) at (3.75,4.5){};
\draw[](tid2) -- (tid4);
\draw[](tid0) -- (tid1);
\draw[](tid0) -- (tid2);
\end{tikzpicture}
\nodepart{two}
\footnotesize{6.36516}
\nodepart{three}
\footnotesize{$33\:33\:33$}
};
 & 
\node[draw=black, rectangle split,  rectangle split parts=3] (sn0x1462bf0){
\begin{tikzpicture}[scale=.2]
\node[circle, scale=0.75, fill] (tid0) at (2.25,1.5){};
\node[circle, scale=0.75, fill] (tid1) at (1.5,3){};
\node[circle, scale=0.75, fill] (tid3) at (1.5,4.5){};
\node[circle, scale=0.75, fill] (tid5) at (1.5,6){};
\node[circle, scale=0.75, fill, red] (tid7) at (0.75,7.5){};
\node[circle, scale=0.75, fill, red] (tid8) at (2.25,7.5){};
\draw[](tid5) -- (tid7);
\draw[](tid5) -- (tid8);
\draw[](tid3) -- (tid5);
\draw[](tid1) -- (tid3);
\node[circle, scale=0.75, fill] (tid2) at (3.75,3){};
\node[circle, scale=0.75, fill] (tid4) at (3.75,4.5){};
\node[circle, scale=0.75, fill, red] (tid6) at (3.75,6){};
\draw[](tid4) -- (tid6);
\draw[](tid2) -- (tid4);
\draw[](tid0) -- (tid1);
\draw[](tid0) -- (tid2);
\end{tikzpicture}
\nodepart{two}
\footnotesize{5.95602}
\nodepart{three}
\footnotesize{$67\:33$}
};
 & 
\\
};
\end{scope}
\begin{scope}[yshift=\leveltopIIII cm]
\matrix (line4) [column sep=1cm] {
\node[draw=black, rectangle split,  rectangle split parts=3] (sn0x145e900){
\begin{tikzpicture}[scale=.2]
\node[circle, scale=0.75, fill] (tid0) at (2.25,1.5){};
\node[circle, scale=0.75, fill] (tid1) at (1.5,3){};
\node[circle, scale=0.75, fill] (tid3) at (1.5,4.5){};
\node[circle, scale=0.75, fill] (tid4) at (1.5,6){};
\node[circle, scale=0.75, fill] (tid5) at (1.5,7.5){};
\node[circle, scale=0.75, fill, red] (tid6) at (0.75,9){};
\node[circle, scale=0.75, fill, red] (tid7) at (2.25,9){};
\draw[](tid5) -- (tid6);
\draw[](tid5) -- (tid7);
\draw[](tid4) -- (tid5);
\draw[](tid3) -- (tid4);
\draw[](tid1) -- (tid3);
\node[circle, scale=0.75, fill, red] (tid2) at (3.75,3){};
\draw[](tid0) -- (tid1);
\draw[](tid0) -- (tid2);
\end{tikzpicture}
\nodepart{two}
\footnotesize{6.52083}
\nodepart{three}
\footnotesize{$33\:67$}
};
 & 
\node[draw=black, rectangle split,  rectangle split parts=3] (sn0x145fa00){
\begin{tikzpicture}[scale=.2]
\node[circle, scale=0.75, fill] (tid0) at (1.5,1.5){};
\node[circle, scale=0.75, fill] (tid1) at (0.75,3){};
\node[circle, scale=0.75, fill] (tid3) at (0.75,4.5){};
\node[circle, scale=0.75, fill] (tid5) at (0.75,6){};
\node[circle, scale=0.75, fill] (tid6) at (0.75,7.5){};
\node[circle, scale=0.75, fill, red] (tid7) at (0.75,9){};
\draw[](tid6) -- (tid7);
\draw[](tid5) -- (tid6);
\draw[](tid3) -- (tid5);
\draw[](tid1) -- (tid3);
\node[circle, scale=0.75, fill] (tid2) at (2.25,3){};
\node[circle, scale=0.75, fill, red] (tid4) at (2.25,4.5){};
\draw[](tid2) -- (tid4);
\draw[](tid0) -- (tid1);
\draw[](tid0) -- (tid2);
\end{tikzpicture}
\nodepart{two}
\footnotesize{6.14062}
\nodepart{three}
\footnotesize{$50\:50$}
};
 & 
\node[draw=black, rectangle split,  rectangle split parts=3] (sn0x145ff80){
\begin{tikzpicture}[scale=.2]
\node[circle, scale=0.75, fill] (tid0) at (1.5,1.5){};
\node[circle, scale=0.75, fill] (tid1) at (0.75,3){};
\node[circle, scale=0.75, fill] (tid3) at (0.75,4.5){};
\node[circle, scale=0.75, fill] (tid5) at (0.75,6){};
\node[circle, scale=0.75, fill, red] (tid7) at (0.75,7.5){};
\draw[](tid5) -- (tid7);
\draw[](tid3) -- (tid5);
\draw[](tid1) -- (tid3);
\node[circle, scale=0.75, fill] (tid2) at (2.25,3){};
\node[circle, scale=0.75, fill] (tid4) at (2.25,4.5){};
\node[circle, scale=0.75, fill, red] (tid6) at (2.25,6){};
\draw[](tid4) -- (tid6);
\draw[](tid2) -- (tid4);
\draw[](tid0) -- (tid1);
\draw[](tid0) -- (tid2);
\end{tikzpicture}
\nodepart{two}
\footnotesize{5.59375}
\nodepart{three}
\footnotesize{$50\:50$}
};
 & 
\node[draw=black, rectangle split,  rectangle split parts=3] (sn0x1462420){
\begin{tikzpicture}[scale=.2]
\node[circle, scale=0.75, fill] (tid0) at (2.25,1.5){};
\node[circle, scale=0.75, fill] (tid1) at (1.5,3){};
\node[circle, scale=0.75, fill] (tid3) at (1.5,4.5){};
\node[circle, scale=0.75, fill] (tid4) at (1.5,6){};
\node[circle, scale=0.75, fill] (tid5) at (0.75,7.5){};
\node[circle, scale=0.75, fill, red] (tid7) at (0.75,9){};
\draw[](tid5) -- (tid7);
\node[circle, scale=0.75, fill, red] (tid6) at (2.25,7.5){};
\draw[](tid4) -- (tid5);
\draw[](tid4) -- (tid6);
\draw[](tid3) -- (tid4);
\draw[](tid1) -- (tid3);
\node[circle, scale=0.75, fill, red] (tid2) at (3.75,3){};
\draw[](tid0) -- (tid1);
\draw[](tid0) -- (tid2);
\end{tikzpicture}
\nodepart{two}
\footnotesize{6.27431}
\nodepart{three}
\footnotesize{$33\:33\:33$}
};
 & 
\node[draw=black, rectangle split,  rectangle split parts=3] (sn0x14624f0){
\begin{tikzpicture}[scale=.2]
\node[circle, scale=0.75, fill] (tid0) at (2.25,1.5){};
\node[circle, scale=0.75, fill] (tid1) at (1.5,3){};
\node[circle, scale=0.75, fill] (tid3) at (1.5,4.5){};
\node[circle, scale=0.75, fill] (tid5) at (1.5,6){};
\node[circle, scale=0.75, fill, red] (tid6) at (0.75,7.5){};
\node[circle, scale=0.75, fill, red] (tid7) at (2.25,7.5){};
\draw[](tid5) -- (tid6);
\draw[](tid5) -- (tid7);
\draw[](tid3) -- (tid5);
\draw[](tid1) -- (tid3);
\node[circle, scale=0.75, fill] (tid2) at (3.75,3){};
\node[circle, scale=0.75, fill, red] (tid4) at (3.75,4.5){};
\draw[](tid2) -- (tid4);
\draw[](tid0) -- (tid1);
\draw[](tid0) -- (tid2);
\end{tikzpicture}
\nodepart{two}
\footnotesize{5.68056}
\nodepart{three}
\footnotesize{$67\:33$}
};
 & 
\\
};
\end{scope}
\begin{scope}[yshift=\leveltopIIIII cm]
\matrix (line5) [column sep=1cm] {
\node[draw=black, rectangle split,  rectangle split parts=3] (sn0x145d480){
\begin{tikzpicture}[scale=.2]
\node[circle, scale=0.75, fill] (tid0) at (1.5,1.5){};
\node[circle, scale=0.75, fill] (tid1) at (1.5,3){};
\node[circle, scale=0.75, fill] (tid2) at (1.5,4.5){};
\node[circle, scale=0.75, fill] (tid3) at (1.5,6){};
\node[circle, scale=0.75, fill] (tid4) at (1.5,7.5){};
\node[circle, scale=0.75, fill, red] (tid5) at (0.75,9){};
\node[circle, scale=0.75, fill, red] (tid6) at (2.25,9){};
\draw[](tid4) -- (tid5);
\draw[](tid4) -- (tid6);
\draw[](tid3) -- (tid4);
\draw[](tid2) -- (tid3);
\draw[](tid1) -- (tid2);
\draw[](tid0) -- (tid1);
\end{tikzpicture}
\nodepart{two}
\footnotesize{6.5}
\nodepart{three}
\footnotesize{$1$}
};
 & 
\node[draw=black, rectangle split,  rectangle split parts=3] (sn0x145e670){
\begin{tikzpicture}[scale=.2]
\node[circle, scale=0.75, fill] (tid0) at (1.5,1.5){};
\node[circle, scale=0.75, fill] (tid1) at (0.75,3){};
\node[circle, scale=0.75, fill] (tid3) at (0.75,4.5){};
\node[circle, scale=0.75, fill] (tid4) at (0.75,6){};
\node[circle, scale=0.75, fill] (tid5) at (0.75,7.5){};
\node[circle, scale=0.75, fill, red] (tid6) at (0.75,9){};
\draw[](tid5) -- (tid6);
\draw[](tid4) -- (tid5);
\draw[](tid3) -- (tid4);
\draw[](tid1) -- (tid3);
\node[circle, scale=0.75, fill, red] (tid2) at (2.25,3){};
\draw[](tid0) -- (tid1);
\draw[](tid0) -- (tid2);
\end{tikzpicture}
\nodepart{two}
\footnotesize{6.03125}
\nodepart{three}
\footnotesize{$50\:50$}
};
 & 
\node[draw=black, rectangle split,  rectangle split parts=3] (sn0x145f770){
\begin{tikzpicture}[scale=.2]
\node[circle, scale=0.75, fill] (tid0) at (1.5,1.5){};
\node[circle, scale=0.75, fill] (tid1) at (0.75,3){};
\node[circle, scale=0.75, fill] (tid3) at (0.75,4.5){};
\node[circle, scale=0.75, fill] (tid5) at (0.75,6){};
\node[circle, scale=0.75, fill, red] (tid6) at (0.75,7.5){};
\draw[](tid5) -- (tid6);
\draw[](tid3) -- (tid5);
\draw[](tid1) -- (tid3);
\node[circle, scale=0.75, fill] (tid2) at (2.25,3){};
\node[circle, scale=0.75, fill, red] (tid4) at (2.25,4.5){};
\draw[](tid2) -- (tid4);
\draw[](tid0) -- (tid1);
\draw[](tid0) -- (tid2);
\end{tikzpicture}
\nodepart{two}
\footnotesize{5.25}
\nodepart{three}
\footnotesize{$50\:50$}
};
 & 
\node[draw=black, rectangle split,  rectangle split parts=3] (sn0x1460b00){
\begin{tikzpicture}[scale=.2]
\node[circle, scale=0.75, fill] (tid0) at (1.5,1.5){};
\node[circle, scale=0.75, fill] (tid1) at (0.75,3){};
\node[circle, scale=0.75, fill] (tid3) at (0.75,4.5){};
\node[circle, scale=0.75, fill, red] (tid5) at (0.75,6){};
\draw[](tid3) -- (tid5);
\draw[](tid1) -- (tid3);
\node[circle, scale=0.75, fill] (tid2) at (2.25,3){};
\node[circle, scale=0.75, fill] (tid4) at (2.25,4.5){};
\node[circle, scale=0.75, fill, red] (tid6) at (2.25,6){};
\draw[](tid4) -- (tid6);
\draw[](tid2) -- (tid4);
\draw[](tid0) -- (tid1);
\draw[](tid0) -- (tid2);
\end{tikzpicture}
\nodepart{two}
\footnotesize{4.9375}
\nodepart{three}
\footnotesize{$1$}
};
 & 
\node[draw=black, rectangle split,  rectangle split parts=3] (sn0x1462350){
\begin{tikzpicture}[scale=.2]
\node[circle, scale=0.75, fill] (tid0) at (1.5,1.5){};
\node[circle, scale=0.75, fill] (tid1) at (1.5,3){};
\node[circle, scale=0.75, fill] (tid2) at (1.5,4.5){};
\node[circle, scale=0.75, fill] (tid3) at (1.5,6){};
\node[circle, scale=0.75, fill] (tid4) at (0.75,7.5){};
\node[circle, scale=0.75, fill, red] (tid6) at (0.75,9){};
\draw[](tid4) -- (tid6);
\node[circle, scale=0.75, fill, red] (tid5) at (2.25,7.5){};
\draw[](tid3) -- (tid4);
\draw[](tid3) -- (tid5);
\draw[](tid2) -- (tid3);
\draw[](tid1) -- (tid2);
\draw[](tid0) -- (tid1);
\end{tikzpicture}
\nodepart{two}
\footnotesize{6.25}
\nodepart{three}
\footnotesize{$50\:50$}
};
 & 
\node[draw=black, rectangle split,  rectangle split parts=3] (sn0x14628e0){
\begin{tikzpicture}[scale=.2]
\node[circle, scale=0.75, fill] (tid0) at (2.25,1.5){};
\node[circle, scale=0.75, fill] (tid1) at (1.5,3){};
\node[circle, scale=0.75, fill] (tid3) at (1.5,4.5){};
\node[circle, scale=0.75, fill] (tid4) at (1.5,6){};
\node[circle, scale=0.75, fill, red] (tid5) at (0.75,7.5){};
\node[circle, scale=0.75, fill, red] (tid6) at (2.25,7.5){};
\draw[](tid4) -- (tid5);
\draw[](tid4) -- (tid6);
\draw[](tid3) -- (tid4);
\draw[](tid1) -- (tid3);
\node[circle, scale=0.75, fill, red] (tid2) at (3.75,3){};
\draw[](tid0) -- (tid1);
\draw[](tid0) -- (tid2);
\end{tikzpicture}
\nodepart{two}
\footnotesize{5.54167}
\nodepart{three}
\footnotesize{$67\:33$}
};
 & 
\\
};
\end{scope}
\begin{scope}[yshift=\leveltopIIIIII cm]
\matrix (line6) [column sep=1cm] {
\node[draw=black, rectangle split,  rectangle split parts=3] (sn0x145d0f0){
\begin{tikzpicture}[scale=.2]
\node[circle, scale=0.75, fill] (tid0) at (0.75,1.5){};
\node[circle, scale=0.75, fill] (tid1) at (0.75,3){};
\node[circle, scale=0.75, fill] (tid2) at (0.75,4.5){};
\node[circle, scale=0.75, fill] (tid3) at (0.75,6){};
\node[circle, scale=0.75, fill] (tid4) at (0.75,7.5){};
\node[circle, scale=0.75, fill, red] (tid5) at (0.75,9){};
\draw[](tid4) -- (tid5);
\draw[](tid3) -- (tid4);
\draw[](tid2) -- (tid3);
\draw[](tid1) -- (tid2);
\draw[](tid0) -- (tid1);
\end{tikzpicture}
\nodepart{two}
\footnotesize{6}
\nodepart{three}
\footnotesize{$1$}
};
 & 
\node[draw=black, rectangle split,  rectangle split parts=3] (sn0x145e380){
\begin{tikzpicture}[scale=.2]
\node[circle, scale=0.75, fill] (tid0) at (1.5,1.5){};
\node[circle, scale=0.75, fill] (tid1) at (0.75,3){};
\node[circle, scale=0.75, fill] (tid3) at (0.75,4.5){};
\node[circle, scale=0.75, fill] (tid4) at (0.75,6){};
\node[circle, scale=0.75, fill, red] (tid5) at (0.75,7.5){};
\draw[](tid4) -- (tid5);
\draw[](tid3) -- (tid4);
\draw[](tid1) -- (tid3);
\node[circle, scale=0.75, fill, red] (tid2) at (2.25,3){};
\draw[](tid0) -- (tid1);
\draw[](tid0) -- (tid2);
\end{tikzpicture}
\nodepart{two}
\footnotesize{5.0625}
\nodepart{three}
\footnotesize{$50\:50$}
};
 & 
\node[draw=black, rectangle split,  rectangle split parts=3] (sn0x145ec40){
\begin{tikzpicture}[scale=.2]
\node[circle, scale=0.75, fill] (tid0) at (1.5,1.5){};
\node[circle, scale=0.75, fill] (tid1) at (0.75,3){};
\node[circle, scale=0.75, fill] (tid3) at (0.75,4.5){};
\node[circle, scale=0.75, fill, red] (tid5) at (0.75,6){};
\draw[](tid3) -- (tid5);
\draw[](tid1) -- (tid3);
\node[circle, scale=0.75, fill] (tid2) at (2.25,3){};
\node[circle, scale=0.75, fill, red] (tid4) at (2.25,4.5){};
\draw[](tid2) -- (tid4);
\draw[](tid0) -- (tid1);
\draw[](tid0) -- (tid2);
\end{tikzpicture}
\nodepart{two}
\footnotesize{4.4375}
\nodepart{three}
\footnotesize{$50\:50$}
};
 & 
\node[draw=black, rectangle split,  rectangle split parts=3] (sn0x1461150){
\begin{tikzpicture}[scale=.2]
\node[circle, scale=0.75, fill] (tid0) at (1.5,1.5){};
\node[circle, scale=0.75, fill] (tid1) at (1.5,3){};
\node[circle, scale=0.75, fill] (tid2) at (1.5,4.5){};
\node[circle, scale=0.75, fill] (tid3) at (1.5,6){};
\node[circle, scale=0.75, fill, red] (tid4) at (0.75,7.5){};
\node[circle, scale=0.75, fill, red] (tid5) at (2.25,7.5){};
\draw[](tid3) -- (tid4);
\draw[](tid3) -- (tid5);
\draw[](tid2) -- (tid3);
\draw[](tid1) -- (tid2);
\draw[](tid0) -- (tid1);
\end{tikzpicture}
\nodepart{two}
\footnotesize{5.5}
\nodepart{three}
\footnotesize{$1$}
};
 & 
\\
};
\end{scope}
\begin{scope}[yshift=\leveltopIIIIIII cm]
\matrix (line7) [column sep=1cm] {
\node[draw=black, rectangle split,  rectangle split parts=3] (sn0x145cdf0){
\begin{tikzpicture}[scale=.2]
\node[circle, scale=0.75, fill] (tid0) at (0.75,1.5){};
\node[circle, scale=0.75, fill] (tid1) at (0.75,3){};
\node[circle, scale=0.75, fill] (tid2) at (0.75,4.5){};
\node[circle, scale=0.75, fill] (tid3) at (0.75,6){};
\node[circle, scale=0.75, fill, red] (tid4) at (0.75,7.5){};
\draw[](tid3) -- (tid4);
\draw[](tid2) -- (tid3);
\draw[](tid1) -- (tid2);
\draw[](tid0) -- (tid1);
\end{tikzpicture}
\nodepart{two}
\footnotesize{5}
\nodepart{three}
\footnotesize{$1$}
};
 & 
\node[draw=black, rectangle split,  rectangle split parts=3] (sn0x145e0d0){
\begin{tikzpicture}[scale=.2]
\node[circle, scale=0.75, fill] (tid0) at (1.5,1.5){};
\node[circle, scale=0.75, fill] (tid1) at (0.75,3){};
\node[circle, scale=0.75, fill] (tid3) at (0.75,4.5){};
\node[circle, scale=0.75, fill, red] (tid4) at (0.75,6){};
\draw[](tid3) -- (tid4);
\draw[](tid1) -- (tid3);
\node[circle, scale=0.75, fill, red] (tid2) at (2.25,3){};
\draw[](tid0) -- (tid1);
\draw[](tid0) -- (tid2);
\end{tikzpicture}
\nodepart{two}
\footnotesize{4.125}
\nodepart{three}
\footnotesize{$50\:50$}
};
 & 
\node[draw=black, rectangle split,  rectangle split parts=3] (sn0x145eb50){
\begin{tikzpicture}[scale=.2]
\node[circle, scale=0.75, fill] (tid0) at (1.5,1.5){};
\node[circle, scale=0.75, fill] (tid1) at (0.75,3){};
\node[circle, scale=0.75, fill, red] (tid3) at (0.75,4.5){};
\draw[](tid1) -- (tid3);
\node[circle, scale=0.75, fill] (tid2) at (2.25,3){};
\node[circle, scale=0.75, fill, red] (tid4) at (2.25,4.5){};
\draw[](tid2) -- (tid4);
\draw[](tid0) -- (tid1);
\draw[](tid0) -- (tid2);
\end{tikzpicture}
\nodepart{two}
\footnotesize{3.75}
\nodepart{three}
\footnotesize{$1$}
};
 & 
\\
};
\end{scope}
\begin{scope}[yshift=\leveltopIIIIIIII cm]
\matrix (line8) [column sep=1cm] {
\node[draw=black, rectangle split,  rectangle split parts=3] (sn0x145cd20){
\begin{tikzpicture}[scale=.2]
\node[circle, scale=0.75, fill] (tid0) at (0.75,1.5){};
\node[circle, scale=0.75, fill] (tid1) at (0.75,3){};
\node[circle, scale=0.75, fill] (tid2) at (0.75,4.5){};
\node[circle, scale=0.75, fill, red] (tid3) at (0.75,6){};
\draw[](tid2) -- (tid3);
\draw[](tid1) -- (tid2);
\draw[](tid0) -- (tid1);
\end{tikzpicture}
\nodepart{two}
\footnotesize{4}
\nodepart{three}
\footnotesize{$1$}
};
 & 
\node[draw=black, rectangle split,  rectangle split parts=3] (sn0x145d820){
\begin{tikzpicture}[scale=.2]
\node[circle, scale=0.75, fill] (tid0) at (1.5,1.5){};
\node[circle, scale=0.75, fill] (tid1) at (0.75,3){};
\node[circle, scale=0.75, fill, red] (tid3) at (0.75,4.5){};
\draw[](tid1) -- (tid3);
\node[circle, scale=0.75, fill, red] (tid2) at (2.25,3){};
\draw[](tid0) -- (tid1);
\draw[](tid0) -- (tid2);
\end{tikzpicture}
\nodepart{two}
\footnotesize{3.25}
\nodepart{three}
\footnotesize{$50\:50$}
};
 & 
\\
};
\end{scope}
\begin{scope}[yshift=\leveltopIIIIIIIII cm]
\matrix (line9) [column sep=1cm] {
\node[draw=black, rectangle split,  rectangle split parts=3] (sn0x145b6b0){
\begin{tikzpicture}[scale=.2]
\node[circle, scale=0.75, fill] (tid0) at (0.75,1.5){};
\node[circle, scale=0.75, fill] (tid1) at (0.75,3){};
\node[circle, scale=0.75, fill, red] (tid2) at (0.75,4.5){};
\draw[](tid1) -- (tid2);
\draw[](tid0) -- (tid1);
\end{tikzpicture}
\nodepart{two}
\footnotesize{3}
\nodepart{three}
\footnotesize{$1$}
};
 & 
\node[draw=black, rectangle split,  rectangle split parts=3] (sn0x145d710){
\begin{tikzpicture}[scale=.2]
\node[circle, scale=0.75, fill] (tid0) at (1.5,1.5){};
\node[circle, scale=0.75, fill, red] (tid1) at (0.75,3){};
\node[circle, scale=0.75, fill, red] (tid2) at (2.25,3){};
\draw[](tid0) -- (tid1);
\draw[](tid0) -- (tid2);
\end{tikzpicture}
\nodepart{two}
\footnotesize{2.5}
\nodepart{three}
\footnotesize{$1$}
};
 & 
\\
};
\end{scope}
\begin{scope}[yshift=\leveltopIIIIIIIIII cm]
\matrix (line10) [column sep=1cm] {
\node[draw=black, rectangle split,  rectangle split parts=3] (sn0x145aea0){
\begin{tikzpicture}[scale=.2]
\node[circle, scale=0.75, fill] (tid0) at (0.75,1.5){};
\node[circle, scale=0.75, fill, red] (tid1) at (0.75,3){};
\draw[](tid0) -- (tid1);
\end{tikzpicture}
\nodepart{two}
\footnotesize{2}
\nodepart{three}
\footnotesize{$1$}
};
 & 
\\
};
\end{scope}
\begin{scope}[yshift=\leveltopIIIIIIIIIII cm]
\matrix (line11) [column sep=1cm] {
\node[draw=black, rectangle split,  rectangle split parts=3] (sn0x145b5e0){
\begin{tikzpicture}[scale=.2]
\node[circle, scale=0.75, fill, red] (tid0) at (0.75,1.5){};
\end{tikzpicture}
\nodepart{two}
\footnotesize{1}
\nodepart{three}
\footnotesize{$$}
};
 & 
\\
};
\end{scope}
\begin{scope}[yshift=\leveltopIIIIIIIIIIII cm]
\matrix (line12) [column sep=1cm] {
\\
};
\end{scope}
\draw (sn0x14639f0.south) -- (sn0x1460d30.north);
\draw (sn0x14639f0.south) -- (sn0x14633d0.north);
\draw (sn0x1460d30.south) -- (sn0x145fad0.north);
\draw (sn0x1460d30.south) -- (sn0x1460050.north);
\draw (sn0x14633d0.south) -- (sn0x1462fe0.north);
\draw (sn0x14633d0.south) -- (sn0x1460050.north);
\draw (sn0x14633d0.south) -- (sn0x1462bf0.north);
\draw (sn0x145fad0.south) -- (sn0x145e900.north);
\draw (sn0x145fad0.south) -- (sn0x145fa00.north);
\draw (sn0x1460050.south) -- (sn0x145fa00.north);
\draw (sn0x1460050.south) -- (sn0x145ff80.north);
\draw (sn0x1462fe0.south) -- (sn0x1462420.north);
\draw (sn0x1462fe0.south) -- (sn0x145fa00.north);
\draw (sn0x1462fe0.south) -- (sn0x14624f0.north);
\draw (sn0x1462bf0.south) -- (sn0x14624f0.north);
\draw (sn0x1462bf0.south) -- (sn0x145ff80.north);
\draw (sn0x145e900.south) -- (sn0x145d480.north);
\draw (sn0x145e900.south) -- (sn0x145e670.north);
\draw (sn0x145fa00.south) -- (sn0x145e670.north);
\draw (sn0x145fa00.south) -- (sn0x145f770.north);
\draw (sn0x145ff80.south) -- (sn0x145f770.north);
\draw (sn0x145ff80.south) -- (sn0x1460b00.north);
\draw (sn0x1462420.south) -- (sn0x1462350.north);
\draw (sn0x1462420.south) -- (sn0x145e670.north);
\draw (sn0x1462420.south) -- (sn0x14628e0.north);
\draw (sn0x14624f0.south) -- (sn0x14628e0.north);
\draw (sn0x14624f0.south) -- (sn0x145f770.north);
\draw (sn0x145d480.south) -- (sn0x145d0f0.north);
\draw (sn0x145e670.south) -- (sn0x145d0f0.north);
\draw (sn0x145e670.south) -- (sn0x145e380.north);
\draw (sn0x145f770.south) -- (sn0x145e380.north);
\draw (sn0x145f770.south) -- (sn0x145ec40.north);
\draw (sn0x1460b00.south) -- (sn0x145ec40.north);
\draw (sn0x1462350.south) -- (sn0x145d0f0.north);
\draw (sn0x1462350.south) -- (sn0x1461150.north);
\draw (sn0x14628e0.south) -- (sn0x1461150.north);
\draw (sn0x14628e0.south) -- (sn0x145e380.north);
\draw (sn0x145d0f0.south) -- (sn0x145cdf0.north);
\draw (sn0x145e380.south) -- (sn0x145cdf0.north);
\draw (sn0x145e380.south) -- (sn0x145e0d0.north);
\draw (sn0x145ec40.south) -- (sn0x145e0d0.north);
\draw (sn0x145ec40.south) -- (sn0x145eb50.north);
\draw (sn0x1461150.south) -- (sn0x145cdf0.north);
\draw (sn0x145cdf0.south) -- (sn0x145cd20.north);
\draw (sn0x145e0d0.south) -- (sn0x145cd20.north);
\draw (sn0x145e0d0.south) -- (sn0x145d820.north);
\draw (sn0x145eb50.south) -- (sn0x145d820.north);
\draw (sn0x145cd20.south) -- (sn0x145b6b0.north);
\draw (sn0x145d820.south) -- (sn0x145b6b0.north);
\draw (sn0x145d820.south) -- (sn0x145d710.north);
\draw (sn0x145b6b0.south) -- (sn0x145aea0.north);
\draw (sn0x145d710.south) -- (sn0x145aea0.north);
\draw (sn0x145aea0.south) -- (sn0x145b5e0.north);
\end{tikzpicture}

%%% Local Variables:
%%% TeX-master: "thesis/thesis.tex"
%%% End: 

  \renewcommand{\leveltopI}{-15cm + \leveltop}
\renewcommand{\leveltopII}{-15cm + \leveltopI}
\renewcommand{\leveltopIII}{-15cm + \leveltopII}
\renewcommand{\leveltopIIII}{-15cm + \leveltopIII}
\renewcommand{\leveltopIIIII}{-15cm + \leveltopIIII}
\renewcommand{\leveltopIIIIII}{-15cm + \leveltopIIIII}
\renewcommand{\leveltopIIIIIII}{-15cm + \leveltopIIIIII}
\renewcommand{\leveltopIIIIIIII}{-15cm + \leveltopIIIIIII}
\renewcommand{\leveltopIIIIIIIII}{-15cm + \leveltopIIIIIIII}
\renewcommand{\leveltopIIIIIIIIII}{-15cm + \leveltopIIIIIIIII}
\renewcommand{\leveltopIIIIIIIIIII}{-15cm + \leveltopIIIIIIIIII}
\begin{tikzpicture}[scale=.2, anchor=south]
\begin{scope}[yshift=\leveltopI cm]
\matrix (line1) [column sep=1cm] {
\node[draw=black, rectangle split,  rectangle split parts=4] (sn0x18908d0){
\footnotesize{1}
\nodepart{two}
\begin{tikzpicture}[scale=.2]
\node[circle, scale=0.75, fill] (tid0) at (3,1.5){};
\node[circle, scale=0.75, fill] (tid1) at (2.25,3){};
\node[circle, scale=0.75, fill] (tid3) at (2.25,4.5){};
\node[circle, scale=0.75, fill] (tid5) at (2.25,6){};
\node[circle, scale=0.75, fill] (tid7) at (1.5,7.5){};
\node[circle, scale=0.75, fill, red] (tid9) at (0.75,9){};
\node[circle, scale=0.75, fill, red] (tid10) at (2.25,9){};
\draw[](tid7) -- (tid9);
\draw[](tid7) -- (tid10);
\node[circle, scale=0.75, fill] (tid8) at (3.75,7.5){};
\draw[](tid5) -- (tid7);
\draw[](tid5) -- (tid8);
\draw[](tid3) -- (tid5);
\draw[](tid1) -- (tid3);
\node[circle, scale=0.75, fill] (tid2) at (5.25,3){};
\node[circle, scale=0.75, fill] (tid4) at (5.25,4.5){};
\node[circle, scale=0.75, fill, red] (tid6) at (5.25,6){};
\draw[](tid4) -- (tid6);
\draw[](tid2) -- (tid4);
\draw[](tid0) -- (tid1);
\draw[](tid0) -- (tid2);
\end{tikzpicture}
\nodepart{three}
\footnotesize{6.96753}
\nodepart{four}
\footnotesize{$33\:67$}
};
 & 
\\
};
\end{scope}
\begin{scope}[yshift=\leveltopII cm]
\matrix (line2) [column sep=1cm] {
\node[draw=black, rectangle split,  rectangle split parts=4] (sn0x188ec30){
\footnotesize{0.333333}
\nodepart{two}
\begin{tikzpicture}[scale=.2]
\node[circle, scale=0.75, fill] (tid0) at (3,1.5){};
\node[circle, scale=0.75, fill] (tid1) at (2.25,3){};
\node[circle, scale=0.75, fill] (tid3) at (2.25,4.5){};
\node[circle, scale=0.75, fill] (tid5) at (2.25,6){};
\node[circle, scale=0.75, fill] (tid6) at (1.5,7.5){};
\node[circle, scale=0.75, fill, red] (tid8) at (0.75,9){};
\node[circle, scale=0.75, fill, red] (tid9) at (2.25,9){};
\draw[](tid6) -- (tid8);
\draw[](tid6) -- (tid9);
\node[circle, scale=0.75, fill, red] (tid7) at (3.75,7.5){};
\draw[](tid5) -- (tid6);
\draw[](tid5) -- (tid7);
\draw[](tid3) -- (tid5);
\draw[](tid1) -- (tid3);
\node[circle, scale=0.75, fill] (tid2) at (5.25,3){};
\node[circle, scale=0.75, fill] (tid4) at (5.25,4.5){};
\draw[](tid2) -- (tid4);
\draw[](tid0) -- (tid1);
\draw[](tid0) -- (tid2);
\end{tikzpicture}
\nodepart{three}
\footnotesize{6.77701}
\nodepart{four}
\footnotesize{$33\:67$}
};
 & 
\node[draw=black, rectangle split,  rectangle split parts=4] (sn0x18906a0){
\footnotesize{0.666667}
\nodepart{two}
\begin{tikzpicture}[scale=.2]
\node[circle, scale=0.75, fill] (tid0) at (2.25,1.5){};
\node[circle, scale=0.75, fill] (tid1) at (1.5,3){};
\node[circle, scale=0.75, fill] (tid3) at (1.5,4.5){};
\node[circle, scale=0.75, fill] (tid5) at (1.5,6){};
\node[circle, scale=0.75, fill] (tid7) at (0.75,7.5){};
\node[circle, scale=0.75, fill, red] (tid9) at (0.75,9){};
\draw[](tid7) -- (tid9);
\node[circle, scale=0.75, fill, red] (tid8) at (2.25,7.5){};
\draw[](tid5) -- (tid7);
\draw[](tid5) -- (tid8);
\draw[](tid3) -- (tid5);
\draw[](tid1) -- (tid3);
\node[circle, scale=0.75, fill] (tid2) at (3.75,3){};
\node[circle, scale=0.75, fill] (tid4) at (3.75,4.5){};
\node[circle, scale=0.75, fill, red] (tid6) at (3.75,6){};
\draw[](tid4) -- (tid6);
\draw[](tid2) -- (tid4);
\draw[](tid0) -- (tid1);
\draw[](tid0) -- (tid2);
\end{tikzpicture}
\nodepart{three}
\footnotesize{6.56279}
\nodepart{four}
\footnotesize{$33\:33\:33$}
};
 & 
\\
};
\end{scope}
\begin{scope}[yshift=\leveltopIII cm]
\matrix (line3) [column sep=1cm] {
\node[draw=black, rectangle split,  rectangle split parts=4] (sn0x188e220){
\footnotesize{0.111111}
\nodepart{two}
\begin{tikzpicture}[scale=.2]
\node[circle, scale=0.75, fill] (tid0) at (2.25,1.5){};
\node[circle, scale=0.75, fill] (tid1) at (1.5,3){};
\node[circle, scale=0.75, fill] (tid3) at (1.5,4.5){};
\node[circle, scale=0.75, fill] (tid5) at (1.5,6){};
\node[circle, scale=0.75, fill] (tid6) at (1.5,7.5){};
\node[circle, scale=0.75, fill, red] (tid7) at (0.75,9){};
\node[circle, scale=0.75, fill, red] (tid8) at (2.25,9){};
\draw[](tid6) -- (tid7);
\draw[](tid6) -- (tid8);
\draw[](tid5) -- (tid6);
\draw[](tid3) -- (tid5);
\draw[](tid1) -- (tid3);
\node[circle, scale=0.75, fill] (tid2) at (3.75,3){};
\node[circle, scale=0.75, fill, red] (tid4) at (3.75,4.5){};
\draw[](tid2) -- (tid4);
\draw[](tid0) -- (tid1);
\draw[](tid0) -- (tid2);
\end{tikzpicture}
\nodepart{three}
\footnotesize{6.60069}
\nodepart{four}
\footnotesize{$33\:67$}
};
 & 
\node[draw=black, rectangle split,  rectangle split parts=4] (sn0x188d8a0){
\footnotesize{0.444444}
\nodepart{two}
\begin{tikzpicture}[scale=.2]
\node[circle, scale=0.75, fill] (tid0) at (2.25,1.5){};
\node[circle, scale=0.75, fill] (tid1) at (1.5,3){};
\node[circle, scale=0.75, fill] (tid3) at (1.5,4.5){};
\node[circle, scale=0.75, fill] (tid5) at (1.5,6){};
\node[circle, scale=0.75, fill] (tid6) at (0.75,7.5){};
\node[circle, scale=0.75, fill, red] (tid8) at (0.75,9){};
\draw[](tid6) -- (tid8);
\node[circle, scale=0.75, fill, red] (tid7) at (2.25,7.5){};
\draw[](tid5) -- (tid6);
\draw[](tid5) -- (tid7);
\draw[](tid3) -- (tid5);
\draw[](tid1) -- (tid3);
\node[circle, scale=0.75, fill] (tid2) at (3.75,3){};
\node[circle, scale=0.75, fill, red] (tid4) at (3.75,4.5){};
\draw[](tid2) -- (tid4);
\draw[](tid0) -- (tid1);
\draw[](tid0) -- (tid2);
\end{tikzpicture}
\nodepart{three}
\footnotesize{6.36516}
\nodepart{four}
\footnotesize{$33\:33\:33$}
};
 & 
\node[draw=black, rectangle split,  rectangle split parts=4] (sn0x188f0f0){
\footnotesize{0.222222}
\nodepart{two}
\begin{tikzpicture}[scale=.2]
\node[circle, scale=0.75, fill] (tid0) at (1.5,1.5){};
\node[circle, scale=0.75, fill] (tid1) at (0.75,3){};
\node[circle, scale=0.75, fill] (tid3) at (0.75,4.5){};
\node[circle, scale=0.75, fill] (tid5) at (0.75,6){};
\node[circle, scale=0.75, fill] (tid7) at (0.75,7.5){};
\node[circle, scale=0.75, fill, red] (tid8) at (0.75,9){};
\draw[](tid7) -- (tid8);
\draw[](tid5) -- (tid7);
\draw[](tid3) -- (tid5);
\draw[](tid1) -- (tid3);
\node[circle, scale=0.75, fill] (tid2) at (2.25,3){};
\node[circle, scale=0.75, fill] (tid4) at (2.25,4.5){};
\node[circle, scale=0.75, fill, red] (tid6) at (2.25,6){};
\draw[](tid4) -- (tid6);
\draw[](tid2) -- (tid4);
\draw[](tid0) -- (tid1);
\draw[](tid0) -- (tid2);
\end{tikzpicture}
\nodepart{three}
\footnotesize{6.36719}
\nodepart{four}
\footnotesize{$50\:50$}
};
 & 
\node[draw=black, rectangle split,  rectangle split parts=4] (sn0x188ff90){
\footnotesize{0.222222}
\nodepart{two}
\begin{tikzpicture}[scale=.2]
\node[circle, scale=0.75, fill] (tid0) at (2.25,1.5){};
\node[circle, scale=0.75, fill] (tid1) at (1.5,3){};
\node[circle, scale=0.75, fill] (tid3) at (1.5,4.5){};
\node[circle, scale=0.75, fill] (tid5) at (1.5,6){};
\node[circle, scale=0.75, fill, red] (tid7) at (0.75,7.5){};
\node[circle, scale=0.75, fill, red] (tid8) at (2.25,7.5){};
\draw[](tid5) -- (tid7);
\draw[](tid5) -- (tid8);
\draw[](tid3) -- (tid5);
\draw[](tid1) -- (tid3);
\node[circle, scale=0.75, fill] (tid2) at (3.75,3){};
\node[circle, scale=0.75, fill] (tid4) at (3.75,4.5){};
\node[circle, scale=0.75, fill, red] (tid6) at (3.75,6){};
\draw[](tid4) -- (tid6);
\draw[](tid2) -- (tid4);
\draw[](tid0) -- (tid1);
\draw[](tid0) -- (tid2);
\end{tikzpicture}
\nodepart{three}
\footnotesize{5.95602}
\nodepart{four}
\footnotesize{$33\:67$}
};
 & 
\\
};
\end{scope}
\begin{scope}[yshift=\leveltopIIII cm]
\matrix (line4) [column sep=1cm] {
\node[draw=black, rectangle split,  rectangle split parts=4] (sn0x188c370){
\footnotesize{0.037037}
\nodepart{two}
\begin{tikzpicture}[scale=.2]
\node[circle, scale=0.75, fill] (tid0) at (2.25,1.5){};
\node[circle, scale=0.75, fill] (tid1) at (1.5,3){};
\node[circle, scale=0.75, fill] (tid3) at (1.5,4.5){};
\node[circle, scale=0.75, fill] (tid4) at (1.5,6){};
\node[circle, scale=0.75, fill] (tid5) at (1.5,7.5){};
\node[circle, scale=0.75, fill, red] (tid6) at (0.75,9){};
\node[circle, scale=0.75, fill, red] (tid7) at (2.25,9){};
\draw[](tid5) -- (tid6);
\draw[](tid5) -- (tid7);
\draw[](tid4) -- (tid5);
\draw[](tid3) -- (tid4);
\draw[](tid1) -- (tid3);
\node[circle, scale=0.75, fill, red] (tid2) at (3.75,3){};
\draw[](tid0) -- (tid1);
\draw[](tid0) -- (tid2);
\end{tikzpicture}
\nodepart{three}
\footnotesize{6.52083}
\nodepart{four}
\footnotesize{$33\:67$}
};
 & 
\node[draw=black, rectangle split,  rectangle split parts=4] (sn0x188d690){
\footnotesize{0.333333}
\nodepart{two}
\begin{tikzpicture}[scale=.2]
\node[circle, scale=0.75, fill] (tid0) at (1.5,1.5){};
\node[circle, scale=0.75, fill] (tid1) at (0.75,3){};
\node[circle, scale=0.75, fill] (tid3) at (0.75,4.5){};
\node[circle, scale=0.75, fill] (tid5) at (0.75,6){};
\node[circle, scale=0.75, fill] (tid6) at (0.75,7.5){};
\node[circle, scale=0.75, fill, red] (tid7) at (0.75,9){};
\draw[](tid6) -- (tid7);
\draw[](tid5) -- (tid6);
\draw[](tid3) -- (tid5);
\draw[](tid1) -- (tid3);
\node[circle, scale=0.75, fill] (tid2) at (2.25,3){};
\node[circle, scale=0.75, fill, red] (tid4) at (2.25,4.5){};
\draw[](tid2) -- (tid4);
\draw[](tid0) -- (tid1);
\draw[](tid0) -- (tid2);
\end{tikzpicture}
\nodepart{three}
\footnotesize{6.14062}
\nodepart{four}
\footnotesize{$50\:50$}
};
 & 
\node[draw=black, rectangle split,  rectangle split parts=4] (sn0x188b180){
\footnotesize{0.148148}
\nodepart{two}
\begin{tikzpicture}[scale=.2]
\node[circle, scale=0.75, fill] (tid0) at (2.25,1.5){};
\node[circle, scale=0.75, fill] (tid1) at (1.5,3){};
\node[circle, scale=0.75, fill] (tid3) at (1.5,4.5){};
\node[circle, scale=0.75, fill] (tid4) at (1.5,6){};
\node[circle, scale=0.75, fill] (tid5) at (0.75,7.5){};
\node[circle, scale=0.75, fill, red] (tid7) at (0.75,9){};
\draw[](tid5) -- (tid7);
\node[circle, scale=0.75, fill, red] (tid6) at (2.25,7.5){};
\draw[](tid4) -- (tid5);
\draw[](tid4) -- (tid6);
\draw[](tid3) -- (tid4);
\draw[](tid1) -- (tid3);
\node[circle, scale=0.75, fill, red] (tid2) at (3.75,3){};
\draw[](tid0) -- (tid1);
\draw[](tid0) -- (tid2);
\end{tikzpicture}
\nodepart{three}
\footnotesize{6.27431}
\nodepart{four}
\footnotesize{$33\:33\:33$}
};
 & 
\node[draw=black, rectangle split,  rectangle split parts=4] (sn0x188de90){
\footnotesize{0.222222}
\nodepart{two}
\begin{tikzpicture}[scale=.2]
\node[circle, scale=0.75, fill] (tid0) at (2.25,1.5){};
\node[circle, scale=0.75, fill] (tid1) at (1.5,3){};
\node[circle, scale=0.75, fill] (tid3) at (1.5,4.5){};
\node[circle, scale=0.75, fill] (tid5) at (1.5,6){};
\node[circle, scale=0.75, fill, red] (tid6) at (0.75,7.5){};
\node[circle, scale=0.75, fill, red] (tid7) at (2.25,7.5){};
\draw[](tid5) -- (tid6);
\draw[](tid5) -- (tid7);
\draw[](tid3) -- (tid5);
\draw[](tid1) -- (tid3);
\node[circle, scale=0.75, fill] (tid2) at (3.75,3){};
\node[circle, scale=0.75, fill, red] (tid4) at (3.75,4.5){};
\draw[](tid2) -- (tid4);
\draw[](tid0) -- (tid1);
\draw[](tid0) -- (tid2);
\end{tikzpicture}
\nodepart{three}
\footnotesize{5.68056}
\nodepart{four}
\footnotesize{$67\:33$}
};
 & 
\node[draw=black, rectangle split,  rectangle split parts=4] (sn0x188f020){
\footnotesize{0.259259}
\nodepart{two}
\begin{tikzpicture}[scale=.2]
\node[circle, scale=0.75, fill] (tid0) at (1.5,1.5){};
\node[circle, scale=0.75, fill] (tid1) at (0.75,3){};
\node[circle, scale=0.75, fill] (tid3) at (0.75,4.5){};
\node[circle, scale=0.75, fill] (tid5) at (0.75,6){};
\node[circle, scale=0.75, fill, red] (tid7) at (0.75,7.5){};
\draw[](tid5) -- (tid7);
\draw[](tid3) -- (tid5);
\draw[](tid1) -- (tid3);
\node[circle, scale=0.75, fill] (tid2) at (2.25,3){};
\node[circle, scale=0.75, fill] (tid4) at (2.25,4.5){};
\node[circle, scale=0.75, fill, red] (tid6) at (2.25,6){};
\draw[](tid4) -- (tid6);
\draw[](tid2) -- (tid4);
\draw[](tid0) -- (tid1);
\draw[](tid0) -- (tid2);
\end{tikzpicture}
\nodepart{three}
\footnotesize{5.59375}
\nodepart{four}
\footnotesize{$50\:50$}
};
 & 
\\
};
\end{scope}
\begin{scope}[yshift=\leveltopIIIII cm]
\matrix (line5) [column sep=1cm] {
\node[draw=black, rectangle split,  rectangle split parts=4] (sn0x1888fa0){
\footnotesize{0.0123457}
\nodepart{two}
\begin{tikzpicture}[scale=.2]
\node[circle, scale=0.75, fill] (tid0) at (1.5,1.5){};
\node[circle, scale=0.75, fill] (tid1) at (1.5,3){};
\node[circle, scale=0.75, fill] (tid2) at (1.5,4.5){};
\node[circle, scale=0.75, fill] (tid3) at (1.5,6){};
\node[circle, scale=0.75, fill] (tid4) at (1.5,7.5){};
\node[circle, scale=0.75, fill, red] (tid5) at (0.75,9){};
\node[circle, scale=0.75, fill, red] (tid6) at (2.25,9){};
\draw[](tid4) -- (tid5);
\draw[](tid4) -- (tid6);
\draw[](tid3) -- (tid4);
\draw[](tid2) -- (tid3);
\draw[](tid1) -- (tid2);
\draw[](tid0) -- (tid1);
\end{tikzpicture}
\nodepart{three}
\footnotesize{6.5}
\nodepart{four}
\footnotesize{$1$}
};
 & 
\node[draw=black, rectangle split,  rectangle split parts=4] (sn0x188ae00){
\footnotesize{0.240741}
\nodepart{two}
\begin{tikzpicture}[scale=.2]
\node[circle, scale=0.75, fill] (tid0) at (1.5,1.5){};
\node[circle, scale=0.75, fill] (tid1) at (0.75,3){};
\node[circle, scale=0.75, fill] (tid3) at (0.75,4.5){};
\node[circle, scale=0.75, fill] (tid4) at (0.75,6){};
\node[circle, scale=0.75, fill] (tid5) at (0.75,7.5){};
\node[circle, scale=0.75, fill, red] (tid6) at (0.75,9){};
\draw[](tid5) -- (tid6);
\draw[](tid4) -- (tid5);
\draw[](tid3) -- (tid4);
\draw[](tid1) -- (tid3);
\node[circle, scale=0.75, fill, red] (tid2) at (2.25,3){};
\draw[](tid0) -- (tid1);
\draw[](tid0) -- (tid2);
\end{tikzpicture}
\nodepart{three}
\footnotesize{6.03125}
\nodepart{four}
\footnotesize{$50\:50$}
};
 & 
\node[draw=black, rectangle split,  rectangle split parts=4] (sn0x188d460){
\footnotesize{0.444444}
\nodepart{two}
\begin{tikzpicture}[scale=.2]
\node[circle, scale=0.75, fill] (tid0) at (1.5,1.5){};
\node[circle, scale=0.75, fill] (tid1) at (0.75,3){};
\node[circle, scale=0.75, fill] (tid3) at (0.75,4.5){};
\node[circle, scale=0.75, fill] (tid5) at (0.75,6){};
\node[circle, scale=0.75, fill, red] (tid6) at (0.75,7.5){};
\draw[](tid5) -- (tid6);
\draw[](tid3) -- (tid5);
\draw[](tid1) -- (tid3);
\node[circle, scale=0.75, fill] (tid2) at (2.25,3){};
\node[circle, scale=0.75, fill, red] (tid4) at (2.25,4.5){};
\draw[](tid2) -- (tid4);
\draw[](tid0) -- (tid1);
\draw[](tid0) -- (tid2);
\end{tikzpicture}
\nodepart{three}
\footnotesize{5.25}
\nodepart{four}
\footnotesize{$50\:50$}
};
 & 
\node[draw=black, rectangle split,  rectangle split parts=4] (sn0x1889a70){
\footnotesize{0.0493827}
\nodepart{two}
\begin{tikzpicture}[scale=.2]
\node[circle, scale=0.75, fill] (tid0) at (1.5,1.5){};
\node[circle, scale=0.75, fill] (tid1) at (1.5,3){};
\node[circle, scale=0.75, fill] (tid2) at (1.5,4.5){};
\node[circle, scale=0.75, fill] (tid3) at (1.5,6){};
\node[circle, scale=0.75, fill] (tid4) at (0.75,7.5){};
\node[circle, scale=0.75, fill, red] (tid6) at (0.75,9){};
\draw[](tid4) -- (tid6);
\node[circle, scale=0.75, fill, red] (tid5) at (2.25,7.5){};
\draw[](tid3) -- (tid4);
\draw[](tid3) -- (tid5);
\draw[](tid2) -- (tid3);
\draw[](tid1) -- (tid2);
\draw[](tid0) -- (tid1);
\end{tikzpicture}
\nodepart{three}
\footnotesize{6.25}
\nodepart{four}
\footnotesize{$50\:50$}
};
 & 
\node[draw=black, rectangle split,  rectangle split parts=4] (sn0x188b630){
\footnotesize{0.123457}
\nodepart{two}
\begin{tikzpicture}[scale=.2]
\node[circle, scale=0.75, fill] (tid0) at (2.25,1.5){};
\node[circle, scale=0.75, fill] (tid1) at (1.5,3){};
\node[circle, scale=0.75, fill] (tid3) at (1.5,4.5){};
\node[circle, scale=0.75, fill] (tid4) at (1.5,6){};
\node[circle, scale=0.75, fill, red] (tid5) at (0.75,7.5){};
\node[circle, scale=0.75, fill, red] (tid6) at (2.25,7.5){};
\draw[](tid4) -- (tid5);
\draw[](tid4) -- (tid6);
\draw[](tid3) -- (tid4);
\draw[](tid1) -- (tid3);
\node[circle, scale=0.75, fill, red] (tid2) at (3.75,3){};
\draw[](tid0) -- (tid1);
\draw[](tid0) -- (tid2);
\end{tikzpicture}
\nodepart{three}
\footnotesize{5.54167}
\nodepart{four}
\footnotesize{$67\:33$}
};
 & 
\node[draw=black, rectangle split,  rectangle split parts=4] (sn0x188fd60){
\footnotesize{0.12963}
\nodepart{two}
\begin{tikzpicture}[scale=.2]
\node[circle, scale=0.75, fill] (tid0) at (1.5,1.5){};
\node[circle, scale=0.75, fill] (tid1) at (0.75,3){};
\node[circle, scale=0.75, fill] (tid3) at (0.75,4.5){};
\node[circle, scale=0.75, fill, red] (tid5) at (0.75,6){};
\draw[](tid3) -- (tid5);
\draw[](tid1) -- (tid3);
\node[circle, scale=0.75, fill] (tid2) at (2.25,3){};
\node[circle, scale=0.75, fill] (tid4) at (2.25,4.5){};
\node[circle, scale=0.75, fill, red] (tid6) at (2.25,6){};
\draw[](tid4) -- (tid6);
\draw[](tid2) -- (tid4);
\draw[](tid0) -- (tid1);
\draw[](tid0) -- (tid2);
\end{tikzpicture}
\nodepart{three}
\footnotesize{4.9375}
\nodepart{four}
\footnotesize{$1$}
};
 & 
\\
};
\end{scope}
\begin{scope}[yshift=\leveltopIIIIII cm]
\matrix (line6) [column sep=1cm] {
\node[draw=black, rectangle split,  rectangle split parts=4] (sn0x1888e50){
\footnotesize{0.157407}
\nodepart{two}
\begin{tikzpicture}[scale=.2]
\node[circle, scale=0.75, fill] (tid0) at (0.75,1.5){};
\node[circle, scale=0.75, fill] (tid1) at (0.75,3){};
\node[circle, scale=0.75, fill] (tid2) at (0.75,4.5){};
\node[circle, scale=0.75, fill] (tid3) at (0.75,6){};
\node[circle, scale=0.75, fill] (tid4) at (0.75,7.5){};
\node[circle, scale=0.75, fill, red] (tid5) at (0.75,9){};
\draw[](tid4) -- (tid5);
\draw[](tid3) -- (tid4);
\draw[](tid2) -- (tid3);
\draw[](tid1) -- (tid2);
\draw[](tid0) -- (tid1);
\end{tikzpicture}
\nodepart{three}
\footnotesize{6}
\nodepart{four}
\footnotesize{$1$}
};
 & 
\node[draw=black, rectangle split,  rectangle split parts=4] (sn0x188acf0){
\footnotesize{0.424897}
\nodepart{two}
\begin{tikzpicture}[scale=.2]
\node[circle, scale=0.75, fill] (tid0) at (1.5,1.5){};
\node[circle, scale=0.75, fill] (tid1) at (0.75,3){};
\node[circle, scale=0.75, fill] (tid3) at (0.75,4.5){};
\node[circle, scale=0.75, fill] (tid4) at (0.75,6){};
\node[circle, scale=0.75, fill, red] (tid5) at (0.75,7.5){};
\draw[](tid4) -- (tid5);
\draw[](tid3) -- (tid4);
\draw[](tid1) -- (tid3);
\node[circle, scale=0.75, fill, red] (tid2) at (2.25,3){};
\draw[](tid0) -- (tid1);
\draw[](tid0) -- (tid2);
\end{tikzpicture}
\nodepart{three}
\footnotesize{5.0625}
\nodepart{four}
\footnotesize{$50\:50$}
};
 & 
\node[draw=black, rectangle split,  rectangle split parts=4] (sn0x188c6a0){
\footnotesize{0.351852}
\nodepart{two}
\begin{tikzpicture}[scale=.2]
\node[circle, scale=0.75, fill] (tid0) at (1.5,1.5){};
\node[circle, scale=0.75, fill] (tid1) at (0.75,3){};
\node[circle, scale=0.75, fill] (tid3) at (0.75,4.5){};
\node[circle, scale=0.75, fill, red] (tid5) at (0.75,6){};
\draw[](tid3) -- (tid5);
\draw[](tid1) -- (tid3);
\node[circle, scale=0.75, fill] (tid2) at (2.25,3){};
\node[circle, scale=0.75, fill, red] (tid4) at (2.25,4.5){};
\draw[](tid2) -- (tid4);
\draw[](tid0) -- (tid1);
\draw[](tid0) -- (tid2);
\end{tikzpicture}
\nodepart{three}
\footnotesize{4.4375}
\nodepart{four}
\footnotesize{$50\:50$}
};
 & 
\node[draw=black, rectangle split,  rectangle split parts=4] (sn0x1889230){
\footnotesize{0.0658436}
\nodepart{two}
\begin{tikzpicture}[scale=.2]
\node[circle, scale=0.75, fill] (tid0) at (1.5,1.5){};
\node[circle, scale=0.75, fill] (tid1) at (1.5,3){};
\node[circle, scale=0.75, fill] (tid2) at (1.5,4.5){};
\node[circle, scale=0.75, fill] (tid3) at (1.5,6){};
\node[circle, scale=0.75, fill, red] (tid4) at (0.75,7.5){};
\node[circle, scale=0.75, fill, red] (tid5) at (2.25,7.5){};
\draw[](tid3) -- (tid4);
\draw[](tid3) -- (tid5);
\draw[](tid2) -- (tid3);
\draw[](tid1) -- (tid2);
\draw[](tid0) -- (tid1);
\end{tikzpicture}
\nodepart{three}
\footnotesize{5.5}
\nodepart{four}
\footnotesize{$1$}
};
 & 
\\
};
\end{scope}
\begin{scope}[yshift=\leveltopIIIIIII cm]
\matrix (line7) [column sep=1cm] {
\node[draw=black, rectangle split,  rectangle split parts=4] (sn0x1888b50){
\footnotesize{0.4357}
\nodepart{two}
\begin{tikzpicture}[scale=.2]
\node[circle, scale=0.75, fill] (tid0) at (0.75,1.5){};
\node[circle, scale=0.75, fill] (tid1) at (0.75,3){};
\node[circle, scale=0.75, fill] (tid2) at (0.75,4.5){};
\node[circle, scale=0.75, fill] (tid3) at (0.75,6){};
\node[circle, scale=0.75, fill, red] (tid4) at (0.75,7.5){};
\draw[](tid3) -- (tid4);
\draw[](tid2) -- (tid3);
\draw[](tid1) -- (tid2);
\draw[](tid0) -- (tid1);
\end{tikzpicture}
\nodepart{three}
\footnotesize{5}
\nodepart{four}
\footnotesize{$1$}
};
 & 
\node[draw=black, rectangle split,  rectangle split parts=4] (sn0x188a9f0){
\footnotesize{0.388375}
\nodepart{two}
\begin{tikzpicture}[scale=.2]
\node[circle, scale=0.75, fill] (tid0) at (1.5,1.5){};
\node[circle, scale=0.75, fill] (tid1) at (0.75,3){};
\node[circle, scale=0.75, fill] (tid3) at (0.75,4.5){};
\node[circle, scale=0.75, fill, red] (tid4) at (0.75,6){};
\draw[](tid3) -- (tid4);
\draw[](tid1) -- (tid3);
\node[circle, scale=0.75, fill, red] (tid2) at (2.25,3){};
\draw[](tid0) -- (tid1);
\draw[](tid0) -- (tid2);
\end{tikzpicture}
\nodepart{three}
\footnotesize{4.125}
\nodepart{four}
\footnotesize{$50\:50$}
};
 & 
\node[draw=black, rectangle split,  rectangle split parts=4] (sn0x188c5b0){
\footnotesize{0.175926}
\nodepart{two}
\begin{tikzpicture}[scale=.2]
\node[circle, scale=0.75, fill] (tid0) at (1.5,1.5){};
\node[circle, scale=0.75, fill] (tid1) at (0.75,3){};
\node[circle, scale=0.75, fill, red] (tid3) at (0.75,4.5){};
\draw[](tid1) -- (tid3);
\node[circle, scale=0.75, fill] (tid2) at (2.25,3){};
\node[circle, scale=0.75, fill, red] (tid4) at (2.25,4.5){};
\draw[](tid2) -- (tid4);
\draw[](tid0) -- (tid1);
\draw[](tid0) -- (tid2);
\end{tikzpicture}
\nodepart{three}
\footnotesize{3.75}
\nodepart{four}
\footnotesize{$1$}
};
 & 
\\
};
\end{scope}
\begin{scope}[yshift=\leveltopIIIIIIII cm]
\matrix (line8) [column sep=1cm] {
\node[draw=black, rectangle split,  rectangle split parts=4] (sn0x1888900){
\footnotesize{0.629887}
\nodepart{two}
\begin{tikzpicture}[scale=.2]
\node[circle, scale=0.75, fill] (tid0) at (0.75,1.5){};
\node[circle, scale=0.75, fill] (tid1) at (0.75,3){};
\node[circle, scale=0.75, fill] (tid2) at (0.75,4.5){};
\node[circle, scale=0.75, fill, red] (tid3) at (0.75,6){};
\draw[](tid2) -- (tid3);
\draw[](tid1) -- (tid2);
\draw[](tid0) -- (tid1);
\end{tikzpicture}
\nodepart{three}
\footnotesize{4}
\nodepart{four}
\footnotesize{$1$}
};
 & 
\node[draw=black, rectangle split,  rectangle split parts=4] (sn0x188a920){
\footnotesize{0.370113}
\nodepart{two}
\begin{tikzpicture}[scale=.2]
\node[circle, scale=0.75, fill] (tid0) at (1.5,1.5){};
\node[circle, scale=0.75, fill] (tid1) at (0.75,3){};
\node[circle, scale=0.75, fill, red] (tid3) at (0.75,4.5){};
\draw[](tid1) -- (tid3);
\node[circle, scale=0.75, fill, red] (tid2) at (2.25,3){};
\draw[](tid0) -- (tid1);
\draw[](tid0) -- (tid2);
\end{tikzpicture}
\nodepart{three}
\footnotesize{3.25}
\nodepart{four}
\footnotesize{$50\:50$}
};
 & 
\\
};
\end{scope}
\begin{scope}[yshift=\leveltopIIIIIIIII cm]
\matrix (line9) [column sep=1cm] {
\node[draw=black, rectangle split,  rectangle split parts=4] (sn0x1888830){
\footnotesize{0.814943}
\nodepart{two}
\begin{tikzpicture}[scale=.2]
\node[circle, scale=0.75, fill] (tid0) at (0.75,1.5){};
\node[circle, scale=0.75, fill] (tid1) at (0.75,3){};
\node[circle, scale=0.75, fill, red] (tid2) at (0.75,4.5){};
\draw[](tid1) -- (tid2);
\draw[](tid0) -- (tid1);
\end{tikzpicture}
\nodepart{three}
\footnotesize{3}
\nodepart{four}
\footnotesize{$1$}
};
 & 
\node[draw=black, rectangle split,  rectangle split parts=4] (sn0x1889c10){
\footnotesize{0.185057}
\nodepart{two}
\begin{tikzpicture}[scale=.2]
\node[circle, scale=0.75, fill] (tid0) at (1.5,1.5){};
\node[circle, scale=0.75, fill, red] (tid1) at (0.75,3){};
\node[circle, scale=0.75, fill, red] (tid2) at (2.25,3){};
\draw[](tid0) -- (tid1);
\draw[](tid0) -- (tid2);
\end{tikzpicture}
\nodepart{three}
\footnotesize{2.5}
\nodepart{four}
\footnotesize{$1$}
};
 & 
\\
};
\end{scope}
\begin{scope}[yshift=\leveltopIIIIIIIIII cm]
\matrix (line10) [column sep=1cm] {
\node[draw=black, rectangle split,  rectangle split parts=4] (sn0x1887520){
\footnotesize{1}
\nodepart{two}
\begin{tikzpicture}[scale=.2]
\node[circle, scale=0.75, fill] (tid0) at (0.75,1.5){};
\node[circle, scale=0.75, fill, red] (tid1) at (0.75,3){};
\draw[](tid0) -- (tid1);
\end{tikzpicture}
\nodepart{three}
\footnotesize{2}
\nodepart{four}
\footnotesize{$1$}
};
 & 
\\
};
\end{scope}
\begin{scope}[yshift=\leveltopIIIIIIIIIII cm]
\matrix (line11) [column sep=1cm] {
\node[draw=black, rectangle split,  rectangle split parts=4] (sn0x1887450){
\footnotesize{1}
\nodepart{two}
\begin{tikzpicture}[scale=.2]
\node[circle, scale=0.75, fill, red] (tid0) at (0.75,1.5){};
\end{tikzpicture}
\nodepart{three}
\footnotesize{1}
\nodepart{four}
\footnotesize{$$}
};
 & 
\\
};
\end{scope}
\begin{scope}[yshift=\leveltopIIIIIIIIIIII cm]
\matrix (line12) [column sep=1cm] {
\\
};
\end{scope}
\draw (sn0x18908d0.south) -- (sn0x188ec30.north);
\draw (sn0x18908d0.south) -- (sn0x18906a0.north);
\draw (sn0x188ec30.south) -- (sn0x188e220.north);
\draw (sn0x188ec30.south) -- (sn0x188d8a0.north);
\draw (sn0x18906a0.south) -- (sn0x188d8a0.north);
\draw (sn0x18906a0.south) -- (sn0x188f0f0.north);
\draw (sn0x18906a0.south) -- (sn0x188ff90.north);
\draw (sn0x188e220.south) -- (sn0x188c370.north);
\draw (sn0x188e220.south) -- (sn0x188d690.north);
\draw (sn0x188d8a0.south) -- (sn0x188b180.north);
\draw (sn0x188d8a0.south) -- (sn0x188d690.north);
\draw (sn0x188d8a0.south) -- (sn0x188de90.north);
\draw (sn0x188f0f0.south) -- (sn0x188d690.north);
\draw (sn0x188f0f0.south) -- (sn0x188f020.north);
\draw (sn0x188ff90.south) -- (sn0x188de90.north);
\draw (sn0x188ff90.south) -- (sn0x188f020.north);
\draw (sn0x188c370.south) -- (sn0x1888fa0.north);
\draw (sn0x188c370.south) -- (sn0x188ae00.north);
\draw (sn0x188d690.south) -- (sn0x188ae00.north);
\draw (sn0x188d690.south) -- (sn0x188d460.north);
\draw (sn0x188b180.south) -- (sn0x1889a70.north);
\draw (sn0x188b180.south) -- (sn0x188ae00.north);
\draw (sn0x188b180.south) -- (sn0x188b630.north);
\draw (sn0x188de90.south) -- (sn0x188b630.north);
\draw (sn0x188de90.south) -- (sn0x188d460.north);
\draw (sn0x188f020.south) -- (sn0x188d460.north);
\draw (sn0x188f020.south) -- (sn0x188fd60.north);
\draw (sn0x1888fa0.south) -- (sn0x1888e50.north);
\draw (sn0x188ae00.south) -- (sn0x1888e50.north);
\draw (sn0x188ae00.south) -- (sn0x188acf0.north);
\draw (sn0x188d460.south) -- (sn0x188acf0.north);
\draw (sn0x188d460.south) -- (sn0x188c6a0.north);
\draw (sn0x1889a70.south) -- (sn0x1888e50.north);
\draw (sn0x1889a70.south) -- (sn0x1889230.north);
\draw (sn0x188b630.south) -- (sn0x1889230.north);
\draw (sn0x188b630.south) -- (sn0x188acf0.north);
\draw (sn0x188fd60.south) -- (sn0x188c6a0.north);
\draw (sn0x1888e50.south) -- (sn0x1888b50.north);
\draw (sn0x188acf0.south) -- (sn0x1888b50.north);
\draw (sn0x188acf0.south) -- (sn0x188a9f0.north);
\draw (sn0x188c6a0.south) -- (sn0x188a9f0.north);
\draw (sn0x188c6a0.south) -- (sn0x188c5b0.north);
\draw (sn0x1889230.south) -- (sn0x1888b50.north);
\draw (sn0x1888b50.south) -- (sn0x1888900.north);
\draw (sn0x188a9f0.south) -- (sn0x1888900.north);
\draw (sn0x188a9f0.south) -- (sn0x188a920.north);
\draw (sn0x188c5b0.south) -- (sn0x188a920.north);
\draw (sn0x1888900.south) -- (sn0x1888830.north);
\draw (sn0x188a920.south) -- (sn0x1888830.north);
\draw (sn0x188a920.south) -- (sn0x1889c10.north);
\draw (sn0x1888830.south) -- (sn0x1887520.north);
\draw (sn0x1889c10.south) -- (sn0x1887520.north);
\draw (sn0x1887520.south) -- (sn0x1887450.north);
\end{tikzpicture}

%%% Local Variables:
%%% TeX-master: "thesis/thesis.tex"
%%% End: 
\renewcommand{\leveltopI}{-15cm + \leveltop}
\renewcommand{\leveltopII}{-15cm + \leveltopI}
\renewcommand{\leveltopIII}{-15cm + \leveltopII}
\renewcommand{\leveltopIIII}{-15cm + \leveltopIII}
\renewcommand{\leveltopIIIII}{-15cm + \leveltopIIII}
\renewcommand{\leveltopIIIIII}{-15cm + \leveltopIIIII}
\renewcommand{\leveltopIIIIIII}{-15cm + \leveltopIIIIII}
\renewcommand{\leveltopIIIIIIII}{-15cm + \leveltopIIIIIII}
\renewcommand{\leveltopIIIIIIIII}{-15cm + \leveltopIIIIIIII}
\renewcommand{\leveltopIIIIIIIIII}{-15cm + \leveltopIIIIIIIII}
\renewcommand{\leveltopIIIIIIIIIII}{-15cm + \leveltopIIIIIIIIII}
\begin{tikzpicture}[scale=.2, anchor=south]
\begin{scope}[yshift=\leveltopI cm]
\matrix (line1) [column sep=1cm] {
\node[draw=black, rectangle split,  rectangle split parts=4] (sn0x1891fa0){
\footnotesize{1}
\nodepart{two}
\begin{tikzpicture}[scale=.2]
\node[circle, scale=0.75, fill] (tid0) at (3,1.5){};
\node[circle, scale=0.75, fill] (tid1) at (2.25,3){};
\node[circle, scale=0.75, fill] (tid3) at (2.25,4.5){};
\node[circle, scale=0.75, fill] (tid5) at (2.25,6){};
\node[circle, scale=0.75, fill] (tid7) at (1.5,7.5){};
\node[circle, scale=0.75, fill, red] (tid9) at (0.75,9){};
\node[circle, scale=0.75, fill] (tid10) at (2.25,9){};
\draw[](tid7) -- (tid9);
\draw[](tid7) -- (tid10);
\node[circle, scale=0.75, fill, red] (tid8) at (3.75,7.5){};
\draw[](tid5) -- (tid7);
\draw[](tid5) -- (tid8);
\draw[](tid3) -- (tid5);
\draw[](tid1) -- (tid3);
\node[circle, scale=0.75, fill] (tid2) at (5.25,3){};
\node[circle, scale=0.75, fill] (tid4) at (5.25,4.5){};
\node[circle, scale=0.75, fill, red] (tid6) at (5.25,6){};
\draw[](tid4) -- (tid6);
\draw[](tid2) -- (tid4);
\draw[](tid0) -- (tid1);
\draw[](tid0) -- (tid2);
\end{tikzpicture}
\nodepart{three}
\footnotesize{7.03938}
\nodepart{four}
\footnotesize{$33\:33\:33$}
};
 & 
\\
};
\end{scope}
\begin{scope}[yshift=\leveltopII cm]
\matrix (line2) [column sep=1cm] {
\node[draw=black, rectangle split,  rectangle split parts=4] (sn0x188ec30){
\footnotesize{0.333333}
\nodepart{two}
\begin{tikzpicture}[scale=.2]
\node[circle, scale=0.75, fill] (tid0) at (3,1.5){};
\node[circle, scale=0.75, fill] (tid1) at (2.25,3){};
\node[circle, scale=0.75, fill] (tid3) at (2.25,4.5){};
\node[circle, scale=0.75, fill] (tid5) at (2.25,6){};
\node[circle, scale=0.75, fill] (tid6) at (1.5,7.5){};
\node[circle, scale=0.75, fill, red] (tid8) at (0.75,9){};
\node[circle, scale=0.75, fill, red] (tid9) at (2.25,9){};
\draw[](tid6) -- (tid8);
\draw[](tid6) -- (tid9);
\node[circle, scale=0.75, fill, red] (tid7) at (3.75,7.5){};
\draw[](tid5) -- (tid6);
\draw[](tid5) -- (tid7);
\draw[](tid3) -- (tid5);
\draw[](tid1) -- (tid3);
\node[circle, scale=0.75, fill] (tid2) at (5.25,3){};
\node[circle, scale=0.75, fill] (tid4) at (5.25,4.5){};
\draw[](tid2) -- (tid4);
\draw[](tid0) -- (tid1);
\draw[](tid0) -- (tid2);
\end{tikzpicture}
\nodepart{three}
\footnotesize{6.77701}
\nodepart{four}
\footnotesize{$33\:67$}
};
 & 
\node[draw=black, rectangle split,  rectangle split parts=4] (sn0x1891ba0){
\footnotesize{0.333333}
\nodepart{two}
\begin{tikzpicture}[scale=.2]
\node[circle, scale=0.75, fill] (tid0) at (2.25,1.5){};
\node[circle, scale=0.75, fill] (tid1) at (1.5,3){};
\node[circle, scale=0.75, fill] (tid3) at (1.5,4.5){};
\node[circle, scale=0.75, fill] (tid5) at (1.5,6){};
\node[circle, scale=0.75, fill] (tid7) at (1.5,7.5){};
\node[circle, scale=0.75, fill, red] (tid8) at (0.75,9){};
\node[circle, scale=0.75, fill, red] (tid9) at (2.25,9){};
\draw[](tid7) -- (tid8);
\draw[](tid7) -- (tid9);
\draw[](tid5) -- (tid7);
\draw[](tid3) -- (tid5);
\draw[](tid1) -- (tid3);
\node[circle, scale=0.75, fill] (tid2) at (3.75,3){};
\node[circle, scale=0.75, fill] (tid4) at (3.75,4.5){};
\node[circle, scale=0.75, fill, red] (tid6) at (3.75,6){};
\draw[](tid4) -- (tid6);
\draw[](tid2) -- (tid4);
\draw[](tid0) -- (tid1);
\draw[](tid0) -- (tid2);
\end{tikzpicture}
\nodepart{three}
\footnotesize{6.77836}
\nodepart{four}
\footnotesize{$33\:67$}
};
 & 
\node[draw=black, rectangle split,  rectangle split parts=4] (sn0x18906a0){
\footnotesize{0.333333}
\nodepart{two}
\begin{tikzpicture}[scale=.2]
\node[circle, scale=0.75, fill] (tid0) at (2.25,1.5){};
\node[circle, scale=0.75, fill] (tid1) at (1.5,3){};
\node[circle, scale=0.75, fill] (tid3) at (1.5,4.5){};
\node[circle, scale=0.75, fill] (tid5) at (1.5,6){};
\node[circle, scale=0.75, fill] (tid7) at (0.75,7.5){};
\node[circle, scale=0.75, fill, red] (tid9) at (0.75,9){};
\draw[](tid7) -- (tid9);
\node[circle, scale=0.75, fill, red] (tid8) at (2.25,7.5){};
\draw[](tid5) -- (tid7);
\draw[](tid5) -- (tid8);
\draw[](tid3) -- (tid5);
\draw[](tid1) -- (tid3);
\node[circle, scale=0.75, fill] (tid2) at (3.75,3){};
\node[circle, scale=0.75, fill] (tid4) at (3.75,4.5){};
\node[circle, scale=0.75, fill, red] (tid6) at (3.75,6){};
\draw[](tid4) -- (tid6);
\draw[](tid2) -- (tid4);
\draw[](tid0) -- (tid1);
\draw[](tid0) -- (tid2);
\end{tikzpicture}
\nodepart{three}
\footnotesize{6.56279}
\nodepart{four}
\footnotesize{$33\:33\:33$}
};
 & 
\\
};
\end{scope}
\begin{scope}[yshift=\leveltopIII cm]
\matrix (line3) [column sep=1cm] {
\node[draw=black, rectangle split,  rectangle split parts=4] (sn0x188e220){
\footnotesize{0.222222}
\nodepart{two}
\begin{tikzpicture}[scale=.2]
\node[circle, scale=0.75, fill] (tid0) at (2.25,1.5){};
\node[circle, scale=0.75, fill] (tid1) at (1.5,3){};
\node[circle, scale=0.75, fill] (tid3) at (1.5,4.5){};
\node[circle, scale=0.75, fill] (tid5) at (1.5,6){};
\node[circle, scale=0.75, fill] (tid6) at (1.5,7.5){};
\node[circle, scale=0.75, fill, red] (tid7) at (0.75,9){};
\node[circle, scale=0.75, fill, red] (tid8) at (2.25,9){};
\draw[](tid6) -- (tid7);
\draw[](tid6) -- (tid8);
\draw[](tid5) -- (tid6);
\draw[](tid3) -- (tid5);
\draw[](tid1) -- (tid3);
\node[circle, scale=0.75, fill] (tid2) at (3.75,3){};
\node[circle, scale=0.75, fill, red] (tid4) at (3.75,4.5){};
\draw[](tid2) -- (tid4);
\draw[](tid0) -- (tid1);
\draw[](tid0) -- (tid2);
\end{tikzpicture}
\nodepart{three}
\footnotesize{6.60069}
\nodepart{four}
\footnotesize{$33\:67$}
};
 & 
\node[draw=black, rectangle split,  rectangle split parts=4] (sn0x188d8a0){
\footnotesize{0.333333}
\nodepart{two}
\begin{tikzpicture}[scale=.2]
\node[circle, scale=0.75, fill] (tid0) at (2.25,1.5){};
\node[circle, scale=0.75, fill] (tid1) at (1.5,3){};
\node[circle, scale=0.75, fill] (tid3) at (1.5,4.5){};
\node[circle, scale=0.75, fill] (tid5) at (1.5,6){};
\node[circle, scale=0.75, fill] (tid6) at (0.75,7.5){};
\node[circle, scale=0.75, fill, red] (tid8) at (0.75,9){};
\draw[](tid6) -- (tid8);
\node[circle, scale=0.75, fill, red] (tid7) at (2.25,7.5){};
\draw[](tid5) -- (tid6);
\draw[](tid5) -- (tid7);
\draw[](tid3) -- (tid5);
\draw[](tid1) -- (tid3);
\node[circle, scale=0.75, fill] (tid2) at (3.75,3){};
\node[circle, scale=0.75, fill, red] (tid4) at (3.75,4.5){};
\draw[](tid2) -- (tid4);
\draw[](tid0) -- (tid1);
\draw[](tid0) -- (tid2);
\end{tikzpicture}
\nodepart{three}
\footnotesize{6.36516}
\nodepart{four}
\footnotesize{$33\:33\:33$}
};
 & 
\node[draw=black, rectangle split,  rectangle split parts=4] (sn0x188f0f0){
\footnotesize{0.333333}
\nodepart{two}
\begin{tikzpicture}[scale=.2]
\node[circle, scale=0.75, fill] (tid0) at (1.5,1.5){};
\node[circle, scale=0.75, fill] (tid1) at (0.75,3){};
\node[circle, scale=0.75, fill] (tid3) at (0.75,4.5){};
\node[circle, scale=0.75, fill] (tid5) at (0.75,6){};
\node[circle, scale=0.75, fill] (tid7) at (0.75,7.5){};
\node[circle, scale=0.75, fill, red] (tid8) at (0.75,9){};
\draw[](tid7) -- (tid8);
\draw[](tid5) -- (tid7);
\draw[](tid3) -- (tid5);
\draw[](tid1) -- (tid3);
\node[circle, scale=0.75, fill] (tid2) at (2.25,3){};
\node[circle, scale=0.75, fill] (tid4) at (2.25,4.5){};
\node[circle, scale=0.75, fill, red] (tid6) at (2.25,6){};
\draw[](tid4) -- (tid6);
\draw[](tid2) -- (tid4);
\draw[](tid0) -- (tid1);
\draw[](tid0) -- (tid2);
\end{tikzpicture}
\nodepart{three}
\footnotesize{6.36719}
\nodepart{four}
\footnotesize{$50\:50$}
};
 & 
\node[draw=black, rectangle split,  rectangle split parts=4] (sn0x188ff90){
\footnotesize{0.111111}
\nodepart{two}
\begin{tikzpicture}[scale=.2]
\node[circle, scale=0.75, fill] (tid0) at (2.25,1.5){};
\node[circle, scale=0.75, fill] (tid1) at (1.5,3){};
\node[circle, scale=0.75, fill] (tid3) at (1.5,4.5){};
\node[circle, scale=0.75, fill] (tid5) at (1.5,6){};
\node[circle, scale=0.75, fill, red] (tid7) at (0.75,7.5){};
\node[circle, scale=0.75, fill, red] (tid8) at (2.25,7.5){};
\draw[](tid5) -- (tid7);
\draw[](tid5) -- (tid8);
\draw[](tid3) -- (tid5);
\draw[](tid1) -- (tid3);
\node[circle, scale=0.75, fill] (tid2) at (3.75,3){};
\node[circle, scale=0.75, fill] (tid4) at (3.75,4.5){};
\node[circle, scale=0.75, fill, red] (tid6) at (3.75,6){};
\draw[](tid4) -- (tid6);
\draw[](tid2) -- (tid4);
\draw[](tid0) -- (tid1);
\draw[](tid0) -- (tid2);
\end{tikzpicture}
\nodepart{three}
\footnotesize{5.95602}
\nodepart{four}
\footnotesize{$33\:67$}
};
 & 
\\
};
\end{scope}
\begin{scope}[yshift=\leveltopIIII cm]
\matrix (line4) [column sep=1cm] {
\node[draw=black, rectangle split,  rectangle split parts=4] (sn0x188c370){
\footnotesize{0.0740741}
\nodepart{two}
\begin{tikzpicture}[scale=.2]
\node[circle, scale=0.75, fill] (tid0) at (2.25,1.5){};
\node[circle, scale=0.75, fill] (tid1) at (1.5,3){};
\node[circle, scale=0.75, fill] (tid3) at (1.5,4.5){};
\node[circle, scale=0.75, fill] (tid4) at (1.5,6){};
\node[circle, scale=0.75, fill] (tid5) at (1.5,7.5){};
\node[circle, scale=0.75, fill, red] (tid6) at (0.75,9){};
\node[circle, scale=0.75, fill, red] (tid7) at (2.25,9){};
\draw[](tid5) -- (tid6);
\draw[](tid5) -- (tid7);
\draw[](tid4) -- (tid5);
\draw[](tid3) -- (tid4);
\draw[](tid1) -- (tid3);
\node[circle, scale=0.75, fill, red] (tid2) at (3.75,3){};
\draw[](tid0) -- (tid1);
\draw[](tid0) -- (tid2);
\end{tikzpicture}
\nodepart{three}
\footnotesize{6.52083}
\nodepart{four}
\footnotesize{$33\:67$}
};
 & 
\node[draw=black, rectangle split,  rectangle split parts=4] (sn0x188d690){
\footnotesize{0.425926}
\nodepart{two}
\begin{tikzpicture}[scale=.2]
\node[circle, scale=0.75, fill] (tid0) at (1.5,1.5){};
\node[circle, scale=0.75, fill] (tid1) at (0.75,3){};
\node[circle, scale=0.75, fill] (tid3) at (0.75,4.5){};
\node[circle, scale=0.75, fill] (tid5) at (0.75,6){};
\node[circle, scale=0.75, fill] (tid6) at (0.75,7.5){};
\node[circle, scale=0.75, fill, red] (tid7) at (0.75,9){};
\draw[](tid6) -- (tid7);
\draw[](tid5) -- (tid6);
\draw[](tid3) -- (tid5);
\draw[](tid1) -- (tid3);
\node[circle, scale=0.75, fill] (tid2) at (2.25,3){};
\node[circle, scale=0.75, fill, red] (tid4) at (2.25,4.5){};
\draw[](tid2) -- (tid4);
\draw[](tid0) -- (tid1);
\draw[](tid0) -- (tid2);
\end{tikzpicture}
\nodepart{three}
\footnotesize{6.14062}
\nodepart{four}
\footnotesize{$50\:50$}
};
 & 
\node[draw=black, rectangle split,  rectangle split parts=4] (sn0x188b180){
\footnotesize{0.111111}
\nodepart{two}
\begin{tikzpicture}[scale=.2]
\node[circle, scale=0.75, fill] (tid0) at (2.25,1.5){};
\node[circle, scale=0.75, fill] (tid1) at (1.5,3){};
\node[circle, scale=0.75, fill] (tid3) at (1.5,4.5){};
\node[circle, scale=0.75, fill] (tid4) at (1.5,6){};
\node[circle, scale=0.75, fill] (tid5) at (0.75,7.5){};
\node[circle, scale=0.75, fill, red] (tid7) at (0.75,9){};
\draw[](tid5) -- (tid7);
\node[circle, scale=0.75, fill, red] (tid6) at (2.25,7.5){};
\draw[](tid4) -- (tid5);
\draw[](tid4) -- (tid6);
\draw[](tid3) -- (tid4);
\draw[](tid1) -- (tid3);
\node[circle, scale=0.75, fill, red] (tid2) at (3.75,3){};
\draw[](tid0) -- (tid1);
\draw[](tid0) -- (tid2);
\end{tikzpicture}
\nodepart{three}
\footnotesize{6.27431}
\nodepart{four}
\footnotesize{$33\:33\:33$}
};
 & 
\node[draw=black, rectangle split,  rectangle split parts=4] (sn0x188de90){
\footnotesize{0.148148}
\nodepart{two}
\begin{tikzpicture}[scale=.2]
\node[circle, scale=0.75, fill] (tid0) at (2.25,1.5){};
\node[circle, scale=0.75, fill] (tid1) at (1.5,3){};
\node[circle, scale=0.75, fill] (tid3) at (1.5,4.5){};
\node[circle, scale=0.75, fill] (tid5) at (1.5,6){};
\node[circle, scale=0.75, fill, red] (tid6) at (0.75,7.5){};
\node[circle, scale=0.75, fill, red] (tid7) at (2.25,7.5){};
\draw[](tid5) -- (tid6);
\draw[](tid5) -- (tid7);
\draw[](tid3) -- (tid5);
\draw[](tid1) -- (tid3);
\node[circle, scale=0.75, fill] (tid2) at (3.75,3){};
\node[circle, scale=0.75, fill, red] (tid4) at (3.75,4.5){};
\draw[](tid2) -- (tid4);
\draw[](tid0) -- (tid1);
\draw[](tid0) -- (tid2);
\end{tikzpicture}
\nodepart{three}
\footnotesize{5.68056}
\nodepart{four}
\footnotesize{$67\:33$}
};
 & 
\node[draw=black, rectangle split,  rectangle split parts=4] (sn0x188f020){
\footnotesize{0.240741}
\nodepart{two}
\begin{tikzpicture}[scale=.2]
\node[circle, scale=0.75, fill] (tid0) at (1.5,1.5){};
\node[circle, scale=0.75, fill] (tid1) at (0.75,3){};
\node[circle, scale=0.75, fill] (tid3) at (0.75,4.5){};
\node[circle, scale=0.75, fill] (tid5) at (0.75,6){};
\node[circle, scale=0.75, fill, red] (tid7) at (0.75,7.5){};
\draw[](tid5) -- (tid7);
\draw[](tid3) -- (tid5);
\draw[](tid1) -- (tid3);
\node[circle, scale=0.75, fill] (tid2) at (2.25,3){};
\node[circle, scale=0.75, fill] (tid4) at (2.25,4.5){};
\node[circle, scale=0.75, fill, red] (tid6) at (2.25,6){};
\draw[](tid4) -- (tid6);
\draw[](tid2) -- (tid4);
\draw[](tid0) -- (tid1);
\draw[](tid0) -- (tid2);
\end{tikzpicture}
\nodepart{three}
\footnotesize{5.59375}
\nodepart{four}
\footnotesize{$50\:50$}
};
 & 
\\
};
\end{scope}
\begin{scope}[yshift=\leveltopIIIII cm]
\matrix (line5) [column sep=1cm] {
\node[draw=black, rectangle split,  rectangle split parts=4] (sn0x1888fa0){
\footnotesize{0.0246914}
\nodepart{two}
\begin{tikzpicture}[scale=.2]
\node[circle, scale=0.75, fill] (tid0) at (1.5,1.5){};
\node[circle, scale=0.75, fill] (tid1) at (1.5,3){};
\node[circle, scale=0.75, fill] (tid2) at (1.5,4.5){};
\node[circle, scale=0.75, fill] (tid3) at (1.5,6){};
\node[circle, scale=0.75, fill] (tid4) at (1.5,7.5){};
\node[circle, scale=0.75, fill, red] (tid5) at (0.75,9){};
\node[circle, scale=0.75, fill, red] (tid6) at (2.25,9){};
\draw[](tid4) -- (tid5);
\draw[](tid4) -- (tid6);
\draw[](tid3) -- (tid4);
\draw[](tid2) -- (tid3);
\draw[](tid1) -- (tid2);
\draw[](tid0) -- (tid1);
\end{tikzpicture}
\nodepart{three}
\footnotesize{6.5}
\nodepart{four}
\footnotesize{$1$}
};
 & 
\node[draw=black, rectangle split,  rectangle split parts=4] (sn0x188ae00){
\footnotesize{0.299383}
\nodepart{two}
\begin{tikzpicture}[scale=.2]
\node[circle, scale=0.75, fill] (tid0) at (1.5,1.5){};
\node[circle, scale=0.75, fill] (tid1) at (0.75,3){};
\node[circle, scale=0.75, fill] (tid3) at (0.75,4.5){};
\node[circle, scale=0.75, fill] (tid4) at (0.75,6){};
\node[circle, scale=0.75, fill] (tid5) at (0.75,7.5){};
\node[circle, scale=0.75, fill, red] (tid6) at (0.75,9){};
\draw[](tid5) -- (tid6);
\draw[](tid4) -- (tid5);
\draw[](tid3) -- (tid4);
\draw[](tid1) -- (tid3);
\node[circle, scale=0.75, fill, red] (tid2) at (2.25,3){};
\draw[](tid0) -- (tid1);
\draw[](tid0) -- (tid2);
\end{tikzpicture}
\nodepart{three}
\footnotesize{6.03125}
\nodepart{four}
\footnotesize{$50\:50$}
};
 & 
\node[draw=black, rectangle split,  rectangle split parts=4] (sn0x188d460){
\footnotesize{0.432099}
\nodepart{two}
\begin{tikzpicture}[scale=.2]
\node[circle, scale=0.75, fill] (tid0) at (1.5,1.5){};
\node[circle, scale=0.75, fill] (tid1) at (0.75,3){};
\node[circle, scale=0.75, fill] (tid3) at (0.75,4.5){};
\node[circle, scale=0.75, fill] (tid5) at (0.75,6){};
\node[circle, scale=0.75, fill, red] (tid6) at (0.75,7.5){};
\draw[](tid5) -- (tid6);
\draw[](tid3) -- (tid5);
\draw[](tid1) -- (tid3);
\node[circle, scale=0.75, fill] (tid2) at (2.25,3){};
\node[circle, scale=0.75, fill, red] (tid4) at (2.25,4.5){};
\draw[](tid2) -- (tid4);
\draw[](tid0) -- (tid1);
\draw[](tid0) -- (tid2);
\end{tikzpicture}
\nodepart{three}
\footnotesize{5.25}
\nodepart{four}
\footnotesize{$50\:50$}
};
 & 
\node[draw=black, rectangle split,  rectangle split parts=4] (sn0x1889a70){
\footnotesize{0.037037}
\nodepart{two}
\begin{tikzpicture}[scale=.2]
\node[circle, scale=0.75, fill] (tid0) at (1.5,1.5){};
\node[circle, scale=0.75, fill] (tid1) at (1.5,3){};
\node[circle, scale=0.75, fill] (tid2) at (1.5,4.5){};
\node[circle, scale=0.75, fill] (tid3) at (1.5,6){};
\node[circle, scale=0.75, fill] (tid4) at (0.75,7.5){};
\node[circle, scale=0.75, fill, red] (tid6) at (0.75,9){};
\draw[](tid4) -- (tid6);
\node[circle, scale=0.75, fill, red] (tid5) at (2.25,7.5){};
\draw[](tid3) -- (tid4);
\draw[](tid3) -- (tid5);
\draw[](tid2) -- (tid3);
\draw[](tid1) -- (tid2);
\draw[](tid0) -- (tid1);
\end{tikzpicture}
\nodepart{three}
\footnotesize{6.25}
\nodepart{four}
\footnotesize{$50\:50$}
};
 & 
\node[draw=black, rectangle split,  rectangle split parts=4] (sn0x188b630){
\footnotesize{0.0864198}
\nodepart{two}
\begin{tikzpicture}[scale=.2]
\node[circle, scale=0.75, fill] (tid0) at (2.25,1.5){};
\node[circle, scale=0.75, fill] (tid1) at (1.5,3){};
\node[circle, scale=0.75, fill] (tid3) at (1.5,4.5){};
\node[circle, scale=0.75, fill] (tid4) at (1.5,6){};
\node[circle, scale=0.75, fill, red] (tid5) at (0.75,7.5){};
\node[circle, scale=0.75, fill, red] (tid6) at (2.25,7.5){};
\draw[](tid4) -- (tid5);
\draw[](tid4) -- (tid6);
\draw[](tid3) -- (tid4);
\draw[](tid1) -- (tid3);
\node[circle, scale=0.75, fill, red] (tid2) at (3.75,3){};
\draw[](tid0) -- (tid1);
\draw[](tid0) -- (tid2);
\end{tikzpicture}
\nodepart{three}
\footnotesize{5.54167}
\nodepart{four}
\footnotesize{$67\:33$}
};
 & 
\node[draw=black, rectangle split,  rectangle split parts=4] (sn0x188fd60){
\footnotesize{0.12037}
\nodepart{two}
\begin{tikzpicture}[scale=.2]
\node[circle, scale=0.75, fill] (tid0) at (1.5,1.5){};
\node[circle, scale=0.75, fill] (tid1) at (0.75,3){};
\node[circle, scale=0.75, fill] (tid3) at (0.75,4.5){};
\node[circle, scale=0.75, fill, red] (tid5) at (0.75,6){};
\draw[](tid3) -- (tid5);
\draw[](tid1) -- (tid3);
\node[circle, scale=0.75, fill] (tid2) at (2.25,3){};
\node[circle, scale=0.75, fill] (tid4) at (2.25,4.5){};
\node[circle, scale=0.75, fill, red] (tid6) at (2.25,6){};
\draw[](tid4) -- (tid6);
\draw[](tid2) -- (tid4);
\draw[](tid0) -- (tid1);
\draw[](tid0) -- (tid2);
\end{tikzpicture}
\nodepart{three}
\footnotesize{4.9375}
\nodepart{four}
\footnotesize{$1$}
};
 & 
\\
};
\end{scope}
\begin{scope}[yshift=\leveltopIIIIII cm]
\matrix (line6) [column sep=1cm] {
\node[draw=black, rectangle split,  rectangle split parts=4] (sn0x1888e50){
\footnotesize{0.192901}
\nodepart{two}
\begin{tikzpicture}[scale=.2]
\node[circle, scale=0.75, fill] (tid0) at (0.75,1.5){};
\node[circle, scale=0.75, fill] (tid1) at (0.75,3){};
\node[circle, scale=0.75, fill] (tid2) at (0.75,4.5){};
\node[circle, scale=0.75, fill] (tid3) at (0.75,6){};
\node[circle, scale=0.75, fill] (tid4) at (0.75,7.5){};
\node[circle, scale=0.75, fill, red] (tid5) at (0.75,9){};
\draw[](tid4) -- (tid5);
\draw[](tid3) -- (tid4);
\draw[](tid2) -- (tid3);
\draw[](tid1) -- (tid2);
\draw[](tid0) -- (tid1);
\end{tikzpicture}
\nodepart{three}
\footnotesize{6}
\nodepart{four}
\footnotesize{$1$}
};
 & 
\node[draw=black, rectangle split,  rectangle split parts=4] (sn0x188acf0){
\footnotesize{0.423354}
\nodepart{two}
\begin{tikzpicture}[scale=.2]
\node[circle, scale=0.75, fill] (tid0) at (1.5,1.5){};
\node[circle, scale=0.75, fill] (tid1) at (0.75,3){};
\node[circle, scale=0.75, fill] (tid3) at (0.75,4.5){};
\node[circle, scale=0.75, fill] (tid4) at (0.75,6){};
\node[circle, scale=0.75, fill, red] (tid5) at (0.75,7.5){};
\draw[](tid4) -- (tid5);
\draw[](tid3) -- (tid4);
\draw[](tid1) -- (tid3);
\node[circle, scale=0.75, fill, red] (tid2) at (2.25,3){};
\draw[](tid0) -- (tid1);
\draw[](tid0) -- (tid2);
\end{tikzpicture}
\nodepart{three}
\footnotesize{5.0625}
\nodepart{four}
\footnotesize{$50\:50$}
};
 & 
\node[draw=black, rectangle split,  rectangle split parts=4] (sn0x188c6a0){
\footnotesize{0.33642}
\nodepart{two}
\begin{tikzpicture}[scale=.2]
\node[circle, scale=0.75, fill] (tid0) at (1.5,1.5){};
\node[circle, scale=0.75, fill] (tid1) at (0.75,3){};
\node[circle, scale=0.75, fill] (tid3) at (0.75,4.5){};
\node[circle, scale=0.75, fill, red] (tid5) at (0.75,6){};
\draw[](tid3) -- (tid5);
\draw[](tid1) -- (tid3);
\node[circle, scale=0.75, fill] (tid2) at (2.25,3){};
\node[circle, scale=0.75, fill, red] (tid4) at (2.25,4.5){};
\draw[](tid2) -- (tid4);
\draw[](tid0) -- (tid1);
\draw[](tid0) -- (tid2);
\end{tikzpicture}
\nodepart{three}
\footnotesize{4.4375}
\nodepart{four}
\footnotesize{$50\:50$}
};
 & 
\node[draw=black, rectangle split,  rectangle split parts=4] (sn0x1889230){
\footnotesize{0.0473251}
\nodepart{two}
\begin{tikzpicture}[scale=.2]
\node[circle, scale=0.75, fill] (tid0) at (1.5,1.5){};
\node[circle, scale=0.75, fill] (tid1) at (1.5,3){};
\node[circle, scale=0.75, fill] (tid2) at (1.5,4.5){};
\node[circle, scale=0.75, fill] (tid3) at (1.5,6){};
\node[circle, scale=0.75, fill, red] (tid4) at (0.75,7.5){};
\node[circle, scale=0.75, fill, red] (tid5) at (2.25,7.5){};
\draw[](tid3) -- (tid4);
\draw[](tid3) -- (tid5);
\draw[](tid2) -- (tid3);
\draw[](tid1) -- (tid2);
\draw[](tid0) -- (tid1);
\end{tikzpicture}
\nodepart{three}
\footnotesize{5.5}
\nodepart{four}
\footnotesize{$1$}
};
 & 
\\
};
\end{scope}
\begin{scope}[yshift=\leveltopIIIIIII cm]
\matrix (line7) [column sep=1cm] {
\node[draw=black, rectangle split,  rectangle split parts=4] (sn0x1888b50){
\footnotesize{0.451903}
\nodepart{two}
\begin{tikzpicture}[scale=.2]
\node[circle, scale=0.75, fill] (tid0) at (0.75,1.5){};
\node[circle, scale=0.75, fill] (tid1) at (0.75,3){};
\node[circle, scale=0.75, fill] (tid2) at (0.75,4.5){};
\node[circle, scale=0.75, fill] (tid3) at (0.75,6){};
\node[circle, scale=0.75, fill, red] (tid4) at (0.75,7.5){};
\draw[](tid3) -- (tid4);
\draw[](tid2) -- (tid3);
\draw[](tid1) -- (tid2);
\draw[](tid0) -- (tid1);
\end{tikzpicture}
\nodepart{three}
\footnotesize{5}
\nodepart{four}
\footnotesize{$1$}
};
 & 
\node[draw=black, rectangle split,  rectangle split parts=4] (sn0x188a9f0){
\footnotesize{0.379887}
\nodepart{two}
\begin{tikzpicture}[scale=.2]
\node[circle, scale=0.75, fill] (tid0) at (1.5,1.5){};
\node[circle, scale=0.75, fill] (tid1) at (0.75,3){};
\node[circle, scale=0.75, fill] (tid3) at (0.75,4.5){};
\node[circle, scale=0.75, fill, red] (tid4) at (0.75,6){};
\draw[](tid3) -- (tid4);
\draw[](tid1) -- (tid3);
\node[circle, scale=0.75, fill, red] (tid2) at (2.25,3){};
\draw[](tid0) -- (tid1);
\draw[](tid0) -- (tid2);
\end{tikzpicture}
\nodepart{three}
\footnotesize{4.125}
\nodepart{four}
\footnotesize{$50\:50$}
};
 & 
\node[draw=black, rectangle split,  rectangle split parts=4] (sn0x188c5b0){
\footnotesize{0.16821}
\nodepart{two}
\begin{tikzpicture}[scale=.2]
\node[circle, scale=0.75, fill] (tid0) at (1.5,1.5){};
\node[circle, scale=0.75, fill] (tid1) at (0.75,3){};
\node[circle, scale=0.75, fill, red] (tid3) at (0.75,4.5){};
\draw[](tid1) -- (tid3);
\node[circle, scale=0.75, fill] (tid2) at (2.25,3){};
\node[circle, scale=0.75, fill, red] (tid4) at (2.25,4.5){};
\draw[](tid2) -- (tid4);
\draw[](tid0) -- (tid1);
\draw[](tid0) -- (tid2);
\end{tikzpicture}
\nodepart{three}
\footnotesize{3.75}
\nodepart{four}
\footnotesize{$1$}
};
 & 
\\
};
\end{scope}
\begin{scope}[yshift=\leveltopIIIIIIII cm]
\matrix (line8) [column sep=1cm] {
\node[draw=black, rectangle split,  rectangle split parts=4] (sn0x1888900){
\footnotesize{0.641847}
\nodepart{two}
\begin{tikzpicture}[scale=.2]
\node[circle, scale=0.75, fill] (tid0) at (0.75,1.5){};
\node[circle, scale=0.75, fill] (tid1) at (0.75,3){};
\node[circle, scale=0.75, fill] (tid2) at (0.75,4.5){};
\node[circle, scale=0.75, fill, red] (tid3) at (0.75,6){};
\draw[](tid2) -- (tid3);
\draw[](tid1) -- (tid2);
\draw[](tid0) -- (tid1);
\end{tikzpicture}
\nodepart{three}
\footnotesize{4}
\nodepart{four}
\footnotesize{$1$}
};
 & 
\node[draw=black, rectangle split,  rectangle split parts=4] (sn0x188a920){
\footnotesize{0.358153}
\nodepart{two}
\begin{tikzpicture}[scale=.2]
\node[circle, scale=0.75, fill] (tid0) at (1.5,1.5){};
\node[circle, scale=0.75, fill] (tid1) at (0.75,3){};
\node[circle, scale=0.75, fill, red] (tid3) at (0.75,4.5){};
\draw[](tid1) -- (tid3);
\node[circle, scale=0.75, fill, red] (tid2) at (2.25,3){};
\draw[](tid0) -- (tid1);
\draw[](tid0) -- (tid2);
\end{tikzpicture}
\nodepart{three}
\footnotesize{3.25}
\nodepart{four}
\footnotesize{$50\:50$}
};
 & 
\\
};
\end{scope}
\begin{scope}[yshift=\leveltopIIIIIIIII cm]
\matrix (line9) [column sep=1cm] {
\node[draw=black, rectangle split,  rectangle split parts=4] (sn0x1888830){
\footnotesize{0.820923}
\nodepart{two}
\begin{tikzpicture}[scale=.2]
\node[circle, scale=0.75, fill] (tid0) at (0.75,1.5){};
\node[circle, scale=0.75, fill] (tid1) at (0.75,3){};
\node[circle, scale=0.75, fill, red] (tid2) at (0.75,4.5){};
\draw[](tid1) -- (tid2);
\draw[](tid0) -- (tid1);
\end{tikzpicture}
\nodepart{three}
\footnotesize{3}
\nodepart{four}
\footnotesize{$1$}
};
 & 
\node[draw=black, rectangle split,  rectangle split parts=4] (sn0x1889c10){
\footnotesize{0.179077}
\nodepart{two}
\begin{tikzpicture}[scale=.2]
\node[circle, scale=0.75, fill] (tid0) at (1.5,1.5){};
\node[circle, scale=0.75, fill, red] (tid1) at (0.75,3){};
\node[circle, scale=0.75, fill, red] (tid2) at (2.25,3){};
\draw[](tid0) -- (tid1);
\draw[](tid0) -- (tid2);
\end{tikzpicture}
\nodepart{three}
\footnotesize{2.5}
\nodepart{four}
\footnotesize{$1$}
};
 & 
\\
};
\end{scope}
\begin{scope}[yshift=\leveltopIIIIIIIIII cm]
\matrix (line10) [column sep=1cm] {
\node[draw=black, rectangle split,  rectangle split parts=4] (sn0x1887520){
\footnotesize{1}
\nodepart{two}
\begin{tikzpicture}[scale=.2]
\node[circle, scale=0.75, fill] (tid0) at (0.75,1.5){};
\node[circle, scale=0.75, fill, red] (tid1) at (0.75,3){};
\draw[](tid0) -- (tid1);
\end{tikzpicture}
\nodepart{three}
\footnotesize{2}
\nodepart{four}
\footnotesize{$1$}
};
 & 
\\
};
\end{scope}
\begin{scope}[yshift=\leveltopIIIIIIIIIII cm]
\matrix (line11) [column sep=1cm] {
\node[draw=black, rectangle split,  rectangle split parts=4] (sn0x1887450){
\footnotesize{1}
\nodepart{two}
\begin{tikzpicture}[scale=.2]
\node[circle, scale=0.75, fill, red] (tid0) at (0.75,1.5){};
\end{tikzpicture}
\nodepart{three}
\footnotesize{1}
\nodepart{four}
\footnotesize{$$}
};
 & 
\\
};
\end{scope}
\begin{scope}[yshift=\leveltopIIIIIIIIIIII cm]
\matrix (line12) [column sep=1cm] {
\\
};
\end{scope}
\draw (sn0x1891fa0.south) -- (sn0x188ec30.north);
\draw (sn0x1891fa0.south) -- (sn0x1891ba0.north);
\draw (sn0x1891fa0.south) -- (sn0x18906a0.north);
\draw (sn0x188ec30.south) -- (sn0x188e220.north);
\draw (sn0x188ec30.south) -- (sn0x188d8a0.north);
\draw (sn0x1891ba0.south) -- (sn0x188e220.north);
\draw (sn0x1891ba0.south) -- (sn0x188f0f0.north);
\draw (sn0x18906a0.south) -- (sn0x188d8a0.north);
\draw (sn0x18906a0.south) -- (sn0x188f0f0.north);
\draw (sn0x18906a0.south) -- (sn0x188ff90.north);
\draw (sn0x188e220.south) -- (sn0x188c370.north);
\draw (sn0x188e220.south) -- (sn0x188d690.north);
\draw (sn0x188d8a0.south) -- (sn0x188b180.north);
\draw (sn0x188d8a0.south) -- (sn0x188d690.north);
\draw (sn0x188d8a0.south) -- (sn0x188de90.north);
\draw (sn0x188f0f0.south) -- (sn0x188d690.north);
\draw (sn0x188f0f0.south) -- (sn0x188f020.north);
\draw (sn0x188ff90.south) -- (sn0x188de90.north);
\draw (sn0x188ff90.south) -- (sn0x188f020.north);
\draw (sn0x188c370.south) -- (sn0x1888fa0.north);
\draw (sn0x188c370.south) -- (sn0x188ae00.north);
\draw (sn0x188d690.south) -- (sn0x188ae00.north);
\draw (sn0x188d690.south) -- (sn0x188d460.north);
\draw (sn0x188b180.south) -- (sn0x1889a70.north);
\draw (sn0x188b180.south) -- (sn0x188ae00.north);
\draw (sn0x188b180.south) -- (sn0x188b630.north);
\draw (sn0x188de90.south) -- (sn0x188b630.north);
\draw (sn0x188de90.south) -- (sn0x188d460.north);
\draw (sn0x188f020.south) -- (sn0x188d460.north);
\draw (sn0x188f020.south) -- (sn0x188fd60.north);
\draw (sn0x1888fa0.south) -- (sn0x1888e50.north);
\draw (sn0x188ae00.south) -- (sn0x1888e50.north);
\draw (sn0x188ae00.south) -- (sn0x188acf0.north);
\draw (sn0x188d460.south) -- (sn0x188acf0.north);
\draw (sn0x188d460.south) -- (sn0x188c6a0.north);
\draw (sn0x1889a70.south) -- (sn0x1888e50.north);
\draw (sn0x1889a70.south) -- (sn0x1889230.north);
\draw (sn0x188b630.south) -- (sn0x1889230.north);
\draw (sn0x188b630.south) -- (sn0x188acf0.north);
\draw (sn0x188fd60.south) -- (sn0x188c6a0.north);
\draw (sn0x1888e50.south) -- (sn0x1888b50.north);
\draw (sn0x188acf0.south) -- (sn0x1888b50.north);
\draw (sn0x188acf0.south) -- (sn0x188a9f0.north);
\draw (sn0x188c6a0.south) -- (sn0x188a9f0.north);
\draw (sn0x188c6a0.south) -- (sn0x188c5b0.north);
\draw (sn0x1889230.south) -- (sn0x1888b50.north);
\draw (sn0x1888b50.south) -- (sn0x1888900.north);
\draw (sn0x188a9f0.south) -- (sn0x1888900.north);
\draw (sn0x188a9f0.south) -- (sn0x188a920.north);
\draw (sn0x188c5b0.south) -- (sn0x188a920.north);
\draw (sn0x1888900.south) -- (sn0x1888830.north);
\draw (sn0x188a920.south) -- (sn0x1888830.north);
\draw (sn0x188a920.south) -- (sn0x1889c10.north);
\draw (sn0x1888830.south) -- (sn0x1887520.north);
\draw (sn0x1889c10.south) -- (sn0x1887520.north);
\draw (sn0x1887520.south) -- (sn0x1887450.north);
\end{tikzpicture}

%%% Local Variables:
%%% TeX-master: "thesis/thesis.tex"
%%% End: 
\renewcommand{\leveltopI}{-15cm + \leveltop}
\renewcommand{\leveltopII}{-15cm + \leveltopI}
\renewcommand{\leveltopIII}{-15cm + \leveltopII}
\renewcommand{\leveltopIIII}{-15cm + \leveltopIII}
\renewcommand{\leveltopIIIII}{-15cm + \leveltopIIII}
\renewcommand{\leveltopIIIIII}{-15cm + \leveltopIIIII}
\renewcommand{\leveltopIIIIIII}{-15cm + \leveltopIIIIII}
\renewcommand{\leveltopIIIIIIII}{-15cm + \leveltopIIIIIII}
\renewcommand{\leveltopIIIIIIIII}{-15cm + \leveltopIIIIIIII}
\renewcommand{\leveltopIIIIIIIIII}{-15cm + \leveltopIIIIIIIII}
\renewcommand{\leveltopIIIIIIIIIII}{-15cm + \leveltopIIIIIIIIII}
\begin{tikzpicture}[scale=.2, anchor=south]
\begin{scope}[yshift=\leveltopI cm]
\matrix (line1) [column sep=1cm] {
\node[draw=black, rectangle split,  rectangle split parts=4] (sn0x1892300){
\footnotesize{1}
\nodepart{two}
\begin{tikzpicture}[scale=.2]
\node[circle, scale=0.75, fill] (tid0) at (3,1.5){};
\node[circle, scale=0.75, fill] (tid1) at (2.25,3){};
\node[circle, scale=0.75, fill] (tid3) at (2.25,4.5){};
\node[circle, scale=0.75, fill] (tid5) at (2.25,6){};
\node[circle, scale=0.75, fill] (tid7) at (1.5,7.5){};
\node[circle, scale=0.75, fill, red] (tid9) at (0.75,9){};
\node[circle, scale=0.75, fill, red] (tid10) at (2.25,9){};
\draw[](tid7) -- (tid9);
\draw[](tid7) -- (tid10);
\node[circle, scale=0.75, fill, red] (tid8) at (3.75,7.5){};
\draw[](tid5) -- (tid7);
\draw[](tid5) -- (tid8);
\draw[](tid3) -- (tid5);
\draw[](tid1) -- (tid3);
\node[circle, scale=0.75, fill] (tid2) at (5.25,3){};
\node[circle, scale=0.75, fill] (tid4) at (5.25,4.5){};
\node[circle, scale=0.75, fill] (tid6) at (5.25,6){};
\draw[](tid4) -- (tid6);
\draw[](tid2) -- (tid4);
\draw[](tid0) -- (tid1);
\draw[](tid0) -- (tid2);
\end{tikzpicture}
\nodepart{three}
\footnotesize{6.96798}
\nodepart{four}
\footnotesize{$33\:67$}
};
 & 
\\
};
\end{scope}
\begin{scope}[yshift=\leveltopII cm]
\matrix (line2) [column sep=1cm] {
\node[draw=black, rectangle split,  rectangle split parts=4] (sn0x1891ba0){
\footnotesize{0.333333}
\nodepart{two}
\begin{tikzpicture}[scale=.2]
\node[circle, scale=0.75, fill] (tid0) at (2.25,1.5){};
\node[circle, scale=0.75, fill] (tid1) at (1.5,3){};
\node[circle, scale=0.75, fill] (tid3) at (1.5,4.5){};
\node[circle, scale=0.75, fill] (tid5) at (1.5,6){};
\node[circle, scale=0.75, fill] (tid7) at (1.5,7.5){};
\node[circle, scale=0.75, fill, red] (tid8) at (0.75,9){};
\node[circle, scale=0.75, fill, red] (tid9) at (2.25,9){};
\draw[](tid7) -- (tid8);
\draw[](tid7) -- (tid9);
\draw[](tid5) -- (tid7);
\draw[](tid3) -- (tid5);
\draw[](tid1) -- (tid3);
\node[circle, scale=0.75, fill] (tid2) at (3.75,3){};
\node[circle, scale=0.75, fill] (tid4) at (3.75,4.5){};
\node[circle, scale=0.75, fill, red] (tid6) at (3.75,6){};
\draw[](tid4) -- (tid6);
\draw[](tid2) -- (tid4);
\draw[](tid0) -- (tid1);
\draw[](tid0) -- (tid2);
\end{tikzpicture}
\nodepart{three}
\footnotesize{6.77836}
\nodepart{four}
\footnotesize{$33\:67$}
};
 & 
\node[draw=black, rectangle split,  rectangle split parts=4] (sn0x18906a0){
\footnotesize{0.666667}
\nodepart{two}
\begin{tikzpicture}[scale=.2]
\node[circle, scale=0.75, fill] (tid0) at (2.25,1.5){};
\node[circle, scale=0.75, fill] (tid1) at (1.5,3){};
\node[circle, scale=0.75, fill] (tid3) at (1.5,4.5){};
\node[circle, scale=0.75, fill] (tid5) at (1.5,6){};
\node[circle, scale=0.75, fill] (tid7) at (0.75,7.5){};
\node[circle, scale=0.75, fill, red] (tid9) at (0.75,9){};
\draw[](tid7) -- (tid9);
\node[circle, scale=0.75, fill, red] (tid8) at (2.25,7.5){};
\draw[](tid5) -- (tid7);
\draw[](tid5) -- (tid8);
\draw[](tid3) -- (tid5);
\draw[](tid1) -- (tid3);
\node[circle, scale=0.75, fill] (tid2) at (3.75,3){};
\node[circle, scale=0.75, fill] (tid4) at (3.75,4.5){};
\node[circle, scale=0.75, fill, red] (tid6) at (3.75,6){};
\draw[](tid4) -- (tid6);
\draw[](tid2) -- (tid4);
\draw[](tid0) -- (tid1);
\draw[](tid0) -- (tid2);
\end{tikzpicture}
\nodepart{three}
\footnotesize{6.56279}
\nodepart{four}
\footnotesize{$33\:33\:33$}
};
 & 
\\
};
\end{scope}
\begin{scope}[yshift=\leveltopIII cm]
\matrix (line3) [column sep=1cm] {
\node[draw=black, rectangle split,  rectangle split parts=4] (sn0x188e220){
\footnotesize{0.111111}
\nodepart{two}
\begin{tikzpicture}[scale=.2]
\node[circle, scale=0.75, fill] (tid0) at (2.25,1.5){};
\node[circle, scale=0.75, fill] (tid1) at (1.5,3){};
\node[circle, scale=0.75, fill] (tid3) at (1.5,4.5){};
\node[circle, scale=0.75, fill] (tid5) at (1.5,6){};
\node[circle, scale=0.75, fill] (tid6) at (1.5,7.5){};
\node[circle, scale=0.75, fill, red] (tid7) at (0.75,9){};
\node[circle, scale=0.75, fill, red] (tid8) at (2.25,9){};
\draw[](tid6) -- (tid7);
\draw[](tid6) -- (tid8);
\draw[](tid5) -- (tid6);
\draw[](tid3) -- (tid5);
\draw[](tid1) -- (tid3);
\node[circle, scale=0.75, fill] (tid2) at (3.75,3){};
\node[circle, scale=0.75, fill, red] (tid4) at (3.75,4.5){};
\draw[](tid2) -- (tid4);
\draw[](tid0) -- (tid1);
\draw[](tid0) -- (tid2);
\end{tikzpicture}
\nodepart{three}
\footnotesize{6.60069}
\nodepart{four}
\footnotesize{$33\:67$}
};
 & 
\node[draw=black, rectangle split,  rectangle split parts=4] (sn0x188f0f0){
\footnotesize{0.444444}
\nodepart{two}
\begin{tikzpicture}[scale=.2]
\node[circle, scale=0.75, fill] (tid0) at (1.5,1.5){};
\node[circle, scale=0.75, fill] (tid1) at (0.75,3){};
\node[circle, scale=0.75, fill] (tid3) at (0.75,4.5){};
\node[circle, scale=0.75, fill] (tid5) at (0.75,6){};
\node[circle, scale=0.75, fill] (tid7) at (0.75,7.5){};
\node[circle, scale=0.75, fill, red] (tid8) at (0.75,9){};
\draw[](tid7) -- (tid8);
\draw[](tid5) -- (tid7);
\draw[](tid3) -- (tid5);
\draw[](tid1) -- (tid3);
\node[circle, scale=0.75, fill] (tid2) at (2.25,3){};
\node[circle, scale=0.75, fill] (tid4) at (2.25,4.5){};
\node[circle, scale=0.75, fill, red] (tid6) at (2.25,6){};
\draw[](tid4) -- (tid6);
\draw[](tid2) -- (tid4);
\draw[](tid0) -- (tid1);
\draw[](tid0) -- (tid2);
\end{tikzpicture}
\nodepart{three}
\footnotesize{6.36719}
\nodepart{four}
\footnotesize{$50\:50$}
};
 & 
\node[draw=black, rectangle split,  rectangle split parts=4] (sn0x188d8a0){
\footnotesize{0.222222}
\nodepart{two}
\begin{tikzpicture}[scale=.2]
\node[circle, scale=0.75, fill] (tid0) at (2.25,1.5){};
\node[circle, scale=0.75, fill] (tid1) at (1.5,3){};
\node[circle, scale=0.75, fill] (tid3) at (1.5,4.5){};
\node[circle, scale=0.75, fill] (tid5) at (1.5,6){};
\node[circle, scale=0.75, fill] (tid6) at (0.75,7.5){};
\node[circle, scale=0.75, fill, red] (tid8) at (0.75,9){};
\draw[](tid6) -- (tid8);
\node[circle, scale=0.75, fill, red] (tid7) at (2.25,7.5){};
\draw[](tid5) -- (tid6);
\draw[](tid5) -- (tid7);
\draw[](tid3) -- (tid5);
\draw[](tid1) -- (tid3);
\node[circle, scale=0.75, fill] (tid2) at (3.75,3){};
\node[circle, scale=0.75, fill, red] (tid4) at (3.75,4.5){};
\draw[](tid2) -- (tid4);
\draw[](tid0) -- (tid1);
\draw[](tid0) -- (tid2);
\end{tikzpicture}
\nodepart{three}
\footnotesize{6.36516}
\nodepart{four}
\footnotesize{$33\:33\:33$}
};
 & 
\node[draw=black, rectangle split,  rectangle split parts=4] (sn0x188ff90){
\footnotesize{0.222222}
\nodepart{two}
\begin{tikzpicture}[scale=.2]
\node[circle, scale=0.75, fill] (tid0) at (2.25,1.5){};
\node[circle, scale=0.75, fill] (tid1) at (1.5,3){};
\node[circle, scale=0.75, fill] (tid3) at (1.5,4.5){};
\node[circle, scale=0.75, fill] (tid5) at (1.5,6){};
\node[circle, scale=0.75, fill, red] (tid7) at (0.75,7.5){};
\node[circle, scale=0.75, fill, red] (tid8) at (2.25,7.5){};
\draw[](tid5) -- (tid7);
\draw[](tid5) -- (tid8);
\draw[](tid3) -- (tid5);
\draw[](tid1) -- (tid3);
\node[circle, scale=0.75, fill] (tid2) at (3.75,3){};
\node[circle, scale=0.75, fill] (tid4) at (3.75,4.5){};
\node[circle, scale=0.75, fill, red] (tid6) at (3.75,6){};
\draw[](tid4) -- (tid6);
\draw[](tid2) -- (tid4);
\draw[](tid0) -- (tid1);
\draw[](tid0) -- (tid2);
\end{tikzpicture}
\nodepart{three}
\footnotesize{5.95602}
\nodepart{four}
\footnotesize{$67\:33$}
};
 & 
\\
};
\end{scope}
\begin{scope}[yshift=\leveltopIIII cm]
\matrix (line4) [column sep=1cm] {
\node[draw=black, rectangle split,  rectangle split parts=4] (sn0x188c370){
\footnotesize{0.037037}
\nodepart{two}
\begin{tikzpicture}[scale=.2]
\node[circle, scale=0.75, fill] (tid0) at (2.25,1.5){};
\node[circle, scale=0.75, fill] (tid1) at (1.5,3){};
\node[circle, scale=0.75, fill] (tid3) at (1.5,4.5){};
\node[circle, scale=0.75, fill] (tid4) at (1.5,6){};
\node[circle, scale=0.75, fill] (tid5) at (1.5,7.5){};
\node[circle, scale=0.75, fill, red] (tid6) at (0.75,9){};
\node[circle, scale=0.75, fill, red] (tid7) at (2.25,9){};
\draw[](tid5) -- (tid6);
\draw[](tid5) -- (tid7);
\draw[](tid4) -- (tid5);
\draw[](tid3) -- (tid4);
\draw[](tid1) -- (tid3);
\node[circle, scale=0.75, fill, red] (tid2) at (3.75,3){};
\draw[](tid0) -- (tid1);
\draw[](tid0) -- (tid2);
\end{tikzpicture}
\nodepart{three}
\footnotesize{6.52083}
\nodepart{four}
\footnotesize{$33\:67$}
};
 & 
\node[draw=black, rectangle split,  rectangle split parts=4] (sn0x188d690){
\footnotesize{0.37037}
\nodepart{two}
\begin{tikzpicture}[scale=.2]
\node[circle, scale=0.75, fill] (tid0) at (1.5,1.5){};
\node[circle, scale=0.75, fill] (tid1) at (0.75,3){};
\node[circle, scale=0.75, fill] (tid3) at (0.75,4.5){};
\node[circle, scale=0.75, fill] (tid5) at (0.75,6){};
\node[circle, scale=0.75, fill] (tid6) at (0.75,7.5){};
\node[circle, scale=0.75, fill, red] (tid7) at (0.75,9){};
\draw[](tid6) -- (tid7);
\draw[](tid5) -- (tid6);
\draw[](tid3) -- (tid5);
\draw[](tid1) -- (tid3);
\node[circle, scale=0.75, fill] (tid2) at (2.25,3){};
\node[circle, scale=0.75, fill, red] (tid4) at (2.25,4.5){};
\draw[](tid2) -- (tid4);
\draw[](tid0) -- (tid1);
\draw[](tid0) -- (tid2);
\end{tikzpicture}
\nodepart{three}
\footnotesize{6.14062}
\nodepart{four}
\footnotesize{$50\:50$}
};
 & 
\node[draw=black, rectangle split,  rectangle split parts=4] (sn0x188f020){
\footnotesize{0.37037}
\nodepart{two}
\begin{tikzpicture}[scale=.2]
\node[circle, scale=0.75, fill] (tid0) at (1.5,1.5){};
\node[circle, scale=0.75, fill] (tid1) at (0.75,3){};
\node[circle, scale=0.75, fill] (tid3) at (0.75,4.5){};
\node[circle, scale=0.75, fill] (tid5) at (0.75,6){};
\node[circle, scale=0.75, fill, red] (tid7) at (0.75,7.5){};
\draw[](tid5) -- (tid7);
\draw[](tid3) -- (tid5);
\draw[](tid1) -- (tid3);
\node[circle, scale=0.75, fill] (tid2) at (2.25,3){};
\node[circle, scale=0.75, fill] (tid4) at (2.25,4.5){};
\node[circle, scale=0.75, fill, red] (tid6) at (2.25,6){};
\draw[](tid4) -- (tid6);
\draw[](tid2) -- (tid4);
\draw[](tid0) -- (tid1);
\draw[](tid0) -- (tid2);
\end{tikzpicture}
\nodepart{three}
\footnotesize{5.59375}
\nodepart{four}
\footnotesize{$50\:50$}
};
 & 
\node[draw=black, rectangle split,  rectangle split parts=4] (sn0x188b180){
\footnotesize{0.0740741}
\nodepart{two}
\begin{tikzpicture}[scale=.2]
\node[circle, scale=0.75, fill] (tid0) at (2.25,1.5){};
\node[circle, scale=0.75, fill] (tid1) at (1.5,3){};
\node[circle, scale=0.75, fill] (tid3) at (1.5,4.5){};
\node[circle, scale=0.75, fill] (tid4) at (1.5,6){};
\node[circle, scale=0.75, fill] (tid5) at (0.75,7.5){};
\node[circle, scale=0.75, fill, red] (tid7) at (0.75,9){};
\draw[](tid5) -- (tid7);
\node[circle, scale=0.75, fill, red] (tid6) at (2.25,7.5){};
\draw[](tid4) -- (tid5);
\draw[](tid4) -- (tid6);
\draw[](tid3) -- (tid4);
\draw[](tid1) -- (tid3);
\node[circle, scale=0.75, fill, red] (tid2) at (3.75,3){};
\draw[](tid0) -- (tid1);
\draw[](tid0) -- (tid2);
\end{tikzpicture}
\nodepart{three}
\footnotesize{6.27431}
\nodepart{four}
\footnotesize{$33\:33\:33$}
};
 & 
\node[draw=black, rectangle split,  rectangle split parts=4] (sn0x188de90){
\footnotesize{0.148148}
\nodepart{two}
\begin{tikzpicture}[scale=.2]
\node[circle, scale=0.75, fill] (tid0) at (2.25,1.5){};
\node[circle, scale=0.75, fill] (tid1) at (1.5,3){};
\node[circle, scale=0.75, fill] (tid3) at (1.5,4.5){};
\node[circle, scale=0.75, fill] (tid5) at (1.5,6){};
\node[circle, scale=0.75, fill, red] (tid6) at (0.75,7.5){};
\node[circle, scale=0.75, fill, red] (tid7) at (2.25,7.5){};
\draw[](tid5) -- (tid6);
\draw[](tid5) -- (tid7);
\draw[](tid3) -- (tid5);
\draw[](tid1) -- (tid3);
\node[circle, scale=0.75, fill] (tid2) at (3.75,3){};
\node[circle, scale=0.75, fill, red] (tid4) at (3.75,4.5){};
\draw[](tid2) -- (tid4);
\draw[](tid0) -- (tid1);
\draw[](tid0) -- (tid2);
\end{tikzpicture}
\nodepart{three}
\footnotesize{5.68056}
\nodepart{four}
\footnotesize{$67\:33$}
};
 & 
\\
};
\end{scope}
\begin{scope}[yshift=\leveltopIIIII cm]
\matrix (line5) [column sep=1cm] {
\node[draw=black, rectangle split,  rectangle split parts=4] (sn0x1888fa0){
\footnotesize{0.0123457}
\nodepart{two}
\begin{tikzpicture}[scale=.2]
\node[circle, scale=0.75, fill] (tid0) at (1.5,1.5){};
\node[circle, scale=0.75, fill] (tid1) at (1.5,3){};
\node[circle, scale=0.75, fill] (tid2) at (1.5,4.5){};
\node[circle, scale=0.75, fill] (tid3) at (1.5,6){};
\node[circle, scale=0.75, fill] (tid4) at (1.5,7.5){};
\node[circle, scale=0.75, fill, red] (tid5) at (0.75,9){};
\node[circle, scale=0.75, fill, red] (tid6) at (2.25,9){};
\draw[](tid4) -- (tid5);
\draw[](tid4) -- (tid6);
\draw[](tid3) -- (tid4);
\draw[](tid2) -- (tid3);
\draw[](tid1) -- (tid2);
\draw[](tid0) -- (tid1);
\end{tikzpicture}
\nodepart{three}
\footnotesize{6.5}
\nodepart{four}
\footnotesize{$1$}
};
 & 
\node[draw=black, rectangle split,  rectangle split parts=4] (sn0x188ae00){
\footnotesize{0.234568}
\nodepart{two}
\begin{tikzpicture}[scale=.2]
\node[circle, scale=0.75, fill] (tid0) at (1.5,1.5){};
\node[circle, scale=0.75, fill] (tid1) at (0.75,3){};
\node[circle, scale=0.75, fill] (tid3) at (0.75,4.5){};
\node[circle, scale=0.75, fill] (tid4) at (0.75,6){};
\node[circle, scale=0.75, fill] (tid5) at (0.75,7.5){};
\node[circle, scale=0.75, fill, red] (tid6) at (0.75,9){};
\draw[](tid5) -- (tid6);
\draw[](tid4) -- (tid5);
\draw[](tid3) -- (tid4);
\draw[](tid1) -- (tid3);
\node[circle, scale=0.75, fill, red] (tid2) at (2.25,3){};
\draw[](tid0) -- (tid1);
\draw[](tid0) -- (tid2);
\end{tikzpicture}
\nodepart{three}
\footnotesize{6.03125}
\nodepart{four}
\footnotesize{$50\:50$}
};
 & 
\node[draw=black, rectangle split,  rectangle split parts=4] (sn0x188d460){
\footnotesize{0.469136}
\nodepart{two}
\begin{tikzpicture}[scale=.2]
\node[circle, scale=0.75, fill] (tid0) at (1.5,1.5){};
\node[circle, scale=0.75, fill] (tid1) at (0.75,3){};
\node[circle, scale=0.75, fill] (tid3) at (0.75,4.5){};
\node[circle, scale=0.75, fill] (tid5) at (0.75,6){};
\node[circle, scale=0.75, fill, red] (tid6) at (0.75,7.5){};
\draw[](tid5) -- (tid6);
\draw[](tid3) -- (tid5);
\draw[](tid1) -- (tid3);
\node[circle, scale=0.75, fill] (tid2) at (2.25,3){};
\node[circle, scale=0.75, fill, red] (tid4) at (2.25,4.5){};
\draw[](tid2) -- (tid4);
\draw[](tid0) -- (tid1);
\draw[](tid0) -- (tid2);
\end{tikzpicture}
\nodepart{three}
\footnotesize{5.25}
\nodepart{four}
\footnotesize{$50\:50$}
};
 & 
\node[draw=black, rectangle split,  rectangle split parts=4] (sn0x188fd60){
\footnotesize{0.185185}
\nodepart{two}
\begin{tikzpicture}[scale=.2]
\node[circle, scale=0.75, fill] (tid0) at (1.5,1.5){};
\node[circle, scale=0.75, fill] (tid1) at (0.75,3){};
\node[circle, scale=0.75, fill] (tid3) at (0.75,4.5){};
\node[circle, scale=0.75, fill, red] (tid5) at (0.75,6){};
\draw[](tid3) -- (tid5);
\draw[](tid1) -- (tid3);
\node[circle, scale=0.75, fill] (tid2) at (2.25,3){};
\node[circle, scale=0.75, fill] (tid4) at (2.25,4.5){};
\node[circle, scale=0.75, fill, red] (tid6) at (2.25,6){};
\draw[](tid4) -- (tid6);
\draw[](tid2) -- (tid4);
\draw[](tid0) -- (tid1);
\draw[](tid0) -- (tid2);
\end{tikzpicture}
\nodepart{three}
\footnotesize{4.9375}
\nodepart{four}
\footnotesize{$1$}
};
 & 
\node[draw=black, rectangle split,  rectangle split parts=4] (sn0x1889a70){
\footnotesize{0.0246914}
\nodepart{two}
\begin{tikzpicture}[scale=.2]
\node[circle, scale=0.75, fill] (tid0) at (1.5,1.5){};
\node[circle, scale=0.75, fill] (tid1) at (1.5,3){};
\node[circle, scale=0.75, fill] (tid2) at (1.5,4.5){};
\node[circle, scale=0.75, fill] (tid3) at (1.5,6){};
\node[circle, scale=0.75, fill] (tid4) at (0.75,7.5){};
\node[circle, scale=0.75, fill, red] (tid6) at (0.75,9){};
\draw[](tid4) -- (tid6);
\node[circle, scale=0.75, fill, red] (tid5) at (2.25,7.5){};
\draw[](tid3) -- (tid4);
\draw[](tid3) -- (tid5);
\draw[](tid2) -- (tid3);
\draw[](tid1) -- (tid2);
\draw[](tid0) -- (tid1);
\end{tikzpicture}
\nodepart{three}
\footnotesize{6.25}
\nodepart{four}
\footnotesize{$50\:50$}
};
 & 
\node[draw=black, rectangle split,  rectangle split parts=4] (sn0x188b630){
\footnotesize{0.0740741}
\nodepart{two}
\begin{tikzpicture}[scale=.2]
\node[circle, scale=0.75, fill] (tid0) at (2.25,1.5){};
\node[circle, scale=0.75, fill] (tid1) at (1.5,3){};
\node[circle, scale=0.75, fill] (tid3) at (1.5,4.5){};
\node[circle, scale=0.75, fill] (tid4) at (1.5,6){};
\node[circle, scale=0.75, fill, red] (tid5) at (0.75,7.5){};
\node[circle, scale=0.75, fill, red] (tid6) at (2.25,7.5){};
\draw[](tid4) -- (tid5);
\draw[](tid4) -- (tid6);
\draw[](tid3) -- (tid4);
\draw[](tid1) -- (tid3);
\node[circle, scale=0.75, fill, red] (tid2) at (3.75,3){};
\draw[](tid0) -- (tid1);
\draw[](tid0) -- (tid2);
\end{tikzpicture}
\nodepart{three}
\footnotesize{5.54167}
\nodepart{four}
\footnotesize{$67\:33$}
};
 & 
\\
};
\end{scope}
\begin{scope}[yshift=\leveltopIIIIII cm]
\matrix (line6) [column sep=1cm] {
\node[draw=black, rectangle split,  rectangle split parts=4] (sn0x1888e50){
\footnotesize{0.141975}
\nodepart{two}
\begin{tikzpicture}[scale=.2]
\node[circle, scale=0.75, fill] (tid0) at (0.75,1.5){};
\node[circle, scale=0.75, fill] (tid1) at (0.75,3){};
\node[circle, scale=0.75, fill] (tid2) at (0.75,4.5){};
\node[circle, scale=0.75, fill] (tid3) at (0.75,6){};
\node[circle, scale=0.75, fill] (tid4) at (0.75,7.5){};
\node[circle, scale=0.75, fill, red] (tid5) at (0.75,9){};
\draw[](tid4) -- (tid5);
\draw[](tid3) -- (tid4);
\draw[](tid2) -- (tid3);
\draw[](tid1) -- (tid2);
\draw[](tid0) -- (tid1);
\end{tikzpicture}
\nodepart{three}
\footnotesize{6}
\nodepart{four}
\footnotesize{$1$}
};
 & 
\node[draw=black, rectangle split,  rectangle split parts=4] (sn0x188acf0){
\footnotesize{0.401235}
\nodepart{two}
\begin{tikzpicture}[scale=.2]
\node[circle, scale=0.75, fill] (tid0) at (1.5,1.5){};
\node[circle, scale=0.75, fill] (tid1) at (0.75,3){};
\node[circle, scale=0.75, fill] (tid3) at (0.75,4.5){};
\node[circle, scale=0.75, fill] (tid4) at (0.75,6){};
\node[circle, scale=0.75, fill, red] (tid5) at (0.75,7.5){};
\draw[](tid4) -- (tid5);
\draw[](tid3) -- (tid4);
\draw[](tid1) -- (tid3);
\node[circle, scale=0.75, fill, red] (tid2) at (2.25,3){};
\draw[](tid0) -- (tid1);
\draw[](tid0) -- (tid2);
\end{tikzpicture}
\nodepart{three}
\footnotesize{5.0625}
\nodepart{four}
\footnotesize{$50\:50$}
};
 & 
\node[draw=black, rectangle split,  rectangle split parts=4] (sn0x188c6a0){
\footnotesize{0.419753}
\nodepart{two}
\begin{tikzpicture}[scale=.2]
\node[circle, scale=0.75, fill] (tid0) at (1.5,1.5){};
\node[circle, scale=0.75, fill] (tid1) at (0.75,3){};
\node[circle, scale=0.75, fill] (tid3) at (0.75,4.5){};
\node[circle, scale=0.75, fill, red] (tid5) at (0.75,6){};
\draw[](tid3) -- (tid5);
\draw[](tid1) -- (tid3);
\node[circle, scale=0.75, fill] (tid2) at (2.25,3){};
\node[circle, scale=0.75, fill, red] (tid4) at (2.25,4.5){};
\draw[](tid2) -- (tid4);
\draw[](tid0) -- (tid1);
\draw[](tid0) -- (tid2);
\end{tikzpicture}
\nodepart{three}
\footnotesize{4.4375}
\nodepart{four}
\footnotesize{$50\:50$}
};
 & 
\node[draw=black, rectangle split,  rectangle split parts=4] (sn0x1889230){
\footnotesize{0.037037}
\nodepart{two}
\begin{tikzpicture}[scale=.2]
\node[circle, scale=0.75, fill] (tid0) at (1.5,1.5){};
\node[circle, scale=0.75, fill] (tid1) at (1.5,3){};
\node[circle, scale=0.75, fill] (tid2) at (1.5,4.5){};
\node[circle, scale=0.75, fill] (tid3) at (1.5,6){};
\node[circle, scale=0.75, fill, red] (tid4) at (0.75,7.5){};
\node[circle, scale=0.75, fill, red] (tid5) at (2.25,7.5){};
\draw[](tid3) -- (tid4);
\draw[](tid3) -- (tid5);
\draw[](tid2) -- (tid3);
\draw[](tid1) -- (tid2);
\draw[](tid0) -- (tid1);
\end{tikzpicture}
\nodepart{three}
\footnotesize{5.5}
\nodepart{four}
\footnotesize{$1$}
};
 & 
\\
};
\end{scope}
\begin{scope}[yshift=\leveltopIIIIIII cm]
\matrix (line7) [column sep=1cm] {
\node[draw=black, rectangle split,  rectangle split parts=4] (sn0x1888b50){
\footnotesize{0.37963}
\nodepart{two}
\begin{tikzpicture}[scale=.2]
\node[circle, scale=0.75, fill] (tid0) at (0.75,1.5){};
\node[circle, scale=0.75, fill] (tid1) at (0.75,3){};
\node[circle, scale=0.75, fill] (tid2) at (0.75,4.5){};
\node[circle, scale=0.75, fill] (tid3) at (0.75,6){};
\node[circle, scale=0.75, fill, red] (tid4) at (0.75,7.5){};
\draw[](tid3) -- (tid4);
\draw[](tid2) -- (tid3);
\draw[](tid1) -- (tid2);
\draw[](tid0) -- (tid1);
\end{tikzpicture}
\nodepart{three}
\footnotesize{5}
\nodepart{four}
\footnotesize{$1$}
};
 & 
\node[draw=black, rectangle split,  rectangle split parts=4] (sn0x188a9f0){
\footnotesize{0.410494}
\nodepart{two}
\begin{tikzpicture}[scale=.2]
\node[circle, scale=0.75, fill] (tid0) at (1.5,1.5){};
\node[circle, scale=0.75, fill] (tid1) at (0.75,3){};
\node[circle, scale=0.75, fill] (tid3) at (0.75,4.5){};
\node[circle, scale=0.75, fill, red] (tid4) at (0.75,6){};
\draw[](tid3) -- (tid4);
\draw[](tid1) -- (tid3);
\node[circle, scale=0.75, fill, red] (tid2) at (2.25,3){};
\draw[](tid0) -- (tid1);
\draw[](tid0) -- (tid2);
\end{tikzpicture}
\nodepart{three}
\footnotesize{4.125}
\nodepart{four}
\footnotesize{$50\:50$}
};
 & 
\node[draw=black, rectangle split,  rectangle split parts=4] (sn0x188c5b0){
\footnotesize{0.209877}
\nodepart{two}
\begin{tikzpicture}[scale=.2]
\node[circle, scale=0.75, fill] (tid0) at (1.5,1.5){};
\node[circle, scale=0.75, fill] (tid1) at (0.75,3){};
\node[circle, scale=0.75, fill, red] (tid3) at (0.75,4.5){};
\draw[](tid1) -- (tid3);
\node[circle, scale=0.75, fill] (tid2) at (2.25,3){};
\node[circle, scale=0.75, fill, red] (tid4) at (2.25,4.5){};
\draw[](tid2) -- (tid4);
\draw[](tid0) -- (tid1);
\draw[](tid0) -- (tid2);
\end{tikzpicture}
\nodepart{three}
\footnotesize{3.75}
\nodepart{four}
\footnotesize{$1$}
};
 & 
\\
};
\end{scope}
\begin{scope}[yshift=\leveltopIIIIIIII cm]
\matrix (line8) [column sep=1cm] {
\node[draw=black, rectangle split,  rectangle split parts=4] (sn0x1888900){
\footnotesize{0.584877}
\nodepart{two}
\begin{tikzpicture}[scale=.2]
\node[circle, scale=0.75, fill] (tid0) at (0.75,1.5){};
\node[circle, scale=0.75, fill] (tid1) at (0.75,3){};
\node[circle, scale=0.75, fill] (tid2) at (0.75,4.5){};
\node[circle, scale=0.75, fill, red] (tid3) at (0.75,6){};
\draw[](tid2) -- (tid3);
\draw[](tid1) -- (tid2);
\draw[](tid0) -- (tid1);
\end{tikzpicture}
\nodepart{three}
\footnotesize{4}
\nodepart{four}
\footnotesize{$1$}
};
 & 
\node[draw=black, rectangle split,  rectangle split parts=4] (sn0x188a920){
\footnotesize{0.415124}
\nodepart{two}
\begin{tikzpicture}[scale=.2]
\node[circle, scale=0.75, fill] (tid0) at (1.5,1.5){};
\node[circle, scale=0.75, fill] (tid1) at (0.75,3){};
\node[circle, scale=0.75, fill, red] (tid3) at (0.75,4.5){};
\draw[](tid1) -- (tid3);
\node[circle, scale=0.75, fill, red] (tid2) at (2.25,3){};
\draw[](tid0) -- (tid1);
\draw[](tid0) -- (tid2);
\end{tikzpicture}
\nodepart{three}
\footnotesize{3.25}
\nodepart{four}
\footnotesize{$50\:50$}
};
 & 
\\
};
\end{scope}
\begin{scope}[yshift=\leveltopIIIIIIIII cm]
\matrix (line9) [column sep=1cm] {
\node[draw=black, rectangle split,  rectangle split parts=4] (sn0x1888830){
\footnotesize{0.792438}
\nodepart{two}
\begin{tikzpicture}[scale=.2]
\node[circle, scale=0.75, fill] (tid0) at (0.75,1.5){};
\node[circle, scale=0.75, fill] (tid1) at (0.75,3){};
\node[circle, scale=0.75, fill, red] (tid2) at (0.75,4.5){};
\draw[](tid1) -- (tid2);
\draw[](tid0) -- (tid1);
\end{tikzpicture}
\nodepart{three}
\footnotesize{3}
\nodepart{four}
\footnotesize{$1$}
};
 & 
\node[draw=black, rectangle split,  rectangle split parts=4] (sn0x1889c10){
\footnotesize{0.207562}
\nodepart{two}
\begin{tikzpicture}[scale=.2]
\node[circle, scale=0.75, fill] (tid0) at (1.5,1.5){};
\node[circle, scale=0.75, fill, red] (tid1) at (0.75,3){};
\node[circle, scale=0.75, fill, red] (tid2) at (2.25,3){};
\draw[](tid0) -- (tid1);
\draw[](tid0) -- (tid2);
\end{tikzpicture}
\nodepart{three}
\footnotesize{2.5}
\nodepart{four}
\footnotesize{$1$}
};
 & 
\\
};
\end{scope}
\begin{scope}[yshift=\leveltopIIIIIIIIII cm]
\matrix (line10) [column sep=1cm] {
\node[draw=black, rectangle split,  rectangle split parts=4] (sn0x1887520){
\footnotesize{1}
\nodepart{two}
\begin{tikzpicture}[scale=.2]
\node[circle, scale=0.75, fill] (tid0) at (0.75,1.5){};
\node[circle, scale=0.75, fill, red] (tid1) at (0.75,3){};
\draw[](tid0) -- (tid1);
\end{tikzpicture}
\nodepart{three}
\footnotesize{2}
\nodepart{four}
\footnotesize{$1$}
};
 & 
\\
};
\end{scope}
\begin{scope}[yshift=\leveltopIIIIIIIIIII cm]
\matrix (line11) [column sep=1cm] {
\node[draw=black, rectangle split,  rectangle split parts=4] (sn0x1887450){
\footnotesize{1}
\nodepart{two}
\begin{tikzpicture}[scale=.2]
\node[circle, scale=0.75, fill, red] (tid0) at (0.75,1.5){};
\end{tikzpicture}
\nodepart{three}
\footnotesize{1}
\nodepart{four}
\footnotesize{$$}
};
 & 
\\
};
\end{scope}
\begin{scope}[yshift=\leveltopIIIIIIIIIIII cm]
\matrix (line12) [column sep=1cm] {
\\
};
\end{scope}
\draw (sn0x1892300.south) -- (sn0x1891ba0.north);
\draw (sn0x1892300.south) -- (sn0x18906a0.north);
\draw (sn0x1891ba0.south) -- (sn0x188e220.north);
\draw (sn0x1891ba0.south) -- (sn0x188f0f0.north);
\draw (sn0x18906a0.south) -- (sn0x188d8a0.north);
\draw (sn0x18906a0.south) -- (sn0x188f0f0.north);
\draw (sn0x18906a0.south) -- (sn0x188ff90.north);
\draw (sn0x188e220.south) -- (sn0x188c370.north);
\draw (sn0x188e220.south) -- (sn0x188d690.north);
\draw (sn0x188f0f0.south) -- (sn0x188d690.north);
\draw (sn0x188f0f0.south) -- (sn0x188f020.north);
\draw (sn0x188d8a0.south) -- (sn0x188b180.north);
\draw (sn0x188d8a0.south) -- (sn0x188d690.north);
\draw (sn0x188d8a0.south) -- (sn0x188de90.north);
\draw (sn0x188ff90.south) -- (sn0x188de90.north);
\draw (sn0x188ff90.south) -- (sn0x188f020.north);
\draw (sn0x188c370.south) -- (sn0x1888fa0.north);
\draw (sn0x188c370.south) -- (sn0x188ae00.north);
\draw (sn0x188d690.south) -- (sn0x188ae00.north);
\draw (sn0x188d690.south) -- (sn0x188d460.north);
\draw (sn0x188f020.south) -- (sn0x188d460.north);
\draw (sn0x188f020.south) -- (sn0x188fd60.north);
\draw (sn0x188b180.south) -- (sn0x1889a70.north);
\draw (sn0x188b180.south) -- (sn0x188ae00.north);
\draw (sn0x188b180.south) -- (sn0x188b630.north);
\draw (sn0x188de90.south) -- (sn0x188b630.north);
\draw (sn0x188de90.south) -- (sn0x188d460.north);
\draw (sn0x1888fa0.south) -- (sn0x1888e50.north);
\draw (sn0x188ae00.south) -- (sn0x1888e50.north);
\draw (sn0x188ae00.south) -- (sn0x188acf0.north);
\draw (sn0x188d460.south) -- (sn0x188acf0.north);
\draw (sn0x188d460.south) -- (sn0x188c6a0.north);
\draw (sn0x188fd60.south) -- (sn0x188c6a0.north);
\draw (sn0x1889a70.south) -- (sn0x1888e50.north);
\draw (sn0x1889a70.south) -- (sn0x1889230.north);
\draw (sn0x188b630.south) -- (sn0x1889230.north);
\draw (sn0x188b630.south) -- (sn0x188acf0.north);
\draw (sn0x1888e50.south) -- (sn0x1888b50.north);
\draw (sn0x188acf0.south) -- (sn0x1888b50.north);
\draw (sn0x188acf0.south) -- (sn0x188a9f0.north);
\draw (sn0x188c6a0.south) -- (sn0x188a9f0.north);
\draw (sn0x188c6a0.south) -- (sn0x188c5b0.north);
\draw (sn0x1889230.south) -- (sn0x1888b50.north);
\draw (sn0x1888b50.south) -- (sn0x1888900.north);
\draw (sn0x188a9f0.south) -- (sn0x1888900.north);
\draw (sn0x188a9f0.south) -- (sn0x188a920.north);
\draw (sn0x188c5b0.south) -- (sn0x188a920.north);
\draw (sn0x1888900.south) -- (sn0x1888830.north);
\draw (sn0x188a920.south) -- (sn0x1888830.north);
\draw (sn0x188a920.south) -- (sn0x1889c10.north);
\draw (sn0x1888830.south) -- (sn0x1887520.north);
\draw (sn0x1889c10.south) -- (sn0x1887520.north);
\draw (sn0x1887520.south) -- (sn0x1887450.north);
\end{tikzpicture}

%%% Local Variables:
%%% TeX-master: "thesis/thesis.tex"
%%% End: 

  \caption{HLF vs. optimal solution for 0012346688}
  \label{fig:hlf-vs-opt-0012346688}
\end{figure}

\begin{figure}[ht]
  \centering
  \renewcommand{\leveltopI}{-19cm + \leveltop}
\renewcommand{\leveltopII}{-19cm + \leveltopI}
\renewcommand{\leveltopIII}{-19cm + \leveltopII}
\renewcommand{\leveltopIIII}{-19cm + \leveltopIII}
\renewcommand{\leveltopIIIII}{-19cm + \leveltopIIII}
\renewcommand{\leveltopIIIIII}{-19cm + \leveltopIIIII}
\renewcommand{\leveltopIIIIIII}{-19cm + \leveltopIIIIII}
\renewcommand{\leveltopIIIIIIII}{-19cm + \leveltopIIIIIII}
\renewcommand{\leveltopIIIIIIIII}{-19cm + \leveltopIIIIIIII}
\renewcommand{\leveltopIIIIIIIIII}{-19cm + \leveltopIIIIIIIII}
\renewcommand{\leveltopIIIIIIIIIII}{-19cm + \leveltopIIIIIIIIII}
\begin{tikzpicture}[scale=.2, anchor=south]
\begin{scope}[yshift=\leveltopI cm]
\matrix (line1) [column sep=.5cm] {
\node[draw=black, rectangle split,  rectangle split parts=4] (sn0x983680){
\footnotesize{100}
\nodepart{two}
\begin{tikzpicture}[scale=.2]
\node[circle, scale=0.75, fill] (tid0) at (3,1.5){};
\node[circle, scale=0.75, fill] (tid1) at (2.25,3){};
\node[circle, scale=0.75, fill] (tid3) at (2.25,4.5){};
\node[circle, scale=0.75, fill] (tid5) at (1.5,6){};
\node[circle, scale=0.75, fill] (tid7) at (1.5,7.5){};
\node[circle, scale=0.75, fill] (tid8) at (1.5,9){};
\node[circle, scale=0.75, fill, red] (tid9) at (0.75,10.5){};
\node[circle, scale=0.75, fill, red] (tid10) at (2.25,10.5){};
\draw[](tid8) -- (tid9);
\draw[](tid8) -- (tid10);
\draw[](tid7) -- (tid8);
\draw[](tid5) -- (tid7);
\node[circle, scale=0.75, fill, red] (tid6) at (3.75,6){};
\draw[](tid3) -- (tid5);
\draw[](tid3) -- (tid6);
\draw[](tid1) -- (tid3);
\node[circle, scale=0.75, fill] (tid2) at (5.25,3){};
\node[circle, scale=0.75, fill] (tid4) at (5.25,4.5){};
\draw[](tid2) -- (tid4);
\draw[](tid0) -- (tid1);
\draw[](tid0) -- (tid2);
\end{tikzpicture}
\nodepart{three}
\footnotesize{7.60833}
\nodepart{four}
\footnotesize{$33\:67$}
};
 & 
\\
};
\end{scope}
\begin{scope}[yshift=\leveltopII cm]
\matrix (line2) [column sep=.5cm] {
\node[draw=black, rectangle split,  rectangle split parts=4] (sn0x984240){
\footnotesize{33.3333}
\nodepart{two}
\begin{tikzpicture}[scale=.2]
\node[circle, scale=0.75, fill] (tid0) at (2.25,1.5){};
\node[circle, scale=0.75, fill] (tid1) at (1.5,3){};
\node[circle, scale=0.75, fill] (tid3) at (1.5,4.5){};
\node[circle, scale=0.75, fill] (tid5) at (1.5,6){};
\node[circle, scale=0.75, fill] (tid6) at (1.5,7.5){};
\node[circle, scale=0.75, fill] (tid7) at (1.5,9){};
\node[circle, scale=0.75, fill, red] (tid8) at (0.75,10.5){};
\node[circle, scale=0.75, fill, red] (tid9) at (2.25,10.5){};
\draw[](tid7) -- (tid8);
\draw[](tid7) -- (tid9);
\draw[](tid6) -- (tid7);
\draw[](tid5) -- (tid6);
\draw[](tid3) -- (tid5);
\draw[](tid1) -- (tid3);
\node[circle, scale=0.75, fill] (tid2) at (3.75,3){};
\node[circle, scale=0.75, fill, red] (tid4) at (3.75,4.5){};
\draw[](tid2) -- (tid4);
\draw[](tid0) -- (tid1);
\draw[](tid0) -- (tid2);
\end{tikzpicture}
\nodepart{three}
\footnotesize{7.55556}
\nodepart{four}
\footnotesize{$33\:67$}
};
 & 
\node[draw=black, rectangle split,  rectangle split parts=4] (sn0x984830){
\footnotesize{66.6667}
\nodepart{two}
\begin{tikzpicture}[scale=.2]
\node[circle, scale=0.75, fill] (tid0) at (2.25,1.5){};
\node[circle, scale=0.75, fill] (tid1) at (1.5,3){};
\node[circle, scale=0.75, fill] (tid3) at (1.5,4.5){};
\node[circle, scale=0.75, fill] (tid5) at (0.75,6){};
\node[circle, scale=0.75, fill] (tid7) at (0.75,7.5){};
\node[circle, scale=0.75, fill] (tid8) at (0.75,9){};
\node[circle, scale=0.75, fill, red] (tid9) at (0.75,10.5){};
\draw[](tid8) -- (tid9);
\draw[](tid7) -- (tid8);
\draw[](tid5) -- (tid7);
\node[circle, scale=0.75, fill, red] (tid6) at (2.25,6){};
\draw[](tid3) -- (tid5);
\draw[](tid3) -- (tid6);
\draw[](tid1) -- (tid3);
\node[circle, scale=0.75, fill] (tid2) at (3.75,3){};
\node[circle, scale=0.75, fill, red] (tid4) at (3.75,4.5){};
\draw[](tid2) -- (tid4);
\draw[](tid0) -- (tid1);
\draw[](tid0) -- (tid2);
\end{tikzpicture}
\nodepart{three}
\footnotesize{7.13471}
\nodepart{four}
\footnotesize{$33\:33\:33$}
};
 & 
\\
};
\end{scope}
\begin{scope}[yshift=\leveltopIII cm]
\matrix (line3) [column sep=.5cm] {
\node[draw=black, rectangle split,  rectangle split parts=4] (sn0x984fb0){
\footnotesize{11.1111}
\nodepart{two}
\begin{tikzpicture}[scale=.2]
\node[circle, scale=0.75, fill] (tid0) at (2.25,1.5){};
\node[circle, scale=0.75, fill] (tid1) at (1.5,3){};
\node[circle, scale=0.75, fill] (tid3) at (1.5,4.5){};
\node[circle, scale=0.75, fill] (tid4) at (1.5,6){};
\node[circle, scale=0.75, fill] (tid5) at (1.5,7.5){};
\node[circle, scale=0.75, fill] (tid6) at (1.5,9){};
\node[circle, scale=0.75, fill, red] (tid7) at (0.75,10.5){};
\node[circle, scale=0.75, fill, red] (tid8) at (2.25,10.5){};
\draw[](tid6) -- (tid7);
\draw[](tid6) -- (tid8);
\draw[](tid5) -- (tid6);
\draw[](tid4) -- (tid5);
\draw[](tid3) -- (tid4);
\draw[](tid1) -- (tid3);
\node[circle, scale=0.75, fill, red] (tid2) at (3.75,3){};
\draw[](tid0) -- (tid1);
\draw[](tid0) -- (tid2);
\end{tikzpicture}
\nodepart{three}
\footnotesize{7.51042}
\nodepart{four}
\footnotesize{$33\:67$}
};
 & 
\node[draw=black, rectangle split,  rectangle split parts=4] (sn0x984d10){
\footnotesize{44.4444}
\nodepart{two}
\begin{tikzpicture}[scale=.2]
\node[circle, scale=0.75, fill] (tid0) at (1.5,1.5){};
\node[circle, scale=0.75, fill] (tid1) at (0.75,3){};
\node[circle, scale=0.75, fill] (tid3) at (0.75,4.5){};
\node[circle, scale=0.75, fill] (tid5) at (0.75,6){};
\node[circle, scale=0.75, fill] (tid6) at (0.75,7.5){};
\node[circle, scale=0.75, fill] (tid7) at (0.75,9){};
\node[circle, scale=0.75, fill, red] (tid8) at (0.75,10.5){};
\draw[](tid7) -- (tid8);
\draw[](tid6) -- (tid7);
\draw[](tid5) -- (tid6);
\draw[](tid3) -- (tid5);
\draw[](tid1) -- (tid3);
\node[circle, scale=0.75, fill] (tid2) at (2.25,3){};
\node[circle, scale=0.75, fill, red] (tid4) at (2.25,4.5){};
\draw[](tid2) -- (tid4);
\draw[](tid0) -- (tid1);
\draw[](tid0) -- (tid2);
\end{tikzpicture}
\nodepart{three}
\footnotesize{7.07812}
\nodepart{four}
\footnotesize{$50\:50$}
};
 & 
\node[draw=black, rectangle split,  rectangle split parts=4] (sn0x9897c0){
\footnotesize{22.2222}
\nodepart{two}
\begin{tikzpicture}[scale=.2]
\node[circle, scale=0.75, fill] (tid0) at (2.25,1.5){};
\node[circle, scale=0.75, fill] (tid1) at (1.5,3){};
\node[circle, scale=0.75, fill] (tid3) at (1.5,4.5){};
\node[circle, scale=0.75, fill] (tid4) at (0.75,6){};
\node[circle, scale=0.75, fill] (tid6) at (0.75,7.5){};
\node[circle, scale=0.75, fill] (tid7) at (0.75,9){};
\node[circle, scale=0.75, fill, red] (tid8) at (0.75,10.5){};
\draw[](tid7) -- (tid8);
\draw[](tid6) -- (tid7);
\draw[](tid4) -- (tid6);
\node[circle, scale=0.75, fill, red] (tid5) at (2.25,6){};
\draw[](tid3) -- (tid4);
\draw[](tid3) -- (tid5);
\draw[](tid1) -- (tid3);
\node[circle, scale=0.75, fill, red] (tid2) at (3.75,3){};
\draw[](tid0) -- (tid1);
\draw[](tid0) -- (tid2);
\end{tikzpicture}
\nodepart{three}
\footnotesize{7.07658}
\nodepart{four}
\footnotesize{$33\:33\:33$}
};
 & 
\node[draw=black, rectangle split,  rectangle split parts=4] (sn0x98a0b0){
\footnotesize{22.2222}
\nodepart{two}
\begin{tikzpicture}[scale=.2]
\node[circle, scale=0.75, fill] (tid0) at (2.25,1.5){};
\node[circle, scale=0.75, fill] (tid1) at (1.5,3){};
\node[circle, scale=0.75, fill] (tid3) at (1.5,4.5){};
\node[circle, scale=0.75, fill] (tid5) at (0.75,6){};
\node[circle, scale=0.75, fill] (tid7) at (0.75,7.5){};
\node[circle, scale=0.75, fill, red] (tid8) at (0.75,9){};
\draw[](tid7) -- (tid8);
\draw[](tid5) -- (tid7);
\node[circle, scale=0.75, fill, red] (tid6) at (2.25,6){};
\draw[](tid3) -- (tid5);
\draw[](tid3) -- (tid6);
\draw[](tid1) -- (tid3);
\node[circle, scale=0.75, fill] (tid2) at (3.75,3){};
\node[circle, scale=0.75, fill, red] (tid4) at (3.75,4.5){};
\draw[](tid2) -- (tid4);
\draw[](tid0) -- (tid1);
\draw[](tid0) -- (tid2);
\end{tikzpicture}
\nodepart{three}
\footnotesize{6.24942}
\nodepart{four}
\footnotesize{$33\:33\:33$}
};
 & 
\\
};
\end{scope}
\draw (sn0x983680.south) -- (sn0x984240.north);
\draw (sn0x983680.south) -- (sn0x984830.north);
\draw (sn0x984240.south) -- (sn0x984fb0.north);
\draw (sn0x984240.south) -- (sn0x984d10.north);
\draw (sn0x984830.south) -- (sn0x9897c0.north);
\draw (sn0x984830.south) -- (sn0x984d10.north);
\draw (sn0x984830.south) -- (sn0x98a0b0.north);
\end{tikzpicture}
%%% Local Variables:
%%% TeX-master: "thesis/thesis.tex"
%%% End: 
  \renewcommand{\leveltopI}{-19cm + \leveltop}
\renewcommand{\leveltopII}{-19cm + \leveltopI}
\renewcommand{\leveltopIII}{-19cm + \leveltopII}
\renewcommand{\leveltopIIII}{-19cm + \leveltopIII}
\renewcommand{\leveltopIIIII}{-19cm + \leveltopIIII}
\renewcommand{\leveltopIIIIII}{-19cm + \leveltopIIIII}
\renewcommand{\leveltopIIIIIII}{-19cm + \leveltopIIIIII}
\renewcommand{\leveltopIIIIIIII}{-19cm + \leveltopIIIIIII}
\renewcommand{\leveltopIIIIIIIII}{-19cm + \leveltopIIIIIIII}
\renewcommand{\leveltopIIIIIIIIII}{-19cm + \leveltopIIIIIIIII}
\renewcommand{\leveltopIIIIIIIIIII}{-19cm + \leveltopIIIIIIIIII}
\begin{tikzpicture}[scale=.2, anchor=south]
\begin{scope}[yshift=\leveltopI cm]
\matrix (line1)[column sep=0.5cm] {
\node[draw=black, rectangle split,  rectangle split parts=4] (sn0x18e59c0){
\footnotesize{100}
\nodepart{two}
\begin{tikzpicture}[scale=.2]
\node[circle, scale=0.75, fill] (tid0) at (3,1.5){};
\node[circle, scale=0.75, fill] (tid1) at (2.25,3){};
\node[circle, scale=0.75, fill] (tid3) at (2.25,4.5){};
\node[circle, scale=0.75, fill] (tid5) at (1.5,6){};
\node[circle, scale=0.75, fill] (tid7) at (1.5,7.5){};
\node[circle, scale=0.75, fill] (tid8) at (1.5,9){};
\node[circle, scale=0.75, fill, red] (tid9) at (0.75,10.5){};
\node[circle, scale=0.75, fill, red] (tid10) at (2.25,10.5){};
\draw[](tid8) -- (tid9);
\draw[](tid8) -- (tid10);
\draw[](tid7) -- (tid8);
\draw[](tid5) -- (tid7);
\node[circle, scale=0.75, fill] (tid6) at (3.75,6){};
\draw[](tid3) -- (tid5);
\draw[](tid3) -- (tid6);
\draw[](tid1) -- (tid3);
\node[circle, scale=0.75, fill] (tid2) at (5.25,3){};
\node[circle, scale=0.75, fill, red] (tid4) at (5.25,4.5){};
\draw[](tid2) -- (tid4);
\draw[](tid0) -- (tid1);
\draw[](tid0) -- (tid2);
\end{tikzpicture}
\nodepart{three}
\footnotesize{7.60798}
\nodepart{four}
\footnotesize{$33\:67$}
};
 & 
\\
};
\end{scope}
\begin{scope}[yshift=\leveltopII cm]
\matrix (line2)[column sep=0.5cm] {
\node[draw=black, rectangle split,  rectangle split parts=4] (sn0x18e2cc0){
\footnotesize{33.3333}
\nodepart{two}
\begin{tikzpicture}[scale=.2]
\node[circle, scale=0.75, fill] (tid0) at (3,1.5){};
\node[circle, scale=0.75, fill] (tid1) at (2.25,3){};
\node[circle, scale=0.75, fill] (tid3) at (2.25,4.5){};
\node[circle, scale=0.75, fill] (tid4) at (1.5,6){};
\node[circle, scale=0.75, fill] (tid6) at (1.5,7.5){};
\node[circle, scale=0.75, fill] (tid7) at (1.5,9){};
\node[circle, scale=0.75, fill, red] (tid8) at (0.75,10.5){};
\node[circle, scale=0.75, fill, red] (tid9) at (2.25,10.5){};
\draw[](tid7) -- (tid8);
\draw[](tid7) -- (tid9);
\draw[](tid6) -- (tid7);
\draw[](tid4) -- (tid6);
\node[circle, scale=0.75, fill, red] (tid5) at (3.75,6){};
\draw[](tid3) -- (tid4);
\draw[](tid3) -- (tid5);
\draw[](tid1) -- (tid3);
\node[circle, scale=0.75, fill] (tid2) at (5.25,3){};
\draw[](tid0) -- (tid1);
\draw[](tid0) -- (tid2);
\end{tikzpicture}
\nodepart{three}
\footnotesize{7.55453}
\nodepart{four}
\footnotesize{$33\:67$}
};
 & 
\node[draw=black, rectangle split,  rectangle split parts=4] (sn0x18e5770){
\footnotesize{66.6667}
\nodepart{two}
\begin{tikzpicture}[scale=.2]
\node[circle, scale=0.75, fill] (tid0) at (2.25,1.5){};
\node[circle, scale=0.75, fill] (tid1) at (1.5,3){};
\node[circle, scale=0.75, fill] (tid3) at (1.5,4.5){};
\node[circle, scale=0.75, fill] (tid5) at (0.75,6){};
\node[circle, scale=0.75, fill] (tid7) at (0.75,7.5){};
\node[circle, scale=0.75, fill] (tid8) at (0.75,9){};
\node[circle, scale=0.75, fill, red] (tid9) at (0.75,10.5){};
\draw[](tid8) -- (tid9);
\draw[](tid7) -- (tid8);
\draw[](tid5) -- (tid7);
\node[circle, scale=0.75, fill, red] (tid6) at (2.25,6){};
\draw[](tid3) -- (tid5);
\draw[](tid3) -- (tid6);
\draw[](tid1) -- (tid3);
\node[circle, scale=0.75, fill] (tid2) at (3.75,3){};
\node[circle, scale=0.75, fill, red] (tid4) at (3.75,4.5){};
\draw[](tid2) -- (tid4);
\draw[](tid0) -- (tid1);
\draw[](tid0) -- (tid2);
\end{tikzpicture}
\nodepart{three}
\footnotesize{7.13471}
\nodepart{four}
\footnotesize{$33\:33\:33$}
};
 & 
\\
};
\end{scope}
\begin{scope}[yshift=\leveltopIII cm]
\matrix (line3)[column sep=0.5cm] {
\node[draw=black, rectangle split,  rectangle split parts=4] (sn0x18e21e0){
\footnotesize{11.1111}
\nodepart{two}
\begin{tikzpicture}[scale=.2]
\node[circle, scale=0.75, fill] (tid0) at (2.25,1.5){};
\node[circle, scale=0.75, fill] (tid1) at (1.5,3){};
\node[circle, scale=0.75, fill] (tid3) at (1.5,4.5){};
\node[circle, scale=0.75, fill] (tid4) at (1.5,6){};
\node[circle, scale=0.75, fill] (tid5) at (1.5,7.5){};
\node[circle, scale=0.75, fill] (tid6) at (1.5,9){};
\node[circle, scale=0.75, fill, red] (tid7) at (0.75,10.5){};
\node[circle, scale=0.75, fill, red] (tid8) at (2.25,10.5){};
\draw[](tid6) -- (tid7);
\draw[](tid6) -- (tid8);
\draw[](tid5) -- (tid6);
\draw[](tid4) -- (tid5);
\draw[](tid3) -- (tid4);
\draw[](tid1) -- (tid3);
\node[circle, scale=0.75, fill, red] (tid2) at (3.75,3){};
\draw[](tid0) -- (tid1);
\draw[](tid0) -- (tid2);
\end{tikzpicture}
\nodepart{three}
\footnotesize{7.51042}
\nodepart{four}
\footnotesize{$33\:67$}
};
 & 
\node[draw=black, rectangle split,  rectangle split parts=4] (sn0x18e1e60){
\footnotesize{44.4444}
\nodepart{two}
\begin{tikzpicture}[scale=.2]
\node[circle, scale=0.75, fill] (tid0) at (2.25,1.5){};
\node[circle, scale=0.75, fill] (tid1) at (1.5,3){};
\node[circle, scale=0.75, fill] (tid3) at (1.5,4.5){};
\node[circle, scale=0.75, fill] (tid4) at (0.75,6){};
\node[circle, scale=0.75, fill] (tid6) at (0.75,7.5){};
\node[circle, scale=0.75, fill] (tid7) at (0.75,9){};
\node[circle, scale=0.75, fill, red] (tid8) at (0.75,10.5){};
\draw[](tid7) -- (tid8);
\draw[](tid6) -- (tid7);
\draw[](tid4) -- (tid6);
\node[circle, scale=0.75, fill, red] (tid5) at (2.25,6){};
\draw[](tid3) -- (tid4);
\draw[](tid3) -- (tid5);
\draw[](tid1) -- (tid3);
\node[circle, scale=0.75, fill, red] (tid2) at (3.75,3){};
\draw[](tid0) -- (tid1);
\draw[](tid0) -- (tid2);
\end{tikzpicture}
\nodepart{three}
\footnotesize{7.07658}
\nodepart{four}
\footnotesize{$33\:33\:33$}
};
 & 
\node[draw=black, rectangle split,  rectangle split parts=4] (sn0x18e45f0){
\footnotesize{22.2222}
\nodepart{two}
\begin{tikzpicture}[scale=.2]
\node[circle, scale=0.75, fill] (tid0) at (1.5,1.5){};
\node[circle, scale=0.75, fill] (tid1) at (0.75,3){};
\node[circle, scale=0.75, fill] (tid3) at (0.75,4.5){};
\node[circle, scale=0.75, fill] (tid5) at (0.75,6){};
\node[circle, scale=0.75, fill] (tid6) at (0.75,7.5){};
\node[circle, scale=0.75, fill] (tid7) at (0.75,9){};
\node[circle, scale=0.75, fill, red] (tid8) at (0.75,10.5){};
\draw[](tid7) -- (tid8);
\draw[](tid6) -- (tid7);
\draw[](tid5) -- (tid6);
\draw[](tid3) -- (tid5);
\draw[](tid1) -- (tid3);
\node[circle, scale=0.75, fill] (tid2) at (2.25,3){};
\node[circle, scale=0.75, fill, red] (tid4) at (2.25,4.5){};
\draw[](tid2) -- (tid4);
\draw[](tid0) -- (tid1);
\draw[](tid0) -- (tid2);
\end{tikzpicture}
\nodepart{three}
\footnotesize{7.07812}
\nodepart{four}
\footnotesize{$50\:50$}
};
 & 
\node[draw=black, rectangle split,  rectangle split parts=4] (sn0x18e49e0){
\footnotesize{22.2222}
\nodepart{two}
\begin{tikzpicture}[scale=.2]
\node[circle, scale=0.75, fill] (tid0) at (2.25,1.5){};
\node[circle, scale=0.75, fill] (tid1) at (1.5,3){};
\node[circle, scale=0.75, fill] (tid3) at (1.5,4.5){};
\node[circle, scale=0.75, fill] (tid5) at (0.75,6){};
\node[circle, scale=0.75, fill] (tid7) at (0.75,7.5){};
\node[circle, scale=0.75, fill, red] (tid8) at (0.75,9){};
\draw[](tid7) -- (tid8);
\draw[](tid5) -- (tid7);
\node[circle, scale=0.75, fill, red] (tid6) at (2.25,6){};
\draw[](tid3) -- (tid5);
\draw[](tid3) -- (tid6);
\draw[](tid1) -- (tid3);
\node[circle, scale=0.75, fill] (tid2) at (3.75,3){};
\node[circle, scale=0.75, fill, red] (tid4) at (3.75,4.5){};
\draw[](tid2) -- (tid4);
\draw[](tid0) -- (tid1);
\draw[](tid0) -- (tid2);
\end{tikzpicture}
\nodepart{three}
\footnotesize{6.24942}
\nodepart{four}
\footnotesize{$33\:33\:33$}
};
 & 
\\
};
\end{scope}
\draw (sn0x18e59c0.south) -- (sn0x18e2cc0.north);
\draw (sn0x18e59c0.south) -- (sn0x18e5770.north);
\draw (sn0x18e2cc0.south) -- (sn0x18e21e0.north);
\draw (sn0x18e2cc0.south) -- (sn0x18e1e60.north);
\draw (sn0x18e5770.south) -- (sn0x18e1e60.north);
\draw (sn0x18e5770.south) -- (sn0x18e45f0.north);
\draw (sn0x18e5770.south) -- (sn0x18e49e0.north);
\end{tikzpicture}
%%% Local Variables:
%%% TeX-master: "thesis/thesis.tex"
%%% End: 

  \caption{HLF vs. optimal solution for 0012446788 (taken from Ernst Mayr)}
  \label{fig:hlf-vs-opt-0012446788}
\end{figure}

\begin{figure}[ht]
  \centering
  \renewcommand{\leveltopI}{-19cm + \leveltop}
\renewcommand{\leveltopII}{-19cm + \leveltopI}
\renewcommand{\leveltopIII}{-19cm + \leveltopII}
\renewcommand{\leveltopIIII}{-19cm + \leveltopIII}
\renewcommand{\leveltopIIIII}{-19cm + \leveltopIIII}
\renewcommand{\leveltopIIIIII}{-19cm + \leveltopIIIII}
\renewcommand{\leveltopIIIIIII}{-19cm + \leveltopIIIIII}
\renewcommand{\leveltopIIIIIIII}{-19cm + \leveltopIIIIIII}
\renewcommand{\leveltopIIIIIIIII}{-19cm + \leveltopIIIIIIII}
\renewcommand{\leveltopIIIIIIIIII}{-19cm + \leveltopIIIIIIIII}
\renewcommand{\leveltopIIIIIIIIIII}{-19cm + \leveltopIIIIIIIIII}
\renewcommand{\leveltopIIIIIIIIIIII}{-19cm + \leveltopIIIIIIIIIII}
\begin{tikzpicture}[scale=.2, anchor=south]
\begin{scope}[yshift=\leveltopI cm]
\matrix (line1)[column sep=0.5cm] {
\node[draw=black, rectangle split,  rectangle split parts=4] (sn0x90bf0b8){
\footnotesize{100}
\nodepart{two}
\begin{tikzpicture}[scale=.2]
\node[circle, scale=0.75, fill] (tid0) at (3,1.5){};
\node[circle, scale=0.75, fill] (tid1) at (2.25,3){};
\node[circle, scale=0.75, fill] (tid3) at (2.25,4.5){};
\node[circle, scale=0.75, fill] (tid5) at (2.25,6){};
\node[circle, scale=0.75, fill] (tid7) at (1.5,7.5){};
\node[circle, scale=0.75, fill] (tid9) at (1.5,9){};
\node[circle, scale=0.75, fill, red] (tid10) at (0.75,10.5){};
\node[circle, scale=0.75, fill, red] (tid11) at (2.25,10.5){};
\draw[](tid9) -- (tid10);
\draw[](tid9) -- (tid11);
\draw[](tid7) -- (tid9);
\node[circle, scale=0.75, fill, red] (tid8) at (3.75,7.5){};
\draw[](tid5) -- (tid7);
\draw[](tid5) -- (tid8);
\draw[](tid3) -- (tid5);
\draw[](tid1) -- (tid3);
\node[circle, scale=0.75, fill] (tid2) at (5.25,3){};
\node[circle, scale=0.75, fill] (tid4) at (5.25,4.5){};
\node[circle, scale=0.75, fill] (tid6) at (5.25,6){};
\draw[](tid4) -- (tid6);
\draw[](tid2) -- (tid4);
\draw[](tid0) -- (tid1);
\draw[](tid0) -- (tid2);
\end{tikzpicture}
\nodepart{three}
\footnotesize{7.77328}
\nodepart{four}
\footnotesize{$33\:67$}
};
 & 
\\
};
\end{scope}
\begin{scope}[yshift=\leveltopII cm]
\matrix (line2)[column sep=0.5cm] {
\node[draw=black, rectangle split,  rectangle split parts=4] (sn0x90be790){
\footnotesize{33.3333}
\nodepart{two}
\begin{tikzpicture}[scale=.2]
\node[circle, scale=0.75, fill] (tid0) at (2.25,1.5){};
\node[circle, scale=0.75, fill] (tid1) at (1.5,3){};
\node[circle, scale=0.75, fill] (tid3) at (1.5,4.5){};
\node[circle, scale=0.75, fill] (tid5) at (1.5,6){};
\node[circle, scale=0.75, fill] (tid7) at (1.5,7.5){};
\node[circle, scale=0.75, fill] (tid8) at (1.5,9){};
\node[circle, scale=0.75, fill, red] (tid9) at (0.75,10.5){};
\node[circle, scale=0.75, fill, red] (tid10) at (2.25,10.5){};
\draw[](tid8) -- (tid9);
\draw[](tid8) -- (tid10);
\draw[](tid7) -- (tid8);
\draw[](tid5) -- (tid7);
\draw[](tid3) -- (tid5);
\draw[](tid1) -- (tid3);
\node[circle, scale=0.75, fill] (tid2) at (3.75,3){};
\node[circle, scale=0.75, fill] (tid4) at (3.75,4.5){};
\node[circle, scale=0.75, fill, red] (tid6) at (3.75,6){};
\draw[](tid4) -- (tid6);
\draw[](tid2) -- (tid4);
\draw[](tid0) -- (tid1);
\draw[](tid0) -- (tid2);
\end{tikzpicture}
\nodepart{three}
\footnotesize{7.66696}
\nodepart{four}
\footnotesize{$33\:67$}
};
 & 
\node[draw=black, rectangle split,  rectangle split parts=4] (sn0x90bd580){
\footnotesize{66.6667}
\nodepart{two}
\begin{tikzpicture}[scale=.2]
\node[circle, scale=0.75, fill] (tid0) at (2.25,1.5){};
\node[circle, scale=0.75, fill] (tid1) at (1.5,3){};
\node[circle, scale=0.75, fill] (tid3) at (1.5,4.5){};
\node[circle, scale=0.75, fill] (tid5) at (1.5,6){};
\node[circle, scale=0.75, fill] (tid7) at (0.75,7.5){};
\node[circle, scale=0.75, fill] (tid9) at (0.75,9){};
\node[circle, scale=0.75, fill, red] (tid10) at (0.75,10.5){};
\draw[](tid9) -- (tid10);
\draw[](tid7) -- (tid9);
\node[circle, scale=0.75, fill, red] (tid8) at (2.25,7.5){};
\draw[](tid5) -- (tid7);
\draw[](tid5) -- (tid8);
\draw[](tid3) -- (tid5);
\draw[](tid1) -- (tid3);
\node[circle, scale=0.75, fill] (tid2) at (3.75,3){};
\node[circle, scale=0.75, fill] (tid4) at (3.75,4.5){};
\node[circle, scale=0.75, fill, red] (tid6) at (3.75,6){};
\draw[](tid4) -- (tid6);
\draw[](tid2) -- (tid4);
\draw[](tid0) -- (tid1);
\draw[](tid0) -- (tid2);
\end{tikzpicture}
\nodepart{three}
\footnotesize{7.32644}
\nodepart{four}
\footnotesize{$33\:33\:33$}
};
 & 
\\
};
\end{scope}
\begin{scope}[yshift=\leveltopIII cm]
\matrix (line3)[column sep=0.5cm] {
\node[draw=black, rectangle split,  rectangle split parts=4] (sn0x90beb08){
\footnotesize{11.1111}
\nodepart{two}
\begin{tikzpicture}[scale=.2]
\node[circle, scale=0.75, fill] (tid0) at (2.25,1.5){};
\node[circle, scale=0.75, fill] (tid1) at (1.5,3){};
\node[circle, scale=0.75, fill] (tid3) at (1.5,4.5){};
\node[circle, scale=0.75, fill] (tid5) at (1.5,6){};
\node[circle, scale=0.75, fill] (tid6) at (1.5,7.5){};
\node[circle, scale=0.75, fill] (tid7) at (1.5,9){};
\node[circle, scale=0.75, fill, red] (tid8) at (0.75,10.5){};
\node[circle, scale=0.75, fill, red] (tid9) at (2.25,10.5){};
\draw[](tid7) -- (tid8);
\draw[](tid7) -- (tid9);
\draw[](tid6) -- (tid7);
\draw[](tid5) -- (tid6);
\draw[](tid3) -- (tid5);
\draw[](tid1) -- (tid3);
\node[circle, scale=0.75, fill] (tid2) at (3.75,3){};
\node[circle, scale=0.75, fill, red] (tid4) at (3.75,4.5){};
\draw[](tid2) -- (tid4);
\draw[](tid0) -- (tid1);
\draw[](tid0) -- (tid2);
\end{tikzpicture}
\nodepart{three}
\footnotesize{7.55556}
\nodepart{four}
\footnotesize{$33\:67$}
};
 & 
\node[draw=black, rectangle split,  rectangle split parts=4] (sn0x90bf898){
\footnotesize{44.4444}
\nodepart{two}
\begin{tikzpicture}[scale=.2]
\node[circle, scale=0.75, fill] (tid0) at (1.5,1.5){};
\node[circle, scale=0.75, fill] (tid1) at (0.75,3){};
\node[circle, scale=0.75, fill] (tid3) at (0.75,4.5){};
\node[circle, scale=0.75, fill] (tid5) at (0.75,6){};
\node[circle, scale=0.75, fill] (tid7) at (0.75,7.5){};
\node[circle, scale=0.75, fill] (tid8) at (0.75,9){};
\node[circle, scale=0.75, fill, red] (tid9) at (0.75,10.5){};
\draw[](tid8) -- (tid9);
\draw[](tid7) -- (tid8);
\draw[](tid5) -- (tid7);
\draw[](tid3) -- (tid5);
\draw[](tid1) -- (tid3);
\node[circle, scale=0.75, fill] (tid2) at (2.25,3){};
\node[circle, scale=0.75, fill] (tid4) at (2.25,4.5){};
\node[circle, scale=0.75, fill, red] (tid6) at (2.25,6){};
\draw[](tid4) -- (tid6);
\draw[](tid2) -- (tid4);
\draw[](tid0) -- (tid1);
\draw[](tid0) -- (tid2);
\end{tikzpicture}
\nodepart{three}
\footnotesize{7.22266}
\nodepart{four}
\footnotesize{$50\:50$}
};
 & 
\node[draw=black, rectangle split,  rectangle split parts=4] (sn0x90c1df8){
\footnotesize{22.2222}
\nodepart{two}
\begin{tikzpicture}[scale=.2]
\node[circle, scale=0.75, fill] (tid0) at (2.25,1.5){};
\node[circle, scale=0.75, fill] (tid1) at (1.5,3){};
\node[circle, scale=0.75, fill] (tid3) at (1.5,4.5){};
\node[circle, scale=0.75, fill] (tid5) at (1.5,6){};
\node[circle, scale=0.75, fill] (tid6) at (0.75,7.5){};
\node[circle, scale=0.75, fill] (tid8) at (0.75,9){};
\node[circle, scale=0.75, fill, red] (tid9) at (0.75,10.5){};
\draw[](tid8) -- (tid9);
\draw[](tid6) -- (tid8);
\node[circle, scale=0.75, fill, red] (tid7) at (2.25,7.5){};
\draw[](tid5) -- (tid6);
\draw[](tid5) -- (tid7);
\draw[](tid3) -- (tid5);
\draw[](tid1) -- (tid3);
\node[circle, scale=0.75, fill] (tid2) at (3.75,3){};
\node[circle, scale=0.75, fill, red] (tid4) at (3.75,4.5){};
\draw[](tid2) -- (tid4);
\draw[](tid0) -- (tid1);
\draw[](tid0) -- (tid2);
\end{tikzpicture}
\nodepart{three}
\footnotesize{7.19387}
\nodepart{four}
\footnotesize{$33\:33\:33$}
};
 & 
\node[draw=black, rectangle split,  rectangle split parts=4] (sn0x90c3bd8){
\footnotesize{22.2222}
\nodepart{two}
\begin{tikzpicture}[scale=.2]
\node[circle, scale=0.75, fill] (tid0) at (2.25,1.5){};
\node[circle, scale=0.75, fill] (tid1) at (1.5,3){};
\node[circle, scale=0.75, fill] (tid3) at (1.5,4.5){};
\node[circle, scale=0.75, fill] (tid5) at (1.5,6){};
\node[circle, scale=0.75, fill] (tid7) at (0.75,7.5){};
\node[circle, scale=0.75, fill, red] (tid9) at (0.75,9){};
\draw[](tid7) -- (tid9);
\node[circle, scale=0.75, fill, red] (tid8) at (2.25,7.5){};
\draw[](tid5) -- (tid7);
\draw[](tid5) -- (tid8);
\draw[](tid3) -- (tid5);
\draw[](tid1) -- (tid3);
\node[circle, scale=0.75, fill] (tid2) at (3.75,3){};
\node[circle, scale=0.75, fill] (tid4) at (3.75,4.5){};
\node[circle, scale=0.75, fill, red] (tid6) at (3.75,6){};
\draw[](tid4) -- (tid6);
\draw[](tid2) -- (tid4);
\draw[](tid0) -- (tid1);
\draw[](tid0) -- (tid2);
\end{tikzpicture}
\nodepart{three}
\footnotesize{6.56279}
\nodepart{four}
\footnotesize{$33\:33\:33$}
};
 & 
\\
};
\end{scope}
\draw (sn0x90bf0b8.south) -- (sn0x90be790.north);
\draw (sn0x90bf0b8.south) -- (sn0x90bd580.north);
\draw (sn0x90be790.south) -- (sn0x90beb08.north);
\draw (sn0x90be790.south) -- (sn0x90bf898.north);
\draw (sn0x90bd580.south) -- (sn0x90c1df8.north);
\draw (sn0x90bd580.south) -- (sn0x90bf898.north);
\draw (sn0x90bd580.south) -- (sn0x90c3bd8.north);
\end{tikzpicture}
%%% Local Variables:
%%% TeX-master: "thesis/thesis.tex"
%%% End: 
  \renewcommand{\leveltopI}{-19cm + \leveltop}
\renewcommand{\leveltopII}{-19cm + \leveltopI}
\renewcommand{\leveltopIII}{-19cm + \leveltopII}
\renewcommand{\leveltopIIII}{-19cm + \leveltopIII}
\renewcommand{\leveltopIIIII}{-19cm + \leveltopIIII}
\renewcommand{\leveltopIIIIII}{-19cm + \leveltopIIIII}
\renewcommand{\leveltopIIIIIII}{-19cm + \leveltopIIIIII}
\renewcommand{\leveltopIIIIIIII}{-19cm + \leveltopIIIIIII}
\renewcommand{\leveltopIIIIIIIII}{-19cm + \leveltopIIIIIIII}
\renewcommand{\leveltopIIIIIIIIII}{-19cm + \leveltopIIIIIIIII}
\renewcommand{\leveltopIIIIIIIIIII}{-19cm + \leveltopIIIIIIIIII}
\renewcommand{\leveltopIIIIIIIIIIII}{-19cm + \leveltopIIIIIIIIIII}
\begin{tikzpicture}[scale=.2, anchor=south]
\begin{scope}[yshift=\leveltopI cm]
\matrix (line1)[column sep=0.5cm] {
\node[draw=black, rectangle split,  rectangle split parts=4] (sn0x9748440){
\footnotesize{100}
\nodepart{two}
\begin{tikzpicture}[scale=.2]
\node[circle, scale=0.75, fill] (tid0) at (3,1.5){};
\node[circle, scale=0.75, fill] (tid1) at (2.25,3){};
\node[circle, scale=0.75, fill] (tid3) at (2.25,4.5){};
\node[circle, scale=0.75, fill] (tid5) at (2.25,6){};
\node[circle, scale=0.75, fill] (tid7) at (1.5,7.5){};
\node[circle, scale=0.75, fill] (tid9) at (1.5,9){};
\node[circle, scale=0.75, fill, red] (tid10) at (0.75,10.5){};
\node[circle, scale=0.75, fill, red] (tid11) at (2.25,10.5){};
\draw[](tid9) -- (tid10);
\draw[](tid9) -- (tid11);
\draw[](tid7) -- (tid9);
\node[circle, scale=0.75, fill] (tid8) at (3.75,7.5){};
\draw[](tid5) -- (tid7);
\draw[](tid5) -- (tid8);
\draw[](tid3) -- (tid5);
\draw[](tid1) -- (tid3);
\node[circle, scale=0.75, fill] (tid2) at (5.25,3){};
\node[circle, scale=0.75, fill] (tid4) at (5.25,4.5){};
\node[circle, scale=0.75, fill, red] (tid6) at (5.25,6){};
\draw[](tid4) -- (tid6);
\draw[](tid2) -- (tid4);
\draw[](tid0) -- (tid1);
\draw[](tid0) -- (tid2);
\end{tikzpicture}
\nodepart{three}
\footnotesize{7.76688}
\nodepart{four}
\footnotesize{$33\:67$}
};
 & 
\\
};
\end{scope}
\begin{scope}[yshift=\leveltopII cm]
\matrix (line2)[column sep=0.5cm] {
\node[draw=black, rectangle split,  rectangle split parts=4] (sn0x97466f8){
\footnotesize{33.3333}
\nodepart{two}
\begin{tikzpicture}[scale=.2]
\node[circle, scale=0.75, fill] (tid0) at (3,1.5){};
\node[circle, scale=0.75, fill] (tid1) at (2.25,3){};
\node[circle, scale=0.75, fill] (tid3) at (2.25,4.5){};
\node[circle, scale=0.75, fill] (tid5) at (2.25,6){};
\node[circle, scale=0.75, fill] (tid6) at (1.5,7.5){};
\node[circle, scale=0.75, fill] (tid8) at (1.5,9){};
\node[circle, scale=0.75, fill, red] (tid9) at (0.75,10.5){};
\node[circle, scale=0.75, fill, red] (tid10) at (2.25,10.5){};
\draw[](tid8) -- (tid9);
\draw[](tid8) -- (tid10);
\draw[](tid6) -- (tid8);
\node[circle, scale=0.75, fill, red] (tid7) at (3.75,7.5){};
\draw[](tid5) -- (tid6);
\draw[](tid5) -- (tid7);
\draw[](tid3) -- (tid5);
\draw[](tid1) -- (tid3);
\node[circle, scale=0.75, fill] (tid2) at (5.25,3){};
\node[circle, scale=0.75, fill] (tid4) at (5.25,4.5){};
\draw[](tid2) -- (tid4);
\draw[](tid0) -- (tid1);
\draw[](tid0) -- (tid2);
\end{tikzpicture}
\nodepart{three}
\footnotesize{7.64776}
\nodepart{four}
\footnotesize{$33\:67$}
};
 & 
\node[draw=black, rectangle split,  rectangle split parts=4] (sn0x9748060){
\footnotesize{66.6667}
\nodepart{two}
\begin{tikzpicture}[scale=.2]
\node[circle, scale=0.75, fill] (tid0) at (2.25,1.5){};
\node[circle, scale=0.75, fill] (tid1) at (1.5,3){};
\node[circle, scale=0.75, fill] (tid3) at (1.5,4.5){};
\node[circle, scale=0.75, fill] (tid5) at (1.5,6){};
\node[circle, scale=0.75, fill] (tid7) at (0.75,7.5){};
\node[circle, scale=0.75, fill] (tid9) at (0.75,9){};
\node[circle, scale=0.75, fill, red] (tid10) at (0.75,10.5){};
\draw[](tid9) -- (tid10);
\draw[](tid7) -- (tid9);
\node[circle, scale=0.75, fill, red] (tid8) at (2.25,7.5){};
\draw[](tid5) -- (tid7);
\draw[](tid5) -- (tid8);
\draw[](tid3) -- (tid5);
\draw[](tid1) -- (tid3);
\node[circle, scale=0.75, fill] (tid2) at (3.75,3){};
\node[circle, scale=0.75, fill] (tid4) at (3.75,4.5){};
\node[circle, scale=0.75, fill, red] (tid6) at (3.75,6){};
\draw[](tid4) -- (tid6);
\draw[](tid2) -- (tid4);
\draw[](tid0) -- (tid1);
\draw[](tid0) -- (tid2);
\end{tikzpicture}
\nodepart{three}
\footnotesize{7.32644}
\nodepart{four}
\footnotesize{$33\:33\:33$}
};
 & 
\\
};
\end{scope}
\begin{scope}[yshift=\leveltopIII cm]
\matrix (line3)[column sep=0.5cm] {
\node[draw=black, rectangle split,  rectangle split parts=4] (sn0x9746d20){
\footnotesize{11.1111}
\nodepart{two}
\begin{tikzpicture}[scale=.2]
\node[circle, scale=0.75, fill] (tid0) at (2.25,1.5){};
\node[circle, scale=0.75, fill] (tid1) at (1.5,3){};
\node[circle, scale=0.75, fill] (tid3) at (1.5,4.5){};
\node[circle, scale=0.75, fill] (tid5) at (1.5,6){};
\node[circle, scale=0.75, fill] (tid6) at (1.5,7.5){};
\node[circle, scale=0.75, fill] (tid7) at (1.5,9){};
\node[circle, scale=0.75, fill, red] (tid8) at (0.75,10.5){};
\node[circle, scale=0.75, fill, red] (tid9) at (2.25,10.5){};
\draw[](tid7) -- (tid8);
\draw[](tid7) -- (tid9);
\draw[](tid6) -- (tid7);
\draw[](tid5) -- (tid6);
\draw[](tid3) -- (tid5);
\draw[](tid1) -- (tid3);
\node[circle, scale=0.75, fill] (tid2) at (3.75,3){};
\node[circle, scale=0.75, fill, red] (tid4) at (3.75,4.5){};
\draw[](tid2) -- (tid4);
\draw[](tid0) -- (tid1);
\draw[](tid0) -- (tid2);
\end{tikzpicture}
\nodepart{three}
\footnotesize{7.55556}
\nodepart{four}
\footnotesize{$33\:67$}
};
 & 
\node[draw=black, rectangle split,  rectangle split parts=4] (sn0x9745bf0){
\footnotesize{44.4444}
\nodepart{two}
\begin{tikzpicture}[scale=.2]
\node[circle, scale=0.75, fill] (tid0) at (2.25,1.5){};
\node[circle, scale=0.75, fill] (tid1) at (1.5,3){};
\node[circle, scale=0.75, fill] (tid3) at (1.5,4.5){};
\node[circle, scale=0.75, fill] (tid5) at (1.5,6){};
\node[circle, scale=0.75, fill] (tid6) at (0.75,7.5){};
\node[circle, scale=0.75, fill] (tid8) at (0.75,9){};
\node[circle, scale=0.75, fill, red] (tid9) at (0.75,10.5){};
\draw[](tid8) -- (tid9);
\draw[](tid6) -- (tid8);
\node[circle, scale=0.75, fill, red] (tid7) at (2.25,7.5){};
\draw[](tid5) -- (tid6);
\draw[](tid5) -- (tid7);
\draw[](tid3) -- (tid5);
\draw[](tid1) -- (tid3);
\node[circle, scale=0.75, fill] (tid2) at (3.75,3){};
\node[circle, scale=0.75, fill, red] (tid4) at (3.75,4.5){};
\draw[](tid2) -- (tid4);
\draw[](tid0) -- (tid1);
\draw[](tid0) -- (tid2);
\end{tikzpicture}
\nodepart{three}
\footnotesize{7.19387}
\nodepart{four}
\footnotesize{$33\:33\:33$}
};
 & 
\node[draw=black, rectangle split,  rectangle split parts=4] (sn0x9747af8){
\footnotesize{22.2222}
\nodepart{two}
\begin{tikzpicture}[scale=.2]
\node[circle, scale=0.75, fill] (tid0) at (1.5,1.5){};
\node[circle, scale=0.75, fill] (tid1) at (0.75,3){};
\node[circle, scale=0.75, fill] (tid3) at (0.75,4.5){};
\node[circle, scale=0.75, fill] (tid5) at (0.75,6){};
\node[circle, scale=0.75, fill] (tid7) at (0.75,7.5){};
\node[circle, scale=0.75, fill] (tid8) at (0.75,9){};
\node[circle, scale=0.75, fill, red] (tid9) at (0.75,10.5){};
\draw[](tid8) -- (tid9);
\draw[](tid7) -- (tid8);
\draw[](tid5) -- (tid7);
\draw[](tid3) -- (tid5);
\draw[](tid1) -- (tid3);
\node[circle, scale=0.75, fill] (tid2) at (2.25,3){};
\node[circle, scale=0.75, fill] (tid4) at (2.25,4.5){};
\node[circle, scale=0.75, fill, red] (tid6) at (2.25,6){};
\draw[](tid4) -- (tid6);
\draw[](tid2) -- (tid4);
\draw[](tid0) -- (tid1);
\draw[](tid0) -- (tid2);
\end{tikzpicture}
\nodepart{three}
\footnotesize{7.22266}
\nodepart{four}
\footnotesize{$50\:50$}
};
 & 
\node[draw=black, rectangle split,  rectangle split parts=4] (sn0x9747bd8){
\footnotesize{22.2222}
\nodepart{two}
\begin{tikzpicture}[scale=.2]
\node[circle, scale=0.75, fill] (tid0) at (2.25,1.5){};
\node[circle, scale=0.75, fill] (tid1) at (1.5,3){};
\node[circle, scale=0.75, fill] (tid3) at (1.5,4.5){};
\node[circle, scale=0.75, fill] (tid5) at (1.5,6){};
\node[circle, scale=0.75, fill] (tid7) at (0.75,7.5){};
\node[circle, scale=0.75, fill, red] (tid9) at (0.75,9){};
\draw[](tid7) -- (tid9);
\node[circle, scale=0.75, fill, red] (tid8) at (2.25,7.5){};
\draw[](tid5) -- (tid7);
\draw[](tid5) -- (tid8);
\draw[](tid3) -- (tid5);
\draw[](tid1) -- (tid3);
\node[circle, scale=0.75, fill] (tid2) at (3.75,3){};
\node[circle, scale=0.75, fill] (tid4) at (3.75,4.5){};
\node[circle, scale=0.75, fill, red] (tid6) at (3.75,6){};
\draw[](tid4) -- (tid6);
\draw[](tid2) -- (tid4);
\draw[](tid0) -- (tid1);
\draw[](tid0) -- (tid2);
\end{tikzpicture}
\nodepart{three}
\footnotesize{6.56279}
\nodepart{four}
\footnotesize{$33\:33\:33$}
};
 & 
\\
};
\end{scope}
\draw (sn0x9748440.south) -- (sn0x97466f8.north);
\draw (sn0x9748440.south) -- (sn0x9748060.north);
\draw (sn0x97466f8.south) -- (sn0x9746d20.north);
\draw (sn0x97466f8.south) -- (sn0x9745bf0.north);
\draw (sn0x9748060.south) -- (sn0x9745bf0.north);
\draw (sn0x9748060.south) -- (sn0x9747af8.north);
\draw (sn0x9748060.south) -- (sn0x9747bd8.north);
\end{tikzpicture}
%%% Local Variables:
%%% TeX-master: "thesis/thesis.tex"
%%% End: 
  \caption{HLF vs. optimal solution for 00123455799 (taken from Chandy/Reynolds)}
  \label{fig:hlf-vs-opt-00123455799}
\end{figure}

%%% Local Variables:
%%% TeX-master: "../thesis.tex"
%%% End: 

\end{document}
