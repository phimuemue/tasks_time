\documentclass[a4paper, 11pt]{report}

\usepackage[english]{babel}
\usepackage{lmodern}
\usepackage[utf8]{inputenc}
\usepackage[
        % colorlinks,
        % linkcolor={black!50!blue},
        % citecolor={blue!70!green},
        % urlcolor={blue!70!red}
        ]
        {hyperref}
\usepackage{amsmath}
\usepackage{amssymb}
\usepackage{amsthm}
\usepackage{graphicx}
\usepackage{tikz}
\usepackage{algorithm}
\usepackage{algorithmicx}
\usepackage{algpseudocode}
\usepackage{chngcntr}
\usepackage{suffix}
\usepackage{caption}
\usepackage{subcaption}
\usepackage{multirow}
\usepackage{stmaryrd}
\usepackage{ifthen}
\usepackage{setspace}
\usepackage[left=2.5cm,right=2.5cm, bottom=3.5cm, top=3.5cm]{geometry}
\usepackage{diagbox}
%\usepackage{natbib}
%\usepackage{cite}
\usepackage[style=alphabetic,backend=biber]{biblatex}
\usepackage{varwidth}

\usetikzlibrary{shapes.multipart,chains}
\usetikzlibrary{positioning}
\usetikzlibrary{matrix}
\usetikzlibrary{automata}
\usetikzlibrary{external} 
\usetikzlibrary{decorations.pathreplacing} 
\usetikzlibrary{decorations.pathmorphing} 


%%% Local Variables:
%%% TeX-master: "thesis.tex"
%%% End: 
\newcommand{\todo}[2][]{\textcolor{red}{TODO\ifthenelse{\equal{#1}{}}{}{[#1]}: #2}}
\newcommand{\done}[2][]{\textcolor{green!50!black}{DONE\ifthenelse{\equal{#1}{}}{}{[#1]}: #2}}
\newcommand{\remark}[2][]{\textcolor{red!70!yellow}{REMARK\ifthenelse{\equal{#1}{}}{}{[#1]}: #2}}

\newtheorem{definition}{Definition}[chapter]
\newtheorem{theorem}[definition]{Theorem}
\newtheorem{lemma}[definition]{Lemma}
\newtheorem{corollary}[definition]{Corollary}
\newtheorem{conjecture}[definition]{Conjecture}
\newtheorem{proposition}[definition]{Proposition}
\counterwithin{algorithm}{chapter}

\newcommand{\E}[1]{\mathbb{E}\left[ #1 \right]}
\newcommand{\naturals}{\mathbb{N}}

\newcommand{\p}[1]{\operatorname{Pr}\left[#1\right]}
\newcommand{\alltasks}{{\mathbb T}}
\newcommand{\neededfor}{\rightarrow}
\WithSuffix\newcommand\neededfor*{\stackrel{*}{\rightarrow}}

\tikzstyle{task_cross}=[
    {path picture={ 
        \draw[black]
        (path picture bounding box.south east) -- 
        (path picture bounding box.north west) 
        (path picture bounding box.south west) -- 
        (path picture bounding box.north east);
      }
    }
]

\tikzstyle{task_scheduled}=[fill=white, draw=black, task_cross]

% \getwidthofnode will measure the width of the node given as its second
% parameter and store it into the first parameter.
\makeatletter
\newcommand\getwidthofnode[2]{%
    \pgfextractx{#1}{\pgfpointanchor{#2}{east}}%
    \pgfextractx{\pgf@xa}{\pgfpointanchor{#2}{west}}% \pgf@xa is a length defined by PGF for temporary storage. No need to create a new temporary length.
    \addtolength{#1}{-\pgf@xa}%
}
\makeatother

% profiles and stuff
\newcommand{\profile}[1]{\left\llbracket #1 \right\rrbracket}
%\newcommand{\profileones}[1]{\mathbb{1}^#1}
\newcommand{\profilerepeat}[2]{{(#1)}^{#2}}
\newcommand{\profileones}[1]{\profilerepeat{1}{#1}}

% stuff to draw diagrams levelwise
\newcommand{\leveltop}{0}
\newcommand{\leveltopI}{0}
\newcommand{\leveltopII}{0}
\newcommand{\leveltopIII}{0}
\newcommand{\leveltopIIII}{0}
\newcommand{\leveltopIIIII}{0}
\newcommand{\leveltopIIIIII}{0}
\newcommand{\leveltopIIIIIII}{0}
\newcommand{\leveltopIIIIIIII}{0}
\newcommand{\leveltopIIIIIIIII}{0}
\newcommand{\leveltopIIIIIIIIII}{0}
\newcommand{\leveltopIIIIIIIIIII}{0}
\newcommand{\leveltopIIIIIIIIIIII}{0}
\newcommand{\leveltopIIIIIIIIIIIII}{0}
\newcommand{\leveltopIIIIIIIIIIIIII}{0}
\newcommand{\leveltopIIIIIIIIIIIIIII}{0}
\newcommand{\leveltopIIIIIIIIIIIIIIII}{0}
\newcommand{\leveltopIIIIIIIIIIIIIIIII}{0}
\newcommand{\leveltopIIIIIIIIIIIIIIIIII}{0}
\newcommand{\leveltopIIIIIIIIIIIIIIIIIII}{0}
\newcommand{\leveltopIIIIIIIIIIIIIIIIIIII}{0}
\newcommand{\leveltopIIIIIIIIIIIIIIIIIIIII}{0}
\newcommand{\leveltopIIIIIIIIIIIIIIIIIIIIII}{0}
\newcommand{\leveltopIIIIIIIIIIIIIIIIIIIIIII}{0}
\newcommand{\leveltopIIIIIIIIIIIIIIIIIIIIIIII}{0}
\newcommand{\leveltopIIIIIIIIIIIIIIIIIIIIIIIII}{0}
\newcommand{\leveltopIIIIIIIIIIIIIIIIIIIIIIIIII}{0}
\newcommand{\leveltopIIIIIIIIIIIIIIIIIIIIIIIIIII}{0}
\newcommand{\leveltopIIIIIIIIIIIIIIIIIIIIIIIIIIII}{0}
\newcommand{\leveltopIIIIIIIIIIIIIIIIIIIIIIIIIIIII}{0}

\widowpenalty=10000
\clubpenalty=10000

%%% Local Variables:
%%% TeX-master: "thesis.tex"
%%% End: 

%\includeonly{p3/p3}

\bibliography{bib}
\addbibresource{bib.bib}

\begin{document}

\chapter{TODO-Lists}
\label{chap:todo}

\section{Current questions}
\label{chap:current-questions}

\begin{itemize}
\item If a tree $T$ is non-HLF-optimal, can there a be a supertree $S$ of this tree ($T\subseteq S$) such that $S$ is HLF-optimal?
\item Are chain collections optimally scheduled by HLF?
\item P3: Is it possible, that there are -- at one level of the optimal snapshot DAG -- multiple snapshots containing the same intree, but a different set of scheduled tasks? \remark{Probably not w.l.o.g.}
\item For each snapshot resulting from an optimal schedule, at least one top-most task is scheduled? \remark{Seems so.}
\item For each snapshot resulting from an optimal schedule, at least two top-most tasks are scheduled?\done{Not necessarily.}
\item If for an intree only non-top tasks are scheduled, you can exchange one of the non-top tasks with a top-tasks and obtain a better run time?\remark{Seems so}
\end{itemize}

\section{Other TODOs}

\begin{itemize}
\item Appendix containing all tree sequences for which optimal schedule is strictly non-HLF.
\item Improve program by using arrays instead of maps (only for canonical snapshots). \done{Works for canonical snapshots, ``normal ones'' broken.} \todo{Correct it!}
\end{itemize}

%\renewcommand{\leveltopI}{-15cm + \leveltop}
\renewcommand{\leveltopII}{-15cm + \leveltopI}
\renewcommand{\leveltopIII}{-15cm + \leveltopII}
\renewcommand{\leveltopIIII}{-15cm + \leveltopIII}
\renewcommand{\leveltopIIIII}{-15cm + \leveltopIIII}
\renewcommand{\leveltopIIIIII}{-15cm + \leveltopIIIII}
\renewcommand{\leveltopIIIIIII}{-15cm + \leveltopIIIIII}
\renewcommand{\leveltopIIIIIIII}{-15cm + \leveltopIIIIIII}
\renewcommand{\leveltopIIIIIIIII}{-15cm + \leveltopIIIIIIII}
\renewcommand{\leveltopIIIIIIIIII}{-15cm + \leveltopIIIIIIIII}
\renewcommand{\leveltopIIIIIIIIIII}{-15cm + \leveltopIIIIIIIIII}
\begin{tikzpicture}[scale=.2, anchor=south]
  \begin{scope}[yshift=\leveltopI cm]
    \matrix (line1) [column sep=1cm] {
      \node[draw=black, rectangle split,  rectangle split parts=3] (sn0x104f980){
        \begin{tikzpicture}[scale=.2]
          \node[circle, scale=0.75, fill] (tid0) at (3.75,1.5){};
          \node[circle, scale=0.75, fill] (tid1) at (2.25,3){};
          \node[circle, scale=0.75, fill] (tid3) at (0.75,4.5){};
          \node[circle, scale=0.75, fill] (tid7) at (0.75,6){};
          \draw[](tid3) -- (tid7);
          \node[circle, scale=0.75, fill] (tid4) at (2.25,4.5){};
          \node[circle, scale=0.75, fill] (tid5) at (3.75,4.5){};
          \draw[](tid1) -- (tid3);
          \draw[](tid1) -- (tid4);
          \draw[](tid1) -- (tid5);
          \node[circle, scale=0.75, fill] (tid2) at (6,3){};
          \node[circle, scale=0.75, fill] (tid6) at (6,4.5){};
          \node[circle, scale=0.75, fill] (tid8) at (5.25,6){};
          \node[circle, scale=0.75, fill, red] (tid10) at (5.25,7.5){};
          \draw[](tid8) -- (tid10);
          \node[circle, scale=0.75, fill, red] (tid9) at (6.75,6){};
          \draw[](tid6) -- (tid8);
          \draw[](tid6) -- (tid9);
          \draw[](tid2) -- (tid6);
          \draw[](tid0) -- (tid1);
          \draw[](tid0) -- (tid2);
        \end{tikzpicture}
        \nodepart{two}
        \footnotesize{6.82812}
        \nodepart{three}
        \footnotesize{$50\:25\:25$}
      };
      & 
      \\
    };
  \end{scope}
  \begin{scope}[yshift=\leveltopII cm]
    \matrix (line2) [column sep=1cm] {
      \node[draw=black, rectangle split,  rectangle split parts=3] (sn0x1050190){
        \begin{tikzpicture}[scale=.2]
          \node[circle, scale=0.75, fill] (tid0) at (3,1.5){};
          \node[circle, scale=0.75, fill] (tid1) at (2.25,3){};
          \node[circle, scale=0.75, fill] (tid3) at (0.75,4.5){};
          \node[circle, scale=0.75, fill, red] (tid7) at (0.75,6){};
          \draw[](tid3) -- (tid7);
          \node[circle, scale=0.75, fill] (tid4) at (2.25,4.5){};
          \node[circle, scale=0.75, fill] (tid5) at (3.75,4.5){};
          \draw[](tid1) -- (tid3);
          \draw[](tid1) -- (tid4);
          \draw[](tid1) -- (tid5);
          \node[circle, scale=0.75, fill] (tid2) at (5.25,3){};
          \node[circle, scale=0.75, fill] (tid6) at (5.25,4.5){};
          \node[circle, scale=0.75, fill] (tid8) at (5.25,6){};
          \node[circle, scale=0.75, fill, red] (tid9) at (5.25,7.5){};
          \draw[](tid8) -- (tid9);
          \draw[](tid6) -- (tid8);
          \draw[](tid2) -- (tid6);
          \draw[](tid0) -- (tid1);
          \draw[](tid0) -- (tid2);
        \end{tikzpicture}
        \nodepart{two}
        \footnotesize{6.35938}
        \nodepart{three}
        \footnotesize{$50\:50$}
      };
      & 
      \node[draw=black, rectangle split,  rectangle split parts=3] (sn0x104cb60){
        \begin{tikzpicture}[scale=.2]
          \node[circle, scale=0.75, fill] (tid0) at (3.75,1.5){};
          \node[circle, scale=0.75, fill] (tid1) at (2.25,3){};
          \node[circle, scale=0.75, fill] (tid3) at (0.75,4.5){};
          \node[circle, scale=0.75, fill, red] (tid7) at (0.75,6){};
          \draw[](tid3) -- (tid7);
          \node[circle, scale=0.75, fill] (tid4) at (2.25,4.5){};
          \node[circle, scale=0.75, fill] (tid5) at (3.75,4.5){};
          \draw[](tid1) -- (tid3);
          \draw[](tid1) -- (tid4);
          \draw[](tid1) -- (tid5);
          \node[circle, scale=0.75, fill] (tid2) at (6,3){};
          \node[circle, scale=0.75, fill] (tid6) at (6,4.5){};
          \node[circle, scale=0.75, fill, red] (tid8) at (5.25,6){};
          \node[circle, scale=0.75, fill] (tid9) at (6.75,6){};
          \draw[](tid6) -- (tid8);
          \draw[](tid6) -- (tid9);
          \draw[](tid2) -- (tid6);
          \draw[](tid0) -- (tid1);
          \draw[](tid0) -- (tid2);
        \end{tikzpicture}
        \nodepart{two}
        \footnotesize{6.29688}
        \nodepart{three}
        \footnotesize{$50\:50$}
      };
      & 
      \node[draw=black, rectangle split,  rectangle split parts=3] (sn0x104dd20){
        \begin{tikzpicture}[scale=.2]
          \node[circle, scale=0.75, fill] (tid0) at (3.75,1.5){};
          \node[circle, scale=0.75, fill] (tid1) at (2.25,3){};
          \node[circle, scale=0.75, fill] (tid3) at (0.75,4.5){};
          \node[circle, scale=0.75, fill] (tid7) at (0.75,6){};
          \draw[](tid3) -- (tid7);
          \node[circle, scale=0.75, fill] (tid4) at (2.25,4.5){};
          \node[circle, scale=0.75, fill] (tid5) at (3.75,4.5){};
          \draw[](tid1) -- (tid3);
          \draw[](tid1) -- (tid4);
          \draw[](tid1) -- (tid5);
          \node[circle, scale=0.75, fill] (tid2) at (6,3){};
          \node[circle, scale=0.75, fill] (tid6) at (6,4.5){};
          \node[circle, scale=0.75, fill, red] (tid8) at (5.25,6){};
          \node[circle, scale=0.75, fill, red] (tid9) at (6.75,6){};
          \draw[](tid6) -- (tid8);
          \draw[](tid6) -- (tid9);
          \draw[](tid2) -- (tid6);
          \draw[](tid0) -- (tid1);
          \draw[](tid0) -- (tid2);
        \end{tikzpicture}
        \nodepart{two}
        \footnotesize{6.29688}
        \nodepart{three}
        \footnotesize{$1$}
      };
      & 
      \\
    };
  \end{scope}
  \begin{scope}[yshift=\leveltopIII cm]
    \matrix (line3) [column sep=1cm] {
      \node[draw=black, rectangle split,  rectangle split parts=3] (sn0x10519d0){
        \begin{tikzpicture}[scale=.2]
          \node[circle, scale=0.75, fill] (tid0) at (3,1.5){};
          \node[circle, scale=0.75, fill] (tid1) at (2.25,3){};
          \node[circle, scale=0.75, fill, red] (tid3) at (0.75,4.5){};
          \node[circle, scale=0.75, fill] (tid4) at (2.25,4.5){};
          \node[circle, scale=0.75, fill] (tid5) at (3.75,4.5){};
          \draw[](tid1) -- (tid3);
          \draw[](tid1) -- (tid4);
          \draw[](tid1) -- (tid5);
          \node[circle, scale=0.75, fill] (tid2) at (5.25,3){};
          \node[circle, scale=0.75, fill] (tid6) at (5.25,4.5){};
          \node[circle, scale=0.75, fill] (tid7) at (5.25,6){};
          \node[circle, scale=0.75, fill, red] (tid8) at (5.25,7.5){};
          \draw[](tid7) -- (tid8);
          \draw[](tid6) -- (tid7);
          \draw[](tid2) -- (tid6);
          \draw[](tid0) -- (tid1);
          \draw[](tid0) -- (tid2);
        \end{tikzpicture}
        \nodepart{two}
        \footnotesize{5.92188}
        \nodepart{three}
        \footnotesize{$50\:50$}
      };
      & 
      \node[draw=black, rectangle split,  rectangle split parts=3] (sn0x104fbd0){
        \begin{tikzpicture}[scale=.2]
          \node[circle, scale=0.75, fill] (tid0) at (3,1.5){};
          \node[circle, scale=0.75, fill] (tid1) at (2.25,3){};
          \node[circle, scale=0.75, fill] (tid3) at (0.75,4.5){};
          \node[circle, scale=0.75, fill, red] (tid7) at (0.75,6){};
          \draw[](tid3) -- (tid7);
          \node[circle, scale=0.75, fill] (tid4) at (2.25,4.5){};
          \node[circle, scale=0.75, fill] (tid5) at (3.75,4.5){};
          \draw[](tid1) -- (tid3);
          \draw[](tid1) -- (tid4);
          \draw[](tid1) -- (tid5);
          \node[circle, scale=0.75, fill] (tid2) at (5.25,3){};
          \node[circle, scale=0.75, fill] (tid6) at (5.25,4.5){};
          \node[circle, scale=0.75, fill, red] (tid8) at (5.25,6){};
          \draw[](tid6) -- (tid8);
          \draw[](tid2) -- (tid6);
          \draw[](tid0) -- (tid1);
          \draw[](tid0) -- (tid2);
        \end{tikzpicture}
        \nodepart{two}
        \footnotesize{5.79688}
        \nodepart{three}
        \footnotesize{$50\:33\:17$}
      };
      & 
      \node[draw=black, rectangle split,  rectangle split parts=3] (sn0x105a080){
        \begin{tikzpicture}[scale=.2]
          \node[circle, scale=0.75, fill] (tid0) at (3.75,1.5){};
          \node[circle, scale=0.75, fill] (tid1) at (2.25,3){};
          \node[circle, scale=0.75, fill] (tid3) at (0.75,4.5){};
          \node[circle, scale=0.75, fill] (tid4) at (2.25,4.5){};
          \node[circle, scale=0.75, fill] (tid5) at (3.75,4.5){};
          \draw[](tid1) -- (tid3);
          \draw[](tid1) -- (tid4);
          \draw[](tid1) -- (tid5);
          \node[circle, scale=0.75, fill] (tid2) at (6,3){};
          \node[circle, scale=0.75, fill] (tid6) at (6,4.5){};
          \node[circle, scale=0.75, fill, red] (tid7) at (5.25,6){};
          \node[circle, scale=0.75, fill, red] (tid8) at (6.75,6){};
          \draw[](tid6) -- (tid7);
          \draw[](tid6) -- (tid8);
          \draw[](tid2) -- (tid6);
          \draw[](tid0) -- (tid1);
          \draw[](tid0) -- (tid2);
        \end{tikzpicture}
        \nodepart{two}
        \footnotesize{5.79688}
        \nodepart{three}
        \footnotesize{$1$}
      };
      & 
      \\
    };
  \end{scope}
  \begin{scope}[yshift=\leveltopIIII cm]
    \matrix (line4) [column sep=1cm] {
      \node[draw=black, rectangle split,  rectangle split parts=3] (sn0x1052250){
        \begin{tikzpicture}[scale=.2]
          \node[circle, scale=0.75, fill] (tid0) at (2.25,1.5){};
          \node[circle, scale=0.75, fill] (tid1) at (0.75,3){};
          \node[circle, scale=0.75, fill] (tid3) at (0.75,4.5){};
          \node[circle, scale=0.75, fill] (tid6) at (0.75,6){};
          \node[circle, scale=0.75, fill, red] (tid7) at (0.75,7.5){};
          \draw[](tid6) -- (tid7);
          \draw[](tid3) -- (tid6);
          \draw[](tid1) -- (tid3);
          \node[circle, scale=0.75, fill] (tid2) at (3,3){};
          \node[circle, scale=0.75, fill, red] (tid4) at (2.25,4.5){};
          \node[circle, scale=0.75, fill] (tid5) at (3.75,4.5){};
          \draw[](tid2) -- (tid4);
          \draw[](tid2) -- (tid5);
          \draw[](tid0) -- (tid1);
          \draw[](tid0) -- (tid2);
        \end{tikzpicture}
        \nodepart{two}
        \footnotesize{5.54688}
        \nodepart{three}
        \footnotesize{$50\:50$}
      };
      & 
      \node[draw=black, rectangle split,  rectangle split parts=3] (sn0x1052960){
        \begin{tikzpicture}[scale=.2]
          \node[circle, scale=0.75, fill] (tid0) at (3,1.5){};
          \node[circle, scale=0.75, fill] (tid1) at (2.25,3){};
          \node[circle, scale=0.75, fill, red] (tid3) at (0.75,4.5){};
          \node[circle, scale=0.75, fill] (tid4) at (2.25,4.5){};
          \node[circle, scale=0.75, fill] (tid5) at (3.75,4.5){};
          \draw[](tid1) -- (tid3);
          \draw[](tid1) -- (tid4);
          \draw[](tid1) -- (tid5);
          \node[circle, scale=0.75, fill] (tid2) at (5.25,3){};
          \node[circle, scale=0.75, fill] (tid6) at (5.25,4.5){};
          \node[circle, scale=0.75, fill, red] (tid7) at (5.25,6){};
          \draw[](tid6) -- (tid7);
          \draw[](tid2) -- (tid6);
          \draw[](tid0) -- (tid1);
          \draw[](tid0) -- (tid2);
        \end{tikzpicture}
        \nodepart{two}
        \footnotesize{5.29688}
        \nodepart{three}
        \footnotesize{$50\:33\:17$}
      };
      & 
      \node[draw=black, rectangle split,  rectangle split parts=3] (sn0x10581e0){
        \begin{tikzpicture}[scale=.2]
          \node[circle, scale=0.75, fill] (tid0) at (3,1.5){};
          \node[circle, scale=0.75, fill] (tid1) at (2.25,3){};
          \node[circle, scale=0.75, fill] (tid3) at (0.75,4.5){};
          \node[circle, scale=0.75, fill, red] (tid7) at (0.75,6){};
          \draw[](tid3) -- (tid7);
          \node[circle, scale=0.75, fill, red] (tid4) at (2.25,4.5){};
          \node[circle, scale=0.75, fill] (tid5) at (3.75,4.5){};
          \draw[](tid1) -- (tid3);
          \draw[](tid1) -- (tid4);
          \draw[](tid1) -- (tid5);
          \node[circle, scale=0.75, fill] (tid2) at (5.25,3){};
          \node[circle, scale=0.75, fill] (tid6) at (5.25,4.5){};
          \draw[](tid2) -- (tid6);
          \draw[](tid0) -- (tid1);
          \draw[](tid0) -- (tid2);
        \end{tikzpicture}
        \nodepart{two}
        \footnotesize{5.29688}
        \nodepart{three}
        \footnotesize{$33\:17\:25\:25$}
      };
      & 
      \node[draw=black, rectangle split,  rectangle split parts=3] (sn0x1058550){
        \begin{tikzpicture}[scale=.2]
          \node[circle, scale=0.75, fill] (tid0) at (3,1.5){};
          \node[circle, scale=0.75, fill] (tid1) at (2.25,3){};
          \node[circle, scale=0.75, fill] (tid3) at (0.75,4.5){};
          \node[circle, scale=0.75, fill, red] (tid7) at (0.75,6){};
          \draw[](tid3) -- (tid7);
          \node[circle, scale=0.75, fill] (tid4) at (2.25,4.5){};
          \node[circle, scale=0.75, fill] (tid5) at (3.75,4.5){};
          \draw[](tid1) -- (tid3);
          \draw[](tid1) -- (tid4);
          \draw[](tid1) -- (tid5);
          \node[circle, scale=0.75, fill] (tid2) at (5.25,3){};
          \node[circle, scale=0.75, fill, red] (tid6) at (5.25,4.5){};
          \draw[](tid2) -- (tid6);
          \draw[](tid0) -- (tid1);
          \draw[](tid0) -- (tid2);
        \end{tikzpicture}
        \nodepart{two}
        \footnotesize{5.29688}
        \nodepart{three}
        \footnotesize{$50\:50$}
      };
      & 
      \\
    };
  \end{scope}
  \begin{scope}[yshift=\leveltopIIIII cm]
    \matrix (line5) [column sep=1cm] {
      \node[draw=black, rectangle split,  rectangle split parts=3] (sn0x10525f0){
        \begin{tikzpicture}[scale=.2]
          \node[circle, scale=0.75, fill] (tid0) at (1.5,1.5){};
          \node[circle, scale=0.75, fill] (tid1) at (0.75,3){};
          \node[circle, scale=0.75, fill] (tid3) at (0.75,4.5){};
          \node[circle, scale=0.75, fill] (tid5) at (0.75,6){};
          \node[circle, scale=0.75, fill, red] (tid6) at (0.75,7.5){};
          \draw[](tid5) -- (tid6);
          \draw[](tid3) -- (tid5);
          \draw[](tid1) -- (tid3);
          \node[circle, scale=0.75, fill] (tid2) at (2.25,3){};
          \node[circle, scale=0.75, fill, red] (tid4) at (2.25,4.5){};
          \draw[](tid2) -- (tid4);
          \draw[](tid0) -- (tid1);
          \draw[](tid0) -- (tid2);
        \end{tikzpicture}
        \nodepart{two}
        \footnotesize{5.25}
        \nodepart{three}
        \footnotesize{$50\:50$}
      };
      & 
      \node[draw=black, rectangle split,  rectangle split parts=3] (sn0x1053850){
        \begin{tikzpicture}[scale=.2]
          \node[circle, scale=0.75, fill] (tid0) at (2.25,1.5){};
          \node[circle, scale=0.75, fill] (tid1) at (1.5,3){};
          \node[circle, scale=0.75, fill, red] (tid3) at (0.75,4.5){};
          \node[circle, scale=0.75, fill] (tid4) at (2.25,4.5){};
          \draw[](tid1) -- (tid3);
          \draw[](tid1) -- (tid4);
          \node[circle, scale=0.75, fill] (tid2) at (3.75,3){};
          \node[circle, scale=0.75, fill] (tid5) at (3.75,4.5){};
          \node[circle, scale=0.75, fill, red] (tid6) at (3.75,6){};
          \draw[](tid5) -- (tid6);
          \draw[](tid2) -- (tid5);
          \draw[](tid0) -- (tid1);
          \draw[](tid0) -- (tid2);
        \end{tikzpicture}
        \nodepart{two}
        \footnotesize{4.84375}
        \nodepart{three}
        \footnotesize{$50\:25\:25$}
      };
      & 
      \node[draw=black, rectangle split,  rectangle split parts=3] (sn0x1056b00){
        \begin{tikzpicture}[scale=.2]
          \node[circle, scale=0.75, fill] (tid0) at (3,1.5){};
          \node[circle, scale=0.75, fill] (tid1) at (2.25,3){};
          \node[circle, scale=0.75, fill, red] (tid3) at (0.75,4.5){};
          \node[circle, scale=0.75, fill, red] (tid4) at (2.25,4.5){};
          \node[circle, scale=0.75, fill] (tid5) at (3.75,4.5){};
          \draw[](tid1) -- (tid3);
          \draw[](tid1) -- (tid4);
          \draw[](tid1) -- (tid5);
          \node[circle, scale=0.75, fill] (tid2) at (5.25,3){};
          \node[circle, scale=0.75, fill] (tid6) at (5.25,4.5){};
          \draw[](tid2) -- (tid6);
          \draw[](tid0) -- (tid1);
          \draw[](tid0) -- (tid2);
        \end{tikzpicture}
        \nodepart{two}
        \footnotesize{4.75}
        \nodepart{three}
        \footnotesize{$50\:50$}
      };
      & 
      \node[draw=black, rectangle split,  rectangle split parts=3] (sn0x1056fb0){
        \begin{tikzpicture}[scale=.2]
          \node[circle, scale=0.75, fill] (tid0) at (3,1.5){};
          \node[circle, scale=0.75, fill] (tid1) at (2.25,3){};
          \node[circle, scale=0.75, fill, red] (tid3) at (0.75,4.5){};
          \node[circle, scale=0.75, fill] (tid4) at (2.25,4.5){};
          \node[circle, scale=0.75, fill] (tid5) at (3.75,4.5){};
          \draw[](tid1) -- (tid3);
          \draw[](tid1) -- (tid4);
          \draw[](tid1) -- (tid5);
          \node[circle, scale=0.75, fill] (tid2) at (5.25,3){};
          \node[circle, scale=0.75, fill, red] (tid6) at (5.25,4.5){};
          \draw[](tid2) -- (tid6);
          \draw[](tid0) -- (tid1);
          \draw[](tid0) -- (tid2);
        \end{tikzpicture}
        \nodepart{two}
        \footnotesize{4.75}
        \nodepart{three}
        \footnotesize{$50\:50$}
      };
      & 
      \node[draw=black, rectangle split,  rectangle split parts=3] (sn0x1058f50){
        \begin{tikzpicture}[scale=.2]
          \node[circle, scale=0.75, fill] (tid0) at (2.25,1.5){};
          \node[circle, scale=0.75, fill] (tid1) at (1.5,3){};
          \node[circle, scale=0.75, fill] (tid3) at (0.75,4.5){};
          \node[circle, scale=0.75, fill, red] (tid6) at (0.75,6){};
          \draw[](tid3) -- (tid6);
          \node[circle, scale=0.75, fill, red] (tid4) at (2.25,4.5){};
          \draw[](tid1) -- (tid3);
          \draw[](tid1) -- (tid4);
          \node[circle, scale=0.75, fill] (tid2) at (3.75,3){};
          \node[circle, scale=0.75, fill] (tid5) at (3.75,4.5){};
          \draw[](tid2) -- (tid5);
          \draw[](tid0) -- (tid1);
          \draw[](tid0) -- (tid2);
        \end{tikzpicture}
        \nodepart{two}
        \footnotesize{4.84375}
        \nodepart{three}
        \footnotesize{$50\:25\:25$}
      };
      & 
      \node[draw=black, rectangle split,  rectangle split parts=3] (sn0x1058a50){
        \begin{tikzpicture}[scale=.2]
          \node[circle, scale=0.75, fill] (tid0) at (2.25,1.5){};
          \node[circle, scale=0.75, fill] (tid1) at (1.5,3){};
          \node[circle, scale=0.75, fill] (tid3) at (0.75,4.5){};
          \node[circle, scale=0.75, fill, red] (tid6) at (0.75,6){};
          \draw[](tid3) -- (tid6);
          \node[circle, scale=0.75, fill] (tid4) at (2.25,4.5){};
          \draw[](tid1) -- (tid3);
          \draw[](tid1) -- (tid4);
          \node[circle, scale=0.75, fill] (tid2) at (3.75,3){};
          \node[circle, scale=0.75, fill, red] (tid5) at (3.75,4.5){};
          \draw[](tid2) -- (tid5);
          \draw[](tid0) -- (tid1);
          \draw[](tid0) -- (tid2);
        \end{tikzpicture}
        \nodepart{two}
        \footnotesize{4.84375}
        \nodepart{three}
        \footnotesize{$50\:50$}
      };
      & 
      \node[draw=black, rectangle split,  rectangle split parts=3] (sn0x10597a0){
        \begin{tikzpicture}[scale=.2]
          \node[circle, scale=0.75, fill] (tid0) at (3,1.5){};
          \node[circle, scale=0.75, fill] (tid1) at (2.25,3){};
          \node[circle, scale=0.75, fill] (tid3) at (0.75,4.5){};
          \node[circle, scale=0.75, fill, red] (tid6) at (0.75,6){};
          \draw[](tid3) -- (tid6);
          \node[circle, scale=0.75, fill, red] (tid4) at (2.25,4.5){};
          \node[circle, scale=0.75, fill] (tid5) at (3.75,4.5){};
          \draw[](tid1) -- (tid3);
          \draw[](tid1) -- (tid4);
          \draw[](tid1) -- (tid5);
          \node[circle, scale=0.75, fill] (tid2) at (5.25,3){};
          \draw[](tid0) -- (tid1);
          \draw[](tid0) -- (tid2);
        \end{tikzpicture}
        \nodepart{two}
        \footnotesize{4.84375}
        \nodepart{three}
        \footnotesize{$50\:50$}
      };
      & 
      \\
    };
  \end{scope}
  \begin{scope}[yshift=\leveltopIIIIII cm]
    \matrix (line6) [column sep=1cm] {
      \node[draw=black, rectangle split,  rectangle split parts=3] (sn0x1053920){
        \begin{tikzpicture}[scale=.2]
          \node[circle, scale=0.75, fill] (tid0) at (1.5,1.5){};
          \node[circle, scale=0.75, fill] (tid1) at (0.75,3){};
          \node[circle, scale=0.75, fill] (tid3) at (0.75,4.5){};
          \node[circle, scale=0.75, fill] (tid4) at (0.75,6){};
          \node[circle, scale=0.75, fill, red] (tid5) at (0.75,7.5){};
          \draw[](tid4) -- (tid5);
          \draw[](tid3) -- (tid4);
          \draw[](tid1) -- (tid3);
          \node[circle, scale=0.75, fill, red] (tid2) at (2.25,3){};
          \draw[](tid0) -- (tid1);
          \draw[](tid0) -- (tid2);
        \end{tikzpicture}
        \nodepart{two}
        \footnotesize{5.0625}
        \nodepart{three}
        \footnotesize{$50\:50$}
      };
      & 
      \node[draw=black, rectangle split,  rectangle split parts=3] (sn0x1053bc0){
        \begin{tikzpicture}[scale=.2]
          \node[circle, scale=0.75, fill] (tid0) at (1.5,1.5){};
          \node[circle, scale=0.75, fill] (tid1) at (0.75,3){};
          \node[circle, scale=0.75, fill] (tid3) at (0.75,4.5){};
          \node[circle, scale=0.75, fill, red] (tid5) at (0.75,6){};
          \draw[](tid3) -- (tid5);
          \draw[](tid1) -- (tid3);
          \node[circle, scale=0.75, fill] (tid2) at (2.25,3){};
          \node[circle, scale=0.75, fill, red] (tid4) at (2.25,4.5){};
          \draw[](tid2) -- (tid4);
          \draw[](tid0) -- (tid1);
          \draw[](tid0) -- (tid2);
        \end{tikzpicture}
        \nodepart{two}
        \footnotesize{4.4375}
        \nodepart{three}
        \footnotesize{$50\:50$}
      };
      & 
      \node[draw=black, rectangle split,  rectangle split parts=3] (sn0x1056090){
        \begin{tikzpicture}[scale=.2]
          \node[circle, scale=0.75, fill] (tid0) at (2.25,1.5){};
          \node[circle, scale=0.75, fill] (tid1) at (1.5,3){};
          \node[circle, scale=0.75, fill, red] (tid3) at (0.75,4.5){};
          \node[circle, scale=0.75, fill, red] (tid4) at (2.25,4.5){};
          \draw[](tid1) -- (tid3);
          \draw[](tid1) -- (tid4);
          \node[circle, scale=0.75, fill] (tid2) at (3.75,3){};
          \node[circle, scale=0.75, fill] (tid5) at (3.75,4.5){};
          \draw[](tid2) -- (tid5);
          \draw[](tid0) -- (tid1);
          \draw[](tid0) -- (tid2);
        \end{tikzpicture}
        \nodepart{two}
        \footnotesize{4.25}
        \nodepart{three}
        \footnotesize{$1$}
      };
      & 
      \node[draw=black, rectangle split,  rectangle split parts=3] (sn0x1056160){
        \begin{tikzpicture}[scale=.2]
          \node[circle, scale=0.75, fill] (tid0) at (2.25,1.5){};
          \node[circle, scale=0.75, fill] (tid1) at (1.5,3){};
          \node[circle, scale=0.75, fill, red] (tid3) at (0.75,4.5){};
          \node[circle, scale=0.75, fill] (tid4) at (2.25,4.5){};
          \draw[](tid1) -- (tid3);
          \draw[](tid1) -- (tid4);
          \node[circle, scale=0.75, fill] (tid2) at (3.75,3){};
          \node[circle, scale=0.75, fill, red] (tid5) at (3.75,4.5){};
          \draw[](tid2) -- (tid5);
          \draw[](tid0) -- (tid1);
          \draw[](tid0) -- (tid2);
        \end{tikzpicture}
        \nodepart{two}
        \footnotesize{4.25}
        \nodepart{three}
        \footnotesize{$50\:50$}
      };
      & 
      \node[draw=black, rectangle split,  rectangle split parts=3] (sn0x1057630){
        \begin{tikzpicture}[scale=.2]
          \node[circle, scale=0.75, fill] (tid0) at (3,1.5){};
          \node[circle, scale=0.75, fill] (tid1) at (2.25,3){};
          \node[circle, scale=0.75, fill, red] (tid3) at (0.75,4.5){};
          \node[circle, scale=0.75, fill, red] (tid4) at (2.25,4.5){};
          \node[circle, scale=0.75, fill] (tid5) at (3.75,4.5){};
          \draw[](tid1) -- (tid3);
          \draw[](tid1) -- (tid4);
          \draw[](tid1) -- (tid5);
          \node[circle, scale=0.75, fill] (tid2) at (5.25,3){};
          \draw[](tid0) -- (tid1);
          \draw[](tid0) -- (tid2);
        \end{tikzpicture}
        \nodepart{two}
        \footnotesize{4.25}
        \nodepart{three}
        \footnotesize{$1$}
      };
      & 
      \node[draw=black, rectangle split,  rectangle split parts=3] (sn0x1058b20){
        \begin{tikzpicture}[scale=.2]
          \node[circle, scale=0.75, fill] (tid0) at (2.25,1.5){};
          \node[circle, scale=0.75, fill] (tid1) at (1.5,3){};
          \node[circle, scale=0.75, fill] (tid3) at (0.75,4.5){};
          \node[circle, scale=0.75, fill, red] (tid5) at (0.75,6){};
          \draw[](tid3) -- (tid5);
          \node[circle, scale=0.75, fill, red] (tid4) at (2.25,4.5){};
          \draw[](tid1) -- (tid3);
          \draw[](tid1) -- (tid4);
          \node[circle, scale=0.75, fill] (tid2) at (3.75,3){};
          \draw[](tid0) -- (tid1);
          \draw[](tid0) -- (tid2);
        \end{tikzpicture}
        \nodepart{two}
        \footnotesize{4.4375}
        \nodepart{three}
        \footnotesize{$50\:50$}
      };
      & 
      \\
    };
  \end{scope}
  \begin{scope}[yshift=\leveltopIIIIIII cm]
    \matrix (line7) [column sep=1cm] {
      \node[draw=black, rectangle split,  rectangle split parts=3] (sn0x10540d0){
        \begin{tikzpicture}[scale=.2]
          \node[circle, scale=0.75, fill] (tid0) at (0.75,1.5){};
          \node[circle, scale=0.75, fill] (tid1) at (0.75,3){};
          \node[circle, scale=0.75, fill] (tid2) at (0.75,4.5){};
          \node[circle, scale=0.75, fill] (tid3) at (0.75,6){};
          \node[circle, scale=0.75, fill, red] (tid4) at (0.75,7.5){};
          \draw[](tid3) -- (tid4);
          \draw[](tid2) -- (tid3);
          \draw[](tid1) -- (tid2);
          \draw[](tid0) -- (tid1);
        \end{tikzpicture}
        \nodepart{two}
        \footnotesize{5}
        \nodepart{three}
        \footnotesize{$1$}
      };
      & 
      \node[draw=black, rectangle split,  rectangle split parts=3] (sn0x1054480){
        \begin{tikzpicture}[scale=.2]
          \node[circle, scale=0.75, fill] (tid0) at (1.5,1.5){};
          \node[circle, scale=0.75, fill] (tid1) at (0.75,3){};
          \node[circle, scale=0.75, fill] (tid3) at (0.75,4.5){};
          \node[circle, scale=0.75, fill, red] (tid4) at (0.75,6){};
          \draw[](tid3) -- (tid4);
          \draw[](tid1) -- (tid3);
          \node[circle, scale=0.75, fill, red] (tid2) at (2.25,3){};
          \draw[](tid0) -- (tid1);
          \draw[](tid0) -- (tid2);
        \end{tikzpicture}
        \nodepart{two}
        \footnotesize{4.125}
        \nodepart{three}
        \footnotesize{$50\:50$}
      };
      & 
      \node[draw=black, rectangle split,  rectangle split parts=3] (sn0x1055dd0){
        \begin{tikzpicture}[scale=.2]
          \node[circle, scale=0.75, fill] (tid0) at (1.5,1.5){};
          \node[circle, scale=0.75, fill] (tid1) at (0.75,3){};
          \node[circle, scale=0.75, fill, red] (tid3) at (0.75,4.5){};
          \draw[](tid1) -- (tid3);
          \node[circle, scale=0.75, fill] (tid2) at (2.25,3){};
          \node[circle, scale=0.75, fill, red] (tid4) at (2.25,4.5){};
          \draw[](tid2) -- (tid4);
          \draw[](tid0) -- (tid1);
          \draw[](tid0) -- (tid2);
        \end{tikzpicture}
        \nodepart{two}
        \footnotesize{3.75}
        \nodepart{three}
        \footnotesize{$1$}
      };
      & 
      \node[draw=black, rectangle split,  rectangle split parts=3] (sn0x10568c0){
        \begin{tikzpicture}[scale=.2]
          \node[circle, scale=0.75, fill] (tid0) at (2.25,1.5){};
          \node[circle, scale=0.75, fill] (tid1) at (1.5,3){};
          \node[circle, scale=0.75, fill, red] (tid3) at (0.75,4.5){};
          \node[circle, scale=0.75, fill, red] (tid4) at (2.25,4.5){};
          \draw[](tid1) -- (tid3);
          \draw[](tid1) -- (tid4);
          \node[circle, scale=0.75, fill] (tid2) at (3.75,3){};
          \draw[](tid0) -- (tid1);
          \draw[](tid0) -- (tid2);
        \end{tikzpicture}
        \nodepart{two}
        \footnotesize{3.75}
        \nodepart{three}
        \footnotesize{$1$}
      };
      & 
      \\
    };
  \end{scope}
  \begin{scope}[yshift=\leveltopIIIIIIII cm]
    \matrix (line8) [column sep=1cm] {
      \node[draw=black, rectangle split,  rectangle split parts=3] (sn0x1054550){
        \begin{tikzpicture}[scale=.2]
          \node[circle, scale=0.75, fill] (tid0) at (0.75,1.5){};
          \node[circle, scale=0.75, fill] (tid1) at (0.75,3){};
          \node[circle, scale=0.75, fill] (tid2) at (0.75,4.5){};
          \node[circle, scale=0.75, fill, red] (tid3) at (0.75,6){};
          \draw[](tid2) -- (tid3);
          \draw[](tid1) -- (tid2);
          \draw[](tid0) -- (tid1);
        \end{tikzpicture}
        \nodepart{two}
        \footnotesize{4}
        \nodepart{three}
        \footnotesize{$1$}
      };
      & 
      \node[draw=black, rectangle split,  rectangle split parts=3] (sn0x1055270){
        \begin{tikzpicture}[scale=.2]
          \node[circle, scale=0.75, fill] (tid0) at (1.5,1.5){};
          \node[circle, scale=0.75, fill] (tid1) at (0.75,3){};
          \node[circle, scale=0.75, fill, red] (tid3) at (0.75,4.5){};
          \draw[](tid1) -- (tid3);
          \node[circle, scale=0.75, fill, red] (tid2) at (2.25,3){};
          \draw[](tid0) -- (tid1);
          \draw[](tid0) -- (tid2);
        \end{tikzpicture}
        \nodepart{two}
        \footnotesize{3.25}
        \nodepart{three}
        \footnotesize{$50\:50$}
      };
      & 
      \\
    };
  \end{scope}
  \begin{scope}[yshift=\leveltopIIIIIIIII cm]
    \matrix (line9) [column sep=1cm] {
      \node[draw=black, rectangle split,  rectangle split parts=3] (sn0x1054a50){
        \begin{tikzpicture}[scale=.2]
          \node[circle, scale=0.75, fill] (tid0) at (0.75,1.5){};
          \node[circle, scale=0.75, fill] (tid1) at (0.75,3){};
          \node[circle, scale=0.75, fill, red] (tid2) at (0.75,4.5){};
          \draw[](tid1) -- (tid2);
          \draw[](tid0) -- (tid1);
        \end{tikzpicture}
        \nodepart{two}
        \footnotesize{3}
        \nodepart{three}
        \footnotesize{$1$}
      };
      & 
      \node[draw=black, rectangle split,  rectangle split parts=3] (sn0x1054cb0){
        \begin{tikzpicture}[scale=.2]
          \node[circle, scale=0.75, fill] (tid0) at (1.5,1.5){};
          \node[circle, scale=0.75, fill, red] (tid1) at (0.75,3){};
          \node[circle, scale=0.75, fill, red] (tid2) at (2.25,3){};
          \draw[](tid0) -- (tid1);
          \draw[](tid0) -- (tid2);
        \end{tikzpicture}
        \nodepart{two}
        \footnotesize{2.5}
        \nodepart{three}
        \footnotesize{$1$}
      };
      & 
      \\
    };
  \end{scope}
  \begin{scope}[yshift=\leveltopIIIIIIIIII cm]
    \matrix (line10) [column sep=1cm] {
      \node[draw=black, rectangle split,  rectangle split parts=3] (sn0x1054b20){
        \begin{tikzpicture}[scale=.2]
          \node[circle, scale=0.75, fill] (tid0) at (0.75,1.5){};
          \node[circle, scale=0.75, fill, red] (tid1) at (0.75,3){};
          \draw[](tid0) -- (tid1);
        \end{tikzpicture}
        \nodepart{two}
        \footnotesize{2}
        \nodepart{three}
        \footnotesize{$1$}
      };
      & 
      \\
    };
  \end{scope}
  \begin{scope}[yshift=\leveltopIIIIIIIIIII cm]
    \matrix (line11) [column sep=1cm] {
      \node[draw=black, rectangle split,  rectangle split parts=3] (sn0x10547e0){
        \begin{tikzpicture}[scale=.2]
          \node[circle, scale=0.75, fill, red] (tid0) at (0.75,1.5){};
        \end{tikzpicture}
        \nodepart{two}
        \footnotesize{1}
        \nodepart{three}
        \footnotesize{$$}
      };
      & 
      \\
    };
  \end{scope}
  \begin{scope}[yshift=\leveltopIIIIIIIIIIII cm]
    \matrix (line12) [column sep=1cm] {
      \\
    };
  \end{scope}
  \draw (sn0x104f980.south) -- (sn0x1050190.north);
  \draw (sn0x104f980.south) -- (sn0x104cb60.north);
  \draw (sn0x104f980.south) -- (sn0x104dd20.north);
  \draw (sn0x1050190.south) -- (sn0x10519d0.north);
  \draw (sn0x1050190.south) -- (sn0x104fbd0.north);
  \draw (sn0x104cb60.south) -- (sn0x105a080.north);
  \draw (sn0x104cb60.south) -- (sn0x104fbd0.north);
  \draw (sn0x104dd20.south) -- (sn0x104fbd0.north);
  \draw (sn0x10519d0.south) -- (sn0x1052250.north);
  \draw (sn0x10519d0.south) -- (sn0x1052960.north);
  \draw (sn0x104fbd0.south) -- (sn0x1052960.north);
  \draw (sn0x104fbd0.south) -- (sn0x10581e0.north);
  \draw (sn0x104fbd0.south) -- (sn0x1058550.north);
  \draw (sn0x105a080.south) -- (sn0x1052960.north);
  \draw (sn0x1052250.south) -- (sn0x10525f0.north);
  \draw (sn0x1052250.south) -- (sn0x1053850.north);
  \draw (sn0x1052960.south) -- (sn0x1053850.north);
  \draw (sn0x1052960.south) -- (sn0x1056b00.north);
  \draw (sn0x1052960.south) -- (sn0x1056fb0.north);
  \draw (sn0x10581e0.south) -- (sn0x1058f50.north);
  \draw (sn0x10581e0.south) -- (sn0x1058a50.north);
  \draw (sn0x10581e0.south) -- (sn0x1056b00.north);
  \draw (sn0x10581e0.south) -- (sn0x1056fb0.north);
  \draw (sn0x1058550.south) -- (sn0x10597a0.north);
  \draw (sn0x1058550.south) -- (sn0x1056fb0.north);
  \draw (sn0x10525f0.south) -- (sn0x1053920.north);
  \draw (sn0x10525f0.south) -- (sn0x1053bc0.north);
  \draw (sn0x1053850.south) -- (sn0x1053bc0.north);
  \draw (sn0x1053850.south) -- (sn0x1056090.north);
  \draw (sn0x1053850.south) -- (sn0x1056160.north);
  \draw (sn0x1056b00.south) -- (sn0x1056090.north);
  \draw (sn0x1056b00.south) -- (sn0x1056160.north);
  \draw (sn0x1056fb0.south) -- (sn0x1056160.north);
  \draw (sn0x1056fb0.south) -- (sn0x1057630.north);
  \draw (sn0x1058f50.south) -- (sn0x1053bc0.north);
  \draw (sn0x1058f50.south) -- (sn0x1056090.north);
  \draw (sn0x1058f50.south) -- (sn0x1056160.north);
  \draw (sn0x1058a50.south) -- (sn0x1058b20.north);
  \draw (sn0x1058a50.south) -- (sn0x1056160.north);
  \draw (sn0x10597a0.south) -- (sn0x1058b20.north);
  \draw (sn0x10597a0.south) -- (sn0x1057630.north);
  \draw (sn0x1053920.south) -- (sn0x10540d0.north);
  \draw (sn0x1053920.south) -- (sn0x1054480.north);
  \draw (sn0x1053bc0.south) -- (sn0x1054480.north);
  \draw (sn0x1053bc0.south) -- (sn0x1055dd0.north);
  \draw (sn0x1056090.south) -- (sn0x1055dd0.north);
  \draw (sn0x1056160.south) -- (sn0x1055dd0.north);
  \draw (sn0x1056160.south) -- (sn0x10568c0.north);
  \draw (sn0x1057630.south) -- (sn0x10568c0.north);
  \draw (sn0x1058b20.south) -- (sn0x1054480.north);
  \draw (sn0x1058b20.south) -- (sn0x10568c0.north);
  \draw (sn0x10540d0.south) -- (sn0x1054550.north);
  \draw (sn0x1054480.south) -- (sn0x1054550.north);
  \draw (sn0x1054480.south) -- (sn0x1055270.north);
  \draw (sn0x1055dd0.south) -- (sn0x1055270.north);
  \draw (sn0x10568c0.south) -- (sn0x1055270.north);
  \draw (sn0x1054550.south) -- (sn0x1054a50.north);
  \draw (sn0x1055270.south) -- (sn0x1054a50.north);
  \draw (sn0x1055270.south) -- (sn0x1054cb0.north);
  \draw (sn0x1054a50.south) -- (sn0x1054b20.north);
  \draw (sn0x1054cb0.south) -- (sn0x1054b20.north);
  \draw (sn0x1054b20.south) -- (sn0x10547e0.north);

  \newcommand{\nd}[4]{
    \node[draw=black, rectangle split, rectangle split parts=3] (n#1#2) {
      $#1/#2$
      \nodepart{two}
      #3
      \nodepart{three}
      #4
    };
  }

  \begin{scope}[yshift=\leveltopI, xshift=70cm, rectangle, draw=black,anchor=south]
    \matrix (test) [column sep=1cm] {
      \nd{5}{6}{6.82812}{50 50};
      \\
    };
  \end{scope}

  \begin{scope}[yshift=\leveltopII, xshift=70cm, rectangle, draw=black,anchor=south]
    \matrix (test) [column sep=1cm] {
      \nd{5}{5}{6.35938}{50 50};
      &
      \nd{4}{6}{6.29688}{1};
      \\
      };
    \end{scope}

    \begin{scope}[yshift=\leveltopIII, xshift=70cm, rectangle, draw=black,anchor=south]
      \matrix (test) [column sep=1cm] {
        \nd{5}{4}{5.92188}{50 50};
        &
        \nd{4}{5}{5.79688}{1};
        \\
      };
    \end{scope}

    \begin{scope}[yshift=\leveltopIIII, xshift=70cm, rectangle, draw=black,anchor=south]
      \matrix (test) [column sep=1cm] {
        \nd{5}{3}{5.54688}{50 50};
        &
        \nd{4}{4}{5.29688}{50 50};
        \\
      };
    \end{scope}

    \begin{scope}[yshift=\leveltopIIIII, xshift=70cm, rectangle, draw=black,anchor=south]
      \matrix (test) [column sep=1cm] {
        \nd{5}{2}{5.25}{50 50};
        &
        \nd{4}{3}{4.84375}{50 50};
        &
        \nd{3}{4}{5.29688}{1};
        \\
      };
    \end{scope}

    \begin{scope}[yshift=\leveltopIIIIII, xshift=70cm, rectangle, draw=black,anchor=south]
      \matrix (test) [column sep=1cm] {
        \nd{5}{1}{5.0625}{50 50};
        &
        \nd{4}{2}{4.4375}{50 50};
        &
        \nd{3}{3}{5.75}{1};
        \\
      };
    \end{scope}

    \begin{scope}[yshift=\leveltopIIIIIII, xshift=70cm, rectangle, draw=black,anchor=south]
      \matrix (test) [column sep=1cm] {
        \nd{5}{0}{5}{50 50};
        &
        \nd{4}{1}{4.125}{50 50};
        &
        \nd{3}{2}{5.25}{1};
        \\
      };
    \end{scope}
    
    \begin{scope}[yshift=\leveltopIIIIIIII, xshift=70cm, rectangle, draw=black,anchor=south]
      \matrix (test) [column sep=1cm] {
        \nd{4}{0}{4}{1};
        &
        \nd{3}{1}{3.25}{50 50};
        \\
      };
    \end{scope}

    \begin{scope}[yshift=\leveltopIIIIIIIII, xshift=70cm, rectangle, draw=black,anchor=south]
      \matrix (test) [column sep=1cm] {
        \nd{3}{0}{3}{1};
        &
        \nd{2}{1}{2.5}{1};
        \\
      };
    \end{scope}

    \begin{scope}[yshift=\leveltopIIIIIIIIII, xshift=70cm, rectangle, draw=black,anchor=south]
      \matrix (test) [column sep=1cm] {
        \nd{2}{0}{2}{1};
        \\
      };
    \end{scope}
    
    \begin{scope}[yshift=\leveltopIIIIIIIIIII, xshift=70cm, rectangle, draw=black,anchor=south]
      \matrix (test) [column sep=1cm] {
        \nd{1}{0}{1}{1};
        \\
      };
    \end{scope}

    \draw (n56.south) -- (n55.north);
    \draw (n56.south) -- (n46.north);
    \draw (n55.south) -- (n54.north);
    \draw (n55.south) -- (n45.north);
    \draw (n46.south) -- (n45.north);
    \draw (n54.south) -- (n53.north);
    \draw (n54.south) -- (n44.north);
    \draw (n45.south) -- (n44.north);
    \draw (n53.south) -- (n52.north);
    \draw (n53.south) -- (n43.north);
    \draw (n44.south) -- (n43.north);
    \draw (n44.south) -- (n34.north);
    \draw (n52.south) -- (n51.north);
    \draw (n52.south) -- (n42.north);
    \draw (n43.south) -- (n42.north);
    \draw (n43.south) -- (n33.north);
    \draw (n34.south) -- (n33.north);
    \draw (n51.south) -- (n50.north);
    \draw (n51.south) -- (n41.north);
    \draw (n42.south) -- (n41.north);
    \draw (n42.south) -- (n32.north);
    \draw (n33.south) -- (n32.north);
    \draw (n32.south) -- (n31.north);
    \draw (n50.south) -- (n40.north);
    \draw (n41.south) -- (n40.north);
    \draw (n41.south) -- (n31.north);
    \draw (n40.south) -- (n30.north);
    \draw (n31.south) -- (n30.north);
    \draw (n31.south) -- (n21.north);
    \draw (n30.south) -- (n20.north);
    \draw (n21.south) -- (n20.north);
    \draw (n20.south) -- (n10.north);

\end{tikzpicture}

%%% Local Variables:
%%% TeX-master: "thesis/thesis.tex"
%%% End: 
\renewcommand{\leveltopI}{-15cm + \leveltop}
\renewcommand{\leveltopII}{-15cm + \leveltopI}
\renewcommand{\leveltopIII}{-15cm + \leveltopII}
\renewcommand{\leveltopIIII}{-15cm + \leveltopIII}
\renewcommand{\leveltopIIIII}{-15cm + \leveltopIIII}
\renewcommand{\leveltopIIIIII}{-15cm + \leveltopIIIII}
\renewcommand{\leveltopIIIIIII}{-15cm + \leveltopIIIIII}
\renewcommand{\leveltopIIIIIIII}{-15cm + \leveltopIIIIIII}
\renewcommand{\leveltopIIIIIIIII}{-15cm + \leveltopIIIIIIII}
\renewcommand{\leveltopIIIIIIIIII}{-15cm + \leveltopIIIIIIIII}
\renewcommand{\leveltopIIIIIIIIIII}{-15cm + \leveltopIIIIIIIIII}
% \begin{tikzpicture}[scale=.2, anchor=south]
%   \begin{scope}[yshift=\leveltopI cm]
%     \matrix (line1) [column sep=1cm] {
%       \node[draw=black, rectangle split,  rectangle split parts=3] (sn0x1050af0){
%         \begin{tikzpicture}[scale=.2]
%           \node[circle, scale=0.75, fill] (tid0) at (3.75,1.5){};
%           \node[circle, scale=0.75, fill] (tid1) at (2.25,3){};
%           \node[circle, scale=0.75, fill] (tid3) at (0.75,4.5){};
%           \node[circle, scale=0.75, fill, red] (tid7) at (0.75,6){};
%           \draw[](tid3) -- (tid7);
%           \node[circle, scale=0.75, fill] (tid4) at (2.25,4.5){};
%           \node[circle, scale=0.75, fill] (tid5) at (3.75,4.5){};
%           \draw[](tid1) -- (tid3);
%           \draw[](tid1) -- (tid4);
%           \draw[](tid1) -- (tid5);
%           \node[circle, scale=0.75, fill] (tid2) at (6,3){};
%           \node[circle, scale=0.75, fill] (tid6) at (6,4.5){};
%           \node[circle, scale=0.75, fill] (tid8) at (5.25,6){};
%           \node[circle, scale=0.75, fill, red] (tid10) at (5.25,7.5){};
%           \draw[](tid8) -- (tid10);
%           \node[circle, scale=0.75, fill] (tid9) at (6.75,6){};
%           \draw[](tid6) -- (tid8);
%           \draw[](tid6) -- (tid9);
%           \draw[](tid2) -- (tid6);
%           \draw[](tid0) -- (tid1);
%           \draw[](tid0) -- (tid2);
%         \end{tikzpicture}
%         \nodepart{two}
%         \footnotesize{6.82812}
%         \nodepart{three}
%         \footnotesize{$50\:50$}
%       };
%       & 
%       \\
%     };
%   \end{scope}
%   \begin{scope}[yshift=\leveltopII cm]
%     \matrix (line2) [column sep=1cm] {
%       \node[draw=black, rectangle split,  rectangle split parts=3] (sn0x105a150){
%         \begin{tikzpicture}[scale=.2]
%           \node[circle, scale=0.75, fill] (tid0) at (3.75,1.5){};
%           \node[circle, scale=0.75, fill] (tid1) at (1.5,3){};
%           \node[circle, scale=0.75, fill] (tid3) at (1.5,4.5){};
%           \node[circle, scale=0.75, fill] (tid7) at (0.75,6){};
%           \node[circle, scale=0.75, fill, red] (tid9) at (0.75,7.5){};
%           \draw[](tid7) -- (tid9);
%           \node[circle, scale=0.75, fill, red] (tid8) at (2.25,6){};
%           \draw[](tid3) -- (tid7);
%           \draw[](tid3) -- (tid8);
%           \draw[](tid1) -- (tid3);
%           \node[circle, scale=0.75, fill] (tid2) at (5.25,3){};
%           \node[circle, scale=0.75, fill] (tid4) at (3.75,4.5){};
%           \node[circle, scale=0.75, fill] (tid5) at (5.25,4.5){};
%           \node[circle, scale=0.75, fill] (tid6) at (6.75,4.5){};
%           \draw[](tid2) -- (tid4);
%           \draw[](tid2) -- (tid5);
%           \draw[](tid2) -- (tid6);
%           \draw[](tid0) -- (tid1);
%           \draw[](tid0) -- (tid2);
%         \end{tikzpicture}
%         \nodepart{two}
%         \footnotesize{6.35938}
%         \nodepart{three}
%         \footnotesize{$50\:50$}
%       };
%       & 
%       \node[draw=black, rectangle split,  rectangle split parts=3] (sn0x104cb60){
%         \begin{tikzpicture}[scale=.2]
%           \node[circle, scale=0.75, fill] (tid0) at (3.75,1.5){};
%           \node[circle, scale=0.75, fill] (tid1) at (2.25,3){};
%           \node[circle, scale=0.75, fill] (tid3) at (0.75,4.5){};
%           \node[circle, scale=0.75, fill, red] (tid7) at (0.75,6){};
%           \draw[](tid3) -- (tid7);
%           \node[circle, scale=0.75, fill] (tid4) at (2.25,4.5){};
%           \node[circle, scale=0.75, fill] (tid5) at (3.75,4.5){};
%           \draw[](tid1) -- (tid3);
%           \draw[](tid1) -- (tid4);
%           \draw[](tid1) -- (tid5);
%           \node[circle, scale=0.75, fill] (tid2) at (6,3){};
%           \node[circle, scale=0.75, fill] (tid6) at (6,4.5){};
%           \node[circle, scale=0.75, fill, red] (tid8) at (5.25,6){};
%           \node[circle, scale=0.75, fill] (tid9) at (6.75,6){};
%           \draw[](tid6) -- (tid8);
%           \draw[](tid6) -- (tid9);
%           \draw[](tid2) -- (tid6);
%           \draw[](tid0) -- (tid1);
%           \draw[](tid0) -- (tid2);
%         \end{tikzpicture}
%         \nodepart{two}
%         \footnotesize{6.29688}
%         \nodepart{three}
%         \footnotesize{$50\:50$}
%       };
%       & 
%       \\
%     };
%   \end{scope}
%   \begin{scope}[yshift=\leveltopIII cm]
%     \matrix (line3) [column sep=1cm] {
%       \node[draw=black, rectangle split,  rectangle split parts=3] (sn0x10519d0){
%         \begin{tikzpicture}[scale=.2]
%           \node[circle, scale=0.75, fill] (tid0) at (3,1.5){};
%           \node[circle, scale=0.75, fill] (tid1) at (2.25,3){};
%           \node[circle, scale=0.75, fill, red] (tid3) at (0.75,4.5){};
%           \node[circle, scale=0.75, fill] (tid4) at (2.25,4.5){};
%           \node[circle, scale=0.75, fill] (tid5) at (3.75,4.5){};
%           \draw[](tid1) -- (tid3);
%           \draw[](tid1) -- (tid4);
%           \draw[](tid1) -- (tid5);
%           \node[circle, scale=0.75, fill] (tid2) at (5.25,3){};
%           \node[circle, scale=0.75, fill] (tid6) at (5.25,4.5){};
%           \node[circle, scale=0.75, fill] (tid7) at (5.25,6){};
%           \node[circle, scale=0.75, fill, red] (tid8) at (5.25,7.5){};
%           \draw[](tid7) -- (tid8);
%           \draw[](tid6) -- (tid7);
%           \draw[](tid2) -- (tid6);
%           \draw[](tid0) -- (tid1);
%           \draw[](tid0) -- (tid2);
%         \end{tikzpicture}
%         \nodepart{two}
%         \footnotesize{5.92188}
%         \nodepart{three}
%         \footnotesize{$50\:50$}
%       };
%       & 
%       \node[draw=black, rectangle split,  rectangle split parts=3] (sn0x105a080){
%         \begin{tikzpicture}[scale=.2]
%           \node[circle, scale=0.75, fill] (tid0) at (3.75,1.5){};
%           \node[circle, scale=0.75, fill] (tid1) at (2.25,3){};
%           \node[circle, scale=0.75, fill] (tid3) at (0.75,4.5){};
%           \node[circle, scale=0.75, fill] (tid4) at (2.25,4.5){};
%           \node[circle, scale=0.75, fill] (tid5) at (3.75,4.5){};
%           \draw[](tid1) -- (tid3);
%           \draw[](tid1) -- (tid4);
%           \draw[](tid1) -- (tid5);
%           \node[circle, scale=0.75, fill] (tid2) at (6,3){};
%           \node[circle, scale=0.75, fill] (tid6) at (6,4.5){};
%           \node[circle, scale=0.75, fill, red] (tid7) at (5.25,6){};
%           \node[circle, scale=0.75, fill, red] (tid8) at (6.75,6){};
%           \draw[](tid6) -- (tid7);
%           \draw[](tid6) -- (tid8);
%           \draw[](tid2) -- (tid6);
%           \draw[](tid0) -- (tid1);
%           \draw[](tid0) -- (tid2);
%         \end{tikzpicture}
%         \nodepart{two}
%         \footnotesize{5.79688}
%         \nodepart{three}
%         \footnotesize{$1$}
%       };
%       & 
%       \node[draw=black, rectangle split,  rectangle split parts=3] (sn0x104fbd0){
%         \begin{tikzpicture}[scale=.2]
%           \node[circle, scale=0.75, fill] (tid0) at (3,1.5){};
%           \node[circle, scale=0.75, fill] (tid1) at (2.25,3){};
%           \node[circle, scale=0.75, fill] (tid3) at (0.75,4.5){};
%           \node[circle, scale=0.75, fill, red] (tid7) at (0.75,6){};
%           \draw[](tid3) -- (tid7);
%           \node[circle, scale=0.75, fill] (tid4) at (2.25,4.5){};
%           \node[circle, scale=0.75, fill] (tid5) at (3.75,4.5){};
%           \draw[](tid1) -- (tid3);
%           \draw[](tid1) -- (tid4);
%           \draw[](tid1) -- (tid5);
%           \node[circle, scale=0.75, fill] (tid2) at (5.25,3){};
%           \node[circle, scale=0.75, fill] (tid6) at (5.25,4.5){};
%           \node[circle, scale=0.75, fill, red] (tid8) at (5.25,6){};
%           \draw[](tid6) -- (tid8);
%           \draw[](tid2) -- (tid6);
%           \draw[](tid0) -- (tid1);
%           \draw[](tid0) -- (tid2);
%         \end{tikzpicture}
%         \nodepart{two}
%         \footnotesize{5.79688}
%         \nodepart{three}
%         \footnotesize{$50\:33\:17$}
%       };
%       & 
%       \\
%     };
%   \end{scope}
%   \begin{scope}[yshift=\leveltopIIII cm]
%     \matrix (line4) [column sep=1cm] {
%       \node[draw=black, rectangle split,  rectangle split parts=3] (sn0x1052250){
%         \begin{tikzpicture}[scale=.2]
%           \node[circle, scale=0.75, fill] (tid0) at (2.25,1.5){};
%           \node[circle, scale=0.75, fill] (tid1) at (0.75,3){};
%           \node[circle, scale=0.75, fill] (tid3) at (0.75,4.5){};
%           \node[circle, scale=0.75, fill] (tid6) at (0.75,6){};
%           \node[circle, scale=0.75, fill, red] (tid7) at (0.75,7.5){};
%           \draw[](tid6) -- (tid7);
%           \draw[](tid3) -- (tid6);
%           \draw[](tid1) -- (tid3);
%           \node[circle, scale=0.75, fill] (tid2) at (3,3){};
%           \node[circle, scale=0.75, fill, red] (tid4) at (2.25,4.5){};
%           \node[circle, scale=0.75, fill] (tid5) at (3.75,4.5){};
%           \draw[](tid2) -- (tid4);
%           \draw[](tid2) -- (tid5);
%           \draw[](tid0) -- (tid1);
%           \draw[](tid0) -- (tid2);
%         \end{tikzpicture}
%         \nodepart{two}
%         \footnotesize{5.54688}
%         \nodepart{three}
%         \footnotesize{$50\:50$}
%       };
%       & 
%       \node[draw=black, rectangle split,  rectangle split parts=3] (sn0x1052960){
%         \begin{tikzpicture}[scale=.2]
%           \node[circle, scale=0.75, fill] (tid0) at (3,1.5){};
%           \node[circle, scale=0.75, fill] (tid1) at (2.25,3){};
%           \node[circle, scale=0.75, fill, red] (tid3) at (0.75,4.5){};
%           \node[circle, scale=0.75, fill] (tid4) at (2.25,4.5){};
%           \node[circle, scale=0.75, fill] (tid5) at (3.75,4.5){};
%           \draw[](tid1) -- (tid3);
%           \draw[](tid1) -- (tid4);
%           \draw[](tid1) -- (tid5);
%           \node[circle, scale=0.75, fill] (tid2) at (5.25,3){};
%           \node[circle, scale=0.75, fill] (tid6) at (5.25,4.5){};
%           \node[circle, scale=0.75, fill, red] (tid7) at (5.25,6){};
%           \draw[](tid6) -- (tid7);
%           \draw[](tid2) -- (tid6);
%           \draw[](tid0) -- (tid1);
%           \draw[](tid0) -- (tid2);
%         \end{tikzpicture}
%         \nodepart{two}
%         \footnotesize{5.29688}
%         \nodepart{three}
%         \footnotesize{$50\:33\:17$}
%       };
%       & 
%       \node[draw=black, rectangle split,  rectangle split parts=3] (sn0x10581e0){
%         \begin{tikzpicture}[scale=.2]
%           \node[circle, scale=0.75, fill] (tid0) at (3,1.5){};
%           \node[circle, scale=0.75, fill] (tid1) at (2.25,3){};
%           \node[circle, scale=0.75, fill] (tid3) at (0.75,4.5){};
%           \node[circle, scale=0.75, fill, red] (tid7) at (0.75,6){};
%           \draw[](tid3) -- (tid7);
%           \node[circle, scale=0.75, fill, red] (tid4) at (2.25,4.5){};
%           \node[circle, scale=0.75, fill] (tid5) at (3.75,4.5){};
%           \draw[](tid1) -- (tid3);
%           \draw[](tid1) -- (tid4);
%           \draw[](tid1) -- (tid5);
%           \node[circle, scale=0.75, fill] (tid2) at (5.25,3){};
%           \node[circle, scale=0.75, fill] (tid6) at (5.25,4.5){};
%           \draw[](tid2) -- (tid6);
%           \draw[](tid0) -- (tid1);
%           \draw[](tid0) -- (tid2);
%         \end{tikzpicture}
%         \nodepart{two}
%         \footnotesize{5.29688}
%         \nodepart{three}
%         \footnotesize{$33\:17\:25\:25$}
%       };
%       & 
%       \node[draw=black, rectangle split,  rectangle split parts=3] (sn0x1058550){
%         \begin{tikzpicture}[scale=.2]
%           \node[circle, scale=0.75, fill] (tid0) at (3,1.5){};
%           \node[circle, scale=0.75, fill] (tid1) at (2.25,3){};
%           \node[circle, scale=0.75, fill] (tid3) at (0.75,4.5){};
%           \node[circle, scale=0.75, fill, red] (tid7) at (0.75,6){};
%           \draw[](tid3) -- (tid7);
%           \node[circle, scale=0.75, fill] (tid4) at (2.25,4.5){};
%           \node[circle, scale=0.75, fill] (tid5) at (3.75,4.5){};
%           \draw[](tid1) -- (tid3);
%           \draw[](tid1) -- (tid4);
%           \draw[](tid1) -- (tid5);
%           \node[circle, scale=0.75, fill] (tid2) at (5.25,3){};
%           \node[circle, scale=0.75, fill, red] (tid6) at (5.25,4.5){};
%           \draw[](tid2) -- (tid6);
%           \draw[](tid0) -- (tid1);
%           \draw[](tid0) -- (tid2);
%         \end{tikzpicture}
%         \nodepart{two}
%         \footnotesize{5.29688}
%         \nodepart{three}
%         \footnotesize{$50\:50$}
%       };
%       & 
%       \\
%     };
%   \end{scope}
%   \begin{scope}[yshift=\leveltopIIIII cm]
%     \matrix (line5) [column sep=1cm] {
%       \node[draw=black, rectangle split,  rectangle split parts=3] (sn0x10525f0){
%         \begin{tikzpicture}[scale=.2]
%           \node[circle, scale=0.75, fill] (tid0) at (1.5,1.5){};
%           \node[circle, scale=0.75, fill] (tid1) at (0.75,3){};
%           \node[circle, scale=0.75, fill] (tid3) at (0.75,4.5){};
%           \node[circle, scale=0.75, fill] (tid5) at (0.75,6){};
%           \node[circle, scale=0.75, fill, red] (tid6) at (0.75,7.5){};
%           \draw[](tid5) -- (tid6);
%           \draw[](tid3) -- (tid5);
%           \draw[](tid1) -- (tid3);
%           \node[circle, scale=0.75, fill] (tid2) at (2.25,3){};
%           \node[circle, scale=0.75, fill, red] (tid4) at (2.25,4.5){};
%           \draw[](tid2) -- (tid4);
%           \draw[](tid0) -- (tid1);
%           \draw[](tid0) -- (tid2);
%         \end{tikzpicture}
%         \nodepart{two}
%         \footnotesize{5.25}
%         \nodepart{three}
%         \footnotesize{$50\:50$}
%       };
%       & 
%       \node[draw=black, rectangle split,  rectangle split parts=3] (sn0x1053850){
%         \begin{tikzpicture}[scale=.2]
%           \node[circle, scale=0.75, fill] (tid0) at (2.25,1.5){};
%           \node[circle, scale=0.75, fill] (tid1) at (1.5,3){};
%           \node[circle, scale=0.75, fill, red] (tid3) at (0.75,4.5){};
%           \node[circle, scale=0.75, fill] (tid4) at (2.25,4.5){};
%           \draw[](tid1) -- (tid3);
%           \draw[](tid1) -- (tid4);
%           \node[circle, scale=0.75, fill] (tid2) at (3.75,3){};
%           \node[circle, scale=0.75, fill] (tid5) at (3.75,4.5){};
%           \node[circle, scale=0.75, fill, red] (tid6) at (3.75,6){};
%           \draw[](tid5) -- (tid6);
%           \draw[](tid2) -- (tid5);
%           \draw[](tid0) -- (tid1);
%           \draw[](tid0) -- (tid2);
%         \end{tikzpicture}
%         \nodepart{two}
%         \footnotesize{4.84375}
%         \nodepart{three}
%         \footnotesize{$50\:25\:25$}
%       };
%       & 
%       \node[draw=black, rectangle split,  rectangle split parts=3] (sn0x1056b00){
%         \begin{tikzpicture}[scale=.2]
%           \node[circle, scale=0.75, fill] (tid0) at (3,1.5){};
%           \node[circle, scale=0.75, fill] (tid1) at (2.25,3){};
%           \node[circle, scale=0.75, fill, red] (tid3) at (0.75,4.5){};
%           \node[circle, scale=0.75, fill, red] (tid4) at (2.25,4.5){};
%           \node[circle, scale=0.75, fill] (tid5) at (3.75,4.5){};
%           \draw[](tid1) -- (tid3);
%           \draw[](tid1) -- (tid4);
%           \draw[](tid1) -- (tid5);
%           \node[circle, scale=0.75, fill] (tid2) at (5.25,3){};
%           \node[circle, scale=0.75, fill] (tid6) at (5.25,4.5){};
%           \draw[](tid2) -- (tid6);
%           \draw[](tid0) -- (tid1);
%           \draw[](tid0) -- (tid2);
%         \end{tikzpicture}
%         \nodepart{two}
%         \footnotesize{4.75}
%         \nodepart{three}
%         \footnotesize{$50\:50$}
%       };
%       & 
%       \node[draw=black, rectangle split,  rectangle split parts=3] (sn0x1056fb0){
%         \begin{tikzpicture}[scale=.2]
%           \node[circle, scale=0.75, fill] (tid0) at (3,1.5){};
%           \node[circle, scale=0.75, fill] (tid1) at (2.25,3){};
%           \node[circle, scale=0.75, fill, red] (tid3) at (0.75,4.5){};
%           \node[circle, scale=0.75, fill] (tid4) at (2.25,4.5){};
%           \node[circle, scale=0.75, fill] (tid5) at (3.75,4.5){};
%           \draw[](tid1) -- (tid3);
%           \draw[](tid1) -- (tid4);
%           \draw[](tid1) -- (tid5);
%           \node[circle, scale=0.75, fill] (tid2) at (5.25,3){};
%           \node[circle, scale=0.75, fill, red] (tid6) at (5.25,4.5){};
%           \draw[](tid2) -- (tid6);
%           \draw[](tid0) -- (tid1);
%           \draw[](tid0) -- (tid2);
%         \end{tikzpicture}
%         \nodepart{two}
%         \footnotesize{4.75}
%         \nodepart{three}
%         \footnotesize{$50\:50$}
%       };
%       & 
%       \node[draw=black, rectangle split,  rectangle split parts=3] (sn0x1058f50){
%         \begin{tikzpicture}[scale=.2]
%           \node[circle, scale=0.75, fill] (tid0) at (2.25,1.5){};
%           \node[circle, scale=0.75, fill] (tid1) at (1.5,3){};
%           \node[circle, scale=0.75, fill] (tid3) at (0.75,4.5){};
%           \node[circle, scale=0.75, fill, red] (tid6) at (0.75,6){};
%           \draw[](tid3) -- (tid6);
%           \node[circle, scale=0.75, fill, red] (tid4) at (2.25,4.5){};
%           \draw[](tid1) -- (tid3);
%           \draw[](tid1) -- (tid4);
%           \node[circle, scale=0.75, fill] (tid2) at (3.75,3){};
%           \node[circle, scale=0.75, fill] (tid5) at (3.75,4.5){};
%           \draw[](tid2) -- (tid5);
%           \draw[](tid0) -- (tid1);
%           \draw[](tid0) -- (tid2);
%         \end{tikzpicture}
%         \nodepart{two}
%         \footnotesize{4.84375}
%         \nodepart{three}
%         \footnotesize{$50\:25\:25$}
%       };
%       & 
%       \node[draw=black, rectangle split,  rectangle split parts=3] (sn0x1058a50){
%         \begin{tikzpicture}[scale=.2]
%           \node[circle, scale=0.75, fill] (tid0) at (2.25,1.5){};
%           \node[circle, scale=0.75, fill] (tid1) at (1.5,3){};
%           \node[circle, scale=0.75, fill] (tid3) at (0.75,4.5){};
%           \node[circle, scale=0.75, fill, red] (tid6) at (0.75,6){};
%           \draw[](tid3) -- (tid6);
%           \node[circle, scale=0.75, fill] (tid4) at (2.25,4.5){};
%           \draw[](tid1) -- (tid3);
%           \draw[](tid1) -- (tid4);
%           \node[circle, scale=0.75, fill] (tid2) at (3.75,3){};
%           \node[circle, scale=0.75, fill, red] (tid5) at (3.75,4.5){};
%           \draw[](tid2) -- (tid5);
%           \draw[](tid0) -- (tid1);
%           \draw[](tid0) -- (tid2);
%         \end{tikzpicture}
%         \nodepart{two}
%         \footnotesize{4.84375}
%         \nodepart{three}
%         \footnotesize{$50\:50$}
%       };
%       & 
%       \node[draw=black, rectangle split,  rectangle split parts=3] (sn0x10597a0){
%         \begin{tikzpicture}[scale=.2]
%           \node[circle, scale=0.75, fill] (tid0) at (3,1.5){};
%           \node[circle, scale=0.75, fill] (tid1) at (2.25,3){};
%           \node[circle, scale=0.75, fill] (tid3) at (0.75,4.5){};
%           \node[circle, scale=0.75, fill, red] (tid6) at (0.75,6){};
%           \draw[](tid3) -- (tid6);
%           \node[circle, scale=0.75, fill, red] (tid4) at (2.25,4.5){};
%           \node[circle, scale=0.75, fill] (tid5) at (3.75,4.5){};
%           \draw[](tid1) -- (tid3);
%           \draw[](tid1) -- (tid4);
%           \draw[](tid1) -- (tid5);
%           \node[circle, scale=0.75, fill] (tid2) at (5.25,3){};
%           \draw[](tid0) -- (tid1);
%           \draw[](tid0) -- (tid2);
%         \end{tikzpicture}
%         \nodepart{two}
%         \footnotesize{4.84375}
%         \nodepart{three}
%         \footnotesize{$50\:50$}
%       };
%       & 
%       \\
%     };
%   \end{scope}
%   \begin{scope}[yshift=\leveltopIIIIII cm]
%     \matrix (line6) [column sep=1cm] {
%       \node[draw=black, rectangle split,  rectangle split parts=3] (sn0x1053920){
%         \begin{tikzpicture}[scale=.2]
%           \node[circle, scale=0.75, fill] (tid0) at (1.5,1.5){};
%           \node[circle, scale=0.75, fill] (tid1) at (0.75,3){};
%           \node[circle, scale=0.75, fill] (tid3) at (0.75,4.5){};
%           \node[circle, scale=0.75, fill] (tid4) at (0.75,6){};
%           \node[circle, scale=0.75, fill, red] (tid5) at (0.75,7.5){};
%           \draw[](tid4) -- (tid5);
%           \draw[](tid3) -- (tid4);
%           \draw[](tid1) -- (tid3);
%           \node[circle, scale=0.75, fill, red] (tid2) at (2.25,3){};
%           \draw[](tid0) -- (tid1);
%           \draw[](tid0) -- (tid2);
%         \end{tikzpicture}
%         \nodepart{two}
%         \footnotesize{5.0625}
%         \nodepart{three}
%         \footnotesize{$50\:50$}
%       };
%       & 
%       \node[draw=black, rectangle split,  rectangle split parts=3] (sn0x1053bc0){
%         \begin{tikzpicture}[scale=.2]
%           \node[circle, scale=0.75, fill] (tid0) at (1.5,1.5){};
%           \node[circle, scale=0.75, fill] (tid1) at (0.75,3){};
%           \node[circle, scale=0.75, fill] (tid3) at (0.75,4.5){};
%           \node[circle, scale=0.75, fill, red] (tid5) at (0.75,6){};
%           \draw[](tid3) -- (tid5);
%           \draw[](tid1) -- (tid3);
%           \node[circle, scale=0.75, fill] (tid2) at (2.25,3){};
%           \node[circle, scale=0.75, fill, red] (tid4) at (2.25,4.5){};
%           \draw[](tid2) -- (tid4);
%           \draw[](tid0) -- (tid1);
%           \draw[](tid0) -- (tid2);
%         \end{tikzpicture}
%         \nodepart{two}
%         \footnotesize{4.4375}
%         \nodepart{three}
%         \footnotesize{$50\:50$}
%       };
%       & 
%       \node[draw=black, rectangle split,  rectangle split parts=3] (sn0x1056090){
%         \begin{tikzpicture}[scale=.2]
%           \node[circle, scale=0.75, fill] (tid0) at (2.25,1.5){};
%           \node[circle, scale=0.75, fill] (tid1) at (1.5,3){};
%           \node[circle, scale=0.75, fill, red] (tid3) at (0.75,4.5){};
%           \node[circle, scale=0.75, fill, red] (tid4) at (2.25,4.5){};
%           \draw[](tid1) -- (tid3);
%           \draw[](tid1) -- (tid4);
%           \node[circle, scale=0.75, fill] (tid2) at (3.75,3){};
%           \node[circle, scale=0.75, fill] (tid5) at (3.75,4.5){};
%           \draw[](tid2) -- (tid5);
%           \draw[](tid0) -- (tid1);
%           \draw[](tid0) -- (tid2);
%         \end{tikzpicture}
%         \nodepart{two}
%         \footnotesize{4.25}
%         \nodepart{three}
%         \footnotesize{$1$}
%       };
%       & 
%       \node[draw=black, rectangle split,  rectangle split parts=3] (sn0x1056160){
%         \begin{tikzpicture}[scale=.2]
%           \node[circle, scale=0.75, fill] (tid0) at (2.25,1.5){};
%           \node[circle, scale=0.75, fill] (tid1) at (1.5,3){};
%           \node[circle, scale=0.75, fill, red] (tid3) at (0.75,4.5){};
%           \node[circle, scale=0.75, fill] (tid4) at (2.25,4.5){};
%           \draw[](tid1) -- (tid3);
%           \draw[](tid1) -- (tid4);
%           \node[circle, scale=0.75, fill] (tid2) at (3.75,3){};
%           \node[circle, scale=0.75, fill, red] (tid5) at (3.75,4.5){};
%           \draw[](tid2) -- (tid5);
%           \draw[](tid0) -- (tid1);
%           \draw[](tid0) -- (tid2);
%         \end{tikzpicture}
%         \nodepart{two}
%         \footnotesize{4.25}
%         \nodepart{three}
%         \footnotesize{$50\:50$}
%       };
%       & 
%       \node[draw=black, rectangle split,  rectangle split parts=3] (sn0x1057630){
%         \begin{tikzpicture}[scale=.2]
%           \node[circle, scale=0.75, fill] (tid0) at (3,1.5){};
%           \node[circle, scale=0.75, fill] (tid1) at (2.25,3){};
%           \node[circle, scale=0.75, fill, red] (tid3) at (0.75,4.5){};
%           \node[circle, scale=0.75, fill, red] (tid4) at (2.25,4.5){};
%           \node[circle, scale=0.75, fill] (tid5) at (3.75,4.5){};
%           \draw[](tid1) -- (tid3);
%           \draw[](tid1) -- (tid4);
%           \draw[](tid1) -- (tid5);
%           \node[circle, scale=0.75, fill] (tid2) at (5.25,3){};
%           \draw[](tid0) -- (tid1);
%           \draw[](tid0) -- (tid2);
%         \end{tikzpicture}
%         \nodepart{two}
%         \footnotesize{4.25}
%         \nodepart{three}
%         \footnotesize{$1$}
%       };
%       & 
%       \node[draw=black, rectangle split,  rectangle split parts=3] (sn0x1058b20){
%         \begin{tikzpicture}[scale=.2]
%           \node[circle, scale=0.75, fill] (tid0) at (2.25,1.5){};
%           \node[circle, scale=0.75, fill] (tid1) at (1.5,3){};
%           \node[circle, scale=0.75, fill] (tid3) at (0.75,4.5){};
%           \node[circle, scale=0.75, fill, red] (tid5) at (0.75,6){};
%           \draw[](tid3) -- (tid5);
%           \node[circle, scale=0.75, fill, red] (tid4) at (2.25,4.5){};
%           \draw[](tid1) -- (tid3);
%           \draw[](tid1) -- (tid4);
%           \node[circle, scale=0.75, fill] (tid2) at (3.75,3){};
%           \draw[](tid0) -- (tid1);
%           \draw[](tid0) -- (tid2);
%         \end{tikzpicture}
%         \nodepart{two}
%         \footnotesize{4.4375}
%         \nodepart{three}
%         \footnotesize{$50\:50$}
%       };
%       & 
%       \\
%     };
%   \end{scope}
%   \begin{scope}[yshift=\leveltopIIIIIII cm]
%     \matrix (line7) [column sep=1cm] {
%       \node[draw=black, rectangle split,  rectangle split parts=3] (sn0x10540d0){
%         \begin{tikzpicture}[scale=.2]
%           \node[circle, scale=0.75, fill] (tid0) at (0.75,1.5){};
%           \node[circle, scale=0.75, fill] (tid1) at (0.75,3){};
%           \node[circle, scale=0.75, fill] (tid2) at (0.75,4.5){};
%           \node[circle, scale=0.75, fill] (tid3) at (0.75,6){};
%           \node[circle, scale=0.75, fill, red] (tid4) at (0.75,7.5){};
%           \draw[](tid3) -- (tid4);
%           \draw[](tid2) -- (tid3);
%           \draw[](tid1) -- (tid2);
%           \draw[](tid0) -- (tid1);
%         \end{tikzpicture}
%         \nodepart{two}
%         \footnotesize{5}
%         \nodepart{three}
%         \footnotesize{$1$}
%       };
%       & 
%       \node[draw=black, rectangle split,  rectangle split parts=3] (sn0x1054480){
%         \begin{tikzpicture}[scale=.2]
%           \node[circle, scale=0.75, fill] (tid0) at (1.5,1.5){};
%           \node[circle, scale=0.75, fill] (tid1) at (0.75,3){};
%           \node[circle, scale=0.75, fill] (tid3) at (0.75,4.5){};
%           \node[circle, scale=0.75, fill, red] (tid4) at (0.75,6){};
%           \draw[](tid3) -- (tid4);
%           \draw[](tid1) -- (tid3);
%           \node[circle, scale=0.75, fill, red] (tid2) at (2.25,3){};
%           \draw[](tid0) -- (tid1);
%           \draw[](tid0) -- (tid2);
%         \end{tikzpicture}
%         \nodepart{two}
%         \footnotesize{4.125}
%         \nodepart{three}
%         \footnotesize{$50\:50$}
%       };
%       & 
%       \node[draw=black, rectangle split,  rectangle split parts=3] (sn0x1055dd0){
%         \begin{tikzpicture}[scale=.2]
%           \node[circle, scale=0.75, fill] (tid0) at (1.5,1.5){};
%           \node[circle, scale=0.75, fill] (tid1) at (0.75,3){};
%           \node[circle, scale=0.75, fill, red] (tid3) at (0.75,4.5){};
%           \draw[](tid1) -- (tid3);
%           \node[circle, scale=0.75, fill] (tid2) at (2.25,3){};
%           \node[circle, scale=0.75, fill, red] (tid4) at (2.25,4.5){};
%           \draw[](tid2) -- (tid4);
%           \draw[](tid0) -- (tid1);
%           \draw[](tid0) -- (tid2);
%         \end{tikzpicture}
%         \nodepart{two}
%         \footnotesize{3.75}
%         \nodepart{three}
%         \footnotesize{$1$}
%       };
%       & 
%       \node[draw=black, rectangle split,  rectangle split parts=3] (sn0x10568c0){
%         \begin{tikzpicture}[scale=.2]
%           \node[circle, scale=0.75, fill] (tid0) at (2.25,1.5){};
%           \node[circle, scale=0.75, fill] (tid1) at (1.5,3){};
%           \node[circle, scale=0.75, fill, red] (tid3) at (0.75,4.5){};
%           \node[circle, scale=0.75, fill, red] (tid4) at (2.25,4.5){};
%           \draw[](tid1) -- (tid3);
%           \draw[](tid1) -- (tid4);
%           \node[circle, scale=0.75, fill] (tid2) at (3.75,3){};
%           \draw[](tid0) -- (tid1);
%           \draw[](tid0) -- (tid2);
%         \end{tikzpicture}
%         \nodepart{two}
%         \footnotesize{3.75}
%         \nodepart{three}
%         \footnotesize{$1$}
%       };
%       & 
%       \\
%     };
%   \end{scope}
%   \begin{scope}[yshift=\leveltopIIIIIIII cm]
%     \matrix (line8) [column sep=1cm] {
%       \node[draw=black, rectangle split,  rectangle split parts=3] (sn0x1054550){
%         \begin{tikzpicture}[scale=.2]
%           \node[circle, scale=0.75, fill] (tid0) at (0.75,1.5){};
%           \node[circle, scale=0.75, fill] (tid1) at (0.75,3){};
%           \node[circle, scale=0.75, fill] (tid2) at (0.75,4.5){};
%           \node[circle, scale=0.75, fill, red] (tid3) at (0.75,6){};
%           \draw[](tid2) -- (tid3);
%           \draw[](tid1) -- (tid2);
%           \draw[](tid0) -- (tid1);
%         \end{tikzpicture}
%         \nodepart{two}
%         \footnotesize{4}
%         \nodepart{three}
%         \footnotesize{$1$}
%       };
%       & 
%       \node[draw=black, rectangle split,  rectangle split parts=3] (sn0x1055270){
%         \begin{tikzpicture}[scale=.2]
%           \node[circle, scale=0.75, fill] (tid0) at (1.5,1.5){};
%           \node[circle, scale=0.75, fill] (tid1) at (0.75,3){};
%           \node[circle, scale=0.75, fill, red] (tid3) at (0.75,4.5){};
%           \draw[](tid1) -- (tid3);
%           \node[circle, scale=0.75, fill, red] (tid2) at (2.25,3){};
%           \draw[](tid0) -- (tid1);
%           \draw[](tid0) -- (tid2);
%         \end{tikzpicture}
%         \nodepart{two}
%         \footnotesize{3.25}
%         \nodepart{three}
%         \footnotesize{$50\:50$}
%       };
%       & 
%       \\
%     };
%   \end{scope}
%   \begin{scope}[yshift=\leveltopIIIIIIIII cm]
%     \matrix (line9) [column sep=1cm] {
%       \node[draw=black, rectangle split,  rectangle split parts=3] (sn0x1054a50){
%         \begin{tikzpicture}[scale=.2]
%           \node[circle, scale=0.75, fill] (tid0) at (0.75,1.5){};
%           \node[circle, scale=0.75, fill] (tid1) at (0.75,3){};
%           \node[circle, scale=0.75, fill, red] (tid2) at (0.75,4.5){};
%           \draw[](tid1) -- (tid2);
%           \draw[](tid0) -- (tid1);
%         \end{tikzpicture}
%         \nodepart{two}
%         \footnotesize{3}
%         \nodepart{three}
%         \footnotesize{$1$}
%       };
%       & 
%       \node[draw=black, rectangle split,  rectangle split parts=3] (sn0x1054cb0){
%         \begin{tikzpicture}[scale=.2]
%           \node[circle, scale=0.75, fill] (tid0) at (1.5,1.5){};
%           \node[circle, scale=0.75, fill, red] (tid1) at (0.75,3){};
%           \node[circle, scale=0.75, fill, red] (tid2) at (2.25,3){};
%           \draw[](tid0) -- (tid1);
%           \draw[](tid0) -- (tid2);
%         \end{tikzpicture}
%         \nodepart{two}
%         \footnotesize{2.5}
%         \nodepart{three}
%         \footnotesize{$1$}
%       };
%       & 
%       \\
%     };
%   \end{scope}
%   \begin{scope}[yshift=\leveltopIIIIIIIIII cm]
%     \matrix (line10) [column sep=1cm] {
%       \node[draw=black, rectangle split,  rectangle split parts=3] (sn0x1054b20){
%         \begin{tikzpicture}[scale=.2]
%           \node[circle, scale=0.75, fill] (tid0) at (0.75,1.5){};
%           \node[circle, scale=0.75, fill, red] (tid1) at (0.75,3){};
%           \draw[](tid0) -- (tid1);
%         \end{tikzpicture}
%         \nodepart{two}
%         \footnotesize{2}
%         \nodepart{three}
%         \footnotesize{$1$}
%       };
%       & 
%       \\
%     };
%   \end{scope}
%   \begin{scope}[yshift=\leveltopIIIIIIIIIII cm]
%     \matrix (line11) [column sep=1cm] {
%       \node[draw=black, rectangle split,  rectangle split parts=3] (sn0x10547e0){
%         \begin{tikzpicture}[scale=.2]
%           \node[circle, scale=0.75, fill, red] (tid0) at (0.75,1.5){};
%         \end{tikzpicture}
%         \nodepart{two}
%         \footnotesize{1}
%         \nodepart{three}
%         \footnotesize{$$}
%       };
%       & 
%       \\
%     };
%   \end{scope}
%   \begin{scope}[yshift=\leveltopIIIIIIIIIIII cm]
%     \matrix (line12) [column sep=1cm] {
%       \\
%     };
%   \end{scope}
%   \draw (sn0x1050af0.south) -- (sn0x105a150.north);
%   \draw (sn0x1050af0.south) -- (sn0x104cb60.north);
%   \draw (sn0x105a150.south) -- (sn0x10519d0.north);
%   \draw (sn0x105a150.south) -- (sn0x105a080.north);
%   \draw (sn0x104cb60.south) -- (sn0x105a080.north);
%   \draw (sn0x104cb60.south) -- (sn0x104fbd0.north);
%   \draw (sn0x10519d0.south) -- (sn0x1052250.north);
%   \draw (sn0x10519d0.south) -- (sn0x1052960.north);
%   \draw (sn0x105a080.south) -- (sn0x1052960.north);
%   \draw (sn0x104fbd0.south) -- (sn0x1052960.north);
%   \draw (sn0x104fbd0.south) -- (sn0x10581e0.north);
%   \draw (sn0x104fbd0.south) -- (sn0x1058550.north);
%   \draw (sn0x1052250.south) -- (sn0x10525f0.north);
%   \draw (sn0x1052250.south) -- (sn0x1053850.north);
%   \draw (sn0x1052960.south) -- (sn0x1053850.north);
%   \draw (sn0x1052960.south) -- (sn0x1056b00.north);
%   \draw (sn0x1052960.south) -- (sn0x1056fb0.north);
%   \draw (sn0x10581e0.south) -- (sn0x1058f50.north);
%   \draw (sn0x10581e0.south) -- (sn0x1058a50.north);
%   \draw (sn0x10581e0.south) -- (sn0x1056b00.north);
%   \draw (sn0x10581e0.south) -- (sn0x1056fb0.north);
%   \draw (sn0x1058550.south) -- (sn0x10597a0.north);
%   \draw (sn0x1058550.south) -- (sn0x1056fb0.north);
%   \draw (sn0x10525f0.south) -- (sn0x1053920.north);
%   \draw (sn0x10525f0.south) -- (sn0x1053bc0.north);
%   \draw (sn0x1053850.south) -- (sn0x1053bc0.north);
%   \draw (sn0x1053850.south) -- (sn0x1056090.north);
%   \draw (sn0x1053850.south) -- (sn0x1056160.north);
%   \draw (sn0x1056b00.south) -- (sn0x1056090.north);
%   \draw (sn0x1056b00.south) -- (sn0x1056160.north);
%   \draw (sn0x1056fb0.south) -- (sn0x1056160.north);
%   \draw (sn0x1056fb0.south) -- (sn0x1057630.north);
%   \draw (sn0x1058f50.south) -- (sn0x1053bc0.north);
%   \draw (sn0x1058f50.south) -- (sn0x1056090.north);
%   \draw (sn0x1058f50.south) -- (sn0x1056160.north);
%   \draw (sn0x1058a50.south) -- (sn0x1058b20.north);
%   \draw (sn0x1058a50.south) -- (sn0x1056160.north);
%   \draw (sn0x10597a0.south) -- (sn0x1058b20.north);
%   \draw (sn0x10597a0.south) -- (sn0x1057630.north);
%   \draw (sn0x1053920.south) -- (sn0x10540d0.north);
%   \draw (sn0x1053920.south) -- (sn0x1054480.north);
%   \draw (sn0x1053bc0.south) -- (sn0x1054480.north);
%   \draw (sn0x1053bc0.south) -- (sn0x1055dd0.north);
%   \draw (sn0x1056090.south) -- (sn0x1055dd0.north);
%   \draw (sn0x1056160.south) -- (sn0x1055dd0.north);
%   \draw (sn0x1056160.south) -- (sn0x10568c0.north);
%   \draw (sn0x1057630.south) -- (sn0x10568c0.north);
%   \draw (sn0x1058b20.south) -- (sn0x1054480.north);
%   \draw (sn0x1058b20.south) -- (sn0x10568c0.north);
%   \draw (sn0x10540d0.south) -- (sn0x1054550.north);
%   \draw (sn0x1054480.south) -- (sn0x1054550.north);
%   \draw (sn0x1054480.south) -- (sn0x1055270.north);
%   \draw (sn0x1055dd0.south) -- (sn0x1055270.north);
%   \draw (sn0x10568c0.south) -- (sn0x1055270.north);
%   \draw (sn0x1054550.south) -- (sn0x1054a50.north);
%   \draw (sn0x1055270.south) -- (sn0x1054a50.north);
%   \draw (sn0x1055270.south) -- (sn0x1054cb0.north);
%   \draw (sn0x1054a50.south) -- (sn0x1054b20.north);
%   \draw (sn0x1054cb0.south) -- (sn0x1054b20.north);
%   \draw (sn0x1054b20.south) -- (sn0x10547e0.north);
% \end{tikzpicture}

%%% Local Variables:
%%% TeX-master: "thesis/thesis.tex"
%%% End: 

%\begin{tikzpicture}[scale=.2]
  \begin{scope}
    \node[draw=black] (sn0x115b5e0W9) at (9, -10) {\begin{tikzpicture}
        [scale=.2]
        \node[circle,scale=0.75,fill]{}[grow=up,sibling distance=4cm]
        child{node[circle,scale=0.75,fill]{}[grow=up]
          child{node[circle,scale=0.75,fill]{}[grow=up,sibling distance=1.5cm]
            child{node[circle,scale=0.75,fill]{}[grow=up]
              child{node[circle,scale=0.75,fill,red]{}[grow=up]
              }
            }
            child{node[circle,scale=0.75,fill,red]{}[grow=up]
            }
          }
        }
        child{node[circle,scale=0.75,fill]{}[grow=up,sibling distance=1.5cm]
          child{node[circle,scale=0.75,fill]{}[grow=up]
            child{node[circle,scale=0.75,fill]{}[grow=up]
            }
          }
          child{node[circle,scale=0.75,fill]{}[grow=up]
          }
          child{node[circle,scale=0.75,fill]{}[grow=up]
          }
        }
        ;
      \end{tikzpicture}
    };
    \node[draw=black] (sn0x115c080W9) at (9, -20) {\begin{tikzpicture}[scale=.2]
        \node[circle,scale=0.75,fill]{}[grow=up,sibling distance=3.5cm]
        child{node[circle,scale=0.75,fill]{}[grow=up,sibling distance=1.5cm]
          child{node[circle,scale=0.75,fill]{}[grow=up]
            child{node[circle,scale=0.75,fill,red]{}[grow=up]
            }
          }
          child{node[circle,scale=0.75,fill]{}[grow=up]
          }
          child{node[circle,scale=0.75,fill]{}[grow=up]
          }
        }
        child{node[circle,scale=0.75,fill]{}[grow=up]
          child{node[circle,scale=0.75,fill]{}[grow=up]
            child{node[circle,scale=0.75,fill]{}[grow=up]
              child{node[circle,scale=0.75,fill,red]{}[grow=up]
              }
            }
          }
        }
        ;
      \end{tikzpicture}
    };
    \node[draw=black] (sn0x115e1f0W9) at (9, -30) {\begin{tikzpicture}[scale=.2]
        \node[circle,scale=0.75,fill]{}[grow=up,sibling distance=3cm]
        child{node[circle,scale=0.75,fill]{}[grow=up]
          child{node[circle,scale=0.75,fill]{}[grow=up]
            child{node[circle,scale=0.75,fill]{}[grow=up]
              child{node[circle,scale=0.75,fill,red]{}[grow=up]
              }
            }
          }
        }
        child{node[circle,scale=0.75,fill]{}[grow=up,sibling distance=1.5cm]
          child{node[circle,scale=0.75,fill,red]{}[grow=up]
          }
          child{node[circle,scale=0.75,fill]{}[grow=up]
          }
          child{node[circle,scale=0.75,fill]{}[grow=up]
          }
        }
        ;
      \end{tikzpicture}
    };
    \node[draw=black] (sn0x115e9e0W9) at (9, -40) {\begin{tikzpicture}[scale=.2]
        \node[circle,scale=0.75,fill]{}[grow=up,sibling distance=2.5cm]
        child{node[circle,scale=0.75,fill]{}[grow=up]
          child{node[circle,scale=0.75,fill]{}[grow=up]
            child{node[circle,scale=0.75,fill]{}[grow=up]
              child{node[circle,scale=0.75,fill,red]{}[grow=up]
              }
            }
          }
        }
        child{node[circle,scale=0.75,fill]{}[grow=up,sibling distance=1.5cm]
          child{node[circle,scale=0.75,fill,red]{}[grow=up]
          }
          child{node[circle,scale=0.75,fill]{}[grow=up]
          }
        }
        ;
      \end{tikzpicture}
    };
    \node[draw=black] (sn0x115f370W9) at (9, -50) {\begin{tikzpicture}[scale=.2]
        \node[circle,scale=0.75,fill]{}[grow=up]
        child{node[circle,scale=0.75,fill]{}[grow=up]
          child{node[circle,scale=0.75,fill]{}[grow=up]
            child{node[circle,scale=0.75,fill]{}[grow=up]
              child{node[circle,scale=0.75,fill,red]{}[grow=up]
              }
            }
          }
        }
        child{node[circle,scale=0.75,fill]{}[grow=up]
          child{node[circle,scale=0.75,fill,red]{}[grow=up]
          }
        }
        ;
      \end{tikzpicture}
    };
    \node[draw=black] (sn0x115f500W9) at (9, -60) {\begin{tikzpicture}[scale=.2]
        \node[circle,scale=0.75,fill]{}[grow=up]
        child{node[circle,scale=0.75,fill]{}[grow=up]
          child{node[circle,scale=0.75,fill]{}[grow=up]
            child{node[circle,scale=0.75,fill]{}[grow=up]
              child{node[circle,scale=0.75,fill,red]{}[grow=up]
              }
            }
          }
        }
        child{node[circle,scale=0.75,fill,red]{}[grow=up]
        }
        ;
      \end{tikzpicture}
    };
    \node[draw=black] (sn0x1160260W9) at (9, -70) {\begin{tikzpicture}[scale=.2]
        \node[circle,scale=0.75,fill]{}[grow=up]
        child{node[circle,scale=0.75,fill]{}[grow=up]
          child{node[circle,scale=0.75,fill]{}[grow=up]
            child{node[circle,scale=0.75,fill]{}[grow=up]
              child{node[circle,scale=0.75,fill,red]{}[grow=up]
              }
            }
          }
        }
        ;
      \end{tikzpicture}
    };
    \node[draw=black] (sn0x11603f0W9) at (9, -80) {\begin{tikzpicture}[scale=.2]
        \node[circle,scale=0.75,fill]{}[grow=up]
        child{node[circle,scale=0.75,fill]{}[grow=up]
          child{node[circle,scale=0.75,fill]{}[grow=up]
            child{node[circle,scale=0.75,fill,red]{}[grow=up]
            }
          }
        }
        ;
      \end{tikzpicture}
    };
    \node[draw=black] (sn0x1160880W9) at (9, -90) {\begin{tikzpicture}[scale=.2]
        \node[circle,scale=0.75,fill]{}[grow=up]
        child{node[circle,scale=0.75,fill]{}[grow=up]
          child{node[circle,scale=0.75,fill,red]{}[grow=up]
          }
        }
        ;
      \end{tikzpicture}
    };
    \node[draw=black] (sn0x1160990W9) at (9, -100) {\begin{tikzpicture}[scale=.2]
        \node[circle,scale=0.75,fill]{}[grow=up]
        child{node[circle,scale=0.75,fill,red]{}[grow=up]
        }
        ;
      \end{tikzpicture}
    };
    \node[draw=black] (sn0x1160ac0W9) at (9, -110) {\begin{tikzpicture}[scale=.2]
        \node[circle,scale=0.75,fill]{}[grow=up]
        ;
      \end{tikzpicture}
    };
    \draw (sn0x1160990W9.south) -- (sn0x1160ac0W9.north);
    \draw (sn0x1160880W9.south) -- (sn0x1160990W9.north);
    \draw (sn0x11603f0W9.south) -- (sn0x1160880W9.north);
    \draw (sn0x1160260W9.south) -- (sn0x11603f0W9.north);
    \node[draw=black] (sn0x1160630W18) at (18, -70) {\begin{tikzpicture}[scale=.2]
        \node[circle,scale=0.75,fill]{}[grow=up]
        child{node[circle,scale=0.75,fill]{}[grow=up]
          child{node[circle,scale=0.75,fill]{}[grow=up]
            child{node[circle,scale=0.75,fill,red]{}[grow=up]
            }
          }
        }
        child{node[circle,scale=0.75,fill,red]{}[grow=up]
        }
        ;
      \end{tikzpicture}
    };
    \node[draw=black] (sn0x1160f10W18) at (18, -80) {\begin{tikzpicture}[scale=.2]
        \node[circle,scale=0.75,fill]{}[grow=up]
        child{node[circle,scale=0.75,fill]{}[grow=up]
          child{node[circle,scale=0.75,fill,red]{}[grow=up]
          }
        }
        child{node[circle,scale=0.75,fill,red]{}[grow=up]
        }
        ;
      \end{tikzpicture}
    };
    \node[draw=black] (sn0x1161350W18) at (18, -90) {\begin{tikzpicture}[scale=.2]
        \node[circle,scale=0.75,fill]{}[grow=up]
        child{node[circle,scale=0.75,fill,red]{}[grow=up]
        }
        child{node[circle,scale=0.75,fill,red]{}[grow=up]
        }
        ;
      \end{tikzpicture}
    };
    \draw (sn0x1161350W18.south) -- (sn0x1160990W9.north);
    \draw (sn0x1160f10W18.south) -- (sn0x1160880W9.north);
    \draw (sn0x1160f10W18.south) -- (sn0x1161350W18.north);
    \draw (sn0x1160630W18.south) -- (sn0x11603f0W9.north);
    \draw (sn0x1160630W18.south) -- (sn0x1160f10W18.north);
    \draw (sn0x115f500W9.south) -- (sn0x1160260W9.north);
    \draw (sn0x115f500W9.south) -- (sn0x1160630W18.north);
    \node[draw=black] (sn0x115f7f0W18) at (18, -60) {\begin{tikzpicture}[scale=.2]
        \node[circle,scale=0.75,fill]{}[grow=up]
        child{node[circle,scale=0.75,fill]{}[grow=up]
          child{node[circle,scale=0.75,fill]{}[grow=up]
            child{node[circle,scale=0.75,fill,red]{}[grow=up]
            }
          }
        }
        child{node[circle,scale=0.75,fill]{}[grow=up]
          child{node[circle,scale=0.75,fill,red]{}[grow=up]
          }
        }
        ;
      \end{tikzpicture}
    };
    \node[draw=black] (sn0x1161980W27) at (27, -70) {\begin{tikzpicture}[scale=.2]
        \node[circle,scale=0.75,fill]{}[grow=up]
        child{node[circle,scale=0.75,fill]{}[grow=up]
          child{node[circle,scale=0.75,fill,red]{}[grow=up]
          }
        }
        child{node[circle,scale=0.75,fill]{}[grow=up]
          child{node[circle,scale=0.75,fill,red]{}[grow=up]
          }
        }
        ;
      \end{tikzpicture}
    };
    \draw (sn0x1161980W27.south) -- (sn0x1160f10W18.north);
    \draw (sn0x115f7f0W18.south) -- (sn0x1160630W18.north);
    \draw (sn0x115f7f0W18.south) -- (sn0x1161980W27.north);
    \draw (sn0x115f370W9.south) -- (sn0x115f500W9.north);
    \draw (sn0x115f370W9.south) -- (sn0x115f7f0W18.north);
    \node[draw=black] (sn0x115f660W18) at (18, -50) {\begin{tikzpicture}[scale=.2]
        \node[circle,scale=0.75,fill]{}[grow=up]
        child{node[circle,scale=0.75,fill]{}[grow=up]
          child{node[circle,scale=0.75,fill]{}[grow=up]
            child{node[circle,scale=0.75,fill,red]{}[grow=up]
            }
          }
        }
        child{node[circle,scale=0.75,fill]{}[grow=up]
          child{node[circle,scale=0.75,fill,red]{}[grow=up]
          }
          child{node[circle,scale=0.75,fill]{}[grow=up]
          }
        }
        ;
      \end{tikzpicture}
    };
    \node[draw=black] (sn0x1161800W27) at (27, -60) {\begin{tikzpicture}[scale=.2]
        \node[circle,scale=0.75,fill]{}[grow=up]
        child{node[circle,scale=0.75,fill]{}[grow=up]
          child{node[circle,scale=0.75,fill,red]{}[grow=up]
          }
          child{node[circle,scale=0.75,fill]{}[grow=up]
          }
        }
        child{node[circle,scale=0.75,fill]{}[grow=up]
          child{node[circle,scale=0.75,fill,red]{}[grow=up]
          }
        }
        ;
      \end{tikzpicture}
    };
    \node[draw=black] (sn0x11624b0W36) at (36, -70) {\begin{tikzpicture}[scale=.2]
        \node[circle,scale=0.75,fill]{}[grow=up]
        child{node[circle,scale=0.75,fill]{}[grow=up]
          child{node[circle,scale=0.75,fill,red]{}[grow=up]
          }
          child{node[circle,scale=0.75,fill,red]{}[grow=up]
          }
        }
        child{node[circle,scale=0.75,fill]{}[grow=up]
        }
        ;
      \end{tikzpicture}
    };
    \draw (sn0x11624b0W36.south) -- (sn0x1160f10W18.north);
    \draw (sn0x1161800W27.south) -- (sn0x1161980W27.north);
    \draw (sn0x1161800W27.south) -- (sn0x11624b0W36.north);
    \node[draw=black] (sn0x1162160W36) at (36, -60) {\begin{tikzpicture}[scale=.2]
        \node[circle,scale=0.75,fill]{}[grow=up]
        child{node[circle,scale=0.75,fill]{}[grow=up]
          child{node[circle,scale=0.75,fill,red]{}[grow=up]
          }
          child{node[circle,scale=0.75,fill,red]{}[grow=up]
          }
        }
        child{node[circle,scale=0.75,fill]{}[grow=up]
          child{node[circle,scale=0.75,fill]{}[grow=up]
          }
        }
        ;
      \end{tikzpicture}
    };
    \draw (sn0x1162160W36.south) -- (sn0x1161980W27.north);
    \draw (sn0x115f660W18.south) -- (sn0x115f7f0W18.north);
    \draw (sn0x115f660W18.south) -- (sn0x1161800W27.north);
    \draw (sn0x115f660W18.south) -- (sn0x1162160W36.north);
    \draw (sn0x115e9e0W9.south) -- (sn0x115f370W9.north);
    \draw (sn0x115e9e0W9.south) -- (sn0x115f660W18.north);
    \node[draw=black] (sn0x115eed0W18) at (18, -40) {\begin{tikzpicture}[scale=.2]
        \node[circle,scale=0.75,fill]{}[grow=up, sibling distance=3cm]
        child{node[circle,scale=0.75,fill]{}[grow=up, sibling distance=1.5cm]
          child{node[circle,scale=0.75,fill,red]{}[grow=up]
          }
          child{node[circle,scale=0.75,fill]{}[grow=up]
          }
          child{node[circle,scale=0.75,fill]{}[grow=up]
          }
        }
        child{node[circle,scale=0.75,fill]{}[grow=up, sibling distance=1.5cm]
          child{node[circle,scale=0.75,fill]{}[grow=up]
            child{node[circle,scale=0.75,fill,red]{}[grow=up]
            }
          }
        }
        ;
      \end{tikzpicture}
    };
    \node[draw=black] (sn0x1162a20W27) at (27, -50) {\begin{tikzpicture}[scale=.2]
        \node[circle,scale=0.75,fill]{}[grow=up,sibling distance=3cm]
        child{node[circle,scale=0.75,fill]{}[grow=up, sibling distance=1.5cm]
          child{node[circle,scale=0.75,fill,red]{}[grow=up]
          }
          child{node[circle,scale=0.75,fill]{}[grow=up]
          }
          child{node[circle,scale=0.75,fill]{}[grow=up]
          }
        }
        child{node[circle,scale=0.75,fill]{}[grow=up,sibling distance=1.5cm]
          child{node[circle,scale=0.75,fill,red]{}[grow=up]
          }
        }
        ;
      \end{tikzpicture}
    };
    \node[draw=black] (sn0x11632c0W45) at (45, -60) {\begin{tikzpicture}[scale=.2]
        \node[circle,scale=0.75,fill]{}[grow=up]
        child{node[circle,scale=0.75,fill]{}[grow=up]
          child{node[circle,scale=0.75,fill,red]{}[grow=up]
          }
          child{node[circle,scale=0.75,fill,red]{}[grow=up]
          }
          child{node[circle,scale=0.75,fill]{}[grow=up]
          }
        }
        child{node[circle,scale=0.75,fill]{}[grow=up]
        }
        ;
      \end{tikzpicture}
    };
    \draw (sn0x11632c0W45.south) -- (sn0x11624b0W36.north);
    \draw (sn0x1162a20W27.south) -- (sn0x1161800W27.north);
    \draw (sn0x1162a20W27.south) -- (sn0x11632c0W45.north);
    \node[draw=black] (sn0x1163090W36) at (36, -50) {\begin{tikzpicture}[scale=.2]
        \node[circle,scale=0.75,fill]{}[grow=up,sibling distance=3cm]
        child{node[circle,scale=0.75,fill]{}[grow=up, sibling distance=1.5cm]
          child{node[circle,scale=0.75,fill,red]{}[grow=up]
          }
          child{node[circle,scale=0.75,fill,red]{}[grow=up]
          }
          child{node[circle,scale=0.75,fill]{}[grow=up]
          }
        }
        child{node[circle,scale=0.75,fill]{}[grow=up,sibling distance=1.5cm]
          child{node[circle,scale=0.75,fill]{}[grow=up]
          }
        }
        ;
      \end{tikzpicture}
    };
    \draw (sn0x1163090W36.south) -- (sn0x1162160W36.north);
    \draw (sn0x1163090W36.south) -- (sn0x1161800W27.north);
    \draw (sn0x115eed0W18.south) -- (sn0x115f660W18.north);
    \draw (sn0x115eed0W18.south) -- (sn0x1162a20W27.north);
    \draw (sn0x115eed0W18.south) -- (sn0x1163090W36.north);
    \draw (sn0x115e1f0W9.south) -- (sn0x115e9e0W9.north);
    \draw (sn0x115e1f0W9.south) -- (sn0x115eed0W18.north);
    \node[draw=black] (sn0x115e5a0W18) at (18, -30) {\begin{tikzpicture}[scale=.2]
        \node[circle,scale=0.75,fill]{}[grow=up, sibling distance=3cm]
        child{node[circle,scale=0.75,fill]{}[grow=up, sibling distance=1.5cm]
          child{node[circle,scale=0.75,fill]{}[grow=up]
            child{node[circle,scale=0.75,fill,red]{}[grow=up]
            }
          }
          child{node[circle,scale=0.75,fill]{}[grow=up]
          }
          child{node[circle,scale=0.75,fill]{}[grow=up]
          }
        }
        child{node[circle,scale=0.75,fill]{}[grow=up]
          child{node[circle,scale=0.75,fill]{}[grow=up]
            child{node[circle,scale=0.75,fill,red]{}[grow=up]
            }
          }
        }
        ;
      \end{tikzpicture}
    };
    \node[draw=black] (sn0x1163900W27) at (27, -40) {\begin{tikzpicture}[scale=.2]
        \node[circle,scale=0.75,fill]{}[grow=up]
        child{node[circle,scale=0.75,fill]{}[grow=up]
          child{node[circle,scale=0.75,fill]{}[grow=up]
            child{node[circle,scale=0.75,fill,red]{}[grow=up]
            }
          }
          child{node[circle,scale=0.75,fill]{}[grow=up]
          }
          child{node[circle,scale=0.75,fill]{}[grow=up]
          }
        }
        child{node[circle,scale=0.75,fill]{}[grow=up]
          child{node[circle,scale=0.75,fill,red]{}[grow=up]
          }
        }
        ;
      \end{tikzpicture}
    };
    \node[draw=black] (sn0x1164010W45) at (45, -50) {\begin{tikzpicture}[scale=.2]
        \node[circle,scale=0.75,fill]{}[grow=up]
        child{node[circle,scale=0.75,fill]{}[grow=up]
          child{node[circle,scale=0.75,fill]{}[grow=up]
            child{node[circle,scale=0.75,fill,red]{}[grow=up]
            }
          }
          child{node[circle,scale=0.75,fill,red]{}[grow=up]
          }
          child{node[circle,scale=0.75,fill]{}[grow=up]
          }
        }
        child{node[circle,scale=0.75,fill]{}[grow=up]
        }
        ;
      \end{tikzpicture}
    };
    \node[draw=black] (sn0x11648d0W54) at (54, -60) {\begin{tikzpicture}[scale=.2]
        \node[circle,scale=0.75,fill]{}[grow=up]
        child{node[circle,scale=0.75,fill]{}[grow=up]
          child{node[circle,scale=0.75,fill]{}[grow=up]
            child{node[circle,scale=0.75,fill,red]{}[grow=up]
            }
          }
          child{node[circle,scale=0.75,fill,red]{}[grow=up]
          }
        }
        child{node[circle,scale=0.75,fill]{}[grow=up]
        }
        ;
      \end{tikzpicture}
    };
    \draw (sn0x11648d0W54.south) -- (sn0x1160630W18.north);
    \draw (sn0x11648d0W54.south) -- (sn0x11624b0W36.north);
    \draw (sn0x1164010W45.south) -- (sn0x11648d0W54.north);
    \draw (sn0x1164010W45.south) -- (sn0x11632c0W45.north);
    \draw (sn0x1163900W27.south) -- (sn0x1164010W45.north);
    \draw (sn0x1163900W27.south) -- (sn0x1162a20W27.north);
    \node[draw=black] (sn0x1163b50W36) at (36, -40) {\begin{tikzpicture}[scale=.2]
        \node[circle,scale=0.75,fill]{}[grow=up]
        child{node[circle,scale=0.75,fill]{}[grow=up]
          child{node[circle,scale=0.75,fill]{}[grow=up]
            child{node[circle,scale=0.75,fill,red]{}[grow=up]
            }
          }
          child{node[circle,scale=0.75,fill,red]{}[grow=up]
          }
          child{node[circle,scale=0.75,fill]{}[grow=up]
          }
        }
        child{node[circle,scale=0.75,fill]{}[grow=up]
          child{node[circle,scale=0.75,fill]{}[grow=up]
          }
        }
        ;
      \end{tikzpicture}
    };
    \node[draw=black] (sn0x11641c0W54) at (54, -50) {\begin{tikzpicture}[scale=.2]
        \node[circle,scale=0.75,fill]{}[grow=up]
        child{node[circle,scale=0.75,fill]{}[grow=up]
          child{node[circle,scale=0.75,fill]{}[grow=up]
            child{node[circle,scale=0.75,fill,red]{}[grow=up]
            }
          }
          child{node[circle,scale=0.75,fill,red]{}[grow=up]
          }
        }
        child{node[circle,scale=0.75,fill]{}[grow=up]
          child{node[circle,scale=0.75,fill]{}[grow=up]
          }
        }
        ;
      \end{tikzpicture}
    };
    \draw (sn0x11641c0W54.south) -- (sn0x115f7f0W18.north);
    \draw (sn0x11641c0W54.south) -- (sn0x1162160W36.north);
    \draw (sn0x11641c0W54.south) -- (sn0x1161800W27.north);
    \node[draw=black] (sn0x11650c0W63) at (63, -50) {\begin{tikzpicture}[scale=.2]
        \node[circle,scale=0.75,fill]{}[grow=up]
        child{node[circle,scale=0.75,fill]{}[grow=up]
          child{node[circle,scale=0.75,fill]{}[grow=up]
            child{node[circle,scale=0.75,fill,red]{}[grow=up]
            }
          }
          child{node[circle,scale=0.75,fill]{}[grow=up]
          }
        }
        child{node[circle,scale=0.75,fill]{}[grow=up]
          child{node[circle,scale=0.75,fill,red]{}[grow=up]
          }
        }
        ;
      \end{tikzpicture}
    };
    \draw (sn0x11650c0W63.south) -- (sn0x11648d0W54.north);
    \draw (sn0x11650c0W63.south) -- (sn0x1161800W27.north);
    \draw (sn0x1163b50W36.south) -- (sn0x11641c0W54.north);
    \draw (sn0x1163b50W36.south) -- (sn0x11650c0W63.north);
    \draw (sn0x1163b50W36.south) -- (sn0x1163090W36.north);
    \draw (sn0x1163b50W36.south) -- (sn0x1162a20W27.north);
    \draw (sn0x115e5a0W18.south) -- (sn0x1163900W27.north);
    \draw (sn0x115e5a0W18.south) -- (sn0x1163b50W36.north);
    \draw (sn0x115e5a0W18.south) -- (sn0x115eed0W18.north);
    \draw (sn0x115c080W9.south) -- (sn0x115e1f0W9.north);
    \draw (sn0x115c080W9.south) -- (sn0x115e5a0W18.north);
    \node[draw=black] (sn0x115c150W18) at (18, -20) {\begin{tikzpicture}[scale=.2]
        \node[circle,scale=0.75,fill]{}[grow=up,sibling distance=3cm]
        child{node[circle,scale=0.75,fill]{}[grow=up,sibling distance=1.5cm]
          child{node[circle,scale=0.75,fill]{}[grow=up]
            child{node[circle,scale=0.75,fill]{}[grow=up]
            }
          }
          child{node[circle,scale=0.75,fill]{}[grow=up]
          }
          child{node[circle,scale=0.75,fill]{}[grow=up]
          }
        }
        child{node[circle,scale=0.75,fill]{}[grow=up, sibling distance=1.5cm]
          child{node[circle,scale=0.75,fill]{}[grow=up]
            child{node[circle,scale=0.75,fill,red]{}[grow=up]
            }
            child{node[circle,scale=0.75,fill,red]{}[grow=up]
            }
          }
        }
        ;
      \end{tikzpicture}
    };
    \draw (sn0x115c150W18.south) -- (sn0x115e5a0W18.north);
    \node[draw=black] (sn0x115dad0W27) at (27, -20) {\begin{tikzpicture}[scale=.2]
        \node[circle,scale=0.75,fill]{}[grow=up,sibling distance=3cm]
        child{node[circle,scale=0.75,fill]{}[grow=up, sibling distance=1.5cm]
          child{node[circle,scale=0.75,fill]{}[grow=up]
            child{node[circle,scale=0.75,fill,red]{}[grow=up]
            }
          }
          child{node[circle,scale=0.75,fill]{}[grow=up]
          }
          child{node[circle,scale=0.75,fill]{}[grow=up]
          }
        }
        child{node[circle,scale=0.75,fill]{}[grow=up,sibling distance=1.5cm]
          child{node[circle,scale=0.75,fill]{}[grow=up]
            child{node[circle,scale=0.75,fill,red]{}[grow=up]
            }
            child{node[circle,scale=0.75,fill]{}[grow=up]
            }
          }
        }
        ;
      \end{tikzpicture}
    };
    \node[draw=black] (sn0x1165740W27) at (27, -30) {\begin{tikzpicture}[scale=.2]
        \node[circle,scale=0.75,fill]{}[grow=up, sibling distance=3cm]
        child{node[circle,scale=0.75,fill]{}[grow=up, sibling distance=1.5cm]
          child{node[circle,scale=0.75,fill]{}[grow=up]
            child{node[circle,scale=0.75,fill,red]{}[grow=up]
            }
            child{node[circle,scale=0.75,fill,red]{}[grow=up]
            }
          }
        }
        child{node[circle,scale=0.75,fill]{}[grow=up, sibling distance=1.5cm]
          child{node[circle,scale=0.75,fill]{}[grow=up]
          }
          child{node[circle,scale=0.75,fill]{}[grow=up]
          }
          child{node[circle,scale=0.75,fill]{}[grow=up]
          }
        }
        ;
      \end{tikzpicture}
    };
    \draw (sn0x1165740W27.south) -- (sn0x115eed0W18.north);
    \draw (sn0x115dad0W27.south) -- (sn0x115e5a0W18.north);
    \draw (sn0x115dad0W27.south) -- (sn0x1165740W27.north);
    \draw (sn0x115b5e0W9.south) -- (sn0x115c080W9.north);
    \draw (sn0x115b5e0W9.south) -- (sn0x115c150W18.north);
    \draw (sn0x115b5e0W9.south) -- (sn0x115dad0W27.north);
  \end{scope}
  \newcommand{\shortnode}[4]{
    \node[draw=black] (s#1#2) at (#3, #4){$#1/#2$};
  }
  \begin{scope}[xshift=70cm,xscale=0.5]
    \shortnode{5}{6}{0}{-10};

    \shortnode{5}{5}{0}{-20};
    \shortnode{4}{6}{10}{-20};
    \draw[->] (s56) -- (s55);
    \draw[->] (s56) -- (s46);

    \shortnode{5}{4}{0}{-30};
    \shortnode{4}{5}{10}{-30};
    \draw[->] (s55) -- (s54);
    \draw[->] (s55) -- (s45);
    \draw[->] (s46) -- (s45);

    \shortnode{5}{3}{0}{-40};
    \shortnode{4}{4}{10}{-40};
    \draw[->] (s54) -- (s53);
    \draw[->] (s54) -- (s44);
    \draw[->] (s45) -- (s44);

    \shortnode{5}{2}{0}{-50};
    \shortnode{4}{3}{10}{-50};
    \shortnode{3}{4}{20}{-50};
    \draw[->] (s53) -- (s52);
    \draw[->] (s53) -- (s43);
    \draw[->] (s44) -- (s43);
    \draw[->] (s44) -- (s34);
    
    \shortnode{5}{1}{0}{-60};
    \shortnode{4}{2}{10}{-60};
    \shortnode{3}{3}{20}{-60};
    \draw[->] (s52) -- (s51);
    \draw[->] (s52) -- (s42);
    \draw[->] (s43) -- (s42);
    \draw[->] (s43) -- (s33);
    \draw[->] (s34) -- (s33);

    \shortnode{5}{0}{0}{-70};
    \shortnode{4}{1}{10}{-70};
    \shortnode{3}{2}{20}{-70};
    \draw[->] (s51) -- (s50);
    \draw[->] (s51) -- (s41);
    \draw[->] (s42) -- (s41);
    \draw[->] (s42) -- (s32);
    \draw[->] (s33) -- (s32);

    \shortnode{4}{0}{0}{-80};
    \shortnode{3}{1}{10}{-80};
    \draw[->] (s50) -- (s40);
    \draw[->] (s41) -- (s40);
    \draw[->] (s41) -- (s31);
    \draw[->] (s32) -- (s31);

    \shortnode{3}{0}{0}{-90};
    \shortnode{2}{1}{10}{-90};
    \draw[->] (s40) -- (s30);
    \draw[->] (s31) -- (s30);
    \draw[->] (s31) -- (s21);

    \shortnode{2}{0}{0}{-100};
    \draw[->] (s30) -- (s20);
    \draw[->] (s21) -- (s20);

    \shortnode{1}{0}{0}{-110};
    \draw[->] (s20) -- (s10);
  \end{scope}
\end{tikzpicture}

%%% Local Variables:
%%% TeX-master: "../../thesis.tex"
%%% End:
%\input{../maxtrees}
\renewcommand{\leveltopI}{-10cm + \leveltop}
\renewcommand{\leveltopII}{-10cm + \leveltopI}
\renewcommand{\leveltopIII}{-10cm + \leveltopII}
\renewcommand{\leveltopIIII}{-10cm + \leveltopIII}
\renewcommand{\leveltopIIIII}{-10cm + \leveltopIIII}
\renewcommand{\leveltopIIIIII}{-10cm + \leveltopIIIII}
\renewcommand{\leveltopIIIIIII}{-10cm + \leveltopIIIIII}
\renewcommand{\leveltopIIIIIIII}{-10cm + \leveltopIIIIIII}
\renewcommand{\leveltopIIIIIIIII}{-10cm + \leveltopIIIIIIII}
\renewcommand{\leveltopIIIIIIIIII}{-10cm + \leveltopIIIIIIIII}
\renewcommand{\leveltopIIIIIIIIIII}{-10cm + \leveltopIIIIIIIIII}
\begin{tikzpicture}[scale=.2, anchor=south, rotate=90]
\begin{scope}[yshift=\leveltopI cm, anchor = center]
\matrix (line1)[row sep=0.5cm] {
\node[draw=black, rectangle split,  rectangle split parts=4] (sn0x22a7ba0){
\footnotesize{100}
\nodepart{two}
\begin{tikzpicture}[scale=.2]
\node[circle, scale=0.75, fill] (tid0) at (4.5,1.5){};
\node[circle, scale=0.75, fill] (tid1) at (2.25,3){};
\node[circle, scale=0.75, fill] (tid4) at (0.75,4.5){};
\node[circle, scale=0.75, fill] (tid5) at (2.25,4.5){};
\node[circle, scale=0.75, fill] (tid6) at (3.75,4.5){};
\draw[](tid1) -- (tid4);
\draw[](tid1) -- (tid5);
\draw[](tid1) -- (tid6);
\node[circle, scale=0.75, fill] (tid2) at (6,3){};
\node[circle, scale=0.75, fill] (tid7) at (5.25,4.5){};
\node[circle, scale=0.75, fill, task_scheduled] (tid10) at (5.25,6){};
\draw[](tid7) -- (tid10);
\node[circle, scale=0.75, fill, task_scheduled] (tid8) at (6.75,4.5){};
\draw[](tid2) -- (tid7);
\draw[](tid2) -- (tid8);
\node[circle, scale=0.75, fill] (tid3) at (8.25,3){};
\node[circle, scale=0.75, fill] (tid9) at (8.25,4.5){};
\draw[](tid3) -- (tid9);
\draw[](tid0) -- (tid1);
\draw[](tid0) -- (tid2);
\draw[](tid0) -- (tid3);
\end{tikzpicture}
\nodepart{three}
\footnotesize{6.63281}
\nodepart{four}
\footnotesize{$38\:12\:30\:10\:10$}
};
 \\ 
\\
};
\end{scope}
\begin{scope}[yshift=\leveltopII cm, anchor = center]
\matrix (line2)[row sep=0.5cm] {
\node[draw=black, rectangle split,  rectangle split parts=4] (sn0x22ab160){
\footnotesize{37.5}
\nodepart{two}
\begin{tikzpicture}[scale=.2]
\node[circle, scale=0.75, fill] (tid0) at (3.75,1.5){};
\node[circle, scale=0.75, fill] (tid1) at (2.25,3){};
\node[circle, scale=0.75, fill, task_scheduled] (tid4) at (0.75,4.5){};
\node[circle, scale=0.75, fill] (tid5) at (2.25,4.5){};
\node[circle, scale=0.75, fill] (tid6) at (3.75,4.5){};
\draw[](tid1) -- (tid4);
\draw[](tid1) -- (tid5);
\draw[](tid1) -- (tid6);
\node[circle, scale=0.75, fill] (tid2) at (5.25,3){};
\node[circle, scale=0.75, fill] (tid7) at (5.25,4.5){};
\node[circle, scale=0.75, fill, task_scheduled] (tid9) at (5.25,6){};
\draw[](tid7) -- (tid9);
\draw[](tid2) -- (tid7);
\node[circle, scale=0.75, fill] (tid3) at (6.75,3){};
\node[circle, scale=0.75, fill] (tid8) at (6.75,4.5){};
\draw[](tid3) -- (tid8);
\draw[](tid0) -- (tid1);
\draw[](tid0) -- (tid2);
\draw[](tid0) -- (tid3);
\end{tikzpicture}
\nodepart{three}
\footnotesize{6.14062}
\nodepart{four}
\footnotesize{$25\:25\:33\:17$}
};
 \\ 
\node[draw=black, rectangle split,  rectangle split parts=4] (sn0x22aaaa0){
\footnotesize{12.5}
\nodepart{two}
\begin{tikzpicture}[scale=.2]
\node[circle, scale=0.75, fill] (tid0) at (3.75,1.5){};
\node[circle, scale=0.75, fill] (tid1) at (2.25,3){};
\node[circle, scale=0.75, fill] (tid4) at (0.75,4.5){};
\node[circle, scale=0.75, fill] (tid5) at (2.25,4.5){};
\node[circle, scale=0.75, fill] (tid6) at (3.75,4.5){};
\draw[](tid1) -- (tid4);
\draw[](tid1) -- (tid5);
\draw[](tid1) -- (tid6);
\node[circle, scale=0.75, fill] (tid2) at (5.25,3){};
\node[circle, scale=0.75, fill] (tid7) at (5.25,4.5){};
\node[circle, scale=0.75, fill, task_scheduled] (tid9) at (5.25,6){};
\draw[](tid7) -- (tid9);
\draw[](tid2) -- (tid7);
\node[circle, scale=0.75, fill] (tid3) at (6.75,3){};
\node[circle, scale=0.75, fill, task_scheduled] (tid8) at (6.75,4.5){};
\draw[](tid3) -- (tid8);
\draw[](tid0) -- (tid1);
\draw[](tid0) -- (tid2);
\draw[](tid0) -- (tid3);
\end{tikzpicture}
\nodepart{three}
\footnotesize{6.14062}
\nodepart{four}
\footnotesize{$50\:38\:12$}
};
 \\ 
\node[draw=black, rectangle split,  rectangle split parts=4] (sn0x22ab2c0){
\footnotesize{30}
\nodepart{two}
\begin{tikzpicture}[scale=.2]
\node[circle, scale=0.75, fill] (tid0) at (4.5,1.5){};
\node[circle, scale=0.75, fill] (tid1) at (2.25,3){};
\node[circle, scale=0.75, fill, task_scheduled] (tid4) at (0.75,4.5){};
\node[circle, scale=0.75, fill] (tid5) at (2.25,4.5){};
\node[circle, scale=0.75, fill] (tid6) at (3.75,4.5){};
\draw[](tid1) -- (tid4);
\draw[](tid1) -- (tid5);
\draw[](tid1) -- (tid6);
\node[circle, scale=0.75, fill] (tid2) at (6,3){};
\node[circle, scale=0.75, fill, task_scheduled] (tid7) at (5.25,4.5){};
\node[circle, scale=0.75, fill] (tid8) at (6.75,4.5){};
\draw[](tid2) -- (tid7);
\draw[](tid2) -- (tid8);
\node[circle, scale=0.75, fill] (tid3) at (8.25,3){};
\node[circle, scale=0.75, fill] (tid9) at (8.25,4.5){};
\draw[](tid3) -- (tid9);
\draw[](tid0) -- (tid1);
\draw[](tid0) -- (tid2);
\draw[](tid0) -- (tid3);
\end{tikzpicture}
\nodepart{three}
\footnotesize{6.125}
\nodepart{four}
\footnotesize{$25\:25\:25\:12\:12$}
};
 \\ 
\node[draw=black, rectangle split,  rectangle split parts=4] (sn0x22aba90){
\footnotesize{10}
\nodepart{two}
\begin{tikzpicture}[scale=.2]
\node[circle, scale=0.75, fill] (tid0) at (4.5,1.5){};
\node[circle, scale=0.75, fill] (tid1) at (2.25,3){};
\node[circle, scale=0.75, fill] (tid4) at (0.75,4.5){};
\node[circle, scale=0.75, fill] (tid5) at (2.25,4.5){};
\node[circle, scale=0.75, fill] (tid6) at (3.75,4.5){};
\draw[](tid1) -- (tid4);
\draw[](tid1) -- (tid5);
\draw[](tid1) -- (tid6);
\node[circle, scale=0.75, fill] (tid2) at (6,3){};
\node[circle, scale=0.75, fill, task_scheduled] (tid7) at (5.25,4.5){};
\node[circle, scale=0.75, fill, task_scheduled] (tid8) at (6.75,4.5){};
\draw[](tid2) -- (tid7);
\draw[](tid2) -- (tid8);
\node[circle, scale=0.75, fill] (tid3) at (8.25,3){};
\node[circle, scale=0.75, fill] (tid9) at (8.25,4.5){};
\draw[](tid3) -- (tid9);
\draw[](tid0) -- (tid1);
\draw[](tid0) -- (tid2);
\draw[](tid0) -- (tid3);
\end{tikzpicture}
\nodepart{three}
\footnotesize{6.125}
\nodepart{four}
\footnotesize{$75\:25$}
};
 \\ 
\node[draw=black, rectangle split,  rectangle split parts=4] (sn0x22ab500){
\footnotesize{10}
\nodepart{two}
\begin{tikzpicture}[scale=.2]
\node[circle, scale=0.75, fill] (tid0) at (4.5,1.5){};
\node[circle, scale=0.75, fill] (tid1) at (2.25,3){};
\node[circle, scale=0.75, fill] (tid4) at (0.75,4.5){};
\node[circle, scale=0.75, fill] (tid5) at (2.25,4.5){};
\node[circle, scale=0.75, fill] (tid6) at (3.75,4.5){};
\draw[](tid1) -- (tid4);
\draw[](tid1) -- (tid5);
\draw[](tid1) -- (tid6);
\node[circle, scale=0.75, fill] (tid2) at (6,3){};
\node[circle, scale=0.75, fill, task_scheduled] (tid7) at (5.25,4.5){};
\node[circle, scale=0.75, fill] (tid8) at (6.75,4.5){};
\draw[](tid2) -- (tid7);
\draw[](tid2) -- (tid8);
\node[circle, scale=0.75, fill] (tid3) at (8.25,3){};
\node[circle, scale=0.75, fill, task_scheduled] (tid9) at (8.25,4.5){};
\draw[](tid3) -- (tid9);
\draw[](tid0) -- (tid1);
\draw[](tid0) -- (tid2);
\draw[](tid0) -- (tid3);
\end{tikzpicture}
\nodepart{three}
\footnotesize{6.125}
\nodepart{four}
\footnotesize{$38\:12\:38\:12$}
};
 \\ 
\\
};
\end{scope}
\begin{scope}[yshift=\leveltopIII cm, anchor = center]
\matrix (line3)[row sep=0.5cm] {
\node[draw=black, rectangle split,  rectangle split parts=4] (sn0x22b4a40){
\footnotesize{6.25}
\nodepart{two}
\begin{tikzpicture}[scale=.2]
\node[circle, scale=0.75, fill] (tid0) at (3.75,1.5){};
\node[circle, scale=0.75, fill] (tid1) at (2.25,3){};
\node[circle, scale=0.75, fill, task_scheduled] (tid4) at (0.75,4.5){};
\node[circle, scale=0.75, fill] (tid5) at (2.25,4.5){};
\node[circle, scale=0.75, fill] (tid6) at (3.75,4.5){};
\draw[](tid1) -- (tid4);
\draw[](tid1) -- (tid5);
\draw[](tid1) -- (tid6);
\node[circle, scale=0.75, fill] (tid2) at (5.25,3){};
\node[circle, scale=0.75, fill] (tid7) at (5.25,4.5){};
\node[circle, scale=0.75, fill, task_scheduled] (tid8) at (5.25,6){};
\draw[](tid7) -- (tid8);
\draw[](tid2) -- (tid7);
\node[circle, scale=0.75, fill] (tid3) at (6.75,3){};
\draw[](tid0) -- (tid1);
\draw[](tid0) -- (tid2);
\draw[](tid0) -- (tid3);
\end{tikzpicture}
\nodepart{three}
\footnotesize{5.65625}
\nodepart{four}
\footnotesize{$33\:17\:50$}
};
 \\ 
\node[draw=black, rectangle split,  rectangle split parts=4] (sn0x22b7500){
\footnotesize{3.75}
\nodepart{two}
\begin{tikzpicture}[scale=.2]
\node[circle, scale=0.75, fill] (tid0) at (4.5,1.5){};
\node[circle, scale=0.75, fill] (tid1) at (2.25,3){};
\node[circle, scale=0.75, fill, task_scheduled] (tid4) at (0.75,4.5){};
\node[circle, scale=0.75, fill] (tid5) at (2.25,4.5){};
\node[circle, scale=0.75, fill] (tid6) at (3.75,4.5){};
\draw[](tid1) -- (tid4);
\draw[](tid1) -- (tid5);
\draw[](tid1) -- (tid6);
\node[circle, scale=0.75, fill] (tid2) at (6,3){};
\node[circle, scale=0.75, fill, task_scheduled] (tid7) at (5.25,4.5){};
\node[circle, scale=0.75, fill] (tid8) at (6.75,4.5){};
\draw[](tid2) -- (tid7);
\draw[](tid2) -- (tid8);
\node[circle, scale=0.75, fill] (tid3) at (8.25,3){};
\draw[](tid0) -- (tid1);
\draw[](tid0) -- (tid2);
\draw[](tid0) -- (tid3);
\end{tikzpicture}
\nodepart{three}
\footnotesize{5.625}
\nodepart{four}
\footnotesize{$33\:17\:33\:17$}
};
 \\ 
\node[draw=black, rectangle split,  rectangle split parts=4] (sn0x22b7aa0){
\footnotesize{1.25}
\nodepart{two}
\begin{tikzpicture}[scale=.2]
\node[circle, scale=0.75, fill] (tid0) at (4.5,1.5){};
\node[circle, scale=0.75, fill] (tid1) at (2.25,3){};
\node[circle, scale=0.75, fill] (tid4) at (0.75,4.5){};
\node[circle, scale=0.75, fill] (tid5) at (2.25,4.5){};
\node[circle, scale=0.75, fill] (tid6) at (3.75,4.5){};
\draw[](tid1) -- (tid4);
\draw[](tid1) -- (tid5);
\draw[](tid1) -- (tid6);
\node[circle, scale=0.75, fill] (tid2) at (6,3){};
\node[circle, scale=0.75, fill, task_scheduled] (tid7) at (5.25,4.5){};
\node[circle, scale=0.75, fill, task_scheduled] (tid8) at (6.75,4.5){};
\draw[](tid2) -- (tid7);
\draw[](tid2) -- (tid8);
\node[circle, scale=0.75, fill] (tid3) at (8.25,3){};
\draw[](tid0) -- (tid1);
\draw[](tid0) -- (tid2);
\draw[](tid0) -- (tid3);
\end{tikzpicture}
\nodepart{three}
\footnotesize{5.625}
\nodepart{four}
\footnotesize{$1$}
};
 \\ 
\node[draw=black, rectangle split,  rectangle split parts=4] (sn0x22ac710){
\footnotesize{16.875}
\nodepart{two}
\begin{tikzpicture}[scale=.2]
\node[circle, scale=0.75, fill] (tid0) at (3.75,1.5){};
\node[circle, scale=0.75, fill] (tid1) at (2.25,3){};
\node[circle, scale=0.75, fill, task_scheduled] (tid4) at (0.75,4.5){};
\node[circle, scale=0.75, fill, task_scheduled] (tid5) at (2.25,4.5){};
\node[circle, scale=0.75, fill] (tid6) at (3.75,4.5){};
\draw[](tid1) -- (tid4);
\draw[](tid1) -- (tid5);
\draw[](tid1) -- (tid6);
\node[circle, scale=0.75, fill] (tid2) at (5.25,3){};
\node[circle, scale=0.75, fill] (tid7) at (5.25,4.5){};
\draw[](tid2) -- (tid7);
\node[circle, scale=0.75, fill] (tid3) at (6.75,3){};
\node[circle, scale=0.75, fill] (tid8) at (6.75,4.5){};
\draw[](tid3) -- (tid8);
\draw[](tid0) -- (tid1);
\draw[](tid0) -- (tid2);
\draw[](tid0) -- (tid3);
\end{tikzpicture}
\nodepart{three}
\footnotesize{5.625}
\nodepart{four}
\footnotesize{$33\:67$}
};
 \\ 
\node[draw=black, rectangle split,  rectangle split parts=4] (sn0x22ada30){
\footnotesize{32.8125}
\nodepart{two}
\begin{tikzpicture}[scale=.2]
\node[circle, scale=0.75, fill] (tid0) at (3.75,1.5){};
\node[circle, scale=0.75, fill] (tid1) at (2.25,3){};
\node[circle, scale=0.75, fill, task_scheduled] (tid4) at (0.75,4.5){};
\node[circle, scale=0.75, fill] (tid5) at (2.25,4.5){};
\node[circle, scale=0.75, fill] (tid6) at (3.75,4.5){};
\draw[](tid1) -- (tid4);
\draw[](tid1) -- (tid5);
\draw[](tid1) -- (tid6);
\node[circle, scale=0.75, fill] (tid2) at (5.25,3){};
\node[circle, scale=0.75, fill, task_scheduled] (tid7) at (5.25,4.5){};
\draw[](tid2) -- (tid7);
\node[circle, scale=0.75, fill] (tid3) at (6.75,3){};
\node[circle, scale=0.75, fill] (tid8) at (6.75,4.5){};
\draw[](tid3) -- (tid8);
\draw[](tid0) -- (tid1);
\draw[](tid0) -- (tid2);
\draw[](tid0) -- (tid3);
\end{tikzpicture}
\nodepart{three}
\footnotesize{5.625}
\nodepart{four}
\footnotesize{$33\:17\:33\:17$}
};
 \\ 
\node[draw=black, rectangle split,  rectangle split parts=4] (sn0x22b4be0){
\footnotesize{5.3125}
\nodepart{two}
\begin{tikzpicture}[scale=.2]
\node[circle, scale=0.75, fill] (tid0) at (3.75,1.5){};
\node[circle, scale=0.75, fill] (tid1) at (2.25,3){};
\node[circle, scale=0.75, fill] (tid4) at (0.75,4.5){};
\node[circle, scale=0.75, fill] (tid5) at (2.25,4.5){};
\node[circle, scale=0.75, fill] (tid6) at (3.75,4.5){};
\draw[](tid1) -- (tid4);
\draw[](tid1) -- (tid5);
\draw[](tid1) -- (tid6);
\node[circle, scale=0.75, fill] (tid2) at (5.25,3){};
\node[circle, scale=0.75, fill, task_scheduled] (tid7) at (5.25,4.5){};
\draw[](tid2) -- (tid7);
\node[circle, scale=0.75, fill] (tid3) at (6.75,3){};
\node[circle, scale=0.75, fill, task_scheduled] (tid8) at (6.75,4.5){};
\draw[](tid3) -- (tid8);
\draw[](tid0) -- (tid1);
\draw[](tid0) -- (tid2);
\draw[](tid0) -- (tid3);
\end{tikzpicture}
\nodepart{three}
\footnotesize{5.625}
\nodepart{four}
\footnotesize{$1$}
};
 \\ 
\node[draw=black, rectangle split,  rectangle split parts=4] (sn0x22ac4a0){
\footnotesize{12.5}
\nodepart{two}
\begin{tikzpicture}[scale=.2]
\node[circle, scale=0.75, fill] (tid0) at (3,1.5){};
\node[circle, scale=0.75, fill] (tid1) at (1.5,3){};
\node[circle, scale=0.75, fill, task_scheduled] (tid4) at (0.75,4.5){};
\node[circle, scale=0.75, fill] (tid5) at (2.25,4.5){};
\draw[](tid1) -- (tid4);
\draw[](tid1) -- (tid5);
\node[circle, scale=0.75, fill] (tid2) at (3.75,3){};
\node[circle, scale=0.75, fill] (tid6) at (3.75,4.5){};
\node[circle, scale=0.75, fill, task_scheduled] (tid8) at (3.75,6){};
\draw[](tid6) -- (tid8);
\draw[](tid2) -- (tid6);
\node[circle, scale=0.75, fill] (tid3) at (5.25,3){};
\node[circle, scale=0.75, fill] (tid7) at (5.25,4.5){};
\draw[](tid3) -- (tid7);
\draw[](tid0) -- (tid1);
\draw[](tid0) -- (tid2);
\draw[](tid0) -- (tid3);
\end{tikzpicture}
\nodepart{three}
\footnotesize{5.65625}
\nodepart{four}
\footnotesize{$50\:17\:33$}
};
 \\ 
\node[draw=black, rectangle split,  rectangle split parts=4] (sn0x22acd50){
\footnotesize{6.25}
\nodepart{two}
\begin{tikzpicture}[scale=.2]
\node[circle, scale=0.75, fill] (tid0) at (3,1.5){};
\node[circle, scale=0.75, fill] (tid1) at (1.5,3){};
\node[circle, scale=0.75, fill] (tid4) at (0.75,4.5){};
\node[circle, scale=0.75, fill] (tid5) at (2.25,4.5){};
\draw[](tid1) -- (tid4);
\draw[](tid1) -- (tid5);
\node[circle, scale=0.75, fill] (tid2) at (3.75,3){};
\node[circle, scale=0.75, fill] (tid6) at (3.75,4.5){};
\node[circle, scale=0.75, fill, task_scheduled] (tid8) at (3.75,6){};
\draw[](tid6) -- (tid8);
\draw[](tid2) -- (tid6);
\node[circle, scale=0.75, fill] (tid3) at (5.25,3){};
\node[circle, scale=0.75, fill, task_scheduled] (tid7) at (5.25,4.5){};
\draw[](tid3) -- (tid7);
\draw[](tid0) -- (tid1);
\draw[](tid0) -- (tid2);
\draw[](tid0) -- (tid3);
\end{tikzpicture}
\nodepart{three}
\footnotesize{5.65625}
\nodepart{four}
\footnotesize{$50\:33\:17$}
};
 \\ 
\node[draw=black, rectangle split,  rectangle split parts=4] (sn0x22b54f0){
\footnotesize{7.5}
\nodepart{two}
\begin{tikzpicture}[scale=.2]
\node[circle, scale=0.75, fill] (tid0) at (3.75,1.5){};
\node[circle, scale=0.75, fill] (tid1) at (1.5,3){};
\node[circle, scale=0.75, fill, task_scheduled] (tid4) at (0.75,4.5){};
\node[circle, scale=0.75, fill] (tid5) at (2.25,4.5){};
\draw[](tid1) -- (tid4);
\draw[](tid1) -- (tid5);
\node[circle, scale=0.75, fill] (tid2) at (4.5,3){};
\node[circle, scale=0.75, fill, task_scheduled] (tid6) at (3.75,4.5){};
\node[circle, scale=0.75, fill] (tid7) at (5.25,4.5){};
\draw[](tid2) -- (tid6);
\draw[](tid2) -- (tid7);
\node[circle, scale=0.75, fill] (tid3) at (6.75,3){};
\node[circle, scale=0.75, fill] (tid8) at (6.75,4.5){};
\draw[](tid3) -- (tid8);
\draw[](tid0) -- (tid1);
\draw[](tid0) -- (tid2);
\draw[](tid0) -- (tid3);
\end{tikzpicture}
\nodepart{three}
\footnotesize{5.625}
\nodepart{four}
\footnotesize{$67\:33$}
};
 \\ 
\node[draw=black, rectangle split,  rectangle split parts=4] (sn0x22b5d10){
\footnotesize{3.75}
\nodepart{two}
\begin{tikzpicture}[scale=.2]
\node[circle, scale=0.75, fill] (tid0) at (3.75,1.5){};
\node[circle, scale=0.75, fill] (tid1) at (1.5,3){};
\node[circle, scale=0.75, fill, task_scheduled] (tid4) at (0.75,4.5){};
\node[circle, scale=0.75, fill, task_scheduled] (tid5) at (2.25,4.5){};
\draw[](tid1) -- (tid4);
\draw[](tid1) -- (tid5);
\node[circle, scale=0.75, fill] (tid2) at (4.5,3){};
\node[circle, scale=0.75, fill] (tid6) at (3.75,4.5){};
\node[circle, scale=0.75, fill] (tid7) at (5.25,4.5){};
\draw[](tid2) -- (tid6);
\draw[](tid2) -- (tid7);
\node[circle, scale=0.75, fill] (tid3) at (6.75,3){};
\node[circle, scale=0.75, fill] (tid8) at (6.75,4.5){};
\draw[](tid3) -- (tid8);
\draw[](tid0) -- (tid1);
\draw[](tid0) -- (tid2);
\draw[](tid0) -- (tid3);
\end{tikzpicture}
\nodepart{three}
\footnotesize{5.625}
\nodepart{four}
\footnotesize{$67\:33$}
};
 \\ 
\node[draw=black, rectangle split,  rectangle split parts=4] (sn0x22b6220){
\footnotesize{3.75}
\nodepart{two}
\begin{tikzpicture}[scale=.2]
\node[circle, scale=0.75, fill] (tid0) at (3.75,1.5){};
\node[circle, scale=0.75, fill] (tid1) at (1.5,3){};
\node[circle, scale=0.75, fill, task_scheduled] (tid4) at (0.75,4.5){};
\node[circle, scale=0.75, fill] (tid5) at (2.25,4.5){};
\draw[](tid1) -- (tid4);
\draw[](tid1) -- (tid5);
\node[circle, scale=0.75, fill] (tid2) at (4.5,3){};
\node[circle, scale=0.75, fill] (tid6) at (3.75,4.5){};
\node[circle, scale=0.75, fill] (tid7) at (5.25,4.5){};
\draw[](tid2) -- (tid6);
\draw[](tid2) -- (tid7);
\node[circle, scale=0.75, fill] (tid3) at (6.75,3){};
\node[circle, scale=0.75, fill, task_scheduled] (tid8) at (6.75,4.5){};
\draw[](tid3) -- (tid8);
\draw[](tid0) -- (tid1);
\draw[](tid0) -- (tid2);
\draw[](tid0) -- (tid3);
\end{tikzpicture}
\nodepart{three}
\footnotesize{5.625}
\nodepart{four}
\footnotesize{$17\:33\:17\:33$}
};
 \\ 
\\
};
\end{scope}
\begin{scope}[yshift=\leveltopIIII cm, anchor = center]
\matrix (line4)[row sep=0.5cm] {
\node[draw=black, rectangle split,  rectangle split parts=4] (sn0x22ad490){
\footnotesize{6.25}
\nodepart{two}
\begin{tikzpicture}[scale=.2]
\node[circle, scale=0.75, fill] (tid0) at (2.25,1.5){};
\node[circle, scale=0.75, fill] (tid1) at (0.75,3){};
\node[circle, scale=0.75, fill] (tid4) at (0.75,4.5){};
\node[circle, scale=0.75, fill, task_scheduled] (tid7) at (0.75,6){};
\draw[](tid4) -- (tid7);
\draw[](tid1) -- (tid4);
\node[circle, scale=0.75, fill] (tid2) at (2.25,3){};
\node[circle, scale=0.75, fill, task_scheduled] (tid5) at (2.25,4.5){};
\draw[](tid2) -- (tid5);
\node[circle, scale=0.75, fill] (tid3) at (3.75,3){};
\node[circle, scale=0.75, fill] (tid6) at (3.75,4.5){};
\draw[](tid3) -- (tid6);
\draw[](tid0) -- (tid1);
\draw[](tid0) -- (tid2);
\draw[](tid0) -- (tid3);
\end{tikzpicture}
\nodepart{three}
\footnotesize{5.1875}
\nodepart{four}
\footnotesize{$50\:50$}
};
 \\ 
\node[draw=black, rectangle split,  rectangle split parts=4] (sn0x22b34f0){
\footnotesize{14.2708}
\nodepart{two}
\begin{tikzpicture}[scale=.2]
\node[circle, scale=0.75, fill] (tid0) at (3.75,1.5){};
\node[circle, scale=0.75, fill] (tid1) at (2.25,3){};
\node[circle, scale=0.75, fill, task_scheduled] (tid4) at (0.75,4.5){};
\node[circle, scale=0.75, fill, task_scheduled] (tid5) at (2.25,4.5){};
\node[circle, scale=0.75, fill] (tid6) at (3.75,4.5){};
\draw[](tid1) -- (tid4);
\draw[](tid1) -- (tid5);
\draw[](tid1) -- (tid6);
\node[circle, scale=0.75, fill] (tid2) at (5.25,3){};
\node[circle, scale=0.75, fill] (tid7) at (5.25,4.5){};
\draw[](tid2) -- (tid7);
\node[circle, scale=0.75, fill] (tid3) at (6.75,3){};
\draw[](tid0) -- (tid1);
\draw[](tid0) -- (tid2);
\draw[](tid0) -- (tid3);
\end{tikzpicture}
\nodepart{three}
\footnotesize{5.125}
\nodepart{four}
\footnotesize{$50\:50$}
};
 \\ 
\node[draw=black, rectangle split,  rectangle split parts=4] (sn0x22b3cf0){
\footnotesize{13.6979}
\nodepart{two}
\begin{tikzpicture}[scale=.2]
\node[circle, scale=0.75, fill] (tid0) at (3.75,1.5){};
\node[circle, scale=0.75, fill] (tid1) at (2.25,3){};
\node[circle, scale=0.75, fill, task_scheduled] (tid4) at (0.75,4.5){};
\node[circle, scale=0.75, fill] (tid5) at (2.25,4.5){};
\node[circle, scale=0.75, fill] (tid6) at (3.75,4.5){};
\draw[](tid1) -- (tid4);
\draw[](tid1) -- (tid5);
\draw[](tid1) -- (tid6);
\node[circle, scale=0.75, fill] (tid2) at (5.25,3){};
\node[circle, scale=0.75, fill, task_scheduled] (tid7) at (5.25,4.5){};
\draw[](tid2) -- (tid7);
\node[circle, scale=0.75, fill] (tid3) at (6.75,3){};
\draw[](tid0) -- (tid1);
\draw[](tid0) -- (tid2);
\draw[](tid0) -- (tid3);
\end{tikzpicture}
\nodepart{three}
\footnotesize{5.125}
\nodepart{four}
\footnotesize{$50\:50$}
};
 \\ 
\node[draw=black, rectangle split,  rectangle split parts=4] (sn0x22b2b90){
\footnotesize{6.25}
\nodepart{two}
\begin{tikzpicture}[scale=.2]
\node[circle, scale=0.75, fill] (tid0) at (3,1.5){};
\node[circle, scale=0.75, fill] (tid1) at (1.5,3){};
\node[circle, scale=0.75, fill, task_scheduled] (tid4) at (0.75,4.5){};
\node[circle, scale=0.75, fill] (tid5) at (2.25,4.5){};
\draw[](tid1) -- (tid4);
\draw[](tid1) -- (tid5);
\node[circle, scale=0.75, fill] (tid2) at (3.75,3){};
\node[circle, scale=0.75, fill] (tid6) at (3.75,4.5){};
\node[circle, scale=0.75, fill, task_scheduled] (tid7) at (3.75,6){};
\draw[](tid6) -- (tid7);
\draw[](tid2) -- (tid6);
\node[circle, scale=0.75, fill] (tid3) at (5.25,3){};
\draw[](tid0) -- (tid1);
\draw[](tid0) -- (tid2);
\draw[](tid0) -- (tid3);
\end{tikzpicture}
\nodepart{three}
\footnotesize{5.1875}
\nodepart{four}
\footnotesize{$50\:25\:25$}
};
 \\ 
\node[draw=black, rectangle split,  rectangle split parts=4] (sn0x22b66d0){
\footnotesize{1.25}
\nodepart{two}
\begin{tikzpicture}[scale=.2]
\node[circle, scale=0.75, fill] (tid0) at (3.75,1.5){};
\node[circle, scale=0.75, fill] (tid1) at (1.5,3){};
\node[circle, scale=0.75, fill, task_scheduled] (tid4) at (0.75,4.5){};
\node[circle, scale=0.75, fill, task_scheduled] (tid5) at (2.25,4.5){};
\draw[](tid1) -- (tid4);
\draw[](tid1) -- (tid5);
\node[circle, scale=0.75, fill] (tid2) at (4.5,3){};
\node[circle, scale=0.75, fill] (tid6) at (3.75,4.5){};
\node[circle, scale=0.75, fill] (tid7) at (5.25,4.5){};
\draw[](tid2) -- (tid6);
\draw[](tid2) -- (tid7);
\node[circle, scale=0.75, fill] (tid3) at (6.75,3){};
\draw[](tid0) -- (tid1);
\draw[](tid0) -- (tid2);
\draw[](tid0) -- (tid3);
\end{tikzpicture}
\nodepart{three}
\footnotesize{5.125}
\nodepart{four}
\footnotesize{$1$}
};
 \\ 
\node[draw=black, rectangle split,  rectangle split parts=4] (sn0x22b6c20){
\footnotesize{2.5}
\nodepart{two}
\begin{tikzpicture}[scale=.2]
\node[circle, scale=0.75, fill] (tid0) at (3.75,1.5){};
\node[circle, scale=0.75, fill] (tid1) at (1.5,3){};
\node[circle, scale=0.75, fill, task_scheduled] (tid4) at (0.75,4.5){};
\node[circle, scale=0.75, fill] (tid5) at (2.25,4.5){};
\draw[](tid1) -- (tid4);
\draw[](tid1) -- (tid5);
\node[circle, scale=0.75, fill] (tid2) at (4.5,3){};
\node[circle, scale=0.75, fill, task_scheduled] (tid6) at (3.75,4.5){};
\node[circle, scale=0.75, fill] (tid7) at (5.25,4.5){};
\draw[](tid2) -- (tid6);
\draw[](tid2) -- (tid7);
\node[circle, scale=0.75, fill] (tid3) at (6.75,3){};
\draw[](tid0) -- (tid1);
\draw[](tid0) -- (tid2);
\draw[](tid0) -- (tid3);
\end{tikzpicture}
\nodepart{three}
\footnotesize{5.125}
\nodepart{four}
\footnotesize{$50\:50$}
};
 \\ 
\node[draw=black, rectangle split,  rectangle split parts=4] (sn0x22ade60){
\footnotesize{10.2083}
\nodepart{two}
\begin{tikzpicture}[scale=.2]
\node[circle, scale=0.75, fill] (tid0) at (3,1.5){};
\node[circle, scale=0.75, fill] (tid1) at (1.5,3){};
\node[circle, scale=0.75, fill, task_scheduled] (tid4) at (0.75,4.5){};
\node[circle, scale=0.75, fill, task_scheduled] (tid5) at (2.25,4.5){};
\draw[](tid1) -- (tid4);
\draw[](tid1) -- (tid5);
\node[circle, scale=0.75, fill] (tid2) at (3.75,3){};
\node[circle, scale=0.75, fill] (tid6) at (3.75,4.5){};
\draw[](tid2) -- (tid6);
\node[circle, scale=0.75, fill] (tid3) at (5.25,3){};
\node[circle, scale=0.75, fill] (tid7) at (5.25,4.5){};
\draw[](tid3) -- (tid7);
\draw[](tid0) -- (tid1);
\draw[](tid0) -- (tid2);
\draw[](tid0) -- (tid3);
\end{tikzpicture}
\nodepart{three}
\footnotesize{5.125}
\nodepart{four}
\footnotesize{$1$}
};
 \\ 
\node[draw=black, rectangle split,  rectangle split parts=4] (sn0x22ae720){
\footnotesize{37.1875}
\nodepart{two}
\begin{tikzpicture}[scale=.2]
\node[circle, scale=0.75, fill] (tid0) at (3,1.5){};
\node[circle, scale=0.75, fill] (tid1) at (1.5,3){};
\node[circle, scale=0.75, fill, task_scheduled] (tid4) at (0.75,4.5){};
\node[circle, scale=0.75, fill] (tid5) at (2.25,4.5){};
\draw[](tid1) -- (tid4);
\draw[](tid1) -- (tid5);
\node[circle, scale=0.75, fill] (tid2) at (3.75,3){};
\node[circle, scale=0.75, fill, task_scheduled] (tid6) at (3.75,4.5){};
\draw[](tid2) -- (tid6);
\node[circle, scale=0.75, fill] (tid3) at (5.25,3){};
\node[circle, scale=0.75, fill] (tid7) at (5.25,4.5){};
\draw[](tid3) -- (tid7);
\draw[](tid0) -- (tid1);
\draw[](tid0) -- (tid2);
\draw[](tid0) -- (tid3);
\end{tikzpicture}
\nodepart{three}
\footnotesize{5.125}
\nodepart{four}
\footnotesize{$25\:25\:50$}
};
 \\ 
\node[draw=black, rectangle split,  rectangle split parts=4] (sn0x22b2d60){
\footnotesize{8.38542}
\nodepart{two}
\begin{tikzpicture}[scale=.2]
\node[circle, scale=0.75, fill] (tid0) at (3,1.5){};
\node[circle, scale=0.75, fill] (tid1) at (1.5,3){};
\node[circle, scale=0.75, fill] (tid4) at (0.75,4.5){};
\node[circle, scale=0.75, fill] (tid5) at (2.25,4.5){};
\draw[](tid1) -- (tid4);
\draw[](tid1) -- (tid5);
\node[circle, scale=0.75, fill] (tid2) at (3.75,3){};
\node[circle, scale=0.75, fill, task_scheduled] (tid6) at (3.75,4.5){};
\draw[](tid2) -- (tid6);
\node[circle, scale=0.75, fill] (tid3) at (5.25,3){};
\node[circle, scale=0.75, fill, task_scheduled] (tid7) at (5.25,4.5){};
\draw[](tid3) -- (tid7);
\draw[](tid0) -- (tid1);
\draw[](tid0) -- (tid2);
\draw[](tid0) -- (tid3);
\end{tikzpicture}
\nodepart{three}
\footnotesize{5.125}
\nodepart{four}
\footnotesize{$1$}
};
 \\ 
\\
};
\end{scope}
\begin{scope}[yshift=\leveltopIIIII cm, anchor = center]
\matrix (line5)[row sep=0.5cm] {
\node[draw=black, rectangle split,  rectangle split parts=4] (sn0x22ae570){
\footnotesize{6.25}
\nodepart{two}
\begin{tikzpicture}[scale=.2]
\node[circle, scale=0.75, fill] (tid0) at (2.25,1.5){};
\node[circle, scale=0.75, fill] (tid1) at (0.75,3){};
\node[circle, scale=0.75, fill] (tid4) at (0.75,4.5){};
\node[circle, scale=0.75, fill, task_scheduled] (tid6) at (0.75,6){};
\draw[](tid4) -- (tid6);
\draw[](tid1) -- (tid4);
\node[circle, scale=0.75, fill] (tid2) at (2.25,3){};
\node[circle, scale=0.75, fill, task_scheduled] (tid5) at (2.25,4.5){};
\draw[](tid2) -- (tid5);
\node[circle, scale=0.75, fill] (tid3) at (3.75,3){};
\draw[](tid0) -- (tid1);
\draw[](tid0) -- (tid2);
\draw[](tid0) -- (tid3);
\end{tikzpicture}
\nodepart{three}
\footnotesize{4.75}
\nodepart{four}
\footnotesize{$50\:50$}
};
 \\ 
\node[draw=black, rectangle split,  rectangle split parts=4] (sn0x22b40d0){
\footnotesize{6.84896}
\nodepart{two}
\begin{tikzpicture}[scale=.2]
\node[circle, scale=0.75, fill] (tid0) at (3.75,1.5){};
\node[circle, scale=0.75, fill] (tid1) at (2.25,3){};
\node[circle, scale=0.75, fill, task_scheduled] (tid4) at (0.75,4.5){};
\node[circle, scale=0.75, fill, task_scheduled] (tid5) at (2.25,4.5){};
\node[circle, scale=0.75, fill] (tid6) at (3.75,4.5){};
\draw[](tid1) -- (tid4);
\draw[](tid1) -- (tid5);
\draw[](tid1) -- (tid6);
\node[circle, scale=0.75, fill] (tid2) at (5.25,3){};
\node[circle, scale=0.75, fill] (tid3) at (6.75,3){};
\draw[](tid0) -- (tid1);
\draw[](tid0) -- (tid2);
\draw[](tid0) -- (tid3);
\end{tikzpicture}
\nodepart{three}
\footnotesize{4.625}
\nodepart{four}
\footnotesize{$1$}
};
 \\ 
\node[draw=black, rectangle split,  rectangle split parts=4] (sn0x22b14e0){
\footnotesize{19.2448}
\nodepart{two}
\begin{tikzpicture}[scale=.2]
\node[circle, scale=0.75, fill] (tid0) at (3,1.5){};
\node[circle, scale=0.75, fill] (tid1) at (1.5,3){};
\node[circle, scale=0.75, fill, task_scheduled] (tid4) at (0.75,4.5){};
\node[circle, scale=0.75, fill, task_scheduled] (tid5) at (2.25,4.5){};
\draw[](tid1) -- (tid4);
\draw[](tid1) -- (tid5);
\node[circle, scale=0.75, fill] (tid2) at (3.75,3){};
\node[circle, scale=0.75, fill] (tid6) at (3.75,4.5){};
\draw[](tid2) -- (tid6);
\node[circle, scale=0.75, fill] (tid3) at (5.25,3){};
\draw[](tid0) -- (tid1);
\draw[](tid0) -- (tid2);
\draw[](tid0) -- (tid3);
\end{tikzpicture}
\nodepart{three}
\footnotesize{4.625}
\nodepart{four}
\footnotesize{$1$}
};
 \\ 
\node[draw=black, rectangle split,  rectangle split parts=4] (sn0x22b1ca0){
\footnotesize{35.7292}
\nodepart{two}
\begin{tikzpicture}[scale=.2]
\node[circle, scale=0.75, fill] (tid0) at (3,1.5){};
\node[circle, scale=0.75, fill] (tid1) at (1.5,3){};
\node[circle, scale=0.75, fill, task_scheduled] (tid4) at (0.75,4.5){};
\node[circle, scale=0.75, fill] (tid5) at (2.25,4.5){};
\draw[](tid1) -- (tid4);
\draw[](tid1) -- (tid5);
\node[circle, scale=0.75, fill] (tid2) at (3.75,3){};
\node[circle, scale=0.75, fill, task_scheduled] (tid6) at (3.75,4.5){};
\draw[](tid2) -- (tid6);
\node[circle, scale=0.75, fill] (tid3) at (5.25,3){};
\draw[](tid0) -- (tid1);
\draw[](tid0) -- (tid2);
\draw[](tid0) -- (tid3);
\end{tikzpicture}
\nodepart{three}
\footnotesize{4.625}
\nodepart{four}
\footnotesize{$50\:50$}
};
 \\ 
\node[draw=black, rectangle split,  rectangle split parts=4] (sn0x22af010){
\footnotesize{31.9271}
\nodepart{two}
\begin{tikzpicture}[scale=.2]
\node[circle, scale=0.75, fill] (tid0) at (2.25,1.5){};
\node[circle, scale=0.75, fill] (tid1) at (0.75,3){};
\node[circle, scale=0.75, fill, task_scheduled] (tid4) at (0.75,4.5){};
\draw[](tid1) -- (tid4);
\node[circle, scale=0.75, fill] (tid2) at (2.25,3){};
\node[circle, scale=0.75, fill, task_scheduled] (tid5) at (2.25,4.5){};
\draw[](tid2) -- (tid5);
\node[circle, scale=0.75, fill] (tid3) at (3.75,3){};
\node[circle, scale=0.75, fill] (tid6) at (3.75,4.5){};
\draw[](tid3) -- (tid6);
\draw[](tid0) -- (tid1);
\draw[](tid0) -- (tid2);
\draw[](tid0) -- (tid3);
\end{tikzpicture}
\nodepart{three}
\footnotesize{4.625}
\nodepart{four}
\footnotesize{$1$}
};
 \\ 
\\
};
\end{scope}
\begin{scope}[yshift=\leveltopIIIIII cm, anchor = center]
\matrix (line6)[row sep=0.5cm] {
\node[draw=black, rectangle split,  rectangle split parts=4] (sn0x22aecc0){
\footnotesize{3.125}
\nodepart{two}
\begin{tikzpicture}[scale=.2]
\node[circle, scale=0.75, fill] (tid0) at (2.25,1.5){};
\node[circle, scale=0.75, fill] (tid1) at (0.75,3){};
\node[circle, scale=0.75, fill] (tid4) at (0.75,4.5){};
\node[circle, scale=0.75, fill, task_scheduled] (tid5) at (0.75,6){};
\draw[](tid4) -- (tid5);
\draw[](tid1) -- (tid4);
\node[circle, scale=0.75, fill, task_scheduled] (tid2) at (2.25,3){};
\node[circle, scale=0.75, fill] (tid3) at (3.75,3){};
\draw[](tid0) -- (tid1);
\draw[](tid0) -- (tid2);
\draw[](tid0) -- (tid3);
\end{tikzpicture}
\nodepart{three}
\footnotesize{4.375}
\nodepart{four}
\footnotesize{$50\:50$}
};
 \\ 
\node[draw=black, rectangle split,  rectangle split parts=4] (sn0x22b21c0){
\footnotesize{24.7135}
\nodepart{two}
\begin{tikzpicture}[scale=.2]
\node[circle, scale=0.75, fill] (tid0) at (3,1.5){};
\node[circle, scale=0.75, fill] (tid1) at (1.5,3){};
\node[circle, scale=0.75, fill, task_scheduled] (tid4) at (0.75,4.5){};
\node[circle, scale=0.75, fill, task_scheduled] (tid5) at (2.25,4.5){};
\draw[](tid1) -- (tid4);
\draw[](tid1) -- (tid5);
\node[circle, scale=0.75, fill] (tid2) at (3.75,3){};
\node[circle, scale=0.75, fill] (tid3) at (5.25,3){};
\draw[](tid0) -- (tid1);
\draw[](tid0) -- (tid2);
\draw[](tid0) -- (tid3);
\end{tikzpicture}
\nodepart{three}
\footnotesize{4.125}
\nodepart{four}
\footnotesize{$1$}
};
 \\ 
\node[draw=black, rectangle split,  rectangle split parts=4] (sn0x22af160){
\footnotesize{72.1615}
\nodepart{two}
\begin{tikzpicture}[scale=.2]
\node[circle, scale=0.75, fill] (tid0) at (2.25,1.5){};
\node[circle, scale=0.75, fill] (tid1) at (0.75,3){};
\node[circle, scale=0.75, fill, task_scheduled] (tid4) at (0.75,4.5){};
\draw[](tid1) -- (tid4);
\node[circle, scale=0.75, fill] (tid2) at (2.25,3){};
\node[circle, scale=0.75, fill, task_scheduled] (tid5) at (2.25,4.5){};
\draw[](tid2) -- (tid5);
\node[circle, scale=0.75, fill] (tid3) at (3.75,3){};
\draw[](tid0) -- (tid1);
\draw[](tid0) -- (tid2);
\draw[](tid0) -- (tid3);
\end{tikzpicture}
\nodepart{three}
\footnotesize{4.125}
\nodepart{four}
\footnotesize{$1$}
};
 \\ 
\\
};
\end{scope}
\begin{scope}[yshift=\leveltopIIIIIII cm, anchor = center]
\matrix (line7)[row sep=0.5cm] {
\node[draw=black, rectangle split,  rectangle split parts=4] (sn0x22af6c0){
\footnotesize{1.5625}
\nodepart{two}
\begin{tikzpicture}[scale=.2]
\node[circle, scale=0.75, fill] (tid0) at (1.5,1.5){};
\node[circle, scale=0.75, fill] (tid1) at (0.75,3){};
\node[circle, scale=0.75, fill] (tid3) at (0.75,4.5){};
\node[circle, scale=0.75, fill, task_scheduled] (tid4) at (0.75,6){};
\draw[](tid3) -- (tid4);
\draw[](tid1) -- (tid3);
\node[circle, scale=0.75, fill, task_scheduled] (tid2) at (2.25,3){};
\draw[](tid0) -- (tid1);
\draw[](tid0) -- (tid2);
\end{tikzpicture}
\nodepart{three}
\footnotesize{4.125}
\nodepart{four}
\footnotesize{$50\:50$}
};
 \\ 
\node[draw=black, rectangle split,  rectangle split parts=4] (sn0x22afb50){
\footnotesize{98.4375}
\nodepart{two}
\begin{tikzpicture}[scale=.2]
\node[circle, scale=0.75, fill] (tid0) at (2.25,1.5){};
\node[circle, scale=0.75, fill] (tid1) at (0.75,3){};
\node[circle, scale=0.75, fill, task_scheduled] (tid4) at (0.75,4.5){};
\draw[](tid1) -- (tid4);
\node[circle, scale=0.75, fill, task_scheduled] (tid2) at (2.25,3){};
\node[circle, scale=0.75, fill] (tid3) at (3.75,3){};
\draw[](tid0) -- (tid1);
\draw[](tid0) -- (tid2);
\draw[](tid0) -- (tid3);
\end{tikzpicture}
\nodepart{three}
\footnotesize{3.625}
\nodepart{four}
\footnotesize{$50\:50$}
};
 \\ 
\\
};
\end{scope}
\draw (sn0x22a7ba0.east) -- (sn0x22ab160.west);
\draw (sn0x22a7ba0.east) -- (sn0x22aaaa0.west);
\draw (sn0x22a7ba0.east) -- (sn0x22ab2c0.west);
\draw (sn0x22a7ba0.east) -- (sn0x22aba90.west);
\draw (sn0x22a7ba0.east) -- (sn0x22ab500.west);
\draw (sn0x22ab160.east) -- (sn0x22ac4a0.west);
\draw (sn0x22ab160.east) -- (sn0x22acd50.west);
\draw (sn0x22ab160.east) -- (sn0x22ac710.west);
\draw (sn0x22ab160.east) -- (sn0x22ada30.west);
\draw (sn0x22aaaa0.east) -- (sn0x22b4a40.west);
\draw (sn0x22aaaa0.east) -- (sn0x22ada30.west);
\draw (sn0x22aaaa0.east) -- (sn0x22b4be0.west);
\draw (sn0x22ab2c0.east) -- (sn0x22b54f0.west);
\draw (sn0x22ab2c0.east) -- (sn0x22b5d10.west);
\draw (sn0x22ab2c0.east) -- (sn0x22b6220.west);
\draw (sn0x22ab2c0.east) -- (sn0x22ac710.west);
\draw (sn0x22ab2c0.east) -- (sn0x22ada30.west);
\draw (sn0x22aba90.east) -- (sn0x22ada30.west);
\draw (sn0x22aba90.east) -- (sn0x22b4be0.west);
\draw (sn0x22ab500.east) -- (sn0x22ada30.west);
\draw (sn0x22ab500.east) -- (sn0x22b4be0.west);
\draw (sn0x22ab500.east) -- (sn0x22b7500.west);
\draw (sn0x22ab500.east) -- (sn0x22b7aa0.west);
\draw (sn0x22b4a40.east) -- (sn0x22b2b90.west);
\draw (sn0x22b4a40.east) -- (sn0x22b34f0.west);
\draw (sn0x22b4a40.east) -- (sn0x22b3cf0.west);
\draw (sn0x22b7500.east) -- (sn0x22b6c20.west);
\draw (sn0x22b7500.east) -- (sn0x22b66d0.west);
\draw (sn0x22b7500.east) -- (sn0x22b34f0.west);
\draw (sn0x22b7500.east) -- (sn0x22b3cf0.west);
\draw (sn0x22b7aa0.east) -- (sn0x22b3cf0.west);
\draw (sn0x22ac710.east) -- (sn0x22ade60.west);
\draw (sn0x22ac710.east) -- (sn0x22ae720.west);
\draw (sn0x22ada30.east) -- (sn0x22ae720.west);
\draw (sn0x22ada30.east) -- (sn0x22b2d60.west);
\draw (sn0x22ada30.east) -- (sn0x22b34f0.west);
\draw (sn0x22ada30.east) -- (sn0x22b3cf0.west);
\draw (sn0x22b4be0.east) -- (sn0x22b3cf0.west);
\draw (sn0x22ac4a0.east) -- (sn0x22ad490.west);
\draw (sn0x22ac4a0.east) -- (sn0x22ade60.west);
\draw (sn0x22ac4a0.east) -- (sn0x22ae720.west);
\draw (sn0x22acd50.east) -- (sn0x22b2b90.west);
\draw (sn0x22acd50.east) -- (sn0x22ae720.west);
\draw (sn0x22acd50.east) -- (sn0x22b2d60.west);
\draw (sn0x22b54f0.east) -- (sn0x22ae720.west);
\draw (sn0x22b54f0.east) -- (sn0x22ade60.west);
\draw (sn0x22b5d10.east) -- (sn0x22ae720.west);
\draw (sn0x22b5d10.east) -- (sn0x22b2d60.west);
\draw (sn0x22b6220.east) -- (sn0x22b2d60.west);
\draw (sn0x22b6220.east) -- (sn0x22ae720.west);
\draw (sn0x22b6220.east) -- (sn0x22b66d0.west);
\draw (sn0x22b6220.east) -- (sn0x22b6c20.west);
\draw (sn0x22ad490.east) -- (sn0x22ae570.west);
\draw (sn0x22ad490.east) -- (sn0x22af010.west);
\draw (sn0x22b34f0.east) -- (sn0x22b14e0.west);
\draw (sn0x22b34f0.east) -- (sn0x22b1ca0.west);
\draw (sn0x22b3cf0.east) -- (sn0x22b1ca0.west);
\draw (sn0x22b3cf0.east) -- (sn0x22b40d0.west);
\draw (sn0x22b2b90.east) -- (sn0x22ae570.west);
\draw (sn0x22b2b90.east) -- (sn0x22b14e0.west);
\draw (sn0x22b2b90.east) -- (sn0x22b1ca0.west);
\draw (sn0x22b66d0.east) -- (sn0x22b1ca0.west);
\draw (sn0x22b6c20.east) -- (sn0x22b1ca0.west);
\draw (sn0x22b6c20.east) -- (sn0x22b14e0.west);
\draw (sn0x22ade60.east) -- (sn0x22af010.west);
\draw (sn0x22ae720.east) -- (sn0x22af010.west);
\draw (sn0x22ae720.east) -- (sn0x22b14e0.west);
\draw (sn0x22ae720.east) -- (sn0x22b1ca0.west);
\draw (sn0x22b2d60.east) -- (sn0x22b1ca0.west);
\draw (sn0x22ae570.east) -- (sn0x22aecc0.west);
\draw (sn0x22ae570.east) -- (sn0x22af160.west);
\draw (sn0x22b40d0.east) -- (sn0x22b21c0.west);
\draw (sn0x22b14e0.east) -- (sn0x22af160.west);
\draw (sn0x22b1ca0.east) -- (sn0x22af160.west);
\draw (sn0x22b1ca0.east) -- (sn0x22b21c0.west);
\draw (sn0x22af010.east) -- (sn0x22af160.west);
\draw (sn0x22aecc0.east) -- (sn0x22af6c0.west);
\draw (sn0x22aecc0.east) -- (sn0x22afb50.west);
\draw (sn0x22b21c0.east) -- (sn0x22afb50.west);
\draw (sn0x22af160.east) -- (sn0x22afb50.west);
\end{tikzpicture}

%%% Local Variables:
%%% TeX-master: "thesis/thesis.tex"
%%% End: 
\renewcommand{\leveltopI}{-10cm + \leveltop}
\renewcommand{\leveltopII}{-10cm + \leveltopI}
\renewcommand{\leveltopIII}{-10cm + \leveltopII}
\renewcommand{\leveltopIIII}{-10cm + \leveltopIII}
\renewcommand{\leveltopIIIII}{-10cm + \leveltopIIII}
\renewcommand{\leveltopIIIIII}{-10cm + \leveltopIIIII}
\renewcommand{\leveltopIIIIIII}{-10cm + \leveltopIIIIII}
\renewcommand{\leveltopIIIIIIII}{-10cm + \leveltopIIIIIII}
\renewcommand{\leveltopIIIIIIIII}{-10cm + \leveltopIIIIIIII}
\renewcommand{\leveltopIIIIIIIIII}{-10cm + \leveltopIIIIIIIII}
\renewcommand{\leveltopIIIIIIIIIII}{-10cm + \leveltopIIIIIIIIII}
\begin{tikzpicture}[scale=.2, anchor=south, rotate=90]
\begin{scope}[yshift=\leveltopI cm, anchor = center]
\matrix (line1)[row sep=0.5cm] {
\node[draw=black, rectangle split,  rectangle split parts=4] (sn0x22a97d0){
\footnotesize{100}
\nodepart{two}
\begin{tikzpicture}[scale=.2]
\node[circle, scale=0.75, fill] (tid0) at (4.5,1.5){};
\node[circle, scale=0.75, fill] (tid1) at (2.25,3){};
\node[circle, scale=0.75, fill, task_scheduled] (tid4) at (0.75,4.5){};
\node[circle, scale=0.75, fill] (tid5) at (2.25,4.5){};
\node[circle, scale=0.75, fill] (tid6) at (3.75,4.5){};
\draw[](tid1) -- (tid4);
\draw[](tid1) -- (tid5);
\draw[](tid1) -- (tid6);
\node[circle, scale=0.75, fill] (tid2) at (6,3){};
\node[circle, scale=0.75, fill] (tid7) at (5.25,4.5){};
\node[circle, scale=0.75, fill, task_scheduled] (tid10) at (5.25,6){};
\draw[](tid7) -- (tid10);
\node[circle, scale=0.75, fill] (tid8) at (6.75,4.5){};
\draw[](tid2) -- (tid7);
\draw[](tid2) -- (tid8);
\node[circle, scale=0.75, fill] (tid3) at (8.25,3){};
\node[circle, scale=0.75, fill] (tid9) at (8.25,4.5){};
\draw[](tid3) -- (tid9);
\draw[](tid0) -- (tid1);
\draw[](tid0) -- (tid2);
\draw[](tid0) -- (tid3);
\end{tikzpicture}
\nodepart{three}
\footnotesize{6.63281}
\nodepart{four}
\footnotesize{$25\:12\:12\:20\:20\:10$}
};
 \\ 
\\
};
\end{scope}
\begin{scope}[yshift=\leveltopII cm, anchor = center]
\matrix (line2)[row sep=0.5cm] {
\node[draw=black, rectangle split,  rectangle split parts=4] (sn0x22b7700){
\footnotesize{25}
\nodepart{two}
\begin{tikzpicture}[scale=.2]
\node[circle, scale=0.75, fill] (tid0) at (3.75,1.5){};
\node[circle, scale=0.75, fill] (tid1) at (1.5,3){};
\node[circle, scale=0.75, fill] (tid4) at (0.75,4.5){};
\node[circle, scale=0.75, fill, task_scheduled] (tid9) at (0.75,6){};
\draw[](tid4) -- (tid9);
\node[circle, scale=0.75, fill] (tid5) at (2.25,4.5){};
\draw[](tid1) -- (tid4);
\draw[](tid1) -- (tid5);
\node[circle, scale=0.75, fill] (tid2) at (4.5,3){};
\node[circle, scale=0.75, fill, task_scheduled] (tid6) at (3.75,4.5){};
\node[circle, scale=0.75, fill] (tid7) at (5.25,4.5){};
\draw[](tid2) -- (tid6);
\draw[](tid2) -- (tid7);
\node[circle, scale=0.75, fill] (tid3) at (6.75,3){};
\node[circle, scale=0.75, fill] (tid8) at (6.75,4.5){};
\draw[](tid3) -- (tid8);
\draw[](tid0) -- (tid1);
\draw[](tid0) -- (tid2);
\draw[](tid0) -- (tid3);
\end{tikzpicture}
\nodepart{three}
\footnotesize{6.14062}
\nodepart{four}
\footnotesize{$17\:33\:25\:12\:12$}
};
 \\ 
\node[draw=black, rectangle split,  rectangle split parts=4] (sn0x22b8ae0){
\footnotesize{12.5}
\nodepart{two}
\begin{tikzpicture}[scale=.2]
\node[circle, scale=0.75, fill] (tid0) at (3.75,1.5){};
\node[circle, scale=0.75, fill] (tid1) at (1.5,3){};
\node[circle, scale=0.75, fill] (tid4) at (0.75,4.5){};
\node[circle, scale=0.75, fill, task_scheduled] (tid9) at (0.75,6){};
\draw[](tid4) -- (tid9);
\node[circle, scale=0.75, fill, task_scheduled] (tid5) at (2.25,4.5){};
\draw[](tid1) -- (tid4);
\draw[](tid1) -- (tid5);
\node[circle, scale=0.75, fill] (tid2) at (4.5,3){};
\node[circle, scale=0.75, fill] (tid6) at (3.75,4.5){};
\node[circle, scale=0.75, fill] (tid7) at (5.25,4.5){};
\draw[](tid2) -- (tid6);
\draw[](tid2) -- (tid7);
\node[circle, scale=0.75, fill] (tid3) at (6.75,3){};
\node[circle, scale=0.75, fill] (tid8) at (6.75,4.5){};
\draw[](tid3) -- (tid8);
\draw[](tid0) -- (tid1);
\draw[](tid0) -- (tid2);
\draw[](tid0) -- (tid3);
\end{tikzpicture}
\nodepart{three}
\footnotesize{6.14062}
\nodepart{four}
\footnotesize{$33\:17\:12\:25\:12$}
};
 \\ 
\node[draw=black, rectangle split,  rectangle split parts=4] (sn0x22b8c20){
\footnotesize{12.5}
\nodepart{two}
\begin{tikzpicture}[scale=.2]
\node[circle, scale=0.75, fill] (tid0) at (3.75,1.5){};
\node[circle, scale=0.75, fill] (tid1) at (1.5,3){};
\node[circle, scale=0.75, fill] (tid4) at (0.75,4.5){};
\node[circle, scale=0.75, fill, task_scheduled] (tid9) at (0.75,6){};
\draw[](tid4) -- (tid9);
\node[circle, scale=0.75, fill] (tid5) at (2.25,4.5){};
\draw[](tid1) -- (tid4);
\draw[](tid1) -- (tid5);
\node[circle, scale=0.75, fill] (tid2) at (4.5,3){};
\node[circle, scale=0.75, fill] (tid6) at (3.75,4.5){};
\node[circle, scale=0.75, fill] (tid7) at (5.25,4.5){};
\draw[](tid2) -- (tid6);
\draw[](tid2) -- (tid7);
\node[circle, scale=0.75, fill] (tid3) at (6.75,3){};
\node[circle, scale=0.75, fill, task_scheduled] (tid8) at (6.75,4.5){};
\draw[](tid3) -- (tid8);
\draw[](tid0) -- (tid1);
\draw[](tid0) -- (tid2);
\draw[](tid0) -- (tid3);
\end{tikzpicture}
\nodepart{three}
\footnotesize{6.14062}
\nodepart{four}
\footnotesize{$17\:33\:50$}
};
 \\ 
\node[draw=black, rectangle split,  rectangle split parts=4] (sn0x22b8400){
\footnotesize{20}
\nodepart{two}
\begin{tikzpicture}[scale=.2]
\node[circle, scale=0.75, fill] (tid0) at (4.5,1.5){};
\node[circle, scale=0.75, fill] (tid1) at (2.25,3){};
\node[circle, scale=0.75, fill, task_scheduled] (tid4) at (0.75,4.5){};
\node[circle, scale=0.75, fill, task_scheduled] (tid5) at (2.25,4.5){};
\node[circle, scale=0.75, fill] (tid6) at (3.75,4.5){};
\draw[](tid1) -- (tid4);
\draw[](tid1) -- (tid5);
\draw[](tid1) -- (tid6);
\node[circle, scale=0.75, fill] (tid2) at (6,3){};
\node[circle, scale=0.75, fill] (tid7) at (5.25,4.5){};
\node[circle, scale=0.75, fill] (tid8) at (6.75,4.5){};
\draw[](tid2) -- (tid7);
\draw[](tid2) -- (tid8);
\node[circle, scale=0.75, fill] (tid3) at (8.25,3){};
\node[circle, scale=0.75, fill] (tid9) at (8.25,4.5){};
\draw[](tid3) -- (tid9);
\draw[](tid0) -- (tid1);
\draw[](tid0) -- (tid2);
\draw[](tid0) -- (tid3);
\end{tikzpicture}
\nodepart{three}
\footnotesize{6.125}
\nodepart{four}
\footnotesize{$25\:50\:25$}
};
 \\ 
\node[draw=black, rectangle split,  rectangle split parts=4] (sn0x22ab2c0){
\footnotesize{20}
\nodepart{two}
\begin{tikzpicture}[scale=.2]
\node[circle, scale=0.75, fill] (tid0) at (4.5,1.5){};
\node[circle, scale=0.75, fill] (tid1) at (2.25,3){};
\node[circle, scale=0.75, fill, task_scheduled] (tid4) at (0.75,4.5){};
\node[circle, scale=0.75, fill] (tid5) at (2.25,4.5){};
\node[circle, scale=0.75, fill] (tid6) at (3.75,4.5){};
\draw[](tid1) -- (tid4);
\draw[](tid1) -- (tid5);
\draw[](tid1) -- (tid6);
\node[circle, scale=0.75, fill] (tid2) at (6,3){};
\node[circle, scale=0.75, fill, task_scheduled] (tid7) at (5.25,4.5){};
\node[circle, scale=0.75, fill] (tid8) at (6.75,4.5){};
\draw[](tid2) -- (tid7);
\draw[](tid2) -- (tid8);
\node[circle, scale=0.75, fill] (tid3) at (8.25,3){};
\node[circle, scale=0.75, fill] (tid9) at (8.25,4.5){};
\draw[](tid3) -- (tid9);
\draw[](tid0) -- (tid1);
\draw[](tid0) -- (tid2);
\draw[](tid0) -- (tid3);
\end{tikzpicture}
\nodepart{three}
\footnotesize{6.125}
\nodepart{four}
\footnotesize{$25\:25\:25\:12\:12$}
};
 \\ 
\node[draw=black, rectangle split,  rectangle split parts=4] (sn0x22b9170){
\footnotesize{10}
\nodepart{two}
\begin{tikzpicture}[scale=.2]
\node[circle, scale=0.75, fill] (tid0) at (4.5,1.5){};
\node[circle, scale=0.75, fill] (tid1) at (2.25,3){};
\node[circle, scale=0.75, fill, task_scheduled] (tid4) at (0.75,4.5){};
\node[circle, scale=0.75, fill] (tid5) at (2.25,4.5){};
\node[circle, scale=0.75, fill] (tid6) at (3.75,4.5){};
\draw[](tid1) -- (tid4);
\draw[](tid1) -- (tid5);
\draw[](tid1) -- (tid6);
\node[circle, scale=0.75, fill] (tid2) at (6,3){};
\node[circle, scale=0.75, fill] (tid7) at (5.25,4.5){};
\node[circle, scale=0.75, fill] (tid8) at (6.75,4.5){};
\draw[](tid2) -- (tid7);
\draw[](tid2) -- (tid8);
\node[circle, scale=0.75, fill] (tid3) at (8.25,3){};
\node[circle, scale=0.75, fill, task_scheduled] (tid9) at (8.25,4.5){};
\draw[](tid3) -- (tid9);
\draw[](tid0) -- (tid1);
\draw[](tid0) -- (tid2);
\draw[](tid0) -- (tid3);
\end{tikzpicture}
\nodepart{three}
\footnotesize{6.125}
\nodepart{four}
\footnotesize{$25\:25\:50$}
};
 \\ 
\\
};
\end{scope}
\begin{scope}[yshift=\leveltopIII cm, anchor = center]
\matrix (line3)[row sep=0.5cm] {
\node[draw=black, rectangle split,  rectangle split parts=4] (sn0x22bb880){
\footnotesize{2.08333}
\nodepart{two}
\begin{tikzpicture}[scale=.2]
\node[circle, scale=0.75, fill] (tid0) at (3.75,1.5){};
\node[circle, scale=0.75, fill] (tid1) at (1.5,3){};
\node[circle, scale=0.75, fill] (tid4) at (0.75,4.5){};
\node[circle, scale=0.75, fill, task_scheduled] (tid8) at (0.75,6){};
\draw[](tid4) -- (tid8);
\node[circle, scale=0.75, fill, task_scheduled] (tid5) at (2.25,4.5){};
\draw[](tid1) -- (tid4);
\draw[](tid1) -- (tid5);
\node[circle, scale=0.75, fill] (tid2) at (4.5,3){};
\node[circle, scale=0.75, fill] (tid6) at (3.75,4.5){};
\node[circle, scale=0.75, fill] (tid7) at (5.25,4.5){};
\draw[](tid2) -- (tid6);
\draw[](tid2) -- (tid7);
\node[circle, scale=0.75, fill] (tid3) at (6.75,3){};
\draw[](tid0) -- (tid1);
\draw[](tid0) -- (tid2);
\draw[](tid0) -- (tid3);
\end{tikzpicture}
\nodepart{three}
\footnotesize{5.65625}
\nodepart{four}
\footnotesize{$50\:17\:33$}
};
 \\ 
\node[draw=black, rectangle split,  rectangle split parts=4] (sn0x22bc030){
\footnotesize{4.16667}
\nodepart{two}
\begin{tikzpicture}[scale=.2]
\node[circle, scale=0.75, fill] (tid0) at (3.75,1.5){};
\node[circle, scale=0.75, fill] (tid1) at (1.5,3){};
\node[circle, scale=0.75, fill] (tid4) at (0.75,4.5){};
\node[circle, scale=0.75, fill, task_scheduled] (tid8) at (0.75,6){};
\draw[](tid4) -- (tid8);
\node[circle, scale=0.75, fill] (tid5) at (2.25,4.5){};
\draw[](tid1) -- (tid4);
\draw[](tid1) -- (tid5);
\node[circle, scale=0.75, fill] (tid2) at (4.5,3){};
\node[circle, scale=0.75, fill, task_scheduled] (tid6) at (3.75,4.5){};
\node[circle, scale=0.75, fill] (tid7) at (5.25,4.5){};
\draw[](tid2) -- (tid6);
\draw[](tid2) -- (tid7);
\node[circle, scale=0.75, fill] (tid3) at (6.75,3){};
\draw[](tid0) -- (tid1);
\draw[](tid0) -- (tid2);
\draw[](tid0) -- (tid3);
\end{tikzpicture}
\nodepart{three}
\footnotesize{5.65625}
\nodepart{four}
\footnotesize{$25\:25\:33\:17$}
};
 \\ 
\node[draw=black, rectangle split,  rectangle split parts=4] (sn0x22b9da0){
\footnotesize{4.16667}
\nodepart{two}
\begin{tikzpicture}[scale=.2]
\node[circle, scale=0.75, fill] (tid0) at (3,1.5){};
\node[circle, scale=0.75, fill] (tid1) at (1.5,3){};
\node[circle, scale=0.75, fill] (tid4) at (0.75,4.5){};
\node[circle, scale=0.75, fill, task_scheduled] (tid8) at (0.75,6){};
\draw[](tid4) -- (tid8);
\node[circle, scale=0.75, fill, task_scheduled] (tid5) at (2.25,4.5){};
\draw[](tid1) -- (tid4);
\draw[](tid1) -- (tid5);
\node[circle, scale=0.75, fill] (tid2) at (3.75,3){};
\node[circle, scale=0.75, fill] (tid6) at (3.75,4.5){};
\draw[](tid2) -- (tid6);
\node[circle, scale=0.75, fill] (tid3) at (5.25,3){};
\node[circle, scale=0.75, fill] (tid7) at (5.25,4.5){};
\draw[](tid3) -- (tid7);
\draw[](tid0) -- (tid1);
\draw[](tid0) -- (tid2);
\draw[](tid0) -- (tid3);
\end{tikzpicture}
\nodepart{three}
\footnotesize{5.65625}
\nodepart{four}
\footnotesize{$50\:17\:33$}
};
 \\ 
\node[draw=black, rectangle split,  rectangle split parts=4] (sn0x22ba8b0){
\footnotesize{8.33333}
\nodepart{two}
\begin{tikzpicture}[scale=.2]
\node[circle, scale=0.75, fill] (tid0) at (3,1.5){};
\node[circle, scale=0.75, fill] (tid1) at (1.5,3){};
\node[circle, scale=0.75, fill] (tid4) at (0.75,4.5){};
\node[circle, scale=0.75, fill, task_scheduled] (tid8) at (0.75,6){};
\draw[](tid4) -- (tid8);
\node[circle, scale=0.75, fill] (tid5) at (2.25,4.5){};
\draw[](tid1) -- (tid4);
\draw[](tid1) -- (tid5);
\node[circle, scale=0.75, fill] (tid2) at (3.75,3){};
\node[circle, scale=0.75, fill, task_scheduled] (tid6) at (3.75,4.5){};
\draw[](tid2) -- (tid6);
\node[circle, scale=0.75, fill] (tid3) at (5.25,3){};
\node[circle, scale=0.75, fill] (tid7) at (5.25,4.5){};
\draw[](tid3) -- (tid7);
\draw[](tid0) -- (tid1);
\draw[](tid0) -- (tid2);
\draw[](tid0) -- (tid3);
\end{tikzpicture}
\nodepart{three}
\footnotesize{5.65625}
\nodepart{four}
\footnotesize{$25\:25\:33\:17$}
};
 \\ 
\node[draw=black, rectangle split,  rectangle split parts=4] (sn0x22bc5d0){
\footnotesize{2.5}
\nodepart{two}
\begin{tikzpicture}[scale=.2]
\node[circle, scale=0.75, fill] (tid0) at (4.5,1.5){};
\node[circle, scale=0.75, fill] (tid1) at (2.25,3){};
\node[circle, scale=0.75, fill, task_scheduled] (tid4) at (0.75,4.5){};
\node[circle, scale=0.75, fill, task_scheduled] (tid5) at (2.25,4.5){};
\node[circle, scale=0.75, fill] (tid6) at (3.75,4.5){};
\draw[](tid1) -- (tid4);
\draw[](tid1) -- (tid5);
\draw[](tid1) -- (tid6);
\node[circle, scale=0.75, fill] (tid2) at (6,3){};
\node[circle, scale=0.75, fill] (tid7) at (5.25,4.5){};
\node[circle, scale=0.75, fill] (tid8) at (6.75,4.5){};
\draw[](tid2) -- (tid7);
\draw[](tid2) -- (tid8);
\node[circle, scale=0.75, fill] (tid3) at (8.25,3){};
\draw[](tid0) -- (tid1);
\draw[](tid0) -- (tid2);
\draw[](tid0) -- (tid3);
\end{tikzpicture}
\nodepart{three}
\footnotesize{5.625}
\nodepart{four}
\footnotesize{$33\:67$}
};
 \\ 
\node[draw=black, rectangle split,  rectangle split parts=4] (sn0x22b7500){
\footnotesize{2.5}
\nodepart{two}
\begin{tikzpicture}[scale=.2]
\node[circle, scale=0.75, fill] (tid0) at (4.5,1.5){};
\node[circle, scale=0.75, fill] (tid1) at (2.25,3){};
\node[circle, scale=0.75, fill, task_scheduled] (tid4) at (0.75,4.5){};
\node[circle, scale=0.75, fill] (tid5) at (2.25,4.5){};
\node[circle, scale=0.75, fill] (tid6) at (3.75,4.5){};
\draw[](tid1) -- (tid4);
\draw[](tid1) -- (tid5);
\draw[](tid1) -- (tid6);
\node[circle, scale=0.75, fill] (tid2) at (6,3){};
\node[circle, scale=0.75, fill, task_scheduled] (tid7) at (5.25,4.5){};
\node[circle, scale=0.75, fill] (tid8) at (6.75,4.5){};
\draw[](tid2) -- (tid7);
\draw[](tid2) -- (tid8);
\node[circle, scale=0.75, fill] (tid3) at (8.25,3){};
\draw[](tid0) -- (tid1);
\draw[](tid0) -- (tid2);
\draw[](tid0) -- (tid3);
\end{tikzpicture}
\nodepart{three}
\footnotesize{5.625}
\nodepart{four}
\footnotesize{$33\:17\:33\:17$}
};
 \\ 
\node[draw=black, rectangle split,  rectangle split parts=4] (sn0x22ac710){
\footnotesize{5}
\nodepart{two}
\begin{tikzpicture}[scale=.2]
\node[circle, scale=0.75, fill] (tid0) at (3.75,1.5){};
\node[circle, scale=0.75, fill] (tid1) at (2.25,3){};
\node[circle, scale=0.75, fill, task_scheduled] (tid4) at (0.75,4.5){};
\node[circle, scale=0.75, fill, task_scheduled] (tid5) at (2.25,4.5){};
\node[circle, scale=0.75, fill] (tid6) at (3.75,4.5){};
\draw[](tid1) -- (tid4);
\draw[](tid1) -- (tid5);
\draw[](tid1) -- (tid6);
\node[circle, scale=0.75, fill] (tid2) at (5.25,3){};
\node[circle, scale=0.75, fill] (tid7) at (5.25,4.5){};
\draw[](tid2) -- (tid7);
\node[circle, scale=0.75, fill] (tid3) at (6.75,3){};
\node[circle, scale=0.75, fill] (tid8) at (6.75,4.5){};
\draw[](tid3) -- (tid8);
\draw[](tid0) -- (tid1);
\draw[](tid0) -- (tid2);
\draw[](tid0) -- (tid3);
\end{tikzpicture}
\nodepart{three}
\footnotesize{5.625}
\nodepart{four}
\footnotesize{$33\:67$}
};
 \\ 
\node[draw=black, rectangle split,  rectangle split parts=4] (sn0x22ada30){
\footnotesize{5}
\nodepart{two}
\begin{tikzpicture}[scale=.2]
\node[circle, scale=0.75, fill] (tid0) at (3.75,1.5){};
\node[circle, scale=0.75, fill] (tid1) at (2.25,3){};
\node[circle, scale=0.75, fill, task_scheduled] (tid4) at (0.75,4.5){};
\node[circle, scale=0.75, fill] (tid5) at (2.25,4.5){};
\node[circle, scale=0.75, fill] (tid6) at (3.75,4.5){};
\draw[](tid1) -- (tid4);
\draw[](tid1) -- (tid5);
\draw[](tid1) -- (tid6);
\node[circle, scale=0.75, fill] (tid2) at (5.25,3){};
\node[circle, scale=0.75, fill, task_scheduled] (tid7) at (5.25,4.5){};
\draw[](tid2) -- (tid7);
\node[circle, scale=0.75, fill] (tid3) at (6.75,3){};
\node[circle, scale=0.75, fill] (tid8) at (6.75,4.5){};
\draw[](tid3) -- (tid8);
\draw[](tid0) -- (tid1);
\draw[](tid0) -- (tid2);
\draw[](tid0) -- (tid3);
\end{tikzpicture}
\nodepart{three}
\footnotesize{5.625}
\nodepart{four}
\footnotesize{$33\:17\:33\:17$}
};
 \\ 
\node[draw=black, rectangle split,  rectangle split parts=4] (sn0x22ac4a0){
\footnotesize{4.16667}
\nodepart{two}
\begin{tikzpicture}[scale=.2]
\node[circle, scale=0.75, fill] (tid0) at (3,1.5){};
\node[circle, scale=0.75, fill] (tid1) at (1.5,3){};
\node[circle, scale=0.75, fill, task_scheduled] (tid4) at (0.75,4.5){};
\node[circle, scale=0.75, fill] (tid5) at (2.25,4.5){};
\draw[](tid1) -- (tid4);
\draw[](tid1) -- (tid5);
\node[circle, scale=0.75, fill] (tid2) at (3.75,3){};
\node[circle, scale=0.75, fill] (tid6) at (3.75,4.5){};
\node[circle, scale=0.75, fill, task_scheduled] (tid8) at (3.75,6){};
\draw[](tid6) -- (tid8);
\draw[](tid2) -- (tid6);
\node[circle, scale=0.75, fill] (tid3) at (5.25,3){};
\node[circle, scale=0.75, fill] (tid7) at (5.25,4.5){};
\draw[](tid3) -- (tid7);
\draw[](tid0) -- (tid1);
\draw[](tid0) -- (tid2);
\draw[](tid0) -- (tid3);
\end{tikzpicture}
\nodepart{three}
\footnotesize{5.65625}
\nodepart{four}
\footnotesize{$50\:17\:33$}
};
 \\ 
\node[draw=black, rectangle split,  rectangle split parts=4] (sn0x22acd50){
\footnotesize{2.08333}
\nodepart{two}
\begin{tikzpicture}[scale=.2]
\node[circle, scale=0.75, fill] (tid0) at (3,1.5){};
\node[circle, scale=0.75, fill] (tid1) at (1.5,3){};
\node[circle, scale=0.75, fill] (tid4) at (0.75,4.5){};
\node[circle, scale=0.75, fill] (tid5) at (2.25,4.5){};
\draw[](tid1) -- (tid4);
\draw[](tid1) -- (tid5);
\node[circle, scale=0.75, fill] (tid2) at (3.75,3){};
\node[circle, scale=0.75, fill] (tid6) at (3.75,4.5){};
\node[circle, scale=0.75, fill, task_scheduled] (tid8) at (3.75,6){};
\draw[](tid6) -- (tid8);
\draw[](tid2) -- (tid6);
\node[circle, scale=0.75, fill] (tid3) at (5.25,3){};
\node[circle, scale=0.75, fill, task_scheduled] (tid7) at (5.25,4.5){};
\draw[](tid3) -- (tid7);
\draw[](tid0) -- (tid1);
\draw[](tid0) -- (tid2);
\draw[](tid0) -- (tid3);
\end{tikzpicture}
\nodepart{three}
\footnotesize{5.65625}
\nodepart{four}
\footnotesize{$50\:33\:17$}
};
 \\ 
\node[draw=black, rectangle split,  rectangle split parts=4] (sn0x22b54f0){
\footnotesize{24.375}
\nodepart{two}
\begin{tikzpicture}[scale=.2]
\node[circle, scale=0.75, fill] (tid0) at (3.75,1.5){};
\node[circle, scale=0.75, fill] (tid1) at (1.5,3){};
\node[circle, scale=0.75, fill, task_scheduled] (tid4) at (0.75,4.5){};
\node[circle, scale=0.75, fill] (tid5) at (2.25,4.5){};
\draw[](tid1) -- (tid4);
\draw[](tid1) -- (tid5);
\node[circle, scale=0.75, fill] (tid2) at (4.5,3){};
\node[circle, scale=0.75, fill, task_scheduled] (tid6) at (3.75,4.5){};
\node[circle, scale=0.75, fill] (tid7) at (5.25,4.5){};
\draw[](tid2) -- (tid6);
\draw[](tid2) -- (tid7);
\node[circle, scale=0.75, fill] (tid3) at (6.75,3){};
\node[circle, scale=0.75, fill] (tid8) at (6.75,4.5){};
\draw[](tid3) -- (tid8);
\draw[](tid0) -- (tid1);
\draw[](tid0) -- (tid2);
\draw[](tid0) -- (tid3);
\end{tikzpicture}
\nodepart{three}
\footnotesize{5.625}
\nodepart{four}
\footnotesize{$67\:33$}
};
 \\ 
\node[draw=black, rectangle split,  rectangle split parts=4] (sn0x22b5d10){
\footnotesize{12.1875}
\nodepart{two}
\begin{tikzpicture}[scale=.2]
\node[circle, scale=0.75, fill] (tid0) at (3.75,1.5){};
\node[circle, scale=0.75, fill] (tid1) at (1.5,3){};
\node[circle, scale=0.75, fill, task_scheduled] (tid4) at (0.75,4.5){};
\node[circle, scale=0.75, fill, task_scheduled] (tid5) at (2.25,4.5){};
\draw[](tid1) -- (tid4);
\draw[](tid1) -- (tid5);
\node[circle, scale=0.75, fill] (tid2) at (4.5,3){};
\node[circle, scale=0.75, fill] (tid6) at (3.75,4.5){};
\node[circle, scale=0.75, fill] (tid7) at (5.25,4.5){};
\draw[](tid2) -- (tid6);
\draw[](tid2) -- (tid7);
\node[circle, scale=0.75, fill] (tid3) at (6.75,3){};
\node[circle, scale=0.75, fill] (tid8) at (6.75,4.5){};
\draw[](tid3) -- (tid8);
\draw[](tid0) -- (tid1);
\draw[](tid0) -- (tid2);
\draw[](tid0) -- (tid3);
\end{tikzpicture}
\nodepart{three}
\footnotesize{5.625}
\nodepart{four}
\footnotesize{$67\:33$}
};
 \\ 
\node[draw=black, rectangle split,  rectangle split parts=4] (sn0x22b6220){
\footnotesize{23.4375}
\nodepart{two}
\begin{tikzpicture}[scale=.2]
\node[circle, scale=0.75, fill] (tid0) at (3.75,1.5){};
\node[circle, scale=0.75, fill] (tid1) at (1.5,3){};
\node[circle, scale=0.75, fill, task_scheduled] (tid4) at (0.75,4.5){};
\node[circle, scale=0.75, fill] (tid5) at (2.25,4.5){};
\draw[](tid1) -- (tid4);
\draw[](tid1) -- (tid5);
\node[circle, scale=0.75, fill] (tid2) at (4.5,3){};
\node[circle, scale=0.75, fill] (tid6) at (3.75,4.5){};
\node[circle, scale=0.75, fill] (tid7) at (5.25,4.5){};
\draw[](tid2) -- (tid6);
\draw[](tid2) -- (tid7);
\node[circle, scale=0.75, fill] (tid3) at (6.75,3){};
\node[circle, scale=0.75, fill, task_scheduled] (tid8) at (6.75,4.5){};
\draw[](tid3) -- (tid8);
\draw[](tid0) -- (tid1);
\draw[](tid0) -- (tid2);
\draw[](tid0) -- (tid3);
\end{tikzpicture}
\nodepart{three}
\footnotesize{5.625}
\nodepart{four}
\footnotesize{$17\:33\:17\:33$}
};
 \\ 
\\
};
\end{scope}
\begin{scope}[yshift=\leveltopIIII cm, anchor = center]
\matrix (line4)[row sep=0.5cm] {
\node[draw=black, rectangle split,  rectangle split parts=4] (sn0x22baab0){
\footnotesize{3.125}
\nodepart{two}
\begin{tikzpicture}[scale=.2]
\node[circle, scale=0.75, fill] (tid0) at (3,1.5){};
\node[circle, scale=0.75, fill] (tid1) at (1.5,3){};
\node[circle, scale=0.75, fill] (tid4) at (0.75,4.5){};
\node[circle, scale=0.75, fill, task_scheduled] (tid7) at (0.75,6){};
\draw[](tid4) -- (tid7);
\node[circle, scale=0.75, fill, task_scheduled] (tid5) at (2.25,4.5){};
\draw[](tid1) -- (tid4);
\draw[](tid1) -- (tid5);
\node[circle, scale=0.75, fill] (tid2) at (3.75,3){};
\node[circle, scale=0.75, fill] (tid6) at (3.75,4.5){};
\draw[](tid2) -- (tid6);
\node[circle, scale=0.75, fill] (tid3) at (5.25,3){};
\draw[](tid0) -- (tid1);
\draw[](tid0) -- (tid2);
\draw[](tid0) -- (tid3);
\end{tikzpicture}
\nodepart{three}
\footnotesize{5.1875}
\nodepart{four}
\footnotesize{$50\:25\:25$}
};
 \\ 
\node[draw=black, rectangle split,  rectangle split parts=4] (sn0x22bafa0){
\footnotesize{3.125}
\nodepart{two}
\begin{tikzpicture}[scale=.2]
\node[circle, scale=0.75, fill] (tid0) at (3,1.5){};
\node[circle, scale=0.75, fill] (tid1) at (1.5,3){};
\node[circle, scale=0.75, fill] (tid4) at (0.75,4.5){};
\node[circle, scale=0.75, fill, task_scheduled] (tid7) at (0.75,6){};
\draw[](tid4) -- (tid7);
\node[circle, scale=0.75, fill] (tid5) at (2.25,4.5){};
\draw[](tid1) -- (tid4);
\draw[](tid1) -- (tid5);
\node[circle, scale=0.75, fill] (tid2) at (3.75,3){};
\node[circle, scale=0.75, fill, task_scheduled] (tid6) at (3.75,4.5){};
\draw[](tid2) -- (tid6);
\node[circle, scale=0.75, fill] (tid3) at (5.25,3){};
\draw[](tid0) -- (tid1);
\draw[](tid0) -- (tid2);
\draw[](tid0) -- (tid3);
\end{tikzpicture}
\nodepart{three}
\footnotesize{5.1875}
\nodepart{four}
\footnotesize{$50\:50$}
};
 \\ 
\node[draw=black, rectangle split,  rectangle split parts=4] (sn0x22ad490){
\footnotesize{4.16667}
\nodepart{two}
\begin{tikzpicture}[scale=.2]
\node[circle, scale=0.75, fill] (tid0) at (2.25,1.5){};
\node[circle, scale=0.75, fill] (tid1) at (0.75,3){};
\node[circle, scale=0.75, fill] (tid4) at (0.75,4.5){};
\node[circle, scale=0.75, fill, task_scheduled] (tid7) at (0.75,6){};
\draw[](tid4) -- (tid7);
\draw[](tid1) -- (tid4);
\node[circle, scale=0.75, fill] (tid2) at (2.25,3){};
\node[circle, scale=0.75, fill, task_scheduled] (tid5) at (2.25,4.5){};
\draw[](tid2) -- (tid5);
\node[circle, scale=0.75, fill] (tid3) at (3.75,3){};
\node[circle, scale=0.75, fill] (tid6) at (3.75,4.5){};
\draw[](tid3) -- (tid6);
\draw[](tid0) -- (tid1);
\draw[](tid0) -- (tid2);
\draw[](tid0) -- (tid3);
\end{tikzpicture}
\nodepart{three}
\footnotesize{5.1875}
\nodepart{four}
\footnotesize{$50\:50$}
};
 \\ 
\node[draw=black, rectangle split,  rectangle split parts=4] (sn0x22b34f0){
\footnotesize{2.5}
\nodepart{two}
\begin{tikzpicture}[scale=.2]
\node[circle, scale=0.75, fill] (tid0) at (3.75,1.5){};
\node[circle, scale=0.75, fill] (tid1) at (2.25,3){};
\node[circle, scale=0.75, fill, task_scheduled] (tid4) at (0.75,4.5){};
\node[circle, scale=0.75, fill, task_scheduled] (tid5) at (2.25,4.5){};
\node[circle, scale=0.75, fill] (tid6) at (3.75,4.5){};
\draw[](tid1) -- (tid4);
\draw[](tid1) -- (tid5);
\draw[](tid1) -- (tid6);
\node[circle, scale=0.75, fill] (tid2) at (5.25,3){};
\node[circle, scale=0.75, fill] (tid7) at (5.25,4.5){};
\draw[](tid2) -- (tid7);
\node[circle, scale=0.75, fill] (tid3) at (6.75,3){};
\draw[](tid0) -- (tid1);
\draw[](tid0) -- (tid2);
\draw[](tid0) -- (tid3);
\end{tikzpicture}
\nodepart{three}
\footnotesize{5.125}
\nodepart{four}
\footnotesize{$50\:50$}
};
 \\ 
\node[draw=black, rectangle split,  rectangle split parts=4] (sn0x22b3cf0){
\footnotesize{1.25}
\nodepart{two}
\begin{tikzpicture}[scale=.2]
\node[circle, scale=0.75, fill] (tid0) at (3.75,1.5){};
\node[circle, scale=0.75, fill] (tid1) at (2.25,3){};
\node[circle, scale=0.75, fill, task_scheduled] (tid4) at (0.75,4.5){};
\node[circle, scale=0.75, fill] (tid5) at (2.25,4.5){};
\node[circle, scale=0.75, fill] (tid6) at (3.75,4.5){};
\draw[](tid1) -- (tid4);
\draw[](tid1) -- (tid5);
\draw[](tid1) -- (tid6);
\node[circle, scale=0.75, fill] (tid2) at (5.25,3){};
\node[circle, scale=0.75, fill, task_scheduled] (tid7) at (5.25,4.5){};
\draw[](tid2) -- (tid7);
\node[circle, scale=0.75, fill] (tid3) at (6.75,3){};
\draw[](tid0) -- (tid1);
\draw[](tid0) -- (tid2);
\draw[](tid0) -- (tid3);
\end{tikzpicture}
\nodepart{three}
\footnotesize{5.125}
\nodepart{four}
\footnotesize{$50\:50$}
};
 \\ 
\node[draw=black, rectangle split,  rectangle split parts=4] (sn0x22b2b90){
\footnotesize{2.08333}
\nodepart{two}
\begin{tikzpicture}[scale=.2]
\node[circle, scale=0.75, fill] (tid0) at (3,1.5){};
\node[circle, scale=0.75, fill] (tid1) at (1.5,3){};
\node[circle, scale=0.75, fill, task_scheduled] (tid4) at (0.75,4.5){};
\node[circle, scale=0.75, fill] (tid5) at (2.25,4.5){};
\draw[](tid1) -- (tid4);
\draw[](tid1) -- (tid5);
\node[circle, scale=0.75, fill] (tid2) at (3.75,3){};
\node[circle, scale=0.75, fill] (tid6) at (3.75,4.5){};
\node[circle, scale=0.75, fill, task_scheduled] (tid7) at (3.75,6){};
\draw[](tid6) -- (tid7);
\draw[](tid2) -- (tid6);
\node[circle, scale=0.75, fill] (tid3) at (5.25,3){};
\draw[](tid0) -- (tid1);
\draw[](tid0) -- (tid2);
\draw[](tid0) -- (tid3);
\end{tikzpicture}
\nodepart{three}
\footnotesize{5.1875}
\nodepart{four}
\footnotesize{$50\:25\:25$}
};
 \\ 
\node[draw=black, rectangle split,  rectangle split parts=4] (sn0x22b66d0){
\footnotesize{6.19792}
\nodepart{two}
\begin{tikzpicture}[scale=.2]
\node[circle, scale=0.75, fill] (tid0) at (3.75,1.5){};
\node[circle, scale=0.75, fill] (tid1) at (1.5,3){};
\node[circle, scale=0.75, fill, task_scheduled] (tid4) at (0.75,4.5){};
\node[circle, scale=0.75, fill, task_scheduled] (tid5) at (2.25,4.5){};
\draw[](tid1) -- (tid4);
\draw[](tid1) -- (tid5);
\node[circle, scale=0.75, fill] (tid2) at (4.5,3){};
\node[circle, scale=0.75, fill] (tid6) at (3.75,4.5){};
\node[circle, scale=0.75, fill] (tid7) at (5.25,4.5){};
\draw[](tid2) -- (tid6);
\draw[](tid2) -- (tid7);
\node[circle, scale=0.75, fill] (tid3) at (6.75,3){};
\draw[](tid0) -- (tid1);
\draw[](tid0) -- (tid2);
\draw[](tid0) -- (tid3);
\end{tikzpicture}
\nodepart{three}
\footnotesize{5.125}
\nodepart{four}
\footnotesize{$1$}
};
 \\ 
\node[draw=black, rectangle split,  rectangle split parts=4] (sn0x22b6c20){
\footnotesize{12.3958}
\nodepart{two}
\begin{tikzpicture}[scale=.2]
\node[circle, scale=0.75, fill] (tid0) at (3.75,1.5){};
\node[circle, scale=0.75, fill] (tid1) at (1.5,3){};
\node[circle, scale=0.75, fill, task_scheduled] (tid4) at (0.75,4.5){};
\node[circle, scale=0.75, fill] (tid5) at (2.25,4.5){};
\draw[](tid1) -- (tid4);
\draw[](tid1) -- (tid5);
\node[circle, scale=0.75, fill] (tid2) at (4.5,3){};
\node[circle, scale=0.75, fill, task_scheduled] (tid6) at (3.75,4.5){};
\node[circle, scale=0.75, fill] (tid7) at (5.25,4.5){};
\draw[](tid2) -- (tid6);
\draw[](tid2) -- (tid7);
\node[circle, scale=0.75, fill] (tid3) at (6.75,3){};
\draw[](tid0) -- (tid1);
\draw[](tid0) -- (tid2);
\draw[](tid0) -- (tid3);
\end{tikzpicture}
\nodepart{three}
\footnotesize{5.125}
\nodepart{four}
\footnotesize{$50\:50$}
};
 \\ 
\node[draw=black, rectangle split,  rectangle split parts=4] (sn0x22ade60){
\footnotesize{11.1806}
\nodepart{two}
\begin{tikzpicture}[scale=.2]
\node[circle, scale=0.75, fill] (tid0) at (3,1.5){};
\node[circle, scale=0.75, fill] (tid1) at (1.5,3){};
\node[circle, scale=0.75, fill, task_scheduled] (tid4) at (0.75,4.5){};
\node[circle, scale=0.75, fill, task_scheduled] (tid5) at (2.25,4.5){};
\draw[](tid1) -- (tid4);
\draw[](tid1) -- (tid5);
\node[circle, scale=0.75, fill] (tid2) at (3.75,3){};
\node[circle, scale=0.75, fill] (tid6) at (3.75,4.5){};
\draw[](tid2) -- (tid6);
\node[circle, scale=0.75, fill] (tid3) at (5.25,3){};
\node[circle, scale=0.75, fill] (tid7) at (5.25,4.5){};
\draw[](tid3) -- (tid7);
\draw[](tid0) -- (tid1);
\draw[](tid0) -- (tid2);
\draw[](tid0) -- (tid3);
\end{tikzpicture}
\nodepart{three}
\footnotesize{5.125}
\nodepart{four}
\footnotesize{$1$}
};
 \\ 
\node[draw=black, rectangle split,  rectangle split parts=4] (sn0x22ae720){
\footnotesize{43.4375}
\nodepart{two}
\begin{tikzpicture}[scale=.2]
\node[circle, scale=0.75, fill] (tid0) at (3,1.5){};
\node[circle, scale=0.75, fill] (tid1) at (1.5,3){};
\node[circle, scale=0.75, fill, task_scheduled] (tid4) at (0.75,4.5){};
\node[circle, scale=0.75, fill] (tid5) at (2.25,4.5){};
\draw[](tid1) -- (tid4);
\draw[](tid1) -- (tid5);
\node[circle, scale=0.75, fill] (tid2) at (3.75,3){};
\node[circle, scale=0.75, fill, task_scheduled] (tid6) at (3.75,4.5){};
\draw[](tid2) -- (tid6);
\node[circle, scale=0.75, fill] (tid3) at (5.25,3){};
\node[circle, scale=0.75, fill] (tid7) at (5.25,4.5){};
\draw[](tid3) -- (tid7);
\draw[](tid0) -- (tid1);
\draw[](tid0) -- (tid2);
\draw[](tid0) -- (tid3);
\end{tikzpicture}
\nodepart{three}
\footnotesize{5.125}
\nodepart{four}
\footnotesize{$25\:25\:50$}
};
 \\ 
\node[draw=black, rectangle split,  rectangle split parts=4] (sn0x22b2d60){
\footnotesize{10.5382}
\nodepart{two}
\begin{tikzpicture}[scale=.2]
\node[circle, scale=0.75, fill] (tid0) at (3,1.5){};
\node[circle, scale=0.75, fill] (tid1) at (1.5,3){};
\node[circle, scale=0.75, fill] (tid4) at (0.75,4.5){};
\node[circle, scale=0.75, fill] (tid5) at (2.25,4.5){};
\draw[](tid1) -- (tid4);
\draw[](tid1) -- (tid5);
\node[circle, scale=0.75, fill] (tid2) at (3.75,3){};
\node[circle, scale=0.75, fill, task_scheduled] (tid6) at (3.75,4.5){};
\draw[](tid2) -- (tid6);
\node[circle, scale=0.75, fill] (tid3) at (5.25,3){};
\node[circle, scale=0.75, fill, task_scheduled] (tid7) at (5.25,4.5){};
\draw[](tid3) -- (tid7);
\draw[](tid0) -- (tid1);
\draw[](tid0) -- (tid2);
\draw[](tid0) -- (tid3);
\end{tikzpicture}
\nodepart{three}
\footnotesize{5.125}
\nodepart{four}
\footnotesize{$1$}
};
 \\ 
\\
};
\end{scope}
\begin{scope}[yshift=\leveltopIIIII cm, anchor = center]
\matrix (line5)[row sep=0.5cm] {
\node[draw=black, rectangle split,  rectangle split parts=4] (sn0x22bb120){
\footnotesize{1.5625}
\nodepart{two}
\begin{tikzpicture}[scale=.2]
\node[circle, scale=0.75, fill] (tid0) at (3,1.5){};
\node[circle, scale=0.75, fill] (tid1) at (1.5,3){};
\node[circle, scale=0.75, fill] (tid4) at (0.75,4.5){};
\node[circle, scale=0.75, fill, task_scheduled] (tid6) at (0.75,6){};
\draw[](tid4) -- (tid6);
\node[circle, scale=0.75, fill, task_scheduled] (tid5) at (2.25,4.5){};
\draw[](tid1) -- (tid4);
\draw[](tid1) -- (tid5);
\node[circle, scale=0.75, fill] (tid2) at (3.75,3){};
\node[circle, scale=0.75, fill] (tid3) at (5.25,3){};
\draw[](tid0) -- (tid1);
\draw[](tid0) -- (tid2);
\draw[](tid0) -- (tid3);
\end{tikzpicture}
\nodepart{three}
\footnotesize{4.75}
\nodepart{four}
\footnotesize{$50\:50$}
};
 \\ 
\node[draw=black, rectangle split,  rectangle split parts=4] (sn0x22ae570){
\footnotesize{4.6875}
\nodepart{two}
\begin{tikzpicture}[scale=.2]
\node[circle, scale=0.75, fill] (tid0) at (2.25,1.5){};
\node[circle, scale=0.75, fill] (tid1) at (0.75,3){};
\node[circle, scale=0.75, fill] (tid4) at (0.75,4.5){};
\node[circle, scale=0.75, fill, task_scheduled] (tid6) at (0.75,6){};
\draw[](tid4) -- (tid6);
\draw[](tid1) -- (tid4);
\node[circle, scale=0.75, fill] (tid2) at (2.25,3){};
\node[circle, scale=0.75, fill, task_scheduled] (tid5) at (2.25,4.5){};
\draw[](tid2) -- (tid5);
\node[circle, scale=0.75, fill] (tid3) at (3.75,3){};
\draw[](tid0) -- (tid1);
\draw[](tid0) -- (tid2);
\draw[](tid0) -- (tid3);
\end{tikzpicture}
\nodepart{three}
\footnotesize{4.75}
\nodepart{four}
\footnotesize{$50\:50$}
};
 \\ 
\node[draw=black, rectangle split,  rectangle split parts=4] (sn0x22b40d0){
\footnotesize{0.625}
\nodepart{two}
\begin{tikzpicture}[scale=.2]
\node[circle, scale=0.75, fill] (tid0) at (3.75,1.5){};
\node[circle, scale=0.75, fill] (tid1) at (2.25,3){};
\node[circle, scale=0.75, fill, task_scheduled] (tid4) at (0.75,4.5){};
\node[circle, scale=0.75, fill, task_scheduled] (tid5) at (2.25,4.5){};
\node[circle, scale=0.75, fill] (tid6) at (3.75,4.5){};
\draw[](tid1) -- (tid4);
\draw[](tid1) -- (tid5);
\draw[](tid1) -- (tid6);
\node[circle, scale=0.75, fill] (tid2) at (5.25,3){};
\node[circle, scale=0.75, fill] (tid3) at (6.75,3){};
\draw[](tid0) -- (tid1);
\draw[](tid0) -- (tid2);
\draw[](tid0) -- (tid3);
\end{tikzpicture}
\nodepart{three}
\footnotesize{4.625}
\nodepart{four}
\footnotesize{$1$}
};
 \\ 
\node[draw=black, rectangle split,  rectangle split parts=4] (sn0x22b14e0){
\footnotesize{19.6094}
\nodepart{two}
\begin{tikzpicture}[scale=.2]
\node[circle, scale=0.75, fill] (tid0) at (3,1.5){};
\node[circle, scale=0.75, fill] (tid1) at (1.5,3){};
\node[circle, scale=0.75, fill, task_scheduled] (tid4) at (0.75,4.5){};
\node[circle, scale=0.75, fill, task_scheduled] (tid5) at (2.25,4.5){};
\draw[](tid1) -- (tid4);
\draw[](tid1) -- (tid5);
\node[circle, scale=0.75, fill] (tid2) at (3.75,3){};
\node[circle, scale=0.75, fill] (tid6) at (3.75,4.5){};
\draw[](tid2) -- (tid6);
\node[circle, scale=0.75, fill] (tid3) at (5.25,3){};
\draw[](tid0) -- (tid1);
\draw[](tid0) -- (tid2);
\draw[](tid0) -- (tid3);
\end{tikzpicture}
\nodepart{three}
\footnotesize{4.625}
\nodepart{four}
\footnotesize{$1$}
};
 \\ 
\node[draw=black, rectangle split,  rectangle split parts=4] (sn0x22b1ca0){
\footnotesize{38.533}
\nodepart{two}
\begin{tikzpicture}[scale=.2]
\node[circle, scale=0.75, fill] (tid0) at (3,1.5){};
\node[circle, scale=0.75, fill] (tid1) at (1.5,3){};
\node[circle, scale=0.75, fill, task_scheduled] (tid4) at (0.75,4.5){};
\node[circle, scale=0.75, fill] (tid5) at (2.25,4.5){};
\draw[](tid1) -- (tid4);
\draw[](tid1) -- (tid5);
\node[circle, scale=0.75, fill] (tid2) at (3.75,3){};
\node[circle, scale=0.75, fill, task_scheduled] (tid6) at (3.75,4.5){};
\draw[](tid2) -- (tid6);
\node[circle, scale=0.75, fill] (tid3) at (5.25,3){};
\draw[](tid0) -- (tid1);
\draw[](tid0) -- (tid2);
\draw[](tid0) -- (tid3);
\end{tikzpicture}
\nodepart{three}
\footnotesize{4.625}
\nodepart{four}
\footnotesize{$50\:50$}
};
 \\ 
\node[draw=black, rectangle split,  rectangle split parts=4] (sn0x22af010){
\footnotesize{34.9826}
\nodepart{two}
\begin{tikzpicture}[scale=.2]
\node[circle, scale=0.75, fill] (tid0) at (2.25,1.5){};
\node[circle, scale=0.75, fill] (tid1) at (0.75,3){};
\node[circle, scale=0.75, fill, task_scheduled] (tid4) at (0.75,4.5){};
\draw[](tid1) -- (tid4);
\node[circle, scale=0.75, fill] (tid2) at (2.25,3){};
\node[circle, scale=0.75, fill, task_scheduled] (tid5) at (2.25,4.5){};
\draw[](tid2) -- (tid5);
\node[circle, scale=0.75, fill] (tid3) at (3.75,3){};
\node[circle, scale=0.75, fill] (tid6) at (3.75,4.5){};
\draw[](tid3) -- (tid6);
\draw[](tid0) -- (tid1);
\draw[](tid0) -- (tid2);
\draw[](tid0) -- (tid3);
\end{tikzpicture}
\nodepart{three}
\footnotesize{4.625}
\nodepart{four}
\footnotesize{$1$}
};
 \\ 
\\
};
\end{scope}
\begin{scope}[yshift=\leveltopIIIIII cm, anchor = center]
\matrix (line6)[row sep=0.5cm] {
\node[draw=black, rectangle split,  rectangle split parts=4] (sn0x22aecc0){
\footnotesize{3.125}
\nodepart{two}
\begin{tikzpicture}[scale=.2]
\node[circle, scale=0.75, fill] (tid0) at (2.25,1.5){};
\node[circle, scale=0.75, fill] (tid1) at (0.75,3){};
\node[circle, scale=0.75, fill] (tid4) at (0.75,4.5){};
\node[circle, scale=0.75, fill, task_scheduled] (tid5) at (0.75,6){};
\draw[](tid4) -- (tid5);
\draw[](tid1) -- (tid4);
\node[circle, scale=0.75, fill, task_scheduled] (tid2) at (2.25,3){};
\node[circle, scale=0.75, fill] (tid3) at (3.75,3){};
\draw[](tid0) -- (tid1);
\draw[](tid0) -- (tid2);
\draw[](tid0) -- (tid3);
\end{tikzpicture}
\nodepart{three}
\footnotesize{4.375}
\nodepart{four}
\footnotesize{$50\:50$}
};
 \\ 
\node[draw=black, rectangle split,  rectangle split parts=4] (sn0x22b21c0){
\footnotesize{20.6727}
\nodepart{two}
\begin{tikzpicture}[scale=.2]
\node[circle, scale=0.75, fill] (tid0) at (3,1.5){};
\node[circle, scale=0.75, fill] (tid1) at (1.5,3){};
\node[circle, scale=0.75, fill, task_scheduled] (tid4) at (0.75,4.5){};
\node[circle, scale=0.75, fill, task_scheduled] (tid5) at (2.25,4.5){};
\draw[](tid1) -- (tid4);
\draw[](tid1) -- (tid5);
\node[circle, scale=0.75, fill] (tid2) at (3.75,3){};
\node[circle, scale=0.75, fill] (tid3) at (5.25,3){};
\draw[](tid0) -- (tid1);
\draw[](tid0) -- (tid2);
\draw[](tid0) -- (tid3);
\end{tikzpicture}
\nodepart{three}
\footnotesize{4.125}
\nodepart{four}
\footnotesize{$1$}
};
 \\ 
\node[draw=black, rectangle split,  rectangle split parts=4] (sn0x22af160){
\footnotesize{76.2023}
\nodepart{two}
\begin{tikzpicture}[scale=.2]
\node[circle, scale=0.75, fill] (tid0) at (2.25,1.5){};
\node[circle, scale=0.75, fill] (tid1) at (0.75,3){};
\node[circle, scale=0.75, fill, task_scheduled] (tid4) at (0.75,4.5){};
\draw[](tid1) -- (tid4);
\node[circle, scale=0.75, fill] (tid2) at (2.25,3){};
\node[circle, scale=0.75, fill, task_scheduled] (tid5) at (2.25,4.5){};
\draw[](tid2) -- (tid5);
\node[circle, scale=0.75, fill] (tid3) at (3.75,3){};
\draw[](tid0) -- (tid1);
\draw[](tid0) -- (tid2);
\draw[](tid0) -- (tid3);
\end{tikzpicture}
\nodepart{three}
\footnotesize{4.125}
\nodepart{four}
\footnotesize{$1$}
};
 \\ 
\\
};
\end{scope}
\begin{scope}[yshift=\leveltopIIIIIII cm, anchor = center]
\matrix (line7)[row sep=0.5cm] {
\node[draw=black, rectangle split,  rectangle split parts=4] (sn0x22af6c0){
\footnotesize{1.5625}
\nodepart{two}
\begin{tikzpicture}[scale=.2]
\node[circle, scale=0.75, fill] (tid0) at (1.5,1.5){};
\node[circle, scale=0.75, fill] (tid1) at (0.75,3){};
\node[circle, scale=0.75, fill] (tid3) at (0.75,4.5){};
\node[circle, scale=0.75, fill, task_scheduled] (tid4) at (0.75,6){};
\draw[](tid3) -- (tid4);
\draw[](tid1) -- (tid3);
\node[circle, scale=0.75, fill, task_scheduled] (tid2) at (2.25,3){};
\draw[](tid0) -- (tid1);
\draw[](tid0) -- (tid2);
\end{tikzpicture}
\nodepart{three}
\footnotesize{4.125}
\nodepart{four}
\footnotesize{$50\:50$}
};
 \\ 
\node[draw=black, rectangle split,  rectangle split parts=4] (sn0x22afb50){
\footnotesize{98.4375}
\nodepart{two}
\begin{tikzpicture}[scale=.2]
\node[circle, scale=0.75, fill] (tid0) at (2.25,1.5){};
\node[circle, scale=0.75, fill] (tid1) at (0.75,3){};
\node[circle, scale=0.75, fill, task_scheduled] (tid4) at (0.75,4.5){};
\draw[](tid1) -- (tid4);
\node[circle, scale=0.75, fill, task_scheduled] (tid2) at (2.25,3){};
\node[circle, scale=0.75, fill] (tid3) at (3.75,3){};
\draw[](tid0) -- (tid1);
\draw[](tid0) -- (tid2);
\draw[](tid0) -- (tid3);
\end{tikzpicture}
\nodepart{three}
\footnotesize{3.625}
\nodepart{four}
\footnotesize{$50\:50$}
};
 \\ 
\\
};
\end{scope}
\draw (sn0x22a97d0.east) -- (sn0x22b7700.west);
\draw (sn0x22a97d0.east) -- (sn0x22b8ae0.west);
\draw (sn0x22a97d0.east) -- (sn0x22b8c20.west);
\draw (sn0x22a97d0.east) -- (sn0x22b8400.west);
\draw (sn0x22a97d0.east) -- (sn0x22ab2c0.west);
\draw (sn0x22a97d0.east) -- (sn0x22b9170.west);
\draw (sn0x22b7700.east) -- (sn0x22b9da0.west);
\draw (sn0x22b7700.east) -- (sn0x22ba8b0.west);
\draw (sn0x22b7700.east) -- (sn0x22b54f0.west);
\draw (sn0x22b7700.east) -- (sn0x22b5d10.west);
\draw (sn0x22b7700.east) -- (sn0x22b6220.west);
\draw (sn0x22b8ae0.east) -- (sn0x22ac4a0.west);
\draw (sn0x22b8ae0.east) -- (sn0x22acd50.west);
\draw (sn0x22b8ae0.east) -- (sn0x22b5d10.west);
\draw (sn0x22b8ae0.east) -- (sn0x22b54f0.west);
\draw (sn0x22b8ae0.east) -- (sn0x22b6220.west);
\draw (sn0x22b8c20.east) -- (sn0x22bb880.west);
\draw (sn0x22b8c20.east) -- (sn0x22bc030.west);
\draw (sn0x22b8c20.east) -- (sn0x22b6220.west);
\draw (sn0x22b8400.east) -- (sn0x22b5d10.west);
\draw (sn0x22b8400.east) -- (sn0x22b54f0.west);
\draw (sn0x22b8400.east) -- (sn0x22b6220.west);
\draw (sn0x22ab2c0.east) -- (sn0x22b54f0.west);
\draw (sn0x22ab2c0.east) -- (sn0x22b5d10.west);
\draw (sn0x22ab2c0.east) -- (sn0x22b6220.west);
\draw (sn0x22ab2c0.east) -- (sn0x22ac710.west);
\draw (sn0x22ab2c0.east) -- (sn0x22ada30.west);
\draw (sn0x22b9170.east) -- (sn0x22b6220.west);
\draw (sn0x22b9170.east) -- (sn0x22bc5d0.west);
\draw (sn0x22b9170.east) -- (sn0x22b7500.west);
\draw (sn0x22bb880.east) -- (sn0x22b2b90.west);
\draw (sn0x22bb880.east) -- (sn0x22b66d0.west);
\draw (sn0x22bb880.east) -- (sn0x22b6c20.west);
\draw (sn0x22bc030.east) -- (sn0x22baab0.west);
\draw (sn0x22bc030.east) -- (sn0x22bafa0.west);
\draw (sn0x22bc030.east) -- (sn0x22b6c20.west);
\draw (sn0x22bc030.east) -- (sn0x22b66d0.west);
\draw (sn0x22b9da0.east) -- (sn0x22ad490.west);
\draw (sn0x22b9da0.east) -- (sn0x22ade60.west);
\draw (sn0x22b9da0.east) -- (sn0x22ae720.west);
\draw (sn0x22ba8b0.east) -- (sn0x22baab0.west);
\draw (sn0x22ba8b0.east) -- (sn0x22bafa0.west);
\draw (sn0x22ba8b0.east) -- (sn0x22ae720.west);
\draw (sn0x22ba8b0.east) -- (sn0x22b2d60.west);
\draw (sn0x22bc5d0.east) -- (sn0x22b66d0.west);
\draw (sn0x22bc5d0.east) -- (sn0x22b6c20.west);
\draw (sn0x22b7500.east) -- (sn0x22b6c20.west);
\draw (sn0x22b7500.east) -- (sn0x22b66d0.west);
\draw (sn0x22b7500.east) -- (sn0x22b34f0.west);
\draw (sn0x22b7500.east) -- (sn0x22b3cf0.west);
\draw (sn0x22ac710.east) -- (sn0x22ade60.west);
\draw (sn0x22ac710.east) -- (sn0x22ae720.west);
\draw (sn0x22ada30.east) -- (sn0x22ae720.west);
\draw (sn0x22ada30.east) -- (sn0x22b2d60.west);
\draw (sn0x22ada30.east) -- (sn0x22b34f0.west);
\draw (sn0x22ada30.east) -- (sn0x22b3cf0.west);
\draw (sn0x22ac4a0.east) -- (sn0x22ad490.west);
\draw (sn0x22ac4a0.east) -- (sn0x22ade60.west);
\draw (sn0x22ac4a0.east) -- (sn0x22ae720.west);
\draw (sn0x22acd50.east) -- (sn0x22b2b90.west);
\draw (sn0x22acd50.east) -- (sn0x22ae720.west);
\draw (sn0x22acd50.east) -- (sn0x22b2d60.west);
\draw (sn0x22b54f0.east) -- (sn0x22ae720.west);
\draw (sn0x22b54f0.east) -- (sn0x22ade60.west);
\draw (sn0x22b5d10.east) -- (sn0x22ae720.west);
\draw (sn0x22b5d10.east) -- (sn0x22b2d60.west);
\draw (sn0x22b6220.east) -- (sn0x22b2d60.west);
\draw (sn0x22b6220.east) -- (sn0x22ae720.west);
\draw (sn0x22b6220.east) -- (sn0x22b66d0.west);
\draw (sn0x22b6220.east) -- (sn0x22b6c20.west);
\draw (sn0x22baab0.east) -- (sn0x22ae570.west);
\draw (sn0x22baab0.east) -- (sn0x22b14e0.west);
\draw (sn0x22baab0.east) -- (sn0x22b1ca0.west);
\draw (sn0x22bafa0.east) -- (sn0x22bb120.west);
\draw (sn0x22bafa0.east) -- (sn0x22b1ca0.west);
\draw (sn0x22ad490.east) -- (sn0x22ae570.west);
\draw (sn0x22ad490.east) -- (sn0x22af010.west);
\draw (sn0x22b34f0.east) -- (sn0x22b14e0.west);
\draw (sn0x22b34f0.east) -- (sn0x22b1ca0.west);
\draw (sn0x22b3cf0.east) -- (sn0x22b1ca0.west);
\draw (sn0x22b3cf0.east) -- (sn0x22b40d0.west);
\draw (sn0x22b2b90.east) -- (sn0x22ae570.west);
\draw (sn0x22b2b90.east) -- (sn0x22b14e0.west);
\draw (sn0x22b2b90.east) -- (sn0x22b1ca0.west);
\draw (sn0x22b66d0.east) -- (sn0x22b1ca0.west);
\draw (sn0x22b6c20.east) -- (sn0x22b1ca0.west);
\draw (sn0x22b6c20.east) -- (sn0x22b14e0.west);
\draw (sn0x22ade60.east) -- (sn0x22af010.west);
\draw (sn0x22ae720.east) -- (sn0x22af010.west);
\draw (sn0x22ae720.east) -- (sn0x22b14e0.west);
\draw (sn0x22ae720.east) -- (sn0x22b1ca0.west);
\draw (sn0x22b2d60.east) -- (sn0x22b1ca0.west);
\draw (sn0x22bb120.east) -- (sn0x22aecc0.west);
\draw (sn0x22bb120.east) -- (sn0x22b21c0.west);
\draw (sn0x22ae570.east) -- (sn0x22aecc0.west);
\draw (sn0x22ae570.east) -- (sn0x22af160.west);
\draw (sn0x22b40d0.east) -- (sn0x22b21c0.west);
\draw (sn0x22b14e0.east) -- (sn0x22af160.west);
\draw (sn0x22b1ca0.east) -- (sn0x22af160.west);
\draw (sn0x22b1ca0.east) -- (sn0x22b21c0.west);
\draw (sn0x22af010.east) -- (sn0x22af160.west);
\draw (sn0x22aecc0.east) -- (sn0x22af6c0.west);
\draw (sn0x22aecc0.east) -- (sn0x22afb50.west);
\draw (sn0x22b21c0.east) -- (sn0x22afb50.west);
\draw (sn0x22af160.east) -- (sn0x22afb50.west);
\end{tikzpicture}

%%% Local Variables:
%%% TeX-master: "thesis/thesis.tex"
%%% End: 
\renewcommand{\leveltopI}{-10cm + \leveltop}
\renewcommand{\leveltopII}{-10cm + \leveltopI}
\renewcommand{\leveltopIII}{-10cm + \leveltopII}
\renewcommand{\leveltopIIII}{-10cm + \leveltopIII}
\renewcommand{\leveltopIIIII}{-10cm + \leveltopIIII}
\renewcommand{\leveltopIIIIII}{-10cm + \leveltopIIIII}
\renewcommand{\leveltopIIIIIII}{-10cm + \leveltopIIIIII}
\renewcommand{\leveltopIIIIIIII}{-10cm + \leveltopIIIIIII}
\renewcommand{\leveltopIIIIIIIII}{-10cm + \leveltopIIIIIIII}
\renewcommand{\leveltopIIIIIIIIII}{-10cm + \leveltopIIIIIIIII}
\renewcommand{\leveltopIIIIIIIIIII}{-10cm + \leveltopIIIIIIIIII}
\begin{tikzpicture}[scale=.2, anchor=south, rotate=90]
\begin{scope}[yshift=\leveltopI cm, anchor = center]
\matrix (line1)[row sep=0.5cm] {
\node[draw=black, rectangle split,  rectangle split parts=4] (sn0x22a97d0){
\footnotesize{100}
\nodepart{two}
\begin{tikzpicture}[scale=.2]
\node[circle, scale=0.75, fill] (tid0) at (4.5,1.5){};
\node[circle, scale=0.75, fill] (tid1) at (2.25,3){};
\node[circle, scale=0.75, fill, task_scheduled] (tid4) at (0.75,4.5){};
\node[circle, scale=0.75, fill] (tid5) at (2.25,4.5){};
\node[circle, scale=0.75, fill] (tid6) at (3.75,4.5){};
\draw[](tid1) -- (tid4);
\draw[](tid1) -- (tid5);
\draw[](tid1) -- (tid6);
\node[circle, scale=0.75, fill] (tid2) at (6,3){};
\node[circle, scale=0.75, fill] (tid7) at (5.25,4.5){};
\node[circle, scale=0.75, fill, task_scheduled] (tid10) at (5.25,6){};
\draw[](tid7) -- (tid10);
\node[circle, scale=0.75, fill] (tid8) at (6.75,4.5){};
\draw[](tid2) -- (tid7);
\draw[](tid2) -- (tid8);
\node[circle, scale=0.75, fill] (tid3) at (8.25,3){};
\node[circle, scale=0.75, fill] (tid9) at (8.25,4.5){};
\draw[](tid3) -- (tid9);
\draw[](tid0) -- (tid1);
\draw[](tid0) -- (tid2);
\draw[](tid0) -- (tid3);
\end{tikzpicture}
\nodepart{three}
\footnotesize{6.63281}
\nodepart{four}
\footnotesize{$25\:12\:12\:20\:20\:10$}
};
 \\ 
\\
};
\end{scope}
\begin{scope}[yshift=\leveltopII cm, anchor = center]
\matrix (line2)[row sep=0.5cm] {
\node[draw=black, rectangle split,  rectangle split parts=4] (sn0x22b7700){
\footnotesize{25}
\nodepart{two}
\begin{tikzpicture}[scale=.2]
\node[circle, scale=0.75, fill] (tid0) at (3.75,1.5){};
\node[circle, scale=0.75, fill] (tid1) at (1.5,3){};
\node[circle, scale=0.75, fill] (tid4) at (0.75,4.5){};
\node[circle, scale=0.75, fill, task_scheduled] (tid9) at (0.75,6){};
\draw[](tid4) -- (tid9);
\node[circle, scale=0.75, fill] (tid5) at (2.25,4.5){};
\draw[](tid1) -- (tid4);
\draw[](tid1) -- (tid5);
\node[circle, scale=0.75, fill] (tid2) at (4.5,3){};
\node[circle, scale=0.75, fill, task_scheduled] (tid6) at (3.75,4.5){};
\node[circle, scale=0.75, fill] (tid7) at (5.25,4.5){};
\draw[](tid2) -- (tid6);
\draw[](tid2) -- (tid7);
\node[circle, scale=0.75, fill] (tid3) at (6.75,3){};
\node[circle, scale=0.75, fill] (tid8) at (6.75,4.5){};
\draw[](tid3) -- (tid8);
\draw[](tid0) -- (tid1);
\draw[](tid0) -- (tid2);
\draw[](tid0) -- (tid3);
\end{tikzpicture}
\nodepart{three}
\footnotesize{6.14062}
\nodepart{four}
\footnotesize{$17\:33\:25\:12\:12$}
};
 \\ 
\node[draw=black, rectangle split,  rectangle split parts=4] (sn0x22b8ae0){
\footnotesize{12.5}
\nodepart{two}
\begin{tikzpicture}[scale=.2]
\node[circle, scale=0.75, fill] (tid0) at (3.75,1.5){};
\node[circle, scale=0.75, fill] (tid1) at (1.5,3){};
\node[circle, scale=0.75, fill] (tid4) at (0.75,4.5){};
\node[circle, scale=0.75, fill, task_scheduled] (tid9) at (0.75,6){};
\draw[](tid4) -- (tid9);
\node[circle, scale=0.75, fill, task_scheduled] (tid5) at (2.25,4.5){};
\draw[](tid1) -- (tid4);
\draw[](tid1) -- (tid5);
\node[circle, scale=0.75, fill] (tid2) at (4.5,3){};
\node[circle, scale=0.75, fill] (tid6) at (3.75,4.5){};
\node[circle, scale=0.75, fill] (tid7) at (5.25,4.5){};
\draw[](tid2) -- (tid6);
\draw[](tid2) -- (tid7);
\node[circle, scale=0.75, fill] (tid3) at (6.75,3){};
\node[circle, scale=0.75, fill] (tid8) at (6.75,4.5){};
\draw[](tid3) -- (tid8);
\draw[](tid0) -- (tid1);
\draw[](tid0) -- (tid2);
\draw[](tid0) -- (tid3);
\end{tikzpicture}
\nodepart{three}
\footnotesize{6.14062}
\nodepart{four}
\footnotesize{$33\:17\:12\:25\:12$}
};
 \\ 
\node[draw=black, rectangle split,  rectangle split parts=4] (sn0x22b8c20){
\footnotesize{12.5}
\nodepart{two}
\begin{tikzpicture}[scale=.2]
\node[circle, scale=0.75, fill] (tid0) at (3.75,1.5){};
\node[circle, scale=0.75, fill] (tid1) at (1.5,3){};
\node[circle, scale=0.75, fill] (tid4) at (0.75,4.5){};
\node[circle, scale=0.75, fill, task_scheduled] (tid9) at (0.75,6){};
\draw[](tid4) -- (tid9);
\node[circle, scale=0.75, fill] (tid5) at (2.25,4.5){};
\draw[](tid1) -- (tid4);
\draw[](tid1) -- (tid5);
\node[circle, scale=0.75, fill] (tid2) at (4.5,3){};
\node[circle, scale=0.75, fill] (tid6) at (3.75,4.5){};
\node[circle, scale=0.75, fill] (tid7) at (5.25,4.5){};
\draw[](tid2) -- (tid6);
\draw[](tid2) -- (tid7);
\node[circle, scale=0.75, fill] (tid3) at (6.75,3){};
\node[circle, scale=0.75, fill, task_scheduled] (tid8) at (6.75,4.5){};
\draw[](tid3) -- (tid8);
\draw[](tid0) -- (tid1);
\draw[](tid0) -- (tid2);
\draw[](tid0) -- (tid3);
\end{tikzpicture}
\nodepart{three}
\footnotesize{6.14062}
\nodepart{four}
\footnotesize{$17\:33\:50$}
};
 \\ 
\node[draw=black, rectangle split,  rectangle split parts=4] (sn0x22b8400){
\footnotesize{20}
\nodepart{two}
\begin{tikzpicture}[scale=.2]
\node[circle, scale=0.75, fill] (tid0) at (4.5,1.5){};
\node[circle, scale=0.75, fill] (tid1) at (2.25,3){};
\node[circle, scale=0.75, fill, task_scheduled] (tid4) at (0.75,4.5){};
\node[circle, scale=0.75, fill, task_scheduled] (tid5) at (2.25,4.5){};
\node[circle, scale=0.75, fill] (tid6) at (3.75,4.5){};
\draw[](tid1) -- (tid4);
\draw[](tid1) -- (tid5);
\draw[](tid1) -- (tid6);
\node[circle, scale=0.75, fill] (tid2) at (6,3){};
\node[circle, scale=0.75, fill] (tid7) at (5.25,4.5){};
\node[circle, scale=0.75, fill] (tid8) at (6.75,4.5){};
\draw[](tid2) -- (tid7);
\draw[](tid2) -- (tid8);
\node[circle, scale=0.75, fill] (tid3) at (8.25,3){};
\node[circle, scale=0.75, fill] (tid9) at (8.25,4.5){};
\draw[](tid3) -- (tid9);
\draw[](tid0) -- (tid1);
\draw[](tid0) -- (tid2);
\draw[](tid0) -- (tid3);
\end{tikzpicture}
\nodepart{three}
\footnotesize{6.125}
\nodepart{four}
\footnotesize{$25\:50\:25$}
};
 \\ 
\node[draw=black, rectangle split,  rectangle split parts=4] (sn0x22ab2c0){
\footnotesize{20}
\nodepart{two}
\begin{tikzpicture}[scale=.2]
\node[circle, scale=0.75, fill] (tid0) at (4.5,1.5){};
\node[circle, scale=0.75, fill] (tid1) at (2.25,3){};
\node[circle, scale=0.75, fill, task_scheduled] (tid4) at (0.75,4.5){};
\node[circle, scale=0.75, fill] (tid5) at (2.25,4.5){};
\node[circle, scale=0.75, fill] (tid6) at (3.75,4.5){};
\draw[](tid1) -- (tid4);
\draw[](tid1) -- (tid5);
\draw[](tid1) -- (tid6);
\node[circle, scale=0.75, fill] (tid2) at (6,3){};
\node[circle, scale=0.75, fill, task_scheduled] (tid7) at (5.25,4.5){};
\node[circle, scale=0.75, fill] (tid8) at (6.75,4.5){};
\draw[](tid2) -- (tid7);
\draw[](tid2) -- (tid8);
\node[circle, scale=0.75, fill] (tid3) at (8.25,3){};
\node[circle, scale=0.75, fill] (tid9) at (8.25,4.5){};
\draw[](tid3) -- (tid9);
\draw[](tid0) -- (tid1);
\draw[](tid0) -- (tid2);
\draw[](tid0) -- (tid3);
\end{tikzpicture}
\nodepart{three}
\footnotesize{6.125}
\nodepart{four}
\footnotesize{$25\:25\:25\:12\:12$}
};
 \\ 
\node[draw=black, rectangle split,  rectangle split parts=4] (sn0x22b9170){
\footnotesize{10}
\nodepart{two}
\begin{tikzpicture}[scale=.2]
\node[circle, scale=0.75, fill] (tid0) at (4.5,1.5){};
\node[circle, scale=0.75, fill] (tid1) at (2.25,3){};
\node[circle, scale=0.75, fill, task_scheduled] (tid4) at (0.75,4.5){};
\node[circle, scale=0.75, fill] (tid5) at (2.25,4.5){};
\node[circle, scale=0.75, fill] (tid6) at (3.75,4.5){};
\draw[](tid1) -- (tid4);
\draw[](tid1) -- (tid5);
\draw[](tid1) -- (tid6);
\node[circle, scale=0.75, fill] (tid2) at (6,3){};
\node[circle, scale=0.75, fill] (tid7) at (5.25,4.5){};
\node[circle, scale=0.75, fill] (tid8) at (6.75,4.5){};
\draw[](tid2) -- (tid7);
\draw[](tid2) -- (tid8);
\node[circle, scale=0.75, fill] (tid3) at (8.25,3){};
\node[circle, scale=0.75, fill, task_scheduled] (tid9) at (8.25,4.5){};
\draw[](tid3) -- (tid9);
\draw[](tid0) -- (tid1);
\draw[](tid0) -- (tid2);
\draw[](tid0) -- (tid3);
\end{tikzpicture}
\nodepart{three}
\footnotesize{6.125}
\nodepart{four}
\footnotesize{$25\:25\:50$}
};
 \\ 
\\
};
\end{scope}
\begin{scope}[yshift=\leveltopIII cm, anchor = center]
\matrix (line3)[row sep=0.5cm] {
\node[draw=black, rectangle split,  rectangle split parts=4] (sn0x22bb880){
\footnotesize{2.08333}
\nodepart{two}
\begin{tikzpicture}[scale=.2]
\node[circle, scale=0.75, fill] (tid0) at (3.75,1.5){};
\node[circle, scale=0.75, fill] (tid1) at (1.5,3){};
\node[circle, scale=0.75, fill] (tid4) at (0.75,4.5){};
\node[circle, scale=0.75, fill, task_scheduled] (tid8) at (0.75,6){};
\draw[](tid4) -- (tid8);
\node[circle, scale=0.75, fill, task_scheduled] (tid5) at (2.25,4.5){};
\draw[](tid1) -- (tid4);
\draw[](tid1) -- (tid5);
\node[circle, scale=0.75, fill] (tid2) at (4.5,3){};
\node[circle, scale=0.75, fill] (tid6) at (3.75,4.5){};
\node[circle, scale=0.75, fill] (tid7) at (5.25,4.5){};
\draw[](tid2) -- (tid6);
\draw[](tid2) -- (tid7);
\node[circle, scale=0.75, fill] (tid3) at (6.75,3){};
\draw[](tid0) -- (tid1);
\draw[](tid0) -- (tid2);
\draw[](tid0) -- (tid3);
\end{tikzpicture}
\nodepart{three}
\footnotesize{5.65625}
\nodepart{four}
\footnotesize{$50\:17\:33$}
};
 \\ 
\node[draw=black, rectangle split,  rectangle split parts=4] (sn0x22bc030){
\footnotesize{4.16667}
\nodepart{two}
\begin{tikzpicture}[scale=.2]
\node[circle, scale=0.75, fill] (tid0) at (3.75,1.5){};
\node[circle, scale=0.75, fill] (tid1) at (1.5,3){};
\node[circle, scale=0.75, fill] (tid4) at (0.75,4.5){};
\node[circle, scale=0.75, fill, task_scheduled] (tid8) at (0.75,6){};
\draw[](tid4) -- (tid8);
\node[circle, scale=0.75, fill] (tid5) at (2.25,4.5){};
\draw[](tid1) -- (tid4);
\draw[](tid1) -- (tid5);
\node[circle, scale=0.75, fill] (tid2) at (4.5,3){};
\node[circle, scale=0.75, fill, task_scheduled] (tid6) at (3.75,4.5){};
\node[circle, scale=0.75, fill] (tid7) at (5.25,4.5){};
\draw[](tid2) -- (tid6);
\draw[](tid2) -- (tid7);
\node[circle, scale=0.75, fill] (tid3) at (6.75,3){};
\draw[](tid0) -- (tid1);
\draw[](tid0) -- (tid2);
\draw[](tid0) -- (tid3);
\end{tikzpicture}
\nodepart{three}
\footnotesize{5.65625}
\nodepart{four}
\footnotesize{$25\:25\:33\:17$}
};
 \\ 
\node[draw=black, rectangle split,  rectangle split parts=4] (sn0x22b9da0){
\footnotesize{4.16667}
\nodepart{two}
\begin{tikzpicture}[scale=.2]
\node[circle, scale=0.75, fill] (tid0) at (3,1.5){};
\node[circle, scale=0.75, fill] (tid1) at (1.5,3){};
\node[circle, scale=0.75, fill] (tid4) at (0.75,4.5){};
\node[circle, scale=0.75, fill, task_scheduled] (tid8) at (0.75,6){};
\draw[](tid4) -- (tid8);
\node[circle, scale=0.75, fill, task_scheduled] (tid5) at (2.25,4.5){};
\draw[](tid1) -- (tid4);
\draw[](tid1) -- (tid5);
\node[circle, scale=0.75, fill] (tid2) at (3.75,3){};
\node[circle, scale=0.75, fill] (tid6) at (3.75,4.5){};
\draw[](tid2) -- (tid6);
\node[circle, scale=0.75, fill] (tid3) at (5.25,3){};
\node[circle, scale=0.75, fill] (tid7) at (5.25,4.5){};
\draw[](tid3) -- (tid7);
\draw[](tid0) -- (tid1);
\draw[](tid0) -- (tid2);
\draw[](tid0) -- (tid3);
\end{tikzpicture}
\nodepart{three}
\footnotesize{5.65625}
\nodepart{four}
\footnotesize{$50\:17\:33$}
};
 \\ 
\node[draw=black, rectangle split,  rectangle split parts=4] (sn0x22ba8b0){
\footnotesize{8.33333}
\nodepart{two}
\begin{tikzpicture}[scale=.2]
\node[circle, scale=0.75, fill] (tid0) at (3,1.5){};
\node[circle, scale=0.75, fill] (tid1) at (1.5,3){};
\node[circle, scale=0.75, fill] (tid4) at (0.75,4.5){};
\node[circle, scale=0.75, fill, task_scheduled] (tid8) at (0.75,6){};
\draw[](tid4) -- (tid8);
\node[circle, scale=0.75, fill] (tid5) at (2.25,4.5){};
\draw[](tid1) -- (tid4);
\draw[](tid1) -- (tid5);
\node[circle, scale=0.75, fill] (tid2) at (3.75,3){};
\node[circle, scale=0.75, fill, task_scheduled] (tid6) at (3.75,4.5){};
\draw[](tid2) -- (tid6);
\node[circle, scale=0.75, fill] (tid3) at (5.25,3){};
\node[circle, scale=0.75, fill] (tid7) at (5.25,4.5){};
\draw[](tid3) -- (tid7);
\draw[](tid0) -- (tid1);
\draw[](tid0) -- (tid2);
\draw[](tid0) -- (tid3);
\end{tikzpicture}
\nodepart{three}
\footnotesize{5.65625}
\nodepart{four}
\footnotesize{$25\:25\:33\:17$}
};
 \\ 
\node[draw=black, rectangle split,  rectangle split parts=4] (sn0x22bc5d0){
\footnotesize{2.5}
\nodepart{two}
\begin{tikzpicture}[scale=.2]
\node[circle, scale=0.75, fill] (tid0) at (4.5,1.5){};
\node[circle, scale=0.75, fill] (tid1) at (2.25,3){};
\node[circle, scale=0.75, fill, task_scheduled] (tid4) at (0.75,4.5){};
\node[circle, scale=0.75, fill, task_scheduled] (tid5) at (2.25,4.5){};
\node[circle, scale=0.75, fill] (tid6) at (3.75,4.5){};
\draw[](tid1) -- (tid4);
\draw[](tid1) -- (tid5);
\draw[](tid1) -- (tid6);
\node[circle, scale=0.75, fill] (tid2) at (6,3){};
\node[circle, scale=0.75, fill] (tid7) at (5.25,4.5){};
\node[circle, scale=0.75, fill] (tid8) at (6.75,4.5){};
\draw[](tid2) -- (tid7);
\draw[](tid2) -- (tid8);
\node[circle, scale=0.75, fill] (tid3) at (8.25,3){};
\draw[](tid0) -- (tid1);
\draw[](tid0) -- (tid2);
\draw[](tid0) -- (tid3);
\end{tikzpicture}
\nodepart{three}
\footnotesize{5.625}
\nodepart{four}
\footnotesize{$33\:67$}
};
 \\ 
\node[draw=black, rectangle split,  rectangle split parts=4] (sn0x22b7500){
\footnotesize{2.5}
\nodepart{two}
\begin{tikzpicture}[scale=.2]
\node[circle, scale=0.75, fill] (tid0) at (4.5,1.5){};
\node[circle, scale=0.75, fill] (tid1) at (2.25,3){};
\node[circle, scale=0.75, fill, task_scheduled] (tid4) at (0.75,4.5){};
\node[circle, scale=0.75, fill] (tid5) at (2.25,4.5){};
\node[circle, scale=0.75, fill] (tid6) at (3.75,4.5){};
\draw[](tid1) -- (tid4);
\draw[](tid1) -- (tid5);
\draw[](tid1) -- (tid6);
\node[circle, scale=0.75, fill] (tid2) at (6,3){};
\node[circle, scale=0.75, fill, task_scheduled] (tid7) at (5.25,4.5){};
\node[circle, scale=0.75, fill] (tid8) at (6.75,4.5){};
\draw[](tid2) -- (tid7);
\draw[](tid2) -- (tid8);
\node[circle, scale=0.75, fill] (tid3) at (8.25,3){};
\draw[](tid0) -- (tid1);
\draw[](tid0) -- (tid2);
\draw[](tid0) -- (tid3);
\end{tikzpicture}
\nodepart{three}
\footnotesize{5.625}
\nodepart{four}
\footnotesize{$33\:17\:33\:17$}
};
 \\ 
\node[draw=black, rectangle split,  rectangle split parts=4] (sn0x22ac710){
\footnotesize{5}
\nodepart{two}
\begin{tikzpicture}[scale=.2]
\node[circle, scale=0.75, fill] (tid0) at (3.75,1.5){};
\node[circle, scale=0.75, fill] (tid1) at (2.25,3){};
\node[circle, scale=0.75, fill, task_scheduled] (tid4) at (0.75,4.5){};
\node[circle, scale=0.75, fill, task_scheduled] (tid5) at (2.25,4.5){};
\node[circle, scale=0.75, fill] (tid6) at (3.75,4.5){};
\draw[](tid1) -- (tid4);
\draw[](tid1) -- (tid5);
\draw[](tid1) -- (tid6);
\node[circle, scale=0.75, fill] (tid2) at (5.25,3){};
\node[circle, scale=0.75, fill] (tid7) at (5.25,4.5){};
\draw[](tid2) -- (tid7);
\node[circle, scale=0.75, fill] (tid3) at (6.75,3){};
\node[circle, scale=0.75, fill] (tid8) at (6.75,4.5){};
\draw[](tid3) -- (tid8);
\draw[](tid0) -- (tid1);
\draw[](tid0) -- (tid2);
\draw[](tid0) -- (tid3);
\end{tikzpicture}
\nodepart{three}
\footnotesize{5.625}
\nodepart{four}
\footnotesize{$33\:67$}
};
 \\ 
\node[draw=black, rectangle split,  rectangle split parts=4] (sn0x22ada30){
\footnotesize{5}
\nodepart{two}
\begin{tikzpicture}[scale=.2]
\node[circle, scale=0.75, fill] (tid0) at (3.75,1.5){};
\node[circle, scale=0.75, fill] (tid1) at (2.25,3){};
\node[circle, scale=0.75, fill, task_scheduled] (tid4) at (0.75,4.5){};
\node[circle, scale=0.75, fill] (tid5) at (2.25,4.5){};
\node[circle, scale=0.75, fill] (tid6) at (3.75,4.5){};
\draw[](tid1) -- (tid4);
\draw[](tid1) -- (tid5);
\draw[](tid1) -- (tid6);
\node[circle, scale=0.75, fill] (tid2) at (5.25,3){};
\node[circle, scale=0.75, fill, task_scheduled] (tid7) at (5.25,4.5){};
\draw[](tid2) -- (tid7);
\node[circle, scale=0.75, fill] (tid3) at (6.75,3){};
\node[circle, scale=0.75, fill] (tid8) at (6.75,4.5){};
\draw[](tid3) -- (tid8);
\draw[](tid0) -- (tid1);
\draw[](tid0) -- (tid2);
\draw[](tid0) -- (tid3);
\end{tikzpicture}
\nodepart{three}
\footnotesize{5.625}
\nodepart{four}
\footnotesize{$33\:17\:33\:17$}
};
 \\ 
\node[draw=black, rectangle split,  rectangle split parts=4] (sn0x22ac4a0){
\footnotesize{4.16667}
\nodepart{two}
\begin{tikzpicture}[scale=.2]
\node[circle, scale=0.75, fill] (tid0) at (3,1.5){};
\node[circle, scale=0.75, fill] (tid1) at (1.5,3){};
\node[circle, scale=0.75, fill, task_scheduled] (tid4) at (0.75,4.5){};
\node[circle, scale=0.75, fill] (tid5) at (2.25,4.5){};
\draw[](tid1) -- (tid4);
\draw[](tid1) -- (tid5);
\node[circle, scale=0.75, fill] (tid2) at (3.75,3){};
\node[circle, scale=0.75, fill] (tid6) at (3.75,4.5){};
\node[circle, scale=0.75, fill, task_scheduled] (tid8) at (3.75,6){};
\draw[](tid6) -- (tid8);
\draw[](tid2) -- (tid6);
\node[circle, scale=0.75, fill] (tid3) at (5.25,3){};
\node[circle, scale=0.75, fill] (tid7) at (5.25,4.5){};
\draw[](tid3) -- (tid7);
\draw[](tid0) -- (tid1);
\draw[](tid0) -- (tid2);
\draw[](tid0) -- (tid3);
\end{tikzpicture}
\nodepart{three}
\footnotesize{5.65625}
\nodepart{four}
\footnotesize{$50\:17\:33$}
};
 \\ 
\node[draw=black, rectangle split,  rectangle split parts=4] (sn0x22acd50){
\footnotesize{2.08333}
\nodepart{two}
\begin{tikzpicture}[scale=.2]
\node[circle, scale=0.75, fill] (tid0) at (3,1.5){};
\node[circle, scale=0.75, fill] (tid1) at (1.5,3){};
\node[circle, scale=0.75, fill] (tid4) at (0.75,4.5){};
\node[circle, scale=0.75, fill] (tid5) at (2.25,4.5){};
\draw[](tid1) -- (tid4);
\draw[](tid1) -- (tid5);
\node[circle, scale=0.75, fill] (tid2) at (3.75,3){};
\node[circle, scale=0.75, fill] (tid6) at (3.75,4.5){};
\node[circle, scale=0.75, fill, task_scheduled] (tid8) at (3.75,6){};
\draw[](tid6) -- (tid8);
\draw[](tid2) -- (tid6);
\node[circle, scale=0.75, fill] (tid3) at (5.25,3){};
\node[circle, scale=0.75, fill, task_scheduled] (tid7) at (5.25,4.5){};
\draw[](tid3) -- (tid7);
\draw[](tid0) -- (tid1);
\draw[](tid0) -- (tid2);
\draw[](tid0) -- (tid3);
\end{tikzpicture}
\nodepart{three}
\footnotesize{5.65625}
\nodepart{four}
\footnotesize{$50\:33\:17$}
};
 \\ 
\node[draw=black, rectangle split,  rectangle split parts=4] (sn0x22b54f0){
\footnotesize{24.375}
\nodepart{two}
\begin{tikzpicture}[scale=.2]
\node[circle, scale=0.75, fill] (tid0) at (3.75,1.5){};
\node[circle, scale=0.75, fill] (tid1) at (1.5,3){};
\node[circle, scale=0.75, fill, task_scheduled] (tid4) at (0.75,4.5){};
\node[circle, scale=0.75, fill] (tid5) at (2.25,4.5){};
\draw[](tid1) -- (tid4);
\draw[](tid1) -- (tid5);
\node[circle, scale=0.75, fill] (tid2) at (4.5,3){};
\node[circle, scale=0.75, fill, task_scheduled] (tid6) at (3.75,4.5){};
\node[circle, scale=0.75, fill] (tid7) at (5.25,4.5){};
\draw[](tid2) -- (tid6);
\draw[](tid2) -- (tid7);
\node[circle, scale=0.75, fill] (tid3) at (6.75,3){};
\node[circle, scale=0.75, fill] (tid8) at (6.75,4.5){};
\draw[](tid3) -- (tid8);
\draw[](tid0) -- (tid1);
\draw[](tid0) -- (tid2);
\draw[](tid0) -- (tid3);
\end{tikzpicture}
\nodepart{three}
\footnotesize{5.625}
\nodepart{four}
\footnotesize{$67\:33$}
};
 \\ 
\node[draw=black, rectangle split,  rectangle split parts=4] (sn0x22b5d10){
\footnotesize{12.1875}
\nodepart{two}
\begin{tikzpicture}[scale=.2]
\node[circle, scale=0.75, fill] (tid0) at (3.75,1.5){};
\node[circle, scale=0.75, fill] (tid1) at (1.5,3){};
\node[circle, scale=0.75, fill, task_scheduled] (tid4) at (0.75,4.5){};
\node[circle, scale=0.75, fill, task_scheduled] (tid5) at (2.25,4.5){};
\draw[](tid1) -- (tid4);
\draw[](tid1) -- (tid5);
\node[circle, scale=0.75, fill] (tid2) at (4.5,3){};
\node[circle, scale=0.75, fill] (tid6) at (3.75,4.5){};
\node[circle, scale=0.75, fill] (tid7) at (5.25,4.5){};
\draw[](tid2) -- (tid6);
\draw[](tid2) -- (tid7);
\node[circle, scale=0.75, fill] (tid3) at (6.75,3){};
\node[circle, scale=0.75, fill] (tid8) at (6.75,4.5){};
\draw[](tid3) -- (tid8);
\draw[](tid0) -- (tid1);
\draw[](tid0) -- (tid2);
\draw[](tid0) -- (tid3);
\end{tikzpicture}
\nodepart{three}
\footnotesize{5.625}
\nodepart{four}
\footnotesize{$67\:33$}
};
 \\ 
\node[draw=black, rectangle split,  rectangle split parts=4] (sn0x22b6220){
\footnotesize{23.4375}
\nodepart{two}
\begin{tikzpicture}[scale=.2]
\node[circle, scale=0.75, fill] (tid0) at (3.75,1.5){};
\node[circle, scale=0.75, fill] (tid1) at (1.5,3){};
\node[circle, scale=0.75, fill, task_scheduled] (tid4) at (0.75,4.5){};
\node[circle, scale=0.75, fill] (tid5) at (2.25,4.5){};
\draw[](tid1) -- (tid4);
\draw[](tid1) -- (tid5);
\node[circle, scale=0.75, fill] (tid2) at (4.5,3){};
\node[circle, scale=0.75, fill] (tid6) at (3.75,4.5){};
\node[circle, scale=0.75, fill] (tid7) at (5.25,4.5){};
\draw[](tid2) -- (tid6);
\draw[](tid2) -- (tid7);
\node[circle, scale=0.75, fill] (tid3) at (6.75,3){};
\node[circle, scale=0.75, fill, task_scheduled] (tid8) at (6.75,4.5){};
\draw[](tid3) -- (tid8);
\draw[](tid0) -- (tid1);
\draw[](tid0) -- (tid2);
\draw[](tid0) -- (tid3);
\end{tikzpicture}
\nodepart{three}
\footnotesize{5.625}
\nodepart{four}
\footnotesize{$17\:33\:17\:33$}
};
 \\ 
\\
};
\end{scope}
\begin{scope}[yshift=\leveltopIIII cm, anchor = center]
\matrix (line4)[row sep=0.5cm] {
\node[draw=black, rectangle split,  rectangle split parts=4] (sn0x22baab0){
\footnotesize{3.125}
\nodepart{two}
\begin{tikzpicture}[scale=.2]
\node[circle, scale=0.75, fill] (tid0) at (3,1.5){};
\node[circle, scale=0.75, fill] (tid1) at (1.5,3){};
\node[circle, scale=0.75, fill] (tid4) at (0.75,4.5){};
\node[circle, scale=0.75, fill, task_scheduled] (tid7) at (0.75,6){};
\draw[](tid4) -- (tid7);
\node[circle, scale=0.75, fill, task_scheduled] (tid5) at (2.25,4.5){};
\draw[](tid1) -- (tid4);
\draw[](tid1) -- (tid5);
\node[circle, scale=0.75, fill] (tid2) at (3.75,3){};
\node[circle, scale=0.75, fill] (tid6) at (3.75,4.5){};
\draw[](tid2) -- (tid6);
\node[circle, scale=0.75, fill] (tid3) at (5.25,3){};
\draw[](tid0) -- (tid1);
\draw[](tid0) -- (tid2);
\draw[](tid0) -- (tid3);
\end{tikzpicture}
\nodepart{three}
\footnotesize{5.1875}
\nodepart{four}
\footnotesize{$50\:25\:25$}
};
 \\ 
\node[draw=black, rectangle split,  rectangle split parts=4] (sn0x22bafa0){
\footnotesize{3.125}
\nodepart{two}
\begin{tikzpicture}[scale=.2]
\node[circle, scale=0.75, fill] (tid0) at (3,1.5){};
\node[circle, scale=0.75, fill] (tid1) at (1.5,3){};
\node[circle, scale=0.75, fill] (tid4) at (0.75,4.5){};
\node[circle, scale=0.75, fill, task_scheduled] (tid7) at (0.75,6){};
\draw[](tid4) -- (tid7);
\node[circle, scale=0.75, fill] (tid5) at (2.25,4.5){};
\draw[](tid1) -- (tid4);
\draw[](tid1) -- (tid5);
\node[circle, scale=0.75, fill] (tid2) at (3.75,3){};
\node[circle, scale=0.75, fill, task_scheduled] (tid6) at (3.75,4.5){};
\draw[](tid2) -- (tid6);
\node[circle, scale=0.75, fill] (tid3) at (5.25,3){};
\draw[](tid0) -- (tid1);
\draw[](tid0) -- (tid2);
\draw[](tid0) -- (tid3);
\end{tikzpicture}
\nodepart{three}
\footnotesize{5.1875}
\nodepart{four}
\footnotesize{$50\:50$}
};
 \\ 
\node[draw=black, rectangle split,  rectangle split parts=4] (sn0x22ad490){
\footnotesize{4.16667}
\nodepart{two}
\begin{tikzpicture}[scale=.2]
\node[circle, scale=0.75, fill] (tid0) at (2.25,1.5){};
\node[circle, scale=0.75, fill] (tid1) at (0.75,3){};
\node[circle, scale=0.75, fill] (tid4) at (0.75,4.5){};
\node[circle, scale=0.75, fill, task_scheduled] (tid7) at (0.75,6){};
\draw[](tid4) -- (tid7);
\draw[](tid1) -- (tid4);
\node[circle, scale=0.75, fill] (tid2) at (2.25,3){};
\node[circle, scale=0.75, fill, task_scheduled] (tid5) at (2.25,4.5){};
\draw[](tid2) -- (tid5);
\node[circle, scale=0.75, fill] (tid3) at (3.75,3){};
\node[circle, scale=0.75, fill] (tid6) at (3.75,4.5){};
\draw[](tid3) -- (tid6);
\draw[](tid0) -- (tid1);
\draw[](tid0) -- (tid2);
\draw[](tid0) -- (tid3);
\end{tikzpicture}
\nodepart{three}
\footnotesize{5.1875}
\nodepart{four}
\footnotesize{$50\:50$}
};
 \\ 
\node[draw=black, rectangle split,  rectangle split parts=4] (sn0x22b34f0){
\footnotesize{2.5}
\nodepart{two}
\begin{tikzpicture}[scale=.2]
\node[circle, scale=0.75, fill] (tid0) at (3.75,1.5){};
\node[circle, scale=0.75, fill] (tid1) at (2.25,3){};
\node[circle, scale=0.75, fill, task_scheduled] (tid4) at (0.75,4.5){};
\node[circle, scale=0.75, fill, task_scheduled] (tid5) at (2.25,4.5){};
\node[circle, scale=0.75, fill] (tid6) at (3.75,4.5){};
\draw[](tid1) -- (tid4);
\draw[](tid1) -- (tid5);
\draw[](tid1) -- (tid6);
\node[circle, scale=0.75, fill] (tid2) at (5.25,3){};
\node[circle, scale=0.75, fill] (tid7) at (5.25,4.5){};
\draw[](tid2) -- (tid7);
\node[circle, scale=0.75, fill] (tid3) at (6.75,3){};
\draw[](tid0) -- (tid1);
\draw[](tid0) -- (tid2);
\draw[](tid0) -- (tid3);
\end{tikzpicture}
\nodepart{three}
\footnotesize{5.125}
\nodepart{four}
\footnotesize{$50\:50$}
};
 \\ 
\node[draw=black, rectangle split,  rectangle split parts=4] (sn0x22b3cf0){
\footnotesize{1.25}
\nodepart{two}
\begin{tikzpicture}[scale=.2]
\node[circle, scale=0.75, fill] (tid0) at (3.75,1.5){};
\node[circle, scale=0.75, fill] (tid1) at (2.25,3){};
\node[circle, scale=0.75, fill, task_scheduled] (tid4) at (0.75,4.5){};
\node[circle, scale=0.75, fill] (tid5) at (2.25,4.5){};
\node[circle, scale=0.75, fill] (tid6) at (3.75,4.5){};
\draw[](tid1) -- (tid4);
\draw[](tid1) -- (tid5);
\draw[](tid1) -- (tid6);
\node[circle, scale=0.75, fill] (tid2) at (5.25,3){};
\node[circle, scale=0.75, fill, task_scheduled] (tid7) at (5.25,4.5){};
\draw[](tid2) -- (tid7);
\node[circle, scale=0.75, fill] (tid3) at (6.75,3){};
\draw[](tid0) -- (tid1);
\draw[](tid0) -- (tid2);
\draw[](tid0) -- (tid3);
\end{tikzpicture}
\nodepart{three}
\footnotesize{5.125}
\nodepart{four}
\footnotesize{$50\:50$}
};
 \\ 
\node[draw=black, rectangle split,  rectangle split parts=4] (sn0x22b2b90){
\footnotesize{2.08333}
\nodepart{two}
\begin{tikzpicture}[scale=.2]
\node[circle, scale=0.75, fill] (tid0) at (3,1.5){};
\node[circle, scale=0.75, fill] (tid1) at (1.5,3){};
\node[circle, scale=0.75, fill, task_scheduled] (tid4) at (0.75,4.5){};
\node[circle, scale=0.75, fill] (tid5) at (2.25,4.5){};
\draw[](tid1) -- (tid4);
\draw[](tid1) -- (tid5);
\node[circle, scale=0.75, fill] (tid2) at (3.75,3){};
\node[circle, scale=0.75, fill] (tid6) at (3.75,4.5){};
\node[circle, scale=0.75, fill, task_scheduled] (tid7) at (3.75,6){};
\draw[](tid6) -- (tid7);
\draw[](tid2) -- (tid6);
\node[circle, scale=0.75, fill] (tid3) at (5.25,3){};
\draw[](tid0) -- (tid1);
\draw[](tid0) -- (tid2);
\draw[](tid0) -- (tid3);
\end{tikzpicture}
\nodepart{three}
\footnotesize{5.1875}
\nodepart{four}
\footnotesize{$50\:25\:25$}
};
 \\ 
\node[draw=black, rectangle split,  rectangle split parts=4] (sn0x22b66d0){
\footnotesize{6.19792}
\nodepart{two}
\begin{tikzpicture}[scale=.2]
\node[circle, scale=0.75, fill] (tid0) at (3.75,1.5){};
\node[circle, scale=0.75, fill] (tid1) at (1.5,3){};
\node[circle, scale=0.75, fill, task_scheduled] (tid4) at (0.75,4.5){};
\node[circle, scale=0.75, fill, task_scheduled] (tid5) at (2.25,4.5){};
\draw[](tid1) -- (tid4);
\draw[](tid1) -- (tid5);
\node[circle, scale=0.75, fill] (tid2) at (4.5,3){};
\node[circle, scale=0.75, fill] (tid6) at (3.75,4.5){};
\node[circle, scale=0.75, fill] (tid7) at (5.25,4.5){};
\draw[](tid2) -- (tid6);
\draw[](tid2) -- (tid7);
\node[circle, scale=0.75, fill] (tid3) at (6.75,3){};
\draw[](tid0) -- (tid1);
\draw[](tid0) -- (tid2);
\draw[](tid0) -- (tid3);
\end{tikzpicture}
\nodepart{three}
\footnotesize{5.125}
\nodepart{four}
\footnotesize{$1$}
};
 \\ 
\node[draw=black, rectangle split,  rectangle split parts=4] (sn0x22b6c20){
\footnotesize{12.3958}
\nodepart{two}
\begin{tikzpicture}[scale=.2]
\node[circle, scale=0.75, fill] (tid0) at (3.75,1.5){};
\node[circle, scale=0.75, fill] (tid1) at (1.5,3){};
\node[circle, scale=0.75, fill, task_scheduled] (tid4) at (0.75,4.5){};
\node[circle, scale=0.75, fill] (tid5) at (2.25,4.5){};
\draw[](tid1) -- (tid4);
\draw[](tid1) -- (tid5);
\node[circle, scale=0.75, fill] (tid2) at (4.5,3){};
\node[circle, scale=0.75, fill, task_scheduled] (tid6) at (3.75,4.5){};
\node[circle, scale=0.75, fill] (tid7) at (5.25,4.5){};
\draw[](tid2) -- (tid6);
\draw[](tid2) -- (tid7);
\node[circle, scale=0.75, fill] (tid3) at (6.75,3){};
\draw[](tid0) -- (tid1);
\draw[](tid0) -- (tid2);
\draw[](tid0) -- (tid3);
\end{tikzpicture}
\nodepart{three}
\footnotesize{5.125}
\nodepart{four}
\footnotesize{$50\:50$}
};
 \\ 
\node[draw=black, rectangle split,  rectangle split parts=4] (sn0x22ade60){
\footnotesize{11.1806}
\nodepart{two}
\begin{tikzpicture}[scale=.2]
\node[circle, scale=0.75, fill] (tid0) at (3,1.5){};
\node[circle, scale=0.75, fill] (tid1) at (1.5,3){};
\node[circle, scale=0.75, fill, task_scheduled] (tid4) at (0.75,4.5){};
\node[circle, scale=0.75, fill, task_scheduled] (tid5) at (2.25,4.5){};
\draw[](tid1) -- (tid4);
\draw[](tid1) -- (tid5);
\node[circle, scale=0.75, fill] (tid2) at (3.75,3){};
\node[circle, scale=0.75, fill] (tid6) at (3.75,4.5){};
\draw[](tid2) -- (tid6);
\node[circle, scale=0.75, fill] (tid3) at (5.25,3){};
\node[circle, scale=0.75, fill] (tid7) at (5.25,4.5){};
\draw[](tid3) -- (tid7);
\draw[](tid0) -- (tid1);
\draw[](tid0) -- (tid2);
\draw[](tid0) -- (tid3);
\end{tikzpicture}
\nodepart{three}
\footnotesize{5.125}
\nodepart{four}
\footnotesize{$1$}
};
 \\ 
\node[draw=black, rectangle split,  rectangle split parts=4] (sn0x22ae720){
\footnotesize{43.4375}
\nodepart{two}
\begin{tikzpicture}[scale=.2]
\node[circle, scale=0.75, fill] (tid0) at (3,1.5){};
\node[circle, scale=0.75, fill] (tid1) at (1.5,3){};
\node[circle, scale=0.75, fill, task_scheduled] (tid4) at (0.75,4.5){};
\node[circle, scale=0.75, fill] (tid5) at (2.25,4.5){};
\draw[](tid1) -- (tid4);
\draw[](tid1) -- (tid5);
\node[circle, scale=0.75, fill] (tid2) at (3.75,3){};
\node[circle, scale=0.75, fill, task_scheduled] (tid6) at (3.75,4.5){};
\draw[](tid2) -- (tid6);
\node[circle, scale=0.75, fill] (tid3) at (5.25,3){};
\node[circle, scale=0.75, fill] (tid7) at (5.25,4.5){};
\draw[](tid3) -- (tid7);
\draw[](tid0) -- (tid1);
\draw[](tid0) -- (tid2);
\draw[](tid0) -- (tid3);
\end{tikzpicture}
\nodepart{three}
\footnotesize{5.125}
\nodepart{four}
\footnotesize{$25\:25\:50$}
};
 \\ 
\node[draw=black, rectangle split,  rectangle split parts=4] (sn0x22b2d60){
\footnotesize{10.5382}
\nodepart{two}
\begin{tikzpicture}[scale=.2]
\node[circle, scale=0.75, fill] (tid0) at (3,1.5){};
\node[circle, scale=0.75, fill] (tid1) at (1.5,3){};
\node[circle, scale=0.75, fill] (tid4) at (0.75,4.5){};
\node[circle, scale=0.75, fill] (tid5) at (2.25,4.5){};
\draw[](tid1) -- (tid4);
\draw[](tid1) -- (tid5);
\node[circle, scale=0.75, fill] (tid2) at (3.75,3){};
\node[circle, scale=0.75, fill, task_scheduled] (tid6) at (3.75,4.5){};
\draw[](tid2) -- (tid6);
\node[circle, scale=0.75, fill] (tid3) at (5.25,3){};
\node[circle, scale=0.75, fill, task_scheduled] (tid7) at (5.25,4.5){};
\draw[](tid3) -- (tid7);
\draw[](tid0) -- (tid1);
\draw[](tid0) -- (tid2);
\draw[](tid0) -- (tid3);
\end{tikzpicture}
\nodepart{three}
\footnotesize{5.125}
\nodepart{four}
\footnotesize{$1$}
};
 \\ 
\\
};
\end{scope}
\begin{scope}[yshift=\leveltopIIIII cm, anchor = center]
\matrix (line5)[row sep=0.5cm] {
\node[draw=black, rectangle split,  rectangle split parts=4] (sn0x22bb120){
\footnotesize{1.5625}
\nodepart{two}
\begin{tikzpicture}[scale=.2]
\node[circle, scale=0.75, fill] (tid0) at (3,1.5){};
\node[circle, scale=0.75, fill] (tid1) at (1.5,3){};
\node[circle, scale=0.75, fill] (tid4) at (0.75,4.5){};
\node[circle, scale=0.75, fill, task_scheduled] (tid6) at (0.75,6){};
\draw[](tid4) -- (tid6);
\node[circle, scale=0.75, fill, task_scheduled] (tid5) at (2.25,4.5){};
\draw[](tid1) -- (tid4);
\draw[](tid1) -- (tid5);
\node[circle, scale=0.75, fill] (tid2) at (3.75,3){};
\node[circle, scale=0.75, fill] (tid3) at (5.25,3){};
\draw[](tid0) -- (tid1);
\draw[](tid0) -- (tid2);
\draw[](tid0) -- (tid3);
\end{tikzpicture}
\nodepart{three}
\footnotesize{4.75}
\nodepart{four}
\footnotesize{$50\:50$}
};
 \\ 
\node[draw=black, rectangle split,  rectangle split parts=4] (sn0x22ae570){
\footnotesize{4.6875}
\nodepart{two}
\begin{tikzpicture}[scale=.2]
\node[circle, scale=0.75, fill] (tid0) at (2.25,1.5){};
\node[circle, scale=0.75, fill] (tid1) at (0.75,3){};
\node[circle, scale=0.75, fill] (tid4) at (0.75,4.5){};
\node[circle, scale=0.75, fill, task_scheduled] (tid6) at (0.75,6){};
\draw[](tid4) -- (tid6);
\draw[](tid1) -- (tid4);
\node[circle, scale=0.75, fill] (tid2) at (2.25,3){};
\node[circle, scale=0.75, fill, task_scheduled] (tid5) at (2.25,4.5){};
\draw[](tid2) -- (tid5);
\node[circle, scale=0.75, fill] (tid3) at (3.75,3){};
\draw[](tid0) -- (tid1);
\draw[](tid0) -- (tid2);
\draw[](tid0) -- (tid3);
\end{tikzpicture}
\nodepart{three}
\footnotesize{4.75}
\nodepart{four}
\footnotesize{$50\:50$}
};
 \\ 
\node[draw=black, rectangle split,  rectangle split parts=4] (sn0x22b40d0){
\footnotesize{0.625}
\nodepart{two}
\begin{tikzpicture}[scale=.2]
\node[circle, scale=0.75, fill] (tid0) at (3.75,1.5){};
\node[circle, scale=0.75, fill] (tid1) at (2.25,3){};
\node[circle, scale=0.75, fill, task_scheduled] (tid4) at (0.75,4.5){};
\node[circle, scale=0.75, fill, task_scheduled] (tid5) at (2.25,4.5){};
\node[circle, scale=0.75, fill] (tid6) at (3.75,4.5){};
\draw[](tid1) -- (tid4);
\draw[](tid1) -- (tid5);
\draw[](tid1) -- (tid6);
\node[circle, scale=0.75, fill] (tid2) at (5.25,3){};
\node[circle, scale=0.75, fill] (tid3) at (6.75,3){};
\draw[](tid0) -- (tid1);
\draw[](tid0) -- (tid2);
\draw[](tid0) -- (tid3);
\end{tikzpicture}
\nodepart{three}
\footnotesize{4.625}
\nodepart{four}
\footnotesize{$1$}
};
 \\ 
\node[draw=black, rectangle split,  rectangle split parts=4] (sn0x22b14e0){
\footnotesize{19.6094}
\nodepart{two}
\begin{tikzpicture}[scale=.2]
\node[circle, scale=0.75, fill] (tid0) at (3,1.5){};
\node[circle, scale=0.75, fill] (tid1) at (1.5,3){};
\node[circle, scale=0.75, fill, task_scheduled] (tid4) at (0.75,4.5){};
\node[circle, scale=0.75, fill, task_scheduled] (tid5) at (2.25,4.5){};
\draw[](tid1) -- (tid4);
\draw[](tid1) -- (tid5);
\node[circle, scale=0.75, fill] (tid2) at (3.75,3){};
\node[circle, scale=0.75, fill] (tid6) at (3.75,4.5){};
\draw[](tid2) -- (tid6);
\node[circle, scale=0.75, fill] (tid3) at (5.25,3){};
\draw[](tid0) -- (tid1);
\draw[](tid0) -- (tid2);
\draw[](tid0) -- (tid3);
\end{tikzpicture}
\nodepart{three}
\footnotesize{4.625}
\nodepart{four}
\footnotesize{$1$}
};
 \\ 
\node[draw=black, rectangle split,  rectangle split parts=4] (sn0x22b1ca0){
\footnotesize{38.533}
\nodepart{two}
\begin{tikzpicture}[scale=.2]
\node[circle, scale=0.75, fill] (tid0) at (3,1.5){};
\node[circle, scale=0.75, fill] (tid1) at (1.5,3){};
\node[circle, scale=0.75, fill, task_scheduled] (tid4) at (0.75,4.5){};
\node[circle, scale=0.75, fill] (tid5) at (2.25,4.5){};
\draw[](tid1) -- (tid4);
\draw[](tid1) -- (tid5);
\node[circle, scale=0.75, fill] (tid2) at (3.75,3){};
\node[circle, scale=0.75, fill, task_scheduled] (tid6) at (3.75,4.5){};
\draw[](tid2) -- (tid6);
\node[circle, scale=0.75, fill] (tid3) at (5.25,3){};
\draw[](tid0) -- (tid1);
\draw[](tid0) -- (tid2);
\draw[](tid0) -- (tid3);
\end{tikzpicture}
\nodepart{three}
\footnotesize{4.625}
\nodepart{four}
\footnotesize{$50\:50$}
};
 \\ 
\node[draw=black, rectangle split,  rectangle split parts=4] (sn0x22af010){
\footnotesize{34.9826}
\nodepart{two}
\begin{tikzpicture}[scale=.2]
\node[circle, scale=0.75, fill] (tid0) at (2.25,1.5){};
\node[circle, scale=0.75, fill] (tid1) at (0.75,3){};
\node[circle, scale=0.75, fill, task_scheduled] (tid4) at (0.75,4.5){};
\draw[](tid1) -- (tid4);
\node[circle, scale=0.75, fill] (tid2) at (2.25,3){};
\node[circle, scale=0.75, fill, task_scheduled] (tid5) at (2.25,4.5){};
\draw[](tid2) -- (tid5);
\node[circle, scale=0.75, fill] (tid3) at (3.75,3){};
\node[circle, scale=0.75, fill] (tid6) at (3.75,4.5){};
\draw[](tid3) -- (tid6);
\draw[](tid0) -- (tid1);
\draw[](tid0) -- (tid2);
\draw[](tid0) -- (tid3);
\end{tikzpicture}
\nodepart{three}
\footnotesize{4.625}
\nodepart{four}
\footnotesize{$1$}
};
 \\ 
\\
};
\end{scope}
\begin{scope}[yshift=\leveltopIIIIII cm, anchor = center]
\matrix (line6)[row sep=0.5cm] {
\node[draw=black, rectangle split,  rectangle split parts=4] (sn0x22aecc0){
\footnotesize{3.125}
\nodepart{two}
\begin{tikzpicture}[scale=.2]
\node[circle, scale=0.75, fill] (tid0) at (2.25,1.5){};
\node[circle, scale=0.75, fill] (tid1) at (0.75,3){};
\node[circle, scale=0.75, fill] (tid4) at (0.75,4.5){};
\node[circle, scale=0.75, fill, task_scheduled] (tid5) at (0.75,6){};
\draw[](tid4) -- (tid5);
\draw[](tid1) -- (tid4);
\node[circle, scale=0.75, fill, task_scheduled] (tid2) at (2.25,3){};
\node[circle, scale=0.75, fill] (tid3) at (3.75,3){};
\draw[](tid0) -- (tid1);
\draw[](tid0) -- (tid2);
\draw[](tid0) -- (tid3);
\end{tikzpicture}
\nodepart{three}
\footnotesize{4.375}
\nodepart{four}
\footnotesize{$50\:50$}
};
 \\ 
\node[draw=black, rectangle split,  rectangle split parts=4] (sn0x22b21c0){
\footnotesize{20.6727}
\nodepart{two}
\begin{tikzpicture}[scale=.2]
\node[circle, scale=0.75, fill] (tid0) at (3,1.5){};
\node[circle, scale=0.75, fill] (tid1) at (1.5,3){};
\node[circle, scale=0.75, fill, task_scheduled] (tid4) at (0.75,4.5){};
\node[circle, scale=0.75, fill, task_scheduled] (tid5) at (2.25,4.5){};
\draw[](tid1) -- (tid4);
\draw[](tid1) -- (tid5);
\node[circle, scale=0.75, fill] (tid2) at (3.75,3){};
\node[circle, scale=0.75, fill] (tid3) at (5.25,3){};
\draw[](tid0) -- (tid1);
\draw[](tid0) -- (tid2);
\draw[](tid0) -- (tid3);
\end{tikzpicture}
\nodepart{three}
\footnotesize{4.125}
\nodepart{four}
\footnotesize{$1$}
};
 \\ 
\node[draw=black, rectangle split,  rectangle split parts=4] (sn0x22af160){
\footnotesize{76.2023}
\nodepart{two}
\begin{tikzpicture}[scale=.2]
\node[circle, scale=0.75, fill] (tid0) at (2.25,1.5){};
\node[circle, scale=0.75, fill] (tid1) at (0.75,3){};
\node[circle, scale=0.75, fill, task_scheduled] (tid4) at (0.75,4.5){};
\draw[](tid1) -- (tid4);
\node[circle, scale=0.75, fill] (tid2) at (2.25,3){};
\node[circle, scale=0.75, fill, task_scheduled] (tid5) at (2.25,4.5){};
\draw[](tid2) -- (tid5);
\node[circle, scale=0.75, fill] (tid3) at (3.75,3){};
\draw[](tid0) -- (tid1);
\draw[](tid0) -- (tid2);
\draw[](tid0) -- (tid3);
\end{tikzpicture}
\nodepart{three}
\footnotesize{4.125}
\nodepart{four}
\footnotesize{$1$}
};
 \\ 
\\
};
\end{scope}
\begin{scope}[yshift=\leveltopIIIIIII cm, anchor = center]
\matrix (line7)[row sep=0.5cm] {
\node[draw=black, rectangle split,  rectangle split parts=4] (sn0x22af6c0){
\footnotesize{1.5625}
\nodepart{two}
\begin{tikzpicture}[scale=.2]
\node[circle, scale=0.75, fill] (tid0) at (1.5,1.5){};
\node[circle, scale=0.75, fill] (tid1) at (0.75,3){};
\node[circle, scale=0.75, fill] (tid3) at (0.75,4.5){};
\node[circle, scale=0.75, fill, task_scheduled] (tid4) at (0.75,6){};
\draw[](tid3) -- (tid4);
\draw[](tid1) -- (tid3);
\node[circle, scale=0.75, fill, task_scheduled] (tid2) at (2.25,3){};
\draw[](tid0) -- (tid1);
\draw[](tid0) -- (tid2);
\end{tikzpicture}
\nodepart{three}
\footnotesize{4.125}
\nodepart{four}
\footnotesize{$50\:50$}
};
 \\ 
\node[draw=black, rectangle split,  rectangle split parts=4] (sn0x22afb50){
\footnotesize{98.4375}
\nodepart{two}
\begin{tikzpicture}[scale=.2]
\node[circle, scale=0.75, fill] (tid0) at (2.25,1.5){};
\node[circle, scale=0.75, fill] (tid1) at (0.75,3){};
\node[circle, scale=0.75, fill, task_scheduled] (tid4) at (0.75,4.5){};
\draw[](tid1) -- (tid4);
\node[circle, scale=0.75, fill, task_scheduled] (tid2) at (2.25,3){};
\node[circle, scale=0.75, fill] (tid3) at (3.75,3){};
\draw[](tid0) -- (tid1);
\draw[](tid0) -- (tid2);
\draw[](tid0) -- (tid3);
\end{tikzpicture}
\nodepart{three}
\footnotesize{3.625}
\nodepart{four}
\footnotesize{$50\:50$}
};
 \\ 
\\
};
\end{scope}
\draw (sn0x22a97d0.east) -- (sn0x22b7700.west);
\draw (sn0x22a97d0.east) -- (sn0x22b8ae0.west);
\draw (sn0x22a97d0.east) -- (sn0x22b8c20.west);
\draw (sn0x22a97d0.east) -- (sn0x22b8400.west);
\draw (sn0x22a97d0.east) -- (sn0x22ab2c0.west);
\draw (sn0x22a97d0.east) -- (sn0x22b9170.west);
\draw (sn0x22b7700.east) -- (sn0x22b9da0.west);
\draw (sn0x22b7700.east) -- (sn0x22ba8b0.west);
\draw (sn0x22b7700.east) -- (sn0x22b54f0.west);
\draw (sn0x22b7700.east) -- (sn0x22b5d10.west);
\draw (sn0x22b7700.east) -- (sn0x22b6220.west);
\draw (sn0x22b8ae0.east) -- (sn0x22ac4a0.west);
\draw (sn0x22b8ae0.east) -- (sn0x22acd50.west);
\draw (sn0x22b8ae0.east) -- (sn0x22b5d10.west);
\draw (sn0x22b8ae0.east) -- (sn0x22b54f0.west);
\draw (sn0x22b8ae0.east) -- (sn0x22b6220.west);
\draw (sn0x22b8c20.east) -- (sn0x22bb880.west);
\draw (sn0x22b8c20.east) -- (sn0x22bc030.west);
\draw (sn0x22b8c20.east) -- (sn0x22b6220.west);
\draw (sn0x22b8400.east) -- (sn0x22b5d10.west);
\draw (sn0x22b8400.east) -- (sn0x22b54f0.west);
\draw (sn0x22b8400.east) -- (sn0x22b6220.west);
\draw (sn0x22ab2c0.east) -- (sn0x22b54f0.west);
\draw (sn0x22ab2c0.east) -- (sn0x22b5d10.west);
\draw (sn0x22ab2c0.east) -- (sn0x22b6220.west);
\draw (sn0x22ab2c0.east) -- (sn0x22ac710.west);
\draw (sn0x22ab2c0.east) -- (sn0x22ada30.west);
\draw (sn0x22b9170.east) -- (sn0x22b6220.west);
\draw (sn0x22b9170.east) -- (sn0x22bc5d0.west);
\draw (sn0x22b9170.east) -- (sn0x22b7500.west);
\draw (sn0x22bb880.east) -- (sn0x22b2b90.west);
\draw (sn0x22bb880.east) -- (sn0x22b66d0.west);
\draw (sn0x22bb880.east) -- (sn0x22b6c20.west);
\draw (sn0x22bc030.east) -- (sn0x22baab0.west);
\draw (sn0x22bc030.east) -- (sn0x22bafa0.west);
\draw (sn0x22bc030.east) -- (sn0x22b6c20.west);
\draw (sn0x22bc030.east) -- (sn0x22b66d0.west);
\draw (sn0x22b9da0.east) -- (sn0x22ad490.west);
\draw (sn0x22b9da0.east) -- (sn0x22ade60.west);
\draw (sn0x22b9da0.east) -- (sn0x22ae720.west);
\draw (sn0x22ba8b0.east) -- (sn0x22baab0.west);
\draw (sn0x22ba8b0.east) -- (sn0x22bafa0.west);
\draw (sn0x22ba8b0.east) -- (sn0x22ae720.west);
\draw (sn0x22ba8b0.east) -- (sn0x22b2d60.west);
\draw (sn0x22bc5d0.east) -- (sn0x22b66d0.west);
\draw (sn0x22bc5d0.east) -- (sn0x22b6c20.west);
\draw (sn0x22b7500.east) -- (sn0x22b6c20.west);
\draw (sn0x22b7500.east) -- (sn0x22b66d0.west);
\draw (sn0x22b7500.east) -- (sn0x22b34f0.west);
\draw (sn0x22b7500.east) -- (sn0x22b3cf0.west);
\draw (sn0x22ac710.east) -- (sn0x22ade60.west);
\draw (sn0x22ac710.east) -- (sn0x22ae720.west);
\draw (sn0x22ada30.east) -- (sn0x22ae720.west);
\draw (sn0x22ada30.east) -- (sn0x22b2d60.west);
\draw (sn0x22ada30.east) -- (sn0x22b34f0.west);
\draw (sn0x22ada30.east) -- (sn0x22b3cf0.west);
\draw (sn0x22ac4a0.east) -- (sn0x22ad490.west);
\draw (sn0x22ac4a0.east) -- (sn0x22ade60.west);
\draw (sn0x22ac4a0.east) -- (sn0x22ae720.west);
\draw (sn0x22acd50.east) -- (sn0x22b2b90.west);
\draw (sn0x22acd50.east) -- (sn0x22ae720.west);
\draw (sn0x22acd50.east) -- (sn0x22b2d60.west);
\draw (sn0x22b54f0.east) -- (sn0x22ae720.west);
\draw (sn0x22b54f0.east) -- (sn0x22ade60.west);
\draw (sn0x22b5d10.east) -- (sn0x22ae720.west);
\draw (sn0x22b5d10.east) -- (sn0x22b2d60.west);
\draw (sn0x22b6220.east) -- (sn0x22b2d60.west);
\draw (sn0x22b6220.east) -- (sn0x22ae720.west);
\draw (sn0x22b6220.east) -- (sn0x22b66d0.west);
\draw (sn0x22b6220.east) -- (sn0x22b6c20.west);
\draw (sn0x22baab0.east) -- (sn0x22ae570.west);
\draw (sn0x22baab0.east) -- (sn0x22b14e0.west);
\draw (sn0x22baab0.east) -- (sn0x22b1ca0.west);
\draw (sn0x22bafa0.east) -- (sn0x22bb120.west);
\draw (sn0x22bafa0.east) -- (sn0x22b1ca0.west);
\draw (sn0x22ad490.east) -- (sn0x22ae570.west);
\draw (sn0x22ad490.east) -- (sn0x22af010.west);
\draw (sn0x22b34f0.east) -- (sn0x22b14e0.west);
\draw (sn0x22b34f0.east) -- (sn0x22b1ca0.west);
\draw (sn0x22b3cf0.east) -- (sn0x22b1ca0.west);
\draw (sn0x22b3cf0.east) -- (sn0x22b40d0.west);
\draw (sn0x22b2b90.east) -- (sn0x22ae570.west);
\draw (sn0x22b2b90.east) -- (sn0x22b14e0.west);
\draw (sn0x22b2b90.east) -- (sn0x22b1ca0.west);
\draw (sn0x22b66d0.east) -- (sn0x22b1ca0.west);
\draw (sn0x22b6c20.east) -- (sn0x22b1ca0.west);
\draw (sn0x22b6c20.east) -- (sn0x22b14e0.west);
\draw (sn0x22ade60.east) -- (sn0x22af010.west);
\draw (sn0x22ae720.east) -- (sn0x22af010.west);
\draw (sn0x22ae720.east) -- (sn0x22b14e0.west);
\draw (sn0x22ae720.east) -- (sn0x22b1ca0.west);
\draw (sn0x22b2d60.east) -- (sn0x22b1ca0.west);
\draw (sn0x22bb120.east) -- (sn0x22aecc0.west);
\draw (sn0x22bb120.east) -- (sn0x22b21c0.west);
\draw (sn0x22ae570.east) -- (sn0x22aecc0.west);
\draw (sn0x22ae570.east) -- (sn0x22af160.west);
\draw (sn0x22b40d0.east) -- (sn0x22b21c0.west);
\draw (sn0x22b14e0.east) -- (sn0x22af160.west);
\draw (sn0x22b1ca0.east) -- (sn0x22af160.west);
\draw (sn0x22b1ca0.east) -- (sn0x22b21c0.west);
\draw (sn0x22af010.east) -- (sn0x22af160.west);
\draw (sn0x22aecc0.east) -- (sn0x22af6c0.west);
\draw (sn0x22aecc0.east) -- (sn0x22afb50.west);
\draw (sn0x22b21c0.east) -- (sn0x22afb50.west);
\draw (sn0x22af160.east) -- (sn0x22afb50.west);
\end{tikzpicture}

%%% Local Variables:
%%% TeX-master: "thesis/thesis.tex"
%%% End: 
\renewcommand{\leveltopI}{-10cm + \leveltop}
\renewcommand{\leveltopII}{-10cm + \leveltopI}
\renewcommand{\leveltopIII}{-10cm + \leveltopII}
\renewcommand{\leveltopIIII}{-10cm + \leveltopIII}
\renewcommand{\leveltopIIIII}{-10cm + \leveltopIIII}
\renewcommand{\leveltopIIIIII}{-10cm + \leveltopIIIII}
\renewcommand{\leveltopIIIIIII}{-10cm + \leveltopIIIIII}
\renewcommand{\leveltopIIIIIIII}{-10cm + \leveltopIIIIIII}
\renewcommand{\leveltopIIIIIIIII}{-10cm + \leveltopIIIIIIII}
\renewcommand{\leveltopIIIIIIIIII}{-10cm + \leveltopIIIIIIIII}
\renewcommand{\leveltopIIIIIIIIIII}{-10cm + \leveltopIIIIIIIIII}
\begin{tikzpicture}[scale=.2, anchor=south, rotate=90]
\begin{scope}[yshift=\leveltopI cm, anchor = center]
\matrix (line1)[row sep=0.5cm] {
\node[draw=black, rectangle split,  rectangle split parts=4] (sn0x22aa490){
\footnotesize{100}
\nodepart{two}
\begin{tikzpicture}[scale=.2]
\node[circle, scale=0.75, fill] (tid0) at (4.5,1.5){};
\node[circle, scale=0.75, fill] (tid1) at (2.25,3){};
\node[circle, scale=0.75, fill] (tid4) at (0.75,4.5){};
\node[circle, scale=0.75, fill] (tid5) at (2.25,4.5){};
\node[circle, scale=0.75, fill] (tid6) at (3.75,4.5){};
\draw[](tid1) -- (tid4);
\draw[](tid1) -- (tid5);
\draw[](tid1) -- (tid6);
\node[circle, scale=0.75, fill] (tid2) at (6,3){};
\node[circle, scale=0.75, fill] (tid7) at (5.25,4.5){};
\node[circle, scale=0.75, fill, task_scheduled] (tid10) at (5.25,6){};
\draw[](tid7) -- (tid10);
\node[circle, scale=0.75, fill] (tid8) at (6.75,4.5){};
\draw[](tid2) -- (tid7);
\draw[](tid2) -- (tid8);
\node[circle, scale=0.75, fill] (tid3) at (8.25,3){};
\node[circle, scale=0.75, fill, task_scheduled] (tid9) at (8.25,4.5){};
\draw[](tid3) -- (tid9);
\draw[](tid0) -- (tid1);
\draw[](tid0) -- (tid2);
\draw[](tid0) -- (tid3);
\end{tikzpicture}
\nodepart{three}
\footnotesize{6.63281}
\nodepart{four}
\footnotesize{$38\:12\:30\:20$}
};
 \\ 
\\
};
\end{scope}
\begin{scope}[yshift=\leveltopII cm, anchor = center]
\matrix (line2)[row sep=0.5cm] {
\node[draw=black, rectangle split,  rectangle split parts=4] (sn0x22bcc60){
\footnotesize{37.5}
\nodepart{two}
\begin{tikzpicture}[scale=.2]
\node[circle, scale=0.75, fill] (tid0) at (4.5,1.5){};
\node[circle, scale=0.75, fill] (tid1) at (2.25,3){};
\node[circle, scale=0.75, fill, task_scheduled] (tid4) at (0.75,4.5){};
\node[circle, scale=0.75, fill] (tid5) at (2.25,4.5){};
\node[circle, scale=0.75, fill] (tid6) at (3.75,4.5){};
\draw[](tid1) -- (tid4);
\draw[](tid1) -- (tid5);
\draw[](tid1) -- (tid6);
\node[circle, scale=0.75, fill] (tid2) at (6,3){};
\node[circle, scale=0.75, fill] (tid7) at (5.25,4.5){};
\node[circle, scale=0.75, fill, task_scheduled] (tid9) at (5.25,6){};
\draw[](tid7) -- (tid9);
\node[circle, scale=0.75, fill] (tid8) at (6.75,4.5){};
\draw[](tid2) -- (tid7);
\draw[](tid2) -- (tid8);
\node[circle, scale=0.75, fill] (tid3) at (8.25,3){};
\draw[](tid0) -- (tid1);
\draw[](tid0) -- (tid2);
\draw[](tid0) -- (tid3);
\end{tikzpicture}
\nodepart{three}
\footnotesize{6.14062}
\nodepart{four}
\footnotesize{$33\:17\:25\:25$}
};
 \\ 
\node[draw=black, rectangle split,  rectangle split parts=4] (sn0x22bcd60){
\footnotesize{12.5}
\nodepart{two}
\begin{tikzpicture}[scale=.2]
\node[circle, scale=0.75, fill] (tid0) at (4.5,1.5){};
\node[circle, scale=0.75, fill] (tid1) at (2.25,3){};
\node[circle, scale=0.75, fill] (tid4) at (0.75,4.5){};
\node[circle, scale=0.75, fill] (tid5) at (2.25,4.5){};
\node[circle, scale=0.75, fill] (tid6) at (3.75,4.5){};
\draw[](tid1) -- (tid4);
\draw[](tid1) -- (tid5);
\draw[](tid1) -- (tid6);
\node[circle, scale=0.75, fill] (tid2) at (6,3){};
\node[circle, scale=0.75, fill] (tid7) at (5.25,4.5){};
\node[circle, scale=0.75, fill, task_scheduled] (tid9) at (5.25,6){};
\draw[](tid7) -- (tid9);
\node[circle, scale=0.75, fill, task_scheduled] (tid8) at (6.75,4.5){};
\draw[](tid2) -- (tid7);
\draw[](tid2) -- (tid8);
\node[circle, scale=0.75, fill] (tid3) at (8.25,3){};
\draw[](tid0) -- (tid1);
\draw[](tid0) -- (tid2);
\draw[](tid0) -- (tid3);
\end{tikzpicture}
\nodepart{three}
\footnotesize{6.14062}
\nodepart{four}
\footnotesize{$50\:38\:12$}
};
 \\ 
\node[draw=black, rectangle split,  rectangle split parts=4] (sn0x22b9170){
\footnotesize{30}
\nodepart{two}
\begin{tikzpicture}[scale=.2]
\node[circle, scale=0.75, fill] (tid0) at (4.5,1.5){};
\node[circle, scale=0.75, fill] (tid1) at (2.25,3){};
\node[circle, scale=0.75, fill, task_scheduled] (tid4) at (0.75,4.5){};
\node[circle, scale=0.75, fill] (tid5) at (2.25,4.5){};
\node[circle, scale=0.75, fill] (tid6) at (3.75,4.5){};
\draw[](tid1) -- (tid4);
\draw[](tid1) -- (tid5);
\draw[](tid1) -- (tid6);
\node[circle, scale=0.75, fill] (tid2) at (6,3){};
\node[circle, scale=0.75, fill] (tid7) at (5.25,4.5){};
\node[circle, scale=0.75, fill] (tid8) at (6.75,4.5){};
\draw[](tid2) -- (tid7);
\draw[](tid2) -- (tid8);
\node[circle, scale=0.75, fill] (tid3) at (8.25,3){};
\node[circle, scale=0.75, fill, task_scheduled] (tid9) at (8.25,4.5){};
\draw[](tid3) -- (tid9);
\draw[](tid0) -- (tid1);
\draw[](tid0) -- (tid2);
\draw[](tid0) -- (tid3);
\end{tikzpicture}
\nodepart{three}
\footnotesize{6.125}
\nodepart{four}
\footnotesize{$25\:25\:50$}
};
 \\ 
\node[draw=black, rectangle split,  rectangle split parts=4] (sn0x22ab500){
\footnotesize{20}
\nodepart{two}
\begin{tikzpicture}[scale=.2]
\node[circle, scale=0.75, fill] (tid0) at (4.5,1.5){};
\node[circle, scale=0.75, fill] (tid1) at (2.25,3){};
\node[circle, scale=0.75, fill] (tid4) at (0.75,4.5){};
\node[circle, scale=0.75, fill] (tid5) at (2.25,4.5){};
\node[circle, scale=0.75, fill] (tid6) at (3.75,4.5){};
\draw[](tid1) -- (tid4);
\draw[](tid1) -- (tid5);
\draw[](tid1) -- (tid6);
\node[circle, scale=0.75, fill] (tid2) at (6,3){};
\node[circle, scale=0.75, fill, task_scheduled] (tid7) at (5.25,4.5){};
\node[circle, scale=0.75, fill] (tid8) at (6.75,4.5){};
\draw[](tid2) -- (tid7);
\draw[](tid2) -- (tid8);
\node[circle, scale=0.75, fill] (tid3) at (8.25,3){};
\node[circle, scale=0.75, fill, task_scheduled] (tid9) at (8.25,4.5){};
\draw[](tid3) -- (tid9);
\draw[](tid0) -- (tid1);
\draw[](tid0) -- (tid2);
\draw[](tid0) -- (tid3);
\end{tikzpicture}
\nodepart{three}
\footnotesize{6.125}
\nodepart{four}
\footnotesize{$38\:12\:38\:12$}
};
 \\ 
\\
};
\end{scope}
\begin{scope}[yshift=\leveltopIII cm, anchor = center]
\matrix (line3)[row sep=0.5cm] {
\node[draw=black, rectangle split,  rectangle split parts=4] (sn0x22bc030){
\footnotesize{12.5}
\nodepart{two}
\begin{tikzpicture}[scale=.2]
\node[circle, scale=0.75, fill] (tid0) at (3.75,1.5){};
\node[circle, scale=0.75, fill] (tid1) at (1.5,3){};
\node[circle, scale=0.75, fill] (tid4) at (0.75,4.5){};
\node[circle, scale=0.75, fill, task_scheduled] (tid8) at (0.75,6){};
\draw[](tid4) -- (tid8);
\node[circle, scale=0.75, fill] (tid5) at (2.25,4.5){};
\draw[](tid1) -- (tid4);
\draw[](tid1) -- (tid5);
\node[circle, scale=0.75, fill] (tid2) at (4.5,3){};
\node[circle, scale=0.75, fill, task_scheduled] (tid6) at (3.75,4.5){};
\node[circle, scale=0.75, fill] (tid7) at (5.25,4.5){};
\draw[](tid2) -- (tid6);
\draw[](tid2) -- (tid7);
\node[circle, scale=0.75, fill] (tid3) at (6.75,3){};
\draw[](tid0) -- (tid1);
\draw[](tid0) -- (tid2);
\draw[](tid0) -- (tid3);
\end{tikzpicture}
\nodepart{three}
\footnotesize{5.65625}
\nodepart{four}
\footnotesize{$25\:25\:33\:17$}
};
 \\ 
\node[draw=black, rectangle split,  rectangle split parts=4] (sn0x22bb880){
\footnotesize{6.25}
\nodepart{two}
\begin{tikzpicture}[scale=.2]
\node[circle, scale=0.75, fill] (tid0) at (3.75,1.5){};
\node[circle, scale=0.75, fill] (tid1) at (1.5,3){};
\node[circle, scale=0.75, fill] (tid4) at (0.75,4.5){};
\node[circle, scale=0.75, fill, task_scheduled] (tid8) at (0.75,6){};
\draw[](tid4) -- (tid8);
\node[circle, scale=0.75, fill, task_scheduled] (tid5) at (2.25,4.5){};
\draw[](tid1) -- (tid4);
\draw[](tid1) -- (tid5);
\node[circle, scale=0.75, fill] (tid2) at (4.5,3){};
\node[circle, scale=0.75, fill] (tid6) at (3.75,4.5){};
\node[circle, scale=0.75, fill] (tid7) at (5.25,4.5){};
\draw[](tid2) -- (tid6);
\draw[](tid2) -- (tid7);
\node[circle, scale=0.75, fill] (tid3) at (6.75,3){};
\draw[](tid0) -- (tid1);
\draw[](tid0) -- (tid2);
\draw[](tid0) -- (tid3);
\end{tikzpicture}
\nodepart{three}
\footnotesize{5.65625}
\nodepart{four}
\footnotesize{$50\:17\:33$}
};
 \\ 
\node[draw=black, rectangle split,  rectangle split parts=4] (sn0x22b4a40){
\footnotesize{6.25}
\nodepart{two}
\begin{tikzpicture}[scale=.2]
\node[circle, scale=0.75, fill] (tid0) at (3.75,1.5){};
\node[circle, scale=0.75, fill] (tid1) at (2.25,3){};
\node[circle, scale=0.75, fill, task_scheduled] (tid4) at (0.75,4.5){};
\node[circle, scale=0.75, fill] (tid5) at (2.25,4.5){};
\node[circle, scale=0.75, fill] (tid6) at (3.75,4.5){};
\draw[](tid1) -- (tid4);
\draw[](tid1) -- (tid5);
\draw[](tid1) -- (tid6);
\node[circle, scale=0.75, fill] (tid2) at (5.25,3){};
\node[circle, scale=0.75, fill] (tid7) at (5.25,4.5){};
\node[circle, scale=0.75, fill, task_scheduled] (tid8) at (5.25,6){};
\draw[](tid7) -- (tid8);
\draw[](tid2) -- (tid7);
\node[circle, scale=0.75, fill] (tid3) at (6.75,3){};
\draw[](tid0) -- (tid1);
\draw[](tid0) -- (tid2);
\draw[](tid0) -- (tid3);
\end{tikzpicture}
\nodepart{three}
\footnotesize{5.65625}
\nodepart{four}
\footnotesize{$33\:17\:50$}
};
 \\ 
\node[draw=black, rectangle split,  rectangle split parts=4] (sn0x22bc5d0){
\footnotesize{16.875}
\nodepart{two}
\begin{tikzpicture}[scale=.2]
\node[circle, scale=0.75, fill] (tid0) at (4.5,1.5){};
\node[circle, scale=0.75, fill] (tid1) at (2.25,3){};
\node[circle, scale=0.75, fill, task_scheduled] (tid4) at (0.75,4.5){};
\node[circle, scale=0.75, fill, task_scheduled] (tid5) at (2.25,4.5){};
\node[circle, scale=0.75, fill] (tid6) at (3.75,4.5){};
\draw[](tid1) -- (tid4);
\draw[](tid1) -- (tid5);
\draw[](tid1) -- (tid6);
\node[circle, scale=0.75, fill] (tid2) at (6,3){};
\node[circle, scale=0.75, fill] (tid7) at (5.25,4.5){};
\node[circle, scale=0.75, fill] (tid8) at (6.75,4.5){};
\draw[](tid2) -- (tid7);
\draw[](tid2) -- (tid8);
\node[circle, scale=0.75, fill] (tid3) at (8.25,3){};
\draw[](tid0) -- (tid1);
\draw[](tid0) -- (tid2);
\draw[](tid0) -- (tid3);
\end{tikzpicture}
\nodepart{three}
\footnotesize{5.625}
\nodepart{four}
\footnotesize{$33\:67$}
};
 \\ 
\node[draw=black, rectangle split,  rectangle split parts=4] (sn0x22b7500){
\footnotesize{29.0625}
\nodepart{two}
\begin{tikzpicture}[scale=.2]
\node[circle, scale=0.75, fill] (tid0) at (4.5,1.5){};
\node[circle, scale=0.75, fill] (tid1) at (2.25,3){};
\node[circle, scale=0.75, fill, task_scheduled] (tid4) at (0.75,4.5){};
\node[circle, scale=0.75, fill] (tid5) at (2.25,4.5){};
\node[circle, scale=0.75, fill] (tid6) at (3.75,4.5){};
\draw[](tid1) -- (tid4);
\draw[](tid1) -- (tid5);
\draw[](tid1) -- (tid6);
\node[circle, scale=0.75, fill] (tid2) at (6,3){};
\node[circle, scale=0.75, fill, task_scheduled] (tid7) at (5.25,4.5){};
\node[circle, scale=0.75, fill] (tid8) at (6.75,4.5){};
\draw[](tid2) -- (tid7);
\draw[](tid2) -- (tid8);
\node[circle, scale=0.75, fill] (tid3) at (8.25,3){};
\draw[](tid0) -- (tid1);
\draw[](tid0) -- (tid2);
\draw[](tid0) -- (tid3);
\end{tikzpicture}
\nodepart{three}
\footnotesize{5.625}
\nodepart{four}
\footnotesize{$33\:17\:33\:17$}
};
 \\ 
\node[draw=black, rectangle split,  rectangle split parts=4] (sn0x22b7aa0){
\footnotesize{4.0625}
\nodepart{two}
\begin{tikzpicture}[scale=.2]
\node[circle, scale=0.75, fill] (tid0) at (4.5,1.5){};
\node[circle, scale=0.75, fill] (tid1) at (2.25,3){};
\node[circle, scale=0.75, fill] (tid4) at (0.75,4.5){};
\node[circle, scale=0.75, fill] (tid5) at (2.25,4.5){};
\node[circle, scale=0.75, fill] (tid6) at (3.75,4.5){};
\draw[](tid1) -- (tid4);
\draw[](tid1) -- (tid5);
\draw[](tid1) -- (tid6);
\node[circle, scale=0.75, fill] (tid2) at (6,3){};
\node[circle, scale=0.75, fill, task_scheduled] (tid7) at (5.25,4.5){};
\node[circle, scale=0.75, fill, task_scheduled] (tid8) at (6.75,4.5){};
\draw[](tid2) -- (tid7);
\draw[](tid2) -- (tid8);
\node[circle, scale=0.75, fill] (tid3) at (8.25,3){};
\draw[](tid0) -- (tid1);
\draw[](tid0) -- (tid2);
\draw[](tid0) -- (tid3);
\end{tikzpicture}
\nodepart{three}
\footnotesize{5.625}
\nodepart{four}
\footnotesize{$1$}
};
 \\ 
\node[draw=black, rectangle split,  rectangle split parts=4] (sn0x22ada30){
\footnotesize{7.5}
\nodepart{two}
\begin{tikzpicture}[scale=.2]
\node[circle, scale=0.75, fill] (tid0) at (3.75,1.5){};
\node[circle, scale=0.75, fill] (tid1) at (2.25,3){};
\node[circle, scale=0.75, fill, task_scheduled] (tid4) at (0.75,4.5){};
\node[circle, scale=0.75, fill] (tid5) at (2.25,4.5){};
\node[circle, scale=0.75, fill] (tid6) at (3.75,4.5){};
\draw[](tid1) -- (tid4);
\draw[](tid1) -- (tid5);
\draw[](tid1) -- (tid6);
\node[circle, scale=0.75, fill] (tid2) at (5.25,3){};
\node[circle, scale=0.75, fill, task_scheduled] (tid7) at (5.25,4.5){};
\draw[](tid2) -- (tid7);
\node[circle, scale=0.75, fill] (tid3) at (6.75,3){};
\node[circle, scale=0.75, fill] (tid8) at (6.75,4.5){};
\draw[](tid3) -- (tid8);
\draw[](tid0) -- (tid1);
\draw[](tid0) -- (tid2);
\draw[](tid0) -- (tid3);
\end{tikzpicture}
\nodepart{three}
\footnotesize{5.625}
\nodepart{four}
\footnotesize{$33\:17\:33\:17$}
};
 \\ 
\node[draw=black, rectangle split,  rectangle split parts=4] (sn0x22b4be0){
\footnotesize{2.5}
\nodepart{two}
\begin{tikzpicture}[scale=.2]
\node[circle, scale=0.75, fill] (tid0) at (3.75,1.5){};
\node[circle, scale=0.75, fill] (tid1) at (2.25,3){};
\node[circle, scale=0.75, fill] (tid4) at (0.75,4.5){};
\node[circle, scale=0.75, fill] (tid5) at (2.25,4.5){};
\node[circle, scale=0.75, fill] (tid6) at (3.75,4.5){};
\draw[](tid1) -- (tid4);
\draw[](tid1) -- (tid5);
\draw[](tid1) -- (tid6);
\node[circle, scale=0.75, fill] (tid2) at (5.25,3){};
\node[circle, scale=0.75, fill, task_scheduled] (tid7) at (5.25,4.5){};
\draw[](tid2) -- (tid7);
\node[circle, scale=0.75, fill] (tid3) at (6.75,3){};
\node[circle, scale=0.75, fill, task_scheduled] (tid8) at (6.75,4.5){};
\draw[](tid3) -- (tid8);
\draw[](tid0) -- (tid1);
\draw[](tid0) -- (tid2);
\draw[](tid0) -- (tid3);
\end{tikzpicture}
\nodepart{three}
\footnotesize{5.625}
\nodepart{four}
\footnotesize{$1$}
};
 \\ 
\node[draw=black, rectangle split,  rectangle split parts=4] (sn0x22b6220){
\footnotesize{15}
\nodepart{two}
\begin{tikzpicture}[scale=.2]
\node[circle, scale=0.75, fill] (tid0) at (3.75,1.5){};
\node[circle, scale=0.75, fill] (tid1) at (1.5,3){};
\node[circle, scale=0.75, fill, task_scheduled] (tid4) at (0.75,4.5){};
\node[circle, scale=0.75, fill] (tid5) at (2.25,4.5){};
\draw[](tid1) -- (tid4);
\draw[](tid1) -- (tid5);
\node[circle, scale=0.75, fill] (tid2) at (4.5,3){};
\node[circle, scale=0.75, fill] (tid6) at (3.75,4.5){};
\node[circle, scale=0.75, fill] (tid7) at (5.25,4.5){};
\draw[](tid2) -- (tid6);
\draw[](tid2) -- (tid7);
\node[circle, scale=0.75, fill] (tid3) at (6.75,3){};
\node[circle, scale=0.75, fill, task_scheduled] (tid8) at (6.75,4.5){};
\draw[](tid3) -- (tid8);
\draw[](tid0) -- (tid1);
\draw[](tid0) -- (tid2);
\draw[](tid0) -- (tid3);
\end{tikzpicture}
\nodepart{three}
\footnotesize{5.625}
\nodepart{four}
\footnotesize{$17\:33\:17\:33$}
};
 \\ 
\\
};
\end{scope}
\begin{scope}[yshift=\leveltopIIII cm, anchor = center]
\matrix (line4)[row sep=0.5cm] {
\node[draw=black, rectangle split,  rectangle split parts=4] (sn0x22baab0){
\footnotesize{3.125}
\nodepart{two}
\begin{tikzpicture}[scale=.2]
\node[circle, scale=0.75, fill] (tid0) at (3,1.5){};
\node[circle, scale=0.75, fill] (tid1) at (1.5,3){};
\node[circle, scale=0.75, fill] (tid4) at (0.75,4.5){};
\node[circle, scale=0.75, fill, task_scheduled] (tid7) at (0.75,6){};
\draw[](tid4) -- (tid7);
\node[circle, scale=0.75, fill, task_scheduled] (tid5) at (2.25,4.5){};
\draw[](tid1) -- (tid4);
\draw[](tid1) -- (tid5);
\node[circle, scale=0.75, fill] (tid2) at (3.75,3){};
\node[circle, scale=0.75, fill] (tid6) at (3.75,4.5){};
\draw[](tid2) -- (tid6);
\node[circle, scale=0.75, fill] (tid3) at (5.25,3){};
\draw[](tid0) -- (tid1);
\draw[](tid0) -- (tid2);
\draw[](tid0) -- (tid3);
\end{tikzpicture}
\nodepart{three}
\footnotesize{5.1875}
\nodepart{four}
\footnotesize{$50\:25\:25$}
};
 \\ 
\node[draw=black, rectangle split,  rectangle split parts=4] (sn0x22bafa0){
\footnotesize{3.125}
\nodepart{two}
\begin{tikzpicture}[scale=.2]
\node[circle, scale=0.75, fill] (tid0) at (3,1.5){};
\node[circle, scale=0.75, fill] (tid1) at (1.5,3){};
\node[circle, scale=0.75, fill] (tid4) at (0.75,4.5){};
\node[circle, scale=0.75, fill, task_scheduled] (tid7) at (0.75,6){};
\draw[](tid4) -- (tid7);
\node[circle, scale=0.75, fill] (tid5) at (2.25,4.5){};
\draw[](tid1) -- (tid4);
\draw[](tid1) -- (tid5);
\node[circle, scale=0.75, fill] (tid2) at (3.75,3){};
\node[circle, scale=0.75, fill, task_scheduled] (tid6) at (3.75,4.5){};
\draw[](tid2) -- (tid6);
\node[circle, scale=0.75, fill] (tid3) at (5.25,3){};
\draw[](tid0) -- (tid1);
\draw[](tid0) -- (tid2);
\draw[](tid0) -- (tid3);
\end{tikzpicture}
\nodepart{three}
\footnotesize{5.1875}
\nodepart{four}
\footnotesize{$50\:50$}
};
 \\ 
\node[draw=black, rectangle split,  rectangle split parts=4] (sn0x22b34f0){
\footnotesize{14.2708}
\nodepart{two}
\begin{tikzpicture}[scale=.2]
\node[circle, scale=0.75, fill] (tid0) at (3.75,1.5){};
\node[circle, scale=0.75, fill] (tid1) at (2.25,3){};
\node[circle, scale=0.75, fill, task_scheduled] (tid4) at (0.75,4.5){};
\node[circle, scale=0.75, fill, task_scheduled] (tid5) at (2.25,4.5){};
\node[circle, scale=0.75, fill] (tid6) at (3.75,4.5){};
\draw[](tid1) -- (tid4);
\draw[](tid1) -- (tid5);
\draw[](tid1) -- (tid6);
\node[circle, scale=0.75, fill] (tid2) at (5.25,3){};
\node[circle, scale=0.75, fill] (tid7) at (5.25,4.5){};
\draw[](tid2) -- (tid7);
\node[circle, scale=0.75, fill] (tid3) at (6.75,3){};
\draw[](tid0) -- (tid1);
\draw[](tid0) -- (tid2);
\draw[](tid0) -- (tid3);
\end{tikzpicture}
\nodepart{three}
\footnotesize{5.125}
\nodepart{four}
\footnotesize{$50\:50$}
};
 \\ 
\node[draw=black, rectangle split,  rectangle split parts=4] (sn0x22b3cf0){
\footnotesize{13.6979}
\nodepart{two}
\begin{tikzpicture}[scale=.2]
\node[circle, scale=0.75, fill] (tid0) at (3.75,1.5){};
\node[circle, scale=0.75, fill] (tid1) at (2.25,3){};
\node[circle, scale=0.75, fill, task_scheduled] (tid4) at (0.75,4.5){};
\node[circle, scale=0.75, fill] (tid5) at (2.25,4.5){};
\node[circle, scale=0.75, fill] (tid6) at (3.75,4.5){};
\draw[](tid1) -- (tid4);
\draw[](tid1) -- (tid5);
\draw[](tid1) -- (tid6);
\node[circle, scale=0.75, fill] (tid2) at (5.25,3){};
\node[circle, scale=0.75, fill, task_scheduled] (tid7) at (5.25,4.5){};
\draw[](tid2) -- (tid7);
\node[circle, scale=0.75, fill] (tid3) at (6.75,3){};
\draw[](tid0) -- (tid1);
\draw[](tid0) -- (tid2);
\draw[](tid0) -- (tid3);
\end{tikzpicture}
\nodepart{three}
\footnotesize{5.125}
\nodepart{four}
\footnotesize{$50\:50$}
};
 \\ 
\node[draw=black, rectangle split,  rectangle split parts=4] (sn0x22b2b90){
\footnotesize{6.25}
\nodepart{two}
\begin{tikzpicture}[scale=.2]
\node[circle, scale=0.75, fill] (tid0) at (3,1.5){};
\node[circle, scale=0.75, fill] (tid1) at (1.5,3){};
\node[circle, scale=0.75, fill, task_scheduled] (tid4) at (0.75,4.5){};
\node[circle, scale=0.75, fill] (tid5) at (2.25,4.5){};
\draw[](tid1) -- (tid4);
\draw[](tid1) -- (tid5);
\node[circle, scale=0.75, fill] (tid2) at (3.75,3){};
\node[circle, scale=0.75, fill] (tid6) at (3.75,4.5){};
\node[circle, scale=0.75, fill, task_scheduled] (tid7) at (3.75,6){};
\draw[](tid6) -- (tid7);
\draw[](tid2) -- (tid6);
\node[circle, scale=0.75, fill] (tid3) at (5.25,3){};
\draw[](tid0) -- (tid1);
\draw[](tid0) -- (tid2);
\draw[](tid0) -- (tid3);
\end{tikzpicture}
\nodepart{three}
\footnotesize{5.1875}
\nodepart{four}
\footnotesize{$50\:25\:25$}
};
 \\ 
\node[draw=black, rectangle split,  rectangle split parts=4] (sn0x22b6c20){
\footnotesize{32.1875}
\nodepart{two}
\begin{tikzpicture}[scale=.2]
\node[circle, scale=0.75, fill] (tid0) at (3.75,1.5){};
\node[circle, scale=0.75, fill] (tid1) at (1.5,3){};
\node[circle, scale=0.75, fill, task_scheduled] (tid4) at (0.75,4.5){};
\node[circle, scale=0.75, fill] (tid5) at (2.25,4.5){};
\draw[](tid1) -- (tid4);
\draw[](tid1) -- (tid5);
\node[circle, scale=0.75, fill] (tid2) at (4.5,3){};
\node[circle, scale=0.75, fill, task_scheduled] (tid6) at (3.75,4.5){};
\node[circle, scale=0.75, fill] (tid7) at (5.25,4.5){};
\draw[](tid2) -- (tid6);
\draw[](tid2) -- (tid7);
\node[circle, scale=0.75, fill] (tid3) at (6.75,3){};
\draw[](tid0) -- (tid1);
\draw[](tid0) -- (tid2);
\draw[](tid0) -- (tid3);
\end{tikzpicture}
\nodepart{three}
\footnotesize{5.125}
\nodepart{four}
\footnotesize{$50\:50$}
};
 \\ 
\node[draw=black, rectangle split,  rectangle split parts=4] (sn0x22b66d0){
\footnotesize{16.0938}
\nodepart{two}
\begin{tikzpicture}[scale=.2]
\node[circle, scale=0.75, fill] (tid0) at (3.75,1.5){};
\node[circle, scale=0.75, fill] (tid1) at (1.5,3){};
\node[circle, scale=0.75, fill, task_scheduled] (tid4) at (0.75,4.5){};
\node[circle, scale=0.75, fill, task_scheduled] (tid5) at (2.25,4.5){};
\draw[](tid1) -- (tid4);
\draw[](tid1) -- (tid5);
\node[circle, scale=0.75, fill] (tid2) at (4.5,3){};
\node[circle, scale=0.75, fill] (tid6) at (3.75,4.5){};
\node[circle, scale=0.75, fill] (tid7) at (5.25,4.5){};
\draw[](tid2) -- (tid6);
\draw[](tid2) -- (tid7);
\node[circle, scale=0.75, fill] (tid3) at (6.75,3){};
\draw[](tid0) -- (tid1);
\draw[](tid0) -- (tid2);
\draw[](tid0) -- (tid3);
\end{tikzpicture}
\nodepart{three}
\footnotesize{5.125}
\nodepart{four}
\footnotesize{$1$}
};
 \\ 
\node[draw=black, rectangle split,  rectangle split parts=4] (sn0x22b2d60){
\footnotesize{3.75}
\nodepart{two}
\begin{tikzpicture}[scale=.2]
\node[circle, scale=0.75, fill] (tid0) at (3,1.5){};
\node[circle, scale=0.75, fill] (tid1) at (1.5,3){};
\node[circle, scale=0.75, fill] (tid4) at (0.75,4.5){};
\node[circle, scale=0.75, fill] (tid5) at (2.25,4.5){};
\draw[](tid1) -- (tid4);
\draw[](tid1) -- (tid5);
\node[circle, scale=0.75, fill] (tid2) at (3.75,3){};
\node[circle, scale=0.75, fill, task_scheduled] (tid6) at (3.75,4.5){};
\draw[](tid2) -- (tid6);
\node[circle, scale=0.75, fill] (tid3) at (5.25,3){};
\node[circle, scale=0.75, fill, task_scheduled] (tid7) at (5.25,4.5){};
\draw[](tid3) -- (tid7);
\draw[](tid0) -- (tid1);
\draw[](tid0) -- (tid2);
\draw[](tid0) -- (tid3);
\end{tikzpicture}
\nodepart{three}
\footnotesize{5.125}
\nodepart{four}
\footnotesize{$1$}
};
 \\ 
\node[draw=black, rectangle split,  rectangle split parts=4] (sn0x22ae720){
\footnotesize{7.5}
\nodepart{two}
\begin{tikzpicture}[scale=.2]
\node[circle, scale=0.75, fill] (tid0) at (3,1.5){};
\node[circle, scale=0.75, fill] (tid1) at (1.5,3){};
\node[circle, scale=0.75, fill, task_scheduled] (tid4) at (0.75,4.5){};
\node[circle, scale=0.75, fill] (tid5) at (2.25,4.5){};
\draw[](tid1) -- (tid4);
\draw[](tid1) -- (tid5);
\node[circle, scale=0.75, fill] (tid2) at (3.75,3){};
\node[circle, scale=0.75, fill, task_scheduled] (tid6) at (3.75,4.5){};
\draw[](tid2) -- (tid6);
\node[circle, scale=0.75, fill] (tid3) at (5.25,3){};
\node[circle, scale=0.75, fill] (tid7) at (5.25,4.5){};
\draw[](tid3) -- (tid7);
\draw[](tid0) -- (tid1);
\draw[](tid0) -- (tid2);
\draw[](tid0) -- (tid3);
\end{tikzpicture}
\nodepart{three}
\footnotesize{5.125}
\nodepart{four}
\footnotesize{$25\:25\:50$}
};
 \\ 
\\
};
\end{scope}
\begin{scope}[yshift=\leveltopIIIII cm, anchor = center]
\matrix (line5)[row sep=0.5cm] {
\node[draw=black, rectangle split,  rectangle split parts=4] (sn0x22bb120){
\footnotesize{1.5625}
\nodepart{two}
\begin{tikzpicture}[scale=.2]
\node[circle, scale=0.75, fill] (tid0) at (3,1.5){};
\node[circle, scale=0.75, fill] (tid1) at (1.5,3){};
\node[circle, scale=0.75, fill] (tid4) at (0.75,4.5){};
\node[circle, scale=0.75, fill, task_scheduled] (tid6) at (0.75,6){};
\draw[](tid4) -- (tid6);
\node[circle, scale=0.75, fill, task_scheduled] (tid5) at (2.25,4.5){};
\draw[](tid1) -- (tid4);
\draw[](tid1) -- (tid5);
\node[circle, scale=0.75, fill] (tid2) at (3.75,3){};
\node[circle, scale=0.75, fill] (tid3) at (5.25,3){};
\draw[](tid0) -- (tid1);
\draw[](tid0) -- (tid2);
\draw[](tid0) -- (tid3);
\end{tikzpicture}
\nodepart{three}
\footnotesize{4.75}
\nodepart{four}
\footnotesize{$50\:50$}
};
 \\ 
\node[draw=black, rectangle split,  rectangle split parts=4] (sn0x22ae570){
\footnotesize{4.6875}
\nodepart{two}
\begin{tikzpicture}[scale=.2]
\node[circle, scale=0.75, fill] (tid0) at (2.25,1.5){};
\node[circle, scale=0.75, fill] (tid1) at (0.75,3){};
\node[circle, scale=0.75, fill] (tid4) at (0.75,4.5){};
\node[circle, scale=0.75, fill, task_scheduled] (tid6) at (0.75,6){};
\draw[](tid4) -- (tid6);
\draw[](tid1) -- (tid4);
\node[circle, scale=0.75, fill] (tid2) at (2.25,3){};
\node[circle, scale=0.75, fill, task_scheduled] (tid5) at (2.25,4.5){};
\draw[](tid2) -- (tid5);
\node[circle, scale=0.75, fill] (tid3) at (3.75,3){};
\draw[](tid0) -- (tid1);
\draw[](tid0) -- (tid2);
\draw[](tid0) -- (tid3);
\end{tikzpicture}
\nodepart{three}
\footnotesize{4.75}
\nodepart{four}
\footnotesize{$50\:50$}
};
 \\ 
\node[draw=black, rectangle split,  rectangle split parts=4] (sn0x22b40d0){
\footnotesize{6.84896}
\nodepart{two}
\begin{tikzpicture}[scale=.2]
\node[circle, scale=0.75, fill] (tid0) at (3.75,1.5){};
\node[circle, scale=0.75, fill] (tid1) at (2.25,3){};
\node[circle, scale=0.75, fill, task_scheduled] (tid4) at (0.75,4.5){};
\node[circle, scale=0.75, fill, task_scheduled] (tid5) at (2.25,4.5){};
\node[circle, scale=0.75, fill] (tid6) at (3.75,4.5){};
\draw[](tid1) -- (tid4);
\draw[](tid1) -- (tid5);
\draw[](tid1) -- (tid6);
\node[circle, scale=0.75, fill] (tid2) at (5.25,3){};
\node[circle, scale=0.75, fill] (tid3) at (6.75,3){};
\draw[](tid0) -- (tid1);
\draw[](tid0) -- (tid2);
\draw[](tid0) -- (tid3);
\end{tikzpicture}
\nodepart{three}
\footnotesize{4.625}
\nodepart{four}
\footnotesize{$1$}
};
 \\ 
\node[draw=black, rectangle split,  rectangle split parts=4] (sn0x22b14e0){
\footnotesize{27.4479}
\nodepart{two}
\begin{tikzpicture}[scale=.2]
\node[circle, scale=0.75, fill] (tid0) at (3,1.5){};
\node[circle, scale=0.75, fill] (tid1) at (1.5,3){};
\node[circle, scale=0.75, fill, task_scheduled] (tid4) at (0.75,4.5){};
\node[circle, scale=0.75, fill, task_scheduled] (tid5) at (2.25,4.5){};
\draw[](tid1) -- (tid4);
\draw[](tid1) -- (tid5);
\node[circle, scale=0.75, fill] (tid2) at (3.75,3){};
\node[circle, scale=0.75, fill] (tid6) at (3.75,4.5){};
\draw[](tid2) -- (tid6);
\node[circle, scale=0.75, fill] (tid3) at (5.25,3){};
\draw[](tid0) -- (tid1);
\draw[](tid0) -- (tid2);
\draw[](tid0) -- (tid3);
\end{tikzpicture}
\nodepart{three}
\footnotesize{4.625}
\nodepart{four}
\footnotesize{$1$}
};
 \\ 
\node[draw=black, rectangle split,  rectangle split parts=4] (sn0x22b1ca0){
\footnotesize{55.7031}
\nodepart{two}
\begin{tikzpicture}[scale=.2]
\node[circle, scale=0.75, fill] (tid0) at (3,1.5){};
\node[circle, scale=0.75, fill] (tid1) at (1.5,3){};
\node[circle, scale=0.75, fill, task_scheduled] (tid4) at (0.75,4.5){};
\node[circle, scale=0.75, fill] (tid5) at (2.25,4.5){};
\draw[](tid1) -- (tid4);
\draw[](tid1) -- (tid5);
\node[circle, scale=0.75, fill] (tid2) at (3.75,3){};
\node[circle, scale=0.75, fill, task_scheduled] (tid6) at (3.75,4.5){};
\draw[](tid2) -- (tid6);
\node[circle, scale=0.75, fill] (tid3) at (5.25,3){};
\draw[](tid0) -- (tid1);
\draw[](tid0) -- (tid2);
\draw[](tid0) -- (tid3);
\end{tikzpicture}
\nodepart{three}
\footnotesize{4.625}
\nodepart{four}
\footnotesize{$50\:50$}
};
 \\ 
\node[draw=black, rectangle split,  rectangle split parts=4] (sn0x22af010){
\footnotesize{3.75}
\nodepart{two}
\begin{tikzpicture}[scale=.2]
\node[circle, scale=0.75, fill] (tid0) at (2.25,1.5){};
\node[circle, scale=0.75, fill] (tid1) at (0.75,3){};
\node[circle, scale=0.75, fill, task_scheduled] (tid4) at (0.75,4.5){};
\draw[](tid1) -- (tid4);
\node[circle, scale=0.75, fill] (tid2) at (2.25,3){};
\node[circle, scale=0.75, fill, task_scheduled] (tid5) at (2.25,4.5){};
\draw[](tid2) -- (tid5);
\node[circle, scale=0.75, fill] (tid3) at (3.75,3){};
\node[circle, scale=0.75, fill] (tid6) at (3.75,4.5){};
\draw[](tid3) -- (tid6);
\draw[](tid0) -- (tid1);
\draw[](tid0) -- (tid2);
\draw[](tid0) -- (tid3);
\end{tikzpicture}
\nodepart{three}
\footnotesize{4.625}
\nodepart{four}
\footnotesize{$1$}
};
 \\ 
\\
};
\end{scope}
\begin{scope}[yshift=\leveltopIIIIII cm, anchor = center]
\matrix (line6)[row sep=0.5cm] {
\node[draw=black, rectangle split,  rectangle split parts=4] (sn0x22aecc0){
\footnotesize{3.125}
\nodepart{two}
\begin{tikzpicture}[scale=.2]
\node[circle, scale=0.75, fill] (tid0) at (2.25,1.5){};
\node[circle, scale=0.75, fill] (tid1) at (0.75,3){};
\node[circle, scale=0.75, fill] (tid4) at (0.75,4.5){};
\node[circle, scale=0.75, fill, task_scheduled] (tid5) at (0.75,6){};
\draw[](tid4) -- (tid5);
\draw[](tid1) -- (tid4);
\node[circle, scale=0.75, fill, task_scheduled] (tid2) at (2.25,3){};
\node[circle, scale=0.75, fill] (tid3) at (3.75,3){};
\draw[](tid0) -- (tid1);
\draw[](tid0) -- (tid2);
\draw[](tid0) -- (tid3);
\end{tikzpicture}
\nodepart{three}
\footnotesize{4.375}
\nodepart{four}
\footnotesize{$50\:50$}
};
 \\ 
\node[draw=black, rectangle split,  rectangle split parts=4] (sn0x22b21c0){
\footnotesize{35.4818}
\nodepart{two}
\begin{tikzpicture}[scale=.2]
\node[circle, scale=0.75, fill] (tid0) at (3,1.5){};
\node[circle, scale=0.75, fill] (tid1) at (1.5,3){};
\node[circle, scale=0.75, fill, task_scheduled] (tid4) at (0.75,4.5){};
\node[circle, scale=0.75, fill, task_scheduled] (tid5) at (2.25,4.5){};
\draw[](tid1) -- (tid4);
\draw[](tid1) -- (tid5);
\node[circle, scale=0.75, fill] (tid2) at (3.75,3){};
\node[circle, scale=0.75, fill] (tid3) at (5.25,3){};
\draw[](tid0) -- (tid1);
\draw[](tid0) -- (tid2);
\draw[](tid0) -- (tid3);
\end{tikzpicture}
\nodepart{three}
\footnotesize{4.125}
\nodepart{four}
\footnotesize{$1$}
};
 \\ 
\node[draw=black, rectangle split,  rectangle split parts=4] (sn0x22af160){
\footnotesize{61.3932}
\nodepart{two}
\begin{tikzpicture}[scale=.2]
\node[circle, scale=0.75, fill] (tid0) at (2.25,1.5){};
\node[circle, scale=0.75, fill] (tid1) at (0.75,3){};
\node[circle, scale=0.75, fill, task_scheduled] (tid4) at (0.75,4.5){};
\draw[](tid1) -- (tid4);
\node[circle, scale=0.75, fill] (tid2) at (2.25,3){};
\node[circle, scale=0.75, fill, task_scheduled] (tid5) at (2.25,4.5){};
\draw[](tid2) -- (tid5);
\node[circle, scale=0.75, fill] (tid3) at (3.75,3){};
\draw[](tid0) -- (tid1);
\draw[](tid0) -- (tid2);
\draw[](tid0) -- (tid3);
\end{tikzpicture}
\nodepart{three}
\footnotesize{4.125}
\nodepart{four}
\footnotesize{$1$}
};
 \\ 
\\
};
\end{scope}
\begin{scope}[yshift=\leveltopIIIIIII cm, anchor = center]
\matrix (line7)[row sep=0.5cm] {
\node[draw=black, rectangle split,  rectangle split parts=4] (sn0x22af6c0){
\footnotesize{1.5625}
\nodepart{two}
\begin{tikzpicture}[scale=.2]
\node[circle, scale=0.75, fill] (tid0) at (1.5,1.5){};
\node[circle, scale=0.75, fill] (tid1) at (0.75,3){};
\node[circle, scale=0.75, fill] (tid3) at (0.75,4.5){};
\node[circle, scale=0.75, fill, task_scheduled] (tid4) at (0.75,6){};
\draw[](tid3) -- (tid4);
\draw[](tid1) -- (tid3);
\node[circle, scale=0.75, fill, task_scheduled] (tid2) at (2.25,3){};
\draw[](tid0) -- (tid1);
\draw[](tid0) -- (tid2);
\end{tikzpicture}
\nodepart{three}
\footnotesize{4.125}
\nodepart{four}
\footnotesize{$50\:50$}
};
 \\ 
\node[draw=black, rectangle split,  rectangle split parts=4] (sn0x22afb50){
\footnotesize{98.4375}
\nodepart{two}
\begin{tikzpicture}[scale=.2]
\node[circle, scale=0.75, fill] (tid0) at (2.25,1.5){};
\node[circle, scale=0.75, fill] (tid1) at (0.75,3){};
\node[circle, scale=0.75, fill, task_scheduled] (tid4) at (0.75,4.5){};
\draw[](tid1) -- (tid4);
\node[circle, scale=0.75, fill, task_scheduled] (tid2) at (2.25,3){};
\node[circle, scale=0.75, fill] (tid3) at (3.75,3){};
\draw[](tid0) -- (tid1);
\draw[](tid0) -- (tid2);
\draw[](tid0) -- (tid3);
\end{tikzpicture}
\nodepart{three}
\footnotesize{3.625}
\nodepart{four}
\footnotesize{$50\:50$}
};
 \\ 
\\
};
\end{scope}
\draw (sn0x22aa490.east) -- (sn0x22bcc60.west);
\draw (sn0x22aa490.east) -- (sn0x22bcd60.west);
\draw (sn0x22aa490.east) -- (sn0x22b9170.west);
\draw (sn0x22aa490.east) -- (sn0x22ab500.west);
\draw (sn0x22bcc60.east) -- (sn0x22bc030.west);
\draw (sn0x22bcc60.east) -- (sn0x22bb880.west);
\draw (sn0x22bcc60.east) -- (sn0x22bc5d0.west);
\draw (sn0x22bcc60.east) -- (sn0x22b7500.west);
\draw (sn0x22bcd60.east) -- (sn0x22b4a40.west);
\draw (sn0x22bcd60.east) -- (sn0x22b7500.west);
\draw (sn0x22bcd60.east) -- (sn0x22b7aa0.west);
\draw (sn0x22b9170.east) -- (sn0x22b6220.west);
\draw (sn0x22b9170.east) -- (sn0x22bc5d0.west);
\draw (sn0x22b9170.east) -- (sn0x22b7500.west);
\draw (sn0x22ab500.east) -- (sn0x22ada30.west);
\draw (sn0x22ab500.east) -- (sn0x22b4be0.west);
\draw (sn0x22ab500.east) -- (sn0x22b7500.west);
\draw (sn0x22ab500.east) -- (sn0x22b7aa0.west);
\draw (sn0x22bc030.east) -- (sn0x22baab0.west);
\draw (sn0x22bc030.east) -- (sn0x22bafa0.west);
\draw (sn0x22bc030.east) -- (sn0x22b6c20.west);
\draw (sn0x22bc030.east) -- (sn0x22b66d0.west);
\draw (sn0x22bb880.east) -- (sn0x22b2b90.west);
\draw (sn0x22bb880.east) -- (sn0x22b66d0.west);
\draw (sn0x22bb880.east) -- (sn0x22b6c20.west);
\draw (sn0x22b4a40.east) -- (sn0x22b2b90.west);
\draw (sn0x22b4a40.east) -- (sn0x22b34f0.west);
\draw (sn0x22b4a40.east) -- (sn0x22b3cf0.west);
\draw (sn0x22bc5d0.east) -- (sn0x22b66d0.west);
\draw (sn0x22bc5d0.east) -- (sn0x22b6c20.west);
\draw (sn0x22b7500.east) -- (sn0x22b6c20.west);
\draw (sn0x22b7500.east) -- (sn0x22b66d0.west);
\draw (sn0x22b7500.east) -- (sn0x22b34f0.west);
\draw (sn0x22b7500.east) -- (sn0x22b3cf0.west);
\draw (sn0x22b7aa0.east) -- (sn0x22b3cf0.west);
\draw (sn0x22ada30.east) -- (sn0x22ae720.west);
\draw (sn0x22ada30.east) -- (sn0x22b2d60.west);
\draw (sn0x22ada30.east) -- (sn0x22b34f0.west);
\draw (sn0x22ada30.east) -- (sn0x22b3cf0.west);
\draw (sn0x22b4be0.east) -- (sn0x22b3cf0.west);
\draw (sn0x22b6220.east) -- (sn0x22b2d60.west);
\draw (sn0x22b6220.east) -- (sn0x22ae720.west);
\draw (sn0x22b6220.east) -- (sn0x22b66d0.west);
\draw (sn0x22b6220.east) -- (sn0x22b6c20.west);
\draw (sn0x22baab0.east) -- (sn0x22ae570.west);
\draw (sn0x22baab0.east) -- (sn0x22b14e0.west);
\draw (sn0x22baab0.east) -- (sn0x22b1ca0.west);
\draw (sn0x22bafa0.east) -- (sn0x22bb120.west);
\draw (sn0x22bafa0.east) -- (sn0x22b1ca0.west);
\draw (sn0x22b34f0.east) -- (sn0x22b14e0.west);
\draw (sn0x22b34f0.east) -- (sn0x22b1ca0.west);
\draw (sn0x22b3cf0.east) -- (sn0x22b1ca0.west);
\draw (sn0x22b3cf0.east) -- (sn0x22b40d0.west);
\draw (sn0x22b2b90.east) -- (sn0x22ae570.west);
\draw (sn0x22b2b90.east) -- (sn0x22b14e0.west);
\draw (sn0x22b2b90.east) -- (sn0x22b1ca0.west);
\draw (sn0x22b6c20.east) -- (sn0x22b1ca0.west);
\draw (sn0x22b6c20.east) -- (sn0x22b14e0.west);
\draw (sn0x22b66d0.east) -- (sn0x22b1ca0.west);
\draw (sn0x22b2d60.east) -- (sn0x22b1ca0.west);
\draw (sn0x22ae720.east) -- (sn0x22af010.west);
\draw (sn0x22ae720.east) -- (sn0x22b14e0.west);
\draw (sn0x22ae720.east) -- (sn0x22b1ca0.west);
\draw (sn0x22bb120.east) -- (sn0x22aecc0.west);
\draw (sn0x22bb120.east) -- (sn0x22b21c0.west);
\draw (sn0x22ae570.east) -- (sn0x22aecc0.west);
\draw (sn0x22ae570.east) -- (sn0x22af160.west);
\draw (sn0x22b40d0.east) -- (sn0x22b21c0.west);
\draw (sn0x22b14e0.east) -- (sn0x22af160.west);
\draw (sn0x22b1ca0.east) -- (sn0x22af160.west);
\draw (sn0x22b1ca0.east) -- (sn0x22b21c0.west);
\draw (sn0x22af010.east) -- (sn0x22af160.west);
\draw (sn0x22aecc0.east) -- (sn0x22af6c0.west);
\draw (sn0x22aecc0.east) -- (sn0x22afb50.west);
\draw (sn0x22b21c0.east) -- (sn0x22afb50.west);
\draw (sn0x22af160.east) -- (sn0x22afb50.west);
\end{tikzpicture}

%%% Local Variables:
%%% TeX-master: "thesis/thesis.tex"
%%% End: 

%\include{../default_optimized}
%\include{../default_profile}
%\include{../default_chain}

% here, the document actually starts to get useful

%\part{Introduction}
\chapter{Theoretical foundations}
\label{chap:theoretical-foundations}

\section{Probability theory}
\label{sec:some-probability}

As we will see later, we will often be using probability distributions (in particular continuous probability distributions).

\subsection{Exponential distribution}
\label{sec:exponential-distribution}

The well-known exponential distribution is the central distribution we are dealing with in this text. All definitions and theorems within this subsection are along the lines of \cite{schickinger2001diskrete}.

\begin{definition}
  A continuous random variable is \emph{exponentially distributed} with parameter $\lambda$ if it has density 
  \begin{equation*}
    f(x) =
    \begin{cases}
      \lambda \cdot e^{-\lambda x} & \text{ if } x \geq 0 
      \\ 0 & \text{ otherwise}
    \end{cases}
    .
  \end{equation*}
\end{definition}

Note that the above definition also determines the distribution function $F$ of an exponentially distributed random variable as follows:

\begin{equation*}
  F(x) =
  \begin{cases}
    1-e^{\lambda x} & \text{ if } x \geq 0 \\
    0 & \text{ otherwise}
  \end{cases}
\end{equation*}

\begin{theorem}
  Let $X$ be an exponentially distributed random variable. Then, the expectancy of $X$ is $\E{X} = \frac 1 \lambda$.
\end{theorem}

\begin{proof}
  We can compute the expectancy for $X$ as follows:
  \begin{eqnarray*}
    \E{X} 
    &=& \int_{-\infty}^\infty x\cdot f(x)\, dx = \\
    &=& \int_{0}^\infty x\cdot \lambda e^{-\lambda x}\, dx \\
    &=& \left[ - \frac{e^{-\lambda x}\cdot \left( \lambda x + 1 \right)}{\lambda} \right]_{0}^\infty \\
    &=& \frac{1}{\lambda}
  \end{eqnarray*}
\end{proof}

\begin{theorem}[Scalability]
  Let $X$ be an exponentially distributed random variable with parameter $\lambda$ and let $a\in \mathbb{R}^+$. Then, the random variable $aX$ is exponentially distributed with parameter $\frac{\lambda}{a}$.
\end{theorem}

\begin{proof}
  We compute the probability that the random variable $aX$ is less than $x$:
  \begin{eqnarray*}
    \p{aX \leq x} 
    &=& \p{X \leq \frac{x}{a}} = \\
    &=& 1 - e^{-\frac{\lambda}{a} \cdot x}
  \end{eqnarray*}
  This result is equivalent to the density function of an exponentially distributed random variable with parameter $\frac{\lambda}{a}$.
\end{proof}

\begin{theorem}
  Let $X_1,\dots,X_n$ be exponentially-distributed random variables with respective parameters $\lambda_1,\dots,\lambda_n$. Then, the ranvom variable $Z:=\min_{i\in\left\{ 1,\dots,n \right\}} \left\{ X_i \right\}$ is exponentially distributed with parameter $\lambda=\lambda_1+\dots+\lambda_n$.
\end{theorem}

\begin{proof}
  We prove the claim by induction. 

  Suppose, we have two exponentially distributed random variables $X_1$ resp. $X_2$ with parameters $\lambda_1$ resp. $\lambda_2$. We then can compute
  \begin{align*}
    \p{min\left\{ X_1,X_2 \right\} \geq x} & = \p{X_1 \geq x \wedge X_2 \geq x} = \\ 
    & = \p{X_1 \geq x}\cdot\p{X_2 \geq x} = \\
    & = e^{-\lambda_1 x} \cdot e^{-\lambda_2 x} = \\
    & = e^{-\lambda_1 x - \lambda_2 x} = \\
    & = e^{-\left( \lambda_1+\lambda_2 \right) \cdot x},
  \end{align*}
  from which we can conclude that $\min\left\{ X_1,X_2 \right\}$ is exponentially distributed with parameter $\lambda_1 + \lambda_2$. By induction, we obtain our claim.
\end{proof}

\begin{definition}[Memorylessness]
  A random variable $X$ is called \emph{memoryless} if
  \begin{equation*}
    \p{X>t+s \mid X>s} = \p{X > t}
  \end{equation*}
\end{definition}

\begin{theorem}
  \label{sec:exponential-memoryless}
  Let $X$ be an exponentially distributed random variable with parameter $\lambda$. Then, $X$ is memoryless.
\end{theorem}

\begin{proof}
  We start by using the definition of conditional probability and rewrite until we arrive at our goal:
  \begin{eqnarray*}
    \p{X>t+s \mid X>s} &=& \frac{\p{X>t+s \wedge X>s}}{\p{X>s}} = \\
    &=& \frac{\p{X>t+s}}{\p{X>s}} = \\
    &=& \frac{e^{-\lambda \cdot(t+s)}}{e^{-\lambda s}} = \\
    &=& e^{-\lambda t} = \\
    &=& \p{X>t}
  \end{eqnarray*}
\end{proof}

This is a very advantageous property that can be exploited in our considerations to follow.

\emph{Remark:} It can even be shown that any memoryless continuous random variable is exponentially distributed, but theorem \ref{sec:exponential-memoryless} is sufficient for our needs.

\subsection{Uniform distribution}
\label{sec:uniform-distribution}

\begin{definition}
  A continuous random variable is \emph{uniformly distributed} over the interval $\left[ a,b \right]$ if it has density
  \begin{equation*}
    f(x) =
    \begin{cases}
      \frac{1}{b-a} & \text{ if } x\in\left[ a,b \right] \\
      0 & \text{ otherwise}
    \end{cases}.
  \end{equation*}
\end{definition}

The density of a uniform random variable is thus given by
\begin{equation*}
  F(x) = \begin{cases}
    0 & \text{ if } x<a \\
    \frac{x-a}{b-a} & \text{ if } x\in\left[ a,b \right] \\
    1 & \text{ if } x>b
  \end{cases}.
\end{equation*}


\subsection{General stuff}
\label{sec:probability-misc}

\begin{theorem}
  Let $X_1,\dots,X_n$ be independent, identically distributed, continuous random variables and let $i\in\left\{ 1,\dots,n \right\}$. Then 
  \begin{equation}
    \label{eq:probability-that-cont-random-variable-is-smallest-out-of-iid-is-one-over-n}
    \p{X_i = \min_{j\in\left\{ 1,\dots,n \right\}}\left\{ X_j \right\}} = \frac{1}{n}.
  \end{equation}
\end{theorem}

\begin{proof}
  It is clear that 
  \begin{equation*}
    \p{X_1 = \min_{j\in\left\{ 1,\dots,n \right\}}\left\{ X_j \right\}} = \p{X_2 = \min_{j\in\left\{ 1,\dots,n \right\}}\left\{ X_j \right\}} = \dots = \p{X_n = \min_{j\in\left\{ 1,\dots,n \right\}}\left\{ X_j \right\}},
  \end{equation*}
  because all random variables $X_1$ through $X_n$ behave the same. Thus we can deduce equation (\ref{eq:probability-that-cont-random-variable-is-smallest-out-of-iid-is-one-over-n}).
\end{proof}

\section{Graph theory}
\label{sec:foundations-graph-theory}

As we will see later, we will constantly deal with \emph{intrees}. In this section we develop simple notation for such trees. We assume that the reader is familiar with the concept of undirected trees and develop our notation on top of the one for undirected trees. For a more detailed introduction on what trees are, see e.g. \cite{diestel2005graph}.

\begin{definition}[Intree]
  Let $I'$ be a undirected tree. Let $v$ be a vertex within $I'$. Let $I$ be the directed version of $I'$ in such a way that all edges are directed towards vertex $v$. Then we call $I$ an intree or a rooted tree with root $v$.

  If there is an edge $(t_1, t_2)$ in $I$, we call $t_1$ a (direct) predecessor of $t_2$ and $t_2$ a (direct) successor of $t_1$. If there is a path from $t_1$ to $t_2$, we simply speak of predecessor and successor.
\end{definition}

Throughout this book, if not stated otherwise, we use the terms intree and tree interchangeably, because we are dealing only with intrees.

Those intrees can naturally be used to describe dependencies between different tasks, as long as each task is the requirement for at most one other tas. Each vertex in an intree then represents a task and if $t_1$ is a direct (direct) predecessor of $t_2$, this means that the task associated with $t_1$ must be executed before the task associated with $t_2$.

We can draw such intrees in a very straightforward manner: We draw the root at the bottom and its direct predecessors one level above. For each predecessor, we inductively repeat this procedure to obtain a ``top-to-bottom-description'' of the tree.

Figure \ref{fig:intree-example-task-names-directed-edges} shows an intree, where tasks 8 is a requirement for task 6, which itself is -- like task 7, 8 and (indirectly) task 10 -- a requirement for task 3. This figure also illustrates the fact that -- in an intree -- each task is a direct requirement for \emph{at most} one other task.

However, we are mostly interested in the \emph{structure} of the tree, which is why we most of the time omit the labellings of the vertices (i.e. we omit the task names) and rely on a \emph{unlabelled} representation as shown in figure \ref{fig:intree-example-structure-version}.

\begin{figure}[t]
  \centering
  \begin{subfigure}{.45\textwidth}
    \centering{}
    \begin{tikzpicture}[scale=.6, anchor=south]
      \node[circle, scale=0.9, draw] (tid0) at (3,1.5){0};
      \node[circle, scale=0.9, draw] (tid1) at (2.25,3){1};
      \node[circle, scale=0.9, draw] (tid2) at (1.5,4.5){3};
      \node[circle, scale=0.9, draw] (tid7) at (0.15,6){6};
      \node[circle, scale=0.9, draw] (tid9) at (0.15,7.5){9};
      \draw[<-, thick](tid7) -- (tid9);
      \node[circle, scale=0.9, draw] (tid10) at (1.5,6){7};
      \draw[<-, thick](tid2) -- (tid7);
      \draw[<-, thick](tid2) -- (tid10);
      \node[circle, scale=0.9, draw] (tid3) at (3.9,4.5){4};
      \node[circle, scale=0.9, draw] (tid5) at (2.85,6){8};
      \node[circle, scale=0.9, draw] (tid6) at (2.85,7.5){\small 10};
      \draw[<-, thick](tid5) -- (tid6);
      \draw[<-, thick](tid2) -- (tid5);
      \draw[<-, thick](tid1) -- (tid2);
      \draw[<-, thick](tid1) -- (tid3);
      \node[circle, scale=0.9, draw] (tid4) at (5.25,3){2};
      \node[circle, scale=0.9, draw] (tid8) at (5.25,4.5){5};
      \draw[<-, thick](tid4) -- (tid8);
      \draw[<-, thick](tid0) -- (tid1);
      \draw[<-, thick](tid0) -- (tid4);
    \end{tikzpicture}
    \caption{Labelled version with vertex labels and edges drawn as arrows.}
    \label{fig:intree-example-task-names-directed-edges}
  \end{subfigure}
  \quad
  \begin{subfigure}{.45\textwidth}
    \centering{}
    \begin{tikzpicture}[scale=.6, anchor=south]
      \node[circle, scale=0.9, fill] (tid0) at (3,1.5){};
      \node[circle, scale=0.9, fill] (tid1) at (2.25,3){};
      \node[circle, scale=0.9, fill] (tid2) at (1.5,4.5){};
      \node[circle, scale=0.9, fill] (tid7) at (0.15,6){};
      \node[circle, scale=0.9, fill] (tid9) at (0.15,7.5){};
      \draw[](tid7) -- (tid9);
      \node[circle, scale=0.9, fill] (tid10) at (1.5,6){};
      \draw[](tid2) -- (tid7);
      \draw[](tid2) -- (tid10);
      \node[circle, scale=0.9, fill] (tid3) at (3.9,4.5){};
      \node[circle, scale=0.9, fill] (tid5) at (2.85,6){};
      \node[circle, scale=0.9, fill] (tid6) at (2.85,7.5){};
      \draw[](tid5) -- (tid6);
      \draw[](tid2) -- (tid5);
      \draw[](tid1) -- (tid2);
      \draw[](tid1) -- (tid3);
      \node[circle, scale=0.9, fill] (tid4) at (5.25,3){};
      \node[circle, scale=0.9, fill] (tid8) at (5.25,4.5){};
      \draw[](tid4) -- (tid8);
      \draw[](tid0) -- (tid1);
      \draw[](tid0) -- (tid4);
      % level separators
      \draw[dashed] (0, 2.6) -- +(10, 0) node[below left, yshift=-.125cm]{Level 0};
      \draw[dashed] (0, 4.1) -- +(10, 0) node[below left, yshift=-.125cm]{Level 1};
      \draw[dashed] (0, 5.6) -- +(10, 0) node[below left, yshift=-.125cm]{Level 2};
      \draw[dashed] (0, 7.1) -- +(10, 0) node[below left, yshift=-.125cm]{Level 3};
      \draw[      ] (0, 8.6)    +(10, 0) node[below left, yshift=-.125cm]{Level 4};
    \end{tikzpicture}
    \caption{Unlabelled without arrows, edges are implicitly towards the root.}
    \label{fig:intree-example-structure-version}
  \end{subfigure}
  \caption{Graphical representation of an intree with 5 levels (numbered 0 to 4). All edges are implicitly directed towards the root, which is drawn at the bottom of the tree. Most of the time, the \emph{structure} of the tree is enough, so we will omit vertex names most of the time.}
  \label{fig:intrees-introductory-explanation}
\end{figure}

\begin{definition}[Level]
  Let $I$ be an intree. Let $v$ be a vertex within $I$. We define $level(v)$ be number of edges along the (unique) path from $v$ to the root.
\end{definition}

The concept of levels is illustrated in figure \ref{fig:intrees-introductory-explanation}.

\todo{Anzahl der rooted trees angeben.}

%%% Local Variables:
%%% TeX-master: "../thesis.tex"
%%% End: 

%\part{Warmup: The two-processor case}
\chapter{Two Processors}
\label{chap:p2}

\section{Profiles}
\label{sec:p2-profiles}

\todo{Prove that same profiles have same runtime.}

If one considers the problem of scheduling an intree onto two processors, it becomes clear that HLF is optimal (\todo{Proof.}). \todo{Is the following correct:} Moreover, we can conclude that we can compute the optimal expected finish time in polynomial time.

This section shows how the original problem of an intree DAG can be mapped onto another, more compact structure.

\subsection{Profiles of Intrees}
\label{sec:p2-simple-method-runtime-profiles-for-intrees}

If we consider the trees in figure \ref{fig:p2-four-intrees-with-same-profile-6-3-1}, we can compute that for two processors HLF always yields an expected run time of $\frac{49}{8}$ for each of them, which is optimal:

\begin{figure}[ht]
  \centering
  \begin{tikzpicture}[scale=.2]
\node[circle, scale=0.75, fill] (tid0) at (4.5,1.5){};
\node[circle, scale=0.75, fill] (tid1) at (2.25,3){};
\node[circle, scale=0.75, fill] (tid4) at (0.75,4.5){};
\node[circle, scale=0.75, fill] (tid5) at (2.25,4.5){};
\node[circle, scale=0.75, fill] (tid6) at (3.75,4.5){};
\draw[](tid1) -- (tid4);
\draw[](tid1) -- (tid5);
\draw[](tid1) -- (tid6);
\node[circle, scale=0.75, fill] (tid2) at (6,3){};
\node[circle, scale=0.75, fill] (tid7) at (5.25,4.5){};
\node[circle, scale=0.75, fill] (tid8) at (6.75,4.5){};
\draw[](tid2) -- (tid7);
\draw[](tid2) -- (tid8);
\node[circle, scale=0.75, fill] (tid3) at (8.25,3){};
\node[circle, scale=0.75, fill] (tid9) at (8.25,4.5){};
\draw[](tid3) -- (tid9);
\draw[](tid0) -- (tid1);
\draw[](tid0) -- (tid2);
\draw[](tid0) -- (tid3);
\end{tikzpicture}
%%% Local Variables:
%%% TeX-master: "thesis/thesis.tex"
%%% End: \hspace{0.5cm}
  \input{p2/000111122_profile}\hspace{0.5cm}
  \begin{tikzpicture}[scale=.2]
\node[circle, scale=0.75, fill] (tid0) at (6,1.5){};
\node[circle, scale=0.75, fill] (tid1) at (4.5,3){};
\node[circle, scale=0.75, fill] (tid4) at (0.75,4.5){};
\node[circle, scale=0.75, fill] (tid5) at (2.25,4.5){};
\node[circle, scale=0.75, fill] (tid6) at (3.75,4.5){};
\node[circle, scale=0.75, fill] (tid7) at (5.25,4.5){};
\node[circle, scale=0.75, fill] (tid8) at (6.75,4.5){};
\node[circle, scale=0.75, fill] (tid9) at (8.25,4.5){};
\draw[](tid1) -- (tid4);
\draw[](tid1) -- (tid5);
\draw[](tid1) -- (tid6);
\draw[](tid1) -- (tid7);
\draw[](tid1) -- (tid8);
\draw[](tid1) -- (tid9);
\node[circle, scale=0.75, fill] (tid2) at (9.75,3){};
\node[circle, scale=0.75, fill] (tid3) at (11.25,3){};
\draw[](tid0) -- (tid1);
\draw[](tid0) -- (tid2);
\draw[](tid0) -- (tid3);
\end{tikzpicture}
%%% Local Variables:
%%% TeX-master: "thesis/thesis.tex"
%%% End: \hspace{0.5cm}
  \begin{tikzpicture}[scale=.2]
\node[circle, scale=0.75, fill] (tid0) at (5.25,1.5){};
\node[circle, scale=0.75, fill] (tid1) at (2.25,3){};
\node[circle, scale=0.75, fill, task_scheduled] (tid4) at (0.75,4.5){};
\node[circle, scale=0.75, fill] (tid5) at (2.25,4.5){};
\node[circle, scale=0.75, fill] (tid6) at (3.75,4.5){};
\draw[](tid1) -- (tid4);
\draw[](tid1) -- (tid5);
\draw[](tid1) -- (tid6);
\node[circle, scale=0.75, fill] (tid2) at (6.75,3){};
\node[circle, scale=0.75, fill, task_scheduled] (tid7) at (5.25,4.5){};
\node[circle, scale=0.75, fill] (tid8) at (6.75,4.5){};
\node[circle, scale=0.75, fill] (tid9) at (8.25,4.5){};
\draw[](tid2) -- (tid7);
\draw[](tid2) -- (tid8);
\draw[](tid2) -- (tid9);
\node[circle, scale=0.75, fill] (tid3) at (9.75,3){};
\draw[](tid0) -- (tid1);
\draw[](tid0) -- (tid2);
\draw[](tid0) -- (tid3);
\end{tikzpicture}
%%% Local Variables:
%%% TeX-master: "thesis/thesis.tex"
%%% End: 
  \caption{Four intrees with the same profile ($\profile{6,3,1}$). All of them have expected run time of $49/8$ if scheduled with HLF on two processors.}
  \label{fig:p2-four-intrees-with-same-profile-6-3-1}
\end{figure}

The intrees in figure \ref{fig:p2-four-intrees-with-same-profile-6-3-1} have the following in common: At each level, they have the same amount of tasks (six tasks at the topmost level, three in the middle one and one at the bottom level).

We can use the number of tasks per level as a (non-bijective) ``encoding'' of intrees. For now, we call this encoding a \emph{profile} of the intree. The above intrees would all be encoded as a profile containing the numbers 6, 3 and 1 in that order. We denote the profile by $\profile{6, 3, 1}$.

Note that not all sequences of numbers can be used as profiles. In particular, the last number in a profile is (w.l.o.g.) 1 (since we have only one task as the root of the tree)\footnote{This, of course, introduces some overhead in notation, but we leave it as it is since it is easier to read this way.}. Moreover, it can not be the case that there is a zero in a profile (since this would imply that there is \emph{no task} on one specific level in the intree).

Moreover, we introduce a abbreviating notation for profiles.

\begin{definition}[Compact notation of profiles]
  For a profile $p$, we introduce a shorthand notation that groups successive ones. That is, instead of writing $j$ consecutive ones, we simply write $\profileones{j}$.
\end{definition}

As a simple example, we rewrite $\profile{2,1,1,1,5,2,1,1,1,1,1,2,1}$ as 
$\profile{2,\profileones{3},5,2,\profileones{5},2,\profileones{1}}$.

\subsection{Profiles and HLF}
\label{sec:p2-simple-method-runtime-profiles-hlf}

For two processors and HLF-scheduling, we can easily conclude the successors of a profile. Let us first of all consider some examples here: If we have the profile $\profile{5,4,2,1}$, then two of the five topmost tasks \emph{have to be scheduled} (since we are using HLF). If one of these two topmost tasks is finished, we reach $\profile{4,4,2,1}$ (see figure \ref{fig:p2-profiles-successors-of-5421-always-same} for reference).

\begin{figure}[ht]
  \centering
  \renewcommand{\leveltopI}{-10cm + \leveltop}
\renewcommand{\leveltopII}{-10cm + \leveltopI}
\renewcommand{\leveltopIII}{-10cm + \leveltopII}
\renewcommand{\leveltopIIII}{-10cm + \leveltopIII}
\renewcommand{\leveltopIIIII}{-10cm + \leveltopIIII}
\renewcommand{\leveltopIIIIII}{-10cm + \leveltopIIIII}
\renewcommand{\leveltopIIIIIII}{-10cm + \leveltopIIIIII}
\renewcommand{\leveltopIIIIIIII}{-10cm + \leveltopIIIIIII}
\renewcommand{\leveltopIIIIIIIII}{-10cm + \leveltopIIIIIIII}
\renewcommand{\leveltopIIIIIIIIII}{-10cm + \leveltopIIIIIIIII}
\renewcommand{\leveltopIIIIIIIIIII}{-10cm + \leveltopIIIIIIIIII}
\renewcommand{\leveltopIIIIIIIIIIII}{-10cm + \leveltopIIIIIIIIIII}
\renewcommand{\leveltopI}{-10cm + \leveltop}
\renewcommand{\leveltopII}{-10cm + \leveltopI}
\renewcommand{\leveltopIII}{-10cm + \leveltopII}
\renewcommand{\leveltopIIII}{-10cm + \leveltopIII}
\renewcommand{\leveltopIIIII}{-10cm + \leveltopIIII}
\renewcommand{\leveltopIIIIII}{-10cm + \leveltopIIIII}
\renewcommand{\leveltopIIIIIII}{-10cm + \leveltopIIIIII}
\renewcommand{\leveltopIIIIIIII}{-10cm + \leveltopIIIIIII}
\renewcommand{\leveltopIIIIIIIII}{-10cm + \leveltopIIIIIIII}
\renewcommand{\leveltopIIIIIIIIII}{-10cm + \leveltopIIIIIIIII}
\renewcommand{\leveltopIIIIIIIIIII}{-10cm + \leveltopIIIIIIIIII}
\renewcommand{\leveltopIIIIIIIIIIII}{-10cm + \leveltopIIIIIIIIIII}
\begin{tikzpicture}[scale=.2, anchor=south]
\begin{scope}[yshift=\leveltopI cm]
\matrix (line1)[column sep=0.1cm] {
\node[draw=black, rectangle split,  rectangle split parts=1] (sn0x9b5d5c8){
\begin{tikzpicture}[scale=.2]
\node[circle, scale=0.75, fill] (tid0) at (4.5,1.5){};
\node[circle, scale=0.75, fill] (tid1) at (3,3){};
\node[circle, scale=0.75, fill] (tid3) at (1.5,4.5){};
\node[circle, scale=0.75, fill, task_scheduled] (tid7) at (0.75,6){};
\node[circle, scale=0.75, fill, task_scheduled] (tid8) at (2.25,6){};
\draw[](tid3) -- (tid7);
\draw[](tid3) -- (tid8);
\node[circle, scale=0.75, fill] (tid4) at (3.75,4.5){};
\node[circle, scale=0.75, fill] (tid9) at (3.75,6){};
\draw[](tid4) -- (tid9);
\node[circle, scale=0.75, fill] (tid5) at (5.25,4.5){};
\draw[](tid1) -- (tid3);
\draw[](tid1) -- (tid4);
\draw[](tid1) -- (tid5);
\node[circle, scale=0.75, fill] (tid2) at (7.5,3){};
\node[circle, scale=0.75, fill] (tid6) at (7.5,4.5){};
\node[circle, scale=0.75, fill] (tid10) at (6.75,6){};
\node[circle, scale=0.75, fill] (tid11) at (8.25,6){};
\draw[](tid6) -- (tid10);
\draw[](tid6) -- (tid11);
\draw[](tid2) -- (tid6);
\draw[](tid0) -- (tid1);
\draw[](tid0) -- (tid2);
\end{tikzpicture}
};
 & 
\\
};
\end{scope}
\begin{scope}[yshift=\leveltopII cm]
\matrix (line2)[column sep=0.1cm] {
\node[draw=black, rectangle split,  rectangle split parts=1] (sn0x9b5de08){
\begin{tikzpicture}[scale=.2]
\node[circle, scale=0.75, fill] (tid0) at (3.75,1.5){};
\node[circle, scale=0.75, fill] (tid1) at (2.25,3){};
\node[circle, scale=0.75, fill] (tid3) at (0.75,4.5){};
\node[circle, scale=0.75, fill, task_scheduled] (tid7) at (0.75,6){};
\draw[](tid3) -- (tid7);
\node[circle, scale=0.75, fill] (tid4) at (2.25,4.5){};
\node[circle, scale=0.75, fill, task_scheduled] (tid8) at (2.25,6){};
\draw[](tid4) -- (tid8);
\node[circle, scale=0.75, fill] (tid5) at (3.75,4.5){};
\draw[](tid1) -- (tid3);
\draw[](tid1) -- (tid4);
\draw[](tid1) -- (tid5);
\node[circle, scale=0.75, fill] (tid2) at (6,3){};
\node[circle, scale=0.75, fill] (tid6) at (6,4.5){};
\node[circle, scale=0.75, fill] (tid9) at (5.25,6){};
\node[circle, scale=0.75, fill] (tid10) at (6.75,6){};
\draw[](tid6) -- (tid9);
\draw[](tid6) -- (tid10);
\draw[](tid2) -- (tid6);
\draw[](tid0) -- (tid1);
\draw[](tid0) -- (tid2);
\end{tikzpicture}
};
 & 
\node[draw=black, rectangle split,  rectangle split parts=1] (sn0x9b58978){
\begin{tikzpicture}[scale=.2]
\node[circle, scale=0.75, fill] (tid0) at (3.75,1.5){};
\node[circle, scale=0.75, fill] (tid1) at (2.25,3){};
\node[circle, scale=0.75, fill] (tid3) at (0.75,4.5){};
\node[circle, scale=0.75, fill, task_scheduled] (tid7) at (0.75,6){};
\draw[](tid3) -- (tid7);
\node[circle, scale=0.75, fill] (tid4) at (2.25,4.5){};
\node[circle, scale=0.75, fill] (tid8) at (2.25,6){};
\draw[](tid4) -- (tid8);
\node[circle, scale=0.75, fill] (tid5) at (3.75,4.5){};
\draw[](tid1) -- (tid3);
\draw[](tid1) -- (tid4);
\draw[](tid1) -- (tid5);
\node[circle, scale=0.75, fill] (tid2) at (6,3){};
\node[circle, scale=0.75, fill] (tid6) at (6,4.5){};
\node[circle, scale=0.75, fill, task_scheduled] (tid9) at (5.25,6){};
\node[circle, scale=0.75, fill] (tid10) at (6.75,6){};
\draw[](tid6) -- (tid9);
\draw[](tid6) -- (tid10);
\draw[](tid2) -- (tid6);
\draw[](tid0) -- (tid1);
\draw[](tid0) -- (tid2);
\end{tikzpicture}
};
 & 
\\
};
\end{scope}
\draw (sn0x9b5d5c8.south) -- (sn0x9b5de08.north);
\draw (sn0x9b5d5c8.south) -- (sn0x9b58978.north);
\end{tikzpicture}
\renewcommand{\leveltopI}{-10cm + \leveltop}
\renewcommand{\leveltopII}{-10cm + \leveltopI}
\renewcommand{\leveltopIII}{-10cm + \leveltopII}
\renewcommand{\leveltopIIII}{-10cm + \leveltopIII}
\renewcommand{\leveltopIIIII}{-10cm + \leveltopIIII}
\renewcommand{\leveltopIIIIII}{-10cm + \leveltopIIIII}
\renewcommand{\leveltopIIIIIII}{-10cm + \leveltopIIIIII}
\renewcommand{\leveltopIIIIIIII}{-10cm + \leveltopIIIIIII}
\renewcommand{\leveltopIIIIIIIII}{-10cm + \leveltopIIIIIIII}
\renewcommand{\leveltopIIIIIIIIII}{-10cm + \leveltopIIIIIIIII}
\renewcommand{\leveltopIIIIIIIIIII}{-10cm + \leveltopIIIIIIIIII}
\renewcommand{\leveltopIIIIIIIIIIII}{-10cm + \leveltopIIIIIIIIIII}
\begin{tikzpicture}[scale=.2, anchor=south]
\begin{scope}[yshift=\leveltopI cm]
\matrix (line1)[column sep=0.1cm] {
\node[draw=black, rectangle split,  rectangle split parts=1] (sn0x9b5ed70){
\begin{tikzpicture}[scale=.2]
\node[circle, scale=0.75, fill] (tid0) at (4.5,1.5){};
\node[circle, scale=0.75, fill] (tid1) at (3,3){};
\node[circle, scale=0.75, fill] (tid3) at (1.5,4.5){};
\node[circle, scale=0.75, fill, task_scheduled] (tid7) at (0.75,6){};
\node[circle, scale=0.75, fill] (tid8) at (2.25,6){};
\draw[](tid3) -- (tid7);
\draw[](tid3) -- (tid8);
\node[circle, scale=0.75, fill] (tid4) at (3.75,4.5){};
\node[circle, scale=0.75, fill, task_scheduled] (tid9) at (3.75,6){};
\draw[](tid4) -- (tid9);
\node[circle, scale=0.75, fill] (tid5) at (5.25,4.5){};
\draw[](tid1) -- (tid3);
\draw[](tid1) -- (tid4);
\draw[](tid1) -- (tid5);
\node[circle, scale=0.75, fill] (tid2) at (7.5,3){};
\node[circle, scale=0.75, fill] (tid6) at (7.5,4.5){};
\node[circle, scale=0.75, fill] (tid10) at (6.75,6){};
\node[circle, scale=0.75, fill] (tid11) at (8.25,6){};
\draw[](tid6) -- (tid10);
\draw[](tid6) -- (tid11);
\draw[](tid2) -- (tid6);
\draw[](tid0) -- (tid1);
\draw[](tid0) -- (tid2);
\end{tikzpicture}
};
 & 
\\
};
\end{scope}
\begin{scope}[yshift=\leveltopII cm]
\matrix (line2)[column sep=0.1cm] {
\node[draw=black, rectangle split,  rectangle split parts=1] (sn0x9b5e2b0){
\begin{tikzpicture}[scale=.2]
\node[circle, scale=0.75, fill] (tid0) at (4.5,1.5){};
\node[circle, scale=0.75, fill] (tid1) at (3,3){};
\node[circle, scale=0.75, fill] (tid3) at (1.5,4.5){};
\node[circle, scale=0.75, fill, task_scheduled] (tid7) at (0.75,6){};
\node[circle, scale=0.75, fill, task_scheduled] (tid8) at (2.25,6){};
\draw[](tid3) -- (tid7);
\draw[](tid3) -- (tid8);
\node[circle, scale=0.75, fill] (tid4) at (3.75,4.5){};
\node[circle, scale=0.75, fill] (tid5) at (5.25,4.5){};
\draw[](tid1) -- (tid3);
\draw[](tid1) -- (tid4);
\draw[](tid1) -- (tid5);
\node[circle, scale=0.75, fill] (tid2) at (7.5,3){};
\node[circle, scale=0.75, fill] (tid6) at (7.5,4.5){};
\node[circle, scale=0.75, fill] (tid9) at (6.75,6){};
\node[circle, scale=0.75, fill] (tid10) at (8.25,6){};
\draw[](tid6) -- (tid9);
\draw[](tid6) -- (tid10);
\draw[](tid2) -- (tid6);
\draw[](tid0) -- (tid1);
\draw[](tid0) -- (tid2);
\end{tikzpicture}
};
 & 
\node[draw=black, rectangle split,  rectangle split parts=1] (sn0x9b5ebe0){
\begin{tikzpicture}[scale=.2]
\node[circle, scale=0.75, fill] (tid0) at (4.5,1.5){};
\node[circle, scale=0.75, fill] (tid1) at (3,3){};
\node[circle, scale=0.75, fill] (tid3) at (1.5,4.5){};
\node[circle, scale=0.75, fill, task_scheduled] (tid7) at (0.75,6){};
\node[circle, scale=0.75, fill] (tid8) at (2.25,6){};
\draw[](tid3) -- (tid7);
\draw[](tid3) -- (tid8);
\node[circle, scale=0.75, fill] (tid4) at (3.75,4.5){};
\node[circle, scale=0.75, fill] (tid5) at (5.25,4.5){};
\draw[](tid1) -- (tid3);
\draw[](tid1) -- (tid4);
\draw[](tid1) -- (tid5);
\node[circle, scale=0.75, fill] (tid2) at (7.5,3){};
\node[circle, scale=0.75, fill] (tid6) at (7.5,4.5){};
\node[circle, scale=0.75, fill, task_scheduled] (tid9) at (6.75,6){};
\node[circle, scale=0.75, fill] (tid10) at (8.25,6){};
\draw[](tid6) -- (tid9);
\draw[](tid6) -- (tid10);
\draw[](tid2) -- (tid6);
\draw[](tid0) -- (tid1);
\draw[](tid0) -- (tid2);
\end{tikzpicture}
};
 & 
\node[draw=black, rectangle split,  rectangle split parts=1] (sn0x9b5de08){
\begin{tikzpicture}[scale=.2]
\node[circle, scale=0.75, fill] (tid0) at (3.75,1.5){};
\node[circle, scale=0.75, fill] (tid1) at (2.25,3){};
\node[circle, scale=0.75, fill] (tid3) at (0.75,4.5){};
\node[circle, scale=0.75, fill, task_scheduled] (tid7) at (0.75,6){};
\draw[](tid3) -- (tid7);
\node[circle, scale=0.75, fill] (tid4) at (2.25,4.5){};
\node[circle, scale=0.75, fill, task_scheduled] (tid8) at (2.25,6){};
\draw[](tid4) -- (tid8);
\node[circle, scale=0.75, fill] (tid5) at (3.75,4.5){};
\draw[](tid1) -- (tid3);
\draw[](tid1) -- (tid4);
\draw[](tid1) -- (tid5);
\node[circle, scale=0.75, fill] (tid2) at (6,3){};
\node[circle, scale=0.75, fill] (tid6) at (6,4.5){};
\node[circle, scale=0.75, fill] (tid9) at (5.25,6){};
\node[circle, scale=0.75, fill] (tid10) at (6.75,6){};
\draw[](tid6) -- (tid9);
\draw[](tid6) -- (tid10);
\draw[](tid2) -- (tid6);
\draw[](tid0) -- (tid1);
\draw[](tid0) -- (tid2);
\end{tikzpicture}
};
 & 
\node[draw=black, rectangle split,  rectangle split parts=1] (sn0x9b58978){
\begin{tikzpicture}[scale=.2]
\node[circle, scale=0.75, fill] (tid0) at (3.75,1.5){};
\node[circle, scale=0.75, fill] (tid1) at (2.25,3){};
\node[circle, scale=0.75, fill] (tid3) at (0.75,4.5){};
\node[circle, scale=0.75, fill, task_scheduled] (tid7) at (0.75,6){};
\draw[](tid3) -- (tid7);
\node[circle, scale=0.75, fill] (tid4) at (2.25,4.5){};
\node[circle, scale=0.75, fill] (tid8) at (2.25,6){};
\draw[](tid4) -- (tid8);
\node[circle, scale=0.75, fill] (tid5) at (3.75,4.5){};
\draw[](tid1) -- (tid3);
\draw[](tid1) -- (tid4);
\draw[](tid1) -- (tid5);
\node[circle, scale=0.75, fill] (tid2) at (6,3){};
\node[circle, scale=0.75, fill] (tid6) at (6,4.5){};
\node[circle, scale=0.75, fill, task_scheduled] (tid9) at (5.25,6){};
\node[circle, scale=0.75, fill] (tid10) at (6.75,6){};
\draw[](tid6) -- (tid9);
\draw[](tid6) -- (tid10);
\draw[](tid2) -- (tid6);
\draw[](tid0) -- (tid1);
\draw[](tid0) -- (tid2);
\end{tikzpicture}
};
 & 
\\
};
\end{scope}
\draw (sn0x9b5ed70.south) -- (sn0x9b5e2b0.north);
\draw (sn0x9b5ed70.south) -- (sn0x9b5ebe0.north);
\draw (sn0x9b5ed70.south) -- (sn0x9b5de08.north);
\draw (sn0x9b5ed70.south) -- (sn0x9b58978.north);
\end{tikzpicture}
%%% Local Variables:
%%% TeX-master: "thesis/thesis.tex"
%%% End: 

  \caption{Intree with profile $\profile{5,4,2,1}$. \emph{All} possible HLF-successors of the original intree have profile $\profile{4,4,2,1}$.}
  \label{fig:p2-profiles-successors-of-5421-always-same}
\end{figure}

\todo{More figures.}

Another interesting case is $\profile{1,5,2,1}$, where the (single) topmost task and one of the five tasks on the second level are scheduled. If the topmost task is finished (which happens with probability $\frac{1}{2}$), we reach $\profile{5,2,1}$. If the scheduled task on the second level finishes first, we reach $\profile{1,4,2,1}$.

The last example we want to give here is $\profile{1,1,1,3,1}$. In this case, the single topmost task and one of the three tasks of the second lowest level have to be scheduled. If the topmost task finishes first (which happens with probability $\frac{1}{2}$), the resulting profile will be $\profile{1,1,3,1}$ (where again the topmost task and one of the tree tasks in the second lowest level are scheduled). If the other scheduled task finishes first, we reach $\profile{1,1,1,2,1}$, where the single topmost task and one of the remaining \emph{two} tasks on the second lowest level are scheduled.

\section{Expected runtime for two processor HLF using profiles}
\label{sec:p2-profiles-hlf-exp-runtime}

We now use the profile notation to denote the expected run time (i.e. we say $\E{\profile{6,3,1}} = \frac{49}{8}$ --- see figure \ref{fig:p2-four-intrees-with-same-profile-6-3-1}).

\subsection{A recursive definition}
\label{sec:p2-profile-exp-run-time-rec-def}

Exploiting profile notation, we can define the following recursive formula that can be used to compute the optimal expected run time:

\begin{equation}
  \label{eq:p2-profile-optimal-exp-run-time}
  \E{\profile{n_1, \dots, n_r}} =
  \begin{cases}
    r, & \text{ if } n_1 = n_2 = \dots = n_r = 1 \\
    \frac{n_1-1}{2} + \E{\profile{1, n_2, n_3, \dots, n_r}} , & \text{ if } n_r\geq 2 \\
    \frac{1}{2} + \frac{1}{2} \cdot \left( \E{\profile{n_2, \dots, n_r}} + \E{SUC(\profile{n_1,\dots,n_r})} \right) ,& \text{ otherwise }
  \end{cases},
\end{equation}
where $SUC(\profile{n_1,\dots,n_r}) = \profile{n_1, n_2, n_3,\dots,n_{j-1},n_j-1,n_{j+1},\dots,n_r}$ such that $j$ is the minimum index such that $n_j>1$.

If we consider the second case of equation (\ref{eq:p2-profile-optimal-exp-run-time}), we see that we can simplify it to the following:

\begin{equation}
  \label{eq:p2-profile-optimal-exp-run-time-def-simplified}
  \E{\profile{n_1, \dots, n_r}} =
  \begin{cases}
    r, & \text{ if } n_1 = n_2 = \dots = n_r = 1 \\
    \frac{n_1}{2} + \frac{1}{2} \cdot \left( \E{\profile{n_2, \dots, n_r}} + \E{SUC(\profile{1,n_2,\dots,n_r})} \right) ,& \text{ otherwise }
  \end{cases},
\end{equation}
with $SUC$ as defined before.

\subsection{Solving the recurrence for special cases}
\label{sec:p2-profile-exp-runtime-closed-form-spec-cases}

Unfortunately, the recurrence relation in equation (\ref{eq:p2-profile-optimal-exp-run-time-def-simplified}) does not significantly simplify the original problem. However, we were able to deduce a closed form that can be used for special cases.

\begin{theorem}
  \label{theo:simple-profiles-exp-runtime-for-p2-hlf}
  Let $\profile{n_1,\profileones{j-2},n_j,\profileones{r-j}}$ be a profile 
  %where 
  %$\left|\left\{ i \in \left\{ 2,3,\dots,r \right\} \mid n_i > 1 \right\}\right| \leq 1$ 
  (i.e. at most the first and one other entry of the profile are different from 1).
  Then it holds that
  \begin{equation*}
    \E{\profile{n_1,\profileones{j-2},n_j,\profileones{r-j}}} = 
    r + \frac{A_0(n_1-2)}{2^{n_1-1}} + \frac{A_{j-1}(n_j-2)}{2^{n_j+j-2}},
  \end{equation*}
  % Then we can compute 
  % \begin{equation*}
  %   \profile{n_1,n_2,\dots,n_r} = 
  %   r + \sum_{i=1}^r \left( \frac{A_{i-1}(n_i - 2)}{2^{n_i+i-2}} \right),
  % \end{equation*}
  where $A_i$ is inductively defined as follows:
  \begin{align*}
    A_0(n) & = (n+1) \cdot 2^n \\
    A_{i+1}(n) & = \sum_{k=0}^n A_{i}(k)
  \end{align*}
\end{theorem}

\begin{table}
  \centering
  \begin{tabular}[ht]{ccccccccl}
    $n$ & -1 & 0 & 1 & 2 & 3 & 4 & 5 & Closed term \\
    \hline
    $A_0(n)$ & 0 & 1 & 4 & 12 & 32 & 80 & 192 & 
    $(n+1)\cdot 2^{n}$ \\
    $A_1(n)$ & 0 & 1 & 5 & 17 & 49 & 129 & 321 & 
    $n\cdot 2^{n+1} + 1$ \\
    $A_2(n)$ & 0 & 1 & 6 & 23 & 72 & 201& 522 & 
    $(n-1)\cdot 2^{n+2}+n+5$ \\
    $A_3(n)$ & 0 & 1 & 7 & 30 & 102 & 303 & 825 & 
    $(n-2)\cdot 2^{n+3}+(n^2+11 n+34)/2$ \\
    $A_4(n)$ & 0 & 1 & 8 & 38 & 140 & 443 & 1268 &
    $(n-3)\cdot2^{n+4}+\binom{n+3}{3}+4\cdot\left(\binom{n+1}{2}+4 n+12\right)$ \\
  \end{tabular}
  \caption{Example values for $A_i(n)$. \todo{OEIS zitieren.}}
  \label{tab:example-values-an-p2-profile}
\end{table}

Before we prove theorem \ref{theo:simple-profiles-exp-runtime-for-p2-hlf}, let us have a look at table \ref{tab:example-values-an-p2-profile} showing values for $A_i(n)$ for small values of $i$ and $n$.
%$(i,n) \in \left( \left\{ 0,1,\dots,4 \right\} \times \left\{ -1,0,1,2,\dots,5 \right\} \right)$. 

From this table and by looking at the definition of $A_i(n)$ we can deduce a simple lemma that will later be useful.

Note that there are closed expressions for $A_i(n)$ for $i\leq 5$ (and possibly for higher values of $i$, as well). However, these formulae are quite complex (also see table \ref{tab:example-values-an-p2-profile}) and we were not able to deduce a \emph{simple} pattern according to which $A_i(n)$ can be constructed\todo[Hr. Mayr]{Sagen Ihnen die Polynombestandteile in Tabelle \ref{tab:example-values-an-p2-profile} etwas?}. It seems that $A_i(n)$ involves the term $(n+1-i)\cdot 2^{n+i}$ in some way, and the remaining term seems to be a polynomial in $n$.

\begin{lemma}
  \label{lemma:p2-hlf-profiles-an-simple-recurrence}
  Let $A_i(n)$ be as defined in theorem \ref{theo:simple-profiles-exp-runtime-for-p2-hlf}. Then, we have
  \begin{equation*}
    A_{j-1}(n) + A_{j}(n-1) = A_{j}(n).
  \end{equation*}
\end{lemma}

\begin{proof}
  Proof is trivial by definition of $A_{j}(n) = \sum_{k=0}^{n} A_{j-1}(k) = A_{j-1}(n) + \sum_{k=0}^{n-1} A_{j-1}(k) = A_{j-1}(n) + A_{j}(n-1)$.
\end{proof}

We can now proof theorem \ref{theo:simple-profiles-exp-runtime-for-p2-hlf}.

\begin{proof}[Proof of theorem \ref{theo:simple-profiles-exp-runtime-for-p2-hlf}]
  We prove theorem \ref{theo:simple-profiles-exp-runtime-for-p2-hlf} by complete induction. The base case $\E{\profile{\profileones{r}}} = r$ is clear because in this case we can always schedule exactly one task. Since there are $r$ tasks in total, this results in an expected run time of $r$ (each task is expected to be exponentially distributed with expectation 1).

  We now consider the special cases where \emph{all elements but one} in the profile are 1. That is, we consider the profile, whose elements are all 1, exctept the element at position $j$, which will be $n$. That is, we examine
  \begin{equation*}
    \profile{\profileones{j-1},n,\profileones{r-j}}.
  \end{equation*}
  We can rewrite this using the definition and afterwards apply the induction hypothesis:

  \begin{eqnarray*}
    \E{\profile{\profileones{j-1},n,\profileones{r-j}}}
    & = & 
    \frac{1}{2} + \frac{1}{2} \cdot 
    \left( 
      \E{\profile{\profileones{j-2},n,\profileones{r-j}}} + 
      \E{\profile{\profileones{j-1},n-1,\profileones{r-j}}}
    \right) = \\
    & = & 
    \frac{1}{2} + \frac{1}{2} \cdot 
    \left( 
      (r-1) + \frac{A_{j-2}(n-2)}{2^{n+(j-1)-2}} +
      r + \frac{A_{j-1}(n-3)}{2^{(n-1)+j-2}}
    \right)
  \end{eqnarray*}
  We now apply lemma Lemma \ref{lemma:p2-hlf-profiles-an-simple-recurrence} and obtain
  \begin{eqnarray*}
    \profile{\profileones{j-1},n,\profileones{r-j}}
    & = & 
    \frac{1}{2} + \frac{1}{2} \cdot 
    \left( 
      (r-1)+r + 
      \frac{A_{j-2}(n-2) + A_{j-1}(n-3)}{2^{n+j-3}}
    \right) = \\
    & = &
    \frac{1}{2} + \frac{1}{2} \cdot 
    \left( 
      2r - 1
      \frac{A_{j-1}(n-2)}{2^{n+j-3}}
    \right) = \\
    & = &
    \frac{1}{2} + 
    r - \frac{1}{2}
    \frac{A_{j-1}(n-2)}{2^{n+j-2}} = \\
    & = &
    r + \frac{A_{j-1}(n-2)}{2^{n+j-2}}
  \end{eqnarray*}

  We conclude the proof by deriving the expected run time for $\profile{m,\profileones{j-2},n,\profileones{r-j}}$. We do this by applying the definition and thereby reducing the problem to $\profile{\profileones{j-1},n,\profileones{r-j}}$:

  \begin{eqnarray*}
    \E{\profile{m,\profileones{j-2},n,\profileones{r-j}}}
    & = & 
    \frac{m-1}{2} + \E{\profile{\profileones{j-1},n,\profileones{r-j}}} = \\
    & = &
    \frac{m-1}{2} + r + \frac{A_{j-1}(n-2)}{2^{n+j-2}} = \\
    & = &
    \frac{(m-1)\cdot 2^{m-2}}{2^{m-1}} + r + \frac{A_{j-1}(n-2)}{2^{n+j-2}} = \\
    & = &
    \frac{A_0(m-2)}{2^{m-1}} + r + \frac{A_{j-1}(n-2)}{2^{n+j-2}}
  \end{eqnarray*}
  
  This concludes the proof.
\end{proof}

Moritz Maaß has shown another property in \cite{MoritzMaasDiploma}, that we are able to generalize.

\begin{theorem}[Intrees with exactly two leaves and intrees with same profiles]
  Let $l, k\in\naturals$, $a\in\naturals_0$ and $\profile{\profilerepeat{1}{l-k}, \profilerepeat{2}{k}, \profilerepeat{1}{a+1}}$ be a profile. Then, it holds that
  \begin{eqnarray*}
    \E{\profile{\profilerepeat{1}{l-k}, \profilerepeat{2}{k}, \profilerepeat{1}{a+1}}}
    = & &
    % the following is an incorrect simplification of Maple
    % a
    % + 2 
    % - \frac{\binom{l+k-1}{k} + \binom{l+k-1}{l}}{2^{l+k}}
    % +  \sum_{i=1}^k \sum_{j=1}^l \left( \frac{1}{2} \right)^{k-i+l-j+1}\cdot \binom{k-i+l-j}{l-j}
    \sum_{i=1}^k \left(\frac{1}{2}\right)^{l+i-1} \cdot \binom{l+i-2}{i-1} \cdot \left( k-i+2 \right) \\
    & + & \sum_{j=1}^l \left(\frac{1}{2}\right)^{k+j-1} \cdot \binom{k+j-2}{j-1} \cdot \left( l-j+2 \right) \\
    & + & \sum_{i=1}^k \sum_{j=1}^l \left( \frac{1}{2}^{k-i+l-j+1}\cdot\binom{ki+l-j}{l-j} \right) \\
    & + & a
    .
  \end{eqnarray*}
\end{theorem}

\begin{proof}
  \todo{Proof.}
\end{proof}

Even if we were not able to deduce a more general formula that holds if more entries in the profile differ from 1, this might serve as a starting point for a more advanced proof.

\todo{Tabelle, die einige Beispiele von Profiles zeigt und deren Erwartungswerte.}

\section{Profile DAGs}
\label{sec:p2-profile-dags}

As seen in the previous chapter, a closed formula for $\E{\profile{n_1,\dots,n_r}}$ seems to be quite complex. This is why we may compute the expected runtime just with the recursive approach given in equation (\ref{eq:p2-profile-optimal-exp-run-time-def-simplified}).

Of course, it is an interesting question how complex this computation is. Therefore, we consider the \emph{profile DAG}. The profile DAG is -- intuitively -- a coarsening of the original snapshot DAG. It is created the following way: We ``merge'' snapshots having the same profile, thereby decreasing the number of vertices in the DAG. Figure \ref{fig:p2-profile-dag-example-000111223-hlfdet} shows a snapshot DAG and its corresponding profile DAG.

\begin{figure}[t]
  \centering
  \renewcommand{\leveltopI}{-11cm + \leveltop}
\renewcommand{\leveltopII}{-11cm + \leveltopI}
\renewcommand{\leveltopIII}{-11cm + \leveltopII}
\renewcommand{\leveltopIIII}{-10cm + \leveltopIII}
\renewcommand{\leveltopIIIII}{-10cm + \leveltopIIII}
\renewcommand{\leveltopIIIIII}{-10cm + \leveltopIIIII}
\renewcommand{\leveltopIIIIIII}{-10cm + \leveltopIIIIII}
\renewcommand{\leveltopIIIIIIII}{-9cm + \leveltopIIIIIII}
\renewcommand{\leveltopIIIIIIIII}{-8cm + \leveltopIIIIIIII}
\renewcommand{\leveltopIIIIIIIIII}{-7cm + \leveltopIIIIIIIII}
\begin{tikzpicture}[scale=.2, anchor=south, rotate=90]
\begin{scope}[yshift=\leveltopI cm, anchor = center]
\matrix (line1)[row sep=0.5cm] {
\node[draw=black, rectangle split,  rectangle split parts=3] (sn0x1c22990){
\footnotesize{100}
\nodepart{two}
\begin{tikzpicture}[scale=.2]
\node[circle, scale=0.75, fill] (tid0) at (4.5,1.5){};
\node[circle, scale=0.75, fill] (tid1) at (2.25,3){};
\node[circle, scale=0.75, fill] (tid4) at (0.75,4.5){};
\node[circle, scale=0.75, fill] (tid5) at (2.25,4.5){};
\node[circle, scale=0.75, fill] (tid6) at (3.75,4.5){};
\draw[](tid1) -- (tid4);
\draw[](tid1) -- (tid5);
\draw[](tid1) -- (tid6);
\node[circle, scale=0.75, fill] (tid2) at (6,3){};
\node[circle, scale=0.75, fill, task_scheduled] (tid7) at (5.25,4.5){};
\node[circle, scale=0.75, fill] (tid8) at (6.75,4.5){};
\draw[](tid2) -- (tid7);
\draw[](tid2) -- (tid8);
\node[circle, scale=0.75, fill] (tid3) at (8.25,3){};
\node[circle, scale=0.75, fill, task_scheduled] (tid9) at (8.25,4.5){};
\draw[](tid3) -- (tid9);
\draw[](tid0) -- (tid1);
\draw[](tid0) -- (tid2);
\draw[](tid0) -- (tid3);
\end{tikzpicture}
\nodepart{three}
\footnotesize{$50\:50$}
};
 \\ 
\\
};
\end{scope}
\begin{scope}[yshift=\leveltopII cm, anchor = center]
\matrix (line2)[row sep=0.5cm] {
\node[draw=black, rectangle split,  rectangle split parts=3] (sn0x1c29d30){
\footnotesize{50}
\nodepart{two}
\begin{tikzpicture}[scale=.2]
\node[circle, scale=0.75, fill] (tid0) at (4.5,1.5){};
\node[circle, scale=0.75, fill] (tid1) at (2.25,3){};
\node[circle, scale=0.75, fill, task_scheduled] (tid4) at (0.75,4.5){};
\node[circle, scale=0.75, fill] (tid5) at (2.25,4.5){};
\node[circle, scale=0.75, fill] (tid6) at (3.75,4.5){};
\draw[](tid1) -- (tid4);
\draw[](tid1) -- (tid5);
\draw[](tid1) -- (tid6);
\node[circle, scale=0.75, fill] (tid2) at (6,3){};
\node[circle, scale=0.75, fill, task_scheduled] (tid7) at (5.25,4.5){};
\node[circle, scale=0.75, fill] (tid8) at (6.75,4.5){};
\draw[](tid2) -- (tid7);
\draw[](tid2) -- (tid8);
\node[circle, scale=0.75, fill] (tid3) at (8.25,3){};
\draw[](tid0) -- (tid1);
\draw[](tid0) -- (tid2);
\draw[](tid0) -- (tid3);
\end{tikzpicture}
\nodepart{three}
\footnotesize{$50\:50$}
};
 \\ 
\node[draw=black, rectangle split,  rectangle split parts=3] (sn0x1c28c10){
\footnotesize{50}
\nodepart{two}
\begin{tikzpicture}[scale=.2]
\node[circle, scale=0.75, fill] (tid0) at (3.75,1.5){};
\node[circle, scale=0.75, fill] (tid1) at (2.25,3){};
\node[circle, scale=0.75, fill, task_scheduled] (tid4) at (0.75,4.5){};
\node[circle, scale=0.75, fill] (tid5) at (2.25,4.5){};
\node[circle, scale=0.75, fill] (tid6) at (3.75,4.5){};
\draw[](tid1) -- (tid4);
\draw[](tid1) -- (tid5);
\draw[](tid1) -- (tid6);
\node[circle, scale=0.75, fill] (tid2) at (5.25,3){};
\node[circle, scale=0.75, fill, task_scheduled] (tid7) at (5.25,4.5){};
\draw[](tid2) -- (tid7);
\node[circle, scale=0.75, fill] (tid3) at (6.75,3){};
\node[circle, scale=0.75, fill] (tid8) at (6.75,4.5){};
\draw[](tid3) -- (tid8);
\draw[](tid0) -- (tid1);
\draw[](tid0) -- (tid2);
\draw[](tid0) -- (tid3);
\end{tikzpicture}
\nodepart{three}
\footnotesize{$50\:50$}
};
 \\ 
\\
};
\end{scope}
\begin{scope}[yshift=\leveltopIII cm, anchor = center]
\matrix (line3)[row sep=0.5cm] {
\node[draw=black, rectangle split,  rectangle split parts=3] (sn0x1c29050){
\footnotesize{50}
\nodepart{two}
\begin{tikzpicture}[scale=.2]
\node[circle, scale=0.75, fill] (tid0) at (3.75,1.5){};
\node[circle, scale=0.75, fill] (tid1) at (2.25,3){};
\node[circle, scale=0.75, fill, task_scheduled] (tid4) at (0.75,4.5){};
\node[circle, scale=0.75, fill, task_scheduled] (tid5) at (2.25,4.5){};
\node[circle, scale=0.75, fill] (tid6) at (3.75,4.5){};
\draw[](tid1) -- (tid4);
\draw[](tid1) -- (tid5);
\draw[](tid1) -- (tid6);
\node[circle, scale=0.75, fill] (tid2) at (5.25,3){};
\node[circle, scale=0.75, fill] (tid7) at (5.25,4.5){};
\draw[](tid2) -- (tid7);
\node[circle, scale=0.75, fill] (tid3) at (6.75,3){};
\draw[](tid0) -- (tid1);
\draw[](tid0) -- (tid2);
\draw[](tid0) -- (tid3);
\end{tikzpicture}
\nodepart{three}
\footnotesize{$1$}
};
 \\ 
\node[draw=black, rectangle split,  rectangle split parts=3] (sn0x1c29fa0){
\footnotesize{25}
\nodepart{two}
\begin{tikzpicture}[scale=.2]
\node[circle, scale=0.75, fill] (tid0) at (3.75,1.5){};
\node[circle, scale=0.75, fill] (tid1) at (1.5,3){};
\node[circle, scale=0.75, fill, task_scheduled] (tid4) at (0.75,4.5){};
\node[circle, scale=0.75, fill] (tid5) at (2.25,4.5){};
\draw[](tid1) -- (tid4);
\draw[](tid1) -- (tid5);
\node[circle, scale=0.75, fill] (tid2) at (4.5,3){};
\node[circle, scale=0.75, fill, task_scheduled] (tid6) at (3.75,4.5){};
\node[circle, scale=0.75, fill] (tid7) at (5.25,4.5){};
\draw[](tid2) -- (tid6);
\draw[](tid2) -- (tid7);
\node[circle, scale=0.75, fill] (tid3) at (6.75,3){};
\draw[](tid0) -- (tid1);
\draw[](tid0) -- (tid2);
\draw[](tid0) -- (tid3);
\end{tikzpicture}
\nodepart{three}
\footnotesize{$50\:50$}
};
 \\ 
\node[draw=black, rectangle split,  rectangle split parts=3] (sn0x1c23690){
\footnotesize{25}
\nodepart{two}
\begin{tikzpicture}[scale=.2]
\node[circle, scale=0.75, fill] (tid0) at (3,1.5){};
\node[circle, scale=0.75, fill] (tid1) at (1.5,3){};
\node[circle, scale=0.75, fill, task_scheduled] (tid4) at (0.75,4.5){};
\node[circle, scale=0.75, fill] (tid5) at (2.25,4.5){};
\draw[](tid1) -- (tid4);
\draw[](tid1) -- (tid5);
\node[circle, scale=0.75, fill] (tid2) at (3.75,3){};
\node[circle, scale=0.75, fill, task_scheduled] (tid6) at (3.75,4.5){};
\draw[](tid2) -- (tid6);
\node[circle, scale=0.75, fill] (tid3) at (5.25,3){};
\node[circle, scale=0.75, fill] (tid7) at (5.25,4.5){};
\draw[](tid3) -- (tid7);
\draw[](tid0) -- (tid1);
\draw[](tid0) -- (tid2);
\draw[](tid0) -- (tid3);
\end{tikzpicture}
\nodepart{three}
\footnotesize{$50\:50$}
};
 \\ 
\\
};
\end{scope}
\begin{scope}[yshift=\leveltopIIII cm, anchor = center]
\matrix (line4)[row sep=0.5cm] {
\node[draw=black, rectangle split,  rectangle split parts=3] (sn0x1c23270){
\footnotesize{75}
\nodepart{two}
\begin{tikzpicture}[scale=.2]
\node[circle, scale=0.75, fill] (tid0) at (3,1.5){};
\node[circle, scale=0.75, fill] (tid1) at (1.5,3){};
\node[circle, scale=0.75, fill, task_scheduled] (tid4) at (0.75,4.5){};
\node[circle, scale=0.75, fill, task_scheduled] (tid5) at (2.25,4.5){};
\draw[](tid1) -- (tid4);
\draw[](tid1) -- (tid5);
\node[circle, scale=0.75, fill] (tid2) at (3.75,3){};
\node[circle, scale=0.75, fill] (tid6) at (3.75,4.5){};
\draw[](tid2) -- (tid6);
\node[circle, scale=0.75, fill] (tid3) at (5.25,3){};
\draw[](tid0) -- (tid1);
\draw[](tid0) -- (tid2);
\draw[](tid0) -- (tid3);
\end{tikzpicture}
\nodepart{three}
\footnotesize{$1$}
};
 \\ 
\node[draw=black, rectangle split,  rectangle split parts=3] (sn0x1c28250){
\footnotesize{12.5}
\nodepart{two}
\begin{tikzpicture}[scale=.2]
\node[circle, scale=0.75, fill] (tid0) at (3,1.5){};
\node[circle, scale=0.75, fill] (tid1) at (1.5,3){};
\node[circle, scale=0.75, fill, task_scheduled] (tid4) at (0.75,4.5){};
\node[circle, scale=0.75, fill] (tid5) at (2.25,4.5){};
\draw[](tid1) -- (tid4);
\draw[](tid1) -- (tid5);
\node[circle, scale=0.75, fill] (tid2) at (3.75,3){};
\node[circle, scale=0.75, fill, task_scheduled] (tid6) at (3.75,4.5){};
\draw[](tid2) -- (tid6);
\node[circle, scale=0.75, fill] (tid3) at (5.25,3){};
\draw[](tid0) -- (tid1);
\draw[](tid0) -- (tid2);
\draw[](tid0) -- (tid3);
\end{tikzpicture}
\nodepart{three}
\footnotesize{$50\:50$}
};
 \\ 
\node[draw=black, rectangle split,  rectangle split parts=3] (sn0x1c23860){
\footnotesize{12.5}
\nodepart{two}
\begin{tikzpicture}[scale=.2]
\node[circle, scale=0.75, fill] (tid0) at (2.25,1.5){};
\node[circle, scale=0.75, fill] (tid1) at (0.75,3){};
\node[circle, scale=0.75, fill, task_scheduled] (tid4) at (0.75,4.5){};
\draw[](tid1) -- (tid4);
\node[circle, scale=0.75, fill] (tid2) at (2.25,3){};
\node[circle, scale=0.75, fill, task_scheduled] (tid5) at (2.25,4.5){};
\draw[](tid2) -- (tid5);
\node[circle, scale=0.75, fill] (tid3) at (3.75,3){};
\node[circle, scale=0.75, fill] (tid6) at (3.75,4.5){};
\draw[](tid3) -- (tid6);
\draw[](tid0) -- (tid1);
\draw[](tid0) -- (tid2);
\draw[](tid0) -- (tid3);
\end{tikzpicture}
\nodepart{three}
\footnotesize{$1$}
};
 \\ 
\\
};
\end{scope}
\begin{scope}[yshift=\leveltopIIIII cm, anchor = center]
\matrix (line5)[row sep=0.5cm] {
\node[draw=black, rectangle split,  rectangle split parts=3] (sn0x1c285b0){
\footnotesize{6.25}
\nodepart{two}
\begin{tikzpicture}[scale=.2]
\node[circle, scale=0.75, fill] (tid0) at (3,1.5){};
\node[circle, scale=0.75, fill] (tid1) at (1.5,3){};
\node[circle, scale=0.75, fill, task_scheduled] (tid4) at (0.75,4.5){};
\node[circle, scale=0.75, fill, task_scheduled] (tid5) at (2.25,4.5){};
\draw[](tid1) -- (tid4);
\draw[](tid1) -- (tid5);
\node[circle, scale=0.75, fill] (tid2) at (3.75,3){};
\node[circle, scale=0.75, fill] (tid3) at (5.25,3){};
\draw[](tid0) -- (tid1);
\draw[](tid0) -- (tid2);
\draw[](tid0) -- (tid3);
\end{tikzpicture}
\nodepart{three}
\footnotesize{$1$}
};
 \\ 
\node[draw=black, rectangle split,  rectangle split parts=3] (sn0x1c23d40){
\footnotesize{93.75}
\nodepart{two}
\begin{tikzpicture}[scale=.2]
\node[circle, scale=0.75, fill] (tid0) at (2.25,1.5){};
\node[circle, scale=0.75, fill] (tid1) at (0.75,3){};
\node[circle, scale=0.75, fill, task_scheduled] (tid4) at (0.75,4.5){};
\draw[](tid1) -- (tid4);
\node[circle, scale=0.75, fill] (tid2) at (2.25,3){};
\node[circle, scale=0.75, fill, task_scheduled] (tid5) at (2.25,4.5){};
\draw[](tid2) -- (tid5);
\node[circle, scale=0.75, fill] (tid3) at (3.75,3){};
\draw[](tid0) -- (tid1);
\draw[](tid0) -- (tid2);
\draw[](tid0) -- (tid3);
\end{tikzpicture}
\nodepart{three}
\footnotesize{$1$}
};
 \\ 
\\
};
\end{scope}
\begin{scope}[yshift=\leveltopIIIIII cm, anchor = center]
\matrix (line6)[row sep=0.5cm] {
\node[draw=black, rectangle split,  rectangle split parts=3] (sn0x1c23fc0){
\footnotesize{100}
\nodepart{two}
\begin{tikzpicture}[scale=.2]
\node[circle, scale=0.75, fill] (tid0) at (2.25,1.5){};
\node[circle, scale=0.75, fill] (tid1) at (0.75,3){};
\node[circle, scale=0.75, fill, task_scheduled] (tid4) at (0.75,4.5){};
\draw[](tid1) -- (tid4);
\node[circle, scale=0.75, fill, task_scheduled] (tid2) at (2.25,3){};
\node[circle, scale=0.75, fill] (tid3) at (3.75,3){};
\draw[](tid0) -- (tid1);
\draw[](tid0) -- (tid2);
\draw[](tid0) -- (tid3);
\end{tikzpicture}
\nodepart{three}
\footnotesize{$50\:50$}
};
 \\ 
\\
};
\end{scope}
\begin{scope}[yshift=\leveltopIIIIIII cm, anchor = center]
\matrix (line7)[row sep=0.5cm] {
\node[draw=black, rectangle split,  rectangle split parts=3] (sn0x1c240d0){
\footnotesize{50}
\nodepart{two}
\begin{tikzpicture}[scale=.2]
\node[circle, scale=0.75, fill] (tid0) at (1.5,1.5){};
\node[circle, scale=0.75, fill] (tid1) at (0.75,3){};
\node[circle, scale=0.75, fill, task_scheduled] (tid3) at (0.75,4.5){};
\draw[](tid1) -- (tid3);
\node[circle, scale=0.75, fill, task_scheduled] (tid2) at (2.25,3){};
\draw[](tid0) -- (tid1);
\draw[](tid0) -- (tid2);
\end{tikzpicture}
\nodepart{three}
\footnotesize{$50\:50$}
};
 \\ 
\node[draw=black, rectangle split,  rectangle split parts=3] (sn0x1c241e0){
\footnotesize{50}
\nodepart{two}
\begin{tikzpicture}[scale=.2]
\node[circle, scale=0.75, fill] (tid0) at (2.25,1.5){};
\node[circle, scale=0.75, fill, task_scheduled] (tid1) at (0.75,3){};
\node[circle, scale=0.75, fill, task_scheduled] (tid2) at (2.25,3){};
\node[circle, scale=0.75, fill] (tid3) at (3.75,3){};
\draw[](tid0) -- (tid1);
\draw[](tid0) -- (tid2);
\draw[](tid0) -- (tid3);
\end{tikzpicture}
\nodepart{three}
\footnotesize{$1$}
};
 \\ 
\\
};
\end{scope}
\begin{scope}[yshift=\leveltopIIIIIIII cm, anchor = center]
\matrix (line8)[row sep=0.5cm] {
\node[draw=black, rectangle split,  rectangle split parts=3] (sn0x1c242f0){
\footnotesize{25}
\nodepart{two}
\begin{tikzpicture}[scale=.2]
\node[circle, scale=0.75, fill] (tid0) at (0.75,1.5){};
\node[circle, scale=0.75, fill] (tid1) at (0.75,3){};
\node[circle, scale=0.75, fill, task_scheduled] (tid2) at (0.75,4.5){};
\draw[](tid1) -- (tid2);
\draw[](tid0) -- (tid1);
\end{tikzpicture}
\nodepart{three}
\footnotesize{$1$}
};
 \\ 
\node[draw=black, rectangle split,  rectangle split parts=3] (sn0x1c245b0){
\footnotesize{75}
\nodepart{two}
\begin{tikzpicture}[scale=.2]
\node[circle, scale=0.75, fill] (tid0) at (1.5,1.5){};
\node[circle, scale=0.75, fill, task_scheduled] (tid1) at (0.75,3){};
\node[circle, scale=0.75, fill, task_scheduled] (tid2) at (2.25,3){};
\draw[](tid0) -- (tid1);
\draw[](tid0) -- (tid2);
\end{tikzpicture}
\nodepart{three}
\footnotesize{$1$}
};
 \\ 
\\
};
\end{scope}
\draw (sn0x1c22990.east) -- (sn0x1c28c10.west);
\draw (sn0x1c22990.east) -- (sn0x1c29d30.west);
\draw (sn0x1c29d30.east) -- (sn0x1c29fa0.west);
\draw (sn0x1c29d30.east) -- (sn0x1c29050.west);
\draw (sn0x1c28c10.east) -- (sn0x1c23690.west);
\draw (sn0x1c28c10.east) -- (sn0x1c29050.west);
\draw (sn0x1c29050.east) -- (sn0x1c23270.west);
\draw (sn0x1c29fa0.east) -- (sn0x1c28250.west);
\draw (sn0x1c29fa0.east) -- (sn0x1c23270.west);
\draw (sn0x1c23690.east) -- (sn0x1c23860.west);
\draw (sn0x1c23690.east) -- (sn0x1c23270.west);
\draw (sn0x1c23270.east) -- (sn0x1c23d40.west);
\draw (sn0x1c28250.east) -- (sn0x1c23d40.west);
\draw (sn0x1c28250.east) -- (sn0x1c285b0.west);
\draw (sn0x1c23860.east) -- (sn0x1c23d40.west);
\draw (sn0x1c285b0.east) -- (sn0x1c23fc0.west);
\draw (sn0x1c23d40.east) -- (sn0x1c23fc0.west);
\draw (sn0x1c23fc0.east) -- (sn0x1c240d0.west);
\draw (sn0x1c23fc0.east) -- (sn0x1c241e0.west);
\draw (sn0x1c240d0.east) -- (sn0x1c242f0.west);
\draw (sn0x1c240d0.east) -- (sn0x1c245b0.west);
\draw (sn0x1c241e0.east) -- (sn0x1c245b0.west);
\end{tikzpicture}
%% profile
\begin{tikzpicture}[scale=.2, anchor=south, rotate=90]
\begin{scope}[yshift=\leveltopI cm, anchor = center]
\matrix (line1)[row sep=0.1cm] {
\node[draw=black, rectangle split,  rectangle split parts=3] (sn0x1c22990){
\footnotesize{100}
\nodepart{two}
\begin{tikzpicture}[scale=.2]
\node[rectangle, scale=0.75] at (0, 0) {$\profile{6, 3, 1}$};
\end{tikzpicture}
\nodepart{three}
\footnotesize{$1$}
};
 \\ 
\\
};
\end{scope}
\begin{scope}[yshift=\leveltopII cm, anchor = center]
\matrix (line2)[row sep=0.1cm] {
\node[draw=black, rectangle split,  rectangle split parts=3] (sn0x1c28c10){
\footnotesize{100}
\nodepart{two}
\begin{tikzpicture}[scale=.2]
\node[rectangle, scale=0.75] at (0, 0) {$\profile{5, 3, 1}$};
\end{tikzpicture}
\nodepart{three}
\footnotesize{$1$}
};
 \\ 
\\
};
\end{scope}
\begin{scope}[yshift=\leveltopIII cm, anchor = center]
\matrix (line3)[row sep=0.1cm] {
\node[draw=black, rectangle split,  rectangle split parts=3] (sn0x1c29fa0){
\footnotesize{100}
\nodepart{two}
\begin{tikzpicture}[scale=.2]
\node[rectangle, scale=0.75] at (0, 0) {$\profile{4, 3, 1}$};
\end{tikzpicture}
\nodepart{three}
\footnotesize{$1$}
};
 \\ 
\\
};
\end{scope}
\begin{scope}[yshift=\leveltopIIII cm, anchor = center]
\matrix (line4)[row sep=0.1cm] {
\node[draw=black, rectangle split,  rectangle split parts=3] (sn0x1c28250){
\footnotesize{100}
\nodepart{two}
\begin{tikzpicture}[scale=.2]
\node[rectangle, scale=0.75] at (0, 0) {$\profile{3, 3, 1}$};
\end{tikzpicture}
\nodepart{three}
\footnotesize{$1$}
};
 \\ 
\\
};
\end{scope}
\begin{scope}[yshift=\leveltopIIIII cm, anchor = center]
\matrix (line5)[row sep=0.1cm] {
\node[draw=black, rectangle split,  rectangle split parts=3] (sn0x1c23d40){
\footnotesize{100}
\nodepart{two}
\begin{tikzpicture}[scale=.2]
\node[rectangle, scale=0.75] at (0, 0) {$\profile{2, 3, 1}$};
\end{tikzpicture}
\nodepart{three}
\footnotesize{$1$}
};
 \\ 
\\
};
\end{scope}
\begin{scope}[yshift=\leveltopIIIIII cm, anchor = center]
\matrix (line6)[row sep=0.1cm] {
\node[draw=black, rectangle split,  rectangle split parts=3] (sn0x1c23fc0){
\footnotesize{100}
\nodepart{two}
\begin{tikzpicture}[scale=.2]
\node[rectangle, scale=0.75] at (0, 0) {$\profile{1, 3, 1}$};
\end{tikzpicture}
\nodepart{three}
\footnotesize{$50\:50$}
};
 \\ 
\\
};
\end{scope}
\begin{scope}[yshift=\leveltopIIIIIII cm, anchor = center]
\matrix (line7)[row sep=0.1cm] {
\node[draw=black, rectangle split,  rectangle split parts=3] (sn0x1c240d0){
\footnotesize{50}
\nodepart{two}
\begin{tikzpicture}[scale=.2]
\node[rectangle, scale=0.75] at (0, 0) {$\profile{1, 2, 1}$};
\end{tikzpicture}
\nodepart{three}
\footnotesize{$50\:50$}
};
 \\ 
\node[draw=black, rectangle split,  rectangle split parts=3] (sn0x1c241e0){
\footnotesize{50}
\nodepart{two}
\begin{tikzpicture}[scale=.2]
\node[rectangle, scale=0.75] at (0, 0) {$\profile{3, 1}$};
\end{tikzpicture}
\nodepart{three}
\footnotesize{$1$}
};
 \\ 
\\
};
\end{scope}
\begin{scope}[yshift=\leveltopIIIIIIII cm, anchor = center]
\matrix (line8)[row sep=0.1cm] {
\node[draw=black, rectangle split,  rectangle split parts=3] (sn0x1c242f0){
\footnotesize{25}
\nodepart{two}
\begin{tikzpicture}[scale=.2]
\node[rectangle, scale=0.75] at (0, 0) {$\profile{1, 1, 1}$};
\end{tikzpicture}
\nodepart{three}
\footnotesize{$1$}
};
 \\ 
\node[draw=black, rectangle split,  rectangle split parts=3] (sn0x1c245b0){
\footnotesize{75}
\nodepart{two}
\begin{tikzpicture}[scale=.2]
\node[rectangle, scale=0.75] at (0, 0) {$\profile{2, 1}$};
\end{tikzpicture}
\nodepart{three}
\footnotesize{$1$}
};
 \\ 
\\
};
\end{scope}
\draw (sn0x1c22990.east) -- (sn0x1c28c10.west);
\draw (sn0x1c28c10.east) -- (sn0x1c29fa0.west);
\draw (sn0x1c29fa0.east) -- (sn0x1c28250.west);
\draw (sn0x1c28250.east) -- (sn0x1c23d40.west);
\draw (sn0x1c23d40.east) -- (sn0x1c23fc0.west);
\draw (sn0x1c23fc0.east) -- (sn0x1c240d0.west);
\draw (sn0x1c23fc0.east) -- (sn0x1c241e0.west);
\draw (sn0x1c240d0.east) -- (sn0x1c242f0.west);
\draw (sn0x1c240d0.east) -- (sn0x1c245b0.west);
\draw (sn0x1c241e0.east) -- (sn0x1c245b0.west);
\end{tikzpicture}
%%% Local Variables:
%%% TeX-master: "../thesis.tex"
%%% End: 
  \caption{A snapshot DAG (HLF, two processors) and its corresponding profile DAG containing much less nodes than the original snaphsot DAG. \emph{Remark:} Snapshots with same profiles have the same expected run time and can thus be merged. }
  \label{fig:p2-profile-dag-example-000111223-hlfdet}
\end{figure}

To be more mathematical: If the snapshot DAG is $(V_s, E_s)$, then the profile DAG can be expressed as $(V_p, E_p)$. These are defined as follows:
\begin{eqnarray*}
  V_p &=& \left\{ \text{PROFILE}(v) \mid v \in V_s \right\} \\
  E_p &=& \left\{ \left(\text{PROFILE}(v_1), \text{PROFILE}(v_2) \right) \mid (v_1, v_2)\in E_s\right\}
\end{eqnarray*}

The function $\text{PROFILE}$ takes a snapshot as input and returns the profile corresponding the the snapshot's intree. Note that $\text{PROFILE}$ then is a homomorphism between the snapthot and the profile DAG.\todo{Stimmt das?}

If we compute the expected run time via the profiles (as equation (\ref{eq:p2-profile-optimal-exp-run-time-def-simplified}) suggests), we only need to compute the profiles arising in the profile DAG. We can cache intermediate results, so we do not need to compute the runtime of any profile twice.

Thus, it is an important question how big these profile DAGs can get. To tackle this question, we examine for a profile $P=\profile{n_1,\dots,n_r}$ how many profiles of a certain length exist as successors of this profile.

We inspect the following example: Consider the profile $\profile{4,3,5,1}$. Its successing profiles of length 4 are
\begin{equation*}
  \profile{3, 3, 5, 1},
  \profile{2, 3, 5, 1},
  \profile{1, 3, 5, 1},
  \profile{1, 2, 5, 1},
  \profile{1, 1, 5, 1},
  \profile{1, 1, 4, 1},
  \profile{1, 1, 3, 1},
  \profile{1, 1, 2, 1} \text{ and }
  \profile{1, 1, 1, 1}.
\end{equation*}

We recognize that the first item in the succesing profiles has to be at most 4 (since the \emph{original} first entry was exactly 4). Moreover, the second entry can only be less then 3 (\emph{original} second entry: 3) if the first entry is 1. Similarily, the third entry in a successing profile can only be less then 5 (original third entry: 5) if the first and the second entry are 1.

More general: In a successing profile, the entry at a certain position can only be less than the original entry at this position if all entries \emph{up to that position} are already 1.

\begin{definition}[Successing profiles]
Let $P=\profile{n_1,\dots,n_r}$ be a profile of lenght $r$. The set of successing profiles of length $j$ of original profile $P$ is
\begin{equation*}
  S^p_{j}
  :=
  \left\{ 
    \profile{m_{r-j+1},\dots,m_{r}}
    \mid
    \exists p \in \left\{ r-j+1,\dots,r \right\}.\,
    \bigwedge_{i=1}^{p-1} m_i = 1 \ \wedge m_p < n_p
  \right\}.
  \todo{Beispiel}
\end{equation*}

The set of \emph{all} successing profiles clearly is then
\begin{equation*}
  S^P
  :=
  \bigcup_{j=1}^r S^P_j.
\end{equation*}
\end{definition}

The size of the profile DAG (starting at a certain profile $P$) is then clearly denoted by the size of $S^P$.

\begin{lemma}[Size of profile DAG]
  \label{lem:profile-dags-exact-size}
  Let $P=\profile{n_1,\dots,n_r}$ be a profile. Then, 
  \begin{equation*}
    \left| S^P \right| = 
    \sum _{j=1}^{r-1} j \cdot n_j 
    -0.5r^2 + 1.5 r.
  \end{equation*}
\end{lemma}

\begin{proof}
  If $P=\profile{n_1,\dots,n_r}$ is a profile, then the size of the set $S^P_j$ is
  \begin{equation*}
    1+ \sum_{i=j}^{r-1} (n_i-1).
  \end{equation*}
  
  In a successing profile, the entry at position $p$ can take exactly the values $1, 2, 3, \dots, n_p-1$ (where $n_p$ is th entry at positon $p$ in profile $P$). This yields the term $(n_i-1)$. Moreover, we stated that the entry at position $p$ can only be smaller if \emph{all previous entries} are 1. Thus, we can simply sum um the terms for the different positions, yielding the above sum.

  We now sum up over the different possible \emph{lengths} (ranging from 1 to $r$) of the successor profiles, and afterwards apply well-known summation rules and formulae for triangular numbers:
  \begin{eqnarray*}
    \left| S^P \right| & = & \sum_{j=1}^{r} \left( 1+ \sum_{i=j}^{r-1} (n_i-1) \right) \\
    &=& \sum_{j=1}^{r} 1 + \sum_{j=1}^{r} \left( \sum_{i=j}^{r-1} n_i \right) - \sum_{j=1}^{r}\sum_{i=j}^{r-1} 1 \\
    &=&r - \sum_{j=1}^{r}(r-j) + \sum_{j=1}^{r} \left( \sum_{i=j}^{r-1} n_i \right) \\
    &=&r - \frac{r^2-r}{2} + \sum_{j=2}^{r-1} \left( \sum_{i=j}^{r-1} n_i \right) \\
    &=&\sum _{j=1}^{r} \left( \sum _{i=j}^{r-1}n_{{i}} \right) + -0.5r^2 + 1.5 r
  \end{eqnarray*}

  It remains to be shown that $\sum _{j=1}^{r} \left( \sum _{i=j}^{r-1}n_{{i}} \right) = \sum_{j=1}^{r-1}(j-1)\cdot n_j$. This can be seen by considering the following:
  \begin{eqnarray*}
    \setlength{\arraycolsep}{2pt}
    \begin{array}{cccccccccccccccc}
      \displaystyle
      \sum_{j=2}^{r-1} \left( \sum _{i=j}^{r-1}n_{{i}} \right) &=&
         n_1 & + & n_2 & + & n_3 & + & n_4 & + & \dots & + & n_{r-2} & + & n_{r-1}  & + \\
      & &    &   & n_2 & + & n_3 & + & n_4 & + & \dots & + & n_{r-2} & + & n_{r-1}  & + \\
      & &    &   &     &   & n_3 & + & n_4 & + & \dots & + & n_{r-2} & + & n_{r-1}  & + \\
      & &    &   &     &   &     & + & n_4 & + & \dots & + & n_{r-2} & + & n_{r-1}  & + \\
      & &    &   &     &   &     &  &  &  & \ddots &  &  &  & \vdots  & + \\
      & &    &   &     &   &     &   &     &   &   &   &  n_{r-2}  & +  & n_{r-1}  & +  \\
      & &    &   &     &   &     &   &     &   &   &   &    &   & n_{r-1}  &   \\
      \\
      &=& n_1 &+& 2 n_2 & + & 3 n_3 &+& 4 n_4 &+& \dots & + & (r-2)\cdot n_{r-2} &+& (r-1)\cdot n_{r-1} & \\
    \end{array}
  \end{eqnarray*}
  This shows that $\left|S^P\right| = \sum _{j=1}^{r-1} j \cdot n_j 
    -0.5r^2 + 1.5 r$.
\end{proof}

Now that we know the size of the profile DAG, this imposes the question how many nodes a profile DAG has \emph{in the worst case} if we consider an intree with exactly $n$ tasks.

\begin{lemma}
  \label{lem:profile-dags-form-of-maximum-profile}
  For an intree with $n$ nodes and $r$ levels (i.e. an intree having a profile with $r$ entries), the maximum size of the profile DAG is reached if the profile is of the form $\profile{\profileones{r-2},n-r+1,1}$.
\end{lemma}

\begin{proof}
  We compute the number of nodes in the profile DAG for the profile $P^*= \profile{\profileones{r-2},n-r+1,1}$. According to lemma  \ref{lem:profile-dags-exact-size}, it is exactly
  \begin{equation*}
    \sum_{j=1}^{r-2} j\cdot 1 + (r-1)\cdot(n-r+1) = \frac{(r-1)\cdot(r-2)}{2} + (r-1)\cdot(n-r+1).
  \end{equation*}
  We consider now an arbitrary profile, but express it in terms of $P^*$. That is, we consider a profile $P=\profile{n_1,\dots,n_r}$, where $n_i = 1 + b_i$, for $i\in\left\{1,2,\dots,r-2\right\}$ and $n_{r-1} = (n-r+1) + b_{r-1}$. Note that this means that each $n_i$ (entry in $P$) is expressed as the corresponding entry in $P^*$ plus some constant $b_i$ chosen appropriately. Of course $n_r = 1$.

  Two observations are important now:

  \begin{itemize}
  \item Since all $n_i \in \naturals$, we can condlude that $b_i \geq 0$ for $i\in\{1,2,\dots,r-2\}$.
  \item Since the number of all tasks has to be the same, we know that $P^*$ and $P$ must have the same number of tasks (namely $n$). This means that
    \begin{equation*}
      n = \sum_{i=1}^r n_i = \sum_{i=1}^{r-2} \left[ 1+b_i \right] + \left[(n-r+1)+b_{r-1} \right] + 1.
    \end{equation*}
    We simplify the above to
    \begin{equation*}
      \sum_{i=1}^{r-2} \left[ 1+b_i \right] + \left[(n-r+1)+b_{r-1} \right] + 1 =
      \sum_{i=1}^{r-2} b_i + r-2 + n-r+1+b_{r-1}+1 = 
      n + \sum_{i=1}^{r-1} b_i
    \end{equation*}
    From this we can conclude that $b_{r-1} = -(\sum_{i=1}^{r-2})$.
  \end{itemize}

  We can now compute the number of nodes for profile $P$ (remember: an arbitrary profile expressed through $P^*$ and the $b_i$'s):

  \begin{eqnarray*}
    \sum_{j=1}^{r-1} j \cdot n_j &=& \sum_{j=1}^{r-2} j\cdot (1+b_j) + (r-1)(n-r+1+b_{r-1}) \\
    &=& \sum_{j=1}^{r-2} j + \sum_{j=1}^{r-2} j\cdot b_j + (r-1)\left(n-r+1-(\sum_{j=1}^{r-2} b_j)\right) \\
    &=& \frac{(r-2)\cdot(r-1)}{2} + (r-1)(n-r+1) + \sum_{j=1}^{r-2} j\cdot b_j - \sum_{j=1}^{r-2} (r-1) \cdot b_j \\
    &=& \underbrace{\frac{(r-2)\cdot(r-1)}{2} + (r-1)(n-r+1)}_{\text{Number of nodes in DAG for $P^*$}} + \sum_{j=1}^{r-2} b_j \cdot (j-r+1) \\
  \end{eqnarray*}
  
  We recognize that the result contains the number of nodes in a profile DAG for profile $P^*=\profile{\profileones{r-2},n-r+1,1}$ --- and some additional term (namely $\sum_{j=1}^{r-2} b_j \cdot (j-r+1)$).

  However, in this term we can see that $(j-r+1) < 0$ (since $j$ ranges from 1 to r-2). This means, that the whole sum $\sum_{j=1}^{r-2} b_j \cdot (j-r+1) \leq 0$ (since all $b_j$ are positive for $j\in\left\{ 1,2,\dots,r-2\right\}$). That means, this sum gets 0 \emph{if and only if} all $b_j$ are 0 for $j\in\left\{ 1,2,\dots,r-2\right\}$.

  This, however, proves that the profile $P^*$ is the profile with exactly $r$ entries and $n$ tasks that maximizes the number of nodes in the profile DAG.
\end{proof}

It remains to explain how to choose $r$ in a way such that given the number of tasks $n$, we can construct a profile with $r$ entries such that the resulting profile DAG has the maximum number of nodes over \emph{all} intrees with $n$ tasks.

\begin{lemma}[Structure of worst-case profile]
  \label{lemma:worst-case-profile-structure}
  Given a natural number $n$, the profile maximizing the number of nodes in the profile DAG is of the form $P^*=\profile{\profileones{r-2},n-r+1,1}$, where $r$ is either $\lfloor n/2\rfloor$ or $\lceil n/2 \rceil$ (one of both --- can be chosen at will).
\end{lemma}

\begin{proof}
  The fact that the profile maximum the number of nodes in the profile DAG is of the form $\profile{\profileones{r-2},n-r+1,1}$ follows directly from lemma \ref{lem:profile-dags-form-of-maximum-profile}. Thus, we can restrict ourselves onto those if we look for profiles maximizing the number of nodes in a profile DAG for a certain number of tasks.
  
  We consider the number of nodes in a profile DAG for a profile of the form $\profile{\profileones{r-2},n-r+1,1}$, given by lemma \ref{lem:profile-dags-exact-size}:
  \begin{equation*}
    \sum_{j=1}^{r-2} j \cdot 1 + (r-1)\cdot(n-r+1) -0.5r^2 +1.5r = \frac{(r-2)\cdot(r-1)}{2} + (r-1)\cdot(n-r+1) -0.5r^2 +1.5r
  \end{equation*}

  Our goal is now to maximize this term by choosing $r$ accordingly depending on $n$. We simplify the above to
  \begin{equation*}
    r+r \left( n-r+1 \right) -1 = -r^2+r\cdot(n+2)-1
  \end{equation*}
  and recognize that this is a downward opened parabola having the derivative
  \begin{equation*}
    3+n-2r,
  \end{equation*}
  meaning that its maximum is at $r=\frac{n+2}{2}$. Since this is only a natural number if $n$ is even and since we have -- as said before -- a parabola, we can simply apply rounding to get the maximum for natural-values. This means, we can derive natural solution: $r^*=\lfloor \frac{n+2}{2} \rfloor$ or $r^*=\lceil \frac{n+2}{2} \rceil$ (can be chosen at will since we have a parabola). Without loss of generality, we focus on $r^*=\lfloor \frac{n+2}{2} \rfloor = \lfloor\frac{n}{2} \rfloor + 1$.

  This means that the profile DAG has a maximum of nodes if we consider the profile
  \begin{equation*}
    \profile{\profileones{r^*-2},n-r^*+1,1} = 
    \profile{
      \profileones{\left( \left\lfloor\frac{n}{2} \right\rfloor + 1 \right)-2},
      n-\left( \left\lfloor\frac{n}{2} \right\rfloor + 1 \right)+1,
      1
    } = 
    \profile{
      \profileones{ \left\lfloor\frac{n}{2} \right\rfloor - 1},
      \left\lceil \frac{n}{2} \right\rceil,
      1
    } 
  \end{equation*}
  Similarily, we could have chosen $r^*=\lceil\frac{n}{2}\rceil+1$. This shows the claim.
\end{proof}

We can now combine the lemmata to the following theorem:

\begin{theorem}[Size of profile DAG]
  \label{the:profile-dags-maximum-size-quadratic}
  For an intree with exactly $n$ tasks, the profile DAG has at most $\lfloor \frac{n}{2} \rfloor \cdot \lceil \frac{n}{2} \rceil +1$ nodes.
\end{theorem}

\begin{proof}
  Lemma \ref{lemma:worst-case-profile-structure} gives us that the wors-case profile is $P^*=\profile{ \profileones{ \left\lfloor\frac{n}{2} \right\rfloor - 1}, \left\lceil \frac{n}{2} \right\rceil, 1 }$, which we use to compute the number of nodes in the corresponding worst-case profile DAG according to lemma \ref{lem:profile-dags-exact-size}:
  \renewcommand{\r}{\left(\left\lfloor\frac{n}{2}\right\rfloor + 1\right)}
  \begin{equation*}
    \frac{(\r-2)\cdot(\r-1)}{2} + 
    \left(\r-1\right)\cdot\left( \left\lceil \frac{n}{2} \right\rceil \right)
    - \frac{1}{2} \cdot \r^2 + \frac{3}{2} \cdot \r     
  \end{equation*}
  We simplify the above to
  \newcommand{\ceiln}{\left\lceil \frac{n}{2} \right\rceil}
  \newcommand{\floorn}{\left\lfloor \frac{n}{2} \right\rfloor}
  \begin{eqnarray*}
    \frac{\left( \floorn -1 \right)\left( \floorn \right)}{2}
    + \floorn \cdot \ceiln
    - \frac{\floorn^2 + 2\cdot \floorn + 1}{2}
    + \frac{3\cdot \floorn + 3}{2}
    & = &
    \ceiln \cdot \floorn + 1,
  \end{eqnarray*}
  proving the claim.
\end{proof}

\section{Snapshot DAG}
\label{sec:p2-snapshot-dag}

Theorem \ref{the:profile-dags-maximum-size-quadratic} shows in particular that the worst-case size of a profile DAG is quadratic in the number of tasks of the intree. Additionally, we can use use the theorem to derive a simple, loose bound on the size of the original snapshot DAG resulting from a HLF schedule for two processors.

\begin{corollary}[Upper bound for the size of the snapshot DAG]
  The size of a snapshot DAG for an intree containing $n$ tasks is $O(n^4)$.
\end{corollary}

\begin{proof}
  Each profile contains at most $n$ tasks (actually, \emph{all but on} profile contain less than $n$ tasks).

  We recognize that there are less than $n^2$ snapshots resulting in the same profile $P$ containing $n$ tasks. This is clearly the case, because, if two processors are present, there are at most $\binom{n}{2} < n^2$ possibilities to choose the two tasks to be scheduled out of at most $n$ tasks.

  \newcommand{\ceiln}{\left\lceil \frac{n}{2} \right\rceil}
  \newcommand{\floorn}{\left\lfloor \frac{n}{2} \right\rfloor}
  We now consider a worst case snapshot DAG, having -- according to theorem \ref{the:profile-dags-maximum-size-quadratic} -- size $\floorn \cdot \ceiln +1\leq \frac{n^2}{4}$. Since each of these profiles corresponds to less than $n^2$ snapshots, it directly follows that the number of nodes in the snapshot DAG is less than $\frac{n^2}{4}\cdot n^2=\frac{n^4}{4}\in O(n^4)$.
\end{proof}

The above bound is far from being tight. Still, it suffices to show that the size of a snapshot DAG can \emph{not} be exponential in the number of nodes.

This, on the other hand, implies a poly-time algorithm that can be used to determine the expected run time for a given intree of tasks whose run times are exponentially distributed: We simply construct the whole snapshot DAG according to HLF (whose size is $O(n^4)$, i.e. polynomial) and compute the expected run time recursively for each snapshot.

%%% Local Variables:
%%% TeX-master: "../thesis.tex"
%%% End: 

%\part{The three processor case}
\chapter{Suboptimal strategies for P3}
\label{chap:p3-suboptimal}

In this section, we take a look at some strategies and use them for scheduling with three processors. We will do so very exemplarily and inspect counterexamples for the respective strategies.

\todo{Introductory text}

\section{HLF}
\label{sec:hlf-p3-suboptimal}

This section showcases some situations, where HLF is not optimal for three processors. We will consider several phenomena that can occur if we use HLF with three processors.

\subsection{HLF does not behave the same for intrees with same profile}
\label{sec:p3-suboptimal-hlf-same-profiles-different-run-times}

In the two-processor case it is known that trees with the same level profile (see section \ref{sec:p2-profiles}) have the same run time. This is not the case for three processors \todo{Erklären, warum der Beweis nicht mehr hinhaut!} Figure \ref{fig:hlf-001112} shows an intree, where HLF can choose at some points, and different choices result in different runtimes.

\begin{figure}[ht]
  \centering
  \begin{subfigure}{.45\linewidth}
    \centering
    \includegraphics{p3/hlf_not_optimal/001112_hlf_subopt.pdf}
    \caption{Suboptimal HLF run}
  \end{subfigure}
  \begin{subfigure}{.45\linewidth}
    \centering
    \includegraphics{p3/hlf_not_optimal/001112_hlf_opt.pdf}
    \caption{Optimal HLF run. This is also the overal optimal schedule.}
    \label{fig:hlf-001112-optimal-version}
  \end{subfigure}
  \caption{HLF on $(0,0,1,1,1,2)$. Different runs of HLF do not necessarily produce the same result.}
  \label{fig:hlf-001112}
\end{figure}

Because HLF can produce different run times depending which task it has chosen, it is clear that HLF in its raw form can not be optimal. The following section reveals even more.

\subsection{Examples where HLF is strictly suboptimal}
\label{sec:p3-suboptimal-hlf-strictly-suboptimal}

The example from figure \ref{fig:hlf-001112-optimal-version} shows the optimal run. We observe that this run is a specific instance of HLF, because at each point of time, always tasks with the highest level numbers are chosen.

However, there are intrees, where \emph{no} HLF-run is optimal. Figures \ref{fig:hlf-vs-opt-0012346688}, \ref{fig:hlf-vs-opt-0012446788} and \ref{fig:hlf-vs-opt-00123455799} show some examples for which this is exactly the case.

\todo{Explizit auf ambiguity bei HLF hinweisen, und auch dass es klassen von Graphen gibt, die keine HLF-Ambiguity zulassen.}

\begin{figure}[ht]
  \centering
  \begin{subfigure}{.45\linewidth}
    \centering
    \includegraphics{p3/hlf_not_optimal/0012346688_subopt.pdf}
    \caption{HLF -- suboptimal}
  \end{subfigure}
  \begin{subfigure}{.45\linewidth}
    \centering
    \includegraphics{p3/hlf_not_optimal/0012346688_opt.pdf}
    \caption{Optimal run is non-HLF}
  \end{subfigure}
  \caption{HLF vs. optimal solution for $(0,0,1,2,3,4,6,6,8,8)$}
  \label{fig:hlf-vs-opt-0012346688}
\end{figure}

\begin{figure}[ht]
  \centering
  \begin{subfigure}{.45\linewidth}
    \centering
    \includegraphics{p3/hlf_not_optimal/0012446788_subopt.pdf}
    \caption{HLF -- suboptimal}
  \end{subfigure}
  \begin{subfigure}{.45\linewidth}
    \centering
    \includegraphics{p3/hlf_not_optimal/0012446788_opt.pdf}
    \caption{Optimal run is non-HLF}
  \end{subfigure}
  \caption{HLF vs. optimal solution for $(0,0,1,2,4,4,6,7,8,8)$ (taken from Ernst Mayr)}
  \label{fig:hlf-vs-opt-0012446788}
\end{figure}
\begin{figure}[ht]
  \centering
  \begin{subfigure}{.45\linewidth}
    \centering
    \includegraphics{p3/hlf_not_optimal/00123455799_subopt.pdf}
    \caption{HLF -- suboptimal}
  \end{subfigure}
  \begin{subfigure}{.45\linewidth}
    \centering
    \includegraphics{p3/hlf_not_optimal/00123455799_opt.pdf}
    \caption{Optimal run is non-HLF}
  \end{subfigure}
  \caption{HLF vs. optimal solution for $(0,0,1,2,3,4,5,5,7,9,9)$ (taken from Chandy/Reynolds)}
  \label{fig:hlf-vs-opt-00123455799}
\end{figure}

\clearpage{}

\section{Maximizing 3-processor-time, minimizing 1-processor time}
\label{sec:p3-disproving-long-p3-and-short-p1-time}

If we have three processors in total, we can split the total run time into three parts: The time where all three processors are processing tasks, the time where one processor is idle and two are working, and the time where only one processor is working.

%We first define some variants of run time. We consider the overall run time and the time where -- within a schedule of an intree -- exactly $p$ processors are working (i.e. where exactly $p$ tasks are scheduled).

\begin{definition}[Run time and its variants]
  We denote by $T$ the expected run time for a schedule associated with an intree. 

  Moreover, we call the time where exactly $p$ taks are scheduled by $T_p$.
\end{definition}

Note that $T$ actually describes an \emph{expected value}. Because of the linearity of expectation, we have that -- for three processors -- $T=T_1 + T_2 + T_3$. If we want to construct an optimal schedule for three processors, we might be tempted to think that (at least) one of the two following criteria should be fulfilled for the optimal schedule:

\begin{description}
\item[P3L] For the optimal schedule, $T_3$ should be maximal (over all schedules), i.e. we should exploit three processors as long as possible (in the expectation).
\item[P1S] For the optimal schedule, $T_1$ should be minimal (over all schedules), i.e. we should try to keep the expected time for which only one processor is working as short as possible.
\end{description}

Surprisingly, \emph{both} of them are wrong (at least if considered separately).

\subsection{Maximizing $T_3$}
\label{sec:p3-disproving-long-p3}

Figure \ref{fig:p3-p3l-suboptimal-example} shows an example, where the optimal schedule keeps three processors busy for expected 0.77777 time steps, while a suboptimal schedule keeps three processors busy for a longer expected time, namely about 0.851852 time steps.

From this we can conclude that it may be advantageous in some cases to accept a shorter time with three busy processors, thereby possibly also decreasing the time where only one processor is busy.

\begin{figure}[ht]
  \centering
  \begin{subfigure}{.45\linewidth}
    \centering
    \includegraphics{p3/keep_3_busy/three_busy_opt.pdf}
    \caption{Optimal schedule. Keeps three processors busy for $7/9\approx 0.78$ time steps ($(T_3, T_2, T_1)=(7/9, 31/24, 37/12)$).}
  \end{subfigure}
  \quad
  \begin{subfigure}{.45\linewidth}
    \centering
    \includegraphics{p3/keep_3_busy/three_busy_subopt.pdf}
    \caption{This suboptimal schedule keeps three processors busy for expectedly $0.851852$ time steps ($(T_3, T_2, T_1)=(23/27,10/9,29/9)$).}
  \end{subfigure}
  \caption{An intree that shows that an optimal P3 schedule needs not keep busy three processors as long as possible. Snapshots with fewer than 6 tasks omitted since they have at most two tasks to be schedlued can be (optimally) processed via ordinary HLF. \todo{See figure \ref{fig:p3-p1s-suboptimal-example}!}}
  \label{fig:p3-p3l-suboptimal-example}
\end{figure}

\subsection{Minimizing $T_1$}
\label{sec:p3-disproving-short-p1}

The ``other direction'', i.e. minimizing the time where only one processor is busy, still is suboptimal.
Figure \ref{fig:p3-p1s-suboptimal-example} shows an intree with the property that the optimal schedule has an expected timespan of roughly 2.59259, within which only one processor is busy. On the other hand, a suboptimal schedule has a timespan of roughly 2.55555 within which only one processor is busy.

\begin{figure}[ht]
  \centering
  \begin{subfigure}{.45\linewidth}
    \centering
    \includegraphics{p3/keep_1_unbusy/one_unbusy_opt.pdf}
    \caption{Optimal schedule. For expectedly $70/27\approx 2.59$ time steps, only one processor is busy $(T_3, T_2, T_1)=(23/27, 25/27, 70/27)$.}
  \end{subfigure}
  \quad
  \begin{subfigure}{.45\linewidth}
    \centering
    \includegraphics{p3/keep_1_unbusy/one_unbusy_subopt.pdf}
    \caption{This suboptimal schedule has an approximated timespan of $23/9\approx 2.55$ time steps, where only one processor is working ($(T_3, T_2, T_1)=(7/9,19/18,23/9)$).}
  \end{subfigure}
  \caption{An intree where the expected time with only one processor being busy is longer within the optimal schedule ($\approx 2.59259$) than within a suboptimal schedule ($\approx 2.555555$).\todo{Write expected times which only one busy processor in subfigures.} \todo{Convert to fractions.}\todo{Which one is optimal? Subfigures!}}
  \label{fig:p3-p1s-suboptimal-example}
\end{figure}

This shows that it can be useful to accept a longer time with only one processor busy, probably acchieving a longer time span where three processors are busy.

\subsection{Maximizing $T_3$ \emph{or} minimizing $T_1$}
\label{sec:p3-suboptimality-maximizing-t3-and-minimizing-t1}

It can also be shown that even combining the two arguments -- in the sense that P3L \emph{or} P1S should be fulfilled for the optimal schedule -- is not correct. This can be observed by examining the intree $(0, 0, 1, 1, 2, 3, 3, 3)$. Figure \ref{fig:p3l-p1s-combo-suboptimal} shows this example.

\begin{figure}[ht]
  \centering
  \renewcommand{\leveltopI}{-10cm + \leveltop}
\renewcommand{\leveltopII}{-10cm + \leveltopI}
\renewcommand{\leveltopIII}{-10cm + \leveltopII}
\renewcommand{\leveltopIIII}{-10cm + \leveltopIII}
\renewcommand{\leveltopIIIII}{-10cm + \leveltopIIII}
\renewcommand{\leveltopIIIIII}{-10cm + \leveltopIIIII}
\renewcommand{\leveltopIIIIIII}{-10cm + \leveltopIIIIII}
\renewcommand{\leveltopIIIIIIII}{-10cm + \leveltopIIIIIII}
\renewcommand{\leveltopIIIIIIIII}{-10cm + \leveltopIIIIIIII}
\begin{tikzpicture}[scale=.2, anchor=south]
\begin{scope}[yshift=\leveltopI cm]
\matrix (line1)[column sep=0.25cm] {
\node[draw=black, rectangle split,  rectangle split parts=2] (sn0x9940640){
\begin{tikzpicture}[scale=.2]
\node[circle, scale=0.75, fill] (tid0) at (3.75,1.5){};
\node[circle, scale=0.75, fill] (tid1) at (3,3){};
\node[circle, scale=0.75, fill] (tid3) at (2.25,4.5){};
\node[circle, scale=0.75, fill, task_scheduled] (tid6) at (0.75,6){};
\node[circle, scale=0.75, fill, task_scheduled] (tid7) at (2.25,6){};
\node[circle, scale=0.75, fill] (tid8) at (3.75,6){};
\draw[](tid3) -- (tid6);
\draw[](tid3) -- (tid7);
\draw[](tid3) -- (tid8);
\node[circle, scale=0.75, fill, task_scheduled] (tid4) at (5.25,4.5){};
\draw[](tid1) -- (tid3);
\draw[](tid1) -- (tid4);
\node[circle, scale=0.75, fill] (tid2) at (6.75,3){};
\node[circle, scale=0.75, fill] (tid5) at (6.75,4.5){};
\draw[](tid2) -- (tid5);
\draw[](tid0) -- (tid1);
\draw[](tid0) -- (tid2);
\end{tikzpicture}
\nodepart{two}
\footnotesize{$33\:67$}
};
 & 
\\
};
\end{scope}
\begin{scope}[yshift=\leveltopII cm]
\matrix (line2)[column sep=0.25cm] {
\node[draw=black, rectangle split,  rectangle split parts=2] (sn0x993e580){
\begin{tikzpicture}[scale=.2]
\node[circle, scale=0.75, fill] (tid0) at (3,1.5){};
\node[circle, scale=0.75, fill] (tid1) at (2.25,3){};
\node[circle, scale=0.75, fill] (tid3) at (2.25,4.5){};
\node[circle, scale=0.75, fill, task_scheduled] (tid5) at (0.75,6){};
\node[circle, scale=0.75, fill, task_scheduled] (tid6) at (2.25,6){};
\node[circle, scale=0.75, fill, task_scheduled] (tid7) at (3.75,6){};
\draw[](tid3) -- (tid5);
\draw[](tid3) -- (tid6);
\draw[](tid3) -- (tid7);
\draw[](tid1) -- (tid3);
\node[circle, scale=0.75, fill] (tid2) at (5.25,3){};
\node[circle, scale=0.75, fill] (tid4) at (5.25,4.5){};
\draw[](tid2) -- (tid4);
\draw[](tid0) -- (tid1);
\draw[](tid0) -- (tid2);
\end{tikzpicture}
\nodepart{two}
\footnotesize{$1$}
};
 & 
\node[draw=black, rectangle split,  rectangle split parts=2] (sn0x9940918){
\begin{tikzpicture}[scale=.2]
\node[circle, scale=0.75, fill] (tid0) at (3,1.5){};
\node[circle, scale=0.75, fill] (tid1) at (2.25,3){};
\node[circle, scale=0.75, fill] (tid3) at (1.5,4.5){};
\node[circle, scale=0.75, fill, task_scheduled] (tid6) at (0.75,6){};
\node[circle, scale=0.75, fill, task_scheduled] (tid7) at (2.25,6){};
\draw[](tid3) -- (tid6);
\draw[](tid3) -- (tid7);
\node[circle, scale=0.75, fill, task_scheduled] (tid4) at (3.75,4.5){};
\draw[](tid1) -- (tid3);
\draw[](tid1) -- (tid4);
\node[circle, scale=0.75, fill] (tid2) at (5.25,3){};
\node[circle, scale=0.75, fill] (tid5) at (5.25,4.5){};
\draw[](tid2) -- (tid5);
\draw[](tid0) -- (tid1);
\draw[](tid0) -- (tid2);
\end{tikzpicture}
\nodepart{two}
\footnotesize{$33\:67$}
};
 & 
\\
};
\end{scope}
\begin{scope}[yshift=\leveltopIII cm]
\matrix (line3)[column sep=0.25cm] {
\node[draw=black, rectangle split,  rectangle split parts=2] (sn0x993e438){
\begin{tikzpicture}[scale=.2]
\node[circle, scale=0.75, fill] (tid0) at (2.25,1.5){};
\node[circle, scale=0.75, fill] (tid1) at (1.5,3){};
\node[circle, scale=0.75, fill] (tid3) at (1.5,4.5){};
\node[circle, scale=0.75, fill, task_scheduled] (tid5) at (0.75,6){};
\node[circle, scale=0.75, fill, task_scheduled] (tid6) at (2.25,6){};
\draw[](tid3) -- (tid5);
\draw[](tid3) -- (tid6);
\draw[](tid1) -- (tid3);
\node[circle, scale=0.75, fill] (tid2) at (3.75,3){};
\node[circle, scale=0.75, fill, task_scheduled] (tid4) at (3.75,4.5){};
\draw[](tid2) -- (tid4);
\draw[](tid0) -- (tid1);
\draw[](tid0) -- (tid2);
\end{tikzpicture}
\nodepart{two}
\footnotesize{$33\:67$}
};
 & 
\node[draw=black, rectangle split,  rectangle split parts=2] (sn0x9940378){
\begin{tikzpicture}[scale=.2]
\node[circle, scale=0.75, fill] (tid0) at (2.25,1.5){};
\node[circle, scale=0.75, fill] (tid1) at (1.5,3){};
\node[circle, scale=0.75, fill] (tid3) at (0.75,4.5){};
\node[circle, scale=0.75, fill, task_scheduled] (tid6) at (0.75,6){};
\draw[](tid3) -- (tid6);
\node[circle, scale=0.75, fill, task_scheduled] (tid4) at (2.25,4.5){};
\draw[](tid1) -- (tid3);
\draw[](tid1) -- (tid4);
\node[circle, scale=0.75, fill] (tid2) at (3.75,3){};
\node[circle, scale=0.75, fill, task_scheduled] (tid5) at (3.75,4.5){};
\draw[](tid2) -- (tid5);
\draw[](tid0) -- (tid1);
\draw[](tid0) -- (tid2);
\end{tikzpicture}
\nodepart{two}
\footnotesize{$33\:33\:33$}
};
 & 
\\
};
\end{scope}
\begin{scope}[yshift=\leveltopIIII cm]
\matrix (line4)[column sep=0.25cm] {
\node[draw=black, rectangle split,  rectangle split parts=2] (sn0x993d988){
\begin{tikzpicture}[scale=.2]
\node[circle, scale=0.75, fill] (tid0) at (2.25,1.5){};
\node[circle, scale=0.75, fill] (tid1) at (1.5,3){};
\node[circle, scale=0.75, fill] (tid3) at (1.5,4.5){};
\node[circle, scale=0.75, fill, task_scheduled] (tid4) at (0.75,6){};
\node[circle, scale=0.75, fill, task_scheduled] (tid5) at (2.25,6){};
\draw[](tid3) -- (tid4);
\draw[](tid3) -- (tid5);
\draw[](tid1) -- (tid3);
\node[circle, scale=0.75, fill, task_scheduled] (tid2) at (3.75,3){};
\draw[](tid0) -- (tid1);
\draw[](tid0) -- (tid2);
\end{tikzpicture}
\nodepart{two}
\footnotesize{$33\:67$}
};
 & 
\node[draw=black, rectangle split,  rectangle split parts=2] (sn0x993f700){
\begin{tikzpicture}[scale=.2]
\node[circle, scale=0.75, fill] (tid0) at (2.25,1.5){};
\node[circle, scale=0.75, fill] (tid1) at (1.5,3){};
\node[circle, scale=0.75, fill] (tid3) at (0.75,4.5){};
\node[circle, scale=0.75, fill, task_scheduled] (tid5) at (0.75,6){};
\draw[](tid3) -- (tid5);
\node[circle, scale=0.75, fill, task_scheduled] (tid4) at (2.25,4.5){};
\draw[](tid1) -- (tid3);
\draw[](tid1) -- (tid4);
\node[circle, scale=0.75, fill, task_scheduled] (tid2) at (3.75,3){};
\draw[](tid0) -- (tid1);
\draw[](tid0) -- (tid2);
\end{tikzpicture}
\nodepart{two}
\footnotesize{$33\:33\:33$}
};
 & 
\node[draw=black, rectangle split,  rectangle split parts=2] (sn0x993e010){
\begin{tikzpicture}[scale=.2]
\node[circle, scale=0.75, fill] (tid0) at (1.5,1.5){};
\node[circle, scale=0.75, fill] (tid1) at (0.75,3){};
\node[circle, scale=0.75, fill] (tid3) at (0.75,4.5){};
\node[circle, scale=0.75, fill, task_scheduled] (tid5) at (0.75,6){};
\draw[](tid3) -- (tid5);
\draw[](tid1) -- (tid3);
\node[circle, scale=0.75, fill] (tid2) at (2.25,3){};
\node[circle, scale=0.75, fill, task_scheduled] (tid4) at (2.25,4.5){};
\draw[](tid2) -- (tid4);
\draw[](tid0) -- (tid1);
\draw[](tid0) -- (tid2);
\end{tikzpicture}
\nodepart{two}
\footnotesize{$50\:50$}
};
 & 
\node[draw=black, rectangle split,  rectangle split parts=2] (sn0x9940050){
\begin{tikzpicture}[scale=.2]
\node[circle, scale=0.75, fill] (tid0) at (2.25,1.5){};
\node[circle, scale=0.75, fill] (tid1) at (1.5,3){};
\node[circle, scale=0.75, fill, task_scheduled] (tid3) at (0.75,4.5){};
\node[circle, scale=0.75, fill, task_scheduled] (tid4) at (2.25,4.5){};
\draw[](tid1) -- (tid3);
\draw[](tid1) -- (tid4);
\node[circle, scale=0.75, fill] (tid2) at (3.75,3){};
\node[circle, scale=0.75, fill, task_scheduled] (tid5) at (3.75,4.5){};
\draw[](tid2) -- (tid5);
\draw[](tid0) -- (tid1);
\draw[](tid0) -- (tid2);
\end{tikzpicture}
\nodepart{two}
\footnotesize{$67\:33$}
};
 & 
\\
};
\end{scope}
\draw (sn0x9940640.south) -- (sn0x993e580.north);
\draw (sn0x9940640.south) -- (sn0x9940918.north);
\draw (sn0x993e580.south) -- (sn0x993e438.north);
\draw (sn0x9940918.south) -- (sn0x993e438.north);
\draw (sn0x9940918.south) -- (sn0x9940378.north);
\draw (sn0x993e438.south) -- (sn0x993d988.north);
\draw (sn0x993e438.south) -- (sn0x993e010.north);
\draw (sn0x9940378.south) -- (sn0x993f700.north);
\draw (sn0x9940378.south) -- (sn0x993e010.north);
\draw (sn0x9940378.south) -- (sn0x9940050.north);
\end{tikzpicture}
\renewcommand{\leveltopI}{-10cm + \leveltop}
\renewcommand{\leveltopII}{-10cm + \leveltopI}
\renewcommand{\leveltopIII}{-10cm + \leveltopII}
\renewcommand{\leveltopIIII}{-10cm + \leveltopIII}
\renewcommand{\leveltopIIIII}{-10cm + \leveltopIIII}
\renewcommand{\leveltopIIIIII}{-10cm + \leveltopIIIII}
\renewcommand{\leveltopIIIIIII}{-10cm + \leveltopIIIIII}
\renewcommand{\leveltopIIIIIIII}{-10cm + \leveltopIIIIIII}
\renewcommand{\leveltopIIIIIIIII}{-10cm + \leveltopIIIIIIII}
\begin{tikzpicture}[scale=.2, anchor=south]
\begin{scope}[yshift=\leveltopI cm]
\matrix (line1)[column sep=0.25cm] {
\node[draw=black, rectangle split,  rectangle split parts=2] (sn0x9941330){
\begin{tikzpicture}[scale=.2]
\node[circle, scale=0.75, fill] (tid0) at (3.75,1.5){};
\node[circle, scale=0.75, fill] (tid1) at (3,3){};
\node[circle, scale=0.75, fill] (tid3) at (2.25,4.5){};
\node[circle, scale=0.75, fill, task_scheduled] (tid6) at (0.75,6){};
\node[circle, scale=0.75, fill] (tid7) at (2.25,6){};
\node[circle, scale=0.75, fill] (tid8) at (3.75,6){};
\draw[](tid3) -- (tid6);
\draw[](tid3) -- (tid7);
\draw[](tid3) -- (tid8);
\node[circle, scale=0.75, fill, task_scheduled] (tid4) at (5.25,4.5){};
\draw[](tid1) -- (tid3);
\draw[](tid1) -- (tid4);
\node[circle, scale=0.75, fill] (tid2) at (6.75,3){};
\node[circle, scale=0.75, fill, task_scheduled] (tid5) at (6.75,4.5){};
\draw[](tid2) -- (tid5);
\draw[](tid0) -- (tid1);
\draw[](tid0) -- (tid2);
\end{tikzpicture}
\nodepart{two}
\footnotesize{$33\:33\:33$}
};
 & 
\\
};
\end{scope}
\begin{scope}[yshift=\leveltopII cm]
\matrix (line2)[column sep=0.25cm] {
\node[draw=black, rectangle split,  rectangle split parts=2] (sn0x99416c8){
\begin{tikzpicture}[scale=.2]
\node[circle, scale=0.75, fill] (tid0) at (3.75,1.5){};
\node[circle, scale=0.75, fill] (tid1) at (3,3){};
\node[circle, scale=0.75, fill] (tid3) at (2.25,4.5){};
\node[circle, scale=0.75, fill, task_scheduled] (tid5) at (0.75,6){};
\node[circle, scale=0.75, fill, task_scheduled] (tid6) at (2.25,6){};
\node[circle, scale=0.75, fill] (tid7) at (3.75,6){};
\draw[](tid3) -- (tid5);
\draw[](tid3) -- (tid6);
\draw[](tid3) -- (tid7);
\node[circle, scale=0.75, fill, task_scheduled] (tid4) at (5.25,4.5){};
\draw[](tid1) -- (tid3);
\draw[](tid1) -- (tid4);
\node[circle, scale=0.75, fill] (tid2) at (6.75,3){};
\draw[](tid0) -- (tid1);
\draw[](tid0) -- (tid2);
\end{tikzpicture}
\nodepart{two}
\footnotesize{$33\:67$}
};
 & 
\node[draw=black, rectangle split,  rectangle split parts=2] (sn0x993e518){
\begin{tikzpicture}[scale=.2]
\node[circle, scale=0.75, fill] (tid0) at (3,1.5){};
\node[circle, scale=0.75, fill] (tid1) at (2.25,3){};
\node[circle, scale=0.75, fill] (tid3) at (2.25,4.5){};
\node[circle, scale=0.75, fill, task_scheduled] (tid5) at (0.75,6){};
\node[circle, scale=0.75, fill, task_scheduled] (tid6) at (2.25,6){};
\node[circle, scale=0.75, fill] (tid7) at (3.75,6){};
\draw[](tid3) -- (tid5);
\draw[](tid3) -- (tid6);
\draw[](tid3) -- (tid7);
\draw[](tid1) -- (tid3);
\node[circle, scale=0.75, fill] (tid2) at (5.25,3){};
\node[circle, scale=0.75, fill, task_scheduled] (tid4) at (5.25,4.5){};
\draw[](tid2) -- (tid4);
\draw[](tid0) -- (tid1);
\draw[](tid0) -- (tid2);
\end{tikzpicture}
\nodepart{two}
\footnotesize{$33\:67$}
};
 & 
\node[draw=black, rectangle split,  rectangle split parts=2] (sn0x99403e0){
\begin{tikzpicture}[scale=.2]
\node[circle, scale=0.75, fill] (tid0) at (3,1.5){};
\node[circle, scale=0.75, fill] (tid1) at (2.25,3){};
\node[circle, scale=0.75, fill] (tid3) at (1.5,4.5){};
\node[circle, scale=0.75, fill, task_scheduled] (tid6) at (0.75,6){};
\node[circle, scale=0.75, fill] (tid7) at (2.25,6){};
\draw[](tid3) -- (tid6);
\draw[](tid3) -- (tid7);
\node[circle, scale=0.75, fill, task_scheduled] (tid4) at (3.75,4.5){};
\draw[](tid1) -- (tid3);
\draw[](tid1) -- (tid4);
\node[circle, scale=0.75, fill] (tid2) at (5.25,3){};
\node[circle, scale=0.75, fill, task_scheduled] (tid5) at (5.25,4.5){};
\draw[](tid2) -- (tid5);
\draw[](tid0) -- (tid1);
\draw[](tid0) -- (tid2);
\end{tikzpicture}
\nodepart{two}
\footnotesize{$33\:33\:33$}
};
 & 
\\
};
\end{scope}
\begin{scope}[yshift=\leveltopIII cm]
\matrix (line3)[column sep=0.25cm] {
\node[draw=black, rectangle split,  rectangle split parts=2] (sn0x993d8b8){
\begin{tikzpicture}[scale=.2]
\node[circle, scale=0.75, fill] (tid0) at (3,1.5){};
\node[circle, scale=0.75, fill] (tid1) at (2.25,3){};
\node[circle, scale=0.75, fill] (tid3) at (2.25,4.5){};
\node[circle, scale=0.75, fill, task_scheduled] (tid4) at (0.75,6){};
\node[circle, scale=0.75, fill, task_scheduled] (tid5) at (2.25,6){};
\node[circle, scale=0.75, fill, task_scheduled] (tid6) at (3.75,6){};
\draw[](tid3) -- (tid4);
\draw[](tid3) -- (tid5);
\draw[](tid3) -- (tid6);
\draw[](tid1) -- (tid3);
\node[circle, scale=0.75, fill] (tid2) at (5.25,3){};
\draw[](tid0) -- (tid1);
\draw[](tid0) -- (tid2);
\end{tikzpicture}
\nodepart{two}
\footnotesize{$1$}
};
 & 
\node[draw=black, rectangle split,  rectangle split parts=2] (sn0x993fb10){
\begin{tikzpicture}[scale=.2]
\node[circle, scale=0.75, fill] (tid0) at (3,1.5){};
\node[circle, scale=0.75, fill] (tid1) at (2.25,3){};
\node[circle, scale=0.75, fill] (tid3) at (1.5,4.5){};
\node[circle, scale=0.75, fill, task_scheduled] (tid5) at (0.75,6){};
\node[circle, scale=0.75, fill, task_scheduled] (tid6) at (2.25,6){};
\draw[](tid3) -- (tid5);
\draw[](tid3) -- (tid6);
\node[circle, scale=0.75, fill, task_scheduled] (tid4) at (3.75,4.5){};
\draw[](tid1) -- (tid3);
\draw[](tid1) -- (tid4);
\node[circle, scale=0.75, fill] (tid2) at (5.25,3){};
\draw[](tid0) -- (tid1);
\draw[](tid0) -- (tid2);
\end{tikzpicture}
\nodepart{two}
\footnotesize{$33\:67$}
};
 & 
\node[draw=black, rectangle split,  rectangle split parts=2] (sn0x993e438){
\begin{tikzpicture}[scale=.2]
\node[circle, scale=0.75, fill] (tid0) at (2.25,1.5){};
\node[circle, scale=0.75, fill] (tid1) at (1.5,3){};
\node[circle, scale=0.75, fill] (tid3) at (1.5,4.5){};
\node[circle, scale=0.75, fill, task_scheduled] (tid5) at (0.75,6){};
\node[circle, scale=0.75, fill, task_scheduled] (tid6) at (2.25,6){};
\draw[](tid3) -- (tid5);
\draw[](tid3) -- (tid6);
\draw[](tid1) -- (tid3);
\node[circle, scale=0.75, fill] (tid2) at (3.75,3){};
\node[circle, scale=0.75, fill, task_scheduled] (tid4) at (3.75,4.5){};
\draw[](tid2) -- (tid4);
\draw[](tid0) -- (tid1);
\draw[](tid0) -- (tid2);
\end{tikzpicture}
\nodepart{two}
\footnotesize{$33\:67$}
};
 & 
\node[draw=black, rectangle split,  rectangle split parts=2] (sn0x9940378){
\begin{tikzpicture}[scale=.2]
\node[circle, scale=0.75, fill] (tid0) at (2.25,1.5){};
\node[circle, scale=0.75, fill] (tid1) at (1.5,3){};
\node[circle, scale=0.75, fill] (tid3) at (0.75,4.5){};
\node[circle, scale=0.75, fill, task_scheduled] (tid6) at (0.75,6){};
\draw[](tid3) -- (tid6);
\node[circle, scale=0.75, fill, task_scheduled] (tid4) at (2.25,4.5){};
\draw[](tid1) -- (tid3);
\draw[](tid1) -- (tid4);
\node[circle, scale=0.75, fill] (tid2) at (3.75,3){};
\node[circle, scale=0.75, fill, task_scheduled] (tid5) at (3.75,4.5){};
\draw[](tid2) -- (tid5);
\draw[](tid0) -- (tid1);
\draw[](tid0) -- (tid2);
\end{tikzpicture}
\nodepart{two}
\footnotesize{$33\:33\:33$}
};
 & 
\\
};
\end{scope}
\begin{scope}[yshift=\leveltopIIII cm]
\matrix (line4)[column sep=0.25cm] {
\node[draw=black, rectangle split,  rectangle split parts=2] (sn0x993d988){
\begin{tikzpicture}[scale=.2]
\node[circle, scale=0.75, fill] (tid0) at (2.25,1.5){};
\node[circle, scale=0.75, fill] (tid1) at (1.5,3){};
\node[circle, scale=0.75, fill] (tid3) at (1.5,4.5){};
\node[circle, scale=0.75, fill, task_scheduled] (tid4) at (0.75,6){};
\node[circle, scale=0.75, fill, task_scheduled] (tid5) at (2.25,6){};
\draw[](tid3) -- (tid4);
\draw[](tid3) -- (tid5);
\draw[](tid1) -- (tid3);
\node[circle, scale=0.75, fill, task_scheduled] (tid2) at (3.75,3){};
\draw[](tid0) -- (tid1);
\draw[](tid0) -- (tid2);
\end{tikzpicture}
\nodepart{two}
\footnotesize{$33\:67$}
};
 & 
\node[draw=black, rectangle split,  rectangle split parts=2] (sn0x993f700){
\begin{tikzpicture}[scale=.2]
\node[circle, scale=0.75, fill] (tid0) at (2.25,1.5){};
\node[circle, scale=0.75, fill] (tid1) at (1.5,3){};
\node[circle, scale=0.75, fill] (tid3) at (0.75,4.5){};
\node[circle, scale=0.75, fill, task_scheduled] (tid5) at (0.75,6){};
\draw[](tid3) -- (tid5);
\node[circle, scale=0.75, fill, task_scheduled] (tid4) at (2.25,4.5){};
\draw[](tid1) -- (tid3);
\draw[](tid1) -- (tid4);
\node[circle, scale=0.75, fill, task_scheduled] (tid2) at (3.75,3){};
\draw[](tid0) -- (tid1);
\draw[](tid0) -- (tid2);
\end{tikzpicture}
\nodepart{two}
\footnotesize{$33\:33\:33$}
};
 & 
\node[draw=black, rectangle split,  rectangle split parts=2] (sn0x993e010){
\begin{tikzpicture}[scale=.2]
\node[circle, scale=0.75, fill] (tid0) at (1.5,1.5){};
\node[circle, scale=0.75, fill] (tid1) at (0.75,3){};
\node[circle, scale=0.75, fill] (tid3) at (0.75,4.5){};
\node[circle, scale=0.75, fill, task_scheduled] (tid5) at (0.75,6){};
\draw[](tid3) -- (tid5);
\draw[](tid1) -- (tid3);
\node[circle, scale=0.75, fill] (tid2) at (2.25,3){};
\node[circle, scale=0.75, fill, task_scheduled] (tid4) at (2.25,4.5){};
\draw[](tid2) -- (tid4);
\draw[](tid0) -- (tid1);
\draw[](tid0) -- (tid2);
\end{tikzpicture}
\nodepart{two}
\footnotesize{$50\:50$}
};
 & 
\node[draw=black, rectangle split,  rectangle split parts=2] (sn0x9940050){
\begin{tikzpicture}[scale=.2]
\node[circle, scale=0.75, fill] (tid0) at (2.25,1.5){};
\node[circle, scale=0.75, fill] (tid1) at (1.5,3){};
\node[circle, scale=0.75, fill, task_scheduled] (tid3) at (0.75,4.5){};
\node[circle, scale=0.75, fill, task_scheduled] (tid4) at (2.25,4.5){};
\draw[](tid1) -- (tid3);
\draw[](tid1) -- (tid4);
\node[circle, scale=0.75, fill] (tid2) at (3.75,3){};
\node[circle, scale=0.75, fill, task_scheduled] (tid5) at (3.75,4.5){};
\draw[](tid2) -- (tid5);
\draw[](tid0) -- (tid1);
\draw[](tid0) -- (tid2);
\end{tikzpicture}
\nodepart{two}
\footnotesize{$33\:67$}
};
 & 
\\
};
\end{scope}
\draw (sn0x9941330.south) -- (sn0x99416c8.north);
\draw (sn0x9941330.south) -- (sn0x993e518.north);
\draw (sn0x9941330.south) -- (sn0x99403e0.north);
\draw (sn0x99416c8.south) -- (sn0x993d8b8.north);
\draw (sn0x99416c8.south) -- (sn0x993fb10.north);
\draw (sn0x993e518.south) -- (sn0x993d8b8.north);
\draw (sn0x993e518.south) -- (sn0x993e438.north);
\draw (sn0x99403e0.south) -- (sn0x993fb10.north);
\draw (sn0x99403e0.south) -- (sn0x993e438.north);
\draw (sn0x99403e0.south) -- (sn0x9940378.north);
\draw (sn0x993d8b8.south) -- (sn0x993d988.north);
\draw (sn0x993fb10.south) -- (sn0x993d988.north);
\draw (sn0x993fb10.south) -- (sn0x993f700.north);
\draw (sn0x993e438.south) -- (sn0x993d988.north);
\draw (sn0x993e438.south) -- (sn0x993e010.north);
\draw (sn0x9940378.south) -- (sn0x993f700.north);
\draw (sn0x9940378.south) -- (sn0x993e010.north);
\draw (sn0x9940378.south) -- (sn0x9940050.north);
\end{tikzpicture}
\renewcommand{\leveltopI}{-10cm + \leveltop}
\renewcommand{\leveltopII}{-10cm + \leveltopI}
\renewcommand{\leveltopIII}{-10cm + \leveltopII}
\renewcommand{\leveltopIIII}{-10cm + \leveltopIII}
\renewcommand{\leveltopIIIII}{-10cm + \leveltopIIII}
\renewcommand{\leveltopIIIIII}{-10cm + \leveltopIIIII}
\renewcommand{\leveltopIIIIIII}{-10cm + \leveltopIIIIII}
\renewcommand{\leveltopIIIIIIII}{-10cm + \leveltopIIIIIII}
\renewcommand{\leveltopIIIIIIIII}{-10cm + \leveltopIIIIIIII}
\begin{tikzpicture}[scale=.2, anchor=south]
\begin{scope}[yshift=\leveltopI cm]
\matrix (line1)[column sep=0.25cm] {
\node[draw=black, rectangle split,  rectangle split parts=2] (sn0x99428e0){
\begin{tikzpicture}[scale=.2]
\node[circle, scale=0.75, fill] (tid0) at (3.75,1.5){};
\node[circle, scale=0.75, fill] (tid1) at (3,3){};
\node[circle, scale=0.75, fill] (tid3) at (2.25,4.5){};
\node[circle, scale=0.75, fill, task_scheduled] (tid6) at (0.75,6){};
\node[circle, scale=0.75, fill, task_scheduled] (tid7) at (2.25,6){};
\node[circle, scale=0.75, fill] (tid8) at (3.75,6){};
\draw[](tid3) -- (tid6);
\draw[](tid3) -- (tid7);
\draw[](tid3) -- (tid8);
\node[circle, scale=0.75, fill] (tid4) at (5.25,4.5){};
\draw[](tid1) -- (tid3);
\draw[](tid1) -- (tid4);
\node[circle, scale=0.75, fill] (tid2) at (6.75,3){};
\node[circle, scale=0.75, fill, task_scheduled] (tid5) at (6.75,4.5){};
\draw[](tid2) -- (tid5);
\draw[](tid0) -- (tid1);
\draw[](tid0) -- (tid2);
\end{tikzpicture}
\nodepart{two}
\footnotesize{$33\:67$}
};
 & 
\\
};
\end{scope}
\begin{scope}[yshift=\leveltopII cm]
\matrix (line2)[column sep=0.25cm] {
\node[draw=black, rectangle split,  rectangle split parts=2] (sn0x99417c8){
\begin{tikzpicture}[scale=.2]
\node[circle, scale=0.75, fill] (tid0) at (3.75,1.5){};
\node[circle, scale=0.75, fill] (tid1) at (3,3){};
\node[circle, scale=0.75, fill] (tid3) at (2.25,4.5){};
\node[circle, scale=0.75, fill, task_scheduled] (tid5) at (0.75,6){};
\node[circle, scale=0.75, fill, task_scheduled] (tid6) at (2.25,6){};
\node[circle, scale=0.75, fill, task_scheduled] (tid7) at (3.75,6){};
\draw[](tid3) -- (tid5);
\draw[](tid3) -- (tid6);
\draw[](tid3) -- (tid7);
\node[circle, scale=0.75, fill] (tid4) at (5.25,4.5){};
\draw[](tid1) -- (tid3);
\draw[](tid1) -- (tid4);
\node[circle, scale=0.75, fill] (tid2) at (6.75,3){};
\draw[](tid0) -- (tid1);
\draw[](tid0) -- (tid2);
\end{tikzpicture}
\nodepart{two}
\footnotesize{$1$}
};
 & 
\node[draw=black, rectangle split,  rectangle split parts=2] (sn0x9942878){
\begin{tikzpicture}[scale=.2]
\node[circle, scale=0.75, fill] (tid0) at (3,1.5){};
\node[circle, scale=0.75, fill] (tid1) at (2.25,3){};
\node[circle, scale=0.75, fill] (tid3) at (1.5,4.5){};
\node[circle, scale=0.75, fill, task_scheduled] (tid6) at (0.75,6){};
\node[circle, scale=0.75, fill, task_scheduled] (tid7) at (2.25,6){};
\draw[](tid3) -- (tid6);
\draw[](tid3) -- (tid7);
\node[circle, scale=0.75, fill] (tid4) at (3.75,4.5){};
\draw[](tid1) -- (tid3);
\draw[](tid1) -- (tid4);
\node[circle, scale=0.75, fill] (tid2) at (5.25,3){};
\node[circle, scale=0.75, fill, task_scheduled] (tid5) at (5.25,4.5){};
\draw[](tid2) -- (tid5);
\draw[](tid0) -- (tid1);
\draw[](tid0) -- (tid2);
\end{tikzpicture}
\nodepart{two}
\footnotesize{$33\:67$}
};
 & 
\\
};
\end{scope}
\begin{scope}[yshift=\leveltopIII cm]
\matrix (line3)[column sep=0.25cm] {
\node[draw=black, rectangle split,  rectangle split parts=2] (sn0x993fb10){
\begin{tikzpicture}[scale=.2]
\node[circle, scale=0.75, fill] (tid0) at (3,1.5){};
\node[circle, scale=0.75, fill] (tid1) at (2.25,3){};
\node[circle, scale=0.75, fill] (tid3) at (1.5,4.5){};
\node[circle, scale=0.75, fill, task_scheduled] (tid5) at (0.75,6){};
\node[circle, scale=0.75, fill, task_scheduled] (tid6) at (2.25,6){};
\draw[](tid3) -- (tid5);
\draw[](tid3) -- (tid6);
\node[circle, scale=0.75, fill, task_scheduled] (tid4) at (3.75,4.5){};
\draw[](tid1) -- (tid3);
\draw[](tid1) -- (tid4);
\node[circle, scale=0.75, fill] (tid2) at (5.25,3){};
\draw[](tid0) -- (tid1);
\draw[](tid0) -- (tid2);
\end{tikzpicture}
\nodepart{two}
\footnotesize{$33\:67$}
};
 & 
\node[draw=black, rectangle split,  rectangle split parts=2] (sn0x9940378){
\begin{tikzpicture}[scale=.2]
\node[circle, scale=0.75, fill] (tid0) at (2.25,1.5){};
\node[circle, scale=0.75, fill] (tid1) at (1.5,3){};
\node[circle, scale=0.75, fill] (tid3) at (0.75,4.5){};
\node[circle, scale=0.75, fill, task_scheduled] (tid6) at (0.75,6){};
\draw[](tid3) -- (tid6);
\node[circle, scale=0.75, fill, task_scheduled] (tid4) at (2.25,4.5){};
\draw[](tid1) -- (tid3);
\draw[](tid1) -- (tid4);
\node[circle, scale=0.75, fill] (tid2) at (3.75,3){};
\node[circle, scale=0.75, fill, task_scheduled] (tid5) at (3.75,4.5){};
\draw[](tid2) -- (tid5);
\draw[](tid0) -- (tid1);
\draw[](tid0) -- (tid2);
\end{tikzpicture}
\nodepart{two}
\footnotesize{$33\:33\:33$}
};
 & 
\\
};
\end{scope}
\begin{scope}[yshift=\leveltopIIII cm]
\matrix (line4)[column sep=0.25cm] {
\node[draw=black, rectangle split,  rectangle split parts=2] (sn0x993d988){
\begin{tikzpicture}[scale=.2]
\node[circle, scale=0.75, fill] (tid0) at (2.25,1.5){};
\node[circle, scale=0.75, fill] (tid1) at (1.5,3){};
\node[circle, scale=0.75, fill] (tid3) at (1.5,4.5){};
\node[circle, scale=0.75, fill, task_scheduled] (tid4) at (0.75,6){};
\node[circle, scale=0.75, fill, task_scheduled] (tid5) at (2.25,6){};
\draw[](tid3) -- (tid4);
\draw[](tid3) -- (tid5);
\draw[](tid1) -- (tid3);
\node[circle, scale=0.75, fill, task_scheduled] (tid2) at (3.75,3){};
\draw[](tid0) -- (tid1);
\draw[](tid0) -- (tid2);
\end{tikzpicture}
\nodepart{two}
\footnotesize{$33\:67$}
};
 & 
\node[draw=black, rectangle split,  rectangle split parts=2] (sn0x993f700){
\begin{tikzpicture}[scale=.2]
\node[circle, scale=0.75, fill] (tid0) at (2.25,1.5){};
\node[circle, scale=0.75, fill] (tid1) at (1.5,3){};
\node[circle, scale=0.75, fill] (tid3) at (0.75,4.5){};
\node[circle, scale=0.75, fill, task_scheduled] (tid5) at (0.75,6){};
\draw[](tid3) -- (tid5);
\node[circle, scale=0.75, fill, task_scheduled] (tid4) at (2.25,4.5){};
\draw[](tid1) -- (tid3);
\draw[](tid1) -- (tid4);
\node[circle, scale=0.75, fill, task_scheduled] (tid2) at (3.75,3){};
\draw[](tid0) -- (tid1);
\draw[](tid0) -- (tid2);
\end{tikzpicture}
\nodepart{two}
\footnotesize{$33\:33\:33$}
};
 & 
\node[draw=black, rectangle split,  rectangle split parts=2] (sn0x993e010){
\begin{tikzpicture}[scale=.2]
\node[circle, scale=0.75, fill] (tid0) at (1.5,1.5){};
\node[circle, scale=0.75, fill] (tid1) at (0.75,3){};
\node[circle, scale=0.75, fill] (tid3) at (0.75,4.5){};
\node[circle, scale=0.75, fill, task_scheduled] (tid5) at (0.75,6){};
\draw[](tid3) -- (tid5);
\draw[](tid1) -- (tid3);
\node[circle, scale=0.75, fill] (tid2) at (2.25,3){};
\node[circle, scale=0.75, fill, task_scheduled] (tid4) at (2.25,4.5){};
\draw[](tid2) -- (tid4);
\draw[](tid0) -- (tid1);
\draw[](tid0) -- (tid2);
\end{tikzpicture}
\nodepart{two}
\footnotesize{$50\:50$}
};
 & 
\node[draw=black, rectangle split,  rectangle split parts=2] (sn0x9940050){
\begin{tikzpicture}[scale=.2]
\node[circle, scale=0.75, fill] (tid0) at (2.25,1.5){};
\node[circle, scale=0.75, fill] (tid1) at (1.5,3){};
\node[circle, scale=0.75, fill, task_scheduled] (tid3) at (0.75,4.5){};
\node[circle, scale=0.75, fill, task_scheduled] (tid4) at (2.25,4.5){};
\draw[](tid1) -- (tid3);
\draw[](tid1) -- (tid4);
\node[circle, scale=0.75, fill] (tid2) at (3.75,3){};
\node[circle, scale=0.75, fill, task_scheduled] (tid5) at (3.75,4.5){};
\draw[](tid2) -- (tid5);
\draw[](tid0) -- (tid1);
\draw[](tid0) -- (tid2);
\end{tikzpicture}
\nodepart{two}
\footnotesize{$33\:67$}
};
 & 
\\
};
\end{scope}
\draw (sn0x99428e0.south) -- (sn0x99417c8.north);
\draw (sn0x99428e0.south) -- (sn0x9942878.north);
\draw (sn0x99417c8.south) -- (sn0x993fb10.north);
\draw (sn0x9942878.south) -- (sn0x993fb10.north);
\draw (sn0x9942878.south) -- (sn0x9940378.north);
\draw (sn0x993fb10.south) -- (sn0x993d988.north);
\draw (sn0x993fb10.south) -- (sn0x993f700.north);
\draw (sn0x9940378.south) -- (sn0x993f700.north);
\draw (sn0x9940378.south) -- (sn0x993e010.north);
\draw (sn0x9940378.south) -- (sn0x9940050.north);
\end{tikzpicture}
\renewcommand{\leveltopI}{-10cm + \leveltop}
\renewcommand{\leveltopII}{-10cm + \leveltopI}
\renewcommand{\leveltopIII}{-10cm + \leveltopII}
\renewcommand{\leveltopIIII}{-10cm + \leveltopIII}
\renewcommand{\leveltopIIIII}{-10cm + \leveltopIIII}
\renewcommand{\leveltopIIIIII}{-10cm + \leveltopIIIII}
\renewcommand{\leveltopIIIIIII}{-10cm + \leveltopIIIIII}
\renewcommand{\leveltopIIIIIIII}{-10cm + \leveltopIIIIIII}
\renewcommand{\leveltopIIIIIIIII}{-10cm + \leveltopIIIIIIII}
\begin{tikzpicture}[scale=.2, anchor=south]
\begin{scope}[yshift=\leveltopI cm]
\matrix (line1)[column sep=0.25cm] {
\node[draw=black, rectangle split,  rectangle split parts=2] (sn0x9942a08){
\begin{tikzpicture}[scale=.2]
\node[circle, scale=0.75, fill] (tid0) at (3.75,1.5){};
\node[circle, scale=0.75, fill] (tid1) at (3,3){};
\node[circle, scale=0.75, fill] (tid3) at (2.25,4.5){};
\node[circle, scale=0.75, fill, task_scheduled] (tid6) at (0.75,6){};
\node[circle, scale=0.75, fill, task_scheduled] (tid7) at (2.25,6){};
\node[circle, scale=0.75, fill, task_scheduled] (tid8) at (3.75,6){};
\draw[](tid3) -- (tid6);
\draw[](tid3) -- (tid7);
\draw[](tid3) -- (tid8);
\node[circle, scale=0.75, fill] (tid4) at (5.25,4.5){};
\draw[](tid1) -- (tid3);
\draw[](tid1) -- (tid4);
\node[circle, scale=0.75, fill] (tid2) at (6.75,3){};
\node[circle, scale=0.75, fill] (tid5) at (6.75,4.5){};
\draw[](tid2) -- (tid5);
\draw[](tid0) -- (tid1);
\draw[](tid0) -- (tid2);
\end{tikzpicture}
\nodepart{two}
\footnotesize{$1$}
};
 & 
\\
};
\end{scope}
\begin{scope}[yshift=\leveltopII cm]
\matrix (line2)[column sep=0.25cm] {
\node[draw=black, rectangle split,  rectangle split parts=2] (sn0x9942878){
\begin{tikzpicture}[scale=.2]
\node[circle, scale=0.75, fill] (tid0) at (3,1.5){};
\node[circle, scale=0.75, fill] (tid1) at (2.25,3){};
\node[circle, scale=0.75, fill] (tid3) at (1.5,4.5){};
\node[circle, scale=0.75, fill, task_scheduled] (tid6) at (0.75,6){};
\node[circle, scale=0.75, fill, task_scheduled] (tid7) at (2.25,6){};
\draw[](tid3) -- (tid6);
\draw[](tid3) -- (tid7);
\node[circle, scale=0.75, fill] (tid4) at (3.75,4.5){};
\draw[](tid1) -- (tid3);
\draw[](tid1) -- (tid4);
\node[circle, scale=0.75, fill] (tid2) at (5.25,3){};
\node[circle, scale=0.75, fill, task_scheduled] (tid5) at (5.25,4.5){};
\draw[](tid2) -- (tid5);
\draw[](tid0) -- (tid1);
\draw[](tid0) -- (tid2);
\end{tikzpicture}
\nodepart{two}
\footnotesize{$33\:67$}
};
 & 
\\
};
\end{scope}
\begin{scope}[yshift=\leveltopIII cm]
\matrix (line3)[column sep=0.25cm] {
\node[draw=black, rectangle split,  rectangle split parts=2] (sn0x993fb10){
\begin{tikzpicture}[scale=.2]
\node[circle, scale=0.75, fill] (tid0) at (3,1.5){};
\node[circle, scale=0.75, fill] (tid1) at (2.25,3){};
\node[circle, scale=0.75, fill] (tid3) at (1.5,4.5){};
\node[circle, scale=0.75, fill, task_scheduled] (tid5) at (0.75,6){};
\node[circle, scale=0.75, fill, task_scheduled] (tid6) at (2.25,6){};
\draw[](tid3) -- (tid5);
\draw[](tid3) -- (tid6);
\node[circle, scale=0.75, fill, task_scheduled] (tid4) at (3.75,4.5){};
\draw[](tid1) -- (tid3);
\draw[](tid1) -- (tid4);
\node[circle, scale=0.75, fill] (tid2) at (5.25,3){};
\draw[](tid0) -- (tid1);
\draw[](tid0) -- (tid2);
\end{tikzpicture}
\nodepart{two}
\footnotesize{$33\:67$}
};
 & 
\node[draw=black, rectangle split,  rectangle split parts=2] (sn0x9940378){
\begin{tikzpicture}[scale=.2]
\node[circle, scale=0.75, fill] (tid0) at (2.25,1.5){};
\node[circle, scale=0.75, fill] (tid1) at (1.5,3){};
\node[circle, scale=0.75, fill] (tid3) at (0.75,4.5){};
\node[circle, scale=0.75, fill, task_scheduled] (tid6) at (0.75,6){};
\draw[](tid3) -- (tid6);
\node[circle, scale=0.75, fill, task_scheduled] (tid4) at (2.25,4.5){};
\draw[](tid1) -- (tid3);
\draw[](tid1) -- (tid4);
\node[circle, scale=0.75, fill] (tid2) at (3.75,3){};
\node[circle, scale=0.75, fill, task_scheduled] (tid5) at (3.75,4.5){};
\draw[](tid2) -- (tid5);
\draw[](tid0) -- (tid1);
\draw[](tid0) -- (tid2);
\end{tikzpicture}
\nodepart{two}
\footnotesize{$33\:33\:33$}
};
 & 
\\
};
\end{scope}
\begin{scope}[yshift=\leveltopIIII cm]
\matrix (line4)[column sep=0.25cm] {
\node[draw=black, rectangle split,  rectangle split parts=2] (sn0x993d988){
\begin{tikzpicture}[scale=.2]
\node[circle, scale=0.75, fill] (tid0) at (2.25,1.5){};
\node[circle, scale=0.75, fill] (tid1) at (1.5,3){};
\node[circle, scale=0.75, fill] (tid3) at (1.5,4.5){};
\node[circle, scale=0.75, fill, task_scheduled] (tid4) at (0.75,6){};
\node[circle, scale=0.75, fill, task_scheduled] (tid5) at (2.25,6){};
\draw[](tid3) -- (tid4);
\draw[](tid3) -- (tid5);
\draw[](tid1) -- (tid3);
\node[circle, scale=0.75, fill, task_scheduled] (tid2) at (3.75,3){};
\draw[](tid0) -- (tid1);
\draw[](tid0) -- (tid2);
\end{tikzpicture}
\nodepart{two}
\footnotesize{$33\:67$}
};
 & 
\node[draw=black, rectangle split,  rectangle split parts=2] (sn0x993f700){
\begin{tikzpicture}[scale=.2]
\node[circle, scale=0.75, fill] (tid0) at (2.25,1.5){};
\node[circle, scale=0.75, fill] (tid1) at (1.5,3){};
\node[circle, scale=0.75, fill] (tid3) at (0.75,4.5){};
\node[circle, scale=0.75, fill, task_scheduled] (tid5) at (0.75,6){};
\draw[](tid3) -- (tid5);
\node[circle, scale=0.75, fill, task_scheduled] (tid4) at (2.25,4.5){};
\draw[](tid1) -- (tid3);
\draw[](tid1) -- (tid4);
\node[circle, scale=0.75, fill, task_scheduled] (tid2) at (3.75,3){};
\draw[](tid0) -- (tid1);
\draw[](tid0) -- (tid2);
\end{tikzpicture}
\nodepart{two}
\footnotesize{$33\:33\:33$}
};
 & 
\node[draw=black, rectangle split,  rectangle split parts=2] (sn0x993e010){
\begin{tikzpicture}[scale=.2]
\node[circle, scale=0.75, fill] (tid0) at (1.5,1.5){};
\node[circle, scale=0.75, fill] (tid1) at (0.75,3){};
\node[circle, scale=0.75, fill] (tid3) at (0.75,4.5){};
\node[circle, scale=0.75, fill, task_scheduled] (tid5) at (0.75,6){};
\draw[](tid3) -- (tid5);
\draw[](tid1) -- (tid3);
\node[circle, scale=0.75, fill] (tid2) at (2.25,3){};
\node[circle, scale=0.75, fill, task_scheduled] (tid4) at (2.25,4.5){};
\draw[](tid2) -- (tid4);
\draw[](tid0) -- (tid1);
\draw[](tid0) -- (tid2);
\end{tikzpicture}
\nodepart{two}
\footnotesize{$50\:50$}
};
 & 
\node[draw=black, rectangle split,  rectangle split parts=2] (sn0x9940050){
\begin{tikzpicture}[scale=.2]
\node[circle, scale=0.75, fill] (tid0) at (2.25,1.5){};
\node[circle, scale=0.75, fill] (tid1) at (1.5,3){};
\node[circle, scale=0.75, fill, task_scheduled] (tid3) at (0.75,4.5){};
\node[circle, scale=0.75, fill, task_scheduled] (tid4) at (2.25,4.5){};
\draw[](tid1) -- (tid3);
\draw[](tid1) -- (tid4);
\node[circle, scale=0.75, fill] (tid2) at (3.75,3){};
\node[circle, scale=0.75, fill, task_scheduled] (tid5) at (3.75,4.5){};
\draw[](tid2) -- (tid5);
\draw[](tid0) -- (tid1);
\draw[](tid0) -- (tid2);
\end{tikzpicture}
\nodepart{two}
\footnotesize{$33\:67$}
};
 & 
\\
};
\end{scope}
\draw (sn0x9942a08.south) -- (sn0x9942878.north);
\draw (sn0x9942878.south) -- (sn0x993fb10.north);
\draw (sn0x9942878.south) -- (sn0x9940378.north);
\draw (sn0x993fb10.south) -- (sn0x993d988.north);
\draw (sn0x993fb10.south) -- (sn0x993f700.north);
\draw (sn0x9940378.south) -- (sn0x993f700.north);
\draw (sn0x9940378.south) -- (sn0x993e010.north);
\draw (sn0x9940378.south) -- (sn0x9940050.north);
\end{tikzpicture}

%%% Local Variables:
%%% TeX-master: "thesis/thesis.tex"
%%% End: 

  \caption{A combination of P3L and P1S is not an optimality criterion for an optimal schedule.\todo{Subfigures.}}
  \label{fig:p3l-p1s-combo-suboptimal}
\end{figure}

\begin{corollary}
  Let $T^s$ denote the overall run time of a schedule $s$ and $T_1^s$, $T_2^s$ and $T_3^s$ be the times where exactly three, two and one tasks are scheduled within this schedule, respectively.

  Let $I$ be an intree and $S$ be the set of all schedules. Let $s^*$ be the optimal schedule, which has associated the optimal run time $T^*$, with $T_1^*, T_2^*, T_3^*$ being its parts.
  \begin{itemize}
  \item It may be the case that there is a schedule $s\in S$ such that $T_3^s \geq T_3^*$.
  \item It may be the case that there is a schedule $s\in S$ such that $T_1^s \leq T_1^*$.
  \end{itemize}
\end{corollary}

That is, it is not necessarily the case that $T_3$ is maximal for the optimal schedule, nor is it necessarily the case that $T_1$ is minimal for the optimal schedule.

However, after some investigation, we are tempted to conjecture the following.

\begin{conjecture}
  Let $I$, $T^s, T_1^s, T_2^s, T_3^s$ and $S$ be as defined above. Let $s^*$ be the optimal schedule for $I$ associated with the respective times $T_1^*, T_2^*, T_3^*$. Then, there is no schedule $s\in S$ such that
  \begin{equation*}
    T^s > T^* \wedge T_1^s \leq T_1^* \wedge T_3^s \geq T_3^*.
  \end{equation*}
\end{conjecture}

Even if this conjecture turns out to be true, it seems complex to transform this knowledge into a scheduling strategy that does something more significantly efficient than ``explore everything, and choose the best'', because $T_3, T_2$ and $T_1$ are not that easy to compute.

\clearpage{}

\section{``No free chains''}
\label{sec:disproving-hlf-no-free-chain}

We can consider all paths from the root of a tree to all its leaves. We might be tempted to think, that it should be the foremost goal to exploit parallelism as good as possible and that this might be acchieved by choosing the currently scheduled tasks in a manner such that as few paths as possible are completely free. That is, we choose the leafs in a way so that the ends of as many different paths as possible are scheduled. This strategy was inspired by looking at the counterexamples against HLF depicted in figures \ref{fig:hlf-001112}, \ref{fig:hlf-vs-opt-0012346688}, \ref{fig:hlf-vs-opt-0012446788} and \ref{fig:hlf-vs-opt-00123455799}. We observe for these intrees that the optimal schedules has no as few free chains as possible.

However, there are examples where this strategy does not yield optimal results. Consider e.g. the tree $(0,0,0,1,1,1,2,2,3)$. 

\begin{figure}[ht]
  \centering
  \begin{subfigure}{.45\textwidth}
    \centering
    \includegraphics{p3/hlfnfc_not_optimal/000111223_hlfnfc.pdf}
    \caption{HLF schedule while choosing tasks such that there are as few free paths as possible -- overall run time of 5.20199.}
  \end{subfigure}
  \quad
  \begin{subfigure}{.45\textwidth}
    \centering
    \includegraphics{p3/hlfnfc_not_optimal/000111223_opt.pdf}
    \caption{Optimal schedule (run time 5.20028) has three free paths at the beginning.}
  \end{subfigure}
  \caption{HLF with no free paths is not necessarily optimal.}
  \label{fig:hlfnfc-is-not-optimal}
\end{figure}

\section{``2-HLF plus 1''}
\label{sec:disproving-2hlf-plus-1}

We examined all intrees with up to 14 tasks, especially the cases where HLF is not optimal. Thereby, we obsered that in all cases where three tasks could be scheduled, the optimal solution scheduled two tasks, that could be chosen by HLF for two processors and only the third task \emph{might} be a task that would not have been chosen by HLF (see figures \ref{fig:hlf-vs-opt-0012346688}, \ref{fig:hlf-vs-opt-0012446788} and \ref{fig:hlf-vs-opt-00123455799} as particular instances of those). Thus, we examined whether an optimal scheduling strategy for three processors has always \emph{at most one} task that is non-HLF. Interestingly, there is an intree with 15 tasks, whose optimal schedule starts out by choosing the single topmost task and \emph{two} non-HLF tasks. This intree ($(0,0,1,2,2,3,3,6,8,9,10,11,12,13)$) is shown in figure \ref{fig:2-hlf-plus-one-not-optimal}. Another tree showing this fact is e.g. $(0, 0, 1, 2, 3, 4, 4, 5, 5, 8, 10, 11, 12)$.

\begin{figure}[ht]
  \centering
  \begin{subfigure}{.45\textwidth}
    \centering
    \includegraphics{p3/2hlf_suboptimal/001223368910111213_opt.pdf}
    \caption{Optimal schedule picking \emph{two} non-HLF tasks.}
  \end{subfigure}
  \quad
  \begin{subfigure}{.45\textwidth}
    \centering
    \includegraphics{p3/2hlf_suboptimal/001223368910111213_subopt.pdf}
    \caption{HLF-Schedule}
  \end{subfigure}
  \caption{Intree $(0,0,1,2,2,3,3,6,8,9,10,11,12,13)$ requires the optimal schedule to start out by choosing two non-HLF tasks.}
  \label{fig:2-hlf-plus-one-not-optimal}
\end{figure}

\section{Only highest or lowest leaves}
\label{sec:disproving-only-highest-or-lowest-leaves}

So far, we have seen several scenarios where HLF was not optimal. The trees seen so far where HLF was not optimal resulted in schedules that picked only combinations highest leaves and lowest leaves possible. However, this is not a criterion for an optimal schedule, as we can observe by scheduling the 14-tasks-intree $(0,0,0,2,3,4,5,7,7,9,10,10,12)$, which is shown in figure \ref{fig:only-high-or-low-not-optimal}.

\begin{figure}[ht]
  \centering
  \begin{subfigure}{.45\textwidth}
    \centering
    \includegraphics{p3/only_high_or_low/0002345779101012_opt.pdf}
    \caption{Optimal schedule picking \emph{two} non-HLF tasks.}
  \end{subfigure}
  \quad
  \begin{subfigure}{.45\textwidth}
    \centering
    \includegraphics{p3/only_high_or_low/0002345779101012_subopt.pdf}
    \caption{Suboptimal HLF-Schedule.}
  \end{subfigure}
  \caption{Intree $(0,0,0,2,3,4,5,7,7,9,10,10,12)$ requires the optimal schedule to start out by choosing two non-HLF tasks. All other schedules yield higher expected run times.}
  \label{fig:only-high-or-low-not-optimal}
\end{figure}

\section{Conclusion}
\label{sec:p3-conclusion}

Unfortunately, we did not find any strategy that always yields an optimal schedule. Of course, it is still possible to compute the optimal schedule by an exhaustive search.

During our research, we recognized some patterns that we are tempted to transform into some conjectures. We were, however, not yet able to prove or disprove them. This section shows the most important ones that could easily be used to decrease the number of snapshots that have to be examined if we want to compute the optimal snapshot by exhaustive search.

\begin{definition}[Topmost task]
  We say that a task is \emph{topmost} if its level is greater or equal than the levels of any other task in the intree. 
\end{definition}

% \begin{conjecture}[Very likely]
%   For each snapshot resulting from an optimal schedule, at least one top-most task is scheduled.
% \end{conjecture}

% \begin{conjecture}[Weak one]
%   For each snapshot resulting from an optimal schedule, at least two top-most tasks are scheduled -- if there are two or more topmost tasks.
% \end{conjecture}

\begin{conjecture}
  An optimal schedule always schedules as many topmost tasks as possible.
\end{conjecture}

\begin{conjecture}
  If for an intree only non-top tasks are scheduled, you can schedule any top-task instead of one non-top scheduled task to obtain a better run time.
\end{conjecture}

\begin{conjecture}
  Parallel chains are optimally scheduled by HLF.
\end{conjecture}

The main problems we faced when we tried to prove the above conjectures can be summarized as follows:
\begin{itemize}
\item The particular intree structure is not necessarily maintained over the induction step --- and if so, many case distinctions may be required.
\item Comparing different intrees seems to be quite cumbersome, especially if we do not know which tasks are scheduled.
\end{itemize}


%%% Local Variables:
%%% TeX-master: "../thesis.tex"
%%% End: 
\chapter{Properties of schedule DAGs and optimal schedules}
\label{chap:p3}

We now researching some properties of snapshot DAGs and optimal schedules. In particular, we will look at a particular non-trivial class of intrees, for which HLF is optimal.

\section{Properties of optimal schedules}
\label{sec:optimal-schedules-properties}

As shown in \cite{chandyreynoldslargepaper1979}, it is known that an optimal scheduling strategy does not keep a processor idle if it could do some work. An intuitive explanation of this fact is as follows: Assume that there is a strategy that keeps a processor idle at some point even if there is a ready task $t$ that could be processed. Then, we construct a new strategy, that schedules $t$ (using the idle processor) and behaves like the original strategy. Then, it can be shown that this new strategy yields a smaller overall expected run time.



\section{Size of the snapshot DAG}
\label{sec:p3-size-of-snapshot-dag-first-attempts}

Similar to the reasoning in section \ref{sec:p2-snapshot-dag}, we can research the size of a snapshot DAG for the P3 case. 
We conducted an experiment and examined the size for snapshot DAGs of intrees containing up to 17 tasks. 
Therefore, we generated all intrees (up to isomorphism) with a certain number of tasks (see section \ref{sec:enumerating-all-intrees} for an algorithm).
Then we computed the following for each intree:
\begin{itemize}
\item Number of distinct (i.e. non-isomorphic) subtrees.
\item Number of snapshots that can be constructed using the LEAF scheduling strategy (i.e. ``try-everything'' scheduling).
\item The size of the \emph{optimal} snapshot DAG.
\end{itemize}

We do so because of the following: It is easily possible to construct an optimal schedule if we take the possible snapshot DAGs of the LEAF scheduler and only leave the choices that yield the best expected run time.

Since the size of the snapshot DAG for an intree with $n$ tasks is at most $n^3$ times as large as the number of subtrees of the original intree, we also computed the number of subtrees for each of these intrees.
That is, we can compare the number of intrees to the number of snapshots to be considered.

The results are summed up in table \ref{tab:num-subtrees-size-of-dags}\todo{Complete!}.

\begin{table}[ht]
  \centering
  \begin{tabular}[ht]{ccccccc}
    \multirow{2}{*}{Tasks} & \multicolumn{2}{c}{Subtrees} & \multicolumn{2}{c}{Snapshots} & \multicolumn{2}{c}{``Optimal DAG''} \\
    & Max & Avg & Max & Avg & Max & Avg \\
    \hline
    3 & 3 & 3.00 & 3 & 3.00 & 3 & 3.00  \\
    4 & 5 & 4.25 & 5 & 4.25 & 5 & 4.25  \\
    5 & 7 & 5.89 & 7 & 5.89 & 7 & 5.89  \\
    6 & 11 & 8.10 & 11 & 8.25 & 11 & 8.05  \\
    7 & 16 & 11.04 & 19 & 11.75 & 16 & 10.81  \\
    8 & 24 &  15.10 & 34 & 17.39 & 22 & 14.37  \\
    9 & 34 &  20.57 & 63 & 26.53 & 31 & 18.76  \\
    10 & 54 &  28.08 & 119 & 41.85 & 41 & 24.16  \\
    11 & 79 &  38.33 & 230 & 67.48 & 55 & 30.67  \\
    12 & 119 & 52.41 & 437 & 112.68 & 71 & 38.41  \\
    13 & 169 &  71.69 & 812 & 184.95 & 89 & 47.49  \\
    14 & 269 &  98.19 & 1510 & 304.41 & 113 & 58.05  \\
    % 15 & 357 &  125.70 & 142 & 67.83  \\
    % 16 & 594 &  171.29 & 184 & 80.55  \\
    % 17 & 850 &  240.39 & 235 & 96.67  \\
  \end{tabular}
  \caption{Number of subtrees, size of the optimized snapshot DAG depending on the number of tasks. ``Subtrees'' denotes the number of distinct subtrees. ``Snapshots'' shows the number of distinct snapshots that have to be examined if we try all possible schedules. The column ``Optimal DAG'' shows the size of the snapshot DAG describing the optimal schedule.}
  \label{tab:num-subtrees-size-of-dags}
\end{table}

As we can see in table \ref{tab:num-subtrees-size-of-dags}, the number of subtrees is (at least for $n\geq  9$) significantly larger than the number of snapshots in the snapshot DAG describing an optimal schedule.

Another interesting fact is that there is no ``strict correlation'' between the number of subtrees and the number of snapshots in the optimal snapshot DAG. That is, there are certain DAGs that have more non-isomorphic subtrees than another DAG, yet -- on the other hand -- more snapshots in the optimal snapshot DAG. As an example, consider the intrees $T_1$ described by 00011111 and $T_2$ described by 00001234: $T_1$ has 19 subtrees and its optimal snapshot DAG contains 13 snapshots, while $T_2$ has only 15 subtrees, but an optimal snapshot DAG containing 14 snapshots.

Moreover, intrees containing $n$ tasks and having the maximal number of subtrees are (at least for $8\leq n \leq 17$) are not the ones having the largest optimal snapshot DAG.

To determine the maximum size of the optimal snapshot DAG for the P3 case, it might be useful to investigate whether the trees that have a large snapshot DAG can be constructed according to a specific pattern. The intrees resulting in snapshot DAGs of maximum size are depicted in figure \ref{fig:intrees-maximum-snapshot-dag-size-p3}. The intrees in this figure seem to behave quite chaotic and we were not able to deduce any pattern according to which they could be generated for general $n$.

\begin{figure}[t]
  \centering
  \includegraphics[scale=1.4]{p3/max_unoptimized.pdf}
  \caption{These are the intrees for which the a brute-force algorithm has to generate maximally many snapshots to generate the optimal schedule (maximal compared to all other intrees with the same number $n$ of vertices). We show $2\leq n \leq 14$.}
  \label{fig:intrees-maximum-unoptimized-p3}
\end{figure}

\begin{figure}[t]
  \centering
  \includegraphics[scale=1.4]{p3/max_snapshot_dag.pdf}
  \caption{These intrees result in \emph{optimal} snapshot DAGs that are larger than all other optimal snapshot DAGs resulting from intrees having the same number of tasks $n$ ($2 \leq n \leq 17$ shown). There seems to be no simple pattern according to which these trees are constructed.}
  \label{fig:intrees-maximum-snapshot-dag-size-p3}
\end{figure}

% \section{Special classes of intrees}
% \label{sec:p3-dag-size-special-class-of-intrees}

% If we consider trees. whose sequence description is of the form $(0, 0, 1, 1, 3, 3, 5, 5, 7,7, 9,9,\dots)$, that have an even number of nodes, then the optimal snapshot DAGs have $\binom{n}{1}+\binom{n}{2}+\binom{n}{3}$ snapshots. \todo{Make this conjecture and nice!}

% \begin{table}[ht]
%   \centering
%   \begin{tabular}{lcc}
%     Class & No. snaps & Opt. size \\
%     $(0,0,1,1,3,3,5,5,\dots)$ & & $\binom{n/2}{1}+\binom{n/2}{2}+\binom{n/2}{3}$ \\
%     $(0,0,0,1,1,1,4,4,4,7,7,7,\dots)$ & & $((n/3)^3 + 2*(n/3))/3$
%   \end{tabular}
%   \caption{Classes and their DAG sizes}
%   \label{tab:special-classes-dag-sizes}
% \end{table}

\section{Degenerate intrees}
\label{sec:p3-degenerate-intrees}

\todo{Definitions: intree, level, suc, adding tasks to trees etc.}

We now focus one one particular class on intrees, namely \emph{degenerate intrees}. A degenerate intree is an intree that consists of one longest chain from the bottom to one leaf, and all other tasks are direct predecessors to one of the tasks within this longest chain. Another characterization is the following: On each level, \emph{at most one task} has predecessors.\todo{Figure zeigen.}

\subsection{Intro: Degenerate binary trees}
\label{sec:p3-degenerate-trees-binary}

We researched degenerate binary trees, i.e. trees whose sequence has the structure
\begin{equation*}
  \left( 0,0,a_0,a_1,a_2,a_3,a_4,\dots,a_n \right),
\end{equation*}
for $n+3$ the total number of tasks within the intree. The values $a_i$ can be recursively defined as follows:
\begin{equation*}
  a_k =
  \begin{cases}
    1, & \text{if } k\leq 1 \\
    a_{k-1}, & \text{if } k>1 \text{ is odd} \\
    a_{k-1}+2, & \text{if } k>1 \text{ is even}
  \end{cases}
\end{equation*}

That is, degenerate binary trees have sequences of the form $(0,0,1,1,3,3,5,5,7,7,9,\dots)$.

We now examine how many snapshots are considered if we compute the optimal P3 schedule by considering \emph{all} possibilities and afterwards discarding the bad ones. The results are summed up in table \ref{tab:p3-degenerate-binary-trees-no-snapshots}. We clearly observe that the number of snapshots grows exponentially (at least within the range for $n$ under consideration). A simple pattern that can be observed from table \ref{tab:p3-degenerate-binary-trees-no-snapshots} is that (at least for $n\leq 26$) that the number of snapshots for a degenerate binary tree with $n$ tasks is greater than twice the number of snapshots for a degenerate binary tree with $n-2$ tasks. If $S(n)$ denotes the number of snapshots for a degenerate binary tree, we can formulate $S(n)>S(n-2)$, which we can (by induction) convert to $S(n) > \sqrt 2 ^ n$.

This can be illustrated by the fact that degenerate binary intrees are fully determined by their profile (please see section \ref{sec:p2-profiles} for the definition of profiles). A degenerate binary tree has a profile of the form
\begin{equation*}
  \profile{a,2,2,2,2,\dots,2,1},
\end{equation*}
where $a$ is either $1$ or $2$. Assume the length of the profile (i.e. the height of the degenerate binary tree) is exactly $l$. Then, we have $2\cdot(l-2)+1+a = 2l-1+a$ tasks in total. Assume for now that $a=2$ (i.e. we are dealing with a complete degenerate binary tree) and $l>2$.

A subtree of a this degenerate binary intree having height $l'$ has a profile of the form
\begin{equation*}
  \profile{a_0,a_1,a_2,\dots,a_{l'-2},1},
\end{equation*}
where $1\leq a_0\leq a$ and $1\leq a_i \leq 2$ for all $i\in\left\{ 1,2,\dots,l-2 \right\}$. Using basic combinatorics, we can tell that there must be
\begin{equation*}
  \sum_{l'=0}^{l-1} 2^{l'} = 2^{l} -1
\end{equation*}
distinct subtrees if $a=2$ for profile length (resp. intree depth) $l$.

If $a=1$ and we have a profile length of $l$, we cann argue that there must be as many subtrees as for the profile without the first item (then of length $l-1$) plus the number of profiles of length exactly $l-1$ with one additional 1 prepended. These are exactly $2^{l-2}$.

This, in total leads to our desired bound for $S(n)$.\todo{Genauer machen.}

\begin{table}[t]
  \centering
  \begin{tabular}[ht]{ccc}
    \multirow{2}{*}{Tasks} & \multicolumn{2}{c}{Snapshots} \\
    & Overall & HLF \\
    \hline
    3  &  3       & 3   \\
    4  &  5       & 5   \\
    5  &  7       & 7   \\
    6  &  11      & 11  \\
    7  &  17      & 14  \\
    8  &  28      & 21  \\
    9  &  48      & 25  \\
    10 &  85      & 36  \\
  \end{tabular}
  \begin{tabular}[ht]{ccc}
    \multirow{2}{*}{Tasks} & \multicolumn{2}{c}{Snapshots} \\
    & Overall & HLF \\
    \hline
    11 &  150     & 41  \\
    12 &  276     & 57  \\
    13 &  477     & 63  \\
    14 &  884     & 85  \\
    15 &  1477    & 92  \\
    16 &  2717    & 121 \\
    17 &  4398    & 129 \\
    18 &  7991    & 166 \\
  \end{tabular}
  \begin{tabular}[ht]{ccc}
    \multirow{2}{*}{Tasks} & \multicolumn{2}{c}{Snapshots} \\
    & Overall & HLF \\
    \hline
    19 &  12600   & 175 \\
    20 &  22594   & 221 \\
    21 &  34883   & 231 \\
    22 &  61774   & 287 \\
    23 &  93775   & 298 \\
    24 &  164187  & 365 \\
    25 &  245852  & 377 \\
    26 &  426089  & 456 \\
  \end{tabular}
  \caption{Number of snapshots for degenerate binary trees in the P3 case. The first column shows the number of tasks. ``Overall'' denotes the number of distinct snapshots that are explored if an optimal schedule is constructed by examining all schedulings. ``HLF'' denotes the number of distinct snapshots for HLF.}
  \label{tab:p3-degenerate-binary-trees-no-snapshots}
\end{table}

Interestingly, degenerate binary intrees, while having a possibly huge amount of snapshots, are probably optimally scheduled by HLF. You can compare the number of HLF snapshots to the number of overal snapshots by looking at table \ref{tab:p3-degenerate-binary-trees-no-snapshots}.

We generalize this fact in the next section.

\subsection{HLF is optimal for degenerate intrees}
\label{sec:p3-degenerate-intrees-hlf-optimal}

\begin{lemma}
  \label{lem:p3-adding-tasks-level-keep-scheduled-same-inequality}
  Let $I$ be a degenerate intree and $x, y$ two (not necessarily distinct) ready tasks within this intree. Let $z_1, z_2$ be two new tasks that will be added to this intree with $level(z_1) > level(z_2)$ in a manner such that $I_1:=I\cup\left\{ z_1 \right\}$ and $I_2:=I\cup\left\{ z_2 \right\}$ are still degenerate intrees. Moreover, the tasks $z_1$ and $z_2$ shall be added in such a way that neither $x$ nor $y$ is a successor of $z_1$ or $z_2$ (i.e. $x,y$ stay ready in $I_1$ resp. $I_2$). 
  
  By $T^*_{t_1,t_2,t_3}(I)$ we denote the optimal expected run time that can be acchieved if we \emph{initially} schedule all tasks from the set $\left\{ t_1,t_2,t_3 \right\}$. \todo{Notation auslagern.} Note that this notation does not necessarily require that we actually have three tasks (e.g. if $t_1=t_2$).
  
  Then, if $x,y$ and $z_1$ resp. $z_2$ are used as initial tasks, we have the following for the best acchievable expected run times (for respective initial tasks):
  \begin{equation}
    \label{eq:lemma-p3-adding-tasks-level-keep-scheduled-same-inequality}
    T^{*}_{x,y,z_1}\left(I\cup\left\{ z_1 \right\}\right) > T^{*}_{x,y,z_2}\left( I\cup\left\{ z_2 \right\} \right)
  \end{equation}

  If we loosen the level condition to $level(z_1)\geq level(z_2)$, we obtain
  \begin{equation*}
    T^{*}_{x,y,z_1}\left(I\cup\left\{ z_1 \right\}\right) \geq T^{*}_{x,y,z_2}\left( I\cup\left\{ z_2 \right\} \right).
  \end{equation*}
\end{lemma}

\begin{proof}
  We focus first on the case where $level(z_1) > level(z_2)$ and prove the claim by induction over the number of nodes:
  
  The induction basis is the case where we have degenerate intrees with 3 tasks\footnote{We start by 3 tasks since these trees are the only ones that allow adding $z_1$ and $z_2$ at different levels such that both $x$ and $y$ stay ready. For an intree with two tasks, the claim can be seen by simply examining that $T(0,0)<T(0,1)$\todo{Improve this footnote.}.} (all of them are depicted in figure \ref{fig:p3-lemma-adding-intrees-induction-start} (only the black nodes)).
  
  \begin{figure}[t]
    \centering
    \begin{tikzpicture}[scale=0.25]
      \newcommand{\treeone}{
        \fill(0,0) circle (0.4);
        \fill(0,1) circle (0.4);
        \fill(0,2) circle (0.4);
        \draw(0,0) -- (0,1);
        \draw(0,1) -- (0,2);
      }
      \newcommand{\treetwo}{
        \fill(0,0) circle (0.4);
        \fill(-.50,1) circle (0.4);
        \fill(.50,1) circle (0.4);
        \draw(0,0) -- (0.5,1);
        \draw(0,0) -- (-.5,1);
      }
      \newcommand{\treethree}{
        \fill(0,0) circle (0.4);
        \fill(0,1) circle (0.4);
        \draw(0,0) -- (0,1);
      }
      % \begin{scope}
      %   \treeone;
      % \end{scope}
      % \begin{scope}[xshift=9cm]
      %   \treetwo;
      % \end{scope}

      \begin{scope}[yshift=-5cm, xshift=-20.5cm]
        \begin{scope}
          \treethree;
          \draw(0,1) -- (0,2);
          \draw[fill=white](0,2) circle (0.4);
          \node at (0,-2) {3};
        \end{scope}
        \node at (2.5,-2) {$>$};
        \begin{scope}[xshift=5cm]
          \treethree;
          \draw(0,0) -- (1,1);
          \draw[fill=white](1,1) circle (0.4);
          \node at (0,-2) {2.5};
        \end{scope}
      \end{scope}

      \begin{scope}[yshift=-5cm, xshift=-3.5cm]
        \begin{scope}
          \treeone;
          \draw(0,2) -- (0,3);
          \draw[fill=white](0,3) circle (0.4);
          \node at (0,-2) {4};
        \end{scope}
        \node at (2.5,-2) {$>$};
        \begin{scope}[xshift=5cm]
          \treeone;
          \draw(0,1) -- (1,2);
          \draw[fill=white](1,2) circle (0.4);
          \node at (0,-2) {3.5};
        \end{scope}
        \node at (8,-2) {$>$};
        \begin{scope}[xshift=11cm]
          \treeone;
          \draw(0,0) -- (1,1);
          \draw[fill=white](1,1) circle (0.4);
          \node at (0,-2) {3.25};
        \end{scope}
        % \node at (4,-1.5) {$4 > 3.5 > 3.25$};
      \end{scope}

      \begin{scope}[yshift=-5cm, xshift=20cm]
        \begin{scope}[xshift=0cm]
          \treetwo;
          \draw(-.5,1) -- (-.5,2);
          \draw[fill=white](-.5,2) circle (0.4);
          \node at (0,-2) {3.25};
        \end{scope}
        \node at (2.5,-2) {$>$};
        \begin{scope}[xshift=5cm]
          \treetwo;
          \draw(0,0) -- (1.5,1);
          \draw[fill=white](1.5,1) circle (0.4);
          \node at (0,-2) {2.83};
        \end{scope}
      \end{scope}
      
    \end{tikzpicture}
    \caption{Adding new tasks to degenerate intrees with less than four nodes such that the resulting intrees are still degenerate. The original (2- resp. 3-node) intrees are drawn black, the newly added tasks are drawn white. Below each intree, we see the corresponding optimal expected run time. The lower the level of the newly added task, the lower the expected run time. This serves as basis for the induction proof for lemma \ref{lem:p3-adding-tasks-level-keep-scheduled-same-inequality}.}
    \label{fig:p3-lemma-adding-intrees-induction-start}
  \end{figure}

  If we add two tasks $z_1$ and $z_2$ with $level(z_1)>level(z_2)$ in a way such that the original ready tasks stay ready and the resulting intrees stay degenerate, we obtain the intrees depicted in figure \ref{fig:p3-lemma-adding-intrees-induction-start} (trees \emph{including} the white nodes). By simply computing the expected optimal run times, we can confirm our claim for intrees with 3 nodes.

  We now do the induction step by considering an intree with $n$ tasks. 
  Let $x,y$ be ready tasks and $z_1$ and $z_2$ to be added with $level(z_1) > level(z_2)$ such that the resulting intrees are degenerate.
  We can now compare the two runs that can occur if $x,y,z_1$ resp. $x,y,z_2$ are initially scheduled. 
  Therefore, we consider what happens in $I_1=I\cup\left\{ z_1 \right\}$ resp. $I_2=I\cup\left\{ z_2 \right\}$ if either $x$, $y$ or $z_1/z_2$ finishes first:

  \begin{itemize}
  \item If $z_1$ resp. $z_2$ is the first task to finish, the resulting intree is exactly $I$. Thus, the remaining run times for these cases are identical if the next task chosen is the same in both trees. We denote the task that may be chosen additionally to $x$ and $y$ by $z'$. If it is the case that only $x$ and $y$ can be scheduled, we set $z'=x$ to simplify notation. The corresponding run time for the resulting intree is then $T^*_{x,y,z'}(I)$.

  \item If $x$ is the first task to finish, then the resulting intrees are 
    \begin{equation*}
      I^x_{1}=I_1\setminus\left\{ x \right\} \quad \text{ resp. } \quad I^x_{2}=I_2\setminus\left\{ x \right\}.
    \end{equation*}

    By $x'$ we denote the task that is scheduled next in the optimal schedule for intree $I_1^x$. 
    If there are only two ready tasks left in $I_1^x$ (which then must be $y$ and $z_1$), we set $x'=y$. The expected optimal runtime for $I_1^x$ in this situaion is then given by $T_{x',y,z_1}^*(I_1^x)$.

    We now examine whether $x'$ is also ready in the intree $I_2^x$:
    \begin{itemize}
    \item If there are only two ready tasks left in $I_1$ (namely $x$ and $y$), we set -- as mentioned before -- $x'=y$. Thus, in $I_2^x$, $x'$ is still ready\footnote{It may even be the case that $I_2^x$ contains some additional ready tasks that are not ready in $I_1^x$.}.
    \item If $x'$ is the direct successor of $x$, then $x$ must have been the \emph{single topmost task} and the \emph{single predecessor} of $x'$ (since $I_1^x$ is a degenerate tree). However, since we assumed that $level(z_2)<level(z_1)$ and $z_1$ can not be a predecessor of $x'$ (since $x'$ is ready in $I_1$), it can not be the case that $z_2$ blocks $x'$ in $I_2^x$. We conclude that in this case $x'$ is ready in $I_2^x$.

      % Moreover, since $x'$ is ready in $I_1^x$, the task $z_1$ can not be a predecessor of $x'$. Since $level(z_2)<level(z_1)$ and since $I_1^x$ and $I_2^x$ are both degenerate intrees, $z_2$ can not block $x'$.\todo{Genauer ausführen. Evtl. auslagern.}
    \item If $x'$ is not the direct successor of $x$, we recognize the fact that $x'$ must reside on a certain level with in the degenerate intree. 

      If $x'$ is \emph{not} in the topmost level, it can not be blocked by $z_2$ because we assumed that $z_2$ is added in a way such that $I\cup\{z_2\}$ is still a degenerate intree.

      Otherwise (if $x'$ is a topmost task \todo{topmost definieren!}), $z_2$ can not be added \emph{above} $x'$ because we assumed $level(z_1)>level(z_2)$.

      Again, $x'$ is ready in $I_2^x$.
      % we still have two subcases:
      % \begin{itemize}
      % \item If there are other tasks at the same level as $x'$ (i.e. at the topmost level), then we can -- without loss of generality\footnote{Because of isomorphism.} -- assume that $z_2$ was added to another task on the same level as $x'$.
      % \item If $x'$ was the \emph{only} topmost task, then it \emph{might} be the case that $z_2$ was chosen in a way such that it is the direct predecessor of $x'$. In this case, we can argue that $level(z_1) > level(z_2)$ and, thus, know that $z_2$ can not block $x'$, because $x'$ is a topmost task.
      % \end{itemize}

      % $x'$ must be on a lower level then $x$ (because we are dealing with a degenerate intree). We assumed that we added $z_2$ in a manner such that $I_2=I\cup\left\{ z_2 \right\}$ is a degenerate subtree. Thus, $z_2$ could not have been added with $x'$ as its successor. Thus, $x'$ is not blocked by $z_2$ in $I_2$.
    \end{itemize}
    We observed that for $I_2^x$, the task $x'$ must be ready.
    
    The intrees $I^x_{1}$ and $I^x_{2}$ have exactly $n$ tasks -- and they have a common subtree, namely
    \begin{equation*}
      I^x := I^x_{1}\setminus\left\{ z_1 \right\}=I^x_{2}\setminus\left\{ z_2 \right\}.
    \end{equation*}

    
    We can now apply the induction hypothesis for the intree $I^x$, since this intree contains only $n-1$ tasks: We have an intree with $n-1$ tasks (namely $I^x$) and two tasks $z_1$ and $z_2$ with $level(z_1)>level(z_2)$, $I^x_1 = I^x\cup\left\{ z_1 \right\}$ and $I^x_2 = I^x\cup\left\{ z_2 \right\}$, implying that $T^*_{x',y,z_1}(I^x_1)>T^*_{x',y,z_2}(I^x_{2})$.
  \item If $y$ is the first task to finish, we argue similar to the $x$ case, thereby considering $I^y_{1},I^y_{2}$ and $I^y$ which are all defined analogously. This finally yields the following inequality: $T^*_{x,y',z_1}(I^y_{1}) > T^*_{x,y',z_2}(I^y_{2})$.
  \end{itemize}

  The above considerations are illustrated in figure 
\ref{fig:p3-adding-tasks-level-keep-scheduled-same-inequality}.\todo{Figure anpassen!}
  
  \begin{figure}[t]
    \centering
    \newcommand{\drawx}{
      \node[draw=black,circle] at (.9,3) {$x$};
    }
    \newcommand{\drawxx}{
      \node[draw=black,circle] at (.7,2) {$x'$};
    }
    \newcommand{\drawy}{
      \node[draw=black,circle] at (-1.5,4) {$y$};
    }
    \newcommand{\drawyy}{
      \node[draw=black,circle] at (-1.5,3) {$y'$};
    }
    \newcommand{\rawtriangle}{
      \draw
      [dotted, very thick,
      %fill=white!90!black, 
      rounded corners]
      (0,0) -- (3,2) -- (3.5,8) -- (-3.5,8) -- (-3,2) -- cycle;
    }
    \newcommand{\treetriangle}{
      \rawtriangle;
      \drawx;
      \drawy;
    }
    \newcommand{\abstand}{7.5cm}
    \begin{tikzpicture}[scale=.4]
      \begin{scope}[xshift=-\abstand]
        \treetriangle
        \node[draw=black,circle] at (0,6.5) {$z_1$};
      \end{scope}
      \begin{scope}[xshift=\abstand]
        \treetriangle
        \node[draw=black,circle] at (1,5) {$z_2$};
      \end{scope}
      
      \begin{scope}[xshift=-2*\abstand, yshift=-9cm]
        \rawtriangle
        \node[draw=black,circle] at (0,6.5) {$z_1$};
        \drawxx;
        \drawy;
      \end{scope}

      \begin{scope}[xshift=-\abstand, yshift=-9cm]
        \rawtriangle
        \node[draw=black,circle] at (0,6.5) {$z_1$};
        \drawx;
        \drawyy;
      \end{scope}

      \begin{scope}[xshift=0, yshift=-9cm]
        \treetriangle
        \node[draw=black,circle] at (.3,5.5) {$z'$};
      \end{scope}
      
      \begin{scope}[xshift=\abstand, yshift=-9cm]
        \rawtriangle
        \node[draw=black,circle] at (1,5) {$z_2$};
        \drawxx;
        \drawy;
      \end{scope}

      \begin{scope}[xshift=2*\abstand, yshift=-9cm]
        \rawtriangle
        \node[draw=black,circle] at (1,5) {$z_2$};
        \drawx;
        \drawyy;
      \end{scope}
      
      \draw[thick,->](-\abstand,0) -- +(0,-.8);
      \draw[thick,->](-\abstand-2cm,1) -- +(-\abstand+4cm,-1.6);
      \draw[thick,->](-\abstand+2cm,1) -- +(\abstand-4cm,-1.6);

      \begin{scope}[xshift=2*\abstand]
        \draw[thick,->](-\abstand,0) -- +(0,-.8);
        \draw[thick,->](-\abstand-2cm,1) -- +(-\abstand+4cm,-1.6);
        \draw[thick,->](-\abstand+2cm,1) -- +(\abstand-4cm,-1.6);
      \end{scope}
      
      % legend      
      \draw[decoration=brace,decorate=true](-11,8.25) --node[above]{$I_1$} +(7,0);
      \draw[decoration=brace,decorate=true](  4,8.25) --node[above]{$I_2$} +(7,0);
      \draw[decoration={brace,mirror},decorate=true](  -18,-9.5) --node[below, yshift=-.2cm]{$I^x_{1}$} +(6,0);
      \draw[decoration={brace,mirror},decorate=true](-10.5,-9.5) --node[below, yshift=-.2cm]{$I^y_{1}$} +(6,0);
      \draw[decoration={brace,mirror},decorate=true](   -3,-9.5) --node[below, yshift=-.2cm]{$I$} +(6,0);
      \draw[decoration={brace,mirror},decorate=true](  4.5,-9.5) --node[below, yshift=-.2cm]{$I^x_{2}$} +(6,0);
      \draw[decoration={brace,mirror},decorate=true](   12,-9.5) --node[below, yshift=-.2cm]{$I^y_{2}$} +(6,0);
    \end{tikzpicture}

    \caption{Proof sketch for lemma \ref{lem:p3-adding-tasks-level-keep-scheduled-same-inequality}. By induction hypothesis, we have that $T^*_{x',y,z_1}(I^x_{1}) > T^*_{x',y,z_2}(I^x_{2})$ and $T^*_{x,y',z_1}(I^y_{1}) > T^*_{x,y',z_2}(I^y_{2})$, from which we deduce $T^*_{x,y,z_1}(I_1) > T^*_{x,y,z_2}(I_2)$. Note that $z_2$ can not block $x'$ or $y'$ since $z_1$ didn't block any of the two and we required that adding $z_1$ resp. $z_2$ still results a degenerate tree.}
    \label{fig:p3-adding-tasks-level-keep-scheduled-same-inequality}
  \end{figure}
  
  Now we argue that the run times for $I_1$ and $I_2$ can be computed as follows:
  \begin{align*}
    T^*_{x,y,z_1}(I_1) & = 
      \frac{1}{3} + 
      \frac{1}{3}\cdot \left( 
        T_{x,y,z'}^*(I) + 
        T^*_{x',y,z_1}(I^x_{1}) +
        T^*_{x,y',z_1}(I^y_{1}) 
      \right)
      \\
    T^*_{x,y,z_2}(I_2) & = 
      \frac{1}{3} + 
      \frac{1}{3}\cdot \left( 
        T_{x,y,z'}^*(I) + 
        T^*_{x',y,z_2}(I^x_{2}) +
        T^*_{x,y',z_2}(I^y_{2}) 
      \right)
  \end{align*}

  We will now use the aforementioned inequalities $T^*_{x',y,z_1}(I^x_{1}) > T^*_{x',y,z_2}(I^x_{2})$ and $T^*_{x,y',z_1}(I^y_{1}) > T^*_{x,y',z_2}(I^y_{2})$:
\begin{align*}
    T^*_{x,y,z_1}(I_1) & = 
      \frac{1}{3} + 
      \frac{1}{3}\cdot \left( 
        T_{x,y,z'}^*(I) + 
        T^*_{x',y,z_1}(I^x_{1}) +
        T^*_{x,y',z_1}(I^y_{1}) 
      \right)
      > 
      \\
      & >
      \frac{1}{3} + 
      \frac{1}{3}\cdot \left( 
        T_{x,y,z'}^*(I) + 
        T^*_{x',y,z_2}(I^x_{2}) +
        T^*_{x,y',z_2}(I^y_{2}) 
      \right) 
      = T^*_{x,y,z_2}(I_2).
  \end{align*}

  This proves our claim for $level(z_1) > level(z_2)$. It is simple to obtain the claim for the version where $level(z_1) \geq level(z_2)$: If the levels are the same, the resulting degenerate intrees $I_1$ and $I_2$ are isomorphic (and there is an isomorphism that maps $z_1$ onto $z_2$), thus the optimal run time is the same. If the levels are different, we argument as in the beginning of the proof.
\end{proof}

\paragraph{Application of lemma \ref{lem:p3-adding-tasks-level-keep-scheduled-same-inequality}}

We presented lemma \ref{lem:p3-adding-tasks-level-keep-scheduled-same-inequality} in a way that involved adding tasks to a certain intree, because this related well to our proof technique. In practice, we can use it to compare two degenerate intrees with $n$ nodes and that have a common subtree containing $n-1$ nodes.

We can now derive the following theorem.

\begin{theorem}
  Degenerate intrees are optimally scheduled by HLF.
\end{theorem}

\begin{proof}
  Consider a degenerate intree with $n=3$ tasks. It is trivially clear that for P3, a HLF schedule is optimal for this intree (see figure \ref{fig:p3-lemma-adding-intrees-induction-start} to see what these intrees look like -- we simply can conclude examine all possible schedules and see that the optimal ones is exactly HLF).
  
  Consider now a degenerate tree $I$ with $n$ nodes and assume that we know that for all degenerate trees with $n-1$ nodes HLF is optimal for three processors. If $I$ has two or less topmost tasks, it is obvious that we have to use HLF (since we can use at most two processors and HLF is known to be optimal for two processors \todo{Referenz und Beweis in P2-Kapitel}).

  Thus, we only have to focus on the case where $I$ has at least three ready tasks.

  If we choose three topmost tasks $x,y,z$ of $I$, we can argue as follows: If $x$ is the task that finishes first, for the resulting subtree $I\setminus \left\{ x \right\}$, we can be sure that we can choose the next task such that we adhere to HLF, thus choosing the optimal solution for $I\setminus\left\{ x \right\}$ (by induction hypothesis). The same holds if $y$ or $z$ finishes first.
  
  We now consider a \todo{(possibly?) }non-HLF schedule for $I$ and compare it to the HLF schedule described before.
  Let $x',y',z'$ be the tasks to be chosen such that at least $x\neq x'$ or $y\neq y'$ or $z\neq z'$. We can -- without loss of generality -- assume $level(x')\leq level(x)$, $level(y')\leq level(y')$, $level(z')\leq level(z)$. If $x'$ is the first task to finish, we consider the degenerate intree $I\setminus\left\{ x' \right\}$. Since $I \setminus \left\{ x' \right\}$ is a degenerate intree with $n-1$ tasks, it would be optimal to use HLF. However, using HLF may or may not be possible depending on our previous choices of $y'$ and $z'$. That is, the optimum for $I\setminus\left\{ x' \right\}$ \emph{might} be acchieved if we chose $y'$ and $z'$ accordingly. From this we can conclude that the optimal expected run time is at least $T_{HLF}\left( I\setminus\left\{ x' \right\} \right)$, where $T_{HLF}$ denotes the run time for HLF (which we know, by induction, is optimal).

  Now we compare the run time for $I\setminus\left\{ x' \right\}$ to the optimal run time for $I \setminus\left\{ x \right\}$, which is exactly given by $T_{HLF}\left( I \setminus\left\{ x \right\} \right)$. We recognize that $I\setminus\left\{ x' \right\}$ and $I\setminus\left\{ x \right\}$ have a common subtree, namely $I\setminus\left\{ x,x' \right\}$ with $n-2$ tasks.
  
  Moreover, we know that for both $I\setminus\left\{ x \right\}$ and $I\setminus\left\{ x' \right\}$ HLF is optimal (because of our induction hypothesis) and that for $I\setminus\left\{ x \right\}$ and $I\setminus\left\{ x' \right\}$ both $y$ and $z$ must be in the optimal (i.e. HLF) schedule since $y$ and $z$ are two of the three topmost tasks. At last, we have $level(x) \geq level(x')$.
  Thus, we can apply lemma \ref{lem:p3-adding-tasks-level-keep-scheduled-same-inequality} for the degenerate tree $I\setminus \{x,x'\}$ with tasks $y$ and $z$ and deduce that
  \begin{equation*}
    T_{x',y,z}(I\setminus\{x\}) 
    \leq 
    \underbrace{T_{x,y,z}(I\setminus\{x'\})}_{\text{Optimal for $I\setminus \{x'\}$}}.
  \end{equation*}
  Equality holds if $x$ and $x'$ are on the same level.

  As stated before, for the intree $I\setminus\{x'\}$, the schedule chosen \emph{might} be a non-HLF schedule. In this case, the schedule performs even worse than $T_{x,y,z}(I\setminus\{ x'  \})$, since the optimal schedule would choose $x,y,z$ as scheduled tasks. This means that
  \begin{equation*}
    T_{x,y,z}(I\setminus\{x'\})
    \leq
    T_{x'',y',z'}(I\setminus\{x'\}),
  \end{equation*}
  where $x''$ is the task chosen by the non-HLF schedule, finally implying (by $T_{HLF}(I\setminus\{x\}) \leq T_{x',y,z}\left( I\setminus\left\{ x \right\} \right)$) that
  \begin{equation*}
    T_{HLF}(I\setminus\{x\})
    \leq
    T_{x'',y',z'}(I\setminus\{x'\}).
  \end{equation*}
  %because the HLF schedule is at least as good as the schedule that starts out with tasks $x',y,z$.
  We argue similar for tasks $y$ and $y'$ resp. $z$ and $z'$ and finally obtain the following:

  \begin{eqnarray*}
    T_{x',y,z}(I\setminus\{x\})
    & \leq &
    T_{x'',y',z'}(I\setminus\{x'\}) \\
    T_{x,y',z}(I\setminus\{y\})
    & \leq &
    T_{x',y'',z'}(I\setminus\{y'\}) \\
    T_{x,y,z'}(I\setminus\{z\})
    & \leq &
    T_{x',y',z''}(I\setminus\{z'\}) \\
  \end{eqnarray*}
  Note that if e.g. $x=x'$, the above three equations are still satisfied.
  Combining the above inequalities yields that
  \begin{eqnarray*}
    %T_{HLF}(I) = 
    T_{x,y,z}(I) 
    & \leq & 
    \frac{1}{3} + \frac{1}{3} \cdot 
    \left( 
      T_{x',y,z}(I\setminus\{x\}) +
      T_{x,y',z}(I\setminus\{y\}) +
      T_{x,y,z'}(I\setminus\{z\})
    \right) 
    \leq \\
    & \leq &
    \frac{1}{3} + \frac{1}{3} \cdot 
    \left( 
      T_{x'',y',z'}(I\setminus\{x'\}) +
      T_{x',y'',z'}(I\setminus\{y'\}) +
      T_{x',y',z''}(I\setminus\{z'\})
    \right) \leq \\
    & \leq &
    T_{x',y',z'}(I)
  \end{eqnarray*}

  This finally shows that a HLF schedule is at least as good as an arbitrary other task, meaning that HLF is optimal for three processors on degenerate trees.
\end{proof}

\section{Parallel chains}
\label{sec:p3-parallel-chains}

We will now consider another class of intrees that can be shown to have HLF as optimal schedules.

\begin{definition}[Parallel chain]
  Let $I$ be an intree. We call $I$ a \emph{parallel chain}, if each task except the root has at most one predecessor. The root may have arbitrarily many predecessors.
\end{definition}

\todo{Figure von einer parallel chain.}

\begin{lemma}
  \label{lem:parallel-chains-chain-switching-higher-is-worse}
  Let $I$ be a parallel chain with three ready tasks $x,y,z$. Such that $level(x)>level(y)$. Let $x'$ resp. $y'$ be two new nodes that are added as predecessors of $x$ resp. $y$. Then, $T_{x',y,z}(I\cup\left\{ x' \right\}) > T_{x,y',z}(I\cup\left\{ y' \right\})$. \todo{T-Notation mit Index mal anständig definieren!}
\end{lemma}

\begin{proof}
  We consider the parallel chain $I\cup\left\{ x' \right\}$ and its optimal schedule. We then construct a schedule for $I\cup\left\{ y' \right\}$ whose run time is less than the optimal run time for $I\cup\left\{ x' \right\}$.
  
  We observe that in $I\cup\left\{ x' \right\}$ either $x'$, $y$ or $z$ finishes first. For each of these cases we now show that we can construct a corresponding schedule in $I\cup\left\{ y' \right\}$ such that the corresponding run time in $I\cup\left\{ y' \right\}$ is at most the one in $I\cup\left\{ x' \right\}$.

  \begin{itemize}
  \item If $x'$ finishes first in $I\cup\left\{ x' \right\}$, then obtain the same intree like if $y'$ finishes first in $I\cup\left\{ y' \right\}$. Thus, we can assume the two resulting schedules to have the same run time.
  \item If $z$ finishes first, we can argue by induction since the trees
    \begin{equation*}
      I\cup\left\{ x' \right\}\setminus\left\{ z \right\}
      \quad
      \text{ and }
      \quad 
      I\cup\left\{ y' \right\}\setminus\left\{ z \right\}
    \end{equation*}
    both have $n-1$ tasks and can be obtained by attaching $x'$ resp. $y'$ to the parallel chain $I\setminus\left\{ z \right\}$. We then know -- by induction -- that $T_{x',y,z}(I\cup\left\{ x' \right\}\setminus\left\{ z \right\}) > T_{x,y',z}(I\cup\left\{ y' \right\}\setminus\left\{ z \right\})$.
  \item It remains to explain what to do if $y$ finishes in $I\cup\left\{ x' \right\}$ compared to if $x$ finishes first in $I\cup\left\{ y' \right\}$.
    
    The two respective intrees then are
    \begin{equation*}
      I\cup\left\{ x' \right\}\setminus\left\{ y \right\} 
      \quad
      \text{ and }
      \quad
      I\cup\left\{ y' \right\}\setminus\left\{ x \right\}.
    \end{equation*}
    
    By induction, we know that
    \begin{equation*}
      T_{x', z}(I\cup\left\{ x' \right\}\setminus\left\{ y \right\})
      >
    \end{equation*}
  \end{itemize}
\end{proof}

\begin{lemma}
  \label{lem:parallel-chains-adding-tasks-level-comparison}
  Let $I$ be a parallel chain and $x, y$ two (not necessarily distinct) redy tasks within $I$. Let $z_1$ and $z_2$ be two new tasks to be added to $I$ with $level(z_1) > level(z_2)$ and $I_1=I\cup\left\{ z_1 \right\}, I_2=I\cup\left\{ z_2 \right\}$.
  
  Then, $T_{x,y,z_1}(I_1) > T_{x,y,z_2}(I_2)$, where $T_{a,b,c}{I}$ denotes the optimal run time that can be acchieved if $a,b,c$ are chosen as initially scheduled tasks.
\end{lemma}

\begin{proof}
  The proof is very similar to the one for lemma \ref{lem:p3-adding-tasks-level-keep-scheduled-same-inequality}. We start by examining small chain trees (with four or less tasks --- see figure \ref{fig:p3-lemma-adding-intrees-induction-start}) and confirm our claim by simply examining the execution times.
  
  We now consider a parallel chain $I$ with $n$ tasks, and tasks $x,y$ and $z_1, z_2$ as specified in the requirements for lemma \ref{lem:parallel-chains-adding-tasks-level-comparison}. We now compare the run times $T_{x,y,z_1}(I_1)$ and $T_{x,y,z_2}(I_2)$, where $I_1=I\cup\left\{ z_1 \right\}$ and $I_2=I\cup\left\{z_2  \right\}$.

  We do a case distinction on which task is to finish first (and compare the situation for $I_1$ and $I_2$):
  \begin{itemize}
  \item If $z_1$ resp. $z_2$ is the first task to finish, we can argue that the run time for the resulting trees can be the same if we choose -- coming from $I_2$ -- the same task that optimally would be chosen coming from $I_1$ (if coming from $I_1$ takes no task at all, we can also do so for $I_2$). 
  \item If $x$ is the first task to finish, we denote by $x'$ the task that would be optimally chosen if we come from $I_1$. We now distinguish:\todo{Was, wenn kein $x'$ gewählt werden kann?}
    \begin{itemize}
    \item If $x'$  is the direct successor of $x$, then $x'$ can only be blocked by $z_2$ if $x'$ is the root. In this case, however, $x'$ would have been blocked by $z_1$ which leads to a contradiction since $x'$ is ready.

      So, we can assume that $x'$ is \emph{not} the root. Thus, we can argue that $x'$ \emph{must} be the end of some chain in $I_1\setminus\left\{ x \right\}$, and can not be blocked by $z_2$. So, $x'$ is ready in $I_2\setminus\left\{ x \right\}$ as well.
    \item If $x'$ is not the direct successor of $x$, then $x'$ can not be the root and $x'$ can be blocked by $z_2$ in $I_2$.
      \begin{itemize}
      \item If $x'$ is \emph{not} blocked by $z_2$, then it is ready in $I_2$.
      \item If $x'$ is exactly the successor of $z_2$, then we 
      \end{itemize}

    \end{itemize}

  \end{itemize}

\end{proof}

%%% Local Variables:
%%% TeX-master: "../thesis.tex"
%%% End: 

%\part{Implementation details}
\chapter{Additional algorithms}
\label{sec:additional-algs}

\newcommand{\treegeq}[1][X]{\stackrel{\text{#1}}{\geq}}

\section{Computing equivalent snapshots}
\label{sec:algorithm-equivalent-snapshot}

As mentioned in section \ref{sec:intro-first-glance-schedules}, it is possible to combine certain snapshots into one single snapshot, thereby avoiding redundant computations. The requirements for two snapshots being equivalent have been discussed there.

It is a notable fact that we can determine in polynomial time (w.r.t. the number of nodes) whether two snapshots are equivalent. This computation involves foremost a check whether the two corresponding intrees are isomorphic. 

%While it seems to be quite hard to check isomorphism for general graphs (\todo{unbedingt Referenz!}), it is feasible for intrees.

While there currently is no known algorithm that checks isomorphism in polynomial time \emph{for general graphs} (see \cite{arora2009computational}), there is a simple algorithm for isomorphism of intrees. The idea behind this algorithm is to recursively sort the predecessors of a task according to the number of their respective predecessors. An early description of this algorithm can be found e.g. in \cite{aho1974design}. We can easily adapt this isomorphism check to our needs in the sense that we can construct an algorithm that constructs a ``canonical snapshot'', and all equivalent snapshots are converted to exactly this canonical snapshot.

One thing to keep in mind is that we have to explicitly take care of the currently scheduled tasks, i.e. we have to adopt the algorithm to distinguish between scheduled and unscheduled tasks.

\subsection{The algorithm}
\label{sec:algorithm-equiv-snapshots-actual-algo}

The algorithm basically relies on an ordering of intrees (there are of course many possible, but we use a simple one).

\begin{definition}[Ordering of intrees with preferred tasks]
  We introduce an ordering denoted by $\treegeq$. For two intrees $I_1$ and $I_2$ and a set $X$ of tasks, we define $I_1 \treegeq I_2$ inductively. 
  \begin{itemize}
  \item If both $I_1$ and $I_2$ consist only of a root, we have $I_1 \treegeq I_2$ if and only if the root of $I_1$ is in $X$ and the root of $I_2$ is not in $X$.
  \item If the root of $I_1$ has less predecessors than the root of $I_2$, then $I_1 \treegeq I_2$.
  \item If the root of $I_1$ has the same number $r$ of predecessors as the root of $I_2$, we sort the corresponding predecessors $p_1,\dots,p_r$ resp. $q_1,\dots,q_r$ (in $I_1$ resp. $I_2$ according to $\treegeq$). Then, if $p_i \treegeq q_i \forall i\in\{1,2,\dots,r \}$, we have $I_1 \treegeq I_2$.
  \end{itemize}
\end{definition}

\todo{Examples!}

The algorithm recursively sorts the tasks in the intree according to the ordering $\treegeq$. It is shown in algorithm \ref{alg:compute-canonical-snapshot}.

\begin{algorithm}
  \begin{algorithmic}
    \Procedure{CanonicalSnapshot}{$s$} \Comment{Returns the canonical snapshot for snapshot $s$}
    \State $t \gets s.intree$ \Comment{Retrieve root of intree}
    \State $X \gets s.scheduled$ \Comment{Retrieve scheduled tasks}
    \State \textbf{return} \Call{CanonicalIntree}{$t, X$} 
    \EndProcedure
    \Statex
    \Procedure{CanonicalIntree}{$t, X$}\Comment{$t$: a (sub)tree, $X$: set of scheduled tasks}
    \State $r \gets t.root$ \Comment{Retrieve root of subtree}
    \State $CanonicalPredecessors \gets 
           \left\{ \Call{CanonicalIntree}{c, X} \mid c \in r.predecessors \right\}$
    \State \textbf{return} root with predecessors in $CanonicalPredecessors$ in sorted order according to $\treegeq$
    \EndProcedure
    \Statex
  \end{algorithmic}
  \caption{Computing canonical snapshots for a snapshot $s$ containing the corresponding intree and the tasks that are currently scheduled (as defined in section \ref{sec:processing-an-intree-of-tasks}).\todo{Algorithmus verbessern.}}
  \label{alg:compute-canonical-snapshot}
\end{algorithm}

\todo{Algorithmus, der equiv. snaps / bzw. canonical snaps berechnet zeigen.}\todo{Laufzeit!}

\subsection{Additional approaches}
\label{sec:algorithm-canonical-snap-additional-approaches}

Unfortunately, computing canonical snapshots has a major impact on the overal performance of our program, which is why we tried other approaches to tackle this problem. For completeness, we will explain them shortly. Unfortunately, they did not work out as good as expected, but maybe they are helpful for future work. In this section, we focus on the part concerning the computation of \emph{canonical intrees} (i.e. we do not distinguish between scheduled and non-scheuled leaves).

\begin{description}
\item[Tree sequence] Most of the time, we are generating a certain subtree and want to know the canonical intree of it. We tried exploiting the fact that we already could \emph{start} with a canonical intree. That is, we tried to construct the tree sequence of a canonical subtree by just examining the tree sequence of the original intree. Sadly, we could not devise a simple pattern that works as a general rule.\todo{Example where equiv intrees yield drastically different sequences.}
\item[Matula numbers] Intrees and their encodings have of course already been subject to research. One example is shown in \cite{matula1968natural}, where a (more or less) natural bijection between intrees and natural numbers, the so-called \emph{Matula numbers}, is shown.\todo{Short description.} While this description could lead to very elegant algorithms, it does not come in handy in practice because the matula numbers can be very big (more than 32 bit needed for matula numbers of 15-node intrees).
\end{description}

\section{Enumerating all intrees with a certain number of nodes}
\label{sec:enumerating-all-intrees}

It is clear that the number of intrees (more precisely, the number of unlabelled rooted trees) with exactly $n$ nodes is exponential in $n$ \todo{proof}. However, for experimental purposes, it is convenient to have an algorithm that is capable of enumerating all these intrees. The main thing that should be kept in mind is that we do \emph{not} generate isomorphic intrees over and over again.

We now show an algorithm to generate \emph{all} intrees with a certain number of nodes (called $n$) up to isomorphism. This algorithm is based on the following two facts: 

\begin{itemize}
  \item The overall root can have any amount of children between 1 and $n-1$. If it has only 1 child, the corresponding predecessor intree must contain exactly $n-1$ vertices. If it has exactly $n-1$ children, each predecessor intree contains exactly 1 vertex. All the cases in between admit several possibilities \todo{Worth stating exact stuff here?}.
  \item If the overall root of the intree with $n$ verices has exactly $r$ predecessors (with $r \in \left\{ 1,2,\dots,n-1 \right\}$, as stated before), then the sum of the vertices with in the predecessor intrees is exactly $n-1$. Moreover, let us denote the predecessor intrees by $T_1,T_2,\dots,T_r$ and call $n_i$ the number of vertices in predecessor intree $T_i$ for all $i\in\left\{1,2,\dots,r \right\}$. Without loss of generality, we can assume $1 \leq n_1 \leq n_2 \leq n_3 \leq \dots \leq n_r$\todo{How many of these \emph{partitions} are there?}.
\end{itemize}

We can exploit these two facts to construct a recursive algorithm which is described in algorithm \ref{alg:generate-intrees}. This algorithm enumerates all intrees with exactly $n$ vertices. It does so by traversing all tuples $(n_1,\dots,n_r)$ fulfilling
\begin{equation*}
  n_1 + n_2 + \dots + n_r = n-1 \quad \text{ and } \quad 1\leq n_1\leq n_2\leq\dots\leq n_r.
\end{equation*}
It then generates all combinations of predecessor intrees $(p_1,\dots,p_r)$ whose respective number of nodes are $n_1,\dots,n_r$. The algorithm thereby omits duplicate combinations. This can easily be acchieved by defining an order $\left(\treegeq\right)$ on intrees as follows ($t_1$ and $t_2$ being two intrees):

\begin{equation}
  \label{eq:definition-treegeq}
  t_1 \treegeq t_2 \equiv (\text{$t_1$ has more vertices than $t_2$}) \vee \exists k \in \left\{ 1,2,\dots,r \right\}. \left( p_{1,k} \treegeq p_{2,k} \wedge \forall i<k. p_{1,i}=p_{2,i} \right)
\end{equation}


\begin{algorithm}
  \begin{algorithmic}
    \Procedure{GenerateIntrees}{$n$} \Comment{Returns the set of all intrees with exactly $n$ vertices}
      \If{$n=1$} 
        \State \textbf{return} $\left\{ \tikz{\fill(0,0) circle (0.1cm);} \right\}$ \Comment{Base case: Intree with just 1 vertex}
      \EndIf
      \State $R \gets \left\{  \right\}$ \Comment{Variable for result}
      \For{$(n_1,\dots,n_r)
            \in 
            \left\{ (n_1,\dots,n_r) \in \naturals^r \mid 
              1 \leq r < n \wedge
              1 \leq n_1 \leq n_2 \leq \dots \leq n_r
            \right\}$}
        \State $P \gets$ (\Call{GenerateIntrees}{$n_1$},\dots,\Call{GenerateIntrees}{$n_r$}) \Comment{Predecessor intrees}
        \For{$(p_1,\dots,p_r) \in P[1] \times P[2] \times \dots \times P[r]$}
          \If{$p_1 \treegeq p_2 \treegeq \dots \treegeq p_r$} \Comment{No duplicates}
          \State $R \gets R \cup \Call{CombinePredecessorIntrees}{p_1,\dots,p_r}$
          \EndIf
        \EndFor
      \EndFor
      \State \textbf{return} $R$
    \EndProcedure
    \Statex
    \Procedure{CombinePredecessorIntrees}{$p_1,\dots,p_r$}
    \State \textbf{return} $\left\{
      \tikz[baseline=(current bounding box.center)]{
        \fill (-0.5,0)circle(0.1cm);
        \node(root) at (-0.5,0){};
        \node[circle](1) at (-2,1) {$p_1$};
        \node[circle](2) at (-1,1) {$p_2$};
        \node[circle](p) at (0,1) {...};
        \node[circle](r) at (1,1) {$p_r$};
        \draw[-] (1) -- (root);
        \draw[-] (2) -- (root);
        \draw[-] (r) -- (root);
      }
      \right\} $
      \Comment{New root with predecessor intrees $p_1,\dots,p_r$}
    \EndProcedure
  \end{algorithmic}
  \caption{Generating all intrees up to isomorphism}
  \label{alg:generate-intrees}
\end{algorithm}


%%% Local Variables:
%%% TeX-master: "../thesis.tex"
%%% End: 
\chapter{Benchmarks}
\label{chap:benchmarks}

While chapter \ref{chap:p3} will focus on more theoretical aspects of our problem, we first summarize in short how the program performs in practice. We therefore ran tests that considered intrees with a certain number of tasks.

\emph{Remark:} It may be the case that the program (and in particular, the data shown in the tables) changed since this thesis was finished.

\section{Canonical vs. non-canonical variant}
\label{sec:benchmark-canonical-vs-non-canonical}

First we compare how the program performs with resp. without the elimination of equivalent snapshots (see section \ref{sec:intro-first-glance-schedules}). We therefore computed the LEAF schedules for all non-trivial intrees with a certain number of tasks and measured the run times. Table \ref{tab:comparison-canonical-vs-non-canonical} shows the results (we did not measure up to milliseconds but only seconds since the results speak for themselves and need not be more accurate). This table was created using a Intel Core2 with 2.13GHz and 2Gb of RAM. For 13 tasks, we had to stop the variant that did not exclude equivalent snapshots because it needed too much memory. 

\begin{table}[th]
  \centering
  \begin{tabular}[ht]{lcccccc}
    Tasks                         & $\leq 8$    & 9 & 10 & 11 & 12 & 13 \\
    \hline
    Equivalent snapshots included & $\leq 0.2s$ & $\approx 0.5s$ & $3s$ & $15s$ & $80s$ & $>360s$ \\
    Equivalent snapshots excluded & $\leq 0.2s$ & $\approx 0.3s$ & $2s$ & $6s$ & $20s$ & $75s$
  \end{tabular}
  \caption{Run time comparison with vs. without elimination of equivalent snapshots.}
  \label{tab:comparison-canonical-vs-non-canonical}
\end{table}

Even more important is the memory consumption: While the variant \emph{with} canonical snapshots used roughly 30mb of memory to process all non-trivial intrees with 12 tasks, we needed over 550mb when we did \emph{not} exclude equivalent snapshots.

We observe that excluding equivalent snapshots results both in a remarkable speedup and in a considerably smaller memory footprint. For this reason, we quite quickly decided to focus on the variant that excludes equivalent snapshots. All following benchmarks are done with equivalent snapshots excluded.

\section{Keeping all intrees in memory}
\label{sec:benchmarks-all-intrees-in-memory}

We now research how much time it takes to compute optimal schedules for intrees with a certain size.

As a first benchmark, we computed optimal schedules for all non-trivial intrees (with a certain amount of tasks) ``in one run'', i.e. we generated all non-trivial intrees, kept them in main memory and computed the optimal schedule for each of these intrees. This technique has the advantage that intermediate results can be re-used and do not need to be recomputed over and over again. On the other hand, keeping that many intrees in RAM leads to enormous memory consumption and, thus, was only done for up to 15 intrees.

Table \ref{tab:time-benchmark} shows the results. This table was created on a reasonably modern machine (with an Intel Core i7). Peak memory consumption was measured with Valgrind for up to 13 tasks. According to the valgrind manual \cite{massifmanual}, these measurements are accurate within 1\%. For more than 13 tasks, Valgrind was too slow, so we went for a more simplistic approach: We simply observed the stats shown by htop. This means, that the memory consumption for 14 and 15 tasks is not \emph{that} accurate, but probably still accurate enough to get an impression of how many memory was needed.

Moreover, we carried out these tests with simple floating point numbers to represent probabilities and expected run times. We also implemented another feature to support fractions --- see section \ref{sec:benchmarks-myfloat-variations} for more details.

\begin{table}[ht]
  \centering
  \begin{tabular}[ht]{ccccc}
    Tasks & Intrees & Snapshots & Time & Memory \\
    \hline{}
    $\leq 9$ & $\leq$ 171 & $\leq$ 891 & $\leq$ 0.5s & $\leq$ 1451040b \\
    10 & 433 & 3004 & 1s & 4941576b \\
    11 & 1123 & 10143 & 4s & 16847304b \\
    12 & 2924 & 34065 & 6s & 57016592b \\
    13 & 7720 & 113492 & 26s & 193781984b \\
    14 & 20487 & 375088 & 2m10s & $\approx$ 410mb \\
    15 & 54838 & 1230391 & 7m & $\approx$ 1.5Gb \\
    
  \end{tabular}
  \caption{Time needed to compute optimal schedules for all non-trivial intrees with a certain number of tasks using a floating-point representation for probabilities and run times.}
  \label{tab:time-benchmark}
\end{table}

The memory consumption might look quite high at first glance, but if we look closer, it becomes clear that quite a lot of things have to be stored. We have to store each snapshot containing the current intree, the scheduled tasks, and its successors. Moreover, we need a ``pool'' of snapshots that is used to avoid over and over recomputing equivalent snapshots. Moreover, the statistics in table \ref{tab:time-benchmark} of course include the memory that is consumed by e.g. auxiliary data structures from the C++ container classes.

The main problem is -- of course -- that the number of non-trivial intrees drastically grows as the number of tasks increases (1,1,2,2,5,11,28,67,171,433,1123,\dots --- see \cite{oeisnumbernontrivialintrees}). For that reason, we continue benchmarking by computing optimal schedules for single intrees (or a small set of intrees).

\section{Grouping intrees and computing them one after another}
\label{sec:benchmarks-clustered-intrees}

The next benchmark we conducted was done for intrees with 15 or more tasks. First, we generated all (nontrivial) intrees with a certain number of tasks. Afterwards, we split the whole collection of intrees into smaller chunks containing a certain amount of intrees. We then processed these chunks one after another. 

The results for this setting are shown in table \ref{tab:benchmark-clustered-run-time}.

In this scenario, the measurements become more difficult to interpret. Especially, the number of snapshots and the amount of memory can not be directly compared to the values shown in section \ref{sec:benchmarks-all-intrees-in-memory}. For the time, it is also non-trivial. This is because we carry out some computations twice. While we can measure the time quite accurately, we can not do this for the memory consumption because Valgrind slowed down the program too much to finish in reasonable time. This is why we used \texttt{htop} to estimate the amount of memory used for 15 and 16 tasks. We did not exactly measure the memory consumption for 17 or more tasks, but it can be said that we experienced no problems on a machine with 8Gb RAM.

\begin{table}[ht]
  \centering
  \begin{tabular}[ht]{ccccc}
    Tasks & Intrees & Chunk size & Time & Memory per chunk \\
    \hline
    15 & 54838 & 5000 & 11m & $<$ 500mb \\
    16 & 147570 & 5000 & 1h & $<$ 800mb \\
    17 & 399466 & 5000 & 3h & n.a. \\
    18 & 1086312 & 6000 & 13.3h & n.a. \\
    19 & 2967517 & 6000 & 24h & n.a.
  \end{tabular}
  \caption{Time needed to compute optimal schedules for all non-trivial intrees  with a certain number using sets of subtrees of tasks using floating point numbers for probabilities and run times.}
  \label{tab:benchmark-clustered-run-time}
\end{table}

One interesting fact that can be observed from table \ref{tab:benchmark-clustered-run-time} is that for 15 tasks, we needed roughly 11 minutes, while in the non-grouped case, it took only about the half of this time (about 6 minutes --- see table \ref{tab:time-benchmark}). That is, as expected, grouping the intrees into smaller chunks considerably increases the run time, but -- on the other hand -- seems to be the only real alternative since the number of trees grows that fast so they do not fit into main memory anymore.

As you can see, the time needed to compute optimal schedules for all non-trivial intrees increases very fast with growing number of tasks. Of course, the run time can be reduced by parallelizing the computations for example in a way that computes different chunks in parallel.

\section{Other representations for probabilities and expected values}
\label{sec:benchmarks-myfloat-variations}

As said before, the user can choose between different representations for probailities and expected values. By default, we simply use C++'s \texttt{float}s but for some scenarios it might be useful to have precise fractions. We implemented those using Boost Rational Number library or GNU Multiple Precision Arithmetic Library (GMP).

We compare the performance of the program with different representations, namely the default \texttt{float} representation, the representation by Boost's Rational Number library with \texttt{unsigned long long}s as denominators and numerators, and representationn by GMP. The results are shown in table \ref{tab:comparison-myfloat-variants} (generated on an Intel Core2 with 2.13GHz).

\begin{table}[th]
  \centering
  \begin{tabular}[ht]{lccccc}
    Tasks & 9 & 10 & 11 & 12 & 13 \\
    \hline 
    \texttt{float} & $\leq 0.3s$ & $2s$ & $6s$ & $20s$ & $75s$ \\
    Boost Raionals & $\leq 0.3$ & $2s$ & $6s$ & $21s$ & $78s$ \\
    GMP & $\leq 0.6s$ & $3s$ & $10s$ & $38s$ & $140s$
  \end{tabular}
  \caption{Comparison of different representations for probabilities and expected values.}
  \label{tab:comparison-myfloat-variants}
\end{table}

We see that Boost's Rational Number library does not considerably increase run time, while GMP almost doubles it. This may come from the fact that GMP uses ``arbitrarily large'' integers as denominators and numerators to represent fractions, thus allowing arbitrary precision (within memory limits), while we can adjust which underlying representation for numerators and denominators Boost should use. Since we are using common \texttt{unsigned long long}s, the operations on these are no very expensive, while the GMP counterparts might be due to overflow checks, dynamic memory allocation for larger numbers, etc. However, \emph{only} GMP offers -- out of the box -- arbitrary precision, while Boost's Rational Number library might yield wrong results if the denominators and numerators get too big for \texttt{unsigned long long}s.

Surprisingly, using Boost's Rationals did not increase memory consumption considerably. On the other hand, GMP also increased the memory consumption by roughly 40\%. This again comes possibly from the fact that GMP offers arbitrary precision, thereby requiring to store arbitrarily large numbers and requiring auxiliary data structures.

Finally, we observer that using rational numbers does not increase run time as much as not eliminating equivalent snapshots, but the additional amount of time with GMP is considerably large -- as a tradeoff we get arbitrary precision.

\emph{Remark:} We almost never ran into problems when we used simple \texttt{float}s to represent probabilities. Only in rare cases we had to rely on GMP because of rounding errors that possibly gave us incorrect results. For this reason, we decided that the default representation shall use \texttt{float}s.

%%% Local Variables:
%%% TeX-master: "../thesis.tex"
%%% End: 

\chapter{First thoughts on implementation}
\label{chap:first-thoughts-on-implementation}

\section{Configuration DAG}
\label{sec:configuration-dag}

Initially we are dealing with an intree (i.e. each node has at most one successor) of tasks that have to be processed by a certain number of processors.

We will call the set of \emph{all} tasks $\alltasks$. If task $t_2$ can only be executed if $t_1$ already has been processed, we write $t_1 \neededfor t_2$. Moreover, we introduce a shorthand notation that allows us to ``chain'' several of these symbols: If there exist tasks $s_1,\dots,s_m$ ($m\in\mathbb N$), we write $t_1 \neededfor* t_2$ if we have $t_1 \neededfor s_1$ and $s_1 \neededfor s_2, s_2 \neededfor s_3, \dots, s_{m-1} \neededfor s_m$ and $s_m \neededfor t_2$ or if $t_1\neededfor t_2$ or if $t_1=t_2$.

\begin{definition}
  Let $\alltasks$ be a set of tasks, and $T \subseteq \alltasks$. We call $T$ an intree (of tasks) if there is one designated task $t_0\in\alltasks$ such that the following two conditions hold:
  \begin{eqnarray*}
    \forall  t \in T. & \quad t \neededfor* t_0 \\
    \forall  t \in T. & \quad t\neededfor s \Rightarrow s\in T
  \end{eqnarray*}
\end{definition}

\begin{definition}
  Let $T$ be an intree of tasks. Let $M\subseteq\alltasks$ be a set of tasks such that the following two conditions hold:
  \begin{itemize}
  \item $\forall t\in M.\, t \in T$
  \item $\forall t\in M.\, \nexists u \in T.\, u\neededfor t $
  \end{itemize}
  We then call the tuple $\left( T, M \right)$ a \emph{configuration}.
\end{definition}

%%% Local Variables:
%%% TeX-master: "../thesis.tex"
%%% End: 

\appendix{}

%\include{counterexamples/counterexamples}

%\bibliographystyle{alphabetic}
\nocite{*}
\printbibliography[notkeyword=oeis,title={Bibliography}]
%\printbibliography[title={Number sequences},keyword=oeis]

\end{document}
