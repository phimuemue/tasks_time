%%% thesis.tex --- 

%% Author: philipp@pmpc
%% Version: $Id: thesis.tex,v 0.0 2013/04/08 12:19:13 philipp Exp$

%%\revision$Header: /home/philipp/Documents/Uni/masterarbeit/thesis/thesis.tex,v 0.0 2013/04/08 12:19:13 philipp Exp$

\documentclass[letter]{report}

\usepackage[english]{babel}
\usepackage{lmodern}
\usepackage[utf8]{inputenc}
\usepackage{hyperref}
\usepackage{amsmath}
\usepackage{amssymb}
\usepackage{amsthm}
\usepackage{graphicx}
\usepackage{tikz}
\usepackage{suffix}
\usepackage[left=3.5cm,right=3.5cm]{geometry}

\newtheorem{definition}{Definition}[chapter]
\newtheorem{theorem}{Theorem}[chapter]

\newcommand{\p}[1]{Pr\left[#1\right]}
\newcommand{\alltasks}{{\mathbb T}}
\newcommand{\neededfor}{\rightarrow}
\WithSuffix\newcommand\neededfor*{\stackrel{*}{\rightarrow}}

\begin{document}

\chapter{Theoretical foundations}
\label{chap:theoretical-foundations}

\section{Some probability}
\label{sec:some-probability}

As we will see later, we will often be using probability distributions (in particular continuous probability distributions).

\subsection{Exponential distribution}
\label{sec:exponential-distribution}

\begin{definition}
  A continuous random variable is \emph{exponentially distributed} with parameter $\lambda$ if it has density 
  \begin{equation*}
    f(x) =
    \begin{cases}
      \lambda \cdot e^{-\lambda x} & \text{ if } x \geq 0 
      \\ 0 & \text{ otherwise}
    \end{cases}
    .
  \end{equation*}
\end{definition}

Note that the above definition also determines the distribution function $F$ of an exponentially distributed random variable as follows:

\begin{equation*}
  F(x) =
  \begin{cases}
    1-e^{\lambda x} & \text{ if } x \geq 0 \\
    0 & \text{ otherwise}
  \end{cases}
\end{equation*}

\begin{theorem}
  Let $X_1,\dots,X_n$ be exponentially-distributed random variables with respective parameters $\lambda_1,\dots,\lambda_n$. Then, the ranvom variable $Z:=\min_{i\in\left\{ 1,\dots,n \right\}} \left\{ X_i \right\}$ is exponentially distributed with parameter $\lambda=\lambda_1+\dots+\lambda_n$.
\end{theorem}

\begin{proof}
  We prove the claim by induction. 

  Suppose, we have two exponentially distributed random variables $X_1$ resp. $X_2$ with parameters $\lambda_1$ resp. $\lambda_2$. We then can compute
  \begin{align*}
    \p{min\left\{ X_1,X_2 \right\} \geq x} & = \p{X_1 \geq x \wedge X_2 \geq x} = \\ 
    & = \p{X_1 \geq x}\cdot\p{X_2 \geq x} = \\
    & = e^{-\lambda_1 x} \cdot e^{-\lambda_2 x} = \\
    & = e^{-\lambda_1 x - \lambda_2 x} = \\
    & = e^{-\left( \lambda_1+\lambda_2 \right) \cdot x},
  \end{align*}
  from which we can conclude that $\min\left\{ X_1,X_2 \right\}$ is exponentially distributed with parameter $\lambda_1 + \lambda_2$. By induction, we obtain our claim.
\end{proof}

\subsection{Uniform distribution}
\label{sec:uniform-distribution}

\begin{definition}
  A continuous random variable is \emph{uniformly distributed} over the interval $\left[ a,b \right]$ if it has density
  \begin{equation*}
    f(x) =
    \begin{cases}
      \frac{1}{b-a} & \text{ if } x\in\left[ a,b \right] \\
      0 & \text{ otherwise}
    \end{cases}.
  \end{equation*}
\end{definition}

The density of a uniform random variable is thus given by
\begin{equation*}
  F(x) = \begin{cases}
    0 & \text{ if } x<a \\
    \frac{x-a}{b-a} & \text{ if } x\in\left[ a,b \right] \\
    1 & \text{ if } x>b
  \end{cases}.
\end{equation*}


\subsection{General stuff}
\label{sec:probability-misc}

\begin{theorem}
Let $X_1,\dots,X_n$ be independent, identically distributed, continuous random variables and let $i\in\left\{ 1,\dots,n \right\}$. Then 
\begin{equation}
\label{eq:probability-that-cont-random-variable-is-smallest-out-of-iid-is-one-over-n}
\p{X_i = \min_{j\in\left\{ 1,\dots,n \right\}}\left\{ X_j \right\}} = \frac{1}{n}.
\end{equation}
\end{theorem}

\begin{proof}
  It is clear that 
  \begin{equation*}
    \p{X_1 = \min_{j\in\left\{ 1,\dots,n \right\}}\left\{ X_j \right\}} = \p{X_2 = \min_{j\in\left\{ 1,\dots,n \right\}}\left\{ X_j \right\}} = \dots = \p{X_n = \min_{j\in\left\{ 1,\dots,n \right\}}\left\{ X_j \right\}},
  \end{equation*}
because all random variables $X_1$ through $X_n$ behave the same. Thus we can deduce equation (\ref{eq:probability-that-cont-random-variable-is-smallest-out-of-iid-is-one-over-n}).
\end{proof}

\chapter{First thoughts on implementation}
\label{chap:first-thoughts-on-implementation}

\section{Configuration DAG}
\label{sec:configuration-dag}

Initially we are dealing with an intree (i.e. each node has at most one successor) of tasks that have to be processed by a certain number of processors.

We will call the set of \emph{all} tasks $\alltasks$. If task $t_2$ can only be executed if $t_1$ already has been processed, we write $t_1 \neededfor t_2$. Moreover, we introduce a shorthand notation that allows us to ``chain'' several of these symbols: If there exist tasks $s_1,\dots,s_m$ ($m\in\mathbb N$), we write $t_1 \neededfor* t_2$ if we have $t_1 \neededfor s_1$ and $s_1 \neededfor s_2, s_2 \neededfor s_3, \dots, s_{m-1} \neededfor s_m$ and $s_m \neededfor t_2$ or if $t_1\neededfor t_2$ or if $t_1=t_2$.

\begin{definition}
  Let $\alltasks$ be a set of tasks, and $T \subseteq \alltasks$. We call $T$ an intree (of tasks) if there is one designated task $t_0\in\alltasks$ such that the following two conditions hold:
  \begin{eqnarray*}
    \forall  t \in T. & \quad t \neededfor* t_0 \\
    \forall  t \in T. & \quad t\neededfor s \Rightarrow s\in T
  \end{eqnarray*}
\end{definition}

\begin{definition}
  Let $T$ be an intree of tasks. Let $M\subseteq\alltasks$ be a set of tasks such that the following two conditions hold:
  \begin{itemize}
  \item $\forall t\in M.\, t \in T$
  \item $\forall t\in M.\, \nexists u \in T.\, u\neededfor t $
  \end{itemize}
  We then call the tuple $\left( T, M \right)$ a \emph{configuration}.
\end{definition}

\chapter{Drawing trees}
\label{chap:drawing-trees}

This is just a test chapter to see how we can draw the intrees and the snapshot DAGs.
\newsavebox{\nodebox}

\begin{tikzpicture}
\tikzstyle{level 1} = [sibling distance = 8cm, level distance = 2cm]
\tikzstyle{level 2} = [sibling distance = 4cm, level distance = 1.8cm]
\tikzstyle{level 3} = [sibling distance = 2.66667cm, level distance = 1.6cm]
\tikzstyle{level 4} = [sibling distance = 2cm, level distance = 1.4cm]
\node[draw=black]{
\sbox{\nodebox}{
\begin{tikzpicture}[scale=.2]
\tikzstyle{level 1} = [sibling distance = 8cm, level distance = 2cm]
\tikzstyle{level 2} = [sibling distance = 4cm, level distance = 1.8cm]
\tikzstyle{level 3} = [sibling distance = 2.66667cm, level distance = 1.6cm]
\tikzstyle{level 4} = [sibling distance = 2cm, level distance = 1.4cm]
\node[fill]{}[grow=up]
child{node[fill]{}[grow=up]
child{node[fill,red]{}[grow=up]
}
child{node[fill,red]{}[grow=up]
}
child{node[fill]{}[grow=up]
}
}
;
\end{tikzpicture}
}
\usebox\nodebox
}
child{node[draw=black]{
\sbox{\nodebox}{
\begin{tikzpicture}[scale=.2]
\tikzstyle{level 1} = [sibling distance = 8cm, level distance = 2cm]
\tikzstyle{level 2} = [sibling distance = 4cm, level distance = 1.8cm]
\tikzstyle{level 3} = [sibling distance = 2.66667cm, level distance = 1.6cm]
\node[fill]{}[grow=up]
child{node[fill]{}[grow=up]
child{node[fill,red]{}[grow=up]
}
child{node[fill,red]{}[grow=up]
}
}
;
\end{tikzpicture}
}
\usebox\nodebox
}
child{node[draw=black]{
\sbox{\nodebox}{
\begin{tikzpicture}[scale=.2]
\tikzstyle{level 1} = [sibling distance = 8cm, level distance = 2cm]
\tikzstyle{level 2} = [sibling distance = 4cm, level distance = 1.8cm]
\node[fill]{}[grow=up]
child{node[fill]{}[grow=up]
child{node[fill,red]{}[grow=up]
}
}
;
\end{tikzpicture}
}
\usebox\nodebox
}
child{node[draw=black]{
\sbox{\nodebox}{
\begin{tikzpicture}[scale=.2]
\tikzstyle{level 1} = [sibling distance = 8cm, level distance = 2cm]
\node[fill]{}[grow=up]
child{node[fill,red]{}[grow=up]
}
;
\end{tikzpicture}
}
\usebox\nodebox
}
child{node[draw=black]{
\sbox{\nodebox}{
\begin{tikzpicture}[scale=.2]
\node[fill,red]{}[grow=up]
;
\end{tikzpicture}
}
\usebox\nodebox
}
}}}child{node[draw=black]{
\sbox{\nodebox}{
\begin{tikzpicture}[scale=.2]
\tikzstyle{level 1} = [sibling distance = 8cm, level distance = 2cm]
\tikzstyle{level 2} = [sibling distance = 4cm, level distance = 1.8cm]
\node[fill]{}[grow=up]
child{node[fill]{}[grow=up]
child{node[fill,red]{}[grow=up]
}
}
;
\end{tikzpicture}
}
\usebox\nodebox
}
child{node[draw=black]{
\sbox{\nodebox}{
\begin{tikzpicture}[scale=.2]
\tikzstyle{level 1} = [sibling distance = 8cm, level distance = 2cm]
\node[fill]{}[grow=up]
child{node[fill,red]{}[grow=up]
}
;
\end{tikzpicture}
}
\usebox\nodebox
}
child{node[draw=black]{
\sbox{\nodebox}{
\begin{tikzpicture}[scale=.2]
\node[fill,red]{}[grow=up]
;
\end{tikzpicture}
}
\usebox\nodebox
}
}}}}child{node[draw=black]{
\sbox{\nodebox}{
\begin{tikzpicture}[scale=.2]
\tikzstyle{level 1} = [sibling distance = 8cm, level distance = 2cm]
\tikzstyle{level 2} = [sibling distance = 4cm, level distance = 1.8cm]
\tikzstyle{level 3} = [sibling distance = 2.66667cm, level distance = 1.6cm]
\node[fill]{}[grow=up]
child{node[fill]{}[grow=up]
child{node[fill,red]{}[grow=up]
}
child{node[fill,red]{}[grow=up]
}
}
;
\end{tikzpicture}
}
\usebox\nodebox
}
child{node[draw=black]{
\sbox{\nodebox}{
\begin{tikzpicture}[scale=.2]
\tikzstyle{level 1} = [sibling distance = 8cm, level distance = 2cm]
\tikzstyle{level 2} = [sibling distance = 4cm, level distance = 1.8cm]
\node[fill]{}[grow=up]
child{node[fill]{}[grow=up]
child{node[fill,red]{}[grow=up]
}
}
;
\end{tikzpicture}
}
\usebox\nodebox
}
child{node[draw=black]{
\sbox{\nodebox}{
\begin{tikzpicture}[scale=.2]
\tikzstyle{level 1} = [sibling distance = 8cm, level distance = 2cm]
\node[fill]{}[grow=up]
child{node[fill,red]{}[grow=up]
}
;
\end{tikzpicture}
}
\usebox\nodebox
}
child{node[draw=black]{
\sbox{\nodebox}{
\begin{tikzpicture}[scale=.2]
\node[fill,red]{}[grow=up]
;
\end{tikzpicture}
}
\usebox\nodebox
}
}}}child{node[draw=black]{
\sbox{\nodebox}{
\begin{tikzpicture}[scale=.2]
\tikzstyle{level 1} = [sibling distance = 8cm, level distance = 2cm]
\tikzstyle{level 2} = [sibling distance = 4cm, level distance = 1.8cm]
\node[fill]{}[grow=up]
child{node[fill]{}[grow=up]
child{node[fill,red]{}[grow=up]
}
}
;
\end{tikzpicture}
}
\usebox\nodebox
}
child{node[draw=black]{
\sbox{\nodebox}{
\begin{tikzpicture}[scale=.2]
\tikzstyle{level 1} = [sibling distance = 8cm, level distance = 2cm]
\node[fill]{}[grow=up]
child{node[fill,red]{}[grow=up]
}
;
\end{tikzpicture}
}
\usebox\nodebox
}
child{node[draw=black]{
\sbox{\nodebox}{
\begin{tikzpicture}[scale=.2]
\node[fill,red]{}[grow=up]
;
\end{tikzpicture}
}
\usebox\nodebox
}
}}}};
\end{tikzpicture}

\begin{tikzpicture}
\tikzstyle{level 1} = [sibling distance = 8cm, level distance = 2cm]
\tikzstyle{level 2} = [sibling distance = 4cm, level distance = 1.8cm]
\tikzstyle{level 3} = [sibling distance = 2.66667cm, level distance = 1.6cm]
\tikzstyle{level 4} = [sibling distance = 2cm, level distance = 1.4cm]
\node[draw=black]{
\sbox{\nodebox}{
\begin{tikzpicture}[scale=.2]
\tikzstyle{level 1} = [sibling distance = 8cm, level distance = 2cm]
\tikzstyle{level 2} = [sibling distance = 4cm, level distance = 1.8cm]
\tikzstyle{level 3} = [sibling distance = 2.66667cm, level distance = 1.6cm]
\tikzstyle{level 4} = [sibling distance = 2cm, level distance = 1.4cm]
\node[fill]{}[grow=up]
child{node[fill]{}[grow=up]
child{node[fill]{}[grow=up]
}
child{node[fill,red]{}[grow=up]
}
child{node[fill,red]{}[grow=up]
}
}
;
\end{tikzpicture}
}
\usebox\nodebox
}
child{node[draw=black]{
\sbox{\nodebox}{
\begin{tikzpicture}[scale=.2]
\tikzstyle{level 1} = [sibling distance = 8cm, level distance = 2cm]
\tikzstyle{level 2} = [sibling distance = 4cm, level distance = 1.8cm]
\tikzstyle{level 3} = [sibling distance = 2.66667cm, level distance = 1.6cm]
\node[fill]{}[grow=up]
child{node[fill]{}[grow=up]
child{node[fill,red]{}[grow=up]
}
child{node[fill,red]{}[grow=up]
}
}
;
\end{tikzpicture}
}
\usebox\nodebox
}
child{node[draw=black]{
\sbox{\nodebox}{
\begin{tikzpicture}[scale=.2]
\tikzstyle{level 1} = [sibling distance = 8cm, level distance = 2cm]
\tikzstyle{level 2} = [sibling distance = 4cm, level distance = 1.8cm]
\node[fill]{}[grow=up]
child{node[fill]{}[grow=up]
child{node[fill,red]{}[grow=up]
}
}
;
\end{tikzpicture}
}
\usebox\nodebox
}
child{node[draw=black]{
\sbox{\nodebox}{
\begin{tikzpicture}[scale=.2]
\tikzstyle{level 1} = [sibling distance = 8cm, level distance = 2cm]
\node[fill]{}[grow=up]
child{node[fill,red]{}[grow=up]
}
;
\end{tikzpicture}
}
\usebox\nodebox
}
child{node[draw=black]{
\sbox{\nodebox}{
\begin{tikzpicture}[scale=.2]
\node[fill,red]{}[grow=up]
;
\end{tikzpicture}
}
\usebox\nodebox
}
}}}child{node[draw=black]{
\sbox{\nodebox}{
\begin{tikzpicture}[scale=.2]
\tikzstyle{level 1} = [sibling distance = 8cm, level distance = 2cm]
\tikzstyle{level 2} = [sibling distance = 4cm, level distance = 1.8cm]
\node[fill]{}[grow=up]
child{node[fill]{}[grow=up]
child{node[fill,red]{}[grow=up]
}
}
;
\end{tikzpicture}
}
\usebox\nodebox
}
child{node[draw=black]{
\sbox{\nodebox}{
\begin{tikzpicture}[scale=.2]
\tikzstyle{level 1} = [sibling distance = 8cm, level distance = 2cm]
\node[fill]{}[grow=up]
child{node[fill,red]{}[grow=up]
}
;
\end{tikzpicture}
}
\usebox\nodebox
}
child{node[draw=black]{
\sbox{\nodebox}{
\begin{tikzpicture}[scale=.2]
\node[fill,red]{}[grow=up]
;
\end{tikzpicture}
}
\usebox\nodebox
}
}}}}child{node[draw=black]{
\sbox{\nodebox}{
\begin{tikzpicture}[scale=.2]
\tikzstyle{level 1} = [sibling distance = 8cm, level distance = 2cm]
\tikzstyle{level 2} = [sibling distance = 4cm, level distance = 1.8cm]
\tikzstyle{level 3} = [sibling distance = 2.66667cm, level distance = 1.6cm]
\node[fill]{}[grow=up]
child{node[fill]{}[grow=up]
child{node[fill,red]{}[grow=up]
}
child{node[fill,red]{}[grow=up]
}
}
;
\end{tikzpicture}
}
\usebox\nodebox
}
child{node[draw=black]{
\sbox{\nodebox}{
\begin{tikzpicture}[scale=.2]
\tikzstyle{level 1} = [sibling distance = 8cm, level distance = 2cm]
\tikzstyle{level 2} = [sibling distance = 4cm, level distance = 1.8cm]
\node[fill]{}[grow=up]
child{node[fill]{}[grow=up]
child{node[fill,red]{}[grow=up]
}
}
;
\end{tikzpicture}
}
\usebox\nodebox
}
child{node[draw=black]{
\sbox{\nodebox}{
\begin{tikzpicture}[scale=.2]
\tikzstyle{level 1} = [sibling distance = 8cm, level distance = 2cm]
\node[fill]{}[grow=up]
child{node[fill,red]{}[grow=up]
}
;
\end{tikzpicture}
}
\usebox\nodebox
}
child{node[draw=black]{
\sbox{\nodebox}{
\begin{tikzpicture}[scale=.2]
\node[fill,red]{}[grow=up]
;
\end{tikzpicture}
}
\usebox\nodebox
}
}}}child{node[draw=black]{
\sbox{\nodebox}{
\begin{tikzpicture}[scale=.2]
\tikzstyle{level 1} = [sibling distance = 8cm, level distance = 2cm]
\tikzstyle{level 2} = [sibling distance = 4cm, level distance = 1.8cm]
\node[fill]{}[grow=up]
child{node[fill]{}[grow=up]
child{node[fill,red]{}[grow=up]
}
}
;
\end{tikzpicture}
}
\usebox\nodebox
}
child{node[draw=black]{
\sbox{\nodebox}{
\begin{tikzpicture}[scale=.2]
\tikzstyle{level 1} = [sibling distance = 8cm, level distance = 2cm]
\node[fill]{}[grow=up]
child{node[fill,red]{}[grow=up]
}
;
\end{tikzpicture}
}
\usebox\nodebox
}
child{node[draw=black]{
\sbox{\nodebox}{
\begin{tikzpicture}[scale=.2]
\node[fill,red]{}[grow=up]
;
\end{tikzpicture}
}
\usebox\nodebox
}
}}}};
\end{tikzpicture}

\begin{tikzpicture}
\tikzstyle{level 1} = [sibling distance = 8cm, level distance = 2cm]
\tikzstyle{level 2} = [sibling distance = 4cm, level distance = 1.8cm]
\tikzstyle{level 3} = [sibling distance = 2.66667cm, level distance = 1.6cm]
\tikzstyle{level 4} = [sibling distance = 2cm, level distance = 1.4cm]
\node[draw=black]{
\sbox{\nodebox}{
\begin{tikzpicture}[scale=.2]
\tikzstyle{level 1} = [sibling distance = 8cm, level distance = 2cm]
\tikzstyle{level 2} = [sibling distance = 4cm, level distance = 1.8cm]
\tikzstyle{level 3} = [sibling distance = 2.66667cm, level distance = 1.6cm]
\tikzstyle{level 4} = [sibling distance = 2cm, level distance = 1.4cm]
\node[fill]{}[grow=up]
child{node[fill]{}[grow=up]
child{node[fill,red]{}[grow=up]
}
child{node[fill]{}[grow=up]
}
child{node[fill,red]{}[grow=up]
}
}
;
\end{tikzpicture}
}
\usebox\nodebox
}
child{node[draw=black]{
\sbox{\nodebox}{
\begin{tikzpicture}[scale=.2]
\tikzstyle{level 1} = [sibling distance = 8cm, level distance = 2cm]
\tikzstyle{level 2} = [sibling distance = 4cm, level distance = 1.8cm]
\tikzstyle{level 3} = [sibling distance = 2.66667cm, level distance = 1.6cm]
\node[fill]{}[grow=up]
child{node[fill]{}[grow=up]
child{node[fill,red]{}[grow=up]
}
child{node[fill,red]{}[grow=up]
}
}
;
\end{tikzpicture}
}
\usebox\nodebox
}
child{node[draw=black]{
\sbox{\nodebox}{
\begin{tikzpicture}[scale=.2]
\tikzstyle{level 1} = [sibling distance = 8cm, level distance = 2cm]
\tikzstyle{level 2} = [sibling distance = 4cm, level distance = 1.8cm]
\node[fill]{}[grow=up]
child{node[fill]{}[grow=up]
child{node[fill,red]{}[grow=up]
}
}
;
\end{tikzpicture}
}
\usebox\nodebox
}
child{node[draw=black]{
\sbox{\nodebox}{
\begin{tikzpicture}[scale=.2]
\tikzstyle{level 1} = [sibling distance = 8cm, level distance = 2cm]
\node[fill]{}[grow=up]
child{node[fill,red]{}[grow=up]
}
;
\end{tikzpicture}
}
\usebox\nodebox
}
child{node[draw=black]{
\sbox{\nodebox}{
\begin{tikzpicture}[scale=.2]
\node[fill,red]{}[grow=up]
;
\end{tikzpicture}
}
\usebox\nodebox
}
}}}child{node[draw=black]{
\sbox{\nodebox}{
\begin{tikzpicture}[scale=.2]
\tikzstyle{level 1} = [sibling distance = 8cm, level distance = 2cm]
\tikzstyle{level 2} = [sibling distance = 4cm, level distance = 1.8cm]
\node[fill]{}[grow=up]
child{node[fill]{}[grow=up]
child{node[fill,red]{}[grow=up]
}
}
;
\end{tikzpicture}
}
\usebox\nodebox
}
child{node[draw=black]{
\sbox{\nodebox}{
\begin{tikzpicture}[scale=.2]
\tikzstyle{level 1} = [sibling distance = 8cm, level distance = 2cm]
\node[fill]{}[grow=up]
child{node[fill,red]{}[grow=up]
}
;
\end{tikzpicture}
}
\usebox\nodebox
}
child{node[draw=black]{
\sbox{\nodebox}{
\begin{tikzpicture}[scale=.2]
\node[fill,red]{}[grow=up]
;
\end{tikzpicture}
}
\usebox\nodebox
}
}}}}child{node[draw=black]{
\sbox{\nodebox}{
\begin{tikzpicture}[scale=.2]
\tikzstyle{level 1} = [sibling distance = 8cm, level distance = 2cm]
\tikzstyle{level 2} = [sibling distance = 4cm, level distance = 1.8cm]
\tikzstyle{level 3} = [sibling distance = 2.66667cm, level distance = 1.6cm]
\node[fill]{}[grow=up]
child{node[fill]{}[grow=up]
child{node[fill,red]{}[grow=up]
}
child{node[fill,red]{}[grow=up]
}
}
;
\end{tikzpicture}
}
\usebox\nodebox
}
child{node[draw=black]{
\sbox{\nodebox}{
\begin{tikzpicture}[scale=.2]
\tikzstyle{level 1} = [sibling distance = 8cm, level distance = 2cm]
\tikzstyle{level 2} = [sibling distance = 4cm, level distance = 1.8cm]
\node[fill]{}[grow=up]
child{node[fill]{}[grow=up]
child{node[fill,red]{}[grow=up]
}
}
;
\end{tikzpicture}
}
\usebox\nodebox
}
child{node[draw=black]{
\sbox{\nodebox}{
\begin{tikzpicture}[scale=.2]
\tikzstyle{level 1} = [sibling distance = 8cm, level distance = 2cm]
\node[fill]{}[grow=up]
child{node[fill,red]{}[grow=up]
}
;
\end{tikzpicture}
}
\usebox\nodebox
}
child{node[draw=black]{
\sbox{\nodebox}{
\begin{tikzpicture}[scale=.2]
\node[fill,red]{}[grow=up]
;
\end{tikzpicture}
}
\usebox\nodebox
}
}}}child{node[draw=black]{
\sbox{\nodebox}{
\begin{tikzpicture}[scale=.2]
\tikzstyle{level 1} = [sibling distance = 8cm, level distance = 2cm]
\tikzstyle{level 2} = [sibling distance = 4cm, level distance = 1.8cm]
\node[fill]{}[grow=up]
child{node[fill]{}[grow=up]
child{node[fill,red]{}[grow=up]
}
}
;
\end{tikzpicture}
}
\usebox\nodebox
}
child{node[draw=black]{
\sbox{\nodebox}{
\begin{tikzpicture}[scale=.2]
\tikzstyle{level 1} = [sibling distance = 8cm, level distance = 2cm]
\node[fill]{}[grow=up]
child{node[fill,red]{}[grow=up]
}
;
\end{tikzpicture}
}
\usebox\nodebox
}
child{node[draw=black]{
\sbox{\nodebox}{
\begin{tikzpicture}[scale=.2]
\node[fill,red]{}[grow=up]
;
\end{tikzpicture}
}
\usebox\nodebox
}
}}}};
\end{tikzpicture}

4
4
4
Total expected run time: 4





\end{document}
