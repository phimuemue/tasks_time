%%% hugegraph.tex --- 

%% Author: philipp@pmpc
%% Version: $Id: thesis.tex,v 0.0 2013/04/08 12:19:13 philipp Exp$

%%\revision$Header: /home/philipp/Documents/Uni/masterarbeit/thesis/thesis.tex,v 0.0 2013/04/08 12:19:13 philipp Exp$

\documentclass{report}

\usepackage[english]{babel}
\usepackage{lmodern}
\usepackage[utf8]{inputenc}
\usepackage{hyperref}
\usepackage{amsmath}
\usepackage{amssymb}
\usepackage{amsthm}
\usepackage{graphicx}
\usepackage{tikz}
\usepackage{suffix}
\usepackage{multicol}
\usepackage[left=3.5cm,right=3.5cm,a0paper]{geometry}

\usetikzlibrary{shapes.multipart,chains}
\usetikzlibrary{positioning}
\usetikzlibrary{matrix}
\usetikzlibrary{external} 
%\tikzexternalize


\tikzstyle{task_cross}=[
    {path picture={ 
        \draw[black]
        (path picture bounding box.south east) -- 
        (path picture bounding box.north west) 
        (path picture bounding box.south west) -- 
        (path picture bounding box.north east);
      }
    }
]

\tikzstyle{task_scheduled}=[fill=white, draw=black, task_cross]

\newtheorem{definition}{Definition}[chapter]
\newtheorem{theorem}{Theorem}[chapter]

\newcommand{\p}[1]{Pr\left[#1\right]}
\newcommand{\alltasks}{{\mathbb T}}
\newcommand{\neededfor}{\rightarrow}
\WithSuffix\newcommand\neededfor*{\stackrel{*}{\rightarrow}}

% \getwidthofnode will measure the width of the node given as its second
% parameter and store it into the first parameter.
\makeatletter
\newcommand\getwidthofnode[2]{%
    \pgfextractx{#1}{\pgfpointanchor{#2}{east}}%
    \pgfextractx{\pgf@xa}{\pgfpointanchor{#2}{west}}% \pgf@xa is a length defined by PGF for temporary storage. No need to create a new temporary length.
    \addtolength{#1}{-\pgf@xa}%
}
\makeatother

\begin{document}

% stuff to draw diagrams levelwise
\newcommand{\leveltop}{0}
\newcommand{\leveltopI}{0}
\newcommand{\leveltopII}{0}
\newcommand{\leveltopIII}{0}
\newcommand{\leveltopIIII}{0}
\newcommand{\leveltopIIIII}{0}
\newcommand{\leveltopIIIIII}{0}
\newcommand{\leveltopIIIIIII}{0}
\newcommand{\leveltopIIIIIIII}{0}
\newcommand{\leveltopIIIIIIIII}{0}
\newcommand{\leveltopIIIIIIIIII}{0}
\newcommand{\leveltopIIIIIIIIIII}{0}
\newcommand{\leveltopIIIIIIIIIIII}{0}
\newcommand{\leveltopIIIIIIIIIIIII}{0}
\newcommand{\leveltopIIIIIIIIIIIIII}{0}
\newcommand{\leveltopIIIIIIIIIIIIIII}{0}
\newcommand{\leveltopIIIIIIIIIIIIIIII}{0}
\newcommand{\leveltopIIIIIIIIIIIIIIIII}{0}
\newcommand{\leveltopIIIIIIIIIIIIIIIIII}{0}
\newcommand{\leveltopIIIIIIIIIIIIIIIIIII}{0}
\newcommand{\leveltopIIIIIIIIIIIIIIIIIIII}{0}
\newcommand{\leveltopIIIIIIIIIIIIIIIIIIIII}{0}
\newcommand{\leveltopIIIIIIIIIIIIIIIIIIIIII}{0}
\newcommand{\leveltopIIIIIIIIIIIIIIIIIIIIIII}{0}
\newcommand{\leveltopIIIIIIIIIIIIIIIIIIIIIIII}{0}
\newcommand{\leveltopIIIIIIIIIIIIIIIIIIIIIIIII}{0}
\newcommand{\leveltopIIIIIIIIIIIIIIIIIIIIIIIIII}{0}
\newcommand{\leveltopIIIIIIIIIIIIIIIIIIIIIIIIIII}{0}
\newcommand{\leveltopIIIIIIIIIIIIIIIIIIIIIIIIIIII}{0}
\newcommand{\leveltopIIIIIIIIIIIIIIIIIIIIIIIIIIIII}{0}

\section{P2: A whole intree-DAG and its condensed counterpart}

\renewcommand{\leveltopI}{-15cm + \leveltop}
\renewcommand{\leveltopII}{-15cm + \leveltopI}
\renewcommand{\leveltopIII}{-15cm + \leveltopII}
\renewcommand{\leveltopIIII}{-15cm + \leveltopIII}
\renewcommand{\leveltopIIIII}{-15cm + \leveltopIIII}
\renewcommand{\leveltopIIIIII}{-15cm + \leveltopIIIII}
\renewcommand{\leveltopIIIIIII}{-15cm + \leveltopIIIIII}
\renewcommand{\leveltopIIIIIIII}{-15cm + \leveltopIIIIIII}
\renewcommand{\leveltopIIIIIIIII}{-15cm + \leveltopIIIIIIII}
\renewcommand{\leveltopIIIIIIIIII}{-15cm + \leveltopIIIIIIIII}
\renewcommand{\leveltopIIIIIIIIIII}{-15cm + \leveltopIIIIIIIIII}
\begin{tikzpicture}[scale=.2, anchor=south]
  \begin{scope}[yshift=\leveltopI cm]
    \matrix (line1) [column sep=1cm] {
      \node[draw=black, rectangle split,  rectangle split parts=3] (sn0x104f980){
        \begin{tikzpicture}[scale=.2]
          \node[circle, scale=0.75, fill] (tid0) at (3.75,1.5){};
          \node[circle, scale=0.75, fill] (tid1) at (2.25,3){};
          \node[circle, scale=0.75, fill] (tid3) at (0.75,4.5){};
          \node[circle, scale=0.75, fill] (tid7) at (0.75,6){};
          \draw[](tid3) -- (tid7);
          \node[circle, scale=0.75, fill] (tid4) at (2.25,4.5){};
          \node[circle, scale=0.75, fill] (tid5) at (3.75,4.5){};
          \draw[](tid1) -- (tid3);
          \draw[](tid1) -- (tid4);
          \draw[](tid1) -- (tid5);
          \node[circle, scale=0.75, fill] (tid2) at (6,3){};
          \node[circle, scale=0.75, fill] (tid6) at (6,4.5){};
          \node[circle, scale=0.75, fill] (tid8) at (5.25,6){};
          \node[circle, scale=0.75, fill, red] (tid10) at (5.25,7.5){};
          \draw[](tid8) -- (tid10);
          \node[circle, scale=0.75, fill, red] (tid9) at (6.75,6){};
          \draw[](tid6) -- (tid8);
          \draw[](tid6) -- (tid9);
          \draw[](tid2) -- (tid6);
          \draw[](tid0) -- (tid1);
          \draw[](tid0) -- (tid2);
        \end{tikzpicture}
        \nodepart{two}
        \footnotesize{6.82812}
        \nodepart{three}
        \footnotesize{$50\:25\:25$}
      };
      & 
      \\
    };
  \end{scope}
  \begin{scope}[yshift=\leveltopII cm]
    \matrix (line2) [column sep=1cm] {
      \node[draw=black, rectangle split,  rectangle split parts=3] (sn0x1050190){
        \begin{tikzpicture}[scale=.2]
          \node[circle, scale=0.75, fill] (tid0) at (3,1.5){};
          \node[circle, scale=0.75, fill] (tid1) at (2.25,3){};
          \node[circle, scale=0.75, fill] (tid3) at (0.75,4.5){};
          \node[circle, scale=0.75, fill, red] (tid7) at (0.75,6){};
          \draw[](tid3) -- (tid7);
          \node[circle, scale=0.75, fill] (tid4) at (2.25,4.5){};
          \node[circle, scale=0.75, fill] (tid5) at (3.75,4.5){};
          \draw[](tid1) -- (tid3);
          \draw[](tid1) -- (tid4);
          \draw[](tid1) -- (tid5);
          \node[circle, scale=0.75, fill] (tid2) at (5.25,3){};
          \node[circle, scale=0.75, fill] (tid6) at (5.25,4.5){};
          \node[circle, scale=0.75, fill] (tid8) at (5.25,6){};
          \node[circle, scale=0.75, fill, red] (tid9) at (5.25,7.5){};
          \draw[](tid8) -- (tid9);
          \draw[](tid6) -- (tid8);
          \draw[](tid2) -- (tid6);
          \draw[](tid0) -- (tid1);
          \draw[](tid0) -- (tid2);
        \end{tikzpicture}
        \nodepart{two}
        \footnotesize{6.35938}
        \nodepart{three}
        \footnotesize{$50\:50$}
      };
      & 
      \node[draw=black, rectangle split,  rectangle split parts=3] (sn0x104cb60){
        \begin{tikzpicture}[scale=.2]
          \node[circle, scale=0.75, fill] (tid0) at (3.75,1.5){};
          \node[circle, scale=0.75, fill] (tid1) at (2.25,3){};
          \node[circle, scale=0.75, fill] (tid3) at (0.75,4.5){};
          \node[circle, scale=0.75, fill, red] (tid7) at (0.75,6){};
          \draw[](tid3) -- (tid7);
          \node[circle, scale=0.75, fill] (tid4) at (2.25,4.5){};
          \node[circle, scale=0.75, fill] (tid5) at (3.75,4.5){};
          \draw[](tid1) -- (tid3);
          \draw[](tid1) -- (tid4);
          \draw[](tid1) -- (tid5);
          \node[circle, scale=0.75, fill] (tid2) at (6,3){};
          \node[circle, scale=0.75, fill] (tid6) at (6,4.5){};
          \node[circle, scale=0.75, fill, red] (tid8) at (5.25,6){};
          \node[circle, scale=0.75, fill] (tid9) at (6.75,6){};
          \draw[](tid6) -- (tid8);
          \draw[](tid6) -- (tid9);
          \draw[](tid2) -- (tid6);
          \draw[](tid0) -- (tid1);
          \draw[](tid0) -- (tid2);
        \end{tikzpicture}
        \nodepart{two}
        \footnotesize{6.29688}
        \nodepart{three}
        \footnotesize{$50\:50$}
      };
      & 
      \node[draw=black, rectangle split,  rectangle split parts=3] (sn0x104dd20){
        \begin{tikzpicture}[scale=.2]
          \node[circle, scale=0.75, fill] (tid0) at (3.75,1.5){};
          \node[circle, scale=0.75, fill] (tid1) at (2.25,3){};
          \node[circle, scale=0.75, fill] (tid3) at (0.75,4.5){};
          \node[circle, scale=0.75, fill] (tid7) at (0.75,6){};
          \draw[](tid3) -- (tid7);
          \node[circle, scale=0.75, fill] (tid4) at (2.25,4.5){};
          \node[circle, scale=0.75, fill] (tid5) at (3.75,4.5){};
          \draw[](tid1) -- (tid3);
          \draw[](tid1) -- (tid4);
          \draw[](tid1) -- (tid5);
          \node[circle, scale=0.75, fill] (tid2) at (6,3){};
          \node[circle, scale=0.75, fill] (tid6) at (6,4.5){};
          \node[circle, scale=0.75, fill, red] (tid8) at (5.25,6){};
          \node[circle, scale=0.75, fill, red] (tid9) at (6.75,6){};
          \draw[](tid6) -- (tid8);
          \draw[](tid6) -- (tid9);
          \draw[](tid2) -- (tid6);
          \draw[](tid0) -- (tid1);
          \draw[](tid0) -- (tid2);
        \end{tikzpicture}
        \nodepart{two}
        \footnotesize{6.29688}
        \nodepart{three}
        \footnotesize{$1$}
      };
      & 
      \\
    };
  \end{scope}
  \begin{scope}[yshift=\leveltopIII cm]
    \matrix (line3) [column sep=1cm] {
      \node[draw=black, rectangle split,  rectangle split parts=3] (sn0x10519d0){
        \begin{tikzpicture}[scale=.2]
          \node[circle, scale=0.75, fill] (tid0) at (3,1.5){};
          \node[circle, scale=0.75, fill] (tid1) at (2.25,3){};
          \node[circle, scale=0.75, fill, red] (tid3) at (0.75,4.5){};
          \node[circle, scale=0.75, fill] (tid4) at (2.25,4.5){};
          \node[circle, scale=0.75, fill] (tid5) at (3.75,4.5){};
          \draw[](tid1) -- (tid3);
          \draw[](tid1) -- (tid4);
          \draw[](tid1) -- (tid5);
          \node[circle, scale=0.75, fill] (tid2) at (5.25,3){};
          \node[circle, scale=0.75, fill] (tid6) at (5.25,4.5){};
          \node[circle, scale=0.75, fill] (tid7) at (5.25,6){};
          \node[circle, scale=0.75, fill, red] (tid8) at (5.25,7.5){};
          \draw[](tid7) -- (tid8);
          \draw[](tid6) -- (tid7);
          \draw[](tid2) -- (tid6);
          \draw[](tid0) -- (tid1);
          \draw[](tid0) -- (tid2);
        \end{tikzpicture}
        \nodepart{two}
        \footnotesize{5.92188}
        \nodepart{three}
        \footnotesize{$50\:50$}
      };
      & 
      \node[draw=black, rectangle split,  rectangle split parts=3] (sn0x104fbd0){
        \begin{tikzpicture}[scale=.2]
          \node[circle, scale=0.75, fill] (tid0) at (3,1.5){};
          \node[circle, scale=0.75, fill] (tid1) at (2.25,3){};
          \node[circle, scale=0.75, fill] (tid3) at (0.75,4.5){};
          \node[circle, scale=0.75, fill, red] (tid7) at (0.75,6){};
          \draw[](tid3) -- (tid7);
          \node[circle, scale=0.75, fill] (tid4) at (2.25,4.5){};
          \node[circle, scale=0.75, fill] (tid5) at (3.75,4.5){};
          \draw[](tid1) -- (tid3);
          \draw[](tid1) -- (tid4);
          \draw[](tid1) -- (tid5);
          \node[circle, scale=0.75, fill] (tid2) at (5.25,3){};
          \node[circle, scale=0.75, fill] (tid6) at (5.25,4.5){};
          \node[circle, scale=0.75, fill, red] (tid8) at (5.25,6){};
          \draw[](tid6) -- (tid8);
          \draw[](tid2) -- (tid6);
          \draw[](tid0) -- (tid1);
          \draw[](tid0) -- (tid2);
        \end{tikzpicture}
        \nodepart{two}
        \footnotesize{5.79688}
        \nodepart{three}
        \footnotesize{$50\:33\:17$}
      };
      & 
      \node[draw=black, rectangle split,  rectangle split parts=3] (sn0x105a080){
        \begin{tikzpicture}[scale=.2]
          \node[circle, scale=0.75, fill] (tid0) at (3.75,1.5){};
          \node[circle, scale=0.75, fill] (tid1) at (2.25,3){};
          \node[circle, scale=0.75, fill] (tid3) at (0.75,4.5){};
          \node[circle, scale=0.75, fill] (tid4) at (2.25,4.5){};
          \node[circle, scale=0.75, fill] (tid5) at (3.75,4.5){};
          \draw[](tid1) -- (tid3);
          \draw[](tid1) -- (tid4);
          \draw[](tid1) -- (tid5);
          \node[circle, scale=0.75, fill] (tid2) at (6,3){};
          \node[circle, scale=0.75, fill] (tid6) at (6,4.5){};
          \node[circle, scale=0.75, fill, red] (tid7) at (5.25,6){};
          \node[circle, scale=0.75, fill, red] (tid8) at (6.75,6){};
          \draw[](tid6) -- (tid7);
          \draw[](tid6) -- (tid8);
          \draw[](tid2) -- (tid6);
          \draw[](tid0) -- (tid1);
          \draw[](tid0) -- (tid2);
        \end{tikzpicture}
        \nodepart{two}
        \footnotesize{5.79688}
        \nodepart{three}
        \footnotesize{$1$}
      };
      & 
      \\
    };
  \end{scope}
  \begin{scope}[yshift=\leveltopIIII cm]
    \matrix (line4) [column sep=1cm] {
      \node[draw=black, rectangle split,  rectangle split parts=3] (sn0x1052250){
        \begin{tikzpicture}[scale=.2]
          \node[circle, scale=0.75, fill] (tid0) at (2.25,1.5){};
          \node[circle, scale=0.75, fill] (tid1) at (0.75,3){};
          \node[circle, scale=0.75, fill] (tid3) at (0.75,4.5){};
          \node[circle, scale=0.75, fill] (tid6) at (0.75,6){};
          \node[circle, scale=0.75, fill, red] (tid7) at (0.75,7.5){};
          \draw[](tid6) -- (tid7);
          \draw[](tid3) -- (tid6);
          \draw[](tid1) -- (tid3);
          \node[circle, scale=0.75, fill] (tid2) at (3,3){};
          \node[circle, scale=0.75, fill, red] (tid4) at (2.25,4.5){};
          \node[circle, scale=0.75, fill] (tid5) at (3.75,4.5){};
          \draw[](tid2) -- (tid4);
          \draw[](tid2) -- (tid5);
          \draw[](tid0) -- (tid1);
          \draw[](tid0) -- (tid2);
        \end{tikzpicture}
        \nodepart{two}
        \footnotesize{5.54688}
        \nodepart{three}
        \footnotesize{$50\:50$}
      };
      & 
      \node[draw=black, rectangle split,  rectangle split parts=3] (sn0x1052960){
        \begin{tikzpicture}[scale=.2]
          \node[circle, scale=0.75, fill] (tid0) at (3,1.5){};
          \node[circle, scale=0.75, fill] (tid1) at (2.25,3){};
          \node[circle, scale=0.75, fill, red] (tid3) at (0.75,4.5){};
          \node[circle, scale=0.75, fill] (tid4) at (2.25,4.5){};
          \node[circle, scale=0.75, fill] (tid5) at (3.75,4.5){};
          \draw[](tid1) -- (tid3);
          \draw[](tid1) -- (tid4);
          \draw[](tid1) -- (tid5);
          \node[circle, scale=0.75, fill] (tid2) at (5.25,3){};
          \node[circle, scale=0.75, fill] (tid6) at (5.25,4.5){};
          \node[circle, scale=0.75, fill, red] (tid7) at (5.25,6){};
          \draw[](tid6) -- (tid7);
          \draw[](tid2) -- (tid6);
          \draw[](tid0) -- (tid1);
          \draw[](tid0) -- (tid2);
        \end{tikzpicture}
        \nodepart{two}
        \footnotesize{5.29688}
        \nodepart{three}
        \footnotesize{$50\:33\:17$}
      };
      & 
      \node[draw=black, rectangle split,  rectangle split parts=3] (sn0x10581e0){
        \begin{tikzpicture}[scale=.2]
          \node[circle, scale=0.75, fill] (tid0) at (3,1.5){};
          \node[circle, scale=0.75, fill] (tid1) at (2.25,3){};
          \node[circle, scale=0.75, fill] (tid3) at (0.75,4.5){};
          \node[circle, scale=0.75, fill, red] (tid7) at (0.75,6){};
          \draw[](tid3) -- (tid7);
          \node[circle, scale=0.75, fill, red] (tid4) at (2.25,4.5){};
          \node[circle, scale=0.75, fill] (tid5) at (3.75,4.5){};
          \draw[](tid1) -- (tid3);
          \draw[](tid1) -- (tid4);
          \draw[](tid1) -- (tid5);
          \node[circle, scale=0.75, fill] (tid2) at (5.25,3){};
          \node[circle, scale=0.75, fill] (tid6) at (5.25,4.5){};
          \draw[](tid2) -- (tid6);
          \draw[](tid0) -- (tid1);
          \draw[](tid0) -- (tid2);
        \end{tikzpicture}
        \nodepart{two}
        \footnotesize{5.29688}
        \nodepart{three}
        \footnotesize{$33\:17\:25\:25$}
      };
      & 
      \node[draw=black, rectangle split,  rectangle split parts=3] (sn0x1058550){
        \begin{tikzpicture}[scale=.2]
          \node[circle, scale=0.75, fill] (tid0) at (3,1.5){};
          \node[circle, scale=0.75, fill] (tid1) at (2.25,3){};
          \node[circle, scale=0.75, fill] (tid3) at (0.75,4.5){};
          \node[circle, scale=0.75, fill, red] (tid7) at (0.75,6){};
          \draw[](tid3) -- (tid7);
          \node[circle, scale=0.75, fill] (tid4) at (2.25,4.5){};
          \node[circle, scale=0.75, fill] (tid5) at (3.75,4.5){};
          \draw[](tid1) -- (tid3);
          \draw[](tid1) -- (tid4);
          \draw[](tid1) -- (tid5);
          \node[circle, scale=0.75, fill] (tid2) at (5.25,3){};
          \node[circle, scale=0.75, fill, red] (tid6) at (5.25,4.5){};
          \draw[](tid2) -- (tid6);
          \draw[](tid0) -- (tid1);
          \draw[](tid0) -- (tid2);
        \end{tikzpicture}
        \nodepart{two}
        \footnotesize{5.29688}
        \nodepart{three}
        \footnotesize{$50\:50$}
      };
      & 
      \\
    };
  \end{scope}
  \begin{scope}[yshift=\leveltopIIIII cm]
    \matrix (line5) [column sep=1cm] {
      \node[draw=black, rectangle split,  rectangle split parts=3] (sn0x10525f0){
        \begin{tikzpicture}[scale=.2]
          \node[circle, scale=0.75, fill] (tid0) at (1.5,1.5){};
          \node[circle, scale=0.75, fill] (tid1) at (0.75,3){};
          \node[circle, scale=0.75, fill] (tid3) at (0.75,4.5){};
          \node[circle, scale=0.75, fill] (tid5) at (0.75,6){};
          \node[circle, scale=0.75, fill, red] (tid6) at (0.75,7.5){};
          \draw[](tid5) -- (tid6);
          \draw[](tid3) -- (tid5);
          \draw[](tid1) -- (tid3);
          \node[circle, scale=0.75, fill] (tid2) at (2.25,3){};
          \node[circle, scale=0.75, fill, red] (tid4) at (2.25,4.5){};
          \draw[](tid2) -- (tid4);
          \draw[](tid0) -- (tid1);
          \draw[](tid0) -- (tid2);
        \end{tikzpicture}
        \nodepart{two}
        \footnotesize{5.25}
        \nodepart{three}
        \footnotesize{$50\:50$}
      };
      & 
      \node[draw=black, rectangle split,  rectangle split parts=3] (sn0x1053850){
        \begin{tikzpicture}[scale=.2]
          \node[circle, scale=0.75, fill] (tid0) at (2.25,1.5){};
          \node[circle, scale=0.75, fill] (tid1) at (1.5,3){};
          \node[circle, scale=0.75, fill, red] (tid3) at (0.75,4.5){};
          \node[circle, scale=0.75, fill] (tid4) at (2.25,4.5){};
          \draw[](tid1) -- (tid3);
          \draw[](tid1) -- (tid4);
          \node[circle, scale=0.75, fill] (tid2) at (3.75,3){};
          \node[circle, scale=0.75, fill] (tid5) at (3.75,4.5){};
          \node[circle, scale=0.75, fill, red] (tid6) at (3.75,6){};
          \draw[](tid5) -- (tid6);
          \draw[](tid2) -- (tid5);
          \draw[](tid0) -- (tid1);
          \draw[](tid0) -- (tid2);
        \end{tikzpicture}
        \nodepart{two}
        \footnotesize{4.84375}
        \nodepart{three}
        \footnotesize{$50\:25\:25$}
      };
      & 
      \node[draw=black, rectangle split,  rectangle split parts=3] (sn0x1056b00){
        \begin{tikzpicture}[scale=.2]
          \node[circle, scale=0.75, fill] (tid0) at (3,1.5){};
          \node[circle, scale=0.75, fill] (tid1) at (2.25,3){};
          \node[circle, scale=0.75, fill, red] (tid3) at (0.75,4.5){};
          \node[circle, scale=0.75, fill, red] (tid4) at (2.25,4.5){};
          \node[circle, scale=0.75, fill] (tid5) at (3.75,4.5){};
          \draw[](tid1) -- (tid3);
          \draw[](tid1) -- (tid4);
          \draw[](tid1) -- (tid5);
          \node[circle, scale=0.75, fill] (tid2) at (5.25,3){};
          \node[circle, scale=0.75, fill] (tid6) at (5.25,4.5){};
          \draw[](tid2) -- (tid6);
          \draw[](tid0) -- (tid1);
          \draw[](tid0) -- (tid2);
        \end{tikzpicture}
        \nodepart{two}
        \footnotesize{4.75}
        \nodepart{three}
        \footnotesize{$50\:50$}
      };
      & 
      \node[draw=black, rectangle split,  rectangle split parts=3] (sn0x1056fb0){
        \begin{tikzpicture}[scale=.2]
          \node[circle, scale=0.75, fill] (tid0) at (3,1.5){};
          \node[circle, scale=0.75, fill] (tid1) at (2.25,3){};
          \node[circle, scale=0.75, fill, red] (tid3) at (0.75,4.5){};
          \node[circle, scale=0.75, fill] (tid4) at (2.25,4.5){};
          \node[circle, scale=0.75, fill] (tid5) at (3.75,4.5){};
          \draw[](tid1) -- (tid3);
          \draw[](tid1) -- (tid4);
          \draw[](tid1) -- (tid5);
          \node[circle, scale=0.75, fill] (tid2) at (5.25,3){};
          \node[circle, scale=0.75, fill, red] (tid6) at (5.25,4.5){};
          \draw[](tid2) -- (tid6);
          \draw[](tid0) -- (tid1);
          \draw[](tid0) -- (tid2);
        \end{tikzpicture}
        \nodepart{two}
        \footnotesize{4.75}
        \nodepart{three}
        \footnotesize{$50\:50$}
      };
      & 
      \node[draw=black, rectangle split,  rectangle split parts=3] (sn0x1058f50){
        \begin{tikzpicture}[scale=.2]
          \node[circle, scale=0.75, fill] (tid0) at (2.25,1.5){};
          \node[circle, scale=0.75, fill] (tid1) at (1.5,3){};
          \node[circle, scale=0.75, fill] (tid3) at (0.75,4.5){};
          \node[circle, scale=0.75, fill, red] (tid6) at (0.75,6){};
          \draw[](tid3) -- (tid6);
          \node[circle, scale=0.75, fill, red] (tid4) at (2.25,4.5){};
          \draw[](tid1) -- (tid3);
          \draw[](tid1) -- (tid4);
          \node[circle, scale=0.75, fill] (tid2) at (3.75,3){};
          \node[circle, scale=0.75, fill] (tid5) at (3.75,4.5){};
          \draw[](tid2) -- (tid5);
          \draw[](tid0) -- (tid1);
          \draw[](tid0) -- (tid2);
        \end{tikzpicture}
        \nodepart{two}
        \footnotesize{4.84375}
        \nodepart{three}
        \footnotesize{$50\:25\:25$}
      };
      & 
      \node[draw=black, rectangle split,  rectangle split parts=3] (sn0x1058a50){
        \begin{tikzpicture}[scale=.2]
          \node[circle, scale=0.75, fill] (tid0) at (2.25,1.5){};
          \node[circle, scale=0.75, fill] (tid1) at (1.5,3){};
          \node[circle, scale=0.75, fill] (tid3) at (0.75,4.5){};
          \node[circle, scale=0.75, fill, red] (tid6) at (0.75,6){};
          \draw[](tid3) -- (tid6);
          \node[circle, scale=0.75, fill] (tid4) at (2.25,4.5){};
          \draw[](tid1) -- (tid3);
          \draw[](tid1) -- (tid4);
          \node[circle, scale=0.75, fill] (tid2) at (3.75,3){};
          \node[circle, scale=0.75, fill, red] (tid5) at (3.75,4.5){};
          \draw[](tid2) -- (tid5);
          \draw[](tid0) -- (tid1);
          \draw[](tid0) -- (tid2);
        \end{tikzpicture}
        \nodepart{two}
        \footnotesize{4.84375}
        \nodepart{three}
        \footnotesize{$50\:50$}
      };
      & 
      \node[draw=black, rectangle split,  rectangle split parts=3] (sn0x10597a0){
        \begin{tikzpicture}[scale=.2]
          \node[circle, scale=0.75, fill] (tid0) at (3,1.5){};
          \node[circle, scale=0.75, fill] (tid1) at (2.25,3){};
          \node[circle, scale=0.75, fill] (tid3) at (0.75,4.5){};
          \node[circle, scale=0.75, fill, red] (tid6) at (0.75,6){};
          \draw[](tid3) -- (tid6);
          \node[circle, scale=0.75, fill, red] (tid4) at (2.25,4.5){};
          \node[circle, scale=0.75, fill] (tid5) at (3.75,4.5){};
          \draw[](tid1) -- (tid3);
          \draw[](tid1) -- (tid4);
          \draw[](tid1) -- (tid5);
          \node[circle, scale=0.75, fill] (tid2) at (5.25,3){};
          \draw[](tid0) -- (tid1);
          \draw[](tid0) -- (tid2);
        \end{tikzpicture}
        \nodepart{two}
        \footnotesize{4.84375}
        \nodepart{three}
        \footnotesize{$50\:50$}
      };
      & 
      \\
    };
  \end{scope}
  \begin{scope}[yshift=\leveltopIIIIII cm]
    \matrix (line6) [column sep=1cm] {
      \node[draw=black, rectangle split,  rectangle split parts=3] (sn0x1053920){
        \begin{tikzpicture}[scale=.2]
          \node[circle, scale=0.75, fill] (tid0) at (1.5,1.5){};
          \node[circle, scale=0.75, fill] (tid1) at (0.75,3){};
          \node[circle, scale=0.75, fill] (tid3) at (0.75,4.5){};
          \node[circle, scale=0.75, fill] (tid4) at (0.75,6){};
          \node[circle, scale=0.75, fill, red] (tid5) at (0.75,7.5){};
          \draw[](tid4) -- (tid5);
          \draw[](tid3) -- (tid4);
          \draw[](tid1) -- (tid3);
          \node[circle, scale=0.75, fill, red] (tid2) at (2.25,3){};
          \draw[](tid0) -- (tid1);
          \draw[](tid0) -- (tid2);
        \end{tikzpicture}
        \nodepart{two}
        \footnotesize{5.0625}
        \nodepart{three}
        \footnotesize{$50\:50$}
      };
      & 
      \node[draw=black, rectangle split,  rectangle split parts=3] (sn0x1053bc0){
        \begin{tikzpicture}[scale=.2]
          \node[circle, scale=0.75, fill] (tid0) at (1.5,1.5){};
          \node[circle, scale=0.75, fill] (tid1) at (0.75,3){};
          \node[circle, scale=0.75, fill] (tid3) at (0.75,4.5){};
          \node[circle, scale=0.75, fill, red] (tid5) at (0.75,6){};
          \draw[](tid3) -- (tid5);
          \draw[](tid1) -- (tid3);
          \node[circle, scale=0.75, fill] (tid2) at (2.25,3){};
          \node[circle, scale=0.75, fill, red] (tid4) at (2.25,4.5){};
          \draw[](tid2) -- (tid4);
          \draw[](tid0) -- (tid1);
          \draw[](tid0) -- (tid2);
        \end{tikzpicture}
        \nodepart{two}
        \footnotesize{4.4375}
        \nodepart{three}
        \footnotesize{$50\:50$}
      };
      & 
      \node[draw=black, rectangle split,  rectangle split parts=3] (sn0x1056090){
        \begin{tikzpicture}[scale=.2]
          \node[circle, scale=0.75, fill] (tid0) at (2.25,1.5){};
          \node[circle, scale=0.75, fill] (tid1) at (1.5,3){};
          \node[circle, scale=0.75, fill, red] (tid3) at (0.75,4.5){};
          \node[circle, scale=0.75, fill, red] (tid4) at (2.25,4.5){};
          \draw[](tid1) -- (tid3);
          \draw[](tid1) -- (tid4);
          \node[circle, scale=0.75, fill] (tid2) at (3.75,3){};
          \node[circle, scale=0.75, fill] (tid5) at (3.75,4.5){};
          \draw[](tid2) -- (tid5);
          \draw[](tid0) -- (tid1);
          \draw[](tid0) -- (tid2);
        \end{tikzpicture}
        \nodepart{two}
        \footnotesize{4.25}
        \nodepart{three}
        \footnotesize{$1$}
      };
      & 
      \node[draw=black, rectangle split,  rectangle split parts=3] (sn0x1056160){
        \begin{tikzpicture}[scale=.2]
          \node[circle, scale=0.75, fill] (tid0) at (2.25,1.5){};
          \node[circle, scale=0.75, fill] (tid1) at (1.5,3){};
          \node[circle, scale=0.75, fill, red] (tid3) at (0.75,4.5){};
          \node[circle, scale=0.75, fill] (tid4) at (2.25,4.5){};
          \draw[](tid1) -- (tid3);
          \draw[](tid1) -- (tid4);
          \node[circle, scale=0.75, fill] (tid2) at (3.75,3){};
          \node[circle, scale=0.75, fill, red] (tid5) at (3.75,4.5){};
          \draw[](tid2) -- (tid5);
          \draw[](tid0) -- (tid1);
          \draw[](tid0) -- (tid2);
        \end{tikzpicture}
        \nodepart{two}
        \footnotesize{4.25}
        \nodepart{three}
        \footnotesize{$50\:50$}
      };
      & 
      \node[draw=black, rectangle split,  rectangle split parts=3] (sn0x1057630){
        \begin{tikzpicture}[scale=.2]
          \node[circle, scale=0.75, fill] (tid0) at (3,1.5){};
          \node[circle, scale=0.75, fill] (tid1) at (2.25,3){};
          \node[circle, scale=0.75, fill, red] (tid3) at (0.75,4.5){};
          \node[circle, scale=0.75, fill, red] (tid4) at (2.25,4.5){};
          \node[circle, scale=0.75, fill] (tid5) at (3.75,4.5){};
          \draw[](tid1) -- (tid3);
          \draw[](tid1) -- (tid4);
          \draw[](tid1) -- (tid5);
          \node[circle, scale=0.75, fill] (tid2) at (5.25,3){};
          \draw[](tid0) -- (tid1);
          \draw[](tid0) -- (tid2);
        \end{tikzpicture}
        \nodepart{two}
        \footnotesize{4.25}
        \nodepart{three}
        \footnotesize{$1$}
      };
      & 
      \node[draw=black, rectangle split,  rectangle split parts=3] (sn0x1058b20){
        \begin{tikzpicture}[scale=.2]
          \node[circle, scale=0.75, fill] (tid0) at (2.25,1.5){};
          \node[circle, scale=0.75, fill] (tid1) at (1.5,3){};
          \node[circle, scale=0.75, fill] (tid3) at (0.75,4.5){};
          \node[circle, scale=0.75, fill, red] (tid5) at (0.75,6){};
          \draw[](tid3) -- (tid5);
          \node[circle, scale=0.75, fill, red] (tid4) at (2.25,4.5){};
          \draw[](tid1) -- (tid3);
          \draw[](tid1) -- (tid4);
          \node[circle, scale=0.75, fill] (tid2) at (3.75,3){};
          \draw[](tid0) -- (tid1);
          \draw[](tid0) -- (tid2);
        \end{tikzpicture}
        \nodepart{two}
        \footnotesize{4.4375}
        \nodepart{three}
        \footnotesize{$50\:50$}
      };
      & 
      \\
    };
  \end{scope}
  \begin{scope}[yshift=\leveltopIIIIIII cm]
    \matrix (line7) [column sep=1cm] {
      \node[draw=black, rectangle split,  rectangle split parts=3] (sn0x10540d0){
        \begin{tikzpicture}[scale=.2]
          \node[circle, scale=0.75, fill] (tid0) at (0.75,1.5){};
          \node[circle, scale=0.75, fill] (tid1) at (0.75,3){};
          \node[circle, scale=0.75, fill] (tid2) at (0.75,4.5){};
          \node[circle, scale=0.75, fill] (tid3) at (0.75,6){};
          \node[circle, scale=0.75, fill, red] (tid4) at (0.75,7.5){};
          \draw[](tid3) -- (tid4);
          \draw[](tid2) -- (tid3);
          \draw[](tid1) -- (tid2);
          \draw[](tid0) -- (tid1);
        \end{tikzpicture}
        \nodepart{two}
        \footnotesize{5}
        \nodepart{three}
        \footnotesize{$1$}
      };
      & 
      \node[draw=black, rectangle split,  rectangle split parts=3] (sn0x1054480){
        \begin{tikzpicture}[scale=.2]
          \node[circle, scale=0.75, fill] (tid0) at (1.5,1.5){};
          \node[circle, scale=0.75, fill] (tid1) at (0.75,3){};
          \node[circle, scale=0.75, fill] (tid3) at (0.75,4.5){};
          \node[circle, scale=0.75, fill, red] (tid4) at (0.75,6){};
          \draw[](tid3) -- (tid4);
          \draw[](tid1) -- (tid3);
          \node[circle, scale=0.75, fill, red] (tid2) at (2.25,3){};
          \draw[](tid0) -- (tid1);
          \draw[](tid0) -- (tid2);
        \end{tikzpicture}
        \nodepart{two}
        \footnotesize{4.125}
        \nodepart{three}
        \footnotesize{$50\:50$}
      };
      & 
      \node[draw=black, rectangle split,  rectangle split parts=3] (sn0x1055dd0){
        \begin{tikzpicture}[scale=.2]
          \node[circle, scale=0.75, fill] (tid0) at (1.5,1.5){};
          \node[circle, scale=0.75, fill] (tid1) at (0.75,3){};
          \node[circle, scale=0.75, fill, red] (tid3) at (0.75,4.5){};
          \draw[](tid1) -- (tid3);
          \node[circle, scale=0.75, fill] (tid2) at (2.25,3){};
          \node[circle, scale=0.75, fill, red] (tid4) at (2.25,4.5){};
          \draw[](tid2) -- (tid4);
          \draw[](tid0) -- (tid1);
          \draw[](tid0) -- (tid2);
        \end{tikzpicture}
        \nodepart{two}
        \footnotesize{3.75}
        \nodepart{three}
        \footnotesize{$1$}
      };
      & 
      \node[draw=black, rectangle split,  rectangle split parts=3] (sn0x10568c0){
        \begin{tikzpicture}[scale=.2]
          \node[circle, scale=0.75, fill] (tid0) at (2.25,1.5){};
          \node[circle, scale=0.75, fill] (tid1) at (1.5,3){};
          \node[circle, scale=0.75, fill, red] (tid3) at (0.75,4.5){};
          \node[circle, scale=0.75, fill, red] (tid4) at (2.25,4.5){};
          \draw[](tid1) -- (tid3);
          \draw[](tid1) -- (tid4);
          \node[circle, scale=0.75, fill] (tid2) at (3.75,3){};
          \draw[](tid0) -- (tid1);
          \draw[](tid0) -- (tid2);
        \end{tikzpicture}
        \nodepart{two}
        \footnotesize{3.75}
        \nodepart{three}
        \footnotesize{$1$}
      };
      & 
      \\
    };
  \end{scope}
  \begin{scope}[yshift=\leveltopIIIIIIII cm]
    \matrix (line8) [column sep=1cm] {
      \node[draw=black, rectangle split,  rectangle split parts=3] (sn0x1054550){
        \begin{tikzpicture}[scale=.2]
          \node[circle, scale=0.75, fill] (tid0) at (0.75,1.5){};
          \node[circle, scale=0.75, fill] (tid1) at (0.75,3){};
          \node[circle, scale=0.75, fill] (tid2) at (0.75,4.5){};
          \node[circle, scale=0.75, fill, red] (tid3) at (0.75,6){};
          \draw[](tid2) -- (tid3);
          \draw[](tid1) -- (tid2);
          \draw[](tid0) -- (tid1);
        \end{tikzpicture}
        \nodepart{two}
        \footnotesize{4}
        \nodepart{three}
        \footnotesize{$1$}
      };
      & 
      \node[draw=black, rectangle split,  rectangle split parts=3] (sn0x1055270){
        \begin{tikzpicture}[scale=.2]
          \node[circle, scale=0.75, fill] (tid0) at (1.5,1.5){};
          \node[circle, scale=0.75, fill] (tid1) at (0.75,3){};
          \node[circle, scale=0.75, fill, red] (tid3) at (0.75,4.5){};
          \draw[](tid1) -- (tid3);
          \node[circle, scale=0.75, fill, red] (tid2) at (2.25,3){};
          \draw[](tid0) -- (tid1);
          \draw[](tid0) -- (tid2);
        \end{tikzpicture}
        \nodepart{two}
        \footnotesize{3.25}
        \nodepart{three}
        \footnotesize{$50\:50$}
      };
      & 
      \\
    };
  \end{scope}
  \begin{scope}[yshift=\leveltopIIIIIIIII cm]
    \matrix (line9) [column sep=1cm] {
      \node[draw=black, rectangle split,  rectangle split parts=3] (sn0x1054a50){
        \begin{tikzpicture}[scale=.2]
          \node[circle, scale=0.75, fill] (tid0) at (0.75,1.5){};
          \node[circle, scale=0.75, fill] (tid1) at (0.75,3){};
          \node[circle, scale=0.75, fill, red] (tid2) at (0.75,4.5){};
          \draw[](tid1) -- (tid2);
          \draw[](tid0) -- (tid1);
        \end{tikzpicture}
        \nodepart{two}
        \footnotesize{3}
        \nodepart{three}
        \footnotesize{$1$}
      };
      & 
      \node[draw=black, rectangle split,  rectangle split parts=3] (sn0x1054cb0){
        \begin{tikzpicture}[scale=.2]
          \node[circle, scale=0.75, fill] (tid0) at (1.5,1.5){};
          \node[circle, scale=0.75, fill, red] (tid1) at (0.75,3){};
          \node[circle, scale=0.75, fill, red] (tid2) at (2.25,3){};
          \draw[](tid0) -- (tid1);
          \draw[](tid0) -- (tid2);
        \end{tikzpicture}
        \nodepart{two}
        \footnotesize{2.5}
        \nodepart{three}
        \footnotesize{$1$}
      };
      & 
      \\
    };
  \end{scope}
  \begin{scope}[yshift=\leveltopIIIIIIIIII cm]
    \matrix (line10) [column sep=1cm] {
      \node[draw=black, rectangle split,  rectangle split parts=3] (sn0x1054b20){
        \begin{tikzpicture}[scale=.2]
          \node[circle, scale=0.75, fill] (tid0) at (0.75,1.5){};
          \node[circle, scale=0.75, fill, red] (tid1) at (0.75,3){};
          \draw[](tid0) -- (tid1);
        \end{tikzpicture}
        \nodepart{two}
        \footnotesize{2}
        \nodepart{three}
        \footnotesize{$1$}
      };
      & 
      \\
    };
  \end{scope}
  \begin{scope}[yshift=\leveltopIIIIIIIIIII cm]
    \matrix (line11) [column sep=1cm] {
      \node[draw=black, rectangle split,  rectangle split parts=3] (sn0x10547e0){
        \begin{tikzpicture}[scale=.2]
          \node[circle, scale=0.75, fill, red] (tid0) at (0.75,1.5){};
        \end{tikzpicture}
        \nodepart{two}
        \footnotesize{1}
        \nodepart{three}
        \footnotesize{$$}
      };
      & 
      \\
    };
  \end{scope}
  \begin{scope}[yshift=\leveltopIIIIIIIIIIII cm]
    \matrix (line12) [column sep=1cm] {
      \\
    };
  \end{scope}
  \draw (sn0x104f980.south) -- (sn0x1050190.north);
  \draw (sn0x104f980.south) -- (sn0x104cb60.north);
  \draw (sn0x104f980.south) -- (sn0x104dd20.north);
  \draw (sn0x1050190.south) -- (sn0x10519d0.north);
  \draw (sn0x1050190.south) -- (sn0x104fbd0.north);
  \draw (sn0x104cb60.south) -- (sn0x105a080.north);
  \draw (sn0x104cb60.south) -- (sn0x104fbd0.north);
  \draw (sn0x104dd20.south) -- (sn0x104fbd0.north);
  \draw (sn0x10519d0.south) -- (sn0x1052250.north);
  \draw (sn0x10519d0.south) -- (sn0x1052960.north);
  \draw (sn0x104fbd0.south) -- (sn0x1052960.north);
  \draw (sn0x104fbd0.south) -- (sn0x10581e0.north);
  \draw (sn0x104fbd0.south) -- (sn0x1058550.north);
  \draw (sn0x105a080.south) -- (sn0x1052960.north);
  \draw (sn0x1052250.south) -- (sn0x10525f0.north);
  \draw (sn0x1052250.south) -- (sn0x1053850.north);
  \draw (sn0x1052960.south) -- (sn0x1053850.north);
  \draw (sn0x1052960.south) -- (sn0x1056b00.north);
  \draw (sn0x1052960.south) -- (sn0x1056fb0.north);
  \draw (sn0x10581e0.south) -- (sn0x1058f50.north);
  \draw (sn0x10581e0.south) -- (sn0x1058a50.north);
  \draw (sn0x10581e0.south) -- (sn0x1056b00.north);
  \draw (sn0x10581e0.south) -- (sn0x1056fb0.north);
  \draw (sn0x1058550.south) -- (sn0x10597a0.north);
  \draw (sn0x1058550.south) -- (sn0x1056fb0.north);
  \draw (sn0x10525f0.south) -- (sn0x1053920.north);
  \draw (sn0x10525f0.south) -- (sn0x1053bc0.north);
  \draw (sn0x1053850.south) -- (sn0x1053bc0.north);
  \draw (sn0x1053850.south) -- (sn0x1056090.north);
  \draw (sn0x1053850.south) -- (sn0x1056160.north);
  \draw (sn0x1056b00.south) -- (sn0x1056090.north);
  \draw (sn0x1056b00.south) -- (sn0x1056160.north);
  \draw (sn0x1056fb0.south) -- (sn0x1056160.north);
  \draw (sn0x1056fb0.south) -- (sn0x1057630.north);
  \draw (sn0x1058f50.south) -- (sn0x1053bc0.north);
  \draw (sn0x1058f50.south) -- (sn0x1056090.north);
  \draw (sn0x1058f50.south) -- (sn0x1056160.north);
  \draw (sn0x1058a50.south) -- (sn0x1058b20.north);
  \draw (sn0x1058a50.south) -- (sn0x1056160.north);
  \draw (sn0x10597a0.south) -- (sn0x1058b20.north);
  \draw (sn0x10597a0.south) -- (sn0x1057630.north);
  \draw (sn0x1053920.south) -- (sn0x10540d0.north);
  \draw (sn0x1053920.south) -- (sn0x1054480.north);
  \draw (sn0x1053bc0.south) -- (sn0x1054480.north);
  \draw (sn0x1053bc0.south) -- (sn0x1055dd0.north);
  \draw (sn0x1056090.south) -- (sn0x1055dd0.north);
  \draw (sn0x1056160.south) -- (sn0x1055dd0.north);
  \draw (sn0x1056160.south) -- (sn0x10568c0.north);
  \draw (sn0x1057630.south) -- (sn0x10568c0.north);
  \draw (sn0x1058b20.south) -- (sn0x1054480.north);
  \draw (sn0x1058b20.south) -- (sn0x10568c0.north);
  \draw (sn0x10540d0.south) -- (sn0x1054550.north);
  \draw (sn0x1054480.south) -- (sn0x1054550.north);
  \draw (sn0x1054480.south) -- (sn0x1055270.north);
  \draw (sn0x1055dd0.south) -- (sn0x1055270.north);
  \draw (sn0x10568c0.south) -- (sn0x1055270.north);
  \draw (sn0x1054550.south) -- (sn0x1054a50.north);
  \draw (sn0x1055270.south) -- (sn0x1054a50.north);
  \draw (sn0x1055270.south) -- (sn0x1054cb0.north);
  \draw (sn0x1054a50.south) -- (sn0x1054b20.north);
  \draw (sn0x1054cb0.south) -- (sn0x1054b20.north);
  \draw (sn0x1054b20.south) -- (sn0x10547e0.north);

  \newcommand{\nd}[4]{
    \node[draw=black, rectangle split, rectangle split parts=3] (n#1#2) {
      $#1/#2$
      \nodepart{two}
      #3
      \nodepart{three}
      #4
    };
  }

  \begin{scope}[yshift=\leveltopI, xshift=70cm, rectangle, draw=black,anchor=south]
    \matrix (test) [column sep=1cm] {
      \nd{5}{6}{6.82812}{50 50};
      \\
    };
  \end{scope}

  \begin{scope}[yshift=\leveltopII, xshift=70cm, rectangle, draw=black,anchor=south]
    \matrix (test) [column sep=1cm] {
      \nd{5}{5}{6.35938}{50 50};
      &
      \nd{4}{6}{6.29688}{1};
      \\
      };
    \end{scope}

    \begin{scope}[yshift=\leveltopIII, xshift=70cm, rectangle, draw=black,anchor=south]
      \matrix (test) [column sep=1cm] {
        \nd{5}{4}{5.92188}{50 50};
        &
        \nd{4}{5}{5.79688}{1};
        \\
      };
    \end{scope}

    \begin{scope}[yshift=\leveltopIIII, xshift=70cm, rectangle, draw=black,anchor=south]
      \matrix (test) [column sep=1cm] {
        \nd{5}{3}{5.54688}{50 50};
        &
        \nd{4}{4}{5.29688}{50 50};
        \\
      };
    \end{scope}

    \begin{scope}[yshift=\leveltopIIIII, xshift=70cm, rectangle, draw=black,anchor=south]
      \matrix (test) [column sep=1cm] {
        \nd{5}{2}{5.25}{50 50};
        &
        \nd{4}{3}{4.84375}{50 50};
        &
        \nd{3}{4}{5.29688}{1};
        \\
      };
    \end{scope}

    \begin{scope}[yshift=\leveltopIIIIII, xshift=70cm, rectangle, draw=black,anchor=south]
      \matrix (test) [column sep=1cm] {
        \nd{5}{1}{5.0625}{50 50};
        &
        \nd{4}{2}{4.4375}{50 50};
        &
        \nd{3}{3}{5.75}{1};
        \\
      };
    \end{scope}

    \begin{scope}[yshift=\leveltopIIIIIII, xshift=70cm, rectangle, draw=black,anchor=south]
      \matrix (test) [column sep=1cm] {
        \nd{5}{0}{5}{50 50};
        &
        \nd{4}{1}{4.125}{50 50};
        &
        \nd{3}{2}{5.25}{1};
        \\
      };
    \end{scope}
    
    \begin{scope}[yshift=\leveltopIIIIIIII, xshift=70cm, rectangle, draw=black,anchor=south]
      \matrix (test) [column sep=1cm] {
        \nd{4}{0}{4}{1};
        &
        \nd{3}{1}{3.25}{50 50};
        \\
      };
    \end{scope}

    \begin{scope}[yshift=\leveltopIIIIIIIII, xshift=70cm, rectangle, draw=black,anchor=south]
      \matrix (test) [column sep=1cm] {
        \nd{3}{0}{3}{1};
        &
        \nd{2}{1}{2.5}{1};
        \\
      };
    \end{scope}

    \begin{scope}[yshift=\leveltopIIIIIIIIII, xshift=70cm, rectangle, draw=black,anchor=south]
      \matrix (test) [column sep=1cm] {
        \nd{2}{0}{2}{1};
        \\
      };
    \end{scope}
    
    \begin{scope}[yshift=\leveltopIIIIIIIIIII, xshift=70cm, rectangle, draw=black,anchor=south]
      \matrix (test) [column sep=1cm] {
        \nd{1}{0}{1}{1};
        \\
      };
    \end{scope}

    \draw (n56.south) -- (n55.north);
    \draw (n56.south) -- (n46.north);
    \draw (n55.south) -- (n54.north);
    \draw (n55.south) -- (n45.north);
    \draw (n46.south) -- (n45.north);
    \draw (n54.south) -- (n53.north);
    \draw (n54.south) -- (n44.north);
    \draw (n45.south) -- (n44.north);
    \draw (n53.south) -- (n52.north);
    \draw (n53.south) -- (n43.north);
    \draw (n44.south) -- (n43.north);
    \draw (n44.south) -- (n34.north);
    \draw (n52.south) -- (n51.north);
    \draw (n52.south) -- (n42.north);
    \draw (n43.south) -- (n42.north);
    \draw (n43.south) -- (n33.north);
    \draw (n34.south) -- (n33.north);
    \draw (n51.south) -- (n50.north);
    \draw (n51.south) -- (n41.north);
    \draw (n42.south) -- (n41.north);
    \draw (n42.south) -- (n32.north);
    \draw (n33.south) -- (n32.north);
    \draw (n32.south) -- (n31.north);
    \draw (n50.south) -- (n40.north);
    \draw (n41.south) -- (n40.north);
    \draw (n41.south) -- (n31.north);
    \draw (n40.south) -- (n30.north);
    \draw (n31.south) -- (n30.north);
    \draw (n31.south) -- (n21.north);
    \draw (n30.south) -- (n20.north);
    \draw (n21.south) -- (n20.north);
    \draw (n20.south) -- (n10.north);

\end{tikzpicture}

%%% Local Variables:
%%% TeX-master: "thesis/thesis.tex"
%%% End: 
\renewcommand{\leveltopI}{-15cm + \leveltop}
\renewcommand{\leveltopII}{-15cm + \leveltopI}
\renewcommand{\leveltopIII}{-15cm + \leveltopII}
\renewcommand{\leveltopIIII}{-15cm + \leveltopIII}
\renewcommand{\leveltopIIIII}{-15cm + \leveltopIIII}
\renewcommand{\leveltopIIIIII}{-15cm + \leveltopIIIII}
\renewcommand{\leveltopIIIIIII}{-15cm + \leveltopIIIIII}
\renewcommand{\leveltopIIIIIIII}{-15cm + \leveltopIIIIIII}
\renewcommand{\leveltopIIIIIIIII}{-15cm + \leveltopIIIIIIII}
\renewcommand{\leveltopIIIIIIIIII}{-15cm + \leveltopIIIIIIIII}
\renewcommand{\leveltopIIIIIIIIIII}{-15cm + \leveltopIIIIIIIIII}
% \begin{tikzpicture}[scale=.2, anchor=south]
%   \begin{scope}[yshift=\leveltopI cm]
%     \matrix (line1) [column sep=1cm] {
%       \node[draw=black, rectangle split,  rectangle split parts=3] (sn0x1050af0){
%         \begin{tikzpicture}[scale=.2]
%           \node[circle, scale=0.75, fill] (tid0) at (3.75,1.5){};
%           \node[circle, scale=0.75, fill] (tid1) at (2.25,3){};
%           \node[circle, scale=0.75, fill] (tid3) at (0.75,4.5){};
%           \node[circle, scale=0.75, fill, red] (tid7) at (0.75,6){};
%           \draw[](tid3) -- (tid7);
%           \node[circle, scale=0.75, fill] (tid4) at (2.25,4.5){};
%           \node[circle, scale=0.75, fill] (tid5) at (3.75,4.5){};
%           \draw[](tid1) -- (tid3);
%           \draw[](tid1) -- (tid4);
%           \draw[](tid1) -- (tid5);
%           \node[circle, scale=0.75, fill] (tid2) at (6,3){};
%           \node[circle, scale=0.75, fill] (tid6) at (6,4.5){};
%           \node[circle, scale=0.75, fill] (tid8) at (5.25,6){};
%           \node[circle, scale=0.75, fill, red] (tid10) at (5.25,7.5){};
%           \draw[](tid8) -- (tid10);
%           \node[circle, scale=0.75, fill] (tid9) at (6.75,6){};
%           \draw[](tid6) -- (tid8);
%           \draw[](tid6) -- (tid9);
%           \draw[](tid2) -- (tid6);
%           \draw[](tid0) -- (tid1);
%           \draw[](tid0) -- (tid2);
%         \end{tikzpicture}
%         \nodepart{two}
%         \footnotesize{6.82812}
%         \nodepart{three}
%         \footnotesize{$50\:50$}
%       };
%       & 
%       \\
%     };
%   \end{scope}
%   \begin{scope}[yshift=\leveltopII cm]
%     \matrix (line2) [column sep=1cm] {
%       \node[draw=black, rectangle split,  rectangle split parts=3] (sn0x105a150){
%         \begin{tikzpicture}[scale=.2]
%           \node[circle, scale=0.75, fill] (tid0) at (3.75,1.5){};
%           \node[circle, scale=0.75, fill] (tid1) at (1.5,3){};
%           \node[circle, scale=0.75, fill] (tid3) at (1.5,4.5){};
%           \node[circle, scale=0.75, fill] (tid7) at (0.75,6){};
%           \node[circle, scale=0.75, fill, red] (tid9) at (0.75,7.5){};
%           \draw[](tid7) -- (tid9);
%           \node[circle, scale=0.75, fill, red] (tid8) at (2.25,6){};
%           \draw[](tid3) -- (tid7);
%           \draw[](tid3) -- (tid8);
%           \draw[](tid1) -- (tid3);
%           \node[circle, scale=0.75, fill] (tid2) at (5.25,3){};
%           \node[circle, scale=0.75, fill] (tid4) at (3.75,4.5){};
%           \node[circle, scale=0.75, fill] (tid5) at (5.25,4.5){};
%           \node[circle, scale=0.75, fill] (tid6) at (6.75,4.5){};
%           \draw[](tid2) -- (tid4);
%           \draw[](tid2) -- (tid5);
%           \draw[](tid2) -- (tid6);
%           \draw[](tid0) -- (tid1);
%           \draw[](tid0) -- (tid2);
%         \end{tikzpicture}
%         \nodepart{two}
%         \footnotesize{6.35938}
%         \nodepart{three}
%         \footnotesize{$50\:50$}
%       };
%       & 
%       \node[draw=black, rectangle split,  rectangle split parts=3] (sn0x104cb60){
%         \begin{tikzpicture}[scale=.2]
%           \node[circle, scale=0.75, fill] (tid0) at (3.75,1.5){};
%           \node[circle, scale=0.75, fill] (tid1) at (2.25,3){};
%           \node[circle, scale=0.75, fill] (tid3) at (0.75,4.5){};
%           \node[circle, scale=0.75, fill, red] (tid7) at (0.75,6){};
%           \draw[](tid3) -- (tid7);
%           \node[circle, scale=0.75, fill] (tid4) at (2.25,4.5){};
%           \node[circle, scale=0.75, fill] (tid5) at (3.75,4.5){};
%           \draw[](tid1) -- (tid3);
%           \draw[](tid1) -- (tid4);
%           \draw[](tid1) -- (tid5);
%           \node[circle, scale=0.75, fill] (tid2) at (6,3){};
%           \node[circle, scale=0.75, fill] (tid6) at (6,4.5){};
%           \node[circle, scale=0.75, fill, red] (tid8) at (5.25,6){};
%           \node[circle, scale=0.75, fill] (tid9) at (6.75,6){};
%           \draw[](tid6) -- (tid8);
%           \draw[](tid6) -- (tid9);
%           \draw[](tid2) -- (tid6);
%           \draw[](tid0) -- (tid1);
%           \draw[](tid0) -- (tid2);
%         \end{tikzpicture}
%         \nodepart{two}
%         \footnotesize{6.29688}
%         \nodepart{three}
%         \footnotesize{$50\:50$}
%       };
%       & 
%       \\
%     };
%   \end{scope}
%   \begin{scope}[yshift=\leveltopIII cm]
%     \matrix (line3) [column sep=1cm] {
%       \node[draw=black, rectangle split,  rectangle split parts=3] (sn0x10519d0){
%         \begin{tikzpicture}[scale=.2]
%           \node[circle, scale=0.75, fill] (tid0) at (3,1.5){};
%           \node[circle, scale=0.75, fill] (tid1) at (2.25,3){};
%           \node[circle, scale=0.75, fill, red] (tid3) at (0.75,4.5){};
%           \node[circle, scale=0.75, fill] (tid4) at (2.25,4.5){};
%           \node[circle, scale=0.75, fill] (tid5) at (3.75,4.5){};
%           \draw[](tid1) -- (tid3);
%           \draw[](tid1) -- (tid4);
%           \draw[](tid1) -- (tid5);
%           \node[circle, scale=0.75, fill] (tid2) at (5.25,3){};
%           \node[circle, scale=0.75, fill] (tid6) at (5.25,4.5){};
%           \node[circle, scale=0.75, fill] (tid7) at (5.25,6){};
%           \node[circle, scale=0.75, fill, red] (tid8) at (5.25,7.5){};
%           \draw[](tid7) -- (tid8);
%           \draw[](tid6) -- (tid7);
%           \draw[](tid2) -- (tid6);
%           \draw[](tid0) -- (tid1);
%           \draw[](tid0) -- (tid2);
%         \end{tikzpicture}
%         \nodepart{two}
%         \footnotesize{5.92188}
%         \nodepart{three}
%         \footnotesize{$50\:50$}
%       };
%       & 
%       \node[draw=black, rectangle split,  rectangle split parts=3] (sn0x105a080){
%         \begin{tikzpicture}[scale=.2]
%           \node[circle, scale=0.75, fill] (tid0) at (3.75,1.5){};
%           \node[circle, scale=0.75, fill] (tid1) at (2.25,3){};
%           \node[circle, scale=0.75, fill] (tid3) at (0.75,4.5){};
%           \node[circle, scale=0.75, fill] (tid4) at (2.25,4.5){};
%           \node[circle, scale=0.75, fill] (tid5) at (3.75,4.5){};
%           \draw[](tid1) -- (tid3);
%           \draw[](tid1) -- (tid4);
%           \draw[](tid1) -- (tid5);
%           \node[circle, scale=0.75, fill] (tid2) at (6,3){};
%           \node[circle, scale=0.75, fill] (tid6) at (6,4.5){};
%           \node[circle, scale=0.75, fill, red] (tid7) at (5.25,6){};
%           \node[circle, scale=0.75, fill, red] (tid8) at (6.75,6){};
%           \draw[](tid6) -- (tid7);
%           \draw[](tid6) -- (tid8);
%           \draw[](tid2) -- (tid6);
%           \draw[](tid0) -- (tid1);
%           \draw[](tid0) -- (tid2);
%         \end{tikzpicture}
%         \nodepart{two}
%         \footnotesize{5.79688}
%         \nodepart{three}
%         \footnotesize{$1$}
%       };
%       & 
%       \node[draw=black, rectangle split,  rectangle split parts=3] (sn0x104fbd0){
%         \begin{tikzpicture}[scale=.2]
%           \node[circle, scale=0.75, fill] (tid0) at (3,1.5){};
%           \node[circle, scale=0.75, fill] (tid1) at (2.25,3){};
%           \node[circle, scale=0.75, fill] (tid3) at (0.75,4.5){};
%           \node[circle, scale=0.75, fill, red] (tid7) at (0.75,6){};
%           \draw[](tid3) -- (tid7);
%           \node[circle, scale=0.75, fill] (tid4) at (2.25,4.5){};
%           \node[circle, scale=0.75, fill] (tid5) at (3.75,4.5){};
%           \draw[](tid1) -- (tid3);
%           \draw[](tid1) -- (tid4);
%           \draw[](tid1) -- (tid5);
%           \node[circle, scale=0.75, fill] (tid2) at (5.25,3){};
%           \node[circle, scale=0.75, fill] (tid6) at (5.25,4.5){};
%           \node[circle, scale=0.75, fill, red] (tid8) at (5.25,6){};
%           \draw[](tid6) -- (tid8);
%           \draw[](tid2) -- (tid6);
%           \draw[](tid0) -- (tid1);
%           \draw[](tid0) -- (tid2);
%         \end{tikzpicture}
%         \nodepart{two}
%         \footnotesize{5.79688}
%         \nodepart{three}
%         \footnotesize{$50\:33\:17$}
%       };
%       & 
%       \\
%     };
%   \end{scope}
%   \begin{scope}[yshift=\leveltopIIII cm]
%     \matrix (line4) [column sep=1cm] {
%       \node[draw=black, rectangle split,  rectangle split parts=3] (sn0x1052250){
%         \begin{tikzpicture}[scale=.2]
%           \node[circle, scale=0.75, fill] (tid0) at (2.25,1.5){};
%           \node[circle, scale=0.75, fill] (tid1) at (0.75,3){};
%           \node[circle, scale=0.75, fill] (tid3) at (0.75,4.5){};
%           \node[circle, scale=0.75, fill] (tid6) at (0.75,6){};
%           \node[circle, scale=0.75, fill, red] (tid7) at (0.75,7.5){};
%           \draw[](tid6) -- (tid7);
%           \draw[](tid3) -- (tid6);
%           \draw[](tid1) -- (tid3);
%           \node[circle, scale=0.75, fill] (tid2) at (3,3){};
%           \node[circle, scale=0.75, fill, red] (tid4) at (2.25,4.5){};
%           \node[circle, scale=0.75, fill] (tid5) at (3.75,4.5){};
%           \draw[](tid2) -- (tid4);
%           \draw[](tid2) -- (tid5);
%           \draw[](tid0) -- (tid1);
%           \draw[](tid0) -- (tid2);
%         \end{tikzpicture}
%         \nodepart{two}
%         \footnotesize{5.54688}
%         \nodepart{three}
%         \footnotesize{$50\:50$}
%       };
%       & 
%       \node[draw=black, rectangle split,  rectangle split parts=3] (sn0x1052960){
%         \begin{tikzpicture}[scale=.2]
%           \node[circle, scale=0.75, fill] (tid0) at (3,1.5){};
%           \node[circle, scale=0.75, fill] (tid1) at (2.25,3){};
%           \node[circle, scale=0.75, fill, red] (tid3) at (0.75,4.5){};
%           \node[circle, scale=0.75, fill] (tid4) at (2.25,4.5){};
%           \node[circle, scale=0.75, fill] (tid5) at (3.75,4.5){};
%           \draw[](tid1) -- (tid3);
%           \draw[](tid1) -- (tid4);
%           \draw[](tid1) -- (tid5);
%           \node[circle, scale=0.75, fill] (tid2) at (5.25,3){};
%           \node[circle, scale=0.75, fill] (tid6) at (5.25,4.5){};
%           \node[circle, scale=0.75, fill, red] (tid7) at (5.25,6){};
%           \draw[](tid6) -- (tid7);
%           \draw[](tid2) -- (tid6);
%           \draw[](tid0) -- (tid1);
%           \draw[](tid0) -- (tid2);
%         \end{tikzpicture}
%         \nodepart{two}
%         \footnotesize{5.29688}
%         \nodepart{three}
%         \footnotesize{$50\:33\:17$}
%       };
%       & 
%       \node[draw=black, rectangle split,  rectangle split parts=3] (sn0x10581e0){
%         \begin{tikzpicture}[scale=.2]
%           \node[circle, scale=0.75, fill] (tid0) at (3,1.5){};
%           \node[circle, scale=0.75, fill] (tid1) at (2.25,3){};
%           \node[circle, scale=0.75, fill] (tid3) at (0.75,4.5){};
%           \node[circle, scale=0.75, fill, red] (tid7) at (0.75,6){};
%           \draw[](tid3) -- (tid7);
%           \node[circle, scale=0.75, fill, red] (tid4) at (2.25,4.5){};
%           \node[circle, scale=0.75, fill] (tid5) at (3.75,4.5){};
%           \draw[](tid1) -- (tid3);
%           \draw[](tid1) -- (tid4);
%           \draw[](tid1) -- (tid5);
%           \node[circle, scale=0.75, fill] (tid2) at (5.25,3){};
%           \node[circle, scale=0.75, fill] (tid6) at (5.25,4.5){};
%           \draw[](tid2) -- (tid6);
%           \draw[](tid0) -- (tid1);
%           \draw[](tid0) -- (tid2);
%         \end{tikzpicture}
%         \nodepart{two}
%         \footnotesize{5.29688}
%         \nodepart{three}
%         \footnotesize{$33\:17\:25\:25$}
%       };
%       & 
%       \node[draw=black, rectangle split,  rectangle split parts=3] (sn0x1058550){
%         \begin{tikzpicture}[scale=.2]
%           \node[circle, scale=0.75, fill] (tid0) at (3,1.5){};
%           \node[circle, scale=0.75, fill] (tid1) at (2.25,3){};
%           \node[circle, scale=0.75, fill] (tid3) at (0.75,4.5){};
%           \node[circle, scale=0.75, fill, red] (tid7) at (0.75,6){};
%           \draw[](tid3) -- (tid7);
%           \node[circle, scale=0.75, fill] (tid4) at (2.25,4.5){};
%           \node[circle, scale=0.75, fill] (tid5) at (3.75,4.5){};
%           \draw[](tid1) -- (tid3);
%           \draw[](tid1) -- (tid4);
%           \draw[](tid1) -- (tid5);
%           \node[circle, scale=0.75, fill] (tid2) at (5.25,3){};
%           \node[circle, scale=0.75, fill, red] (tid6) at (5.25,4.5){};
%           \draw[](tid2) -- (tid6);
%           \draw[](tid0) -- (tid1);
%           \draw[](tid0) -- (tid2);
%         \end{tikzpicture}
%         \nodepart{two}
%         \footnotesize{5.29688}
%         \nodepart{three}
%         \footnotesize{$50\:50$}
%       };
%       & 
%       \\
%     };
%   \end{scope}
%   \begin{scope}[yshift=\leveltopIIIII cm]
%     \matrix (line5) [column sep=1cm] {
%       \node[draw=black, rectangle split,  rectangle split parts=3] (sn0x10525f0){
%         \begin{tikzpicture}[scale=.2]
%           \node[circle, scale=0.75, fill] (tid0) at (1.5,1.5){};
%           \node[circle, scale=0.75, fill] (tid1) at (0.75,3){};
%           \node[circle, scale=0.75, fill] (tid3) at (0.75,4.5){};
%           \node[circle, scale=0.75, fill] (tid5) at (0.75,6){};
%           \node[circle, scale=0.75, fill, red] (tid6) at (0.75,7.5){};
%           \draw[](tid5) -- (tid6);
%           \draw[](tid3) -- (tid5);
%           \draw[](tid1) -- (tid3);
%           \node[circle, scale=0.75, fill] (tid2) at (2.25,3){};
%           \node[circle, scale=0.75, fill, red] (tid4) at (2.25,4.5){};
%           \draw[](tid2) -- (tid4);
%           \draw[](tid0) -- (tid1);
%           \draw[](tid0) -- (tid2);
%         \end{tikzpicture}
%         \nodepart{two}
%         \footnotesize{5.25}
%         \nodepart{three}
%         \footnotesize{$50\:50$}
%       };
%       & 
%       \node[draw=black, rectangle split,  rectangle split parts=3] (sn0x1053850){
%         \begin{tikzpicture}[scale=.2]
%           \node[circle, scale=0.75, fill] (tid0) at (2.25,1.5){};
%           \node[circle, scale=0.75, fill] (tid1) at (1.5,3){};
%           \node[circle, scale=0.75, fill, red] (tid3) at (0.75,4.5){};
%           \node[circle, scale=0.75, fill] (tid4) at (2.25,4.5){};
%           \draw[](tid1) -- (tid3);
%           \draw[](tid1) -- (tid4);
%           \node[circle, scale=0.75, fill] (tid2) at (3.75,3){};
%           \node[circle, scale=0.75, fill] (tid5) at (3.75,4.5){};
%           \node[circle, scale=0.75, fill, red] (tid6) at (3.75,6){};
%           \draw[](tid5) -- (tid6);
%           \draw[](tid2) -- (tid5);
%           \draw[](tid0) -- (tid1);
%           \draw[](tid0) -- (tid2);
%         \end{tikzpicture}
%         \nodepart{two}
%         \footnotesize{4.84375}
%         \nodepart{three}
%         \footnotesize{$50\:25\:25$}
%       };
%       & 
%       \node[draw=black, rectangle split,  rectangle split parts=3] (sn0x1056b00){
%         \begin{tikzpicture}[scale=.2]
%           \node[circle, scale=0.75, fill] (tid0) at (3,1.5){};
%           \node[circle, scale=0.75, fill] (tid1) at (2.25,3){};
%           \node[circle, scale=0.75, fill, red] (tid3) at (0.75,4.5){};
%           \node[circle, scale=0.75, fill, red] (tid4) at (2.25,4.5){};
%           \node[circle, scale=0.75, fill] (tid5) at (3.75,4.5){};
%           \draw[](tid1) -- (tid3);
%           \draw[](tid1) -- (tid4);
%           \draw[](tid1) -- (tid5);
%           \node[circle, scale=0.75, fill] (tid2) at (5.25,3){};
%           \node[circle, scale=0.75, fill] (tid6) at (5.25,4.5){};
%           \draw[](tid2) -- (tid6);
%           \draw[](tid0) -- (tid1);
%           \draw[](tid0) -- (tid2);
%         \end{tikzpicture}
%         \nodepart{two}
%         \footnotesize{4.75}
%         \nodepart{three}
%         \footnotesize{$50\:50$}
%       };
%       & 
%       \node[draw=black, rectangle split,  rectangle split parts=3] (sn0x1056fb0){
%         \begin{tikzpicture}[scale=.2]
%           \node[circle, scale=0.75, fill] (tid0) at (3,1.5){};
%           \node[circle, scale=0.75, fill] (tid1) at (2.25,3){};
%           \node[circle, scale=0.75, fill, red] (tid3) at (0.75,4.5){};
%           \node[circle, scale=0.75, fill] (tid4) at (2.25,4.5){};
%           \node[circle, scale=0.75, fill] (tid5) at (3.75,4.5){};
%           \draw[](tid1) -- (tid3);
%           \draw[](tid1) -- (tid4);
%           \draw[](tid1) -- (tid5);
%           \node[circle, scale=0.75, fill] (tid2) at (5.25,3){};
%           \node[circle, scale=0.75, fill, red] (tid6) at (5.25,4.5){};
%           \draw[](tid2) -- (tid6);
%           \draw[](tid0) -- (tid1);
%           \draw[](tid0) -- (tid2);
%         \end{tikzpicture}
%         \nodepart{two}
%         \footnotesize{4.75}
%         \nodepart{three}
%         \footnotesize{$50\:50$}
%       };
%       & 
%       \node[draw=black, rectangle split,  rectangle split parts=3] (sn0x1058f50){
%         \begin{tikzpicture}[scale=.2]
%           \node[circle, scale=0.75, fill] (tid0) at (2.25,1.5){};
%           \node[circle, scale=0.75, fill] (tid1) at (1.5,3){};
%           \node[circle, scale=0.75, fill] (tid3) at (0.75,4.5){};
%           \node[circle, scale=0.75, fill, red] (tid6) at (0.75,6){};
%           \draw[](tid3) -- (tid6);
%           \node[circle, scale=0.75, fill, red] (tid4) at (2.25,4.5){};
%           \draw[](tid1) -- (tid3);
%           \draw[](tid1) -- (tid4);
%           \node[circle, scale=0.75, fill] (tid2) at (3.75,3){};
%           \node[circle, scale=0.75, fill] (tid5) at (3.75,4.5){};
%           \draw[](tid2) -- (tid5);
%           \draw[](tid0) -- (tid1);
%           \draw[](tid0) -- (tid2);
%         \end{tikzpicture}
%         \nodepart{two}
%         \footnotesize{4.84375}
%         \nodepart{three}
%         \footnotesize{$50\:25\:25$}
%       };
%       & 
%       \node[draw=black, rectangle split,  rectangle split parts=3] (sn0x1058a50){
%         \begin{tikzpicture}[scale=.2]
%           \node[circle, scale=0.75, fill] (tid0) at (2.25,1.5){};
%           \node[circle, scale=0.75, fill] (tid1) at (1.5,3){};
%           \node[circle, scale=0.75, fill] (tid3) at (0.75,4.5){};
%           \node[circle, scale=0.75, fill, red] (tid6) at (0.75,6){};
%           \draw[](tid3) -- (tid6);
%           \node[circle, scale=0.75, fill] (tid4) at (2.25,4.5){};
%           \draw[](tid1) -- (tid3);
%           \draw[](tid1) -- (tid4);
%           \node[circle, scale=0.75, fill] (tid2) at (3.75,3){};
%           \node[circle, scale=0.75, fill, red] (tid5) at (3.75,4.5){};
%           \draw[](tid2) -- (tid5);
%           \draw[](tid0) -- (tid1);
%           \draw[](tid0) -- (tid2);
%         \end{tikzpicture}
%         \nodepart{two}
%         \footnotesize{4.84375}
%         \nodepart{three}
%         \footnotesize{$50\:50$}
%       };
%       & 
%       \node[draw=black, rectangle split,  rectangle split parts=3] (sn0x10597a0){
%         \begin{tikzpicture}[scale=.2]
%           \node[circle, scale=0.75, fill] (tid0) at (3,1.5){};
%           \node[circle, scale=0.75, fill] (tid1) at (2.25,3){};
%           \node[circle, scale=0.75, fill] (tid3) at (0.75,4.5){};
%           \node[circle, scale=0.75, fill, red] (tid6) at (0.75,6){};
%           \draw[](tid3) -- (tid6);
%           \node[circle, scale=0.75, fill, red] (tid4) at (2.25,4.5){};
%           \node[circle, scale=0.75, fill] (tid5) at (3.75,4.5){};
%           \draw[](tid1) -- (tid3);
%           \draw[](tid1) -- (tid4);
%           \draw[](tid1) -- (tid5);
%           \node[circle, scale=0.75, fill] (tid2) at (5.25,3){};
%           \draw[](tid0) -- (tid1);
%           \draw[](tid0) -- (tid2);
%         \end{tikzpicture}
%         \nodepart{two}
%         \footnotesize{4.84375}
%         \nodepart{three}
%         \footnotesize{$50\:50$}
%       };
%       & 
%       \\
%     };
%   \end{scope}
%   \begin{scope}[yshift=\leveltopIIIIII cm]
%     \matrix (line6) [column sep=1cm] {
%       \node[draw=black, rectangle split,  rectangle split parts=3] (sn0x1053920){
%         \begin{tikzpicture}[scale=.2]
%           \node[circle, scale=0.75, fill] (tid0) at (1.5,1.5){};
%           \node[circle, scale=0.75, fill] (tid1) at (0.75,3){};
%           \node[circle, scale=0.75, fill] (tid3) at (0.75,4.5){};
%           \node[circle, scale=0.75, fill] (tid4) at (0.75,6){};
%           \node[circle, scale=0.75, fill, red] (tid5) at (0.75,7.5){};
%           \draw[](tid4) -- (tid5);
%           \draw[](tid3) -- (tid4);
%           \draw[](tid1) -- (tid3);
%           \node[circle, scale=0.75, fill, red] (tid2) at (2.25,3){};
%           \draw[](tid0) -- (tid1);
%           \draw[](tid0) -- (tid2);
%         \end{tikzpicture}
%         \nodepart{two}
%         \footnotesize{5.0625}
%         \nodepart{three}
%         \footnotesize{$50\:50$}
%       };
%       & 
%       \node[draw=black, rectangle split,  rectangle split parts=3] (sn0x1053bc0){
%         \begin{tikzpicture}[scale=.2]
%           \node[circle, scale=0.75, fill] (tid0) at (1.5,1.5){};
%           \node[circle, scale=0.75, fill] (tid1) at (0.75,3){};
%           \node[circle, scale=0.75, fill] (tid3) at (0.75,4.5){};
%           \node[circle, scale=0.75, fill, red] (tid5) at (0.75,6){};
%           \draw[](tid3) -- (tid5);
%           \draw[](tid1) -- (tid3);
%           \node[circle, scale=0.75, fill] (tid2) at (2.25,3){};
%           \node[circle, scale=0.75, fill, red] (tid4) at (2.25,4.5){};
%           \draw[](tid2) -- (tid4);
%           \draw[](tid0) -- (tid1);
%           \draw[](tid0) -- (tid2);
%         \end{tikzpicture}
%         \nodepart{two}
%         \footnotesize{4.4375}
%         \nodepart{three}
%         \footnotesize{$50\:50$}
%       };
%       & 
%       \node[draw=black, rectangle split,  rectangle split parts=3] (sn0x1056090){
%         \begin{tikzpicture}[scale=.2]
%           \node[circle, scale=0.75, fill] (tid0) at (2.25,1.5){};
%           \node[circle, scale=0.75, fill] (tid1) at (1.5,3){};
%           \node[circle, scale=0.75, fill, red] (tid3) at (0.75,4.5){};
%           \node[circle, scale=0.75, fill, red] (tid4) at (2.25,4.5){};
%           \draw[](tid1) -- (tid3);
%           \draw[](tid1) -- (tid4);
%           \node[circle, scale=0.75, fill] (tid2) at (3.75,3){};
%           \node[circle, scale=0.75, fill] (tid5) at (3.75,4.5){};
%           \draw[](tid2) -- (tid5);
%           \draw[](tid0) -- (tid1);
%           \draw[](tid0) -- (tid2);
%         \end{tikzpicture}
%         \nodepart{two}
%         \footnotesize{4.25}
%         \nodepart{three}
%         \footnotesize{$1$}
%       };
%       & 
%       \node[draw=black, rectangle split,  rectangle split parts=3] (sn0x1056160){
%         \begin{tikzpicture}[scale=.2]
%           \node[circle, scale=0.75, fill] (tid0) at (2.25,1.5){};
%           \node[circle, scale=0.75, fill] (tid1) at (1.5,3){};
%           \node[circle, scale=0.75, fill, red] (tid3) at (0.75,4.5){};
%           \node[circle, scale=0.75, fill] (tid4) at (2.25,4.5){};
%           \draw[](tid1) -- (tid3);
%           \draw[](tid1) -- (tid4);
%           \node[circle, scale=0.75, fill] (tid2) at (3.75,3){};
%           \node[circle, scale=0.75, fill, red] (tid5) at (3.75,4.5){};
%           \draw[](tid2) -- (tid5);
%           \draw[](tid0) -- (tid1);
%           \draw[](tid0) -- (tid2);
%         \end{tikzpicture}
%         \nodepart{two}
%         \footnotesize{4.25}
%         \nodepart{three}
%         \footnotesize{$50\:50$}
%       };
%       & 
%       \node[draw=black, rectangle split,  rectangle split parts=3] (sn0x1057630){
%         \begin{tikzpicture}[scale=.2]
%           \node[circle, scale=0.75, fill] (tid0) at (3,1.5){};
%           \node[circle, scale=0.75, fill] (tid1) at (2.25,3){};
%           \node[circle, scale=0.75, fill, red] (tid3) at (0.75,4.5){};
%           \node[circle, scale=0.75, fill, red] (tid4) at (2.25,4.5){};
%           \node[circle, scale=0.75, fill] (tid5) at (3.75,4.5){};
%           \draw[](tid1) -- (tid3);
%           \draw[](tid1) -- (tid4);
%           \draw[](tid1) -- (tid5);
%           \node[circle, scale=0.75, fill] (tid2) at (5.25,3){};
%           \draw[](tid0) -- (tid1);
%           \draw[](tid0) -- (tid2);
%         \end{tikzpicture}
%         \nodepart{two}
%         \footnotesize{4.25}
%         \nodepart{three}
%         \footnotesize{$1$}
%       };
%       & 
%       \node[draw=black, rectangle split,  rectangle split parts=3] (sn0x1058b20){
%         \begin{tikzpicture}[scale=.2]
%           \node[circle, scale=0.75, fill] (tid0) at (2.25,1.5){};
%           \node[circle, scale=0.75, fill] (tid1) at (1.5,3){};
%           \node[circle, scale=0.75, fill] (tid3) at (0.75,4.5){};
%           \node[circle, scale=0.75, fill, red] (tid5) at (0.75,6){};
%           \draw[](tid3) -- (tid5);
%           \node[circle, scale=0.75, fill, red] (tid4) at (2.25,4.5){};
%           \draw[](tid1) -- (tid3);
%           \draw[](tid1) -- (tid4);
%           \node[circle, scale=0.75, fill] (tid2) at (3.75,3){};
%           \draw[](tid0) -- (tid1);
%           \draw[](tid0) -- (tid2);
%         \end{tikzpicture}
%         \nodepart{two}
%         \footnotesize{4.4375}
%         \nodepart{three}
%         \footnotesize{$50\:50$}
%       };
%       & 
%       \\
%     };
%   \end{scope}
%   \begin{scope}[yshift=\leveltopIIIIIII cm]
%     \matrix (line7) [column sep=1cm] {
%       \node[draw=black, rectangle split,  rectangle split parts=3] (sn0x10540d0){
%         \begin{tikzpicture}[scale=.2]
%           \node[circle, scale=0.75, fill] (tid0) at (0.75,1.5){};
%           \node[circle, scale=0.75, fill] (tid1) at (0.75,3){};
%           \node[circle, scale=0.75, fill] (tid2) at (0.75,4.5){};
%           \node[circle, scale=0.75, fill] (tid3) at (0.75,6){};
%           \node[circle, scale=0.75, fill, red] (tid4) at (0.75,7.5){};
%           \draw[](tid3) -- (tid4);
%           \draw[](tid2) -- (tid3);
%           \draw[](tid1) -- (tid2);
%           \draw[](tid0) -- (tid1);
%         \end{tikzpicture}
%         \nodepart{two}
%         \footnotesize{5}
%         \nodepart{three}
%         \footnotesize{$1$}
%       };
%       & 
%       \node[draw=black, rectangle split,  rectangle split parts=3] (sn0x1054480){
%         \begin{tikzpicture}[scale=.2]
%           \node[circle, scale=0.75, fill] (tid0) at (1.5,1.5){};
%           \node[circle, scale=0.75, fill] (tid1) at (0.75,3){};
%           \node[circle, scale=0.75, fill] (tid3) at (0.75,4.5){};
%           \node[circle, scale=0.75, fill, red] (tid4) at (0.75,6){};
%           \draw[](tid3) -- (tid4);
%           \draw[](tid1) -- (tid3);
%           \node[circle, scale=0.75, fill, red] (tid2) at (2.25,3){};
%           \draw[](tid0) -- (tid1);
%           \draw[](tid0) -- (tid2);
%         \end{tikzpicture}
%         \nodepart{two}
%         \footnotesize{4.125}
%         \nodepart{three}
%         \footnotesize{$50\:50$}
%       };
%       & 
%       \node[draw=black, rectangle split,  rectangle split parts=3] (sn0x1055dd0){
%         \begin{tikzpicture}[scale=.2]
%           \node[circle, scale=0.75, fill] (tid0) at (1.5,1.5){};
%           \node[circle, scale=0.75, fill] (tid1) at (0.75,3){};
%           \node[circle, scale=0.75, fill, red] (tid3) at (0.75,4.5){};
%           \draw[](tid1) -- (tid3);
%           \node[circle, scale=0.75, fill] (tid2) at (2.25,3){};
%           \node[circle, scale=0.75, fill, red] (tid4) at (2.25,4.5){};
%           \draw[](tid2) -- (tid4);
%           \draw[](tid0) -- (tid1);
%           \draw[](tid0) -- (tid2);
%         \end{tikzpicture}
%         \nodepart{two}
%         \footnotesize{3.75}
%         \nodepart{three}
%         \footnotesize{$1$}
%       };
%       & 
%       \node[draw=black, rectangle split,  rectangle split parts=3] (sn0x10568c0){
%         \begin{tikzpicture}[scale=.2]
%           \node[circle, scale=0.75, fill] (tid0) at (2.25,1.5){};
%           \node[circle, scale=0.75, fill] (tid1) at (1.5,3){};
%           \node[circle, scale=0.75, fill, red] (tid3) at (0.75,4.5){};
%           \node[circle, scale=0.75, fill, red] (tid4) at (2.25,4.5){};
%           \draw[](tid1) -- (tid3);
%           \draw[](tid1) -- (tid4);
%           \node[circle, scale=0.75, fill] (tid2) at (3.75,3){};
%           \draw[](tid0) -- (tid1);
%           \draw[](tid0) -- (tid2);
%         \end{tikzpicture}
%         \nodepart{two}
%         \footnotesize{3.75}
%         \nodepart{three}
%         \footnotesize{$1$}
%       };
%       & 
%       \\
%     };
%   \end{scope}
%   \begin{scope}[yshift=\leveltopIIIIIIII cm]
%     \matrix (line8) [column sep=1cm] {
%       \node[draw=black, rectangle split,  rectangle split parts=3] (sn0x1054550){
%         \begin{tikzpicture}[scale=.2]
%           \node[circle, scale=0.75, fill] (tid0) at (0.75,1.5){};
%           \node[circle, scale=0.75, fill] (tid1) at (0.75,3){};
%           \node[circle, scale=0.75, fill] (tid2) at (0.75,4.5){};
%           \node[circle, scale=0.75, fill, red] (tid3) at (0.75,6){};
%           \draw[](tid2) -- (tid3);
%           \draw[](tid1) -- (tid2);
%           \draw[](tid0) -- (tid1);
%         \end{tikzpicture}
%         \nodepart{two}
%         \footnotesize{4}
%         \nodepart{three}
%         \footnotesize{$1$}
%       };
%       & 
%       \node[draw=black, rectangle split,  rectangle split parts=3] (sn0x1055270){
%         \begin{tikzpicture}[scale=.2]
%           \node[circle, scale=0.75, fill] (tid0) at (1.5,1.5){};
%           \node[circle, scale=0.75, fill] (tid1) at (0.75,3){};
%           \node[circle, scale=0.75, fill, red] (tid3) at (0.75,4.5){};
%           \draw[](tid1) -- (tid3);
%           \node[circle, scale=0.75, fill, red] (tid2) at (2.25,3){};
%           \draw[](tid0) -- (tid1);
%           \draw[](tid0) -- (tid2);
%         \end{tikzpicture}
%         \nodepart{two}
%         \footnotesize{3.25}
%         \nodepart{three}
%         \footnotesize{$50\:50$}
%       };
%       & 
%       \\
%     };
%   \end{scope}
%   \begin{scope}[yshift=\leveltopIIIIIIIII cm]
%     \matrix (line9) [column sep=1cm] {
%       \node[draw=black, rectangle split,  rectangle split parts=3] (sn0x1054a50){
%         \begin{tikzpicture}[scale=.2]
%           \node[circle, scale=0.75, fill] (tid0) at (0.75,1.5){};
%           \node[circle, scale=0.75, fill] (tid1) at (0.75,3){};
%           \node[circle, scale=0.75, fill, red] (tid2) at (0.75,4.5){};
%           \draw[](tid1) -- (tid2);
%           \draw[](tid0) -- (tid1);
%         \end{tikzpicture}
%         \nodepart{two}
%         \footnotesize{3}
%         \nodepart{three}
%         \footnotesize{$1$}
%       };
%       & 
%       \node[draw=black, rectangle split,  rectangle split parts=3] (sn0x1054cb0){
%         \begin{tikzpicture}[scale=.2]
%           \node[circle, scale=0.75, fill] (tid0) at (1.5,1.5){};
%           \node[circle, scale=0.75, fill, red] (tid1) at (0.75,3){};
%           \node[circle, scale=0.75, fill, red] (tid2) at (2.25,3){};
%           \draw[](tid0) -- (tid1);
%           \draw[](tid0) -- (tid2);
%         \end{tikzpicture}
%         \nodepart{two}
%         \footnotesize{2.5}
%         \nodepart{three}
%         \footnotesize{$1$}
%       };
%       & 
%       \\
%     };
%   \end{scope}
%   \begin{scope}[yshift=\leveltopIIIIIIIIII cm]
%     \matrix (line10) [column sep=1cm] {
%       \node[draw=black, rectangle split,  rectangle split parts=3] (sn0x1054b20){
%         \begin{tikzpicture}[scale=.2]
%           \node[circle, scale=0.75, fill] (tid0) at (0.75,1.5){};
%           \node[circle, scale=0.75, fill, red] (tid1) at (0.75,3){};
%           \draw[](tid0) -- (tid1);
%         \end{tikzpicture}
%         \nodepart{two}
%         \footnotesize{2}
%         \nodepart{three}
%         \footnotesize{$1$}
%       };
%       & 
%       \\
%     };
%   \end{scope}
%   \begin{scope}[yshift=\leveltopIIIIIIIIIII cm]
%     \matrix (line11) [column sep=1cm] {
%       \node[draw=black, rectangle split,  rectangle split parts=3] (sn0x10547e0){
%         \begin{tikzpicture}[scale=.2]
%           \node[circle, scale=0.75, fill, red] (tid0) at (0.75,1.5){};
%         \end{tikzpicture}
%         \nodepart{two}
%         \footnotesize{1}
%         \nodepart{three}
%         \footnotesize{$$}
%       };
%       & 
%       \\
%     };
%   \end{scope}
%   \begin{scope}[yshift=\leveltopIIIIIIIIIIII cm]
%     \matrix (line12) [column sep=1cm] {
%       \\
%     };
%   \end{scope}
%   \draw (sn0x1050af0.south) -- (sn0x105a150.north);
%   \draw (sn0x1050af0.south) -- (sn0x104cb60.north);
%   \draw (sn0x105a150.south) -- (sn0x10519d0.north);
%   \draw (sn0x105a150.south) -- (sn0x105a080.north);
%   \draw (sn0x104cb60.south) -- (sn0x105a080.north);
%   \draw (sn0x104cb60.south) -- (sn0x104fbd0.north);
%   \draw (sn0x10519d0.south) -- (sn0x1052250.north);
%   \draw (sn0x10519d0.south) -- (sn0x1052960.north);
%   \draw (sn0x105a080.south) -- (sn0x1052960.north);
%   \draw (sn0x104fbd0.south) -- (sn0x1052960.north);
%   \draw (sn0x104fbd0.south) -- (sn0x10581e0.north);
%   \draw (sn0x104fbd0.south) -- (sn0x1058550.north);
%   \draw (sn0x1052250.south) -- (sn0x10525f0.north);
%   \draw (sn0x1052250.south) -- (sn0x1053850.north);
%   \draw (sn0x1052960.south) -- (sn0x1053850.north);
%   \draw (sn0x1052960.south) -- (sn0x1056b00.north);
%   \draw (sn0x1052960.south) -- (sn0x1056fb0.north);
%   \draw (sn0x10581e0.south) -- (sn0x1058f50.north);
%   \draw (sn0x10581e0.south) -- (sn0x1058a50.north);
%   \draw (sn0x10581e0.south) -- (sn0x1056b00.north);
%   \draw (sn0x10581e0.south) -- (sn0x1056fb0.north);
%   \draw (sn0x1058550.south) -- (sn0x10597a0.north);
%   \draw (sn0x1058550.south) -- (sn0x1056fb0.north);
%   \draw (sn0x10525f0.south) -- (sn0x1053920.north);
%   \draw (sn0x10525f0.south) -- (sn0x1053bc0.north);
%   \draw (sn0x1053850.south) -- (sn0x1053bc0.north);
%   \draw (sn0x1053850.south) -- (sn0x1056090.north);
%   \draw (sn0x1053850.south) -- (sn0x1056160.north);
%   \draw (sn0x1056b00.south) -- (sn0x1056090.north);
%   \draw (sn0x1056b00.south) -- (sn0x1056160.north);
%   \draw (sn0x1056fb0.south) -- (sn0x1056160.north);
%   \draw (sn0x1056fb0.south) -- (sn0x1057630.north);
%   \draw (sn0x1058f50.south) -- (sn0x1053bc0.north);
%   \draw (sn0x1058f50.south) -- (sn0x1056090.north);
%   \draw (sn0x1058f50.south) -- (sn0x1056160.north);
%   \draw (sn0x1058a50.south) -- (sn0x1058b20.north);
%   \draw (sn0x1058a50.south) -- (sn0x1056160.north);
%   \draw (sn0x10597a0.south) -- (sn0x1058b20.north);
%   \draw (sn0x10597a0.south) -- (sn0x1057630.north);
%   \draw (sn0x1053920.south) -- (sn0x10540d0.north);
%   \draw (sn0x1053920.south) -- (sn0x1054480.north);
%   \draw (sn0x1053bc0.south) -- (sn0x1054480.north);
%   \draw (sn0x1053bc0.south) -- (sn0x1055dd0.north);
%   \draw (sn0x1056090.south) -- (sn0x1055dd0.north);
%   \draw (sn0x1056160.south) -- (sn0x1055dd0.north);
%   \draw (sn0x1056160.south) -- (sn0x10568c0.north);
%   \draw (sn0x1057630.south) -- (sn0x10568c0.north);
%   \draw (sn0x1058b20.south) -- (sn0x1054480.north);
%   \draw (sn0x1058b20.south) -- (sn0x10568c0.north);
%   \draw (sn0x10540d0.south) -- (sn0x1054550.north);
%   \draw (sn0x1054480.south) -- (sn0x1054550.north);
%   \draw (sn0x1054480.south) -- (sn0x1055270.north);
%   \draw (sn0x1055dd0.south) -- (sn0x1055270.north);
%   \draw (sn0x10568c0.south) -- (sn0x1055270.north);
%   \draw (sn0x1054550.south) -- (sn0x1054a50.north);
%   \draw (sn0x1055270.south) -- (sn0x1054a50.north);
%   \draw (sn0x1055270.south) -- (sn0x1054cb0.north);
%   \draw (sn0x1054a50.south) -- (sn0x1054b20.north);
%   \draw (sn0x1054cb0.south) -- (sn0x1054b20.north);
%   \draw (sn0x1054b20.south) -- (sn0x10547e0.north);
% \end{tikzpicture}

%%% Local Variables:
%%% TeX-master: "thesis/thesis.tex"
%%% End: 


\hrule



\section{Several proofs that HLF is not optimal for P3}
\frame{
  \begin{minipage}{.25\textwidth}
    \subsection{Different runs of P3-HLF yield diffe\-rent results on 001112}
    \centering{}
    \input{../001112hlf.tex}
  \end{minipage}
}
\frame{
  \begin{minipage}{.25\textwidth}
    \subsection{P3-HLF not optimal for 0012346688}
    \renewcommand{\leveltopI}{-15cm + \leveltop}
\renewcommand{\leveltopII}{-15cm + \leveltopI}
\renewcommand{\leveltopIII}{-15cm + \leveltopII}
\renewcommand{\leveltopIIII}{-15cm + \leveltopIII}
\renewcommand{\leveltopIIIII}{-15cm + \leveltopIIII}
\renewcommand{\leveltopIIIIII}{-15cm + \leveltopIIIII}
\renewcommand{\leveltopIIIIIII}{-15cm + \leveltopIIIIII}
\renewcommand{\leveltopIIIIIIII}{-15cm + \leveltopIIIIIII}
\renewcommand{\leveltopIIIIIIIII}{-15cm + \leveltopIIIIIIII}
\renewcommand{\leveltopIIIIIIIIII}{-15cm + \leveltopIIIIIIIII}
\renewcommand{\leveltopIIIIIIIIIII}{-15cm + \leveltopIIIIIIIIII}
\begin{tikzpicture}[scale=.2, anchor=south]
\begin{scope}[yshift=\leveltopI cm]
\matrix (line1) [column sep=1cm] {
\node[draw=black, rectangle split,  rectangle split parts=4] (sn0x83df00){
\footnotesize{1}
\nodepart{two}
\begin{tikzpicture}[scale=.2]
\node[circle, scale=0.75, fill] (tid0) at (3,1.5){};
\node[circle, scale=0.75, fill] (tid1) at (2.25,3){};
\node[circle, scale=0.75, fill] (tid3) at (2.25,4.5){};
\node[circle, scale=0.75, fill] (tid5) at (2.25,6){};
\node[circle, scale=0.75, fill] (tid7) at (1.5,7.5){};
\node[circle, scale=0.75, fill, red] (tid9) at (0.75,9){};
\node[circle, scale=0.75, fill, red] (tid10) at (2.25,9){};
\draw[](tid7) -- (tid9);
\draw[](tid7) -- (tid10);
\node[circle, scale=0.75, fill, red] (tid8) at (3.75,7.5){};
\draw[](tid5) -- (tid7);
\draw[](tid5) -- (tid8);
\draw[](tid3) -- (tid5);
\draw[](tid1) -- (tid3);
\node[circle, scale=0.75, fill] (tid2) at (5.25,3){};
\node[circle, scale=0.75, fill] (tid4) at (5.25,4.5){};
\node[circle, scale=0.75, fill] (tid6) at (5.25,6){};
\draw[](tid4) -- (tid6);
\draw[](tid2) -- (tid4);
\draw[](tid0) -- (tid1);
\draw[](tid0) -- (tid2);
\end{tikzpicture}
\nodepart{three}
\footnotesize{6.96798}
\nodepart{four}
\footnotesize{$33\:67$}
};
 & 
\\
};
\end{scope}
\begin{scope}[yshift=\leveltopII cm]
\matrix (line2) [column sep=1cm] {
\node[draw=black, rectangle split,  rectangle split parts=4] (sn0x83fb80){
\footnotesize{0.333333}
\nodepart{two}
\begin{tikzpicture}[scale=.2]
\node[circle, scale=0.75, fill] (tid0) at (2.25,1.5){};
\node[circle, scale=0.75, fill] (tid1) at (1.5,3){};
\node[circle, scale=0.75, fill] (tid3) at (1.5,4.5){};
\node[circle, scale=0.75, fill] (tid5) at (1.5,6){};
\node[circle, scale=0.75, fill] (tid7) at (1.5,7.5){};
\node[circle, scale=0.75, fill, red] (tid8) at (0.75,9){};
\node[circle, scale=0.75, fill, red] (tid9) at (2.25,9){};
\draw[](tid7) -- (tid8);
\draw[](tid7) -- (tid9);
\draw[](tid5) -- (tid7);
\draw[](tid3) -- (tid5);
\draw[](tid1) -- (tid3);
\node[circle, scale=0.75, fill] (tid2) at (3.75,3){};
\node[circle, scale=0.75, fill] (tid4) at (3.75,4.5){};
\node[circle, scale=0.75, fill, red] (tid6) at (3.75,6){};
\draw[](tid4) -- (tid6);
\draw[](tid2) -- (tid4);
\draw[](tid0) -- (tid1);
\draw[](tid0) -- (tid2);
\end{tikzpicture}
\nodepart{three}
\footnotesize{6.77836}
\nodepart{four}
\footnotesize{$33\:67$}
};
 & 
\node[draw=black, rectangle split,  rectangle split parts=4] (sn0x840c40){
\footnotesize{0.666667}
\nodepart{two}
\begin{tikzpicture}[scale=.2]
\node[circle, scale=0.75, fill] (tid0) at (2.25,1.5){};
\node[circle, scale=0.75, fill] (tid1) at (1.5,3){};
\node[circle, scale=0.75, fill] (tid3) at (1.5,4.5){};
\node[circle, scale=0.75, fill] (tid5) at (1.5,6){};
\node[circle, scale=0.75, fill] (tid7) at (0.75,7.5){};
\node[circle, scale=0.75, fill, red] (tid9) at (0.75,9){};
\draw[](tid7) -- (tid9);
\node[circle, scale=0.75, fill, red] (tid8) at (2.25,7.5){};
\draw[](tid5) -- (tid7);
\draw[](tid5) -- (tid8);
\draw[](tid3) -- (tid5);
\draw[](tid1) -- (tid3);
\node[circle, scale=0.75, fill] (tid2) at (3.75,3){};
\node[circle, scale=0.75, fill] (tid4) at (3.75,4.5){};
\node[circle, scale=0.75, fill, red] (tid6) at (3.75,6){};
\draw[](tid4) -- (tid6);
\draw[](tid2) -- (tid4);
\draw[](tid0) -- (tid1);
\draw[](tid0) -- (tid2);
\end{tikzpicture}
\nodepart{three}
\footnotesize{6.56279}
\nodepart{four}
\footnotesize{$33\:33\:33$}
};
 & 
\\
};
\end{scope}
\begin{scope}[yshift=\leveltopIII cm]
\matrix (line3) [column sep=1cm] {
\node[draw=black, rectangle split,  rectangle split parts=4] (sn0x8401e0){
\footnotesize{0.111111}
\nodepart{two}
\begin{tikzpicture}[scale=.2]
\node[circle, scale=0.75, fill] (tid0) at (2.25,1.5){};
\node[circle, scale=0.75, fill] (tid1) at (1.5,3){};
\node[circle, scale=0.75, fill] (tid3) at (1.5,4.5){};
\node[circle, scale=0.75, fill] (tid5) at (1.5,6){};
\node[circle, scale=0.75, fill] (tid6) at (1.5,7.5){};
\node[circle, scale=0.75, fill, red] (tid7) at (0.75,9){};
\node[circle, scale=0.75, fill, red] (tid8) at (2.25,9){};
\draw[](tid6) -- (tid7);
\draw[](tid6) -- (tid8);
\draw[](tid5) -- (tid6);
\draw[](tid3) -- (tid5);
\draw[](tid1) -- (tid3);
\node[circle, scale=0.75, fill] (tid2) at (3.75,3){};
\node[circle, scale=0.75, fill, red] (tid4) at (3.75,4.5){};
\draw[](tid2) -- (tid4);
\draw[](tid0) -- (tid1);
\draw[](tid0) -- (tid2);
\end{tikzpicture}
\nodepart{three}
\footnotesize{6.60069}
\nodepart{four}
\footnotesize{$33\:67$}
};
 & 
\node[draw=black, rectangle split,  rectangle split parts=4] (sn0x841280){
\footnotesize{0.444444}
\nodepart{two}
\begin{tikzpicture}[scale=.2]
\node[circle, scale=0.75, fill] (tid0) at (1.5,1.5){};
\node[circle, scale=0.75, fill] (tid1) at (0.75,3){};
\node[circle, scale=0.75, fill] (tid3) at (0.75,4.5){};
\node[circle, scale=0.75, fill] (tid5) at (0.75,6){};
\node[circle, scale=0.75, fill] (tid7) at (0.75,7.5){};
\node[circle, scale=0.75, fill, red] (tid8) at (0.75,9){};
\draw[](tid7) -- (tid8);
\draw[](tid5) -- (tid7);
\draw[](tid3) -- (tid5);
\draw[](tid1) -- (tid3);
\node[circle, scale=0.75, fill] (tid2) at (2.25,3){};
\node[circle, scale=0.75, fill] (tid4) at (2.25,4.5){};
\node[circle, scale=0.75, fill, red] (tid6) at (2.25,6){};
\draw[](tid4) -- (tid6);
\draw[](tid2) -- (tid4);
\draw[](tid0) -- (tid1);
\draw[](tid0) -- (tid2);
\end{tikzpicture}
\nodepart{three}
\footnotesize{6.36719}
\nodepart{four}
\footnotesize{$50\:50$}
};
 & 
\node[draw=black, rectangle split,  rectangle split parts=4] (sn0x8453f0){
\footnotesize{0.222222}
\nodepart{two}
\begin{tikzpicture}[scale=.2]
\node[circle, scale=0.75, fill] (tid0) at (2.25,1.5){};
\node[circle, scale=0.75, fill] (tid1) at (1.5,3){};
\node[circle, scale=0.75, fill] (tid3) at (1.5,4.5){};
\node[circle, scale=0.75, fill] (tid5) at (1.5,6){};
\node[circle, scale=0.75, fill] (tid6) at (0.75,7.5){};
\node[circle, scale=0.75, fill, red] (tid8) at (0.75,9){};
\draw[](tid6) -- (tid8);
\node[circle, scale=0.75, fill, red] (tid7) at (2.25,7.5){};
\draw[](tid5) -- (tid6);
\draw[](tid5) -- (tid7);
\draw[](tid3) -- (tid5);
\draw[](tid1) -- (tid3);
\node[circle, scale=0.75, fill] (tid2) at (3.75,3){};
\node[circle, scale=0.75, fill, red] (tid4) at (3.75,4.5){};
\draw[](tid2) -- (tid4);
\draw[](tid0) -- (tid1);
\draw[](tid0) -- (tid2);
\end{tikzpicture}
\nodepart{three}
\footnotesize{6.36516}
\nodepart{four}
\footnotesize{$33\:33\:33$}
};
 & 
\node[draw=black, rectangle split,  rectangle split parts=4] (sn0x8460d0){
\footnotesize{0.222222}
\nodepart{two}
\begin{tikzpicture}[scale=.2]
\node[circle, scale=0.75, fill] (tid0) at (2.25,1.5){};
\node[circle, scale=0.75, fill] (tid1) at (1.5,3){};
\node[circle, scale=0.75, fill] (tid3) at (1.5,4.5){};
\node[circle, scale=0.75, fill] (tid5) at (1.5,6){};
\node[circle, scale=0.75, fill, red] (tid7) at (0.75,7.5){};
\node[circle, scale=0.75, fill, red] (tid8) at (2.25,7.5){};
\draw[](tid5) -- (tid7);
\draw[](tid5) -- (tid8);
\draw[](tid3) -- (tid5);
\draw[](tid1) -- (tid3);
\node[circle, scale=0.75, fill] (tid2) at (3.75,3){};
\node[circle, scale=0.75, fill] (tid4) at (3.75,4.5){};
\node[circle, scale=0.75, fill, red] (tid6) at (3.75,6){};
\draw[](tid4) -- (tid6);
\draw[](tid2) -- (tid4);
\draw[](tid0) -- (tid1);
\draw[](tid0) -- (tid2);
\end{tikzpicture}
\nodepart{three}
\footnotesize{5.95602}
\nodepart{four}
\footnotesize{$67\:33$}
};
 & 
\\
};
\end{scope}
\begin{scope}[yshift=\leveltopIIII cm]
\matrix (line4) [column sep=1cm] {
\node[draw=black, rectangle split,  rectangle split parts=4] (sn0x841a80){
\footnotesize{0.037037}
\nodepart{two}
\begin{tikzpicture}[scale=.2]
\node[circle, scale=0.75, fill] (tid0) at (2.25,1.5){};
\node[circle, scale=0.75, fill] (tid1) at (1.5,3){};
\node[circle, scale=0.75, fill] (tid3) at (1.5,4.5){};
\node[circle, scale=0.75, fill] (tid4) at (1.5,6){};
\node[circle, scale=0.75, fill] (tid5) at (1.5,7.5){};
\node[circle, scale=0.75, fill, red] (tid6) at (0.75,9){};
\node[circle, scale=0.75, fill, red] (tid7) at (2.25,9){};
\draw[](tid5) -- (tid6);
\draw[](tid5) -- (tid7);
\draw[](tid4) -- (tid5);
\draw[](tid3) -- (tid4);
\draw[](tid1) -- (tid3);
\node[circle, scale=0.75, fill, red] (tid2) at (3.75,3){};
\draw[](tid0) -- (tid1);
\draw[](tid0) -- (tid2);
\end{tikzpicture}
\nodepart{three}
\footnotesize{6.52083}
\nodepart{four}
\footnotesize{$33\:67$}
};
 & 
\node[draw=black, rectangle split,  rectangle split parts=4] (sn0x841ff0){
\footnotesize{0.37037}
\nodepart{two}
\begin{tikzpicture}[scale=.2]
\node[circle, scale=0.75, fill] (tid0) at (1.5,1.5){};
\node[circle, scale=0.75, fill] (tid1) at (0.75,3){};
\node[circle, scale=0.75, fill] (tid3) at (0.75,4.5){};
\node[circle, scale=0.75, fill] (tid5) at (0.75,6){};
\node[circle, scale=0.75, fill] (tid6) at (0.75,7.5){};
\node[circle, scale=0.75, fill, red] (tid7) at (0.75,9){};
\draw[](tid6) -- (tid7);
\draw[](tid5) -- (tid6);
\draw[](tid3) -- (tid5);
\draw[](tid1) -- (tid3);
\node[circle, scale=0.75, fill] (tid2) at (2.25,3){};
\node[circle, scale=0.75, fill, red] (tid4) at (2.25,4.5){};
\draw[](tid2) -- (tid4);
\draw[](tid0) -- (tid1);
\draw[](tid0) -- (tid2);
\end{tikzpicture}
\nodepart{three}
\footnotesize{6.14062}
\nodepart{four}
\footnotesize{$50\:50$}
};
 & 
\node[draw=black, rectangle split,  rectangle split parts=4] (sn0x844380){
\footnotesize{0.37037}
\nodepart{two}
\begin{tikzpicture}[scale=.2]
\node[circle, scale=0.75, fill] (tid0) at (1.5,1.5){};
\node[circle, scale=0.75, fill] (tid1) at (0.75,3){};
\node[circle, scale=0.75, fill] (tid3) at (0.75,4.5){};
\node[circle, scale=0.75, fill] (tid5) at (0.75,6){};
\node[circle, scale=0.75, fill, red] (tid7) at (0.75,7.5){};
\draw[](tid5) -- (tid7);
\draw[](tid3) -- (tid5);
\draw[](tid1) -- (tid3);
\node[circle, scale=0.75, fill] (tid2) at (2.25,3){};
\node[circle, scale=0.75, fill] (tid4) at (2.25,4.5){};
\node[circle, scale=0.75, fill, red] (tid6) at (2.25,6){};
\draw[](tid4) -- (tid6);
\draw[](tid2) -- (tid4);
\draw[](tid0) -- (tid1);
\draw[](tid0) -- (tid2);
\end{tikzpicture}
\nodepart{three}
\footnotesize{5.59375}
\nodepart{four}
\footnotesize{$50\:50$}
};
 & 
\node[draw=black, rectangle split,  rectangle split parts=4] (sn0x846b30){
\footnotesize{0.0740741}
\nodepart{two}
\begin{tikzpicture}[scale=.2]
\node[circle, scale=0.75, fill] (tid0) at (2.25,1.5){};
\node[circle, scale=0.75, fill] (tid1) at (1.5,3){};
\node[circle, scale=0.75, fill] (tid3) at (1.5,4.5){};
\node[circle, scale=0.75, fill] (tid4) at (1.5,6){};
\node[circle, scale=0.75, fill] (tid5) at (0.75,7.5){};
\node[circle, scale=0.75, fill, red] (tid7) at (0.75,9){};
\draw[](tid5) -- (tid7);
\node[circle, scale=0.75, fill, red] (tid6) at (2.25,7.5){};
\draw[](tid4) -- (tid5);
\draw[](tid4) -- (tid6);
\draw[](tid3) -- (tid4);
\draw[](tid1) -- (tid3);
\node[circle, scale=0.75, fill, red] (tid2) at (3.75,3){};
\draw[](tid0) -- (tid1);
\draw[](tid0) -- (tid2);
\end{tikzpicture}
\nodepart{three}
\footnotesize{6.27431}
\nodepart{four}
\footnotesize{$33\:33\:33$}
};
 & 
\node[draw=black, rectangle split,  rectangle split parts=4] (sn0x846f20){
\footnotesize{0.148148}
\nodepart{two}
\begin{tikzpicture}[scale=.2]
\node[circle, scale=0.75, fill] (tid0) at (2.25,1.5){};
\node[circle, scale=0.75, fill] (tid1) at (1.5,3){};
\node[circle, scale=0.75, fill] (tid3) at (1.5,4.5){};
\node[circle, scale=0.75, fill] (tid5) at (1.5,6){};
\node[circle, scale=0.75, fill, red] (tid6) at (0.75,7.5){};
\node[circle, scale=0.75, fill, red] (tid7) at (2.25,7.5){};
\draw[](tid5) -- (tid6);
\draw[](tid5) -- (tid7);
\draw[](tid3) -- (tid5);
\draw[](tid1) -- (tid3);
\node[circle, scale=0.75, fill] (tid2) at (3.75,3){};
\node[circle, scale=0.75, fill, red] (tid4) at (3.75,4.5){};
\draw[](tid2) -- (tid4);
\draw[](tid0) -- (tid1);
\draw[](tid0) -- (tid2);
\end{tikzpicture}
\nodepart{three}
\footnotesize{5.68056}
\nodepart{four}
\footnotesize{$67\:33$}
};
 & 
\\
};
\end{scope}
\begin{scope}[yshift=\leveltopIIIII cm]
\matrix (line5) [column sep=1cm] {
\node[draw=black, rectangle split,  rectangle split parts=4] (sn0x8420d0){
\footnotesize{0.0123457}
\nodepart{two}
\begin{tikzpicture}[scale=.2]
\node[circle, scale=0.75, fill] (tid0) at (1.5,1.5){};
\node[circle, scale=0.75, fill] (tid1) at (1.5,3){};
\node[circle, scale=0.75, fill] (tid2) at (1.5,4.5){};
\node[circle, scale=0.75, fill] (tid3) at (1.5,6){};
\node[circle, scale=0.75, fill] (tid4) at (1.5,7.5){};
\node[circle, scale=0.75, fill, red] (tid5) at (0.75,9){};
\node[circle, scale=0.75, fill, red] (tid6) at (2.25,9){};
\draw[](tid4) -- (tid5);
\draw[](tid4) -- (tid6);
\draw[](tid3) -- (tid4);
\draw[](tid2) -- (tid3);
\draw[](tid1) -- (tid2);
\draw[](tid0) -- (tid1);
\end{tikzpicture}
\nodepart{three}
\footnotesize{6.5}
\nodepart{four}
\footnotesize{$1$}
};
 & 
\node[draw=black, rectangle split,  rectangle split parts=4] (sn0x842470){
\footnotesize{0.234568}
\nodepart{two}
\begin{tikzpicture}[scale=.2]
\node[circle, scale=0.75, fill] (tid0) at (1.5,1.5){};
\node[circle, scale=0.75, fill] (tid1) at (0.75,3){};
\node[circle, scale=0.75, fill] (tid3) at (0.75,4.5){};
\node[circle, scale=0.75, fill] (tid4) at (0.75,6){};
\node[circle, scale=0.75, fill] (tid5) at (0.75,7.5){};
\node[circle, scale=0.75, fill, red] (tid6) at (0.75,9){};
\draw[](tid5) -- (tid6);
\draw[](tid4) -- (tid5);
\draw[](tid3) -- (tid4);
\draw[](tid1) -- (tid3);
\node[circle, scale=0.75, fill, red] (tid2) at (2.25,3){};
\draw[](tid0) -- (tid1);
\draw[](tid0) -- (tid2);
\end{tikzpicture}
\nodepart{three}
\footnotesize{6.03125}
\nodepart{four}
\footnotesize{$50\:50$}
};
 & 
\node[draw=black, rectangle split,  rectangle split parts=4] (sn0x843c90){
\footnotesize{0.469136}
\nodepart{two}
\begin{tikzpicture}[scale=.2]
\node[circle, scale=0.75, fill] (tid0) at (1.5,1.5){};
\node[circle, scale=0.75, fill] (tid1) at (0.75,3){};
\node[circle, scale=0.75, fill] (tid3) at (0.75,4.5){};
\node[circle, scale=0.75, fill] (tid5) at (0.75,6){};
\node[circle, scale=0.75, fill, red] (tid6) at (0.75,7.5){};
\draw[](tid5) -- (tid6);
\draw[](tid3) -- (tid5);
\draw[](tid1) -- (tid3);
\node[circle, scale=0.75, fill] (tid2) at (2.25,3){};
\node[circle, scale=0.75, fill, red] (tid4) at (2.25,4.5){};
\draw[](tid2) -- (tid4);
\draw[](tid0) -- (tid1);
\draw[](tid0) -- (tid2);
\end{tikzpicture}
\nodepart{three}
\footnotesize{5.25}
\nodepart{four}
\footnotesize{$50\:50$}
};
 & 
\node[draw=black, rectangle split,  rectangle split parts=4] (sn0x8459d0){
\footnotesize{0.185185}
\nodepart{two}
\begin{tikzpicture}[scale=.2]
\node[circle, scale=0.75, fill] (tid0) at (1.5,1.5){};
\node[circle, scale=0.75, fill] (tid1) at (0.75,3){};
\node[circle, scale=0.75, fill] (tid3) at (0.75,4.5){};
\node[circle, scale=0.75, fill, red] (tid5) at (0.75,6){};
\draw[](tid3) -- (tid5);
\draw[](tid1) -- (tid3);
\node[circle, scale=0.75, fill] (tid2) at (2.25,3){};
\node[circle, scale=0.75, fill] (tid4) at (2.25,4.5){};
\node[circle, scale=0.75, fill, red] (tid6) at (2.25,6){};
\draw[](tid4) -- (tid6);
\draw[](tid2) -- (tid4);
\draw[](tid0) -- (tid1);
\draw[](tid0) -- (tid2);
\end{tikzpicture}
\nodepart{three}
\footnotesize{4.9375}
\nodepart{four}
\footnotesize{$1$}
};
 & 
\node[draw=black, rectangle split,  rectangle split parts=4] (sn0x847120){
\footnotesize{0.0246914}
\nodepart{two}
\begin{tikzpicture}[scale=.2]
\node[circle, scale=0.75, fill] (tid0) at (1.5,1.5){};
\node[circle, scale=0.75, fill] (tid1) at (1.5,3){};
\node[circle, scale=0.75, fill] (tid2) at (1.5,4.5){};
\node[circle, scale=0.75, fill] (tid3) at (1.5,6){};
\node[circle, scale=0.75, fill] (tid4) at (0.75,7.5){};
\node[circle, scale=0.75, fill, red] (tid6) at (0.75,9){};
\draw[](tid4) -- (tid6);
\node[circle, scale=0.75, fill, red] (tid5) at (2.25,7.5){};
\draw[](tid3) -- (tid4);
\draw[](tid3) -- (tid5);
\draw[](tid2) -- (tid3);
\draw[](tid1) -- (tid2);
\draw[](tid0) -- (tid1);
\end{tikzpicture}
\nodepart{three}
\footnotesize{6.25}
\nodepart{four}
\footnotesize{$50\:50$}
};
 & 
\node[draw=black, rectangle split,  rectangle split parts=4] (sn0x846630){
\footnotesize{0.0740741}
\nodepart{two}
\begin{tikzpicture}[scale=.2]
\node[circle, scale=0.75, fill] (tid0) at (2.25,1.5){};
\node[circle, scale=0.75, fill] (tid1) at (1.5,3){};
\node[circle, scale=0.75, fill] (tid3) at (1.5,4.5){};
\node[circle, scale=0.75, fill] (tid4) at (1.5,6){};
\node[circle, scale=0.75, fill, red] (tid5) at (0.75,7.5){};
\node[circle, scale=0.75, fill, red] (tid6) at (2.25,7.5){};
\draw[](tid4) -- (tid5);
\draw[](tid4) -- (tid6);
\draw[](tid3) -- (tid4);
\draw[](tid1) -- (tid3);
\node[circle, scale=0.75, fill, red] (tid2) at (3.75,3){};
\draw[](tid0) -- (tid1);
\draw[](tid0) -- (tid2);
\end{tikzpicture}
\nodepart{three}
\footnotesize{5.54167}
\nodepart{four}
\footnotesize{$67\:33$}
};
 & 
\\
};
\end{scope}
\begin{scope}[yshift=\leveltopIIIIII cm]
\matrix (line6) [column sep=1cm] {
\node[draw=black, rectangle split,  rectangle split parts=4] (sn0x842a70){
\footnotesize{0.141975}
\nodepart{two}
\begin{tikzpicture}[scale=.2]
\node[circle, scale=0.75, fill] (tid0) at (0.75,1.5){};
\node[circle, scale=0.75, fill] (tid1) at (0.75,3){};
\node[circle, scale=0.75, fill] (tid2) at (0.75,4.5){};
\node[circle, scale=0.75, fill] (tid3) at (0.75,6){};
\node[circle, scale=0.75, fill] (tid4) at (0.75,7.5){};
\node[circle, scale=0.75, fill, red] (tid5) at (0.75,9){};
\draw[](tid4) -- (tid5);
\draw[](tid3) -- (tid4);
\draw[](tid2) -- (tid3);
\draw[](tid1) -- (tid2);
\draw[](tid0) -- (tid1);
\end{tikzpicture}
\nodepart{three}
\footnotesize{6}
\nodepart{four}
\footnotesize{$1$}
};
 & 
\node[draw=black, rectangle split,  rectangle split parts=4] (sn0x8438b0){
\footnotesize{0.401235}
\nodepart{two}
\begin{tikzpicture}[scale=.2]
\node[circle, scale=0.75, fill] (tid0) at (1.5,1.5){};
\node[circle, scale=0.75, fill] (tid1) at (0.75,3){};
\node[circle, scale=0.75, fill] (tid3) at (0.75,4.5){};
\node[circle, scale=0.75, fill] (tid4) at (0.75,6){};
\node[circle, scale=0.75, fill, red] (tid5) at (0.75,7.5){};
\draw[](tid4) -- (tid5);
\draw[](tid3) -- (tid4);
\draw[](tid1) -- (tid3);
\node[circle, scale=0.75, fill, red] (tid2) at (2.25,3){};
\draw[](tid0) -- (tid1);
\draw[](tid0) -- (tid2);
\end{tikzpicture}
\nodepart{three}
\footnotesize{5.0625}
\nodepart{four}
\footnotesize{$50\:50$}
};
 & 
\node[draw=black, rectangle split,  rectangle split parts=4] (sn0x844db0){
\footnotesize{0.419753}
\nodepart{two}
\begin{tikzpicture}[scale=.2]
\node[circle, scale=0.75, fill] (tid0) at (1.5,1.5){};
\node[circle, scale=0.75, fill] (tid1) at (0.75,3){};
\node[circle, scale=0.75, fill] (tid3) at (0.75,4.5){};
\node[circle, scale=0.75, fill, red] (tid5) at (0.75,6){};
\draw[](tid3) -- (tid5);
\draw[](tid1) -- (tid3);
\node[circle, scale=0.75, fill] (tid2) at (2.25,3){};
\node[circle, scale=0.75, fill, red] (tid4) at (2.25,4.5){};
\draw[](tid2) -- (tid4);
\draw[](tid0) -- (tid1);
\draw[](tid0) -- (tid2);
\end{tikzpicture}
\nodepart{three}
\footnotesize{4.4375}
\nodepart{four}
\footnotesize{$50\:50$}
};
 & 
\node[draw=black, rectangle split,  rectangle split parts=4] (sn0x847d10){
\footnotesize{0.037037}
\nodepart{two}
\begin{tikzpicture}[scale=.2]
\node[circle, scale=0.75, fill] (tid0) at (1.5,1.5){};
\node[circle, scale=0.75, fill] (tid1) at (1.5,3){};
\node[circle, scale=0.75, fill] (tid2) at (1.5,4.5){};
\node[circle, scale=0.75, fill] (tid3) at (1.5,6){};
\node[circle, scale=0.75, fill, red] (tid4) at (0.75,7.5){};
\node[circle, scale=0.75, fill, red] (tid5) at (2.25,7.5){};
\draw[](tid3) -- (tid4);
\draw[](tid3) -- (tid5);
\draw[](tid2) -- (tid3);
\draw[](tid1) -- (tid2);
\draw[](tid0) -- (tid1);
\end{tikzpicture}
\nodepart{three}
\footnotesize{5.5}
\nodepart{four}
\footnotesize{$1$}
};
 & 
\\
};
\end{scope}
\begin{scope}[yshift=\leveltopIIIIIII cm]
\matrix (line7) [column sep=1cm] {
\node[draw=black, rectangle split,  rectangle split parts=4] (sn0x842780){
\footnotesize{0.37963}
\nodepart{two}
\begin{tikzpicture}[scale=.2]
\node[circle, scale=0.75, fill] (tid0) at (0.75,1.5){};
\node[circle, scale=0.75, fill] (tid1) at (0.75,3){};
\node[circle, scale=0.75, fill] (tid2) at (0.75,4.5){};
\node[circle, scale=0.75, fill] (tid3) at (0.75,6){};
\node[circle, scale=0.75, fill, red] (tid4) at (0.75,7.5){};
\draw[](tid3) -- (tid4);
\draw[](tid2) -- (tid3);
\draw[](tid1) -- (tid2);
\draw[](tid0) -- (tid1);
\end{tikzpicture}
\nodepart{three}
\footnotesize{5}
\nodepart{four}
\footnotesize{$1$}
};
 & 
\node[draw=black, rectangle split,  rectangle split parts=4] (sn0x843ab0){
\footnotesize{0.410494}
\nodepart{two}
\begin{tikzpicture}[scale=.2]
\node[circle, scale=0.75, fill] (tid0) at (1.5,1.5){};
\node[circle, scale=0.75, fill] (tid1) at (0.75,3){};
\node[circle, scale=0.75, fill] (tid3) at (0.75,4.5){};
\node[circle, scale=0.75, fill, red] (tid4) at (0.75,6){};
\draw[](tid3) -- (tid4);
\draw[](tid1) -- (tid3);
\node[circle, scale=0.75, fill, red] (tid2) at (2.25,3){};
\draw[](tid0) -- (tid1);
\draw[](tid0) -- (tid2);
\end{tikzpicture}
\nodepart{three}
\footnotesize{4.125}
\nodepart{four}
\footnotesize{$50\:50$}
};
 & 
\node[draw=black, rectangle split,  rectangle split parts=4] (sn0x8450c0){
\footnotesize{0.209877}
\nodepart{two}
\begin{tikzpicture}[scale=.2]
\node[circle, scale=0.75, fill] (tid0) at (1.5,1.5){};
\node[circle, scale=0.75, fill] (tid1) at (0.75,3){};
\node[circle, scale=0.75, fill, red] (tid3) at (0.75,4.5){};
\draw[](tid1) -- (tid3);
\node[circle, scale=0.75, fill] (tid2) at (2.25,3){};
\node[circle, scale=0.75, fill, red] (tid4) at (2.25,4.5){};
\draw[](tid2) -- (tid4);
\draw[](tid0) -- (tid1);
\draw[](tid0) -- (tid2);
\end{tikzpicture}
\nodepart{three}
\footnotesize{3.75}
\nodepart{four}
\footnotesize{$1$}
};
 & 
\\
};
\end{scope}
\begin{scope}[yshift=\leveltopIIIIIIII cm]
\matrix (line8) [column sep=1cm] {
\node[draw=black, rectangle split,  rectangle split parts=4] (sn0x8428f0){
\footnotesize{0.584877}
\nodepart{two}
\begin{tikzpicture}[scale=.2]
\node[circle, scale=0.75, fill] (tid0) at (0.75,1.5){};
\node[circle, scale=0.75, fill] (tid1) at (0.75,3){};
\node[circle, scale=0.75, fill] (tid2) at (0.75,4.5){};
\node[circle, scale=0.75, fill, red] (tid3) at (0.75,6){};
\draw[](tid2) -- (tid3);
\draw[](tid1) -- (tid2);
\draw[](tid0) -- (tid1);
\end{tikzpicture}
\nodepart{three}
\footnotesize{4}
\nodepart{four}
\footnotesize{$1$}
};
 & 
\node[draw=black, rectangle split,  rectangle split parts=4] (sn0x843590){
\footnotesize{0.415124}
\nodepart{two}
\begin{tikzpicture}[scale=.2]
\node[circle, scale=0.75, fill] (tid0) at (1.5,1.5){};
\node[circle, scale=0.75, fill] (tid1) at (0.75,3){};
\node[circle, scale=0.75, fill, red] (tid3) at (0.75,4.5){};
\draw[](tid1) -- (tid3);
\node[circle, scale=0.75, fill, red] (tid2) at (2.25,3){};
\draw[](tid0) -- (tid1);
\draw[](tid0) -- (tid2);
\end{tikzpicture}
\nodepart{three}
\footnotesize{3.25}
\nodepart{four}
\footnotesize{$50\:50$}
};
 & 
\\
};
\end{scope}
\begin{scope}[yshift=\leveltopIIIIIIIII cm]
\matrix (line9) [column sep=1cm] {
\node[draw=black, rectangle split,  rectangle split parts=4] (sn0x842ce0){
\footnotesize{0.792438}
\nodepart{two}
\begin{tikzpicture}[scale=.2]
\node[circle, scale=0.75, fill] (tid0) at (0.75,1.5){};
\node[circle, scale=0.75, fill] (tid1) at (0.75,3){};
\node[circle, scale=0.75, fill, red] (tid2) at (0.75,4.5){};
\draw[](tid1) -- (tid2);
\draw[](tid0) -- (tid1);
\end{tikzpicture}
\nodepart{three}
\footnotesize{3}
\nodepart{four}
\footnotesize{$1$}
};
 & 
\node[draw=black, rectangle split,  rectangle split parts=4] (sn0x843b80){
\footnotesize{0.207562}
\nodepart{two}
\begin{tikzpicture}[scale=.2]
\node[circle, scale=0.75, fill] (tid0) at (1.5,1.5){};
\node[circle, scale=0.75, fill, red] (tid1) at (0.75,3){};
\node[circle, scale=0.75, fill, red] (tid2) at (2.25,3){};
\draw[](tid0) -- (tid1);
\draw[](tid0) -- (tid2);
\end{tikzpicture}
\nodepart{three}
\footnotesize{2.5}
\nodepart{four}
\footnotesize{$1$}
};
 & 
\\
};
\end{scope}
\begin{scope}[yshift=\leveltopIIIIIIIIII cm]
\matrix (line10) [column sep=1cm] {
\node[draw=black, rectangle split,  rectangle split parts=4] (sn0x842db0){
\footnotesize{1}
\nodepart{two}
\begin{tikzpicture}[scale=.2]
\node[circle, scale=0.75, fill] (tid0) at (0.75,1.5){};
\node[circle, scale=0.75, fill, red] (tid1) at (0.75,3){};
\draw[](tid0) -- (tid1);
\end{tikzpicture}
\nodepart{three}
\footnotesize{2}
\nodepart{four}
\footnotesize{$1$}
};
 & 
\\
};
\end{scope}
\begin{scope}[yshift=\leveltopIIIIIIIIIII cm]
\matrix (line11) [column sep=1cm] {
\node[draw=black, rectangle split,  rectangle split parts=4] (sn0x842c00){
\footnotesize{1}
\nodepart{two}
\begin{tikzpicture}[scale=.2]
\node[circle, scale=0.75, fill, red] (tid0) at (0.75,1.5){};
\end{tikzpicture}
\nodepart{three}
\footnotesize{1}
\nodepart{four}
\footnotesize{$$}
};
 & 
\\
};
\end{scope}
\begin{scope}[yshift=\leveltopIIIIIIIIIIII cm]
\matrix (line12) [column sep=1cm] {
\\
};
\end{scope}
\draw (sn0x83df00.south) -- (sn0x83fb80.north);
\draw (sn0x83df00.south) -- (sn0x840c40.north);
\draw (sn0x83fb80.south) -- (sn0x8401e0.north);
\draw (sn0x83fb80.south) -- (sn0x841280.north);
\draw (sn0x840c40.south) -- (sn0x8453f0.north);
\draw (sn0x840c40.south) -- (sn0x841280.north);
\draw (sn0x840c40.south) -- (sn0x8460d0.north);
\draw (sn0x8401e0.south) -- (sn0x841a80.north);
\draw (sn0x8401e0.south) -- (sn0x841ff0.north);
\draw (sn0x841280.south) -- (sn0x841ff0.north);
\draw (sn0x841280.south) -- (sn0x844380.north);
\draw (sn0x8453f0.south) -- (sn0x846b30.north);
\draw (sn0x8453f0.south) -- (sn0x841ff0.north);
\draw (sn0x8453f0.south) -- (sn0x846f20.north);
\draw (sn0x8460d0.south) -- (sn0x846f20.north);
\draw (sn0x8460d0.south) -- (sn0x844380.north);
\draw (sn0x841a80.south) -- (sn0x8420d0.north);
\draw (sn0x841a80.south) -- (sn0x842470.north);
\draw (sn0x841ff0.south) -- (sn0x842470.north);
\draw (sn0x841ff0.south) -- (sn0x843c90.north);
\draw (sn0x844380.south) -- (sn0x843c90.north);
\draw (sn0x844380.south) -- (sn0x8459d0.north);
\draw (sn0x846b30.south) -- (sn0x847120.north);
\draw (sn0x846b30.south) -- (sn0x842470.north);
\draw (sn0x846b30.south) -- (sn0x846630.north);
\draw (sn0x846f20.south) -- (sn0x846630.north);
\draw (sn0x846f20.south) -- (sn0x843c90.north);
\draw (sn0x8420d0.south) -- (sn0x842a70.north);
\draw (sn0x842470.south) -- (sn0x842a70.north);
\draw (sn0x842470.south) -- (sn0x8438b0.north);
\draw (sn0x843c90.south) -- (sn0x8438b0.north);
\draw (sn0x843c90.south) -- (sn0x844db0.north);
\draw (sn0x8459d0.south) -- (sn0x844db0.north);
\draw (sn0x847120.south) -- (sn0x842a70.north);
\draw (sn0x847120.south) -- (sn0x847d10.north);
\draw (sn0x846630.south) -- (sn0x847d10.north);
\draw (sn0x846630.south) -- (sn0x8438b0.north);
\draw (sn0x842a70.south) -- (sn0x842780.north);
\draw (sn0x8438b0.south) -- (sn0x842780.north);
\draw (sn0x8438b0.south) -- (sn0x843ab0.north);
\draw (sn0x844db0.south) -- (sn0x843ab0.north);
\draw (sn0x844db0.south) -- (sn0x8450c0.north);
\draw (sn0x847d10.south) -- (sn0x842780.north);
\draw (sn0x842780.south) -- (sn0x8428f0.north);
\draw (sn0x843ab0.south) -- (sn0x8428f0.north);
\draw (sn0x843ab0.south) -- (sn0x843590.north);
\draw (sn0x8450c0.south) -- (sn0x843590.north);
\draw (sn0x8428f0.south) -- (sn0x842ce0.north);
\draw (sn0x843590.south) -- (sn0x842ce0.north);
\draw (sn0x843590.south) -- (sn0x843b80.north);
\draw (sn0x842ce0.south) -- (sn0x842db0.north);
\draw (sn0x843b80.south) -- (sn0x842db0.north);
\draw (sn0x842db0.south) -- (sn0x842c00.north);
\end{tikzpicture}

%%% Local Variables:
%%% TeX-master: "thesis/thesis.tex"
%%% End: 

    \renewcommand{\leveltopI}{-15cm + \leveltop}
\renewcommand{\leveltopII}{-15cm + \leveltopI}
\renewcommand{\leveltopIII}{-15cm + \leveltopII}
\renewcommand{\leveltopIIII}{-15cm + \leveltopIII}
\renewcommand{\leveltopIIIII}{-15cm + \leveltopIIII}
\renewcommand{\leveltopIIIIII}{-15cm + \leveltopIIIII}
\renewcommand{\leveltopIIIIIII}{-15cm + \leveltopIIIIII}
\renewcommand{\leveltopIIIIIIII}{-15cm + \leveltopIIIIIII}
\renewcommand{\leveltopIIIIIIIII}{-15cm + \leveltopIIIIIIII}
\renewcommand{\leveltopIIIIIIIIII}{-15cm + \leveltopIIIIIIIII}
\renewcommand{\leveltopIIIIIIIIIII}{-15cm + \leveltopIIIIIIIIII}
\begin{tikzpicture}[scale=.2, anchor=south]
\begin{scope}[yshift=\leveltopI cm]
\matrix (line1) [column sep=1cm] {
\node[draw=black, rectangle split,  rectangle split parts=3] (sn0x801680){
\begin{tikzpicture}[scale=.2]
\node[circle, scale=0.75, fill] (tid0) at (3,1.5){};
\node[circle, scale=0.75, fill] (tid1) at (2.25,3){};
\node[circle, scale=0.75, fill] (tid3) at (2.25,4.5){};
\node[circle, scale=0.75, fill] (tid5) at (2.25,6){};
\node[circle, scale=0.75, fill] (tid7) at (1.5,7.5){};
\node[circle, scale=0.75, fill, red] (tid9) at (0.75,9){};
\node[circle, scale=0.75, fill, red] (tid10) at (2.25,9){};
\draw[](tid7) -- (tid9);
\draw[](tid7) -- (tid10);
\node[circle, scale=0.75, fill] (tid8) at (3.75,7.5){};
\draw[](tid5) -- (tid7);
\draw[](tid5) -- (tid8);
\draw[](tid3) -- (tid5);
\draw[](tid1) -- (tid3);
\node[circle, scale=0.75, fill] (tid2) at (5.25,3){};
\node[circle, scale=0.75, fill] (tid4) at (5.25,4.5){};
\node[circle, scale=0.75, fill, red] (tid6) at (5.25,6){};
\draw[](tid4) -- (tid6);
\draw[](tid2) -- (tid4);
\draw[](tid0) -- (tid1);
\draw[](tid0) -- (tid2);
\end{tikzpicture}
\nodepart{two}
\footnotesize{6.96753}
\nodepart{three}
\footnotesize{$33\:67$}
};
 & 
\\
};
\end{scope}
\begin{scope}[yshift=\leveltopII cm]
\matrix (line2) [column sep=1cm] {
\node[draw=black, rectangle split,  rectangle split parts=3] (sn0x7ffa30){
\begin{tikzpicture}[scale=.2]
\node[circle, scale=0.75, fill] (tid0) at (3,1.5){};
\node[circle, scale=0.75, fill] (tid1) at (2.25,3){};
\node[circle, scale=0.75, fill] (tid3) at (2.25,4.5){};
\node[circle, scale=0.75, fill] (tid5) at (2.25,6){};
\node[circle, scale=0.75, fill] (tid6) at (1.5,7.5){};
\node[circle, scale=0.75, fill, red] (tid8) at (0.75,9){};
\node[circle, scale=0.75, fill, red] (tid9) at (2.25,9){};
\draw[](tid6) -- (tid8);
\draw[](tid6) -- (tid9);
\node[circle, scale=0.75, fill, red] (tid7) at (3.75,7.5){};
\draw[](tid5) -- (tid6);
\draw[](tid5) -- (tid7);
\draw[](tid3) -- (tid5);
\draw[](tid1) -- (tid3);
\node[circle, scale=0.75, fill] (tid2) at (5.25,3){};
\node[circle, scale=0.75, fill] (tid4) at (5.25,4.5){};
\draw[](tid2) -- (tid4);
\draw[](tid0) -- (tid1);
\draw[](tid0) -- (tid2);
\end{tikzpicture}
\nodepart{two}
\footnotesize{6.77701}
\nodepart{three}
\footnotesize{$33\:67$}
};
 & 
\node[draw=black, rectangle split,  rectangle split parts=3] (sn0x801450){
\begin{tikzpicture}[scale=.2]
\node[circle, scale=0.75, fill] (tid0) at (2.25,1.5){};
\node[circle, scale=0.75, fill] (tid1) at (1.5,3){};
\node[circle, scale=0.75, fill] (tid3) at (1.5,4.5){};
\node[circle, scale=0.75, fill] (tid5) at (1.5,6){};
\node[circle, scale=0.75, fill] (tid7) at (0.75,7.5){};
\node[circle, scale=0.75, fill, red] (tid9) at (0.75,9){};
\draw[](tid7) -- (tid9);
\node[circle, scale=0.75, fill, red] (tid8) at (2.25,7.5){};
\draw[](tid5) -- (tid7);
\draw[](tid5) -- (tid8);
\draw[](tid3) -- (tid5);
\draw[](tid1) -- (tid3);
\node[circle, scale=0.75, fill] (tid2) at (3.75,3){};
\node[circle, scale=0.75, fill] (tid4) at (3.75,4.5){};
\node[circle, scale=0.75, fill, red] (tid6) at (3.75,6){};
\draw[](tid4) -- (tid6);
\draw[](tid2) -- (tid4);
\draw[](tid0) -- (tid1);
\draw[](tid0) -- (tid2);
\end{tikzpicture}
\nodepart{two}
\footnotesize{6.56279}
\nodepart{three}
\footnotesize{$33\:33\:33$}
};
 & 
\\
};
\end{scope}
\begin{scope}[yshift=\leveltopIII cm]
\matrix (line3) [column sep=1cm] {
\node[draw=black, rectangle split,  rectangle split parts=3] (sn0x7fefb0){
\begin{tikzpicture}[scale=.2]
\node[circle, scale=0.75, fill] (tid0) at (2.25,1.5){};
\node[circle, scale=0.75, fill] (tid1) at (1.5,3){};
\node[circle, scale=0.75, fill] (tid3) at (1.5,4.5){};
\node[circle, scale=0.75, fill] (tid5) at (1.5,6){};
\node[circle, scale=0.75, fill] (tid6) at (1.5,7.5){};
\node[circle, scale=0.75, fill, red] (tid7) at (0.75,9){};
\node[circle, scale=0.75, fill, red] (tid8) at (2.25,9){};
\draw[](tid6) -- (tid7);
\draw[](tid6) -- (tid8);
\draw[](tid5) -- (tid6);
\draw[](tid3) -- (tid5);
\draw[](tid1) -- (tid3);
\node[circle, scale=0.75, fill] (tid2) at (3.75,3){};
\node[circle, scale=0.75, fill, red] (tid4) at (3.75,4.5){};
\draw[](tid2) -- (tid4);
\draw[](tid0) -- (tid1);
\draw[](tid0) -- (tid2);
\end{tikzpicture}
\nodepart{two}
\footnotesize{6.60069}
\nodepart{three}
\footnotesize{$33\:67$}
};
 & 
\node[draw=black, rectangle split,  rectangle split parts=3] (sn0x7fe4c0){
\begin{tikzpicture}[scale=.2]
\node[circle, scale=0.75, fill] (tid0) at (2.25,1.5){};
\node[circle, scale=0.75, fill] (tid1) at (1.5,3){};
\node[circle, scale=0.75, fill] (tid3) at (1.5,4.5){};
\node[circle, scale=0.75, fill] (tid5) at (1.5,6){};
\node[circle, scale=0.75, fill] (tid6) at (0.75,7.5){};
\node[circle, scale=0.75, fill, red] (tid8) at (0.75,9){};
\draw[](tid6) -- (tid8);
\node[circle, scale=0.75, fill, red] (tid7) at (2.25,7.5){};
\draw[](tid5) -- (tid6);
\draw[](tid5) -- (tid7);
\draw[](tid3) -- (tid5);
\draw[](tid1) -- (tid3);
\node[circle, scale=0.75, fill] (tid2) at (3.75,3){};
\node[circle, scale=0.75, fill, red] (tid4) at (3.75,4.5){};
\draw[](tid2) -- (tid4);
\draw[](tid0) -- (tid1);
\draw[](tid0) -- (tid2);
\end{tikzpicture}
\nodepart{two}
\footnotesize{6.36516}
\nodepart{three}
\footnotesize{$33\:33\:33$}
};
 & 
\node[draw=black, rectangle split,  rectangle split parts=3] (sn0x7ffea0){
\begin{tikzpicture}[scale=.2]
\node[circle, scale=0.75, fill] (tid0) at (1.5,1.5){};
\node[circle, scale=0.75, fill] (tid1) at (0.75,3){};
\node[circle, scale=0.75, fill] (tid3) at (0.75,4.5){};
\node[circle, scale=0.75, fill] (tid5) at (0.75,6){};
\node[circle, scale=0.75, fill] (tid7) at (0.75,7.5){};
\node[circle, scale=0.75, fill, red] (tid8) at (0.75,9){};
\draw[](tid7) -- (tid8);
\draw[](tid5) -- (tid7);
\draw[](tid3) -- (tid5);
\draw[](tid1) -- (tid3);
\node[circle, scale=0.75, fill] (tid2) at (2.25,3){};
\node[circle, scale=0.75, fill] (tid4) at (2.25,4.5){};
\node[circle, scale=0.75, fill, red] (tid6) at (2.25,6){};
\draw[](tid4) -- (tid6);
\draw[](tid2) -- (tid4);
\draw[](tid0) -- (tid1);
\draw[](tid0) -- (tid2);
\end{tikzpicture}
\nodepart{two}
\footnotesize{6.36719}
\nodepart{three}
\footnotesize{$50\:50$}
};
 & 
\node[draw=black, rectangle split,  rectangle split parts=3] (sn0x800d40){
\begin{tikzpicture}[scale=.2]
\node[circle, scale=0.75, fill] (tid0) at (2.25,1.5){};
\node[circle, scale=0.75, fill] (tid1) at (1.5,3){};
\node[circle, scale=0.75, fill] (tid3) at (1.5,4.5){};
\node[circle, scale=0.75, fill] (tid5) at (1.5,6){};
\node[circle, scale=0.75, fill, red] (tid7) at (0.75,7.5){};
\node[circle, scale=0.75, fill, red] (tid8) at (2.25,7.5){};
\draw[](tid5) -- (tid7);
\draw[](tid5) -- (tid8);
\draw[](tid3) -- (tid5);
\draw[](tid1) -- (tid3);
\node[circle, scale=0.75, fill] (tid2) at (3.75,3){};
\node[circle, scale=0.75, fill] (tid4) at (3.75,4.5){};
\node[circle, scale=0.75, fill, red] (tid6) at (3.75,6){};
\draw[](tid4) -- (tid6);
\draw[](tid2) -- (tid4);
\draw[](tid0) -- (tid1);
\draw[](tid0) -- (tid2);
\end{tikzpicture}
\nodepart{two}
\footnotesize{5.95602}
\nodepart{three}
\footnotesize{$33\:67$}
};
 & 
\\
};
\end{scope}
\begin{scope}[yshift=\leveltopIIII cm]
\matrix (line4) [column sep=1cm] {
\node[draw=black, rectangle split,  rectangle split parts=3] (sn0x7fd000){
\begin{tikzpicture}[scale=.2]
\node[circle, scale=0.75, fill] (tid0) at (2.25,1.5){};
\node[circle, scale=0.75, fill] (tid1) at (1.5,3){};
\node[circle, scale=0.75, fill] (tid3) at (1.5,4.5){};
\node[circle, scale=0.75, fill] (tid4) at (1.5,6){};
\node[circle, scale=0.75, fill] (tid5) at (1.5,7.5){};
\node[circle, scale=0.75, fill, red] (tid6) at (0.75,9){};
\node[circle, scale=0.75, fill, red] (tid7) at (2.25,9){};
\draw[](tid5) -- (tid6);
\draw[](tid5) -- (tid7);
\draw[](tid4) -- (tid5);
\draw[](tid3) -- (tid4);
\draw[](tid1) -- (tid3);
\node[circle, scale=0.75, fill, red] (tid2) at (3.75,3){};
\draw[](tid0) -- (tid1);
\draw[](tid0) -- (tid2);
\end{tikzpicture}
\nodepart{two}
\footnotesize{6.52083}
\nodepart{three}
\footnotesize{$33\:67$}
};
 & 
\node[draw=black, rectangle split,  rectangle split parts=3] (sn0x7fe3f0){
\begin{tikzpicture}[scale=.2]
\node[circle, scale=0.75, fill] (tid0) at (1.5,1.5){};
\node[circle, scale=0.75, fill] (tid1) at (0.75,3){};
\node[circle, scale=0.75, fill] (tid3) at (0.75,4.5){};
\node[circle, scale=0.75, fill] (tid5) at (0.75,6){};
\node[circle, scale=0.75, fill] (tid6) at (0.75,7.5){};
\node[circle, scale=0.75, fill, red] (tid7) at (0.75,9){};
\draw[](tid6) -- (tid7);
\draw[](tid5) -- (tid6);
\draw[](tid3) -- (tid5);
\draw[](tid1) -- (tid3);
\node[circle, scale=0.75, fill] (tid2) at (2.25,3){};
\node[circle, scale=0.75, fill, red] (tid4) at (2.25,4.5){};
\draw[](tid2) -- (tid4);
\draw[](tid0) -- (tid1);
\draw[](tid0) -- (tid2);
\end{tikzpicture}
\nodepart{two}
\footnotesize{6.14062}
\nodepart{three}
\footnotesize{$50\:50$}
};
 & 
\node[draw=black, rectangle split,  rectangle split parts=3] (sn0x7fbdd0){
\begin{tikzpicture}[scale=.2]
\node[circle, scale=0.75, fill] (tid0) at (2.25,1.5){};
\node[circle, scale=0.75, fill] (tid1) at (1.5,3){};
\node[circle, scale=0.75, fill] (tid3) at (1.5,4.5){};
\node[circle, scale=0.75, fill] (tid4) at (1.5,6){};
\node[circle, scale=0.75, fill] (tid5) at (0.75,7.5){};
\node[circle, scale=0.75, fill, red] (tid7) at (0.75,9){};
\draw[](tid5) -- (tid7);
\node[circle, scale=0.75, fill, red] (tid6) at (2.25,7.5){};
\draw[](tid4) -- (tid5);
\draw[](tid4) -- (tid6);
\draw[](tid3) -- (tid4);
\draw[](tid1) -- (tid3);
\node[circle, scale=0.75, fill, red] (tid2) at (3.75,3){};
\draw[](tid0) -- (tid1);
\draw[](tid0) -- (tid2);
\end{tikzpicture}
\nodepart{two}
\footnotesize{6.27431}
\nodepart{three}
\footnotesize{$33\:33\:33$}
};
 & 
\node[draw=black, rectangle split,  rectangle split parts=3] (sn0x7feb40){
\begin{tikzpicture}[scale=.2]
\node[circle, scale=0.75, fill] (tid0) at (2.25,1.5){};
\node[circle, scale=0.75, fill] (tid1) at (1.5,3){};
\node[circle, scale=0.75, fill] (tid3) at (1.5,4.5){};
\node[circle, scale=0.75, fill] (tid5) at (1.5,6){};
\node[circle, scale=0.75, fill, red] (tid6) at (0.75,7.5){};
\node[circle, scale=0.75, fill, red] (tid7) at (2.25,7.5){};
\draw[](tid5) -- (tid6);
\draw[](tid5) -- (tid7);
\draw[](tid3) -- (tid5);
\draw[](tid1) -- (tid3);
\node[circle, scale=0.75, fill] (tid2) at (3.75,3){};
\node[circle, scale=0.75, fill, red] (tid4) at (3.75,4.5){};
\draw[](tid2) -- (tid4);
\draw[](tid0) -- (tid1);
\draw[](tid0) -- (tid2);
\end{tikzpicture}
\nodepart{two}
\footnotesize{5.68056}
\nodepart{three}
\footnotesize{$67\:33$}
};
 & 
\node[draw=black, rectangle split,  rectangle split parts=3] (sn0x7ffdd0){
\begin{tikzpicture}[scale=.2]
\node[circle, scale=0.75, fill] (tid0) at (1.5,1.5){};
\node[circle, scale=0.75, fill] (tid1) at (0.75,3){};
\node[circle, scale=0.75, fill] (tid3) at (0.75,4.5){};
\node[circle, scale=0.75, fill] (tid5) at (0.75,6){};
\node[circle, scale=0.75, fill, red] (tid7) at (0.75,7.5){};
\draw[](tid5) -- (tid7);
\draw[](tid3) -- (tid5);
\draw[](tid1) -- (tid3);
\node[circle, scale=0.75, fill] (tid2) at (2.25,3){};
\node[circle, scale=0.75, fill] (tid4) at (2.25,4.5){};
\node[circle, scale=0.75, fill, red] (tid6) at (2.25,6){};
\draw[](tid4) -- (tid6);
\draw[](tid2) -- (tid4);
\draw[](tid0) -- (tid1);
\draw[](tid0) -- (tid2);
\end{tikzpicture}
\nodepart{two}
\footnotesize{5.59375}
\nodepart{three}
\footnotesize{$50\:50$}
};
 & 
\\
};
\end{scope}
\begin{scope}[yshift=\leveltopIIIII cm]
\matrix (line5) [column sep=1cm] {
\node[draw=black, rectangle split,  rectangle split parts=3] (sn0x7f9bb0){
\begin{tikzpicture}[scale=.2]
\node[circle, scale=0.75, fill] (tid0) at (1.5,1.5){};
\node[circle, scale=0.75, fill] (tid1) at (1.5,3){};
\node[circle, scale=0.75, fill] (tid2) at (1.5,4.5){};
\node[circle, scale=0.75, fill] (tid3) at (1.5,6){};
\node[circle, scale=0.75, fill] (tid4) at (1.5,7.5){};
\node[circle, scale=0.75, fill, red] (tid5) at (0.75,9){};
\node[circle, scale=0.75, fill, red] (tid6) at (2.25,9){};
\draw[](tid4) -- (tid5);
\draw[](tid4) -- (tid6);
\draw[](tid3) -- (tid4);
\draw[](tid2) -- (tid3);
\draw[](tid1) -- (tid2);
\draw[](tid0) -- (tid1);
\end{tikzpicture}
\nodepart{two}
\footnotesize{6.5}
\nodepart{three}
\footnotesize{$1$}
};
 & 
\node[draw=black, rectangle split,  rectangle split parts=3] (sn0x7fba00){
\begin{tikzpicture}[scale=.2]
\node[circle, scale=0.75, fill] (tid0) at (1.5,1.5){};
\node[circle, scale=0.75, fill] (tid1) at (0.75,3){};
\node[circle, scale=0.75, fill] (tid3) at (0.75,4.5){};
\node[circle, scale=0.75, fill] (tid4) at (0.75,6){};
\node[circle, scale=0.75, fill] (tid5) at (0.75,7.5){};
\node[circle, scale=0.75, fill, red] (tid6) at (0.75,9){};
\draw[](tid5) -- (tid6);
\draw[](tid4) -- (tid5);
\draw[](tid3) -- (tid4);
\draw[](tid1) -- (tid3);
\node[circle, scale=0.75, fill, red] (tid2) at (2.25,3){};
\draw[](tid0) -- (tid1);
\draw[](tid0) -- (tid2);
\end{tikzpicture}
\nodepart{two}
\footnotesize{6.03125}
\nodepart{three}
\footnotesize{$50\:50$}
};
 & 
\node[draw=black, rectangle split,  rectangle split parts=3] (sn0x7fe1a0){
\begin{tikzpicture}[scale=.2]
\node[circle, scale=0.75, fill] (tid0) at (1.5,1.5){};
\node[circle, scale=0.75, fill] (tid1) at (0.75,3){};
\node[circle, scale=0.75, fill] (tid3) at (0.75,4.5){};
\node[circle, scale=0.75, fill] (tid5) at (0.75,6){};
\node[circle, scale=0.75, fill, red] (tid6) at (0.75,7.5){};
\draw[](tid5) -- (tid6);
\draw[](tid3) -- (tid5);
\draw[](tid1) -- (tid3);
\node[circle, scale=0.75, fill] (tid2) at (2.25,3){};
\node[circle, scale=0.75, fill, red] (tid4) at (2.25,4.5){};
\draw[](tid2) -- (tid4);
\draw[](tid0) -- (tid1);
\draw[](tid0) -- (tid2);
\end{tikzpicture}
\nodepart{two}
\footnotesize{5.25}
\nodepart{three}
\footnotesize{$50\:50$}
};
 & 
\node[draw=black, rectangle split,  rectangle split parts=3] (sn0x7fa6c0){
\begin{tikzpicture}[scale=.2]
\node[circle, scale=0.75, fill] (tid0) at (1.5,1.5){};
\node[circle, scale=0.75, fill] (tid1) at (1.5,3){};
\node[circle, scale=0.75, fill] (tid2) at (1.5,4.5){};
\node[circle, scale=0.75, fill] (tid3) at (1.5,6){};
\node[circle, scale=0.75, fill] (tid4) at (0.75,7.5){};
\node[circle, scale=0.75, fill, red] (tid6) at (0.75,9){};
\draw[](tid4) -- (tid6);
\node[circle, scale=0.75, fill, red] (tid5) at (2.25,7.5){};
\draw[](tid3) -- (tid4);
\draw[](tid3) -- (tid5);
\draw[](tid2) -- (tid3);
\draw[](tid1) -- (tid2);
\draw[](tid0) -- (tid1);
\end{tikzpicture}
\nodepart{two}
\footnotesize{6.25}
\nodepart{three}
\footnotesize{$50\:50$}
};
 & 
\node[draw=black, rectangle split,  rectangle split parts=3] (sn0x7fc2a0){
\begin{tikzpicture}[scale=.2]
\node[circle, scale=0.75, fill] (tid0) at (2.25,1.5){};
\node[circle, scale=0.75, fill] (tid1) at (1.5,3){};
\node[circle, scale=0.75, fill] (tid3) at (1.5,4.5){};
\node[circle, scale=0.75, fill] (tid4) at (1.5,6){};
\node[circle, scale=0.75, fill, red] (tid5) at (0.75,7.5){};
\node[circle, scale=0.75, fill, red] (tid6) at (2.25,7.5){};
\draw[](tid4) -- (tid5);
\draw[](tid4) -- (tid6);
\draw[](tid3) -- (tid4);
\draw[](tid1) -- (tid3);
\node[circle, scale=0.75, fill, red] (tid2) at (3.75,3){};
\draw[](tid0) -- (tid1);
\draw[](tid0) -- (tid2);
\end{tikzpicture}
\nodepart{two}
\footnotesize{5.54167}
\nodepart{three}
\footnotesize{$67\:33$}
};
 & 
\node[draw=black, rectangle split,  rectangle split parts=3] (sn0x800b10){
\begin{tikzpicture}[scale=.2]
\node[circle, scale=0.75, fill] (tid0) at (1.5,1.5){};
\node[circle, scale=0.75, fill] (tid1) at (0.75,3){};
\node[circle, scale=0.75, fill] (tid3) at (0.75,4.5){};
\node[circle, scale=0.75, fill, red] (tid5) at (0.75,6){};
\draw[](tid3) -- (tid5);
\draw[](tid1) -- (tid3);
\node[circle, scale=0.75, fill] (tid2) at (2.25,3){};
\node[circle, scale=0.75, fill] (tid4) at (2.25,4.5){};
\node[circle, scale=0.75, fill, red] (tid6) at (2.25,6){};
\draw[](tid4) -- (tid6);
\draw[](tid2) -- (tid4);
\draw[](tid0) -- (tid1);
\draw[](tid0) -- (tid2);
\end{tikzpicture}
\nodepart{two}
\footnotesize{4.9375}
\nodepart{three}
\footnotesize{$1$}
};
 & 
\\
};
\end{scope}
\begin{scope}[yshift=\leveltopIIIIII cm]
\matrix (line6) [column sep=1cm] {
\node[draw=black, rectangle split,  rectangle split parts=3] (sn0x7f9a80){
\begin{tikzpicture}[scale=.2]
\node[circle, scale=0.75, fill] (tid0) at (0.75,1.5){};
\node[circle, scale=0.75, fill] (tid1) at (0.75,3){};
\node[circle, scale=0.75, fill] (tid2) at (0.75,4.5){};
\node[circle, scale=0.75, fill] (tid3) at (0.75,6){};
\node[circle, scale=0.75, fill] (tid4) at (0.75,7.5){};
\node[circle, scale=0.75, fill, red] (tid5) at (0.75,9){};
\draw[](tid4) -- (tid5);
\draw[](tid3) -- (tid4);
\draw[](tid2) -- (tid3);
\draw[](tid1) -- (tid2);
\draw[](tid0) -- (tid1);
\end{tikzpicture}
\nodepart{two}
\footnotesize{6}
\nodepart{three}
\footnotesize{$1$}
};
 & 
\node[draw=black, rectangle split,  rectangle split parts=3] (sn0x7fb8f0){
\begin{tikzpicture}[scale=.2]
\node[circle, scale=0.75, fill] (tid0) at (1.5,1.5){};
\node[circle, scale=0.75, fill] (tid1) at (0.75,3){};
\node[circle, scale=0.75, fill] (tid3) at (0.75,4.5){};
\node[circle, scale=0.75, fill] (tid4) at (0.75,6){};
\node[circle, scale=0.75, fill, red] (tid5) at (0.75,7.5){};
\draw[](tid4) -- (tid5);
\draw[](tid3) -- (tid4);
\draw[](tid1) -- (tid3);
\node[circle, scale=0.75, fill, red] (tid2) at (2.25,3){};
\draw[](tid0) -- (tid1);
\draw[](tid0) -- (tid2);
\end{tikzpicture}
\nodepart{two}
\footnotesize{5.0625}
\nodepart{three}
\footnotesize{$50\:50$}
};
 & 
\node[draw=black, rectangle split,  rectangle split parts=3] (sn0x7fd460){
\begin{tikzpicture}[scale=.2]
\node[circle, scale=0.75, fill] (tid0) at (1.5,1.5){};
\node[circle, scale=0.75, fill] (tid1) at (0.75,3){};
\node[circle, scale=0.75, fill] (tid3) at (0.75,4.5){};
\node[circle, scale=0.75, fill, red] (tid5) at (0.75,6){};
\draw[](tid3) -- (tid5);
\draw[](tid1) -- (tid3);
\node[circle, scale=0.75, fill] (tid2) at (2.25,3){};
\node[circle, scale=0.75, fill, red] (tid4) at (2.25,4.5){};
\draw[](tid2) -- (tid4);
\draw[](tid0) -- (tid1);
\draw[](tid0) -- (tid2);
\end{tikzpicture}
\nodepart{two}
\footnotesize{4.4375}
\nodepart{three}
\footnotesize{$50\:50$}
};
 & 
\node[draw=black, rectangle split,  rectangle split parts=3] (sn0x7f9e40){
\begin{tikzpicture}[scale=.2]
\node[circle, scale=0.75, fill] (tid0) at (1.5,1.5){};
\node[circle, scale=0.75, fill] (tid1) at (1.5,3){};
\node[circle, scale=0.75, fill] (tid2) at (1.5,4.5){};
\node[circle, scale=0.75, fill] (tid3) at (1.5,6){};
\node[circle, scale=0.75, fill, red] (tid4) at (0.75,7.5){};
\node[circle, scale=0.75, fill, red] (tid5) at (2.25,7.5){};
\draw[](tid3) -- (tid4);
\draw[](tid3) -- (tid5);
\draw[](tid2) -- (tid3);
\draw[](tid1) -- (tid2);
\draw[](tid0) -- (tid1);
\end{tikzpicture}
\nodepart{two}
\footnotesize{5.5}
\nodepart{three}
\footnotesize{$1$}
};
 & 
\\
};
\end{scope}
\begin{scope}[yshift=\leveltopIIIIIII cm]
\matrix (line7) [column sep=1cm] {
\node[draw=black, rectangle split,  rectangle split parts=3] (sn0x7f97c0){
\begin{tikzpicture}[scale=.2]
\node[circle, scale=0.75, fill] (tid0) at (0.75,1.5){};
\node[circle, scale=0.75, fill] (tid1) at (0.75,3){};
\node[circle, scale=0.75, fill] (tid2) at (0.75,4.5){};
\node[circle, scale=0.75, fill] (tid3) at (0.75,6){};
\node[circle, scale=0.75, fill, red] (tid4) at (0.75,7.5){};
\draw[](tid3) -- (tid4);
\draw[](tid2) -- (tid3);
\draw[](tid1) -- (tid2);
\draw[](tid0) -- (tid1);
\end{tikzpicture}
\nodepart{two}
\footnotesize{5}
\nodepart{three}
\footnotesize{$1$}
};
 & 
\node[draw=black, rectangle split,  rectangle split parts=3] (sn0x7fb5f0){
\begin{tikzpicture}[scale=.2]
\node[circle, scale=0.75, fill] (tid0) at (1.5,1.5){};
\node[circle, scale=0.75, fill] (tid1) at (0.75,3){};
\node[circle, scale=0.75, fill] (tid3) at (0.75,4.5){};
\node[circle, scale=0.75, fill, red] (tid4) at (0.75,6){};
\draw[](tid3) -- (tid4);
\draw[](tid1) -- (tid3);
\node[circle, scale=0.75, fill, red] (tid2) at (2.25,3){};
\draw[](tid0) -- (tid1);
\draw[](tid0) -- (tid2);
\end{tikzpicture}
\nodepart{two}
\footnotesize{4.125}
\nodepart{three}
\footnotesize{$50\:50$}
};
 & 
\node[draw=black, rectangle split,  rectangle split parts=3] (sn0x7fd350){
\begin{tikzpicture}[scale=.2]
\node[circle, scale=0.75, fill] (tid0) at (1.5,1.5){};
\node[circle, scale=0.75, fill] (tid1) at (0.75,3){};
\node[circle, scale=0.75, fill, red] (tid3) at (0.75,4.5){};
\draw[](tid1) -- (tid3);
\node[circle, scale=0.75, fill] (tid2) at (2.25,3){};
\node[circle, scale=0.75, fill, red] (tid4) at (2.25,4.5){};
\draw[](tid2) -- (tid4);
\draw[](tid0) -- (tid1);
\draw[](tid0) -- (tid2);
\end{tikzpicture}
\nodepart{two}
\footnotesize{3.75}
\nodepart{three}
\footnotesize{$1$}
};
 & 
\\
};
\end{scope}
\begin{scope}[yshift=\leveltopIIIIIIII cm]
\matrix (line8) [column sep=1cm] {
\node[draw=black, rectangle split,  rectangle split parts=3] (sn0x7f9570){
\begin{tikzpicture}[scale=.2]
\node[circle, scale=0.75, fill] (tid0) at (0.75,1.5){};
\node[circle, scale=0.75, fill] (tid1) at (0.75,3){};
\node[circle, scale=0.75, fill] (tid2) at (0.75,4.5){};
\node[circle, scale=0.75, fill, red] (tid3) at (0.75,6){};
\draw[](tid2) -- (tid3);
\draw[](tid1) -- (tid2);
\draw[](tid0) -- (tid1);
\end{tikzpicture}
\nodepart{two}
\footnotesize{4}
\nodepart{three}
\footnotesize{$1$}
};
 & 
\node[draw=black, rectangle split,  rectangle split parts=3] (sn0x7fb520){
\begin{tikzpicture}[scale=.2]
\node[circle, scale=0.75, fill] (tid0) at (1.5,1.5){};
\node[circle, scale=0.75, fill] (tid1) at (0.75,3){};
\node[circle, scale=0.75, fill, red] (tid3) at (0.75,4.5){};
\draw[](tid1) -- (tid3);
\node[circle, scale=0.75, fill, red] (tid2) at (2.25,3){};
\draw[](tid0) -- (tid1);
\draw[](tid0) -- (tid2);
\end{tikzpicture}
\nodepart{two}
\footnotesize{3.25}
\nodepart{three}
\footnotesize{$50\:50$}
};
 & 
\\
};
\end{scope}
\begin{scope}[yshift=\leveltopIIIIIIIII cm]
\matrix (line9) [column sep=1cm] {
\node[draw=black, rectangle split,  rectangle split parts=3] (sn0x7f94a0){
\begin{tikzpicture}[scale=.2]
\node[circle, scale=0.75, fill] (tid0) at (0.75,1.5){};
\node[circle, scale=0.75, fill] (tid1) at (0.75,3){};
\node[circle, scale=0.75, fill, red] (tid2) at (0.75,4.5){};
\draw[](tid1) -- (tid2);
\draw[](tid0) -- (tid1);
\end{tikzpicture}
\nodepart{two}
\footnotesize{3}
\nodepart{three}
\footnotesize{$1$}
};
 & 
\node[draw=black, rectangle split,  rectangle split parts=3] (sn0x7fa860){
\begin{tikzpicture}[scale=.2]
\node[circle, scale=0.75, fill] (tid0) at (1.5,1.5){};
\node[circle, scale=0.75, fill, red] (tid1) at (0.75,3){};
\node[circle, scale=0.75, fill, red] (tid2) at (2.25,3){};
\draw[](tid0) -- (tid1);
\draw[](tid0) -- (tid2);
\end{tikzpicture}
\nodepart{two}
\footnotesize{2.5}
\nodepart{three}
\footnotesize{$1$}
};
 & 
\\
};
\end{scope}
\begin{scope}[yshift=\leveltopIIIIIIIIII cm]
\matrix (line10) [column sep=1cm] {
\node[draw=black, rectangle split,  rectangle split parts=3] (sn0x7f8230){
\begin{tikzpicture}[scale=.2]
\node[circle, scale=0.75, fill] (tid0) at (0.75,1.5){};
\node[circle, scale=0.75, fill, red] (tid1) at (0.75,3){};
\draw[](tid0) -- (tid1);
\end{tikzpicture}
\nodepart{two}
\footnotesize{2}
\nodepart{three}
\footnotesize{$1$}
};
 & 
\\
};
\end{scope}
\begin{scope}[yshift=\leveltopIIIIIIIIIII cm]
\matrix (line11) [column sep=1cm] {
\node[draw=black, rectangle split,  rectangle split parts=3] (sn0x7f8160){
\begin{tikzpicture}[scale=.2]
\node[circle, scale=0.75, fill, red] (tid0) at (0.75,1.5){};
\end{tikzpicture}
\nodepart{two}
\footnotesize{1}
\nodepart{three}
\footnotesize{$$}
};
 & 
\\
};
\end{scope}
\begin{scope}[yshift=\leveltopIIIIIIIIIIII cm]
\matrix (line12) [column sep=1cm] {
\\
};
\end{scope}
\draw (sn0x801680.south) -- (sn0x7ffa30.north);
\draw (sn0x801680.south) -- (sn0x801450.north);
\draw (sn0x7ffa30.south) -- (sn0x7fefb0.north);
\draw (sn0x7ffa30.south) -- (sn0x7fe4c0.north);
\draw (sn0x801450.south) -- (sn0x7fe4c0.north);
\draw (sn0x801450.south) -- (sn0x7ffea0.north);
\draw (sn0x801450.south) -- (sn0x800d40.north);
\draw (sn0x7fefb0.south) -- (sn0x7fd000.north);
\draw (sn0x7fefb0.south) -- (sn0x7fe3f0.north);
\draw (sn0x7fe4c0.south) -- (sn0x7fbdd0.north);
\draw (sn0x7fe4c0.south) -- (sn0x7fe3f0.north);
\draw (sn0x7fe4c0.south) -- (sn0x7feb40.north);
\draw (sn0x7ffea0.south) -- (sn0x7fe3f0.north);
\draw (sn0x7ffea0.south) -- (sn0x7ffdd0.north);
\draw (sn0x800d40.south) -- (sn0x7feb40.north);
\draw (sn0x800d40.south) -- (sn0x7ffdd0.north);
\draw (sn0x7fd000.south) -- (sn0x7f9bb0.north);
\draw (sn0x7fd000.south) -- (sn0x7fba00.north);
\draw (sn0x7fe3f0.south) -- (sn0x7fba00.north);
\draw (sn0x7fe3f0.south) -- (sn0x7fe1a0.north);
\draw (sn0x7fbdd0.south) -- (sn0x7fa6c0.north);
\draw (sn0x7fbdd0.south) -- (sn0x7fba00.north);
\draw (sn0x7fbdd0.south) -- (sn0x7fc2a0.north);
\draw (sn0x7feb40.south) -- (sn0x7fc2a0.north);
\draw (sn0x7feb40.south) -- (sn0x7fe1a0.north);
\draw (sn0x7ffdd0.south) -- (sn0x7fe1a0.north);
\draw (sn0x7ffdd0.south) -- (sn0x800b10.north);
\draw (sn0x7f9bb0.south) -- (sn0x7f9a80.north);
\draw (sn0x7fba00.south) -- (sn0x7f9a80.north);
\draw (sn0x7fba00.south) -- (sn0x7fb8f0.north);
\draw (sn0x7fe1a0.south) -- (sn0x7fb8f0.north);
\draw (sn0x7fe1a0.south) -- (sn0x7fd460.north);
\draw (sn0x7fa6c0.south) -- (sn0x7f9a80.north);
\draw (sn0x7fa6c0.south) -- (sn0x7f9e40.north);
\draw (sn0x7fc2a0.south) -- (sn0x7f9e40.north);
\draw (sn0x7fc2a0.south) -- (sn0x7fb8f0.north);
\draw (sn0x800b10.south) -- (sn0x7fd460.north);
\draw (sn0x7f9a80.south) -- (sn0x7f97c0.north);
\draw (sn0x7fb8f0.south) -- (sn0x7f97c0.north);
\draw (sn0x7fb8f0.south) -- (sn0x7fb5f0.north);
\draw (sn0x7fd460.south) -- (sn0x7fb5f0.north);
\draw (sn0x7fd460.south) -- (sn0x7fd350.north);
\draw (sn0x7f9e40.south) -- (sn0x7f97c0.north);
\draw (sn0x7f97c0.south) -- (sn0x7f9570.north);
\draw (sn0x7fb5f0.south) -- (sn0x7f9570.north);
\draw (sn0x7fb5f0.south) -- (sn0x7fb520.north);
\draw (sn0x7fd350.south) -- (sn0x7fb520.north);
\draw (sn0x7f9570.south) -- (sn0x7f94a0.north);
\draw (sn0x7fb520.south) -- (sn0x7f94a0.north);
\draw (sn0x7fb520.south) -- (sn0x7fa860.north);
\draw (sn0x7f94a0.south) -- (sn0x7f8230.north);
\draw (sn0x7fa860.south) -- (sn0x7f8230.north);
\draw (sn0x7f8230.south) -- (sn0x7f8160.north);
\end{tikzpicture}

%%% Local Variables:
%%% TeX-master: "thesis/thesis.tex"
%%% End: 

  \end{minipage}
}
\frame{
  \begin{minipage}{.25\textwidth}
    \subsection{P3-HLF not optimal  for 0012446788}
    \renewcommand{\leveltopI}{-19cm + \leveltop}
\renewcommand{\leveltopII}{-19cm + \leveltopI}
\renewcommand{\leveltopIII}{-19cm + \leveltopII}
\renewcommand{\leveltopIIII}{-19cm + \leveltopIII}
\renewcommand{\leveltopIIIII}{-19cm + \leveltopIIII}
\renewcommand{\leveltopIIIIII}{-19cm + \leveltopIIIII}
\renewcommand{\leveltopIIIIIII}{-19cm + \leveltopIIIIII}
\renewcommand{\leveltopIIIIIIII}{-19cm + \leveltopIIIIIII}
\renewcommand{\leveltopIIIIIIIII}{-19cm + \leveltopIIIIIIII}
\renewcommand{\leveltopIIIIIIIIII}{-19cm + \leveltopIIIIIIIII}
\renewcommand{\leveltopIIIIIIIIIII}{-19cm + \leveltopIIIIIIIIII}
\begin{tikzpicture}[scale=.2, anchor=south]
\begin{scope}[yshift=\leveltopI cm]
\matrix (line1) [column sep=.5cm] {
\node[draw=black, rectangle split,  rectangle split parts=4] (sn0x983680){
\footnotesize{100}
\nodepart{two}
\begin{tikzpicture}[scale=.2]
\node[circle, scale=0.75, fill] (tid0) at (3,1.5){};
\node[circle, scale=0.75, fill] (tid1) at (2.25,3){};
\node[circle, scale=0.75, fill] (tid3) at (2.25,4.5){};
\node[circle, scale=0.75, fill] (tid5) at (1.5,6){};
\node[circle, scale=0.75, fill] (tid7) at (1.5,7.5){};
\node[circle, scale=0.75, fill] (tid8) at (1.5,9){};
\node[circle, scale=0.75, fill, red] (tid9) at (0.75,10.5){};
\node[circle, scale=0.75, fill, red] (tid10) at (2.25,10.5){};
\draw[](tid8) -- (tid9);
\draw[](tid8) -- (tid10);
\draw[](tid7) -- (tid8);
\draw[](tid5) -- (tid7);
\node[circle, scale=0.75, fill, red] (tid6) at (3.75,6){};
\draw[](tid3) -- (tid5);
\draw[](tid3) -- (tid6);
\draw[](tid1) -- (tid3);
\node[circle, scale=0.75, fill] (tid2) at (5.25,3){};
\node[circle, scale=0.75, fill] (tid4) at (5.25,4.5){};
\draw[](tid2) -- (tid4);
\draw[](tid0) -- (tid1);
\draw[](tid0) -- (tid2);
\end{tikzpicture}
\nodepart{three}
\footnotesize{7.60833}
\nodepart{four}
\footnotesize{$33\:67$}
};
 & 
\\
};
\end{scope}
\begin{scope}[yshift=\leveltopII cm]
\matrix (line2) [column sep=.5cm] {
\node[draw=black, rectangle split,  rectangle split parts=4] (sn0x984240){
\footnotesize{33.3333}
\nodepart{two}
\begin{tikzpicture}[scale=.2]
\node[circle, scale=0.75, fill] (tid0) at (2.25,1.5){};
\node[circle, scale=0.75, fill] (tid1) at (1.5,3){};
\node[circle, scale=0.75, fill] (tid3) at (1.5,4.5){};
\node[circle, scale=0.75, fill] (tid5) at (1.5,6){};
\node[circle, scale=0.75, fill] (tid6) at (1.5,7.5){};
\node[circle, scale=0.75, fill] (tid7) at (1.5,9){};
\node[circle, scale=0.75, fill, red] (tid8) at (0.75,10.5){};
\node[circle, scale=0.75, fill, red] (tid9) at (2.25,10.5){};
\draw[](tid7) -- (tid8);
\draw[](tid7) -- (tid9);
\draw[](tid6) -- (tid7);
\draw[](tid5) -- (tid6);
\draw[](tid3) -- (tid5);
\draw[](tid1) -- (tid3);
\node[circle, scale=0.75, fill] (tid2) at (3.75,3){};
\node[circle, scale=0.75, fill, red] (tid4) at (3.75,4.5){};
\draw[](tid2) -- (tid4);
\draw[](tid0) -- (tid1);
\draw[](tid0) -- (tid2);
\end{tikzpicture}
\nodepart{three}
\footnotesize{7.55556}
\nodepart{four}
\footnotesize{$33\:67$}
};
 & 
\node[draw=black, rectangle split,  rectangle split parts=4] (sn0x984830){
\footnotesize{66.6667}
\nodepart{two}
\begin{tikzpicture}[scale=.2]
\node[circle, scale=0.75, fill] (tid0) at (2.25,1.5){};
\node[circle, scale=0.75, fill] (tid1) at (1.5,3){};
\node[circle, scale=0.75, fill] (tid3) at (1.5,4.5){};
\node[circle, scale=0.75, fill] (tid5) at (0.75,6){};
\node[circle, scale=0.75, fill] (tid7) at (0.75,7.5){};
\node[circle, scale=0.75, fill] (tid8) at (0.75,9){};
\node[circle, scale=0.75, fill, red] (tid9) at (0.75,10.5){};
\draw[](tid8) -- (tid9);
\draw[](tid7) -- (tid8);
\draw[](tid5) -- (tid7);
\node[circle, scale=0.75, fill, red] (tid6) at (2.25,6){};
\draw[](tid3) -- (tid5);
\draw[](tid3) -- (tid6);
\draw[](tid1) -- (tid3);
\node[circle, scale=0.75, fill] (tid2) at (3.75,3){};
\node[circle, scale=0.75, fill, red] (tid4) at (3.75,4.5){};
\draw[](tid2) -- (tid4);
\draw[](tid0) -- (tid1);
\draw[](tid0) -- (tid2);
\end{tikzpicture}
\nodepart{three}
\footnotesize{7.13471}
\nodepart{four}
\footnotesize{$33\:33\:33$}
};
 & 
\\
};
\end{scope}
\begin{scope}[yshift=\leveltopIII cm]
\matrix (line3) [column sep=.5cm] {
\node[draw=black, rectangle split,  rectangle split parts=4] (sn0x984fb0){
\footnotesize{11.1111}
\nodepart{two}
\begin{tikzpicture}[scale=.2]
\node[circle, scale=0.75, fill] (tid0) at (2.25,1.5){};
\node[circle, scale=0.75, fill] (tid1) at (1.5,3){};
\node[circle, scale=0.75, fill] (tid3) at (1.5,4.5){};
\node[circle, scale=0.75, fill] (tid4) at (1.5,6){};
\node[circle, scale=0.75, fill] (tid5) at (1.5,7.5){};
\node[circle, scale=0.75, fill] (tid6) at (1.5,9){};
\node[circle, scale=0.75, fill, red] (tid7) at (0.75,10.5){};
\node[circle, scale=0.75, fill, red] (tid8) at (2.25,10.5){};
\draw[](tid6) -- (tid7);
\draw[](tid6) -- (tid8);
\draw[](tid5) -- (tid6);
\draw[](tid4) -- (tid5);
\draw[](tid3) -- (tid4);
\draw[](tid1) -- (tid3);
\node[circle, scale=0.75, fill, red] (tid2) at (3.75,3){};
\draw[](tid0) -- (tid1);
\draw[](tid0) -- (tid2);
\end{tikzpicture}
\nodepart{three}
\footnotesize{7.51042}
\nodepart{four}
\footnotesize{$33\:67$}
};
 & 
\node[draw=black, rectangle split,  rectangle split parts=4] (sn0x984d10){
\footnotesize{44.4444}
\nodepart{two}
\begin{tikzpicture}[scale=.2]
\node[circle, scale=0.75, fill] (tid0) at (1.5,1.5){};
\node[circle, scale=0.75, fill] (tid1) at (0.75,3){};
\node[circle, scale=0.75, fill] (tid3) at (0.75,4.5){};
\node[circle, scale=0.75, fill] (tid5) at (0.75,6){};
\node[circle, scale=0.75, fill] (tid6) at (0.75,7.5){};
\node[circle, scale=0.75, fill] (tid7) at (0.75,9){};
\node[circle, scale=0.75, fill, red] (tid8) at (0.75,10.5){};
\draw[](tid7) -- (tid8);
\draw[](tid6) -- (tid7);
\draw[](tid5) -- (tid6);
\draw[](tid3) -- (tid5);
\draw[](tid1) -- (tid3);
\node[circle, scale=0.75, fill] (tid2) at (2.25,3){};
\node[circle, scale=0.75, fill, red] (tid4) at (2.25,4.5){};
\draw[](tid2) -- (tid4);
\draw[](tid0) -- (tid1);
\draw[](tid0) -- (tid2);
\end{tikzpicture}
\nodepart{three}
\footnotesize{7.07812}
\nodepart{four}
\footnotesize{$50\:50$}
};
 & 
\node[draw=black, rectangle split,  rectangle split parts=4] (sn0x9897c0){
\footnotesize{22.2222}
\nodepart{two}
\begin{tikzpicture}[scale=.2]
\node[circle, scale=0.75, fill] (tid0) at (2.25,1.5){};
\node[circle, scale=0.75, fill] (tid1) at (1.5,3){};
\node[circle, scale=0.75, fill] (tid3) at (1.5,4.5){};
\node[circle, scale=0.75, fill] (tid4) at (0.75,6){};
\node[circle, scale=0.75, fill] (tid6) at (0.75,7.5){};
\node[circle, scale=0.75, fill] (tid7) at (0.75,9){};
\node[circle, scale=0.75, fill, red] (tid8) at (0.75,10.5){};
\draw[](tid7) -- (tid8);
\draw[](tid6) -- (tid7);
\draw[](tid4) -- (tid6);
\node[circle, scale=0.75, fill, red] (tid5) at (2.25,6){};
\draw[](tid3) -- (tid4);
\draw[](tid3) -- (tid5);
\draw[](tid1) -- (tid3);
\node[circle, scale=0.75, fill, red] (tid2) at (3.75,3){};
\draw[](tid0) -- (tid1);
\draw[](tid0) -- (tid2);
\end{tikzpicture}
\nodepart{three}
\footnotesize{7.07658}
\nodepart{four}
\footnotesize{$33\:33\:33$}
};
 & 
\node[draw=black, rectangle split,  rectangle split parts=4] (sn0x98a0b0){
\footnotesize{22.2222}
\nodepart{two}
\begin{tikzpicture}[scale=.2]
\node[circle, scale=0.75, fill] (tid0) at (2.25,1.5){};
\node[circle, scale=0.75, fill] (tid1) at (1.5,3){};
\node[circle, scale=0.75, fill] (tid3) at (1.5,4.5){};
\node[circle, scale=0.75, fill] (tid5) at (0.75,6){};
\node[circle, scale=0.75, fill] (tid7) at (0.75,7.5){};
\node[circle, scale=0.75, fill, red] (tid8) at (0.75,9){};
\draw[](tid7) -- (tid8);
\draw[](tid5) -- (tid7);
\node[circle, scale=0.75, fill, red] (tid6) at (2.25,6){};
\draw[](tid3) -- (tid5);
\draw[](tid3) -- (tid6);
\draw[](tid1) -- (tid3);
\node[circle, scale=0.75, fill] (tid2) at (3.75,3){};
\node[circle, scale=0.75, fill, red] (tid4) at (3.75,4.5){};
\draw[](tid2) -- (tid4);
\draw[](tid0) -- (tid1);
\draw[](tid0) -- (tid2);
\end{tikzpicture}
\nodepart{three}
\footnotesize{6.24942}
\nodepart{four}
\footnotesize{$33\:33\:33$}
};
 & 
\\
};
\end{scope}
\draw (sn0x983680.south) -- (sn0x984240.north);
\draw (sn0x983680.south) -- (sn0x984830.north);
\draw (sn0x984240.south) -- (sn0x984fb0.north);
\draw (sn0x984240.south) -- (sn0x984d10.north);
\draw (sn0x984830.south) -- (sn0x9897c0.north);
\draw (sn0x984830.south) -- (sn0x984d10.north);
\draw (sn0x984830.south) -- (sn0x98a0b0.north);
\end{tikzpicture}
%%% Local Variables:
%%% TeX-master: "thesis/thesis.tex"
%%% End: 
    \renewcommand{\leveltopI}{-19cm + \leveltop}
\renewcommand{\leveltopII}{-19cm + \leveltopI}
\renewcommand{\leveltopIII}{-19cm + \leveltopII}
\renewcommand{\leveltopIIII}{-19cm + \leveltopIII}
\renewcommand{\leveltopIIIII}{-19cm + \leveltopIIII}
\renewcommand{\leveltopIIIIII}{-19cm + \leveltopIIIII}
\renewcommand{\leveltopIIIIIII}{-19cm + \leveltopIIIIII}
\renewcommand{\leveltopIIIIIIII}{-19cm + \leveltopIIIIIII}
\renewcommand{\leveltopIIIIIIIII}{-19cm + \leveltopIIIIIIII}
\renewcommand{\leveltopIIIIIIIIII}{-19cm + \leveltopIIIIIIIII}
\renewcommand{\leveltopIIIIIIIIIII}{-19cm + \leveltopIIIIIIIIII}
\begin{tikzpicture}[scale=.2, anchor=south]
\begin{scope}[yshift=\leveltopI cm]
\matrix (line1)[column sep=0.5cm] {
\node[draw=black, rectangle split,  rectangle split parts=4] (sn0x18e59c0){
\footnotesize{100}
\nodepart{two}
\begin{tikzpicture}[scale=.2]
\node[circle, scale=0.75, fill] (tid0) at (3,1.5){};
\node[circle, scale=0.75, fill] (tid1) at (2.25,3){};
\node[circle, scale=0.75, fill] (tid3) at (2.25,4.5){};
\node[circle, scale=0.75, fill] (tid5) at (1.5,6){};
\node[circle, scale=0.75, fill] (tid7) at (1.5,7.5){};
\node[circle, scale=0.75, fill] (tid8) at (1.5,9){};
\node[circle, scale=0.75, fill, red] (tid9) at (0.75,10.5){};
\node[circle, scale=0.75, fill, red] (tid10) at (2.25,10.5){};
\draw[](tid8) -- (tid9);
\draw[](tid8) -- (tid10);
\draw[](tid7) -- (tid8);
\draw[](tid5) -- (tid7);
\node[circle, scale=0.75, fill] (tid6) at (3.75,6){};
\draw[](tid3) -- (tid5);
\draw[](tid3) -- (tid6);
\draw[](tid1) -- (tid3);
\node[circle, scale=0.75, fill] (tid2) at (5.25,3){};
\node[circle, scale=0.75, fill, red] (tid4) at (5.25,4.5){};
\draw[](tid2) -- (tid4);
\draw[](tid0) -- (tid1);
\draw[](tid0) -- (tid2);
\end{tikzpicture}
\nodepart{three}
\footnotesize{7.60798}
\nodepart{four}
\footnotesize{$33\:67$}
};
 & 
\\
};
\end{scope}
\begin{scope}[yshift=\leveltopII cm]
\matrix (line2)[column sep=0.5cm] {
\node[draw=black, rectangle split,  rectangle split parts=4] (sn0x18e2cc0){
\footnotesize{33.3333}
\nodepart{two}
\begin{tikzpicture}[scale=.2]
\node[circle, scale=0.75, fill] (tid0) at (3,1.5){};
\node[circle, scale=0.75, fill] (tid1) at (2.25,3){};
\node[circle, scale=0.75, fill] (tid3) at (2.25,4.5){};
\node[circle, scale=0.75, fill] (tid4) at (1.5,6){};
\node[circle, scale=0.75, fill] (tid6) at (1.5,7.5){};
\node[circle, scale=0.75, fill] (tid7) at (1.5,9){};
\node[circle, scale=0.75, fill, red] (tid8) at (0.75,10.5){};
\node[circle, scale=0.75, fill, red] (tid9) at (2.25,10.5){};
\draw[](tid7) -- (tid8);
\draw[](tid7) -- (tid9);
\draw[](tid6) -- (tid7);
\draw[](tid4) -- (tid6);
\node[circle, scale=0.75, fill, red] (tid5) at (3.75,6){};
\draw[](tid3) -- (tid4);
\draw[](tid3) -- (tid5);
\draw[](tid1) -- (tid3);
\node[circle, scale=0.75, fill] (tid2) at (5.25,3){};
\draw[](tid0) -- (tid1);
\draw[](tid0) -- (tid2);
\end{tikzpicture}
\nodepart{three}
\footnotesize{7.55453}
\nodepart{four}
\footnotesize{$33\:67$}
};
 & 
\node[draw=black, rectangle split,  rectangle split parts=4] (sn0x18e5770){
\footnotesize{66.6667}
\nodepart{two}
\begin{tikzpicture}[scale=.2]
\node[circle, scale=0.75, fill] (tid0) at (2.25,1.5){};
\node[circle, scale=0.75, fill] (tid1) at (1.5,3){};
\node[circle, scale=0.75, fill] (tid3) at (1.5,4.5){};
\node[circle, scale=0.75, fill] (tid5) at (0.75,6){};
\node[circle, scale=0.75, fill] (tid7) at (0.75,7.5){};
\node[circle, scale=0.75, fill] (tid8) at (0.75,9){};
\node[circle, scale=0.75, fill, red] (tid9) at (0.75,10.5){};
\draw[](tid8) -- (tid9);
\draw[](tid7) -- (tid8);
\draw[](tid5) -- (tid7);
\node[circle, scale=0.75, fill, red] (tid6) at (2.25,6){};
\draw[](tid3) -- (tid5);
\draw[](tid3) -- (tid6);
\draw[](tid1) -- (tid3);
\node[circle, scale=0.75, fill] (tid2) at (3.75,3){};
\node[circle, scale=0.75, fill, red] (tid4) at (3.75,4.5){};
\draw[](tid2) -- (tid4);
\draw[](tid0) -- (tid1);
\draw[](tid0) -- (tid2);
\end{tikzpicture}
\nodepart{three}
\footnotesize{7.13471}
\nodepart{four}
\footnotesize{$33\:33\:33$}
};
 & 
\\
};
\end{scope}
\begin{scope}[yshift=\leveltopIII cm]
\matrix (line3)[column sep=0.5cm] {
\node[draw=black, rectangle split,  rectangle split parts=4] (sn0x18e21e0){
\footnotesize{11.1111}
\nodepart{two}
\begin{tikzpicture}[scale=.2]
\node[circle, scale=0.75, fill] (tid0) at (2.25,1.5){};
\node[circle, scale=0.75, fill] (tid1) at (1.5,3){};
\node[circle, scale=0.75, fill] (tid3) at (1.5,4.5){};
\node[circle, scale=0.75, fill] (tid4) at (1.5,6){};
\node[circle, scale=0.75, fill] (tid5) at (1.5,7.5){};
\node[circle, scale=0.75, fill] (tid6) at (1.5,9){};
\node[circle, scale=0.75, fill, red] (tid7) at (0.75,10.5){};
\node[circle, scale=0.75, fill, red] (tid8) at (2.25,10.5){};
\draw[](tid6) -- (tid7);
\draw[](tid6) -- (tid8);
\draw[](tid5) -- (tid6);
\draw[](tid4) -- (tid5);
\draw[](tid3) -- (tid4);
\draw[](tid1) -- (tid3);
\node[circle, scale=0.75, fill, red] (tid2) at (3.75,3){};
\draw[](tid0) -- (tid1);
\draw[](tid0) -- (tid2);
\end{tikzpicture}
\nodepart{three}
\footnotesize{7.51042}
\nodepart{four}
\footnotesize{$33\:67$}
};
 & 
\node[draw=black, rectangle split,  rectangle split parts=4] (sn0x18e1e60){
\footnotesize{44.4444}
\nodepart{two}
\begin{tikzpicture}[scale=.2]
\node[circle, scale=0.75, fill] (tid0) at (2.25,1.5){};
\node[circle, scale=0.75, fill] (tid1) at (1.5,3){};
\node[circle, scale=0.75, fill] (tid3) at (1.5,4.5){};
\node[circle, scale=0.75, fill] (tid4) at (0.75,6){};
\node[circle, scale=0.75, fill] (tid6) at (0.75,7.5){};
\node[circle, scale=0.75, fill] (tid7) at (0.75,9){};
\node[circle, scale=0.75, fill, red] (tid8) at (0.75,10.5){};
\draw[](tid7) -- (tid8);
\draw[](tid6) -- (tid7);
\draw[](tid4) -- (tid6);
\node[circle, scale=0.75, fill, red] (tid5) at (2.25,6){};
\draw[](tid3) -- (tid4);
\draw[](tid3) -- (tid5);
\draw[](tid1) -- (tid3);
\node[circle, scale=0.75, fill, red] (tid2) at (3.75,3){};
\draw[](tid0) -- (tid1);
\draw[](tid0) -- (tid2);
\end{tikzpicture}
\nodepart{three}
\footnotesize{7.07658}
\nodepart{four}
\footnotesize{$33\:33\:33$}
};
 & 
\node[draw=black, rectangle split,  rectangle split parts=4] (sn0x18e45f0){
\footnotesize{22.2222}
\nodepart{two}
\begin{tikzpicture}[scale=.2]
\node[circle, scale=0.75, fill] (tid0) at (1.5,1.5){};
\node[circle, scale=0.75, fill] (tid1) at (0.75,3){};
\node[circle, scale=0.75, fill] (tid3) at (0.75,4.5){};
\node[circle, scale=0.75, fill] (tid5) at (0.75,6){};
\node[circle, scale=0.75, fill] (tid6) at (0.75,7.5){};
\node[circle, scale=0.75, fill] (tid7) at (0.75,9){};
\node[circle, scale=0.75, fill, red] (tid8) at (0.75,10.5){};
\draw[](tid7) -- (tid8);
\draw[](tid6) -- (tid7);
\draw[](tid5) -- (tid6);
\draw[](tid3) -- (tid5);
\draw[](tid1) -- (tid3);
\node[circle, scale=0.75, fill] (tid2) at (2.25,3){};
\node[circle, scale=0.75, fill, red] (tid4) at (2.25,4.5){};
\draw[](tid2) -- (tid4);
\draw[](tid0) -- (tid1);
\draw[](tid0) -- (tid2);
\end{tikzpicture}
\nodepart{three}
\footnotesize{7.07812}
\nodepart{four}
\footnotesize{$50\:50$}
};
 & 
\node[draw=black, rectangle split,  rectangle split parts=4] (sn0x18e49e0){
\footnotesize{22.2222}
\nodepart{two}
\begin{tikzpicture}[scale=.2]
\node[circle, scale=0.75, fill] (tid0) at (2.25,1.5){};
\node[circle, scale=0.75, fill] (tid1) at (1.5,3){};
\node[circle, scale=0.75, fill] (tid3) at (1.5,4.5){};
\node[circle, scale=0.75, fill] (tid5) at (0.75,6){};
\node[circle, scale=0.75, fill] (tid7) at (0.75,7.5){};
\node[circle, scale=0.75, fill, red] (tid8) at (0.75,9){};
\draw[](tid7) -- (tid8);
\draw[](tid5) -- (tid7);
\node[circle, scale=0.75, fill, red] (tid6) at (2.25,6){};
\draw[](tid3) -- (tid5);
\draw[](tid3) -- (tid6);
\draw[](tid1) -- (tid3);
\node[circle, scale=0.75, fill] (tid2) at (3.75,3){};
\node[circle, scale=0.75, fill, red] (tid4) at (3.75,4.5){};
\draw[](tid2) -- (tid4);
\draw[](tid0) -- (tid1);
\draw[](tid0) -- (tid2);
\end{tikzpicture}
\nodepart{three}
\footnotesize{6.24942}
\nodepart{four}
\footnotesize{$33\:33\:33$}
};
 & 
\\
};
\end{scope}
\draw (sn0x18e59c0.south) -- (sn0x18e2cc0.north);
\draw (sn0x18e59c0.south) -- (sn0x18e5770.north);
\draw (sn0x18e2cc0.south) -- (sn0x18e21e0.north);
\draw (sn0x18e2cc0.south) -- (sn0x18e1e60.north);
\draw (sn0x18e5770.south) -- (sn0x18e1e60.north);
\draw (sn0x18e5770.south) -- (sn0x18e45f0.north);
\draw (sn0x18e5770.south) -- (sn0x18e49e0.north);
\end{tikzpicture}
%%% Local Variables:
%%% TeX-master: "thesis/thesis.tex"
%%% End: 

  \end{minipage}
}
\frame{
  \begin{minipage}{.25\textwidth}
    \subsection{P3-HLF not optimal for 00123455799}
    \renewcommand{\leveltopI}{-19cm + \leveltop}
\renewcommand{\leveltopII}{-19cm + \leveltopI}
\renewcommand{\leveltopIII}{-19cm + \leveltopII}
\renewcommand{\leveltopIIII}{-19cm + \leveltopIII}
\renewcommand{\leveltopIIIII}{-19cm + \leveltopIIII}
\renewcommand{\leveltopIIIIII}{-19cm + \leveltopIIIII}
\renewcommand{\leveltopIIIIIII}{-19cm + \leveltopIIIIII}
\renewcommand{\leveltopIIIIIIII}{-19cm + \leveltopIIIIIII}
\renewcommand{\leveltopIIIIIIIII}{-19cm + \leveltopIIIIIIII}
\renewcommand{\leveltopIIIIIIIIII}{-19cm + \leveltopIIIIIIIII}
\renewcommand{\leveltopIIIIIIIIIII}{-19cm + \leveltopIIIIIIIIII}
\renewcommand{\leveltopIIIIIIIIIIII}{-19cm + \leveltopIIIIIIIIIII}
\begin{tikzpicture}[scale=.2, anchor=south]
\begin{scope}[yshift=\leveltopI cm]
\matrix (line1)[column sep=0.5cm] {
\node[draw=black, rectangle split,  rectangle split parts=4] (sn0x90bf0b8){
\footnotesize{100}
\nodepart{two}
\begin{tikzpicture}[scale=.2]
\node[circle, scale=0.75, fill] (tid0) at (3,1.5){};
\node[circle, scale=0.75, fill] (tid1) at (2.25,3){};
\node[circle, scale=0.75, fill] (tid3) at (2.25,4.5){};
\node[circle, scale=0.75, fill] (tid5) at (2.25,6){};
\node[circle, scale=0.75, fill] (tid7) at (1.5,7.5){};
\node[circle, scale=0.75, fill] (tid9) at (1.5,9){};
\node[circle, scale=0.75, fill, red] (tid10) at (0.75,10.5){};
\node[circle, scale=0.75, fill, red] (tid11) at (2.25,10.5){};
\draw[](tid9) -- (tid10);
\draw[](tid9) -- (tid11);
\draw[](tid7) -- (tid9);
\node[circle, scale=0.75, fill, red] (tid8) at (3.75,7.5){};
\draw[](tid5) -- (tid7);
\draw[](tid5) -- (tid8);
\draw[](tid3) -- (tid5);
\draw[](tid1) -- (tid3);
\node[circle, scale=0.75, fill] (tid2) at (5.25,3){};
\node[circle, scale=0.75, fill] (tid4) at (5.25,4.5){};
\node[circle, scale=0.75, fill] (tid6) at (5.25,6){};
\draw[](tid4) -- (tid6);
\draw[](tid2) -- (tid4);
\draw[](tid0) -- (tid1);
\draw[](tid0) -- (tid2);
\end{tikzpicture}
\nodepart{three}
\footnotesize{7.77328}
\nodepart{four}
\footnotesize{$33\:67$}
};
 & 
\\
};
\end{scope}
\begin{scope}[yshift=\leveltopII cm]
\matrix (line2)[column sep=0.5cm] {
\node[draw=black, rectangle split,  rectangle split parts=4] (sn0x90be790){
\footnotesize{33.3333}
\nodepart{two}
\begin{tikzpicture}[scale=.2]
\node[circle, scale=0.75, fill] (tid0) at (2.25,1.5){};
\node[circle, scale=0.75, fill] (tid1) at (1.5,3){};
\node[circle, scale=0.75, fill] (tid3) at (1.5,4.5){};
\node[circle, scale=0.75, fill] (tid5) at (1.5,6){};
\node[circle, scale=0.75, fill] (tid7) at (1.5,7.5){};
\node[circle, scale=0.75, fill] (tid8) at (1.5,9){};
\node[circle, scale=0.75, fill, red] (tid9) at (0.75,10.5){};
\node[circle, scale=0.75, fill, red] (tid10) at (2.25,10.5){};
\draw[](tid8) -- (tid9);
\draw[](tid8) -- (tid10);
\draw[](tid7) -- (tid8);
\draw[](tid5) -- (tid7);
\draw[](tid3) -- (tid5);
\draw[](tid1) -- (tid3);
\node[circle, scale=0.75, fill] (tid2) at (3.75,3){};
\node[circle, scale=0.75, fill] (tid4) at (3.75,4.5){};
\node[circle, scale=0.75, fill, red] (tid6) at (3.75,6){};
\draw[](tid4) -- (tid6);
\draw[](tid2) -- (tid4);
\draw[](tid0) -- (tid1);
\draw[](tid0) -- (tid2);
\end{tikzpicture}
\nodepart{three}
\footnotesize{7.66696}
\nodepart{four}
\footnotesize{$33\:67$}
};
 & 
\node[draw=black, rectangle split,  rectangle split parts=4] (sn0x90bd580){
\footnotesize{66.6667}
\nodepart{two}
\begin{tikzpicture}[scale=.2]
\node[circle, scale=0.75, fill] (tid0) at (2.25,1.5){};
\node[circle, scale=0.75, fill] (tid1) at (1.5,3){};
\node[circle, scale=0.75, fill] (tid3) at (1.5,4.5){};
\node[circle, scale=0.75, fill] (tid5) at (1.5,6){};
\node[circle, scale=0.75, fill] (tid7) at (0.75,7.5){};
\node[circle, scale=0.75, fill] (tid9) at (0.75,9){};
\node[circle, scale=0.75, fill, red] (tid10) at (0.75,10.5){};
\draw[](tid9) -- (tid10);
\draw[](tid7) -- (tid9);
\node[circle, scale=0.75, fill, red] (tid8) at (2.25,7.5){};
\draw[](tid5) -- (tid7);
\draw[](tid5) -- (tid8);
\draw[](tid3) -- (tid5);
\draw[](tid1) -- (tid3);
\node[circle, scale=0.75, fill] (tid2) at (3.75,3){};
\node[circle, scale=0.75, fill] (tid4) at (3.75,4.5){};
\node[circle, scale=0.75, fill, red] (tid6) at (3.75,6){};
\draw[](tid4) -- (tid6);
\draw[](tid2) -- (tid4);
\draw[](tid0) -- (tid1);
\draw[](tid0) -- (tid2);
\end{tikzpicture}
\nodepart{three}
\footnotesize{7.32644}
\nodepart{four}
\footnotesize{$33\:33\:33$}
};
 & 
\\
};
\end{scope}
\begin{scope}[yshift=\leveltopIII cm]
\matrix (line3)[column sep=0.5cm] {
\node[draw=black, rectangle split,  rectangle split parts=4] (sn0x90beb08){
\footnotesize{11.1111}
\nodepart{two}
\begin{tikzpicture}[scale=.2]
\node[circle, scale=0.75, fill] (tid0) at (2.25,1.5){};
\node[circle, scale=0.75, fill] (tid1) at (1.5,3){};
\node[circle, scale=0.75, fill] (tid3) at (1.5,4.5){};
\node[circle, scale=0.75, fill] (tid5) at (1.5,6){};
\node[circle, scale=0.75, fill] (tid6) at (1.5,7.5){};
\node[circle, scale=0.75, fill] (tid7) at (1.5,9){};
\node[circle, scale=0.75, fill, red] (tid8) at (0.75,10.5){};
\node[circle, scale=0.75, fill, red] (tid9) at (2.25,10.5){};
\draw[](tid7) -- (tid8);
\draw[](tid7) -- (tid9);
\draw[](tid6) -- (tid7);
\draw[](tid5) -- (tid6);
\draw[](tid3) -- (tid5);
\draw[](tid1) -- (tid3);
\node[circle, scale=0.75, fill] (tid2) at (3.75,3){};
\node[circle, scale=0.75, fill, red] (tid4) at (3.75,4.5){};
\draw[](tid2) -- (tid4);
\draw[](tid0) -- (tid1);
\draw[](tid0) -- (tid2);
\end{tikzpicture}
\nodepart{three}
\footnotesize{7.55556}
\nodepart{four}
\footnotesize{$33\:67$}
};
 & 
\node[draw=black, rectangle split,  rectangle split parts=4] (sn0x90bf898){
\footnotesize{44.4444}
\nodepart{two}
\begin{tikzpicture}[scale=.2]
\node[circle, scale=0.75, fill] (tid0) at (1.5,1.5){};
\node[circle, scale=0.75, fill] (tid1) at (0.75,3){};
\node[circle, scale=0.75, fill] (tid3) at (0.75,4.5){};
\node[circle, scale=0.75, fill] (tid5) at (0.75,6){};
\node[circle, scale=0.75, fill] (tid7) at (0.75,7.5){};
\node[circle, scale=0.75, fill] (tid8) at (0.75,9){};
\node[circle, scale=0.75, fill, red] (tid9) at (0.75,10.5){};
\draw[](tid8) -- (tid9);
\draw[](tid7) -- (tid8);
\draw[](tid5) -- (tid7);
\draw[](tid3) -- (tid5);
\draw[](tid1) -- (tid3);
\node[circle, scale=0.75, fill] (tid2) at (2.25,3){};
\node[circle, scale=0.75, fill] (tid4) at (2.25,4.5){};
\node[circle, scale=0.75, fill, red] (tid6) at (2.25,6){};
\draw[](tid4) -- (tid6);
\draw[](tid2) -- (tid4);
\draw[](tid0) -- (tid1);
\draw[](tid0) -- (tid2);
\end{tikzpicture}
\nodepart{three}
\footnotesize{7.22266}
\nodepart{four}
\footnotesize{$50\:50$}
};
 & 
\node[draw=black, rectangle split,  rectangle split parts=4] (sn0x90c1df8){
\footnotesize{22.2222}
\nodepart{two}
\begin{tikzpicture}[scale=.2]
\node[circle, scale=0.75, fill] (tid0) at (2.25,1.5){};
\node[circle, scale=0.75, fill] (tid1) at (1.5,3){};
\node[circle, scale=0.75, fill] (tid3) at (1.5,4.5){};
\node[circle, scale=0.75, fill] (tid5) at (1.5,6){};
\node[circle, scale=0.75, fill] (tid6) at (0.75,7.5){};
\node[circle, scale=0.75, fill] (tid8) at (0.75,9){};
\node[circle, scale=0.75, fill, red] (tid9) at (0.75,10.5){};
\draw[](tid8) -- (tid9);
\draw[](tid6) -- (tid8);
\node[circle, scale=0.75, fill, red] (tid7) at (2.25,7.5){};
\draw[](tid5) -- (tid6);
\draw[](tid5) -- (tid7);
\draw[](tid3) -- (tid5);
\draw[](tid1) -- (tid3);
\node[circle, scale=0.75, fill] (tid2) at (3.75,3){};
\node[circle, scale=0.75, fill, red] (tid4) at (3.75,4.5){};
\draw[](tid2) -- (tid4);
\draw[](tid0) -- (tid1);
\draw[](tid0) -- (tid2);
\end{tikzpicture}
\nodepart{three}
\footnotesize{7.19387}
\nodepart{four}
\footnotesize{$33\:33\:33$}
};
 & 
\node[draw=black, rectangle split,  rectangle split parts=4] (sn0x90c3bd8){
\footnotesize{22.2222}
\nodepart{two}
\begin{tikzpicture}[scale=.2]
\node[circle, scale=0.75, fill] (tid0) at (2.25,1.5){};
\node[circle, scale=0.75, fill] (tid1) at (1.5,3){};
\node[circle, scale=0.75, fill] (tid3) at (1.5,4.5){};
\node[circle, scale=0.75, fill] (tid5) at (1.5,6){};
\node[circle, scale=0.75, fill] (tid7) at (0.75,7.5){};
\node[circle, scale=0.75, fill, red] (tid9) at (0.75,9){};
\draw[](tid7) -- (tid9);
\node[circle, scale=0.75, fill, red] (tid8) at (2.25,7.5){};
\draw[](tid5) -- (tid7);
\draw[](tid5) -- (tid8);
\draw[](tid3) -- (tid5);
\draw[](tid1) -- (tid3);
\node[circle, scale=0.75, fill] (tid2) at (3.75,3){};
\node[circle, scale=0.75, fill] (tid4) at (3.75,4.5){};
\node[circle, scale=0.75, fill, red] (tid6) at (3.75,6){};
\draw[](tid4) -- (tid6);
\draw[](tid2) -- (tid4);
\draw[](tid0) -- (tid1);
\draw[](tid0) -- (tid2);
\end{tikzpicture}
\nodepart{three}
\footnotesize{6.56279}
\nodepart{four}
\footnotesize{$33\:33\:33$}
};
 & 
\\
};
\end{scope}
\draw (sn0x90bf0b8.south) -- (sn0x90be790.north);
\draw (sn0x90bf0b8.south) -- (sn0x90bd580.north);
\draw (sn0x90be790.south) -- (sn0x90beb08.north);
\draw (sn0x90be790.south) -- (sn0x90bf898.north);
\draw (sn0x90bd580.south) -- (sn0x90c1df8.north);
\draw (sn0x90bd580.south) -- (sn0x90bf898.north);
\draw (sn0x90bd580.south) -- (sn0x90c3bd8.north);
\end{tikzpicture}
%%% Local Variables:
%%% TeX-master: "thesis/thesis.tex"
%%% End: 
    \input{../00123455799opt.tex}
  \end{minipage}
}

\renewcommand{\leveltopI}{-10cm + \leveltop}
\renewcommand{\leveltopII}{-10cm + \leveltopI}
\renewcommand{\leveltopIII}{-10cm + \leveltopII}
\renewcommand{\leveltopIIII}{-10cm + \leveltopIII}
\renewcommand{\leveltopIIIII}{-10cm + \leveltopIIII}
\renewcommand{\leveltopIIIIII}{-10cm + \leveltopIIIII}
\renewcommand{\leveltopIIIIIII}{-10cm + \leveltopIIIIII}
\renewcommand{\leveltopIIIIIIII}{-10cm + \leveltopIIIIIII}
\renewcommand{\leveltopIIIIIIIII}{-10cm + \leveltopIIIIIIII}
\renewcommand{\leveltopIIIIIIIIII}{-10cm + \leveltopIIIIIIIII}
\renewcommand{\leveltopIIIIIIIIIII}{-10cm + \leveltopIIIIIIIIII}
\begin{tikzpicture}[scale=.2, anchor=south, rotate=90]
\begin{scope}[yshift=\leveltopI cm, anchor = center]
\matrix (line1)[row sep=0.5cm] {
\node[draw=black, rectangle split,  rectangle split parts=4] (sn0x22a7ba0){
\footnotesize{100}
\nodepart{two}
\begin{tikzpicture}[scale=.2]
\node[circle, scale=0.75, fill] (tid0) at (4.5,1.5){};
\node[circle, scale=0.75, fill] (tid1) at (2.25,3){};
\node[circle, scale=0.75, fill] (tid4) at (0.75,4.5){};
\node[circle, scale=0.75, fill] (tid5) at (2.25,4.5){};
\node[circle, scale=0.75, fill] (tid6) at (3.75,4.5){};
\draw[](tid1) -- (tid4);
\draw[](tid1) -- (tid5);
\draw[](tid1) -- (tid6);
\node[circle, scale=0.75, fill] (tid2) at (6,3){};
\node[circle, scale=0.75, fill] (tid7) at (5.25,4.5){};
\node[circle, scale=0.75, fill, task_scheduled] (tid10) at (5.25,6){};
\draw[](tid7) -- (tid10);
\node[circle, scale=0.75, fill, task_scheduled] (tid8) at (6.75,4.5){};
\draw[](tid2) -- (tid7);
\draw[](tid2) -- (tid8);
\node[circle, scale=0.75, fill] (tid3) at (8.25,3){};
\node[circle, scale=0.75, fill] (tid9) at (8.25,4.5){};
\draw[](tid3) -- (tid9);
\draw[](tid0) -- (tid1);
\draw[](tid0) -- (tid2);
\draw[](tid0) -- (tid3);
\end{tikzpicture}
\nodepart{three}
\footnotesize{6.63281}
\nodepart{four}
\footnotesize{$38\:12\:30\:10\:10$}
};
 \\ 
\\
};
\end{scope}
\begin{scope}[yshift=\leveltopII cm, anchor = center]
\matrix (line2)[row sep=0.5cm] {
\node[draw=black, rectangle split,  rectangle split parts=4] (sn0x22ab160){
\footnotesize{37.5}
\nodepart{two}
\begin{tikzpicture}[scale=.2]
\node[circle, scale=0.75, fill] (tid0) at (3.75,1.5){};
\node[circle, scale=0.75, fill] (tid1) at (2.25,3){};
\node[circle, scale=0.75, fill, task_scheduled] (tid4) at (0.75,4.5){};
\node[circle, scale=0.75, fill] (tid5) at (2.25,4.5){};
\node[circle, scale=0.75, fill] (tid6) at (3.75,4.5){};
\draw[](tid1) -- (tid4);
\draw[](tid1) -- (tid5);
\draw[](tid1) -- (tid6);
\node[circle, scale=0.75, fill] (tid2) at (5.25,3){};
\node[circle, scale=0.75, fill] (tid7) at (5.25,4.5){};
\node[circle, scale=0.75, fill, task_scheduled] (tid9) at (5.25,6){};
\draw[](tid7) -- (tid9);
\draw[](tid2) -- (tid7);
\node[circle, scale=0.75, fill] (tid3) at (6.75,3){};
\node[circle, scale=0.75, fill] (tid8) at (6.75,4.5){};
\draw[](tid3) -- (tid8);
\draw[](tid0) -- (tid1);
\draw[](tid0) -- (tid2);
\draw[](tid0) -- (tid3);
\end{tikzpicture}
\nodepart{three}
\footnotesize{6.14062}
\nodepart{four}
\footnotesize{$25\:25\:33\:17$}
};
 \\ 
\node[draw=black, rectangle split,  rectangle split parts=4] (sn0x22aaaa0){
\footnotesize{12.5}
\nodepart{two}
\begin{tikzpicture}[scale=.2]
\node[circle, scale=0.75, fill] (tid0) at (3.75,1.5){};
\node[circle, scale=0.75, fill] (tid1) at (2.25,3){};
\node[circle, scale=0.75, fill] (tid4) at (0.75,4.5){};
\node[circle, scale=0.75, fill] (tid5) at (2.25,4.5){};
\node[circle, scale=0.75, fill] (tid6) at (3.75,4.5){};
\draw[](tid1) -- (tid4);
\draw[](tid1) -- (tid5);
\draw[](tid1) -- (tid6);
\node[circle, scale=0.75, fill] (tid2) at (5.25,3){};
\node[circle, scale=0.75, fill] (tid7) at (5.25,4.5){};
\node[circle, scale=0.75, fill, task_scheduled] (tid9) at (5.25,6){};
\draw[](tid7) -- (tid9);
\draw[](tid2) -- (tid7);
\node[circle, scale=0.75, fill] (tid3) at (6.75,3){};
\node[circle, scale=0.75, fill, task_scheduled] (tid8) at (6.75,4.5){};
\draw[](tid3) -- (tid8);
\draw[](tid0) -- (tid1);
\draw[](tid0) -- (tid2);
\draw[](tid0) -- (tid3);
\end{tikzpicture}
\nodepart{three}
\footnotesize{6.14062}
\nodepart{four}
\footnotesize{$50\:38\:12$}
};
 \\ 
\node[draw=black, rectangle split,  rectangle split parts=4] (sn0x22ab2c0){
\footnotesize{30}
\nodepart{two}
\begin{tikzpicture}[scale=.2]
\node[circle, scale=0.75, fill] (tid0) at (4.5,1.5){};
\node[circle, scale=0.75, fill] (tid1) at (2.25,3){};
\node[circle, scale=0.75, fill, task_scheduled] (tid4) at (0.75,4.5){};
\node[circle, scale=0.75, fill] (tid5) at (2.25,4.5){};
\node[circle, scale=0.75, fill] (tid6) at (3.75,4.5){};
\draw[](tid1) -- (tid4);
\draw[](tid1) -- (tid5);
\draw[](tid1) -- (tid6);
\node[circle, scale=0.75, fill] (tid2) at (6,3){};
\node[circle, scale=0.75, fill, task_scheduled] (tid7) at (5.25,4.5){};
\node[circle, scale=0.75, fill] (tid8) at (6.75,4.5){};
\draw[](tid2) -- (tid7);
\draw[](tid2) -- (tid8);
\node[circle, scale=0.75, fill] (tid3) at (8.25,3){};
\node[circle, scale=0.75, fill] (tid9) at (8.25,4.5){};
\draw[](tid3) -- (tid9);
\draw[](tid0) -- (tid1);
\draw[](tid0) -- (tid2);
\draw[](tid0) -- (tid3);
\end{tikzpicture}
\nodepart{three}
\footnotesize{6.125}
\nodepart{four}
\footnotesize{$25\:25\:25\:12\:12$}
};
 \\ 
\node[draw=black, rectangle split,  rectangle split parts=4] (sn0x22aba90){
\footnotesize{10}
\nodepart{two}
\begin{tikzpicture}[scale=.2]
\node[circle, scale=0.75, fill] (tid0) at (4.5,1.5){};
\node[circle, scale=0.75, fill] (tid1) at (2.25,3){};
\node[circle, scale=0.75, fill] (tid4) at (0.75,4.5){};
\node[circle, scale=0.75, fill] (tid5) at (2.25,4.5){};
\node[circle, scale=0.75, fill] (tid6) at (3.75,4.5){};
\draw[](tid1) -- (tid4);
\draw[](tid1) -- (tid5);
\draw[](tid1) -- (tid6);
\node[circle, scale=0.75, fill] (tid2) at (6,3){};
\node[circle, scale=0.75, fill, task_scheduled] (tid7) at (5.25,4.5){};
\node[circle, scale=0.75, fill, task_scheduled] (tid8) at (6.75,4.5){};
\draw[](tid2) -- (tid7);
\draw[](tid2) -- (tid8);
\node[circle, scale=0.75, fill] (tid3) at (8.25,3){};
\node[circle, scale=0.75, fill] (tid9) at (8.25,4.5){};
\draw[](tid3) -- (tid9);
\draw[](tid0) -- (tid1);
\draw[](tid0) -- (tid2);
\draw[](tid0) -- (tid3);
\end{tikzpicture}
\nodepart{three}
\footnotesize{6.125}
\nodepart{four}
\footnotesize{$75\:25$}
};
 \\ 
\node[draw=black, rectangle split,  rectangle split parts=4] (sn0x22ab500){
\footnotesize{10}
\nodepart{two}
\begin{tikzpicture}[scale=.2]
\node[circle, scale=0.75, fill] (tid0) at (4.5,1.5){};
\node[circle, scale=0.75, fill] (tid1) at (2.25,3){};
\node[circle, scale=0.75, fill] (tid4) at (0.75,4.5){};
\node[circle, scale=0.75, fill] (tid5) at (2.25,4.5){};
\node[circle, scale=0.75, fill] (tid6) at (3.75,4.5){};
\draw[](tid1) -- (tid4);
\draw[](tid1) -- (tid5);
\draw[](tid1) -- (tid6);
\node[circle, scale=0.75, fill] (tid2) at (6,3){};
\node[circle, scale=0.75, fill, task_scheduled] (tid7) at (5.25,4.5){};
\node[circle, scale=0.75, fill] (tid8) at (6.75,4.5){};
\draw[](tid2) -- (tid7);
\draw[](tid2) -- (tid8);
\node[circle, scale=0.75, fill] (tid3) at (8.25,3){};
\node[circle, scale=0.75, fill, task_scheduled] (tid9) at (8.25,4.5){};
\draw[](tid3) -- (tid9);
\draw[](tid0) -- (tid1);
\draw[](tid0) -- (tid2);
\draw[](tid0) -- (tid3);
\end{tikzpicture}
\nodepart{three}
\footnotesize{6.125}
\nodepart{four}
\footnotesize{$38\:12\:38\:12$}
};
 \\ 
\\
};
\end{scope}
\begin{scope}[yshift=\leveltopIII cm, anchor = center]
\matrix (line3)[row sep=0.5cm] {
\node[draw=black, rectangle split,  rectangle split parts=4] (sn0x22b4a40){
\footnotesize{6.25}
\nodepart{two}
\begin{tikzpicture}[scale=.2]
\node[circle, scale=0.75, fill] (tid0) at (3.75,1.5){};
\node[circle, scale=0.75, fill] (tid1) at (2.25,3){};
\node[circle, scale=0.75, fill, task_scheduled] (tid4) at (0.75,4.5){};
\node[circle, scale=0.75, fill] (tid5) at (2.25,4.5){};
\node[circle, scale=0.75, fill] (tid6) at (3.75,4.5){};
\draw[](tid1) -- (tid4);
\draw[](tid1) -- (tid5);
\draw[](tid1) -- (tid6);
\node[circle, scale=0.75, fill] (tid2) at (5.25,3){};
\node[circle, scale=0.75, fill] (tid7) at (5.25,4.5){};
\node[circle, scale=0.75, fill, task_scheduled] (tid8) at (5.25,6){};
\draw[](tid7) -- (tid8);
\draw[](tid2) -- (tid7);
\node[circle, scale=0.75, fill] (tid3) at (6.75,3){};
\draw[](tid0) -- (tid1);
\draw[](tid0) -- (tid2);
\draw[](tid0) -- (tid3);
\end{tikzpicture}
\nodepart{three}
\footnotesize{5.65625}
\nodepart{four}
\footnotesize{$33\:17\:50$}
};
 \\ 
\node[draw=black, rectangle split,  rectangle split parts=4] (sn0x22b7500){
\footnotesize{3.75}
\nodepart{two}
\begin{tikzpicture}[scale=.2]
\node[circle, scale=0.75, fill] (tid0) at (4.5,1.5){};
\node[circle, scale=0.75, fill] (tid1) at (2.25,3){};
\node[circle, scale=0.75, fill, task_scheduled] (tid4) at (0.75,4.5){};
\node[circle, scale=0.75, fill] (tid5) at (2.25,4.5){};
\node[circle, scale=0.75, fill] (tid6) at (3.75,4.5){};
\draw[](tid1) -- (tid4);
\draw[](tid1) -- (tid5);
\draw[](tid1) -- (tid6);
\node[circle, scale=0.75, fill] (tid2) at (6,3){};
\node[circle, scale=0.75, fill, task_scheduled] (tid7) at (5.25,4.5){};
\node[circle, scale=0.75, fill] (tid8) at (6.75,4.5){};
\draw[](tid2) -- (tid7);
\draw[](tid2) -- (tid8);
\node[circle, scale=0.75, fill] (tid3) at (8.25,3){};
\draw[](tid0) -- (tid1);
\draw[](tid0) -- (tid2);
\draw[](tid0) -- (tid3);
\end{tikzpicture}
\nodepart{three}
\footnotesize{5.625}
\nodepart{four}
\footnotesize{$33\:17\:33\:17$}
};
 \\ 
\node[draw=black, rectangle split,  rectangle split parts=4] (sn0x22b7aa0){
\footnotesize{1.25}
\nodepart{two}
\begin{tikzpicture}[scale=.2]
\node[circle, scale=0.75, fill] (tid0) at (4.5,1.5){};
\node[circle, scale=0.75, fill] (tid1) at (2.25,3){};
\node[circle, scale=0.75, fill] (tid4) at (0.75,4.5){};
\node[circle, scale=0.75, fill] (tid5) at (2.25,4.5){};
\node[circle, scale=0.75, fill] (tid6) at (3.75,4.5){};
\draw[](tid1) -- (tid4);
\draw[](tid1) -- (tid5);
\draw[](tid1) -- (tid6);
\node[circle, scale=0.75, fill] (tid2) at (6,3){};
\node[circle, scale=0.75, fill, task_scheduled] (tid7) at (5.25,4.5){};
\node[circle, scale=0.75, fill, task_scheduled] (tid8) at (6.75,4.5){};
\draw[](tid2) -- (tid7);
\draw[](tid2) -- (tid8);
\node[circle, scale=0.75, fill] (tid3) at (8.25,3){};
\draw[](tid0) -- (tid1);
\draw[](tid0) -- (tid2);
\draw[](tid0) -- (tid3);
\end{tikzpicture}
\nodepart{three}
\footnotesize{5.625}
\nodepart{four}
\footnotesize{$1$}
};
 \\ 
\node[draw=black, rectangle split,  rectangle split parts=4] (sn0x22ac710){
\footnotesize{16.875}
\nodepart{two}
\begin{tikzpicture}[scale=.2]
\node[circle, scale=0.75, fill] (tid0) at (3.75,1.5){};
\node[circle, scale=0.75, fill] (tid1) at (2.25,3){};
\node[circle, scale=0.75, fill, task_scheduled] (tid4) at (0.75,4.5){};
\node[circle, scale=0.75, fill, task_scheduled] (tid5) at (2.25,4.5){};
\node[circle, scale=0.75, fill] (tid6) at (3.75,4.5){};
\draw[](tid1) -- (tid4);
\draw[](tid1) -- (tid5);
\draw[](tid1) -- (tid6);
\node[circle, scale=0.75, fill] (tid2) at (5.25,3){};
\node[circle, scale=0.75, fill] (tid7) at (5.25,4.5){};
\draw[](tid2) -- (tid7);
\node[circle, scale=0.75, fill] (tid3) at (6.75,3){};
\node[circle, scale=0.75, fill] (tid8) at (6.75,4.5){};
\draw[](tid3) -- (tid8);
\draw[](tid0) -- (tid1);
\draw[](tid0) -- (tid2);
\draw[](tid0) -- (tid3);
\end{tikzpicture}
\nodepart{three}
\footnotesize{5.625}
\nodepart{four}
\footnotesize{$33\:67$}
};
 \\ 
\node[draw=black, rectangle split,  rectangle split parts=4] (sn0x22ada30){
\footnotesize{32.8125}
\nodepart{two}
\begin{tikzpicture}[scale=.2]
\node[circle, scale=0.75, fill] (tid0) at (3.75,1.5){};
\node[circle, scale=0.75, fill] (tid1) at (2.25,3){};
\node[circle, scale=0.75, fill, task_scheduled] (tid4) at (0.75,4.5){};
\node[circle, scale=0.75, fill] (tid5) at (2.25,4.5){};
\node[circle, scale=0.75, fill] (tid6) at (3.75,4.5){};
\draw[](tid1) -- (tid4);
\draw[](tid1) -- (tid5);
\draw[](tid1) -- (tid6);
\node[circle, scale=0.75, fill] (tid2) at (5.25,3){};
\node[circle, scale=0.75, fill, task_scheduled] (tid7) at (5.25,4.5){};
\draw[](tid2) -- (tid7);
\node[circle, scale=0.75, fill] (tid3) at (6.75,3){};
\node[circle, scale=0.75, fill] (tid8) at (6.75,4.5){};
\draw[](tid3) -- (tid8);
\draw[](tid0) -- (tid1);
\draw[](tid0) -- (tid2);
\draw[](tid0) -- (tid3);
\end{tikzpicture}
\nodepart{three}
\footnotesize{5.625}
\nodepart{four}
\footnotesize{$33\:17\:33\:17$}
};
 \\ 
\node[draw=black, rectangle split,  rectangle split parts=4] (sn0x22b4be0){
\footnotesize{5.3125}
\nodepart{two}
\begin{tikzpicture}[scale=.2]
\node[circle, scale=0.75, fill] (tid0) at (3.75,1.5){};
\node[circle, scale=0.75, fill] (tid1) at (2.25,3){};
\node[circle, scale=0.75, fill] (tid4) at (0.75,4.5){};
\node[circle, scale=0.75, fill] (tid5) at (2.25,4.5){};
\node[circle, scale=0.75, fill] (tid6) at (3.75,4.5){};
\draw[](tid1) -- (tid4);
\draw[](tid1) -- (tid5);
\draw[](tid1) -- (tid6);
\node[circle, scale=0.75, fill] (tid2) at (5.25,3){};
\node[circle, scale=0.75, fill, task_scheduled] (tid7) at (5.25,4.5){};
\draw[](tid2) -- (tid7);
\node[circle, scale=0.75, fill] (tid3) at (6.75,3){};
\node[circle, scale=0.75, fill, task_scheduled] (tid8) at (6.75,4.5){};
\draw[](tid3) -- (tid8);
\draw[](tid0) -- (tid1);
\draw[](tid0) -- (tid2);
\draw[](tid0) -- (tid3);
\end{tikzpicture}
\nodepart{three}
\footnotesize{5.625}
\nodepart{four}
\footnotesize{$1$}
};
 \\ 
\node[draw=black, rectangle split,  rectangle split parts=4] (sn0x22ac4a0){
\footnotesize{12.5}
\nodepart{two}
\begin{tikzpicture}[scale=.2]
\node[circle, scale=0.75, fill] (tid0) at (3,1.5){};
\node[circle, scale=0.75, fill] (tid1) at (1.5,3){};
\node[circle, scale=0.75, fill, task_scheduled] (tid4) at (0.75,4.5){};
\node[circle, scale=0.75, fill] (tid5) at (2.25,4.5){};
\draw[](tid1) -- (tid4);
\draw[](tid1) -- (tid5);
\node[circle, scale=0.75, fill] (tid2) at (3.75,3){};
\node[circle, scale=0.75, fill] (tid6) at (3.75,4.5){};
\node[circle, scale=0.75, fill, task_scheduled] (tid8) at (3.75,6){};
\draw[](tid6) -- (tid8);
\draw[](tid2) -- (tid6);
\node[circle, scale=0.75, fill] (tid3) at (5.25,3){};
\node[circle, scale=0.75, fill] (tid7) at (5.25,4.5){};
\draw[](tid3) -- (tid7);
\draw[](tid0) -- (tid1);
\draw[](tid0) -- (tid2);
\draw[](tid0) -- (tid3);
\end{tikzpicture}
\nodepart{three}
\footnotesize{5.65625}
\nodepart{four}
\footnotesize{$50\:17\:33$}
};
 \\ 
\node[draw=black, rectangle split,  rectangle split parts=4] (sn0x22acd50){
\footnotesize{6.25}
\nodepart{two}
\begin{tikzpicture}[scale=.2]
\node[circle, scale=0.75, fill] (tid0) at (3,1.5){};
\node[circle, scale=0.75, fill] (tid1) at (1.5,3){};
\node[circle, scale=0.75, fill] (tid4) at (0.75,4.5){};
\node[circle, scale=0.75, fill] (tid5) at (2.25,4.5){};
\draw[](tid1) -- (tid4);
\draw[](tid1) -- (tid5);
\node[circle, scale=0.75, fill] (tid2) at (3.75,3){};
\node[circle, scale=0.75, fill] (tid6) at (3.75,4.5){};
\node[circle, scale=0.75, fill, task_scheduled] (tid8) at (3.75,6){};
\draw[](tid6) -- (tid8);
\draw[](tid2) -- (tid6);
\node[circle, scale=0.75, fill] (tid3) at (5.25,3){};
\node[circle, scale=0.75, fill, task_scheduled] (tid7) at (5.25,4.5){};
\draw[](tid3) -- (tid7);
\draw[](tid0) -- (tid1);
\draw[](tid0) -- (tid2);
\draw[](tid0) -- (tid3);
\end{tikzpicture}
\nodepart{three}
\footnotesize{5.65625}
\nodepart{four}
\footnotesize{$50\:33\:17$}
};
 \\ 
\node[draw=black, rectangle split,  rectangle split parts=4] (sn0x22b54f0){
\footnotesize{7.5}
\nodepart{two}
\begin{tikzpicture}[scale=.2]
\node[circle, scale=0.75, fill] (tid0) at (3.75,1.5){};
\node[circle, scale=0.75, fill] (tid1) at (1.5,3){};
\node[circle, scale=0.75, fill, task_scheduled] (tid4) at (0.75,4.5){};
\node[circle, scale=0.75, fill] (tid5) at (2.25,4.5){};
\draw[](tid1) -- (tid4);
\draw[](tid1) -- (tid5);
\node[circle, scale=0.75, fill] (tid2) at (4.5,3){};
\node[circle, scale=0.75, fill, task_scheduled] (tid6) at (3.75,4.5){};
\node[circle, scale=0.75, fill] (tid7) at (5.25,4.5){};
\draw[](tid2) -- (tid6);
\draw[](tid2) -- (tid7);
\node[circle, scale=0.75, fill] (tid3) at (6.75,3){};
\node[circle, scale=0.75, fill] (tid8) at (6.75,4.5){};
\draw[](tid3) -- (tid8);
\draw[](tid0) -- (tid1);
\draw[](tid0) -- (tid2);
\draw[](tid0) -- (tid3);
\end{tikzpicture}
\nodepart{three}
\footnotesize{5.625}
\nodepart{four}
\footnotesize{$67\:33$}
};
 \\ 
\node[draw=black, rectangle split,  rectangle split parts=4] (sn0x22b5d10){
\footnotesize{3.75}
\nodepart{two}
\begin{tikzpicture}[scale=.2]
\node[circle, scale=0.75, fill] (tid0) at (3.75,1.5){};
\node[circle, scale=0.75, fill] (tid1) at (1.5,3){};
\node[circle, scale=0.75, fill, task_scheduled] (tid4) at (0.75,4.5){};
\node[circle, scale=0.75, fill, task_scheduled] (tid5) at (2.25,4.5){};
\draw[](tid1) -- (tid4);
\draw[](tid1) -- (tid5);
\node[circle, scale=0.75, fill] (tid2) at (4.5,3){};
\node[circle, scale=0.75, fill] (tid6) at (3.75,4.5){};
\node[circle, scale=0.75, fill] (tid7) at (5.25,4.5){};
\draw[](tid2) -- (tid6);
\draw[](tid2) -- (tid7);
\node[circle, scale=0.75, fill] (tid3) at (6.75,3){};
\node[circle, scale=0.75, fill] (tid8) at (6.75,4.5){};
\draw[](tid3) -- (tid8);
\draw[](tid0) -- (tid1);
\draw[](tid0) -- (tid2);
\draw[](tid0) -- (tid3);
\end{tikzpicture}
\nodepart{three}
\footnotesize{5.625}
\nodepart{four}
\footnotesize{$67\:33$}
};
 \\ 
\node[draw=black, rectangle split,  rectangle split parts=4] (sn0x22b6220){
\footnotesize{3.75}
\nodepart{two}
\begin{tikzpicture}[scale=.2]
\node[circle, scale=0.75, fill] (tid0) at (3.75,1.5){};
\node[circle, scale=0.75, fill] (tid1) at (1.5,3){};
\node[circle, scale=0.75, fill, task_scheduled] (tid4) at (0.75,4.5){};
\node[circle, scale=0.75, fill] (tid5) at (2.25,4.5){};
\draw[](tid1) -- (tid4);
\draw[](tid1) -- (tid5);
\node[circle, scale=0.75, fill] (tid2) at (4.5,3){};
\node[circle, scale=0.75, fill] (tid6) at (3.75,4.5){};
\node[circle, scale=0.75, fill] (tid7) at (5.25,4.5){};
\draw[](tid2) -- (tid6);
\draw[](tid2) -- (tid7);
\node[circle, scale=0.75, fill] (tid3) at (6.75,3){};
\node[circle, scale=0.75, fill, task_scheduled] (tid8) at (6.75,4.5){};
\draw[](tid3) -- (tid8);
\draw[](tid0) -- (tid1);
\draw[](tid0) -- (tid2);
\draw[](tid0) -- (tid3);
\end{tikzpicture}
\nodepart{three}
\footnotesize{5.625}
\nodepart{four}
\footnotesize{$17\:33\:17\:33$}
};
 \\ 
\\
};
\end{scope}
\begin{scope}[yshift=\leveltopIIII cm, anchor = center]
\matrix (line4)[row sep=0.5cm] {
\node[draw=black, rectangle split,  rectangle split parts=4] (sn0x22ad490){
\footnotesize{6.25}
\nodepart{two}
\begin{tikzpicture}[scale=.2]
\node[circle, scale=0.75, fill] (tid0) at (2.25,1.5){};
\node[circle, scale=0.75, fill] (tid1) at (0.75,3){};
\node[circle, scale=0.75, fill] (tid4) at (0.75,4.5){};
\node[circle, scale=0.75, fill, task_scheduled] (tid7) at (0.75,6){};
\draw[](tid4) -- (tid7);
\draw[](tid1) -- (tid4);
\node[circle, scale=0.75, fill] (tid2) at (2.25,3){};
\node[circle, scale=0.75, fill, task_scheduled] (tid5) at (2.25,4.5){};
\draw[](tid2) -- (tid5);
\node[circle, scale=0.75, fill] (tid3) at (3.75,3){};
\node[circle, scale=0.75, fill] (tid6) at (3.75,4.5){};
\draw[](tid3) -- (tid6);
\draw[](tid0) -- (tid1);
\draw[](tid0) -- (tid2);
\draw[](tid0) -- (tid3);
\end{tikzpicture}
\nodepart{three}
\footnotesize{5.1875}
\nodepart{four}
\footnotesize{$50\:50$}
};
 \\ 
\node[draw=black, rectangle split,  rectangle split parts=4] (sn0x22b34f0){
\footnotesize{14.2708}
\nodepart{two}
\begin{tikzpicture}[scale=.2]
\node[circle, scale=0.75, fill] (tid0) at (3.75,1.5){};
\node[circle, scale=0.75, fill] (tid1) at (2.25,3){};
\node[circle, scale=0.75, fill, task_scheduled] (tid4) at (0.75,4.5){};
\node[circle, scale=0.75, fill, task_scheduled] (tid5) at (2.25,4.5){};
\node[circle, scale=0.75, fill] (tid6) at (3.75,4.5){};
\draw[](tid1) -- (tid4);
\draw[](tid1) -- (tid5);
\draw[](tid1) -- (tid6);
\node[circle, scale=0.75, fill] (tid2) at (5.25,3){};
\node[circle, scale=0.75, fill] (tid7) at (5.25,4.5){};
\draw[](tid2) -- (tid7);
\node[circle, scale=0.75, fill] (tid3) at (6.75,3){};
\draw[](tid0) -- (tid1);
\draw[](tid0) -- (tid2);
\draw[](tid0) -- (tid3);
\end{tikzpicture}
\nodepart{three}
\footnotesize{5.125}
\nodepart{four}
\footnotesize{$50\:50$}
};
 \\ 
\node[draw=black, rectangle split,  rectangle split parts=4] (sn0x22b3cf0){
\footnotesize{13.6979}
\nodepart{two}
\begin{tikzpicture}[scale=.2]
\node[circle, scale=0.75, fill] (tid0) at (3.75,1.5){};
\node[circle, scale=0.75, fill] (tid1) at (2.25,3){};
\node[circle, scale=0.75, fill, task_scheduled] (tid4) at (0.75,4.5){};
\node[circle, scale=0.75, fill] (tid5) at (2.25,4.5){};
\node[circle, scale=0.75, fill] (tid6) at (3.75,4.5){};
\draw[](tid1) -- (tid4);
\draw[](tid1) -- (tid5);
\draw[](tid1) -- (tid6);
\node[circle, scale=0.75, fill] (tid2) at (5.25,3){};
\node[circle, scale=0.75, fill, task_scheduled] (tid7) at (5.25,4.5){};
\draw[](tid2) -- (tid7);
\node[circle, scale=0.75, fill] (tid3) at (6.75,3){};
\draw[](tid0) -- (tid1);
\draw[](tid0) -- (tid2);
\draw[](tid0) -- (tid3);
\end{tikzpicture}
\nodepart{three}
\footnotesize{5.125}
\nodepart{four}
\footnotesize{$50\:50$}
};
 \\ 
\node[draw=black, rectangle split,  rectangle split parts=4] (sn0x22b2b90){
\footnotesize{6.25}
\nodepart{two}
\begin{tikzpicture}[scale=.2]
\node[circle, scale=0.75, fill] (tid0) at (3,1.5){};
\node[circle, scale=0.75, fill] (tid1) at (1.5,3){};
\node[circle, scale=0.75, fill, task_scheduled] (tid4) at (0.75,4.5){};
\node[circle, scale=0.75, fill] (tid5) at (2.25,4.5){};
\draw[](tid1) -- (tid4);
\draw[](tid1) -- (tid5);
\node[circle, scale=0.75, fill] (tid2) at (3.75,3){};
\node[circle, scale=0.75, fill] (tid6) at (3.75,4.5){};
\node[circle, scale=0.75, fill, task_scheduled] (tid7) at (3.75,6){};
\draw[](tid6) -- (tid7);
\draw[](tid2) -- (tid6);
\node[circle, scale=0.75, fill] (tid3) at (5.25,3){};
\draw[](tid0) -- (tid1);
\draw[](tid0) -- (tid2);
\draw[](tid0) -- (tid3);
\end{tikzpicture}
\nodepart{three}
\footnotesize{5.1875}
\nodepart{four}
\footnotesize{$50\:25\:25$}
};
 \\ 
\node[draw=black, rectangle split,  rectangle split parts=4] (sn0x22b66d0){
\footnotesize{1.25}
\nodepart{two}
\begin{tikzpicture}[scale=.2]
\node[circle, scale=0.75, fill] (tid0) at (3.75,1.5){};
\node[circle, scale=0.75, fill] (tid1) at (1.5,3){};
\node[circle, scale=0.75, fill, task_scheduled] (tid4) at (0.75,4.5){};
\node[circle, scale=0.75, fill, task_scheduled] (tid5) at (2.25,4.5){};
\draw[](tid1) -- (tid4);
\draw[](tid1) -- (tid5);
\node[circle, scale=0.75, fill] (tid2) at (4.5,3){};
\node[circle, scale=0.75, fill] (tid6) at (3.75,4.5){};
\node[circle, scale=0.75, fill] (tid7) at (5.25,4.5){};
\draw[](tid2) -- (tid6);
\draw[](tid2) -- (tid7);
\node[circle, scale=0.75, fill] (tid3) at (6.75,3){};
\draw[](tid0) -- (tid1);
\draw[](tid0) -- (tid2);
\draw[](tid0) -- (tid3);
\end{tikzpicture}
\nodepart{three}
\footnotesize{5.125}
\nodepart{four}
\footnotesize{$1$}
};
 \\ 
\node[draw=black, rectangle split,  rectangle split parts=4] (sn0x22b6c20){
\footnotesize{2.5}
\nodepart{two}
\begin{tikzpicture}[scale=.2]
\node[circle, scale=0.75, fill] (tid0) at (3.75,1.5){};
\node[circle, scale=0.75, fill] (tid1) at (1.5,3){};
\node[circle, scale=0.75, fill, task_scheduled] (tid4) at (0.75,4.5){};
\node[circle, scale=0.75, fill] (tid5) at (2.25,4.5){};
\draw[](tid1) -- (tid4);
\draw[](tid1) -- (tid5);
\node[circle, scale=0.75, fill] (tid2) at (4.5,3){};
\node[circle, scale=0.75, fill, task_scheduled] (tid6) at (3.75,4.5){};
\node[circle, scale=0.75, fill] (tid7) at (5.25,4.5){};
\draw[](tid2) -- (tid6);
\draw[](tid2) -- (tid7);
\node[circle, scale=0.75, fill] (tid3) at (6.75,3){};
\draw[](tid0) -- (tid1);
\draw[](tid0) -- (tid2);
\draw[](tid0) -- (tid3);
\end{tikzpicture}
\nodepart{three}
\footnotesize{5.125}
\nodepart{four}
\footnotesize{$50\:50$}
};
 \\ 
\node[draw=black, rectangle split,  rectangle split parts=4] (sn0x22ade60){
\footnotesize{10.2083}
\nodepart{two}
\begin{tikzpicture}[scale=.2]
\node[circle, scale=0.75, fill] (tid0) at (3,1.5){};
\node[circle, scale=0.75, fill] (tid1) at (1.5,3){};
\node[circle, scale=0.75, fill, task_scheduled] (tid4) at (0.75,4.5){};
\node[circle, scale=0.75, fill, task_scheduled] (tid5) at (2.25,4.5){};
\draw[](tid1) -- (tid4);
\draw[](tid1) -- (tid5);
\node[circle, scale=0.75, fill] (tid2) at (3.75,3){};
\node[circle, scale=0.75, fill] (tid6) at (3.75,4.5){};
\draw[](tid2) -- (tid6);
\node[circle, scale=0.75, fill] (tid3) at (5.25,3){};
\node[circle, scale=0.75, fill] (tid7) at (5.25,4.5){};
\draw[](tid3) -- (tid7);
\draw[](tid0) -- (tid1);
\draw[](tid0) -- (tid2);
\draw[](tid0) -- (tid3);
\end{tikzpicture}
\nodepart{three}
\footnotesize{5.125}
\nodepart{four}
\footnotesize{$1$}
};
 \\ 
\node[draw=black, rectangle split,  rectangle split parts=4] (sn0x22ae720){
\footnotesize{37.1875}
\nodepart{two}
\begin{tikzpicture}[scale=.2]
\node[circle, scale=0.75, fill] (tid0) at (3,1.5){};
\node[circle, scale=0.75, fill] (tid1) at (1.5,3){};
\node[circle, scale=0.75, fill, task_scheduled] (tid4) at (0.75,4.5){};
\node[circle, scale=0.75, fill] (tid5) at (2.25,4.5){};
\draw[](tid1) -- (tid4);
\draw[](tid1) -- (tid5);
\node[circle, scale=0.75, fill] (tid2) at (3.75,3){};
\node[circle, scale=0.75, fill, task_scheduled] (tid6) at (3.75,4.5){};
\draw[](tid2) -- (tid6);
\node[circle, scale=0.75, fill] (tid3) at (5.25,3){};
\node[circle, scale=0.75, fill] (tid7) at (5.25,4.5){};
\draw[](tid3) -- (tid7);
\draw[](tid0) -- (tid1);
\draw[](tid0) -- (tid2);
\draw[](tid0) -- (tid3);
\end{tikzpicture}
\nodepart{three}
\footnotesize{5.125}
\nodepart{four}
\footnotesize{$25\:25\:50$}
};
 \\ 
\node[draw=black, rectangle split,  rectangle split parts=4] (sn0x22b2d60){
\footnotesize{8.38542}
\nodepart{two}
\begin{tikzpicture}[scale=.2]
\node[circle, scale=0.75, fill] (tid0) at (3,1.5){};
\node[circle, scale=0.75, fill] (tid1) at (1.5,3){};
\node[circle, scale=0.75, fill] (tid4) at (0.75,4.5){};
\node[circle, scale=0.75, fill] (tid5) at (2.25,4.5){};
\draw[](tid1) -- (tid4);
\draw[](tid1) -- (tid5);
\node[circle, scale=0.75, fill] (tid2) at (3.75,3){};
\node[circle, scale=0.75, fill, task_scheduled] (tid6) at (3.75,4.5){};
\draw[](tid2) -- (tid6);
\node[circle, scale=0.75, fill] (tid3) at (5.25,3){};
\node[circle, scale=0.75, fill, task_scheduled] (tid7) at (5.25,4.5){};
\draw[](tid3) -- (tid7);
\draw[](tid0) -- (tid1);
\draw[](tid0) -- (tid2);
\draw[](tid0) -- (tid3);
\end{tikzpicture}
\nodepart{three}
\footnotesize{5.125}
\nodepart{four}
\footnotesize{$1$}
};
 \\ 
\\
};
\end{scope}
\begin{scope}[yshift=\leveltopIIIII cm, anchor = center]
\matrix (line5)[row sep=0.5cm] {
\node[draw=black, rectangle split,  rectangle split parts=4] (sn0x22ae570){
\footnotesize{6.25}
\nodepart{two}
\begin{tikzpicture}[scale=.2]
\node[circle, scale=0.75, fill] (tid0) at (2.25,1.5){};
\node[circle, scale=0.75, fill] (tid1) at (0.75,3){};
\node[circle, scale=0.75, fill] (tid4) at (0.75,4.5){};
\node[circle, scale=0.75, fill, task_scheduled] (tid6) at (0.75,6){};
\draw[](tid4) -- (tid6);
\draw[](tid1) -- (tid4);
\node[circle, scale=0.75, fill] (tid2) at (2.25,3){};
\node[circle, scale=0.75, fill, task_scheduled] (tid5) at (2.25,4.5){};
\draw[](tid2) -- (tid5);
\node[circle, scale=0.75, fill] (tid3) at (3.75,3){};
\draw[](tid0) -- (tid1);
\draw[](tid0) -- (tid2);
\draw[](tid0) -- (tid3);
\end{tikzpicture}
\nodepart{three}
\footnotesize{4.75}
\nodepart{four}
\footnotesize{$50\:50$}
};
 \\ 
\node[draw=black, rectangle split,  rectangle split parts=4] (sn0x22b40d0){
\footnotesize{6.84896}
\nodepart{two}
\begin{tikzpicture}[scale=.2]
\node[circle, scale=0.75, fill] (tid0) at (3.75,1.5){};
\node[circle, scale=0.75, fill] (tid1) at (2.25,3){};
\node[circle, scale=0.75, fill, task_scheduled] (tid4) at (0.75,4.5){};
\node[circle, scale=0.75, fill, task_scheduled] (tid5) at (2.25,4.5){};
\node[circle, scale=0.75, fill] (tid6) at (3.75,4.5){};
\draw[](tid1) -- (tid4);
\draw[](tid1) -- (tid5);
\draw[](tid1) -- (tid6);
\node[circle, scale=0.75, fill] (tid2) at (5.25,3){};
\node[circle, scale=0.75, fill] (tid3) at (6.75,3){};
\draw[](tid0) -- (tid1);
\draw[](tid0) -- (tid2);
\draw[](tid0) -- (tid3);
\end{tikzpicture}
\nodepart{three}
\footnotesize{4.625}
\nodepart{four}
\footnotesize{$1$}
};
 \\ 
\node[draw=black, rectangle split,  rectangle split parts=4] (sn0x22b14e0){
\footnotesize{19.2448}
\nodepart{two}
\begin{tikzpicture}[scale=.2]
\node[circle, scale=0.75, fill] (tid0) at (3,1.5){};
\node[circle, scale=0.75, fill] (tid1) at (1.5,3){};
\node[circle, scale=0.75, fill, task_scheduled] (tid4) at (0.75,4.5){};
\node[circle, scale=0.75, fill, task_scheduled] (tid5) at (2.25,4.5){};
\draw[](tid1) -- (tid4);
\draw[](tid1) -- (tid5);
\node[circle, scale=0.75, fill] (tid2) at (3.75,3){};
\node[circle, scale=0.75, fill] (tid6) at (3.75,4.5){};
\draw[](tid2) -- (tid6);
\node[circle, scale=0.75, fill] (tid3) at (5.25,3){};
\draw[](tid0) -- (tid1);
\draw[](tid0) -- (tid2);
\draw[](tid0) -- (tid3);
\end{tikzpicture}
\nodepart{three}
\footnotesize{4.625}
\nodepart{four}
\footnotesize{$1$}
};
 \\ 
\node[draw=black, rectangle split,  rectangle split parts=4] (sn0x22b1ca0){
\footnotesize{35.7292}
\nodepart{two}
\begin{tikzpicture}[scale=.2]
\node[circle, scale=0.75, fill] (tid0) at (3,1.5){};
\node[circle, scale=0.75, fill] (tid1) at (1.5,3){};
\node[circle, scale=0.75, fill, task_scheduled] (tid4) at (0.75,4.5){};
\node[circle, scale=0.75, fill] (tid5) at (2.25,4.5){};
\draw[](tid1) -- (tid4);
\draw[](tid1) -- (tid5);
\node[circle, scale=0.75, fill] (tid2) at (3.75,3){};
\node[circle, scale=0.75, fill, task_scheduled] (tid6) at (3.75,4.5){};
\draw[](tid2) -- (tid6);
\node[circle, scale=0.75, fill] (tid3) at (5.25,3){};
\draw[](tid0) -- (tid1);
\draw[](tid0) -- (tid2);
\draw[](tid0) -- (tid3);
\end{tikzpicture}
\nodepart{three}
\footnotesize{4.625}
\nodepart{four}
\footnotesize{$50\:50$}
};
 \\ 
\node[draw=black, rectangle split,  rectangle split parts=4] (sn0x22af010){
\footnotesize{31.9271}
\nodepart{two}
\begin{tikzpicture}[scale=.2]
\node[circle, scale=0.75, fill] (tid0) at (2.25,1.5){};
\node[circle, scale=0.75, fill] (tid1) at (0.75,3){};
\node[circle, scale=0.75, fill, task_scheduled] (tid4) at (0.75,4.5){};
\draw[](tid1) -- (tid4);
\node[circle, scale=0.75, fill] (tid2) at (2.25,3){};
\node[circle, scale=0.75, fill, task_scheduled] (tid5) at (2.25,4.5){};
\draw[](tid2) -- (tid5);
\node[circle, scale=0.75, fill] (tid3) at (3.75,3){};
\node[circle, scale=0.75, fill] (tid6) at (3.75,4.5){};
\draw[](tid3) -- (tid6);
\draw[](tid0) -- (tid1);
\draw[](tid0) -- (tid2);
\draw[](tid0) -- (tid3);
\end{tikzpicture}
\nodepart{three}
\footnotesize{4.625}
\nodepart{four}
\footnotesize{$1$}
};
 \\ 
\\
};
\end{scope}
\begin{scope}[yshift=\leveltopIIIIII cm, anchor = center]
\matrix (line6)[row sep=0.5cm] {
\node[draw=black, rectangle split,  rectangle split parts=4] (sn0x22aecc0){
\footnotesize{3.125}
\nodepart{two}
\begin{tikzpicture}[scale=.2]
\node[circle, scale=0.75, fill] (tid0) at (2.25,1.5){};
\node[circle, scale=0.75, fill] (tid1) at (0.75,3){};
\node[circle, scale=0.75, fill] (tid4) at (0.75,4.5){};
\node[circle, scale=0.75, fill, task_scheduled] (tid5) at (0.75,6){};
\draw[](tid4) -- (tid5);
\draw[](tid1) -- (tid4);
\node[circle, scale=0.75, fill, task_scheduled] (tid2) at (2.25,3){};
\node[circle, scale=0.75, fill] (tid3) at (3.75,3){};
\draw[](tid0) -- (tid1);
\draw[](tid0) -- (tid2);
\draw[](tid0) -- (tid3);
\end{tikzpicture}
\nodepart{three}
\footnotesize{4.375}
\nodepart{four}
\footnotesize{$50\:50$}
};
 \\ 
\node[draw=black, rectangle split,  rectangle split parts=4] (sn0x22b21c0){
\footnotesize{24.7135}
\nodepart{two}
\begin{tikzpicture}[scale=.2]
\node[circle, scale=0.75, fill] (tid0) at (3,1.5){};
\node[circle, scale=0.75, fill] (tid1) at (1.5,3){};
\node[circle, scale=0.75, fill, task_scheduled] (tid4) at (0.75,4.5){};
\node[circle, scale=0.75, fill, task_scheduled] (tid5) at (2.25,4.5){};
\draw[](tid1) -- (tid4);
\draw[](tid1) -- (tid5);
\node[circle, scale=0.75, fill] (tid2) at (3.75,3){};
\node[circle, scale=0.75, fill] (tid3) at (5.25,3){};
\draw[](tid0) -- (tid1);
\draw[](tid0) -- (tid2);
\draw[](tid0) -- (tid3);
\end{tikzpicture}
\nodepart{three}
\footnotesize{4.125}
\nodepart{four}
\footnotesize{$1$}
};
 \\ 
\node[draw=black, rectangle split,  rectangle split parts=4] (sn0x22af160){
\footnotesize{72.1615}
\nodepart{two}
\begin{tikzpicture}[scale=.2]
\node[circle, scale=0.75, fill] (tid0) at (2.25,1.5){};
\node[circle, scale=0.75, fill] (tid1) at (0.75,3){};
\node[circle, scale=0.75, fill, task_scheduled] (tid4) at (0.75,4.5){};
\draw[](tid1) -- (tid4);
\node[circle, scale=0.75, fill] (tid2) at (2.25,3){};
\node[circle, scale=0.75, fill, task_scheduled] (tid5) at (2.25,4.5){};
\draw[](tid2) -- (tid5);
\node[circle, scale=0.75, fill] (tid3) at (3.75,3){};
\draw[](tid0) -- (tid1);
\draw[](tid0) -- (tid2);
\draw[](tid0) -- (tid3);
\end{tikzpicture}
\nodepart{three}
\footnotesize{4.125}
\nodepart{four}
\footnotesize{$1$}
};
 \\ 
\\
};
\end{scope}
\begin{scope}[yshift=\leveltopIIIIIII cm, anchor = center]
\matrix (line7)[row sep=0.5cm] {
\node[draw=black, rectangle split,  rectangle split parts=4] (sn0x22af6c0){
\footnotesize{1.5625}
\nodepart{two}
\begin{tikzpicture}[scale=.2]
\node[circle, scale=0.75, fill] (tid0) at (1.5,1.5){};
\node[circle, scale=0.75, fill] (tid1) at (0.75,3){};
\node[circle, scale=0.75, fill] (tid3) at (0.75,4.5){};
\node[circle, scale=0.75, fill, task_scheduled] (tid4) at (0.75,6){};
\draw[](tid3) -- (tid4);
\draw[](tid1) -- (tid3);
\node[circle, scale=0.75, fill, task_scheduled] (tid2) at (2.25,3){};
\draw[](tid0) -- (tid1);
\draw[](tid0) -- (tid2);
\end{tikzpicture}
\nodepart{three}
\footnotesize{4.125}
\nodepart{four}
\footnotesize{$50\:50$}
};
 \\ 
\node[draw=black, rectangle split,  rectangle split parts=4] (sn0x22afb50){
\footnotesize{98.4375}
\nodepart{two}
\begin{tikzpicture}[scale=.2]
\node[circle, scale=0.75, fill] (tid0) at (2.25,1.5){};
\node[circle, scale=0.75, fill] (tid1) at (0.75,3){};
\node[circle, scale=0.75, fill, task_scheduled] (tid4) at (0.75,4.5){};
\draw[](tid1) -- (tid4);
\node[circle, scale=0.75, fill, task_scheduled] (tid2) at (2.25,3){};
\node[circle, scale=0.75, fill] (tid3) at (3.75,3){};
\draw[](tid0) -- (tid1);
\draw[](tid0) -- (tid2);
\draw[](tid0) -- (tid3);
\end{tikzpicture}
\nodepart{three}
\footnotesize{3.625}
\nodepart{four}
\footnotesize{$50\:50$}
};
 \\ 
\\
};
\end{scope}
\draw (sn0x22a7ba0.east) -- (sn0x22ab160.west);
\draw (sn0x22a7ba0.east) -- (sn0x22aaaa0.west);
\draw (sn0x22a7ba0.east) -- (sn0x22ab2c0.west);
\draw (sn0x22a7ba0.east) -- (sn0x22aba90.west);
\draw (sn0x22a7ba0.east) -- (sn0x22ab500.west);
\draw (sn0x22ab160.east) -- (sn0x22ac4a0.west);
\draw (sn0x22ab160.east) -- (sn0x22acd50.west);
\draw (sn0x22ab160.east) -- (sn0x22ac710.west);
\draw (sn0x22ab160.east) -- (sn0x22ada30.west);
\draw (sn0x22aaaa0.east) -- (sn0x22b4a40.west);
\draw (sn0x22aaaa0.east) -- (sn0x22ada30.west);
\draw (sn0x22aaaa0.east) -- (sn0x22b4be0.west);
\draw (sn0x22ab2c0.east) -- (sn0x22b54f0.west);
\draw (sn0x22ab2c0.east) -- (sn0x22b5d10.west);
\draw (sn0x22ab2c0.east) -- (sn0x22b6220.west);
\draw (sn0x22ab2c0.east) -- (sn0x22ac710.west);
\draw (sn0x22ab2c0.east) -- (sn0x22ada30.west);
\draw (sn0x22aba90.east) -- (sn0x22ada30.west);
\draw (sn0x22aba90.east) -- (sn0x22b4be0.west);
\draw (sn0x22ab500.east) -- (sn0x22ada30.west);
\draw (sn0x22ab500.east) -- (sn0x22b4be0.west);
\draw (sn0x22ab500.east) -- (sn0x22b7500.west);
\draw (sn0x22ab500.east) -- (sn0x22b7aa0.west);
\draw (sn0x22b4a40.east) -- (sn0x22b2b90.west);
\draw (sn0x22b4a40.east) -- (sn0x22b34f0.west);
\draw (sn0x22b4a40.east) -- (sn0x22b3cf0.west);
\draw (sn0x22b7500.east) -- (sn0x22b6c20.west);
\draw (sn0x22b7500.east) -- (sn0x22b66d0.west);
\draw (sn0x22b7500.east) -- (sn0x22b34f0.west);
\draw (sn0x22b7500.east) -- (sn0x22b3cf0.west);
\draw (sn0x22b7aa0.east) -- (sn0x22b3cf0.west);
\draw (sn0x22ac710.east) -- (sn0x22ade60.west);
\draw (sn0x22ac710.east) -- (sn0x22ae720.west);
\draw (sn0x22ada30.east) -- (sn0x22ae720.west);
\draw (sn0x22ada30.east) -- (sn0x22b2d60.west);
\draw (sn0x22ada30.east) -- (sn0x22b34f0.west);
\draw (sn0x22ada30.east) -- (sn0x22b3cf0.west);
\draw (sn0x22b4be0.east) -- (sn0x22b3cf0.west);
\draw (sn0x22ac4a0.east) -- (sn0x22ad490.west);
\draw (sn0x22ac4a0.east) -- (sn0x22ade60.west);
\draw (sn0x22ac4a0.east) -- (sn0x22ae720.west);
\draw (sn0x22acd50.east) -- (sn0x22b2b90.west);
\draw (sn0x22acd50.east) -- (sn0x22ae720.west);
\draw (sn0x22acd50.east) -- (sn0x22b2d60.west);
\draw (sn0x22b54f0.east) -- (sn0x22ae720.west);
\draw (sn0x22b54f0.east) -- (sn0x22ade60.west);
\draw (sn0x22b5d10.east) -- (sn0x22ae720.west);
\draw (sn0x22b5d10.east) -- (sn0x22b2d60.west);
\draw (sn0x22b6220.east) -- (sn0x22b2d60.west);
\draw (sn0x22b6220.east) -- (sn0x22ae720.west);
\draw (sn0x22b6220.east) -- (sn0x22b66d0.west);
\draw (sn0x22b6220.east) -- (sn0x22b6c20.west);
\draw (sn0x22ad490.east) -- (sn0x22ae570.west);
\draw (sn0x22ad490.east) -- (sn0x22af010.west);
\draw (sn0x22b34f0.east) -- (sn0x22b14e0.west);
\draw (sn0x22b34f0.east) -- (sn0x22b1ca0.west);
\draw (sn0x22b3cf0.east) -- (sn0x22b1ca0.west);
\draw (sn0x22b3cf0.east) -- (sn0x22b40d0.west);
\draw (sn0x22b2b90.east) -- (sn0x22ae570.west);
\draw (sn0x22b2b90.east) -- (sn0x22b14e0.west);
\draw (sn0x22b2b90.east) -- (sn0x22b1ca0.west);
\draw (sn0x22b66d0.east) -- (sn0x22b1ca0.west);
\draw (sn0x22b6c20.east) -- (sn0x22b1ca0.west);
\draw (sn0x22b6c20.east) -- (sn0x22b14e0.west);
\draw (sn0x22ade60.east) -- (sn0x22af010.west);
\draw (sn0x22ae720.east) -- (sn0x22af010.west);
\draw (sn0x22ae720.east) -- (sn0x22b14e0.west);
\draw (sn0x22ae720.east) -- (sn0x22b1ca0.west);
\draw (sn0x22b2d60.east) -- (sn0x22b1ca0.west);
\draw (sn0x22ae570.east) -- (sn0x22aecc0.west);
\draw (sn0x22ae570.east) -- (sn0x22af160.west);
\draw (sn0x22b40d0.east) -- (sn0x22b21c0.west);
\draw (sn0x22b14e0.east) -- (sn0x22af160.west);
\draw (sn0x22b1ca0.east) -- (sn0x22af160.west);
\draw (sn0x22b1ca0.east) -- (sn0x22b21c0.west);
\draw (sn0x22af010.east) -- (sn0x22af160.west);
\draw (sn0x22aecc0.east) -- (sn0x22af6c0.west);
\draw (sn0x22aecc0.east) -- (sn0x22afb50.west);
\draw (sn0x22b21c0.east) -- (sn0x22afb50.west);
\draw (sn0x22af160.east) -- (sn0x22afb50.west);
\end{tikzpicture}

%%% Local Variables:
%%% TeX-master: "thesis/thesis.tex"
%%% End: 
\renewcommand{\leveltopI}{-10cm + \leveltop}
\renewcommand{\leveltopII}{-10cm + \leveltopI}
\renewcommand{\leveltopIII}{-10cm + \leveltopII}
\renewcommand{\leveltopIIII}{-10cm + \leveltopIII}
\renewcommand{\leveltopIIIII}{-10cm + \leveltopIIII}
\renewcommand{\leveltopIIIIII}{-10cm + \leveltopIIIII}
\renewcommand{\leveltopIIIIIII}{-10cm + \leveltopIIIIII}
\renewcommand{\leveltopIIIIIIII}{-10cm + \leveltopIIIIIII}
\renewcommand{\leveltopIIIIIIIII}{-10cm + \leveltopIIIIIIII}
\renewcommand{\leveltopIIIIIIIIII}{-10cm + \leveltopIIIIIIIII}
\renewcommand{\leveltopIIIIIIIIIII}{-10cm + \leveltopIIIIIIIIII}
\begin{tikzpicture}[scale=.2, anchor=south, rotate=90]
\begin{scope}[yshift=\leveltopI cm, anchor = center]
\matrix (line1)[row sep=0.5cm] {
\node[draw=black, rectangle split,  rectangle split parts=4] (sn0x22a97d0){
\footnotesize{100}
\nodepart{two}
\begin{tikzpicture}[scale=.2]
\node[circle, scale=0.75, fill] (tid0) at (4.5,1.5){};
\node[circle, scale=0.75, fill] (tid1) at (2.25,3){};
\node[circle, scale=0.75, fill, task_scheduled] (tid4) at (0.75,4.5){};
\node[circle, scale=0.75, fill] (tid5) at (2.25,4.5){};
\node[circle, scale=0.75, fill] (tid6) at (3.75,4.5){};
\draw[](tid1) -- (tid4);
\draw[](tid1) -- (tid5);
\draw[](tid1) -- (tid6);
\node[circle, scale=0.75, fill] (tid2) at (6,3){};
\node[circle, scale=0.75, fill] (tid7) at (5.25,4.5){};
\node[circle, scale=0.75, fill, task_scheduled] (tid10) at (5.25,6){};
\draw[](tid7) -- (tid10);
\node[circle, scale=0.75, fill] (tid8) at (6.75,4.5){};
\draw[](tid2) -- (tid7);
\draw[](tid2) -- (tid8);
\node[circle, scale=0.75, fill] (tid3) at (8.25,3){};
\node[circle, scale=0.75, fill] (tid9) at (8.25,4.5){};
\draw[](tid3) -- (tid9);
\draw[](tid0) -- (tid1);
\draw[](tid0) -- (tid2);
\draw[](tid0) -- (tid3);
\end{tikzpicture}
\nodepart{three}
\footnotesize{6.63281}
\nodepart{four}
\footnotesize{$25\:12\:12\:20\:20\:10$}
};
 \\ 
\\
};
\end{scope}
\begin{scope}[yshift=\leveltopII cm, anchor = center]
\matrix (line2)[row sep=0.5cm] {
\node[draw=black, rectangle split,  rectangle split parts=4] (sn0x22b7700){
\footnotesize{25}
\nodepart{two}
\begin{tikzpicture}[scale=.2]
\node[circle, scale=0.75, fill] (tid0) at (3.75,1.5){};
\node[circle, scale=0.75, fill] (tid1) at (1.5,3){};
\node[circle, scale=0.75, fill] (tid4) at (0.75,4.5){};
\node[circle, scale=0.75, fill, task_scheduled] (tid9) at (0.75,6){};
\draw[](tid4) -- (tid9);
\node[circle, scale=0.75, fill] (tid5) at (2.25,4.5){};
\draw[](tid1) -- (tid4);
\draw[](tid1) -- (tid5);
\node[circle, scale=0.75, fill] (tid2) at (4.5,3){};
\node[circle, scale=0.75, fill, task_scheduled] (tid6) at (3.75,4.5){};
\node[circle, scale=0.75, fill] (tid7) at (5.25,4.5){};
\draw[](tid2) -- (tid6);
\draw[](tid2) -- (tid7);
\node[circle, scale=0.75, fill] (tid3) at (6.75,3){};
\node[circle, scale=0.75, fill] (tid8) at (6.75,4.5){};
\draw[](tid3) -- (tid8);
\draw[](tid0) -- (tid1);
\draw[](tid0) -- (tid2);
\draw[](tid0) -- (tid3);
\end{tikzpicture}
\nodepart{three}
\footnotesize{6.14062}
\nodepart{four}
\footnotesize{$17\:33\:25\:12\:12$}
};
 \\ 
\node[draw=black, rectangle split,  rectangle split parts=4] (sn0x22b8ae0){
\footnotesize{12.5}
\nodepart{two}
\begin{tikzpicture}[scale=.2]
\node[circle, scale=0.75, fill] (tid0) at (3.75,1.5){};
\node[circle, scale=0.75, fill] (tid1) at (1.5,3){};
\node[circle, scale=0.75, fill] (tid4) at (0.75,4.5){};
\node[circle, scale=0.75, fill, task_scheduled] (tid9) at (0.75,6){};
\draw[](tid4) -- (tid9);
\node[circle, scale=0.75, fill, task_scheduled] (tid5) at (2.25,4.5){};
\draw[](tid1) -- (tid4);
\draw[](tid1) -- (tid5);
\node[circle, scale=0.75, fill] (tid2) at (4.5,3){};
\node[circle, scale=0.75, fill] (tid6) at (3.75,4.5){};
\node[circle, scale=0.75, fill] (tid7) at (5.25,4.5){};
\draw[](tid2) -- (tid6);
\draw[](tid2) -- (tid7);
\node[circle, scale=0.75, fill] (tid3) at (6.75,3){};
\node[circle, scale=0.75, fill] (tid8) at (6.75,4.5){};
\draw[](tid3) -- (tid8);
\draw[](tid0) -- (tid1);
\draw[](tid0) -- (tid2);
\draw[](tid0) -- (tid3);
\end{tikzpicture}
\nodepart{three}
\footnotesize{6.14062}
\nodepart{four}
\footnotesize{$33\:17\:12\:25\:12$}
};
 \\ 
\node[draw=black, rectangle split,  rectangle split parts=4] (sn0x22b8c20){
\footnotesize{12.5}
\nodepart{two}
\begin{tikzpicture}[scale=.2]
\node[circle, scale=0.75, fill] (tid0) at (3.75,1.5){};
\node[circle, scale=0.75, fill] (tid1) at (1.5,3){};
\node[circle, scale=0.75, fill] (tid4) at (0.75,4.5){};
\node[circle, scale=0.75, fill, task_scheduled] (tid9) at (0.75,6){};
\draw[](tid4) -- (tid9);
\node[circle, scale=0.75, fill] (tid5) at (2.25,4.5){};
\draw[](tid1) -- (tid4);
\draw[](tid1) -- (tid5);
\node[circle, scale=0.75, fill] (tid2) at (4.5,3){};
\node[circle, scale=0.75, fill] (tid6) at (3.75,4.5){};
\node[circle, scale=0.75, fill] (tid7) at (5.25,4.5){};
\draw[](tid2) -- (tid6);
\draw[](tid2) -- (tid7);
\node[circle, scale=0.75, fill] (tid3) at (6.75,3){};
\node[circle, scale=0.75, fill, task_scheduled] (tid8) at (6.75,4.5){};
\draw[](tid3) -- (tid8);
\draw[](tid0) -- (tid1);
\draw[](tid0) -- (tid2);
\draw[](tid0) -- (tid3);
\end{tikzpicture}
\nodepart{three}
\footnotesize{6.14062}
\nodepart{four}
\footnotesize{$17\:33\:50$}
};
 \\ 
\node[draw=black, rectangle split,  rectangle split parts=4] (sn0x22b8400){
\footnotesize{20}
\nodepart{two}
\begin{tikzpicture}[scale=.2]
\node[circle, scale=0.75, fill] (tid0) at (4.5,1.5){};
\node[circle, scale=0.75, fill] (tid1) at (2.25,3){};
\node[circle, scale=0.75, fill, task_scheduled] (tid4) at (0.75,4.5){};
\node[circle, scale=0.75, fill, task_scheduled] (tid5) at (2.25,4.5){};
\node[circle, scale=0.75, fill] (tid6) at (3.75,4.5){};
\draw[](tid1) -- (tid4);
\draw[](tid1) -- (tid5);
\draw[](tid1) -- (tid6);
\node[circle, scale=0.75, fill] (tid2) at (6,3){};
\node[circle, scale=0.75, fill] (tid7) at (5.25,4.5){};
\node[circle, scale=0.75, fill] (tid8) at (6.75,4.5){};
\draw[](tid2) -- (tid7);
\draw[](tid2) -- (tid8);
\node[circle, scale=0.75, fill] (tid3) at (8.25,3){};
\node[circle, scale=0.75, fill] (tid9) at (8.25,4.5){};
\draw[](tid3) -- (tid9);
\draw[](tid0) -- (tid1);
\draw[](tid0) -- (tid2);
\draw[](tid0) -- (tid3);
\end{tikzpicture}
\nodepart{three}
\footnotesize{6.125}
\nodepart{four}
\footnotesize{$25\:50\:25$}
};
 \\ 
\node[draw=black, rectangle split,  rectangle split parts=4] (sn0x22ab2c0){
\footnotesize{20}
\nodepart{two}
\begin{tikzpicture}[scale=.2]
\node[circle, scale=0.75, fill] (tid0) at (4.5,1.5){};
\node[circle, scale=0.75, fill] (tid1) at (2.25,3){};
\node[circle, scale=0.75, fill, task_scheduled] (tid4) at (0.75,4.5){};
\node[circle, scale=0.75, fill] (tid5) at (2.25,4.5){};
\node[circle, scale=0.75, fill] (tid6) at (3.75,4.5){};
\draw[](tid1) -- (tid4);
\draw[](tid1) -- (tid5);
\draw[](tid1) -- (tid6);
\node[circle, scale=0.75, fill] (tid2) at (6,3){};
\node[circle, scale=0.75, fill, task_scheduled] (tid7) at (5.25,4.5){};
\node[circle, scale=0.75, fill] (tid8) at (6.75,4.5){};
\draw[](tid2) -- (tid7);
\draw[](tid2) -- (tid8);
\node[circle, scale=0.75, fill] (tid3) at (8.25,3){};
\node[circle, scale=0.75, fill] (tid9) at (8.25,4.5){};
\draw[](tid3) -- (tid9);
\draw[](tid0) -- (tid1);
\draw[](tid0) -- (tid2);
\draw[](tid0) -- (tid3);
\end{tikzpicture}
\nodepart{three}
\footnotesize{6.125}
\nodepart{four}
\footnotesize{$25\:25\:25\:12\:12$}
};
 \\ 
\node[draw=black, rectangle split,  rectangle split parts=4] (sn0x22b9170){
\footnotesize{10}
\nodepart{two}
\begin{tikzpicture}[scale=.2]
\node[circle, scale=0.75, fill] (tid0) at (4.5,1.5){};
\node[circle, scale=0.75, fill] (tid1) at (2.25,3){};
\node[circle, scale=0.75, fill, task_scheduled] (tid4) at (0.75,4.5){};
\node[circle, scale=0.75, fill] (tid5) at (2.25,4.5){};
\node[circle, scale=0.75, fill] (tid6) at (3.75,4.5){};
\draw[](tid1) -- (tid4);
\draw[](tid1) -- (tid5);
\draw[](tid1) -- (tid6);
\node[circle, scale=0.75, fill] (tid2) at (6,3){};
\node[circle, scale=0.75, fill] (tid7) at (5.25,4.5){};
\node[circle, scale=0.75, fill] (tid8) at (6.75,4.5){};
\draw[](tid2) -- (tid7);
\draw[](tid2) -- (tid8);
\node[circle, scale=0.75, fill] (tid3) at (8.25,3){};
\node[circle, scale=0.75, fill, task_scheduled] (tid9) at (8.25,4.5){};
\draw[](tid3) -- (tid9);
\draw[](tid0) -- (tid1);
\draw[](tid0) -- (tid2);
\draw[](tid0) -- (tid3);
\end{tikzpicture}
\nodepart{three}
\footnotesize{6.125}
\nodepart{four}
\footnotesize{$25\:25\:50$}
};
 \\ 
\\
};
\end{scope}
\begin{scope}[yshift=\leveltopIII cm, anchor = center]
\matrix (line3)[row sep=0.5cm] {
\node[draw=black, rectangle split,  rectangle split parts=4] (sn0x22bb880){
\footnotesize{2.08333}
\nodepart{two}
\begin{tikzpicture}[scale=.2]
\node[circle, scale=0.75, fill] (tid0) at (3.75,1.5){};
\node[circle, scale=0.75, fill] (tid1) at (1.5,3){};
\node[circle, scale=0.75, fill] (tid4) at (0.75,4.5){};
\node[circle, scale=0.75, fill, task_scheduled] (tid8) at (0.75,6){};
\draw[](tid4) -- (tid8);
\node[circle, scale=0.75, fill, task_scheduled] (tid5) at (2.25,4.5){};
\draw[](tid1) -- (tid4);
\draw[](tid1) -- (tid5);
\node[circle, scale=0.75, fill] (tid2) at (4.5,3){};
\node[circle, scale=0.75, fill] (tid6) at (3.75,4.5){};
\node[circle, scale=0.75, fill] (tid7) at (5.25,4.5){};
\draw[](tid2) -- (tid6);
\draw[](tid2) -- (tid7);
\node[circle, scale=0.75, fill] (tid3) at (6.75,3){};
\draw[](tid0) -- (tid1);
\draw[](tid0) -- (tid2);
\draw[](tid0) -- (tid3);
\end{tikzpicture}
\nodepart{three}
\footnotesize{5.65625}
\nodepart{four}
\footnotesize{$50\:17\:33$}
};
 \\ 
\node[draw=black, rectangle split,  rectangle split parts=4] (sn0x22bc030){
\footnotesize{4.16667}
\nodepart{two}
\begin{tikzpicture}[scale=.2]
\node[circle, scale=0.75, fill] (tid0) at (3.75,1.5){};
\node[circle, scale=0.75, fill] (tid1) at (1.5,3){};
\node[circle, scale=0.75, fill] (tid4) at (0.75,4.5){};
\node[circle, scale=0.75, fill, task_scheduled] (tid8) at (0.75,6){};
\draw[](tid4) -- (tid8);
\node[circle, scale=0.75, fill] (tid5) at (2.25,4.5){};
\draw[](tid1) -- (tid4);
\draw[](tid1) -- (tid5);
\node[circle, scale=0.75, fill] (tid2) at (4.5,3){};
\node[circle, scale=0.75, fill, task_scheduled] (tid6) at (3.75,4.5){};
\node[circle, scale=0.75, fill] (tid7) at (5.25,4.5){};
\draw[](tid2) -- (tid6);
\draw[](tid2) -- (tid7);
\node[circle, scale=0.75, fill] (tid3) at (6.75,3){};
\draw[](tid0) -- (tid1);
\draw[](tid0) -- (tid2);
\draw[](tid0) -- (tid3);
\end{tikzpicture}
\nodepart{three}
\footnotesize{5.65625}
\nodepart{four}
\footnotesize{$25\:25\:33\:17$}
};
 \\ 
\node[draw=black, rectangle split,  rectangle split parts=4] (sn0x22b9da0){
\footnotesize{4.16667}
\nodepart{two}
\begin{tikzpicture}[scale=.2]
\node[circle, scale=0.75, fill] (tid0) at (3,1.5){};
\node[circle, scale=0.75, fill] (tid1) at (1.5,3){};
\node[circle, scale=0.75, fill] (tid4) at (0.75,4.5){};
\node[circle, scale=0.75, fill, task_scheduled] (tid8) at (0.75,6){};
\draw[](tid4) -- (tid8);
\node[circle, scale=0.75, fill, task_scheduled] (tid5) at (2.25,4.5){};
\draw[](tid1) -- (tid4);
\draw[](tid1) -- (tid5);
\node[circle, scale=0.75, fill] (tid2) at (3.75,3){};
\node[circle, scale=0.75, fill] (tid6) at (3.75,4.5){};
\draw[](tid2) -- (tid6);
\node[circle, scale=0.75, fill] (tid3) at (5.25,3){};
\node[circle, scale=0.75, fill] (tid7) at (5.25,4.5){};
\draw[](tid3) -- (tid7);
\draw[](tid0) -- (tid1);
\draw[](tid0) -- (tid2);
\draw[](tid0) -- (tid3);
\end{tikzpicture}
\nodepart{three}
\footnotesize{5.65625}
\nodepart{four}
\footnotesize{$50\:17\:33$}
};
 \\ 
\node[draw=black, rectangle split,  rectangle split parts=4] (sn0x22ba8b0){
\footnotesize{8.33333}
\nodepart{two}
\begin{tikzpicture}[scale=.2]
\node[circle, scale=0.75, fill] (tid0) at (3,1.5){};
\node[circle, scale=0.75, fill] (tid1) at (1.5,3){};
\node[circle, scale=0.75, fill] (tid4) at (0.75,4.5){};
\node[circle, scale=0.75, fill, task_scheduled] (tid8) at (0.75,6){};
\draw[](tid4) -- (tid8);
\node[circle, scale=0.75, fill] (tid5) at (2.25,4.5){};
\draw[](tid1) -- (tid4);
\draw[](tid1) -- (tid5);
\node[circle, scale=0.75, fill] (tid2) at (3.75,3){};
\node[circle, scale=0.75, fill, task_scheduled] (tid6) at (3.75,4.5){};
\draw[](tid2) -- (tid6);
\node[circle, scale=0.75, fill] (tid3) at (5.25,3){};
\node[circle, scale=0.75, fill] (tid7) at (5.25,4.5){};
\draw[](tid3) -- (tid7);
\draw[](tid0) -- (tid1);
\draw[](tid0) -- (tid2);
\draw[](tid0) -- (tid3);
\end{tikzpicture}
\nodepart{three}
\footnotesize{5.65625}
\nodepart{four}
\footnotesize{$25\:25\:33\:17$}
};
 \\ 
\node[draw=black, rectangle split,  rectangle split parts=4] (sn0x22bc5d0){
\footnotesize{2.5}
\nodepart{two}
\begin{tikzpicture}[scale=.2]
\node[circle, scale=0.75, fill] (tid0) at (4.5,1.5){};
\node[circle, scale=0.75, fill] (tid1) at (2.25,3){};
\node[circle, scale=0.75, fill, task_scheduled] (tid4) at (0.75,4.5){};
\node[circle, scale=0.75, fill, task_scheduled] (tid5) at (2.25,4.5){};
\node[circle, scale=0.75, fill] (tid6) at (3.75,4.5){};
\draw[](tid1) -- (tid4);
\draw[](tid1) -- (tid5);
\draw[](tid1) -- (tid6);
\node[circle, scale=0.75, fill] (tid2) at (6,3){};
\node[circle, scale=0.75, fill] (tid7) at (5.25,4.5){};
\node[circle, scale=0.75, fill] (tid8) at (6.75,4.5){};
\draw[](tid2) -- (tid7);
\draw[](tid2) -- (tid8);
\node[circle, scale=0.75, fill] (tid3) at (8.25,3){};
\draw[](tid0) -- (tid1);
\draw[](tid0) -- (tid2);
\draw[](tid0) -- (tid3);
\end{tikzpicture}
\nodepart{three}
\footnotesize{5.625}
\nodepart{four}
\footnotesize{$33\:67$}
};
 \\ 
\node[draw=black, rectangle split,  rectangle split parts=4] (sn0x22b7500){
\footnotesize{2.5}
\nodepart{two}
\begin{tikzpicture}[scale=.2]
\node[circle, scale=0.75, fill] (tid0) at (4.5,1.5){};
\node[circle, scale=0.75, fill] (tid1) at (2.25,3){};
\node[circle, scale=0.75, fill, task_scheduled] (tid4) at (0.75,4.5){};
\node[circle, scale=0.75, fill] (tid5) at (2.25,4.5){};
\node[circle, scale=0.75, fill] (tid6) at (3.75,4.5){};
\draw[](tid1) -- (tid4);
\draw[](tid1) -- (tid5);
\draw[](tid1) -- (tid6);
\node[circle, scale=0.75, fill] (tid2) at (6,3){};
\node[circle, scale=0.75, fill, task_scheduled] (tid7) at (5.25,4.5){};
\node[circle, scale=0.75, fill] (tid8) at (6.75,4.5){};
\draw[](tid2) -- (tid7);
\draw[](tid2) -- (tid8);
\node[circle, scale=0.75, fill] (tid3) at (8.25,3){};
\draw[](tid0) -- (tid1);
\draw[](tid0) -- (tid2);
\draw[](tid0) -- (tid3);
\end{tikzpicture}
\nodepart{three}
\footnotesize{5.625}
\nodepart{four}
\footnotesize{$33\:17\:33\:17$}
};
 \\ 
\node[draw=black, rectangle split,  rectangle split parts=4] (sn0x22ac710){
\footnotesize{5}
\nodepart{two}
\begin{tikzpicture}[scale=.2]
\node[circle, scale=0.75, fill] (tid0) at (3.75,1.5){};
\node[circle, scale=0.75, fill] (tid1) at (2.25,3){};
\node[circle, scale=0.75, fill, task_scheduled] (tid4) at (0.75,4.5){};
\node[circle, scale=0.75, fill, task_scheduled] (tid5) at (2.25,4.5){};
\node[circle, scale=0.75, fill] (tid6) at (3.75,4.5){};
\draw[](tid1) -- (tid4);
\draw[](tid1) -- (tid5);
\draw[](tid1) -- (tid6);
\node[circle, scale=0.75, fill] (tid2) at (5.25,3){};
\node[circle, scale=0.75, fill] (tid7) at (5.25,4.5){};
\draw[](tid2) -- (tid7);
\node[circle, scale=0.75, fill] (tid3) at (6.75,3){};
\node[circle, scale=0.75, fill] (tid8) at (6.75,4.5){};
\draw[](tid3) -- (tid8);
\draw[](tid0) -- (tid1);
\draw[](tid0) -- (tid2);
\draw[](tid0) -- (tid3);
\end{tikzpicture}
\nodepart{three}
\footnotesize{5.625}
\nodepart{four}
\footnotesize{$33\:67$}
};
 \\ 
\node[draw=black, rectangle split,  rectangle split parts=4] (sn0x22ada30){
\footnotesize{5}
\nodepart{two}
\begin{tikzpicture}[scale=.2]
\node[circle, scale=0.75, fill] (tid0) at (3.75,1.5){};
\node[circle, scale=0.75, fill] (tid1) at (2.25,3){};
\node[circle, scale=0.75, fill, task_scheduled] (tid4) at (0.75,4.5){};
\node[circle, scale=0.75, fill] (tid5) at (2.25,4.5){};
\node[circle, scale=0.75, fill] (tid6) at (3.75,4.5){};
\draw[](tid1) -- (tid4);
\draw[](tid1) -- (tid5);
\draw[](tid1) -- (tid6);
\node[circle, scale=0.75, fill] (tid2) at (5.25,3){};
\node[circle, scale=0.75, fill, task_scheduled] (tid7) at (5.25,4.5){};
\draw[](tid2) -- (tid7);
\node[circle, scale=0.75, fill] (tid3) at (6.75,3){};
\node[circle, scale=0.75, fill] (tid8) at (6.75,4.5){};
\draw[](tid3) -- (tid8);
\draw[](tid0) -- (tid1);
\draw[](tid0) -- (tid2);
\draw[](tid0) -- (tid3);
\end{tikzpicture}
\nodepart{three}
\footnotesize{5.625}
\nodepart{four}
\footnotesize{$33\:17\:33\:17$}
};
 \\ 
\node[draw=black, rectangle split,  rectangle split parts=4] (sn0x22ac4a0){
\footnotesize{4.16667}
\nodepart{two}
\begin{tikzpicture}[scale=.2]
\node[circle, scale=0.75, fill] (tid0) at (3,1.5){};
\node[circle, scale=0.75, fill] (tid1) at (1.5,3){};
\node[circle, scale=0.75, fill, task_scheduled] (tid4) at (0.75,4.5){};
\node[circle, scale=0.75, fill] (tid5) at (2.25,4.5){};
\draw[](tid1) -- (tid4);
\draw[](tid1) -- (tid5);
\node[circle, scale=0.75, fill] (tid2) at (3.75,3){};
\node[circle, scale=0.75, fill] (tid6) at (3.75,4.5){};
\node[circle, scale=0.75, fill, task_scheduled] (tid8) at (3.75,6){};
\draw[](tid6) -- (tid8);
\draw[](tid2) -- (tid6);
\node[circle, scale=0.75, fill] (tid3) at (5.25,3){};
\node[circle, scale=0.75, fill] (tid7) at (5.25,4.5){};
\draw[](tid3) -- (tid7);
\draw[](tid0) -- (tid1);
\draw[](tid0) -- (tid2);
\draw[](tid0) -- (tid3);
\end{tikzpicture}
\nodepart{three}
\footnotesize{5.65625}
\nodepart{four}
\footnotesize{$50\:17\:33$}
};
 \\ 
\node[draw=black, rectangle split,  rectangle split parts=4] (sn0x22acd50){
\footnotesize{2.08333}
\nodepart{two}
\begin{tikzpicture}[scale=.2]
\node[circle, scale=0.75, fill] (tid0) at (3,1.5){};
\node[circle, scale=0.75, fill] (tid1) at (1.5,3){};
\node[circle, scale=0.75, fill] (tid4) at (0.75,4.5){};
\node[circle, scale=0.75, fill] (tid5) at (2.25,4.5){};
\draw[](tid1) -- (tid4);
\draw[](tid1) -- (tid5);
\node[circle, scale=0.75, fill] (tid2) at (3.75,3){};
\node[circle, scale=0.75, fill] (tid6) at (3.75,4.5){};
\node[circle, scale=0.75, fill, task_scheduled] (tid8) at (3.75,6){};
\draw[](tid6) -- (tid8);
\draw[](tid2) -- (tid6);
\node[circle, scale=0.75, fill] (tid3) at (5.25,3){};
\node[circle, scale=0.75, fill, task_scheduled] (tid7) at (5.25,4.5){};
\draw[](tid3) -- (tid7);
\draw[](tid0) -- (tid1);
\draw[](tid0) -- (tid2);
\draw[](tid0) -- (tid3);
\end{tikzpicture}
\nodepart{three}
\footnotesize{5.65625}
\nodepart{four}
\footnotesize{$50\:33\:17$}
};
 \\ 
\node[draw=black, rectangle split,  rectangle split parts=4] (sn0x22b54f0){
\footnotesize{24.375}
\nodepart{two}
\begin{tikzpicture}[scale=.2]
\node[circle, scale=0.75, fill] (tid0) at (3.75,1.5){};
\node[circle, scale=0.75, fill] (tid1) at (1.5,3){};
\node[circle, scale=0.75, fill, task_scheduled] (tid4) at (0.75,4.5){};
\node[circle, scale=0.75, fill] (tid5) at (2.25,4.5){};
\draw[](tid1) -- (tid4);
\draw[](tid1) -- (tid5);
\node[circle, scale=0.75, fill] (tid2) at (4.5,3){};
\node[circle, scale=0.75, fill, task_scheduled] (tid6) at (3.75,4.5){};
\node[circle, scale=0.75, fill] (tid7) at (5.25,4.5){};
\draw[](tid2) -- (tid6);
\draw[](tid2) -- (tid7);
\node[circle, scale=0.75, fill] (tid3) at (6.75,3){};
\node[circle, scale=0.75, fill] (tid8) at (6.75,4.5){};
\draw[](tid3) -- (tid8);
\draw[](tid0) -- (tid1);
\draw[](tid0) -- (tid2);
\draw[](tid0) -- (tid3);
\end{tikzpicture}
\nodepart{three}
\footnotesize{5.625}
\nodepart{four}
\footnotesize{$67\:33$}
};
 \\ 
\node[draw=black, rectangle split,  rectangle split parts=4] (sn0x22b5d10){
\footnotesize{12.1875}
\nodepart{two}
\begin{tikzpicture}[scale=.2]
\node[circle, scale=0.75, fill] (tid0) at (3.75,1.5){};
\node[circle, scale=0.75, fill] (tid1) at (1.5,3){};
\node[circle, scale=0.75, fill, task_scheduled] (tid4) at (0.75,4.5){};
\node[circle, scale=0.75, fill, task_scheduled] (tid5) at (2.25,4.5){};
\draw[](tid1) -- (tid4);
\draw[](tid1) -- (tid5);
\node[circle, scale=0.75, fill] (tid2) at (4.5,3){};
\node[circle, scale=0.75, fill] (tid6) at (3.75,4.5){};
\node[circle, scale=0.75, fill] (tid7) at (5.25,4.5){};
\draw[](tid2) -- (tid6);
\draw[](tid2) -- (tid7);
\node[circle, scale=0.75, fill] (tid3) at (6.75,3){};
\node[circle, scale=0.75, fill] (tid8) at (6.75,4.5){};
\draw[](tid3) -- (tid8);
\draw[](tid0) -- (tid1);
\draw[](tid0) -- (tid2);
\draw[](tid0) -- (tid3);
\end{tikzpicture}
\nodepart{three}
\footnotesize{5.625}
\nodepart{four}
\footnotesize{$67\:33$}
};
 \\ 
\node[draw=black, rectangle split,  rectangle split parts=4] (sn0x22b6220){
\footnotesize{23.4375}
\nodepart{two}
\begin{tikzpicture}[scale=.2]
\node[circle, scale=0.75, fill] (tid0) at (3.75,1.5){};
\node[circle, scale=0.75, fill] (tid1) at (1.5,3){};
\node[circle, scale=0.75, fill, task_scheduled] (tid4) at (0.75,4.5){};
\node[circle, scale=0.75, fill] (tid5) at (2.25,4.5){};
\draw[](tid1) -- (tid4);
\draw[](tid1) -- (tid5);
\node[circle, scale=0.75, fill] (tid2) at (4.5,3){};
\node[circle, scale=0.75, fill] (tid6) at (3.75,4.5){};
\node[circle, scale=0.75, fill] (tid7) at (5.25,4.5){};
\draw[](tid2) -- (tid6);
\draw[](tid2) -- (tid7);
\node[circle, scale=0.75, fill] (tid3) at (6.75,3){};
\node[circle, scale=0.75, fill, task_scheduled] (tid8) at (6.75,4.5){};
\draw[](tid3) -- (tid8);
\draw[](tid0) -- (tid1);
\draw[](tid0) -- (tid2);
\draw[](tid0) -- (tid3);
\end{tikzpicture}
\nodepart{three}
\footnotesize{5.625}
\nodepart{four}
\footnotesize{$17\:33\:17\:33$}
};
 \\ 
\\
};
\end{scope}
\begin{scope}[yshift=\leveltopIIII cm, anchor = center]
\matrix (line4)[row sep=0.5cm] {
\node[draw=black, rectangle split,  rectangle split parts=4] (sn0x22baab0){
\footnotesize{3.125}
\nodepart{two}
\begin{tikzpicture}[scale=.2]
\node[circle, scale=0.75, fill] (tid0) at (3,1.5){};
\node[circle, scale=0.75, fill] (tid1) at (1.5,3){};
\node[circle, scale=0.75, fill] (tid4) at (0.75,4.5){};
\node[circle, scale=0.75, fill, task_scheduled] (tid7) at (0.75,6){};
\draw[](tid4) -- (tid7);
\node[circle, scale=0.75, fill, task_scheduled] (tid5) at (2.25,4.5){};
\draw[](tid1) -- (tid4);
\draw[](tid1) -- (tid5);
\node[circle, scale=0.75, fill] (tid2) at (3.75,3){};
\node[circle, scale=0.75, fill] (tid6) at (3.75,4.5){};
\draw[](tid2) -- (tid6);
\node[circle, scale=0.75, fill] (tid3) at (5.25,3){};
\draw[](tid0) -- (tid1);
\draw[](tid0) -- (tid2);
\draw[](tid0) -- (tid3);
\end{tikzpicture}
\nodepart{three}
\footnotesize{5.1875}
\nodepart{four}
\footnotesize{$50\:25\:25$}
};
 \\ 
\node[draw=black, rectangle split,  rectangle split parts=4] (sn0x22bafa0){
\footnotesize{3.125}
\nodepart{two}
\begin{tikzpicture}[scale=.2]
\node[circle, scale=0.75, fill] (tid0) at (3,1.5){};
\node[circle, scale=0.75, fill] (tid1) at (1.5,3){};
\node[circle, scale=0.75, fill] (tid4) at (0.75,4.5){};
\node[circle, scale=0.75, fill, task_scheduled] (tid7) at (0.75,6){};
\draw[](tid4) -- (tid7);
\node[circle, scale=0.75, fill] (tid5) at (2.25,4.5){};
\draw[](tid1) -- (tid4);
\draw[](tid1) -- (tid5);
\node[circle, scale=0.75, fill] (tid2) at (3.75,3){};
\node[circle, scale=0.75, fill, task_scheduled] (tid6) at (3.75,4.5){};
\draw[](tid2) -- (tid6);
\node[circle, scale=0.75, fill] (tid3) at (5.25,3){};
\draw[](tid0) -- (tid1);
\draw[](tid0) -- (tid2);
\draw[](tid0) -- (tid3);
\end{tikzpicture}
\nodepart{three}
\footnotesize{5.1875}
\nodepart{four}
\footnotesize{$50\:50$}
};
 \\ 
\node[draw=black, rectangle split,  rectangle split parts=4] (sn0x22ad490){
\footnotesize{4.16667}
\nodepart{two}
\begin{tikzpicture}[scale=.2]
\node[circle, scale=0.75, fill] (tid0) at (2.25,1.5){};
\node[circle, scale=0.75, fill] (tid1) at (0.75,3){};
\node[circle, scale=0.75, fill] (tid4) at (0.75,4.5){};
\node[circle, scale=0.75, fill, task_scheduled] (tid7) at (0.75,6){};
\draw[](tid4) -- (tid7);
\draw[](tid1) -- (tid4);
\node[circle, scale=0.75, fill] (tid2) at (2.25,3){};
\node[circle, scale=0.75, fill, task_scheduled] (tid5) at (2.25,4.5){};
\draw[](tid2) -- (tid5);
\node[circle, scale=0.75, fill] (tid3) at (3.75,3){};
\node[circle, scale=0.75, fill] (tid6) at (3.75,4.5){};
\draw[](tid3) -- (tid6);
\draw[](tid0) -- (tid1);
\draw[](tid0) -- (tid2);
\draw[](tid0) -- (tid3);
\end{tikzpicture}
\nodepart{three}
\footnotesize{5.1875}
\nodepart{four}
\footnotesize{$50\:50$}
};
 \\ 
\node[draw=black, rectangle split,  rectangle split parts=4] (sn0x22b34f0){
\footnotesize{2.5}
\nodepart{two}
\begin{tikzpicture}[scale=.2]
\node[circle, scale=0.75, fill] (tid0) at (3.75,1.5){};
\node[circle, scale=0.75, fill] (tid1) at (2.25,3){};
\node[circle, scale=0.75, fill, task_scheduled] (tid4) at (0.75,4.5){};
\node[circle, scale=0.75, fill, task_scheduled] (tid5) at (2.25,4.5){};
\node[circle, scale=0.75, fill] (tid6) at (3.75,4.5){};
\draw[](tid1) -- (tid4);
\draw[](tid1) -- (tid5);
\draw[](tid1) -- (tid6);
\node[circle, scale=0.75, fill] (tid2) at (5.25,3){};
\node[circle, scale=0.75, fill] (tid7) at (5.25,4.5){};
\draw[](tid2) -- (tid7);
\node[circle, scale=0.75, fill] (tid3) at (6.75,3){};
\draw[](tid0) -- (tid1);
\draw[](tid0) -- (tid2);
\draw[](tid0) -- (tid3);
\end{tikzpicture}
\nodepart{three}
\footnotesize{5.125}
\nodepart{four}
\footnotesize{$50\:50$}
};
 \\ 
\node[draw=black, rectangle split,  rectangle split parts=4] (sn0x22b3cf0){
\footnotesize{1.25}
\nodepart{two}
\begin{tikzpicture}[scale=.2]
\node[circle, scale=0.75, fill] (tid0) at (3.75,1.5){};
\node[circle, scale=0.75, fill] (tid1) at (2.25,3){};
\node[circle, scale=0.75, fill, task_scheduled] (tid4) at (0.75,4.5){};
\node[circle, scale=0.75, fill] (tid5) at (2.25,4.5){};
\node[circle, scale=0.75, fill] (tid6) at (3.75,4.5){};
\draw[](tid1) -- (tid4);
\draw[](tid1) -- (tid5);
\draw[](tid1) -- (tid6);
\node[circle, scale=0.75, fill] (tid2) at (5.25,3){};
\node[circle, scale=0.75, fill, task_scheduled] (tid7) at (5.25,4.5){};
\draw[](tid2) -- (tid7);
\node[circle, scale=0.75, fill] (tid3) at (6.75,3){};
\draw[](tid0) -- (tid1);
\draw[](tid0) -- (tid2);
\draw[](tid0) -- (tid3);
\end{tikzpicture}
\nodepart{three}
\footnotesize{5.125}
\nodepart{four}
\footnotesize{$50\:50$}
};
 \\ 
\node[draw=black, rectangle split,  rectangle split parts=4] (sn0x22b2b90){
\footnotesize{2.08333}
\nodepart{two}
\begin{tikzpicture}[scale=.2]
\node[circle, scale=0.75, fill] (tid0) at (3,1.5){};
\node[circle, scale=0.75, fill] (tid1) at (1.5,3){};
\node[circle, scale=0.75, fill, task_scheduled] (tid4) at (0.75,4.5){};
\node[circle, scale=0.75, fill] (tid5) at (2.25,4.5){};
\draw[](tid1) -- (tid4);
\draw[](tid1) -- (tid5);
\node[circle, scale=0.75, fill] (tid2) at (3.75,3){};
\node[circle, scale=0.75, fill] (tid6) at (3.75,4.5){};
\node[circle, scale=0.75, fill, task_scheduled] (tid7) at (3.75,6){};
\draw[](tid6) -- (tid7);
\draw[](tid2) -- (tid6);
\node[circle, scale=0.75, fill] (tid3) at (5.25,3){};
\draw[](tid0) -- (tid1);
\draw[](tid0) -- (tid2);
\draw[](tid0) -- (tid3);
\end{tikzpicture}
\nodepart{three}
\footnotesize{5.1875}
\nodepart{four}
\footnotesize{$50\:25\:25$}
};
 \\ 
\node[draw=black, rectangle split,  rectangle split parts=4] (sn0x22b66d0){
\footnotesize{6.19792}
\nodepart{two}
\begin{tikzpicture}[scale=.2]
\node[circle, scale=0.75, fill] (tid0) at (3.75,1.5){};
\node[circle, scale=0.75, fill] (tid1) at (1.5,3){};
\node[circle, scale=0.75, fill, task_scheduled] (tid4) at (0.75,4.5){};
\node[circle, scale=0.75, fill, task_scheduled] (tid5) at (2.25,4.5){};
\draw[](tid1) -- (tid4);
\draw[](tid1) -- (tid5);
\node[circle, scale=0.75, fill] (tid2) at (4.5,3){};
\node[circle, scale=0.75, fill] (tid6) at (3.75,4.5){};
\node[circle, scale=0.75, fill] (tid7) at (5.25,4.5){};
\draw[](tid2) -- (tid6);
\draw[](tid2) -- (tid7);
\node[circle, scale=0.75, fill] (tid3) at (6.75,3){};
\draw[](tid0) -- (tid1);
\draw[](tid0) -- (tid2);
\draw[](tid0) -- (tid3);
\end{tikzpicture}
\nodepart{three}
\footnotesize{5.125}
\nodepart{four}
\footnotesize{$1$}
};
 \\ 
\node[draw=black, rectangle split,  rectangle split parts=4] (sn0x22b6c20){
\footnotesize{12.3958}
\nodepart{two}
\begin{tikzpicture}[scale=.2]
\node[circle, scale=0.75, fill] (tid0) at (3.75,1.5){};
\node[circle, scale=0.75, fill] (tid1) at (1.5,3){};
\node[circle, scale=0.75, fill, task_scheduled] (tid4) at (0.75,4.5){};
\node[circle, scale=0.75, fill] (tid5) at (2.25,4.5){};
\draw[](tid1) -- (tid4);
\draw[](tid1) -- (tid5);
\node[circle, scale=0.75, fill] (tid2) at (4.5,3){};
\node[circle, scale=0.75, fill, task_scheduled] (tid6) at (3.75,4.5){};
\node[circle, scale=0.75, fill] (tid7) at (5.25,4.5){};
\draw[](tid2) -- (tid6);
\draw[](tid2) -- (tid7);
\node[circle, scale=0.75, fill] (tid3) at (6.75,3){};
\draw[](tid0) -- (tid1);
\draw[](tid0) -- (tid2);
\draw[](tid0) -- (tid3);
\end{tikzpicture}
\nodepart{three}
\footnotesize{5.125}
\nodepart{four}
\footnotesize{$50\:50$}
};
 \\ 
\node[draw=black, rectangle split,  rectangle split parts=4] (sn0x22ade60){
\footnotesize{11.1806}
\nodepart{two}
\begin{tikzpicture}[scale=.2]
\node[circle, scale=0.75, fill] (tid0) at (3,1.5){};
\node[circle, scale=0.75, fill] (tid1) at (1.5,3){};
\node[circle, scale=0.75, fill, task_scheduled] (tid4) at (0.75,4.5){};
\node[circle, scale=0.75, fill, task_scheduled] (tid5) at (2.25,4.5){};
\draw[](tid1) -- (tid4);
\draw[](tid1) -- (tid5);
\node[circle, scale=0.75, fill] (tid2) at (3.75,3){};
\node[circle, scale=0.75, fill] (tid6) at (3.75,4.5){};
\draw[](tid2) -- (tid6);
\node[circle, scale=0.75, fill] (tid3) at (5.25,3){};
\node[circle, scale=0.75, fill] (tid7) at (5.25,4.5){};
\draw[](tid3) -- (tid7);
\draw[](tid0) -- (tid1);
\draw[](tid0) -- (tid2);
\draw[](tid0) -- (tid3);
\end{tikzpicture}
\nodepart{three}
\footnotesize{5.125}
\nodepart{four}
\footnotesize{$1$}
};
 \\ 
\node[draw=black, rectangle split,  rectangle split parts=4] (sn0x22ae720){
\footnotesize{43.4375}
\nodepart{two}
\begin{tikzpicture}[scale=.2]
\node[circle, scale=0.75, fill] (tid0) at (3,1.5){};
\node[circle, scale=0.75, fill] (tid1) at (1.5,3){};
\node[circle, scale=0.75, fill, task_scheduled] (tid4) at (0.75,4.5){};
\node[circle, scale=0.75, fill] (tid5) at (2.25,4.5){};
\draw[](tid1) -- (tid4);
\draw[](tid1) -- (tid5);
\node[circle, scale=0.75, fill] (tid2) at (3.75,3){};
\node[circle, scale=0.75, fill, task_scheduled] (tid6) at (3.75,4.5){};
\draw[](tid2) -- (tid6);
\node[circle, scale=0.75, fill] (tid3) at (5.25,3){};
\node[circle, scale=0.75, fill] (tid7) at (5.25,4.5){};
\draw[](tid3) -- (tid7);
\draw[](tid0) -- (tid1);
\draw[](tid0) -- (tid2);
\draw[](tid0) -- (tid3);
\end{tikzpicture}
\nodepart{three}
\footnotesize{5.125}
\nodepart{four}
\footnotesize{$25\:25\:50$}
};
 \\ 
\node[draw=black, rectangle split,  rectangle split parts=4] (sn0x22b2d60){
\footnotesize{10.5382}
\nodepart{two}
\begin{tikzpicture}[scale=.2]
\node[circle, scale=0.75, fill] (tid0) at (3,1.5){};
\node[circle, scale=0.75, fill] (tid1) at (1.5,3){};
\node[circle, scale=0.75, fill] (tid4) at (0.75,4.5){};
\node[circle, scale=0.75, fill] (tid5) at (2.25,4.5){};
\draw[](tid1) -- (tid4);
\draw[](tid1) -- (tid5);
\node[circle, scale=0.75, fill] (tid2) at (3.75,3){};
\node[circle, scale=0.75, fill, task_scheduled] (tid6) at (3.75,4.5){};
\draw[](tid2) -- (tid6);
\node[circle, scale=0.75, fill] (tid3) at (5.25,3){};
\node[circle, scale=0.75, fill, task_scheduled] (tid7) at (5.25,4.5){};
\draw[](tid3) -- (tid7);
\draw[](tid0) -- (tid1);
\draw[](tid0) -- (tid2);
\draw[](tid0) -- (tid3);
\end{tikzpicture}
\nodepart{three}
\footnotesize{5.125}
\nodepart{four}
\footnotesize{$1$}
};
 \\ 
\\
};
\end{scope}
\begin{scope}[yshift=\leveltopIIIII cm, anchor = center]
\matrix (line5)[row sep=0.5cm] {
\node[draw=black, rectangle split,  rectangle split parts=4] (sn0x22bb120){
\footnotesize{1.5625}
\nodepart{two}
\begin{tikzpicture}[scale=.2]
\node[circle, scale=0.75, fill] (tid0) at (3,1.5){};
\node[circle, scale=0.75, fill] (tid1) at (1.5,3){};
\node[circle, scale=0.75, fill] (tid4) at (0.75,4.5){};
\node[circle, scale=0.75, fill, task_scheduled] (tid6) at (0.75,6){};
\draw[](tid4) -- (tid6);
\node[circle, scale=0.75, fill, task_scheduled] (tid5) at (2.25,4.5){};
\draw[](tid1) -- (tid4);
\draw[](tid1) -- (tid5);
\node[circle, scale=0.75, fill] (tid2) at (3.75,3){};
\node[circle, scale=0.75, fill] (tid3) at (5.25,3){};
\draw[](tid0) -- (tid1);
\draw[](tid0) -- (tid2);
\draw[](tid0) -- (tid3);
\end{tikzpicture}
\nodepart{three}
\footnotesize{4.75}
\nodepart{four}
\footnotesize{$50\:50$}
};
 \\ 
\node[draw=black, rectangle split,  rectangle split parts=4] (sn0x22ae570){
\footnotesize{4.6875}
\nodepart{two}
\begin{tikzpicture}[scale=.2]
\node[circle, scale=0.75, fill] (tid0) at (2.25,1.5){};
\node[circle, scale=0.75, fill] (tid1) at (0.75,3){};
\node[circle, scale=0.75, fill] (tid4) at (0.75,4.5){};
\node[circle, scale=0.75, fill, task_scheduled] (tid6) at (0.75,6){};
\draw[](tid4) -- (tid6);
\draw[](tid1) -- (tid4);
\node[circle, scale=0.75, fill] (tid2) at (2.25,3){};
\node[circle, scale=0.75, fill, task_scheduled] (tid5) at (2.25,4.5){};
\draw[](tid2) -- (tid5);
\node[circle, scale=0.75, fill] (tid3) at (3.75,3){};
\draw[](tid0) -- (tid1);
\draw[](tid0) -- (tid2);
\draw[](tid0) -- (tid3);
\end{tikzpicture}
\nodepart{three}
\footnotesize{4.75}
\nodepart{four}
\footnotesize{$50\:50$}
};
 \\ 
\node[draw=black, rectangle split,  rectangle split parts=4] (sn0x22b40d0){
\footnotesize{0.625}
\nodepart{two}
\begin{tikzpicture}[scale=.2]
\node[circle, scale=0.75, fill] (tid0) at (3.75,1.5){};
\node[circle, scale=0.75, fill] (tid1) at (2.25,3){};
\node[circle, scale=0.75, fill, task_scheduled] (tid4) at (0.75,4.5){};
\node[circle, scale=0.75, fill, task_scheduled] (tid5) at (2.25,4.5){};
\node[circle, scale=0.75, fill] (tid6) at (3.75,4.5){};
\draw[](tid1) -- (tid4);
\draw[](tid1) -- (tid5);
\draw[](tid1) -- (tid6);
\node[circle, scale=0.75, fill] (tid2) at (5.25,3){};
\node[circle, scale=0.75, fill] (tid3) at (6.75,3){};
\draw[](tid0) -- (tid1);
\draw[](tid0) -- (tid2);
\draw[](tid0) -- (tid3);
\end{tikzpicture}
\nodepart{three}
\footnotesize{4.625}
\nodepart{four}
\footnotesize{$1$}
};
 \\ 
\node[draw=black, rectangle split,  rectangle split parts=4] (sn0x22b14e0){
\footnotesize{19.6094}
\nodepart{two}
\begin{tikzpicture}[scale=.2]
\node[circle, scale=0.75, fill] (tid0) at (3,1.5){};
\node[circle, scale=0.75, fill] (tid1) at (1.5,3){};
\node[circle, scale=0.75, fill, task_scheduled] (tid4) at (0.75,4.5){};
\node[circle, scale=0.75, fill, task_scheduled] (tid5) at (2.25,4.5){};
\draw[](tid1) -- (tid4);
\draw[](tid1) -- (tid5);
\node[circle, scale=0.75, fill] (tid2) at (3.75,3){};
\node[circle, scale=0.75, fill] (tid6) at (3.75,4.5){};
\draw[](tid2) -- (tid6);
\node[circle, scale=0.75, fill] (tid3) at (5.25,3){};
\draw[](tid0) -- (tid1);
\draw[](tid0) -- (tid2);
\draw[](tid0) -- (tid3);
\end{tikzpicture}
\nodepart{three}
\footnotesize{4.625}
\nodepart{four}
\footnotesize{$1$}
};
 \\ 
\node[draw=black, rectangle split,  rectangle split parts=4] (sn0x22b1ca0){
\footnotesize{38.533}
\nodepart{two}
\begin{tikzpicture}[scale=.2]
\node[circle, scale=0.75, fill] (tid0) at (3,1.5){};
\node[circle, scale=0.75, fill] (tid1) at (1.5,3){};
\node[circle, scale=0.75, fill, task_scheduled] (tid4) at (0.75,4.5){};
\node[circle, scale=0.75, fill] (tid5) at (2.25,4.5){};
\draw[](tid1) -- (tid4);
\draw[](tid1) -- (tid5);
\node[circle, scale=0.75, fill] (tid2) at (3.75,3){};
\node[circle, scale=0.75, fill, task_scheduled] (tid6) at (3.75,4.5){};
\draw[](tid2) -- (tid6);
\node[circle, scale=0.75, fill] (tid3) at (5.25,3){};
\draw[](tid0) -- (tid1);
\draw[](tid0) -- (tid2);
\draw[](tid0) -- (tid3);
\end{tikzpicture}
\nodepart{three}
\footnotesize{4.625}
\nodepart{four}
\footnotesize{$50\:50$}
};
 \\ 
\node[draw=black, rectangle split,  rectangle split parts=4] (sn0x22af010){
\footnotesize{34.9826}
\nodepart{two}
\begin{tikzpicture}[scale=.2]
\node[circle, scale=0.75, fill] (tid0) at (2.25,1.5){};
\node[circle, scale=0.75, fill] (tid1) at (0.75,3){};
\node[circle, scale=0.75, fill, task_scheduled] (tid4) at (0.75,4.5){};
\draw[](tid1) -- (tid4);
\node[circle, scale=0.75, fill] (tid2) at (2.25,3){};
\node[circle, scale=0.75, fill, task_scheduled] (tid5) at (2.25,4.5){};
\draw[](tid2) -- (tid5);
\node[circle, scale=0.75, fill] (tid3) at (3.75,3){};
\node[circle, scale=0.75, fill] (tid6) at (3.75,4.5){};
\draw[](tid3) -- (tid6);
\draw[](tid0) -- (tid1);
\draw[](tid0) -- (tid2);
\draw[](tid0) -- (tid3);
\end{tikzpicture}
\nodepart{three}
\footnotesize{4.625}
\nodepart{four}
\footnotesize{$1$}
};
 \\ 
\\
};
\end{scope}
\begin{scope}[yshift=\leveltopIIIIII cm, anchor = center]
\matrix (line6)[row sep=0.5cm] {
\node[draw=black, rectangle split,  rectangle split parts=4] (sn0x22aecc0){
\footnotesize{3.125}
\nodepart{two}
\begin{tikzpicture}[scale=.2]
\node[circle, scale=0.75, fill] (tid0) at (2.25,1.5){};
\node[circle, scale=0.75, fill] (tid1) at (0.75,3){};
\node[circle, scale=0.75, fill] (tid4) at (0.75,4.5){};
\node[circle, scale=0.75, fill, task_scheduled] (tid5) at (0.75,6){};
\draw[](tid4) -- (tid5);
\draw[](tid1) -- (tid4);
\node[circle, scale=0.75, fill, task_scheduled] (tid2) at (2.25,3){};
\node[circle, scale=0.75, fill] (tid3) at (3.75,3){};
\draw[](tid0) -- (tid1);
\draw[](tid0) -- (tid2);
\draw[](tid0) -- (tid3);
\end{tikzpicture}
\nodepart{three}
\footnotesize{4.375}
\nodepart{four}
\footnotesize{$50\:50$}
};
 \\ 
\node[draw=black, rectangle split,  rectangle split parts=4] (sn0x22b21c0){
\footnotesize{20.6727}
\nodepart{two}
\begin{tikzpicture}[scale=.2]
\node[circle, scale=0.75, fill] (tid0) at (3,1.5){};
\node[circle, scale=0.75, fill] (tid1) at (1.5,3){};
\node[circle, scale=0.75, fill, task_scheduled] (tid4) at (0.75,4.5){};
\node[circle, scale=0.75, fill, task_scheduled] (tid5) at (2.25,4.5){};
\draw[](tid1) -- (tid4);
\draw[](tid1) -- (tid5);
\node[circle, scale=0.75, fill] (tid2) at (3.75,3){};
\node[circle, scale=0.75, fill] (tid3) at (5.25,3){};
\draw[](tid0) -- (tid1);
\draw[](tid0) -- (tid2);
\draw[](tid0) -- (tid3);
\end{tikzpicture}
\nodepart{three}
\footnotesize{4.125}
\nodepart{four}
\footnotesize{$1$}
};
 \\ 
\node[draw=black, rectangle split,  rectangle split parts=4] (sn0x22af160){
\footnotesize{76.2023}
\nodepart{two}
\begin{tikzpicture}[scale=.2]
\node[circle, scale=0.75, fill] (tid0) at (2.25,1.5){};
\node[circle, scale=0.75, fill] (tid1) at (0.75,3){};
\node[circle, scale=0.75, fill, task_scheduled] (tid4) at (0.75,4.5){};
\draw[](tid1) -- (tid4);
\node[circle, scale=0.75, fill] (tid2) at (2.25,3){};
\node[circle, scale=0.75, fill, task_scheduled] (tid5) at (2.25,4.5){};
\draw[](tid2) -- (tid5);
\node[circle, scale=0.75, fill] (tid3) at (3.75,3){};
\draw[](tid0) -- (tid1);
\draw[](tid0) -- (tid2);
\draw[](tid0) -- (tid3);
\end{tikzpicture}
\nodepart{three}
\footnotesize{4.125}
\nodepart{four}
\footnotesize{$1$}
};
 \\ 
\\
};
\end{scope}
\begin{scope}[yshift=\leveltopIIIIIII cm, anchor = center]
\matrix (line7)[row sep=0.5cm] {
\node[draw=black, rectangle split,  rectangle split parts=4] (sn0x22af6c0){
\footnotesize{1.5625}
\nodepart{two}
\begin{tikzpicture}[scale=.2]
\node[circle, scale=0.75, fill] (tid0) at (1.5,1.5){};
\node[circle, scale=0.75, fill] (tid1) at (0.75,3){};
\node[circle, scale=0.75, fill] (tid3) at (0.75,4.5){};
\node[circle, scale=0.75, fill, task_scheduled] (tid4) at (0.75,6){};
\draw[](tid3) -- (tid4);
\draw[](tid1) -- (tid3);
\node[circle, scale=0.75, fill, task_scheduled] (tid2) at (2.25,3){};
\draw[](tid0) -- (tid1);
\draw[](tid0) -- (tid2);
\end{tikzpicture}
\nodepart{three}
\footnotesize{4.125}
\nodepart{four}
\footnotesize{$50\:50$}
};
 \\ 
\node[draw=black, rectangle split,  rectangle split parts=4] (sn0x22afb50){
\footnotesize{98.4375}
\nodepart{two}
\begin{tikzpicture}[scale=.2]
\node[circle, scale=0.75, fill] (tid0) at (2.25,1.5){};
\node[circle, scale=0.75, fill] (tid1) at (0.75,3){};
\node[circle, scale=0.75, fill, task_scheduled] (tid4) at (0.75,4.5){};
\draw[](tid1) -- (tid4);
\node[circle, scale=0.75, fill, task_scheduled] (tid2) at (2.25,3){};
\node[circle, scale=0.75, fill] (tid3) at (3.75,3){};
\draw[](tid0) -- (tid1);
\draw[](tid0) -- (tid2);
\draw[](tid0) -- (tid3);
\end{tikzpicture}
\nodepart{three}
\footnotesize{3.625}
\nodepart{four}
\footnotesize{$50\:50$}
};
 \\ 
\\
};
\end{scope}
\draw (sn0x22a97d0.east) -- (sn0x22b7700.west);
\draw (sn0x22a97d0.east) -- (sn0x22b8ae0.west);
\draw (sn0x22a97d0.east) -- (sn0x22b8c20.west);
\draw (sn0x22a97d0.east) -- (sn0x22b8400.west);
\draw (sn0x22a97d0.east) -- (sn0x22ab2c0.west);
\draw (sn0x22a97d0.east) -- (sn0x22b9170.west);
\draw (sn0x22b7700.east) -- (sn0x22b9da0.west);
\draw (sn0x22b7700.east) -- (sn0x22ba8b0.west);
\draw (sn0x22b7700.east) -- (sn0x22b54f0.west);
\draw (sn0x22b7700.east) -- (sn0x22b5d10.west);
\draw (sn0x22b7700.east) -- (sn0x22b6220.west);
\draw (sn0x22b8ae0.east) -- (sn0x22ac4a0.west);
\draw (sn0x22b8ae0.east) -- (sn0x22acd50.west);
\draw (sn0x22b8ae0.east) -- (sn0x22b5d10.west);
\draw (sn0x22b8ae0.east) -- (sn0x22b54f0.west);
\draw (sn0x22b8ae0.east) -- (sn0x22b6220.west);
\draw (sn0x22b8c20.east) -- (sn0x22bb880.west);
\draw (sn0x22b8c20.east) -- (sn0x22bc030.west);
\draw (sn0x22b8c20.east) -- (sn0x22b6220.west);
\draw (sn0x22b8400.east) -- (sn0x22b5d10.west);
\draw (sn0x22b8400.east) -- (sn0x22b54f0.west);
\draw (sn0x22b8400.east) -- (sn0x22b6220.west);
\draw (sn0x22ab2c0.east) -- (sn0x22b54f0.west);
\draw (sn0x22ab2c0.east) -- (sn0x22b5d10.west);
\draw (sn0x22ab2c0.east) -- (sn0x22b6220.west);
\draw (sn0x22ab2c0.east) -- (sn0x22ac710.west);
\draw (sn0x22ab2c0.east) -- (sn0x22ada30.west);
\draw (sn0x22b9170.east) -- (sn0x22b6220.west);
\draw (sn0x22b9170.east) -- (sn0x22bc5d0.west);
\draw (sn0x22b9170.east) -- (sn0x22b7500.west);
\draw (sn0x22bb880.east) -- (sn0x22b2b90.west);
\draw (sn0x22bb880.east) -- (sn0x22b66d0.west);
\draw (sn0x22bb880.east) -- (sn0x22b6c20.west);
\draw (sn0x22bc030.east) -- (sn0x22baab0.west);
\draw (sn0x22bc030.east) -- (sn0x22bafa0.west);
\draw (sn0x22bc030.east) -- (sn0x22b6c20.west);
\draw (sn0x22bc030.east) -- (sn0x22b66d0.west);
\draw (sn0x22b9da0.east) -- (sn0x22ad490.west);
\draw (sn0x22b9da0.east) -- (sn0x22ade60.west);
\draw (sn0x22b9da0.east) -- (sn0x22ae720.west);
\draw (sn0x22ba8b0.east) -- (sn0x22baab0.west);
\draw (sn0x22ba8b0.east) -- (sn0x22bafa0.west);
\draw (sn0x22ba8b0.east) -- (sn0x22ae720.west);
\draw (sn0x22ba8b0.east) -- (sn0x22b2d60.west);
\draw (sn0x22bc5d0.east) -- (sn0x22b66d0.west);
\draw (sn0x22bc5d0.east) -- (sn0x22b6c20.west);
\draw (sn0x22b7500.east) -- (sn0x22b6c20.west);
\draw (sn0x22b7500.east) -- (sn0x22b66d0.west);
\draw (sn0x22b7500.east) -- (sn0x22b34f0.west);
\draw (sn0x22b7500.east) -- (sn0x22b3cf0.west);
\draw (sn0x22ac710.east) -- (sn0x22ade60.west);
\draw (sn0x22ac710.east) -- (sn0x22ae720.west);
\draw (sn0x22ada30.east) -- (sn0x22ae720.west);
\draw (sn0x22ada30.east) -- (sn0x22b2d60.west);
\draw (sn0x22ada30.east) -- (sn0x22b34f0.west);
\draw (sn0x22ada30.east) -- (sn0x22b3cf0.west);
\draw (sn0x22ac4a0.east) -- (sn0x22ad490.west);
\draw (sn0x22ac4a0.east) -- (sn0x22ade60.west);
\draw (sn0x22ac4a0.east) -- (sn0x22ae720.west);
\draw (sn0x22acd50.east) -- (sn0x22b2b90.west);
\draw (sn0x22acd50.east) -- (sn0x22ae720.west);
\draw (sn0x22acd50.east) -- (sn0x22b2d60.west);
\draw (sn0x22b54f0.east) -- (sn0x22ae720.west);
\draw (sn0x22b54f0.east) -- (sn0x22ade60.west);
\draw (sn0x22b5d10.east) -- (sn0x22ae720.west);
\draw (sn0x22b5d10.east) -- (sn0x22b2d60.west);
\draw (sn0x22b6220.east) -- (sn0x22b2d60.west);
\draw (sn0x22b6220.east) -- (sn0x22ae720.west);
\draw (sn0x22b6220.east) -- (sn0x22b66d0.west);
\draw (sn0x22b6220.east) -- (sn0x22b6c20.west);
\draw (sn0x22baab0.east) -- (sn0x22ae570.west);
\draw (sn0x22baab0.east) -- (sn0x22b14e0.west);
\draw (sn0x22baab0.east) -- (sn0x22b1ca0.west);
\draw (sn0x22bafa0.east) -- (sn0x22bb120.west);
\draw (sn0x22bafa0.east) -- (sn0x22b1ca0.west);
\draw (sn0x22ad490.east) -- (sn0x22ae570.west);
\draw (sn0x22ad490.east) -- (sn0x22af010.west);
\draw (sn0x22b34f0.east) -- (sn0x22b14e0.west);
\draw (sn0x22b34f0.east) -- (sn0x22b1ca0.west);
\draw (sn0x22b3cf0.east) -- (sn0x22b1ca0.west);
\draw (sn0x22b3cf0.east) -- (sn0x22b40d0.west);
\draw (sn0x22b2b90.east) -- (sn0x22ae570.west);
\draw (sn0x22b2b90.east) -- (sn0x22b14e0.west);
\draw (sn0x22b2b90.east) -- (sn0x22b1ca0.west);
\draw (sn0x22b66d0.east) -- (sn0x22b1ca0.west);
\draw (sn0x22b6c20.east) -- (sn0x22b1ca0.west);
\draw (sn0x22b6c20.east) -- (sn0x22b14e0.west);
\draw (sn0x22ade60.east) -- (sn0x22af010.west);
\draw (sn0x22ae720.east) -- (sn0x22af010.west);
\draw (sn0x22ae720.east) -- (sn0x22b14e0.west);
\draw (sn0x22ae720.east) -- (sn0x22b1ca0.west);
\draw (sn0x22b2d60.east) -- (sn0x22b1ca0.west);
\draw (sn0x22bb120.east) -- (sn0x22aecc0.west);
\draw (sn0x22bb120.east) -- (sn0x22b21c0.west);
\draw (sn0x22ae570.east) -- (sn0x22aecc0.west);
\draw (sn0x22ae570.east) -- (sn0x22af160.west);
\draw (sn0x22b40d0.east) -- (sn0x22b21c0.west);
\draw (sn0x22b14e0.east) -- (sn0x22af160.west);
\draw (sn0x22b1ca0.east) -- (sn0x22af160.west);
\draw (sn0x22b1ca0.east) -- (sn0x22b21c0.west);
\draw (sn0x22af010.east) -- (sn0x22af160.west);
\draw (sn0x22aecc0.east) -- (sn0x22af6c0.west);
\draw (sn0x22aecc0.east) -- (sn0x22afb50.west);
\draw (sn0x22b21c0.east) -- (sn0x22afb50.west);
\draw (sn0x22af160.east) -- (sn0x22afb50.west);
\end{tikzpicture}

%%% Local Variables:
%%% TeX-master: "thesis/thesis.tex"
%%% End: 
\renewcommand{\leveltopI}{-10cm + \leveltop}
\renewcommand{\leveltopII}{-10cm + \leveltopI}
\renewcommand{\leveltopIII}{-10cm + \leveltopII}
\renewcommand{\leveltopIIII}{-10cm + \leveltopIII}
\renewcommand{\leveltopIIIII}{-10cm + \leveltopIIII}
\renewcommand{\leveltopIIIIII}{-10cm + \leveltopIIIII}
\renewcommand{\leveltopIIIIIII}{-10cm + \leveltopIIIIII}
\renewcommand{\leveltopIIIIIIII}{-10cm + \leveltopIIIIIII}
\renewcommand{\leveltopIIIIIIIII}{-10cm + \leveltopIIIIIIII}
\renewcommand{\leveltopIIIIIIIIII}{-10cm + \leveltopIIIIIIIII}
\renewcommand{\leveltopIIIIIIIIIII}{-10cm + \leveltopIIIIIIIIII}
\begin{tikzpicture}[scale=.2, anchor=south, rotate=90]
\begin{scope}[yshift=\leveltopI cm, anchor = center]
\matrix (line1)[row sep=0.5cm] {
\node[draw=black, rectangle split,  rectangle split parts=4] (sn0x22a97d0){
\footnotesize{100}
\nodepart{two}
\begin{tikzpicture}[scale=.2]
\node[circle, scale=0.75, fill] (tid0) at (4.5,1.5){};
\node[circle, scale=0.75, fill] (tid1) at (2.25,3){};
\node[circle, scale=0.75, fill, task_scheduled] (tid4) at (0.75,4.5){};
\node[circle, scale=0.75, fill] (tid5) at (2.25,4.5){};
\node[circle, scale=0.75, fill] (tid6) at (3.75,4.5){};
\draw[](tid1) -- (tid4);
\draw[](tid1) -- (tid5);
\draw[](tid1) -- (tid6);
\node[circle, scale=0.75, fill] (tid2) at (6,3){};
\node[circle, scale=0.75, fill] (tid7) at (5.25,4.5){};
\node[circle, scale=0.75, fill, task_scheduled] (tid10) at (5.25,6){};
\draw[](tid7) -- (tid10);
\node[circle, scale=0.75, fill] (tid8) at (6.75,4.5){};
\draw[](tid2) -- (tid7);
\draw[](tid2) -- (tid8);
\node[circle, scale=0.75, fill] (tid3) at (8.25,3){};
\node[circle, scale=0.75, fill] (tid9) at (8.25,4.5){};
\draw[](tid3) -- (tid9);
\draw[](tid0) -- (tid1);
\draw[](tid0) -- (tid2);
\draw[](tid0) -- (tid3);
\end{tikzpicture}
\nodepart{three}
\footnotesize{6.63281}
\nodepart{four}
\footnotesize{$25\:12\:12\:20\:20\:10$}
};
 \\ 
\\
};
\end{scope}
\begin{scope}[yshift=\leveltopII cm, anchor = center]
\matrix (line2)[row sep=0.5cm] {
\node[draw=black, rectangle split,  rectangle split parts=4] (sn0x22b7700){
\footnotesize{25}
\nodepart{two}
\begin{tikzpicture}[scale=.2]
\node[circle, scale=0.75, fill] (tid0) at (3.75,1.5){};
\node[circle, scale=0.75, fill] (tid1) at (1.5,3){};
\node[circle, scale=0.75, fill] (tid4) at (0.75,4.5){};
\node[circle, scale=0.75, fill, task_scheduled] (tid9) at (0.75,6){};
\draw[](tid4) -- (tid9);
\node[circle, scale=0.75, fill] (tid5) at (2.25,4.5){};
\draw[](tid1) -- (tid4);
\draw[](tid1) -- (tid5);
\node[circle, scale=0.75, fill] (tid2) at (4.5,3){};
\node[circle, scale=0.75, fill, task_scheduled] (tid6) at (3.75,4.5){};
\node[circle, scale=0.75, fill] (tid7) at (5.25,4.5){};
\draw[](tid2) -- (tid6);
\draw[](tid2) -- (tid7);
\node[circle, scale=0.75, fill] (tid3) at (6.75,3){};
\node[circle, scale=0.75, fill] (tid8) at (6.75,4.5){};
\draw[](tid3) -- (tid8);
\draw[](tid0) -- (tid1);
\draw[](tid0) -- (tid2);
\draw[](tid0) -- (tid3);
\end{tikzpicture}
\nodepart{three}
\footnotesize{6.14062}
\nodepart{four}
\footnotesize{$17\:33\:25\:12\:12$}
};
 \\ 
\node[draw=black, rectangle split,  rectangle split parts=4] (sn0x22b8ae0){
\footnotesize{12.5}
\nodepart{two}
\begin{tikzpicture}[scale=.2]
\node[circle, scale=0.75, fill] (tid0) at (3.75,1.5){};
\node[circle, scale=0.75, fill] (tid1) at (1.5,3){};
\node[circle, scale=0.75, fill] (tid4) at (0.75,4.5){};
\node[circle, scale=0.75, fill, task_scheduled] (tid9) at (0.75,6){};
\draw[](tid4) -- (tid9);
\node[circle, scale=0.75, fill, task_scheduled] (tid5) at (2.25,4.5){};
\draw[](tid1) -- (tid4);
\draw[](tid1) -- (tid5);
\node[circle, scale=0.75, fill] (tid2) at (4.5,3){};
\node[circle, scale=0.75, fill] (tid6) at (3.75,4.5){};
\node[circle, scale=0.75, fill] (tid7) at (5.25,4.5){};
\draw[](tid2) -- (tid6);
\draw[](tid2) -- (tid7);
\node[circle, scale=0.75, fill] (tid3) at (6.75,3){};
\node[circle, scale=0.75, fill] (tid8) at (6.75,4.5){};
\draw[](tid3) -- (tid8);
\draw[](tid0) -- (tid1);
\draw[](tid0) -- (tid2);
\draw[](tid0) -- (tid3);
\end{tikzpicture}
\nodepart{three}
\footnotesize{6.14062}
\nodepart{four}
\footnotesize{$33\:17\:12\:25\:12$}
};
 \\ 
\node[draw=black, rectangle split,  rectangle split parts=4] (sn0x22b8c20){
\footnotesize{12.5}
\nodepart{two}
\begin{tikzpicture}[scale=.2]
\node[circle, scale=0.75, fill] (tid0) at (3.75,1.5){};
\node[circle, scale=0.75, fill] (tid1) at (1.5,3){};
\node[circle, scale=0.75, fill] (tid4) at (0.75,4.5){};
\node[circle, scale=0.75, fill, task_scheduled] (tid9) at (0.75,6){};
\draw[](tid4) -- (tid9);
\node[circle, scale=0.75, fill] (tid5) at (2.25,4.5){};
\draw[](tid1) -- (tid4);
\draw[](tid1) -- (tid5);
\node[circle, scale=0.75, fill] (tid2) at (4.5,3){};
\node[circle, scale=0.75, fill] (tid6) at (3.75,4.5){};
\node[circle, scale=0.75, fill] (tid7) at (5.25,4.5){};
\draw[](tid2) -- (tid6);
\draw[](tid2) -- (tid7);
\node[circle, scale=0.75, fill] (tid3) at (6.75,3){};
\node[circle, scale=0.75, fill, task_scheduled] (tid8) at (6.75,4.5){};
\draw[](tid3) -- (tid8);
\draw[](tid0) -- (tid1);
\draw[](tid0) -- (tid2);
\draw[](tid0) -- (tid3);
\end{tikzpicture}
\nodepart{three}
\footnotesize{6.14062}
\nodepart{four}
\footnotesize{$17\:33\:50$}
};
 \\ 
\node[draw=black, rectangle split,  rectangle split parts=4] (sn0x22b8400){
\footnotesize{20}
\nodepart{two}
\begin{tikzpicture}[scale=.2]
\node[circle, scale=0.75, fill] (tid0) at (4.5,1.5){};
\node[circle, scale=0.75, fill] (tid1) at (2.25,3){};
\node[circle, scale=0.75, fill, task_scheduled] (tid4) at (0.75,4.5){};
\node[circle, scale=0.75, fill, task_scheduled] (tid5) at (2.25,4.5){};
\node[circle, scale=0.75, fill] (tid6) at (3.75,4.5){};
\draw[](tid1) -- (tid4);
\draw[](tid1) -- (tid5);
\draw[](tid1) -- (tid6);
\node[circle, scale=0.75, fill] (tid2) at (6,3){};
\node[circle, scale=0.75, fill] (tid7) at (5.25,4.5){};
\node[circle, scale=0.75, fill] (tid8) at (6.75,4.5){};
\draw[](tid2) -- (tid7);
\draw[](tid2) -- (tid8);
\node[circle, scale=0.75, fill] (tid3) at (8.25,3){};
\node[circle, scale=0.75, fill] (tid9) at (8.25,4.5){};
\draw[](tid3) -- (tid9);
\draw[](tid0) -- (tid1);
\draw[](tid0) -- (tid2);
\draw[](tid0) -- (tid3);
\end{tikzpicture}
\nodepart{three}
\footnotesize{6.125}
\nodepart{four}
\footnotesize{$25\:50\:25$}
};
 \\ 
\node[draw=black, rectangle split,  rectangle split parts=4] (sn0x22ab2c0){
\footnotesize{20}
\nodepart{two}
\begin{tikzpicture}[scale=.2]
\node[circle, scale=0.75, fill] (tid0) at (4.5,1.5){};
\node[circle, scale=0.75, fill] (tid1) at (2.25,3){};
\node[circle, scale=0.75, fill, task_scheduled] (tid4) at (0.75,4.5){};
\node[circle, scale=0.75, fill] (tid5) at (2.25,4.5){};
\node[circle, scale=0.75, fill] (tid6) at (3.75,4.5){};
\draw[](tid1) -- (tid4);
\draw[](tid1) -- (tid5);
\draw[](tid1) -- (tid6);
\node[circle, scale=0.75, fill] (tid2) at (6,3){};
\node[circle, scale=0.75, fill, task_scheduled] (tid7) at (5.25,4.5){};
\node[circle, scale=0.75, fill] (tid8) at (6.75,4.5){};
\draw[](tid2) -- (tid7);
\draw[](tid2) -- (tid8);
\node[circle, scale=0.75, fill] (tid3) at (8.25,3){};
\node[circle, scale=0.75, fill] (tid9) at (8.25,4.5){};
\draw[](tid3) -- (tid9);
\draw[](tid0) -- (tid1);
\draw[](tid0) -- (tid2);
\draw[](tid0) -- (tid3);
\end{tikzpicture}
\nodepart{three}
\footnotesize{6.125}
\nodepart{four}
\footnotesize{$25\:25\:25\:12\:12$}
};
 \\ 
\node[draw=black, rectangle split,  rectangle split parts=4] (sn0x22b9170){
\footnotesize{10}
\nodepart{two}
\begin{tikzpicture}[scale=.2]
\node[circle, scale=0.75, fill] (tid0) at (4.5,1.5){};
\node[circle, scale=0.75, fill] (tid1) at (2.25,3){};
\node[circle, scale=0.75, fill, task_scheduled] (tid4) at (0.75,4.5){};
\node[circle, scale=0.75, fill] (tid5) at (2.25,4.5){};
\node[circle, scale=0.75, fill] (tid6) at (3.75,4.5){};
\draw[](tid1) -- (tid4);
\draw[](tid1) -- (tid5);
\draw[](tid1) -- (tid6);
\node[circle, scale=0.75, fill] (tid2) at (6,3){};
\node[circle, scale=0.75, fill] (tid7) at (5.25,4.5){};
\node[circle, scale=0.75, fill] (tid8) at (6.75,4.5){};
\draw[](tid2) -- (tid7);
\draw[](tid2) -- (tid8);
\node[circle, scale=0.75, fill] (tid3) at (8.25,3){};
\node[circle, scale=0.75, fill, task_scheduled] (tid9) at (8.25,4.5){};
\draw[](tid3) -- (tid9);
\draw[](tid0) -- (tid1);
\draw[](tid0) -- (tid2);
\draw[](tid0) -- (tid3);
\end{tikzpicture}
\nodepart{three}
\footnotesize{6.125}
\nodepart{four}
\footnotesize{$25\:25\:50$}
};
 \\ 
\\
};
\end{scope}
\begin{scope}[yshift=\leveltopIII cm, anchor = center]
\matrix (line3)[row sep=0.5cm] {
\node[draw=black, rectangle split,  rectangle split parts=4] (sn0x22bb880){
\footnotesize{2.08333}
\nodepart{two}
\begin{tikzpicture}[scale=.2]
\node[circle, scale=0.75, fill] (tid0) at (3.75,1.5){};
\node[circle, scale=0.75, fill] (tid1) at (1.5,3){};
\node[circle, scale=0.75, fill] (tid4) at (0.75,4.5){};
\node[circle, scale=0.75, fill, task_scheduled] (tid8) at (0.75,6){};
\draw[](tid4) -- (tid8);
\node[circle, scale=0.75, fill, task_scheduled] (tid5) at (2.25,4.5){};
\draw[](tid1) -- (tid4);
\draw[](tid1) -- (tid5);
\node[circle, scale=0.75, fill] (tid2) at (4.5,3){};
\node[circle, scale=0.75, fill] (tid6) at (3.75,4.5){};
\node[circle, scale=0.75, fill] (tid7) at (5.25,4.5){};
\draw[](tid2) -- (tid6);
\draw[](tid2) -- (tid7);
\node[circle, scale=0.75, fill] (tid3) at (6.75,3){};
\draw[](tid0) -- (tid1);
\draw[](tid0) -- (tid2);
\draw[](tid0) -- (tid3);
\end{tikzpicture}
\nodepart{three}
\footnotesize{5.65625}
\nodepart{four}
\footnotesize{$50\:17\:33$}
};
 \\ 
\node[draw=black, rectangle split,  rectangle split parts=4] (sn0x22bc030){
\footnotesize{4.16667}
\nodepart{two}
\begin{tikzpicture}[scale=.2]
\node[circle, scale=0.75, fill] (tid0) at (3.75,1.5){};
\node[circle, scale=0.75, fill] (tid1) at (1.5,3){};
\node[circle, scale=0.75, fill] (tid4) at (0.75,4.5){};
\node[circle, scale=0.75, fill, task_scheduled] (tid8) at (0.75,6){};
\draw[](tid4) -- (tid8);
\node[circle, scale=0.75, fill] (tid5) at (2.25,4.5){};
\draw[](tid1) -- (tid4);
\draw[](tid1) -- (tid5);
\node[circle, scale=0.75, fill] (tid2) at (4.5,3){};
\node[circle, scale=0.75, fill, task_scheduled] (tid6) at (3.75,4.5){};
\node[circle, scale=0.75, fill] (tid7) at (5.25,4.5){};
\draw[](tid2) -- (tid6);
\draw[](tid2) -- (tid7);
\node[circle, scale=0.75, fill] (tid3) at (6.75,3){};
\draw[](tid0) -- (tid1);
\draw[](tid0) -- (tid2);
\draw[](tid0) -- (tid3);
\end{tikzpicture}
\nodepart{three}
\footnotesize{5.65625}
\nodepart{four}
\footnotesize{$25\:25\:33\:17$}
};
 \\ 
\node[draw=black, rectangle split,  rectangle split parts=4] (sn0x22b9da0){
\footnotesize{4.16667}
\nodepart{two}
\begin{tikzpicture}[scale=.2]
\node[circle, scale=0.75, fill] (tid0) at (3,1.5){};
\node[circle, scale=0.75, fill] (tid1) at (1.5,3){};
\node[circle, scale=0.75, fill] (tid4) at (0.75,4.5){};
\node[circle, scale=0.75, fill, task_scheduled] (tid8) at (0.75,6){};
\draw[](tid4) -- (tid8);
\node[circle, scale=0.75, fill, task_scheduled] (tid5) at (2.25,4.5){};
\draw[](tid1) -- (tid4);
\draw[](tid1) -- (tid5);
\node[circle, scale=0.75, fill] (tid2) at (3.75,3){};
\node[circle, scale=0.75, fill] (tid6) at (3.75,4.5){};
\draw[](tid2) -- (tid6);
\node[circle, scale=0.75, fill] (tid3) at (5.25,3){};
\node[circle, scale=0.75, fill] (tid7) at (5.25,4.5){};
\draw[](tid3) -- (tid7);
\draw[](tid0) -- (tid1);
\draw[](tid0) -- (tid2);
\draw[](tid0) -- (tid3);
\end{tikzpicture}
\nodepart{three}
\footnotesize{5.65625}
\nodepart{four}
\footnotesize{$50\:17\:33$}
};
 \\ 
\node[draw=black, rectangle split,  rectangle split parts=4] (sn0x22ba8b0){
\footnotesize{8.33333}
\nodepart{two}
\begin{tikzpicture}[scale=.2]
\node[circle, scale=0.75, fill] (tid0) at (3,1.5){};
\node[circle, scale=0.75, fill] (tid1) at (1.5,3){};
\node[circle, scale=0.75, fill] (tid4) at (0.75,4.5){};
\node[circle, scale=0.75, fill, task_scheduled] (tid8) at (0.75,6){};
\draw[](tid4) -- (tid8);
\node[circle, scale=0.75, fill] (tid5) at (2.25,4.5){};
\draw[](tid1) -- (tid4);
\draw[](tid1) -- (tid5);
\node[circle, scale=0.75, fill] (tid2) at (3.75,3){};
\node[circle, scale=0.75, fill, task_scheduled] (tid6) at (3.75,4.5){};
\draw[](tid2) -- (tid6);
\node[circle, scale=0.75, fill] (tid3) at (5.25,3){};
\node[circle, scale=0.75, fill] (tid7) at (5.25,4.5){};
\draw[](tid3) -- (tid7);
\draw[](tid0) -- (tid1);
\draw[](tid0) -- (tid2);
\draw[](tid0) -- (tid3);
\end{tikzpicture}
\nodepart{three}
\footnotesize{5.65625}
\nodepart{four}
\footnotesize{$25\:25\:33\:17$}
};
 \\ 
\node[draw=black, rectangle split,  rectangle split parts=4] (sn0x22bc5d0){
\footnotesize{2.5}
\nodepart{two}
\begin{tikzpicture}[scale=.2]
\node[circle, scale=0.75, fill] (tid0) at (4.5,1.5){};
\node[circle, scale=0.75, fill] (tid1) at (2.25,3){};
\node[circle, scale=0.75, fill, task_scheduled] (tid4) at (0.75,4.5){};
\node[circle, scale=0.75, fill, task_scheduled] (tid5) at (2.25,4.5){};
\node[circle, scale=0.75, fill] (tid6) at (3.75,4.5){};
\draw[](tid1) -- (tid4);
\draw[](tid1) -- (tid5);
\draw[](tid1) -- (tid6);
\node[circle, scale=0.75, fill] (tid2) at (6,3){};
\node[circle, scale=0.75, fill] (tid7) at (5.25,4.5){};
\node[circle, scale=0.75, fill] (tid8) at (6.75,4.5){};
\draw[](tid2) -- (tid7);
\draw[](tid2) -- (tid8);
\node[circle, scale=0.75, fill] (tid3) at (8.25,3){};
\draw[](tid0) -- (tid1);
\draw[](tid0) -- (tid2);
\draw[](tid0) -- (tid3);
\end{tikzpicture}
\nodepart{three}
\footnotesize{5.625}
\nodepart{four}
\footnotesize{$33\:67$}
};
 \\ 
\node[draw=black, rectangle split,  rectangle split parts=4] (sn0x22b7500){
\footnotesize{2.5}
\nodepart{two}
\begin{tikzpicture}[scale=.2]
\node[circle, scale=0.75, fill] (tid0) at (4.5,1.5){};
\node[circle, scale=0.75, fill] (tid1) at (2.25,3){};
\node[circle, scale=0.75, fill, task_scheduled] (tid4) at (0.75,4.5){};
\node[circle, scale=0.75, fill] (tid5) at (2.25,4.5){};
\node[circle, scale=0.75, fill] (tid6) at (3.75,4.5){};
\draw[](tid1) -- (tid4);
\draw[](tid1) -- (tid5);
\draw[](tid1) -- (tid6);
\node[circle, scale=0.75, fill] (tid2) at (6,3){};
\node[circle, scale=0.75, fill, task_scheduled] (tid7) at (5.25,4.5){};
\node[circle, scale=0.75, fill] (tid8) at (6.75,4.5){};
\draw[](tid2) -- (tid7);
\draw[](tid2) -- (tid8);
\node[circle, scale=0.75, fill] (tid3) at (8.25,3){};
\draw[](tid0) -- (tid1);
\draw[](tid0) -- (tid2);
\draw[](tid0) -- (tid3);
\end{tikzpicture}
\nodepart{three}
\footnotesize{5.625}
\nodepart{four}
\footnotesize{$33\:17\:33\:17$}
};
 \\ 
\node[draw=black, rectangle split,  rectangle split parts=4] (sn0x22ac710){
\footnotesize{5}
\nodepart{two}
\begin{tikzpicture}[scale=.2]
\node[circle, scale=0.75, fill] (tid0) at (3.75,1.5){};
\node[circle, scale=0.75, fill] (tid1) at (2.25,3){};
\node[circle, scale=0.75, fill, task_scheduled] (tid4) at (0.75,4.5){};
\node[circle, scale=0.75, fill, task_scheduled] (tid5) at (2.25,4.5){};
\node[circle, scale=0.75, fill] (tid6) at (3.75,4.5){};
\draw[](tid1) -- (tid4);
\draw[](tid1) -- (tid5);
\draw[](tid1) -- (tid6);
\node[circle, scale=0.75, fill] (tid2) at (5.25,3){};
\node[circle, scale=0.75, fill] (tid7) at (5.25,4.5){};
\draw[](tid2) -- (tid7);
\node[circle, scale=0.75, fill] (tid3) at (6.75,3){};
\node[circle, scale=0.75, fill] (tid8) at (6.75,4.5){};
\draw[](tid3) -- (tid8);
\draw[](tid0) -- (tid1);
\draw[](tid0) -- (tid2);
\draw[](tid0) -- (tid3);
\end{tikzpicture}
\nodepart{three}
\footnotesize{5.625}
\nodepart{four}
\footnotesize{$33\:67$}
};
 \\ 
\node[draw=black, rectangle split,  rectangle split parts=4] (sn0x22ada30){
\footnotesize{5}
\nodepart{two}
\begin{tikzpicture}[scale=.2]
\node[circle, scale=0.75, fill] (tid0) at (3.75,1.5){};
\node[circle, scale=0.75, fill] (tid1) at (2.25,3){};
\node[circle, scale=0.75, fill, task_scheduled] (tid4) at (0.75,4.5){};
\node[circle, scale=0.75, fill] (tid5) at (2.25,4.5){};
\node[circle, scale=0.75, fill] (tid6) at (3.75,4.5){};
\draw[](tid1) -- (tid4);
\draw[](tid1) -- (tid5);
\draw[](tid1) -- (tid6);
\node[circle, scale=0.75, fill] (tid2) at (5.25,3){};
\node[circle, scale=0.75, fill, task_scheduled] (tid7) at (5.25,4.5){};
\draw[](tid2) -- (tid7);
\node[circle, scale=0.75, fill] (tid3) at (6.75,3){};
\node[circle, scale=0.75, fill] (tid8) at (6.75,4.5){};
\draw[](tid3) -- (tid8);
\draw[](tid0) -- (tid1);
\draw[](tid0) -- (tid2);
\draw[](tid0) -- (tid3);
\end{tikzpicture}
\nodepart{three}
\footnotesize{5.625}
\nodepart{four}
\footnotesize{$33\:17\:33\:17$}
};
 \\ 
\node[draw=black, rectangle split,  rectangle split parts=4] (sn0x22ac4a0){
\footnotesize{4.16667}
\nodepart{two}
\begin{tikzpicture}[scale=.2]
\node[circle, scale=0.75, fill] (tid0) at (3,1.5){};
\node[circle, scale=0.75, fill] (tid1) at (1.5,3){};
\node[circle, scale=0.75, fill, task_scheduled] (tid4) at (0.75,4.5){};
\node[circle, scale=0.75, fill] (tid5) at (2.25,4.5){};
\draw[](tid1) -- (tid4);
\draw[](tid1) -- (tid5);
\node[circle, scale=0.75, fill] (tid2) at (3.75,3){};
\node[circle, scale=0.75, fill] (tid6) at (3.75,4.5){};
\node[circle, scale=0.75, fill, task_scheduled] (tid8) at (3.75,6){};
\draw[](tid6) -- (tid8);
\draw[](tid2) -- (tid6);
\node[circle, scale=0.75, fill] (tid3) at (5.25,3){};
\node[circle, scale=0.75, fill] (tid7) at (5.25,4.5){};
\draw[](tid3) -- (tid7);
\draw[](tid0) -- (tid1);
\draw[](tid0) -- (tid2);
\draw[](tid0) -- (tid3);
\end{tikzpicture}
\nodepart{three}
\footnotesize{5.65625}
\nodepart{four}
\footnotesize{$50\:17\:33$}
};
 \\ 
\node[draw=black, rectangle split,  rectangle split parts=4] (sn0x22acd50){
\footnotesize{2.08333}
\nodepart{two}
\begin{tikzpicture}[scale=.2]
\node[circle, scale=0.75, fill] (tid0) at (3,1.5){};
\node[circle, scale=0.75, fill] (tid1) at (1.5,3){};
\node[circle, scale=0.75, fill] (tid4) at (0.75,4.5){};
\node[circle, scale=0.75, fill] (tid5) at (2.25,4.5){};
\draw[](tid1) -- (tid4);
\draw[](tid1) -- (tid5);
\node[circle, scale=0.75, fill] (tid2) at (3.75,3){};
\node[circle, scale=0.75, fill] (tid6) at (3.75,4.5){};
\node[circle, scale=0.75, fill, task_scheduled] (tid8) at (3.75,6){};
\draw[](tid6) -- (tid8);
\draw[](tid2) -- (tid6);
\node[circle, scale=0.75, fill] (tid3) at (5.25,3){};
\node[circle, scale=0.75, fill, task_scheduled] (tid7) at (5.25,4.5){};
\draw[](tid3) -- (tid7);
\draw[](tid0) -- (tid1);
\draw[](tid0) -- (tid2);
\draw[](tid0) -- (tid3);
\end{tikzpicture}
\nodepart{three}
\footnotesize{5.65625}
\nodepart{four}
\footnotesize{$50\:33\:17$}
};
 \\ 
\node[draw=black, rectangle split,  rectangle split parts=4] (sn0x22b54f0){
\footnotesize{24.375}
\nodepart{two}
\begin{tikzpicture}[scale=.2]
\node[circle, scale=0.75, fill] (tid0) at (3.75,1.5){};
\node[circle, scale=0.75, fill] (tid1) at (1.5,3){};
\node[circle, scale=0.75, fill, task_scheduled] (tid4) at (0.75,4.5){};
\node[circle, scale=0.75, fill] (tid5) at (2.25,4.5){};
\draw[](tid1) -- (tid4);
\draw[](tid1) -- (tid5);
\node[circle, scale=0.75, fill] (tid2) at (4.5,3){};
\node[circle, scale=0.75, fill, task_scheduled] (tid6) at (3.75,4.5){};
\node[circle, scale=0.75, fill] (tid7) at (5.25,4.5){};
\draw[](tid2) -- (tid6);
\draw[](tid2) -- (tid7);
\node[circle, scale=0.75, fill] (tid3) at (6.75,3){};
\node[circle, scale=0.75, fill] (tid8) at (6.75,4.5){};
\draw[](tid3) -- (tid8);
\draw[](tid0) -- (tid1);
\draw[](tid0) -- (tid2);
\draw[](tid0) -- (tid3);
\end{tikzpicture}
\nodepart{three}
\footnotesize{5.625}
\nodepart{four}
\footnotesize{$67\:33$}
};
 \\ 
\node[draw=black, rectangle split,  rectangle split parts=4] (sn0x22b5d10){
\footnotesize{12.1875}
\nodepart{two}
\begin{tikzpicture}[scale=.2]
\node[circle, scale=0.75, fill] (tid0) at (3.75,1.5){};
\node[circle, scale=0.75, fill] (tid1) at (1.5,3){};
\node[circle, scale=0.75, fill, task_scheduled] (tid4) at (0.75,4.5){};
\node[circle, scale=0.75, fill, task_scheduled] (tid5) at (2.25,4.5){};
\draw[](tid1) -- (tid4);
\draw[](tid1) -- (tid5);
\node[circle, scale=0.75, fill] (tid2) at (4.5,3){};
\node[circle, scale=0.75, fill] (tid6) at (3.75,4.5){};
\node[circle, scale=0.75, fill] (tid7) at (5.25,4.5){};
\draw[](tid2) -- (tid6);
\draw[](tid2) -- (tid7);
\node[circle, scale=0.75, fill] (tid3) at (6.75,3){};
\node[circle, scale=0.75, fill] (tid8) at (6.75,4.5){};
\draw[](tid3) -- (tid8);
\draw[](tid0) -- (tid1);
\draw[](tid0) -- (tid2);
\draw[](tid0) -- (tid3);
\end{tikzpicture}
\nodepart{three}
\footnotesize{5.625}
\nodepart{four}
\footnotesize{$67\:33$}
};
 \\ 
\node[draw=black, rectangle split,  rectangle split parts=4] (sn0x22b6220){
\footnotesize{23.4375}
\nodepart{two}
\begin{tikzpicture}[scale=.2]
\node[circle, scale=0.75, fill] (tid0) at (3.75,1.5){};
\node[circle, scale=0.75, fill] (tid1) at (1.5,3){};
\node[circle, scale=0.75, fill, task_scheduled] (tid4) at (0.75,4.5){};
\node[circle, scale=0.75, fill] (tid5) at (2.25,4.5){};
\draw[](tid1) -- (tid4);
\draw[](tid1) -- (tid5);
\node[circle, scale=0.75, fill] (tid2) at (4.5,3){};
\node[circle, scale=0.75, fill] (tid6) at (3.75,4.5){};
\node[circle, scale=0.75, fill] (tid7) at (5.25,4.5){};
\draw[](tid2) -- (tid6);
\draw[](tid2) -- (tid7);
\node[circle, scale=0.75, fill] (tid3) at (6.75,3){};
\node[circle, scale=0.75, fill, task_scheduled] (tid8) at (6.75,4.5){};
\draw[](tid3) -- (tid8);
\draw[](tid0) -- (tid1);
\draw[](tid0) -- (tid2);
\draw[](tid0) -- (tid3);
\end{tikzpicture}
\nodepart{three}
\footnotesize{5.625}
\nodepart{four}
\footnotesize{$17\:33\:17\:33$}
};
 \\ 
\\
};
\end{scope}
\begin{scope}[yshift=\leveltopIIII cm, anchor = center]
\matrix (line4)[row sep=0.5cm] {
\node[draw=black, rectangle split,  rectangle split parts=4] (sn0x22baab0){
\footnotesize{3.125}
\nodepart{two}
\begin{tikzpicture}[scale=.2]
\node[circle, scale=0.75, fill] (tid0) at (3,1.5){};
\node[circle, scale=0.75, fill] (tid1) at (1.5,3){};
\node[circle, scale=0.75, fill] (tid4) at (0.75,4.5){};
\node[circle, scale=0.75, fill, task_scheduled] (tid7) at (0.75,6){};
\draw[](tid4) -- (tid7);
\node[circle, scale=0.75, fill, task_scheduled] (tid5) at (2.25,4.5){};
\draw[](tid1) -- (tid4);
\draw[](tid1) -- (tid5);
\node[circle, scale=0.75, fill] (tid2) at (3.75,3){};
\node[circle, scale=0.75, fill] (tid6) at (3.75,4.5){};
\draw[](tid2) -- (tid6);
\node[circle, scale=0.75, fill] (tid3) at (5.25,3){};
\draw[](tid0) -- (tid1);
\draw[](tid0) -- (tid2);
\draw[](tid0) -- (tid3);
\end{tikzpicture}
\nodepart{three}
\footnotesize{5.1875}
\nodepart{four}
\footnotesize{$50\:25\:25$}
};
 \\ 
\node[draw=black, rectangle split,  rectangle split parts=4] (sn0x22bafa0){
\footnotesize{3.125}
\nodepart{two}
\begin{tikzpicture}[scale=.2]
\node[circle, scale=0.75, fill] (tid0) at (3,1.5){};
\node[circle, scale=0.75, fill] (tid1) at (1.5,3){};
\node[circle, scale=0.75, fill] (tid4) at (0.75,4.5){};
\node[circle, scale=0.75, fill, task_scheduled] (tid7) at (0.75,6){};
\draw[](tid4) -- (tid7);
\node[circle, scale=0.75, fill] (tid5) at (2.25,4.5){};
\draw[](tid1) -- (tid4);
\draw[](tid1) -- (tid5);
\node[circle, scale=0.75, fill] (tid2) at (3.75,3){};
\node[circle, scale=0.75, fill, task_scheduled] (tid6) at (3.75,4.5){};
\draw[](tid2) -- (tid6);
\node[circle, scale=0.75, fill] (tid3) at (5.25,3){};
\draw[](tid0) -- (tid1);
\draw[](tid0) -- (tid2);
\draw[](tid0) -- (tid3);
\end{tikzpicture}
\nodepart{three}
\footnotesize{5.1875}
\nodepart{four}
\footnotesize{$50\:50$}
};
 \\ 
\node[draw=black, rectangle split,  rectangle split parts=4] (sn0x22ad490){
\footnotesize{4.16667}
\nodepart{two}
\begin{tikzpicture}[scale=.2]
\node[circle, scale=0.75, fill] (tid0) at (2.25,1.5){};
\node[circle, scale=0.75, fill] (tid1) at (0.75,3){};
\node[circle, scale=0.75, fill] (tid4) at (0.75,4.5){};
\node[circle, scale=0.75, fill, task_scheduled] (tid7) at (0.75,6){};
\draw[](tid4) -- (tid7);
\draw[](tid1) -- (tid4);
\node[circle, scale=0.75, fill] (tid2) at (2.25,3){};
\node[circle, scale=0.75, fill, task_scheduled] (tid5) at (2.25,4.5){};
\draw[](tid2) -- (tid5);
\node[circle, scale=0.75, fill] (tid3) at (3.75,3){};
\node[circle, scale=0.75, fill] (tid6) at (3.75,4.5){};
\draw[](tid3) -- (tid6);
\draw[](tid0) -- (tid1);
\draw[](tid0) -- (tid2);
\draw[](tid0) -- (tid3);
\end{tikzpicture}
\nodepart{three}
\footnotesize{5.1875}
\nodepart{four}
\footnotesize{$50\:50$}
};
 \\ 
\node[draw=black, rectangle split,  rectangle split parts=4] (sn0x22b34f0){
\footnotesize{2.5}
\nodepart{two}
\begin{tikzpicture}[scale=.2]
\node[circle, scale=0.75, fill] (tid0) at (3.75,1.5){};
\node[circle, scale=0.75, fill] (tid1) at (2.25,3){};
\node[circle, scale=0.75, fill, task_scheduled] (tid4) at (0.75,4.5){};
\node[circle, scale=0.75, fill, task_scheduled] (tid5) at (2.25,4.5){};
\node[circle, scale=0.75, fill] (tid6) at (3.75,4.5){};
\draw[](tid1) -- (tid4);
\draw[](tid1) -- (tid5);
\draw[](tid1) -- (tid6);
\node[circle, scale=0.75, fill] (tid2) at (5.25,3){};
\node[circle, scale=0.75, fill] (tid7) at (5.25,4.5){};
\draw[](tid2) -- (tid7);
\node[circle, scale=0.75, fill] (tid3) at (6.75,3){};
\draw[](tid0) -- (tid1);
\draw[](tid0) -- (tid2);
\draw[](tid0) -- (tid3);
\end{tikzpicture}
\nodepart{three}
\footnotesize{5.125}
\nodepart{four}
\footnotesize{$50\:50$}
};
 \\ 
\node[draw=black, rectangle split,  rectangle split parts=4] (sn0x22b3cf0){
\footnotesize{1.25}
\nodepart{two}
\begin{tikzpicture}[scale=.2]
\node[circle, scale=0.75, fill] (tid0) at (3.75,1.5){};
\node[circle, scale=0.75, fill] (tid1) at (2.25,3){};
\node[circle, scale=0.75, fill, task_scheduled] (tid4) at (0.75,4.5){};
\node[circle, scale=0.75, fill] (tid5) at (2.25,4.5){};
\node[circle, scale=0.75, fill] (tid6) at (3.75,4.5){};
\draw[](tid1) -- (tid4);
\draw[](tid1) -- (tid5);
\draw[](tid1) -- (tid6);
\node[circle, scale=0.75, fill] (tid2) at (5.25,3){};
\node[circle, scale=0.75, fill, task_scheduled] (tid7) at (5.25,4.5){};
\draw[](tid2) -- (tid7);
\node[circle, scale=0.75, fill] (tid3) at (6.75,3){};
\draw[](tid0) -- (tid1);
\draw[](tid0) -- (tid2);
\draw[](tid0) -- (tid3);
\end{tikzpicture}
\nodepart{three}
\footnotesize{5.125}
\nodepart{four}
\footnotesize{$50\:50$}
};
 \\ 
\node[draw=black, rectangle split,  rectangle split parts=4] (sn0x22b2b90){
\footnotesize{2.08333}
\nodepart{two}
\begin{tikzpicture}[scale=.2]
\node[circle, scale=0.75, fill] (tid0) at (3,1.5){};
\node[circle, scale=0.75, fill] (tid1) at (1.5,3){};
\node[circle, scale=0.75, fill, task_scheduled] (tid4) at (0.75,4.5){};
\node[circle, scale=0.75, fill] (tid5) at (2.25,4.5){};
\draw[](tid1) -- (tid4);
\draw[](tid1) -- (tid5);
\node[circle, scale=0.75, fill] (tid2) at (3.75,3){};
\node[circle, scale=0.75, fill] (tid6) at (3.75,4.5){};
\node[circle, scale=0.75, fill, task_scheduled] (tid7) at (3.75,6){};
\draw[](tid6) -- (tid7);
\draw[](tid2) -- (tid6);
\node[circle, scale=0.75, fill] (tid3) at (5.25,3){};
\draw[](tid0) -- (tid1);
\draw[](tid0) -- (tid2);
\draw[](tid0) -- (tid3);
\end{tikzpicture}
\nodepart{three}
\footnotesize{5.1875}
\nodepart{four}
\footnotesize{$50\:25\:25$}
};
 \\ 
\node[draw=black, rectangle split,  rectangle split parts=4] (sn0x22b66d0){
\footnotesize{6.19792}
\nodepart{two}
\begin{tikzpicture}[scale=.2]
\node[circle, scale=0.75, fill] (tid0) at (3.75,1.5){};
\node[circle, scale=0.75, fill] (tid1) at (1.5,3){};
\node[circle, scale=0.75, fill, task_scheduled] (tid4) at (0.75,4.5){};
\node[circle, scale=0.75, fill, task_scheduled] (tid5) at (2.25,4.5){};
\draw[](tid1) -- (tid4);
\draw[](tid1) -- (tid5);
\node[circle, scale=0.75, fill] (tid2) at (4.5,3){};
\node[circle, scale=0.75, fill] (tid6) at (3.75,4.5){};
\node[circle, scale=0.75, fill] (tid7) at (5.25,4.5){};
\draw[](tid2) -- (tid6);
\draw[](tid2) -- (tid7);
\node[circle, scale=0.75, fill] (tid3) at (6.75,3){};
\draw[](tid0) -- (tid1);
\draw[](tid0) -- (tid2);
\draw[](tid0) -- (tid3);
\end{tikzpicture}
\nodepart{three}
\footnotesize{5.125}
\nodepart{four}
\footnotesize{$1$}
};
 \\ 
\node[draw=black, rectangle split,  rectangle split parts=4] (sn0x22b6c20){
\footnotesize{12.3958}
\nodepart{two}
\begin{tikzpicture}[scale=.2]
\node[circle, scale=0.75, fill] (tid0) at (3.75,1.5){};
\node[circle, scale=0.75, fill] (tid1) at (1.5,3){};
\node[circle, scale=0.75, fill, task_scheduled] (tid4) at (0.75,4.5){};
\node[circle, scale=0.75, fill] (tid5) at (2.25,4.5){};
\draw[](tid1) -- (tid4);
\draw[](tid1) -- (tid5);
\node[circle, scale=0.75, fill] (tid2) at (4.5,3){};
\node[circle, scale=0.75, fill, task_scheduled] (tid6) at (3.75,4.5){};
\node[circle, scale=0.75, fill] (tid7) at (5.25,4.5){};
\draw[](tid2) -- (tid6);
\draw[](tid2) -- (tid7);
\node[circle, scale=0.75, fill] (tid3) at (6.75,3){};
\draw[](tid0) -- (tid1);
\draw[](tid0) -- (tid2);
\draw[](tid0) -- (tid3);
\end{tikzpicture}
\nodepart{three}
\footnotesize{5.125}
\nodepart{four}
\footnotesize{$50\:50$}
};
 \\ 
\node[draw=black, rectangle split,  rectangle split parts=4] (sn0x22ade60){
\footnotesize{11.1806}
\nodepart{two}
\begin{tikzpicture}[scale=.2]
\node[circle, scale=0.75, fill] (tid0) at (3,1.5){};
\node[circle, scale=0.75, fill] (tid1) at (1.5,3){};
\node[circle, scale=0.75, fill, task_scheduled] (tid4) at (0.75,4.5){};
\node[circle, scale=0.75, fill, task_scheduled] (tid5) at (2.25,4.5){};
\draw[](tid1) -- (tid4);
\draw[](tid1) -- (tid5);
\node[circle, scale=0.75, fill] (tid2) at (3.75,3){};
\node[circle, scale=0.75, fill] (tid6) at (3.75,4.5){};
\draw[](tid2) -- (tid6);
\node[circle, scale=0.75, fill] (tid3) at (5.25,3){};
\node[circle, scale=0.75, fill] (tid7) at (5.25,4.5){};
\draw[](tid3) -- (tid7);
\draw[](tid0) -- (tid1);
\draw[](tid0) -- (tid2);
\draw[](tid0) -- (tid3);
\end{tikzpicture}
\nodepart{three}
\footnotesize{5.125}
\nodepart{four}
\footnotesize{$1$}
};
 \\ 
\node[draw=black, rectangle split,  rectangle split parts=4] (sn0x22ae720){
\footnotesize{43.4375}
\nodepart{two}
\begin{tikzpicture}[scale=.2]
\node[circle, scale=0.75, fill] (tid0) at (3,1.5){};
\node[circle, scale=0.75, fill] (tid1) at (1.5,3){};
\node[circle, scale=0.75, fill, task_scheduled] (tid4) at (0.75,4.5){};
\node[circle, scale=0.75, fill] (tid5) at (2.25,4.5){};
\draw[](tid1) -- (tid4);
\draw[](tid1) -- (tid5);
\node[circle, scale=0.75, fill] (tid2) at (3.75,3){};
\node[circle, scale=0.75, fill, task_scheduled] (tid6) at (3.75,4.5){};
\draw[](tid2) -- (tid6);
\node[circle, scale=0.75, fill] (tid3) at (5.25,3){};
\node[circle, scale=0.75, fill] (tid7) at (5.25,4.5){};
\draw[](tid3) -- (tid7);
\draw[](tid0) -- (tid1);
\draw[](tid0) -- (tid2);
\draw[](tid0) -- (tid3);
\end{tikzpicture}
\nodepart{three}
\footnotesize{5.125}
\nodepart{four}
\footnotesize{$25\:25\:50$}
};
 \\ 
\node[draw=black, rectangle split,  rectangle split parts=4] (sn0x22b2d60){
\footnotesize{10.5382}
\nodepart{two}
\begin{tikzpicture}[scale=.2]
\node[circle, scale=0.75, fill] (tid0) at (3,1.5){};
\node[circle, scale=0.75, fill] (tid1) at (1.5,3){};
\node[circle, scale=0.75, fill] (tid4) at (0.75,4.5){};
\node[circle, scale=0.75, fill] (tid5) at (2.25,4.5){};
\draw[](tid1) -- (tid4);
\draw[](tid1) -- (tid5);
\node[circle, scale=0.75, fill] (tid2) at (3.75,3){};
\node[circle, scale=0.75, fill, task_scheduled] (tid6) at (3.75,4.5){};
\draw[](tid2) -- (tid6);
\node[circle, scale=0.75, fill] (tid3) at (5.25,3){};
\node[circle, scale=0.75, fill, task_scheduled] (tid7) at (5.25,4.5){};
\draw[](tid3) -- (tid7);
\draw[](tid0) -- (tid1);
\draw[](tid0) -- (tid2);
\draw[](tid0) -- (tid3);
\end{tikzpicture}
\nodepart{three}
\footnotesize{5.125}
\nodepart{four}
\footnotesize{$1$}
};
 \\ 
\\
};
\end{scope}
\begin{scope}[yshift=\leveltopIIIII cm, anchor = center]
\matrix (line5)[row sep=0.5cm] {
\node[draw=black, rectangle split,  rectangle split parts=4] (sn0x22bb120){
\footnotesize{1.5625}
\nodepart{two}
\begin{tikzpicture}[scale=.2]
\node[circle, scale=0.75, fill] (tid0) at (3,1.5){};
\node[circle, scale=0.75, fill] (tid1) at (1.5,3){};
\node[circle, scale=0.75, fill] (tid4) at (0.75,4.5){};
\node[circle, scale=0.75, fill, task_scheduled] (tid6) at (0.75,6){};
\draw[](tid4) -- (tid6);
\node[circle, scale=0.75, fill, task_scheduled] (tid5) at (2.25,4.5){};
\draw[](tid1) -- (tid4);
\draw[](tid1) -- (tid5);
\node[circle, scale=0.75, fill] (tid2) at (3.75,3){};
\node[circle, scale=0.75, fill] (tid3) at (5.25,3){};
\draw[](tid0) -- (tid1);
\draw[](tid0) -- (tid2);
\draw[](tid0) -- (tid3);
\end{tikzpicture}
\nodepart{three}
\footnotesize{4.75}
\nodepart{four}
\footnotesize{$50\:50$}
};
 \\ 
\node[draw=black, rectangle split,  rectangle split parts=4] (sn0x22ae570){
\footnotesize{4.6875}
\nodepart{two}
\begin{tikzpicture}[scale=.2]
\node[circle, scale=0.75, fill] (tid0) at (2.25,1.5){};
\node[circle, scale=0.75, fill] (tid1) at (0.75,3){};
\node[circle, scale=0.75, fill] (tid4) at (0.75,4.5){};
\node[circle, scale=0.75, fill, task_scheduled] (tid6) at (0.75,6){};
\draw[](tid4) -- (tid6);
\draw[](tid1) -- (tid4);
\node[circle, scale=0.75, fill] (tid2) at (2.25,3){};
\node[circle, scale=0.75, fill, task_scheduled] (tid5) at (2.25,4.5){};
\draw[](tid2) -- (tid5);
\node[circle, scale=0.75, fill] (tid3) at (3.75,3){};
\draw[](tid0) -- (tid1);
\draw[](tid0) -- (tid2);
\draw[](tid0) -- (tid3);
\end{tikzpicture}
\nodepart{three}
\footnotesize{4.75}
\nodepart{four}
\footnotesize{$50\:50$}
};
 \\ 
\node[draw=black, rectangle split,  rectangle split parts=4] (sn0x22b40d0){
\footnotesize{0.625}
\nodepart{two}
\begin{tikzpicture}[scale=.2]
\node[circle, scale=0.75, fill] (tid0) at (3.75,1.5){};
\node[circle, scale=0.75, fill] (tid1) at (2.25,3){};
\node[circle, scale=0.75, fill, task_scheduled] (tid4) at (0.75,4.5){};
\node[circle, scale=0.75, fill, task_scheduled] (tid5) at (2.25,4.5){};
\node[circle, scale=0.75, fill] (tid6) at (3.75,4.5){};
\draw[](tid1) -- (tid4);
\draw[](tid1) -- (tid5);
\draw[](tid1) -- (tid6);
\node[circle, scale=0.75, fill] (tid2) at (5.25,3){};
\node[circle, scale=0.75, fill] (tid3) at (6.75,3){};
\draw[](tid0) -- (tid1);
\draw[](tid0) -- (tid2);
\draw[](tid0) -- (tid3);
\end{tikzpicture}
\nodepart{three}
\footnotesize{4.625}
\nodepart{four}
\footnotesize{$1$}
};
 \\ 
\node[draw=black, rectangle split,  rectangle split parts=4] (sn0x22b14e0){
\footnotesize{19.6094}
\nodepart{two}
\begin{tikzpicture}[scale=.2]
\node[circle, scale=0.75, fill] (tid0) at (3,1.5){};
\node[circle, scale=0.75, fill] (tid1) at (1.5,3){};
\node[circle, scale=0.75, fill, task_scheduled] (tid4) at (0.75,4.5){};
\node[circle, scale=0.75, fill, task_scheduled] (tid5) at (2.25,4.5){};
\draw[](tid1) -- (tid4);
\draw[](tid1) -- (tid5);
\node[circle, scale=0.75, fill] (tid2) at (3.75,3){};
\node[circle, scale=0.75, fill] (tid6) at (3.75,4.5){};
\draw[](tid2) -- (tid6);
\node[circle, scale=0.75, fill] (tid3) at (5.25,3){};
\draw[](tid0) -- (tid1);
\draw[](tid0) -- (tid2);
\draw[](tid0) -- (tid3);
\end{tikzpicture}
\nodepart{three}
\footnotesize{4.625}
\nodepart{four}
\footnotesize{$1$}
};
 \\ 
\node[draw=black, rectangle split,  rectangle split parts=4] (sn0x22b1ca0){
\footnotesize{38.533}
\nodepart{two}
\begin{tikzpicture}[scale=.2]
\node[circle, scale=0.75, fill] (tid0) at (3,1.5){};
\node[circle, scale=0.75, fill] (tid1) at (1.5,3){};
\node[circle, scale=0.75, fill, task_scheduled] (tid4) at (0.75,4.5){};
\node[circle, scale=0.75, fill] (tid5) at (2.25,4.5){};
\draw[](tid1) -- (tid4);
\draw[](tid1) -- (tid5);
\node[circle, scale=0.75, fill] (tid2) at (3.75,3){};
\node[circle, scale=0.75, fill, task_scheduled] (tid6) at (3.75,4.5){};
\draw[](tid2) -- (tid6);
\node[circle, scale=0.75, fill] (tid3) at (5.25,3){};
\draw[](tid0) -- (tid1);
\draw[](tid0) -- (tid2);
\draw[](tid0) -- (tid3);
\end{tikzpicture}
\nodepart{three}
\footnotesize{4.625}
\nodepart{four}
\footnotesize{$50\:50$}
};
 \\ 
\node[draw=black, rectangle split,  rectangle split parts=4] (sn0x22af010){
\footnotesize{34.9826}
\nodepart{two}
\begin{tikzpicture}[scale=.2]
\node[circle, scale=0.75, fill] (tid0) at (2.25,1.5){};
\node[circle, scale=0.75, fill] (tid1) at (0.75,3){};
\node[circle, scale=0.75, fill, task_scheduled] (tid4) at (0.75,4.5){};
\draw[](tid1) -- (tid4);
\node[circle, scale=0.75, fill] (tid2) at (2.25,3){};
\node[circle, scale=0.75, fill, task_scheduled] (tid5) at (2.25,4.5){};
\draw[](tid2) -- (tid5);
\node[circle, scale=0.75, fill] (tid3) at (3.75,3){};
\node[circle, scale=0.75, fill] (tid6) at (3.75,4.5){};
\draw[](tid3) -- (tid6);
\draw[](tid0) -- (tid1);
\draw[](tid0) -- (tid2);
\draw[](tid0) -- (tid3);
\end{tikzpicture}
\nodepart{three}
\footnotesize{4.625}
\nodepart{four}
\footnotesize{$1$}
};
 \\ 
\\
};
\end{scope}
\begin{scope}[yshift=\leveltopIIIIII cm, anchor = center]
\matrix (line6)[row sep=0.5cm] {
\node[draw=black, rectangle split,  rectangle split parts=4] (sn0x22aecc0){
\footnotesize{3.125}
\nodepart{two}
\begin{tikzpicture}[scale=.2]
\node[circle, scale=0.75, fill] (tid0) at (2.25,1.5){};
\node[circle, scale=0.75, fill] (tid1) at (0.75,3){};
\node[circle, scale=0.75, fill] (tid4) at (0.75,4.5){};
\node[circle, scale=0.75, fill, task_scheduled] (tid5) at (0.75,6){};
\draw[](tid4) -- (tid5);
\draw[](tid1) -- (tid4);
\node[circle, scale=0.75, fill, task_scheduled] (tid2) at (2.25,3){};
\node[circle, scale=0.75, fill] (tid3) at (3.75,3){};
\draw[](tid0) -- (tid1);
\draw[](tid0) -- (tid2);
\draw[](tid0) -- (tid3);
\end{tikzpicture}
\nodepart{three}
\footnotesize{4.375}
\nodepart{four}
\footnotesize{$50\:50$}
};
 \\ 
\node[draw=black, rectangle split,  rectangle split parts=4] (sn0x22b21c0){
\footnotesize{20.6727}
\nodepart{two}
\begin{tikzpicture}[scale=.2]
\node[circle, scale=0.75, fill] (tid0) at (3,1.5){};
\node[circle, scale=0.75, fill] (tid1) at (1.5,3){};
\node[circle, scale=0.75, fill, task_scheduled] (tid4) at (0.75,4.5){};
\node[circle, scale=0.75, fill, task_scheduled] (tid5) at (2.25,4.5){};
\draw[](tid1) -- (tid4);
\draw[](tid1) -- (tid5);
\node[circle, scale=0.75, fill] (tid2) at (3.75,3){};
\node[circle, scale=0.75, fill] (tid3) at (5.25,3){};
\draw[](tid0) -- (tid1);
\draw[](tid0) -- (tid2);
\draw[](tid0) -- (tid3);
\end{tikzpicture}
\nodepart{three}
\footnotesize{4.125}
\nodepart{four}
\footnotesize{$1$}
};
 \\ 
\node[draw=black, rectangle split,  rectangle split parts=4] (sn0x22af160){
\footnotesize{76.2023}
\nodepart{two}
\begin{tikzpicture}[scale=.2]
\node[circle, scale=0.75, fill] (tid0) at (2.25,1.5){};
\node[circle, scale=0.75, fill] (tid1) at (0.75,3){};
\node[circle, scale=0.75, fill, task_scheduled] (tid4) at (0.75,4.5){};
\draw[](tid1) -- (tid4);
\node[circle, scale=0.75, fill] (tid2) at (2.25,3){};
\node[circle, scale=0.75, fill, task_scheduled] (tid5) at (2.25,4.5){};
\draw[](tid2) -- (tid5);
\node[circle, scale=0.75, fill] (tid3) at (3.75,3){};
\draw[](tid0) -- (tid1);
\draw[](tid0) -- (tid2);
\draw[](tid0) -- (tid3);
\end{tikzpicture}
\nodepart{three}
\footnotesize{4.125}
\nodepart{four}
\footnotesize{$1$}
};
 \\ 
\\
};
\end{scope}
\begin{scope}[yshift=\leveltopIIIIIII cm, anchor = center]
\matrix (line7)[row sep=0.5cm] {
\node[draw=black, rectangle split,  rectangle split parts=4] (sn0x22af6c0){
\footnotesize{1.5625}
\nodepart{two}
\begin{tikzpicture}[scale=.2]
\node[circle, scale=0.75, fill] (tid0) at (1.5,1.5){};
\node[circle, scale=0.75, fill] (tid1) at (0.75,3){};
\node[circle, scale=0.75, fill] (tid3) at (0.75,4.5){};
\node[circle, scale=0.75, fill, task_scheduled] (tid4) at (0.75,6){};
\draw[](tid3) -- (tid4);
\draw[](tid1) -- (tid3);
\node[circle, scale=0.75, fill, task_scheduled] (tid2) at (2.25,3){};
\draw[](tid0) -- (tid1);
\draw[](tid0) -- (tid2);
\end{tikzpicture}
\nodepart{three}
\footnotesize{4.125}
\nodepart{four}
\footnotesize{$50\:50$}
};
 \\ 
\node[draw=black, rectangle split,  rectangle split parts=4] (sn0x22afb50){
\footnotesize{98.4375}
\nodepart{two}
\begin{tikzpicture}[scale=.2]
\node[circle, scale=0.75, fill] (tid0) at (2.25,1.5){};
\node[circle, scale=0.75, fill] (tid1) at (0.75,3){};
\node[circle, scale=0.75, fill, task_scheduled] (tid4) at (0.75,4.5){};
\draw[](tid1) -- (tid4);
\node[circle, scale=0.75, fill, task_scheduled] (tid2) at (2.25,3){};
\node[circle, scale=0.75, fill] (tid3) at (3.75,3){};
\draw[](tid0) -- (tid1);
\draw[](tid0) -- (tid2);
\draw[](tid0) -- (tid3);
\end{tikzpicture}
\nodepart{three}
\footnotesize{3.625}
\nodepart{four}
\footnotesize{$50\:50$}
};
 \\ 
\\
};
\end{scope}
\draw (sn0x22a97d0.east) -- (sn0x22b7700.west);
\draw (sn0x22a97d0.east) -- (sn0x22b8ae0.west);
\draw (sn0x22a97d0.east) -- (sn0x22b8c20.west);
\draw (sn0x22a97d0.east) -- (sn0x22b8400.west);
\draw (sn0x22a97d0.east) -- (sn0x22ab2c0.west);
\draw (sn0x22a97d0.east) -- (sn0x22b9170.west);
\draw (sn0x22b7700.east) -- (sn0x22b9da0.west);
\draw (sn0x22b7700.east) -- (sn0x22ba8b0.west);
\draw (sn0x22b7700.east) -- (sn0x22b54f0.west);
\draw (sn0x22b7700.east) -- (sn0x22b5d10.west);
\draw (sn0x22b7700.east) -- (sn0x22b6220.west);
\draw (sn0x22b8ae0.east) -- (sn0x22ac4a0.west);
\draw (sn0x22b8ae0.east) -- (sn0x22acd50.west);
\draw (sn0x22b8ae0.east) -- (sn0x22b5d10.west);
\draw (sn0x22b8ae0.east) -- (sn0x22b54f0.west);
\draw (sn0x22b8ae0.east) -- (sn0x22b6220.west);
\draw (sn0x22b8c20.east) -- (sn0x22bb880.west);
\draw (sn0x22b8c20.east) -- (sn0x22bc030.west);
\draw (sn0x22b8c20.east) -- (sn0x22b6220.west);
\draw (sn0x22b8400.east) -- (sn0x22b5d10.west);
\draw (sn0x22b8400.east) -- (sn0x22b54f0.west);
\draw (sn0x22b8400.east) -- (sn0x22b6220.west);
\draw (sn0x22ab2c0.east) -- (sn0x22b54f0.west);
\draw (sn0x22ab2c0.east) -- (sn0x22b5d10.west);
\draw (sn0x22ab2c0.east) -- (sn0x22b6220.west);
\draw (sn0x22ab2c0.east) -- (sn0x22ac710.west);
\draw (sn0x22ab2c0.east) -- (sn0x22ada30.west);
\draw (sn0x22b9170.east) -- (sn0x22b6220.west);
\draw (sn0x22b9170.east) -- (sn0x22bc5d0.west);
\draw (sn0x22b9170.east) -- (sn0x22b7500.west);
\draw (sn0x22bb880.east) -- (sn0x22b2b90.west);
\draw (sn0x22bb880.east) -- (sn0x22b66d0.west);
\draw (sn0x22bb880.east) -- (sn0x22b6c20.west);
\draw (sn0x22bc030.east) -- (sn0x22baab0.west);
\draw (sn0x22bc030.east) -- (sn0x22bafa0.west);
\draw (sn0x22bc030.east) -- (sn0x22b6c20.west);
\draw (sn0x22bc030.east) -- (sn0x22b66d0.west);
\draw (sn0x22b9da0.east) -- (sn0x22ad490.west);
\draw (sn0x22b9da0.east) -- (sn0x22ade60.west);
\draw (sn0x22b9da0.east) -- (sn0x22ae720.west);
\draw (sn0x22ba8b0.east) -- (sn0x22baab0.west);
\draw (sn0x22ba8b0.east) -- (sn0x22bafa0.west);
\draw (sn0x22ba8b0.east) -- (sn0x22ae720.west);
\draw (sn0x22ba8b0.east) -- (sn0x22b2d60.west);
\draw (sn0x22bc5d0.east) -- (sn0x22b66d0.west);
\draw (sn0x22bc5d0.east) -- (sn0x22b6c20.west);
\draw (sn0x22b7500.east) -- (sn0x22b6c20.west);
\draw (sn0x22b7500.east) -- (sn0x22b66d0.west);
\draw (sn0x22b7500.east) -- (sn0x22b34f0.west);
\draw (sn0x22b7500.east) -- (sn0x22b3cf0.west);
\draw (sn0x22ac710.east) -- (sn0x22ade60.west);
\draw (sn0x22ac710.east) -- (sn0x22ae720.west);
\draw (sn0x22ada30.east) -- (sn0x22ae720.west);
\draw (sn0x22ada30.east) -- (sn0x22b2d60.west);
\draw (sn0x22ada30.east) -- (sn0x22b34f0.west);
\draw (sn0x22ada30.east) -- (sn0x22b3cf0.west);
\draw (sn0x22ac4a0.east) -- (sn0x22ad490.west);
\draw (sn0x22ac4a0.east) -- (sn0x22ade60.west);
\draw (sn0x22ac4a0.east) -- (sn0x22ae720.west);
\draw (sn0x22acd50.east) -- (sn0x22b2b90.west);
\draw (sn0x22acd50.east) -- (sn0x22ae720.west);
\draw (sn0x22acd50.east) -- (sn0x22b2d60.west);
\draw (sn0x22b54f0.east) -- (sn0x22ae720.west);
\draw (sn0x22b54f0.east) -- (sn0x22ade60.west);
\draw (sn0x22b5d10.east) -- (sn0x22ae720.west);
\draw (sn0x22b5d10.east) -- (sn0x22b2d60.west);
\draw (sn0x22b6220.east) -- (sn0x22b2d60.west);
\draw (sn0x22b6220.east) -- (sn0x22ae720.west);
\draw (sn0x22b6220.east) -- (sn0x22b66d0.west);
\draw (sn0x22b6220.east) -- (sn0x22b6c20.west);
\draw (sn0x22baab0.east) -- (sn0x22ae570.west);
\draw (sn0x22baab0.east) -- (sn0x22b14e0.west);
\draw (sn0x22baab0.east) -- (sn0x22b1ca0.west);
\draw (sn0x22bafa0.east) -- (sn0x22bb120.west);
\draw (sn0x22bafa0.east) -- (sn0x22b1ca0.west);
\draw (sn0x22ad490.east) -- (sn0x22ae570.west);
\draw (sn0x22ad490.east) -- (sn0x22af010.west);
\draw (sn0x22b34f0.east) -- (sn0x22b14e0.west);
\draw (sn0x22b34f0.east) -- (sn0x22b1ca0.west);
\draw (sn0x22b3cf0.east) -- (sn0x22b1ca0.west);
\draw (sn0x22b3cf0.east) -- (sn0x22b40d0.west);
\draw (sn0x22b2b90.east) -- (sn0x22ae570.west);
\draw (sn0x22b2b90.east) -- (sn0x22b14e0.west);
\draw (sn0x22b2b90.east) -- (sn0x22b1ca0.west);
\draw (sn0x22b66d0.east) -- (sn0x22b1ca0.west);
\draw (sn0x22b6c20.east) -- (sn0x22b1ca0.west);
\draw (sn0x22b6c20.east) -- (sn0x22b14e0.west);
\draw (sn0x22ade60.east) -- (sn0x22af010.west);
\draw (sn0x22ae720.east) -- (sn0x22af010.west);
\draw (sn0x22ae720.east) -- (sn0x22b14e0.west);
\draw (sn0x22ae720.east) -- (sn0x22b1ca0.west);
\draw (sn0x22b2d60.east) -- (sn0x22b1ca0.west);
\draw (sn0x22bb120.east) -- (sn0x22aecc0.west);
\draw (sn0x22bb120.east) -- (sn0x22b21c0.west);
\draw (sn0x22ae570.east) -- (sn0x22aecc0.west);
\draw (sn0x22ae570.east) -- (sn0x22af160.west);
\draw (sn0x22b40d0.east) -- (sn0x22b21c0.west);
\draw (sn0x22b14e0.east) -- (sn0x22af160.west);
\draw (sn0x22b1ca0.east) -- (sn0x22af160.west);
\draw (sn0x22b1ca0.east) -- (sn0x22b21c0.west);
\draw (sn0x22af010.east) -- (sn0x22af160.west);
\draw (sn0x22aecc0.east) -- (sn0x22af6c0.west);
\draw (sn0x22aecc0.east) -- (sn0x22afb50.west);
\draw (sn0x22b21c0.east) -- (sn0x22afb50.west);
\draw (sn0x22af160.east) -- (sn0x22afb50.west);
\end{tikzpicture}

%%% Local Variables:
%%% TeX-master: "thesis/thesis.tex"
%%% End: 
\renewcommand{\leveltopI}{-10cm + \leveltop}
\renewcommand{\leveltopII}{-10cm + \leveltopI}
\renewcommand{\leveltopIII}{-10cm + \leveltopII}
\renewcommand{\leveltopIIII}{-10cm + \leveltopIII}
\renewcommand{\leveltopIIIII}{-10cm + \leveltopIIII}
\renewcommand{\leveltopIIIIII}{-10cm + \leveltopIIIII}
\renewcommand{\leveltopIIIIIII}{-10cm + \leveltopIIIIII}
\renewcommand{\leveltopIIIIIIII}{-10cm + \leveltopIIIIIII}
\renewcommand{\leveltopIIIIIIIII}{-10cm + \leveltopIIIIIIII}
\renewcommand{\leveltopIIIIIIIIII}{-10cm + \leveltopIIIIIIIII}
\renewcommand{\leveltopIIIIIIIIIII}{-10cm + \leveltopIIIIIIIIII}
\begin{tikzpicture}[scale=.2, anchor=south, rotate=90]
\begin{scope}[yshift=\leveltopI cm, anchor = center]
\matrix (line1)[row sep=0.5cm] {
\node[draw=black, rectangle split,  rectangle split parts=4] (sn0x22aa490){
\footnotesize{100}
\nodepart{two}
\begin{tikzpicture}[scale=.2]
\node[circle, scale=0.75, fill] (tid0) at (4.5,1.5){};
\node[circle, scale=0.75, fill] (tid1) at (2.25,3){};
\node[circle, scale=0.75, fill] (tid4) at (0.75,4.5){};
\node[circle, scale=0.75, fill] (tid5) at (2.25,4.5){};
\node[circle, scale=0.75, fill] (tid6) at (3.75,4.5){};
\draw[](tid1) -- (tid4);
\draw[](tid1) -- (tid5);
\draw[](tid1) -- (tid6);
\node[circle, scale=0.75, fill] (tid2) at (6,3){};
\node[circle, scale=0.75, fill] (tid7) at (5.25,4.5){};
\node[circle, scale=0.75, fill, task_scheduled] (tid10) at (5.25,6){};
\draw[](tid7) -- (tid10);
\node[circle, scale=0.75, fill] (tid8) at (6.75,4.5){};
\draw[](tid2) -- (tid7);
\draw[](tid2) -- (tid8);
\node[circle, scale=0.75, fill] (tid3) at (8.25,3){};
\node[circle, scale=0.75, fill, task_scheduled] (tid9) at (8.25,4.5){};
\draw[](tid3) -- (tid9);
\draw[](tid0) -- (tid1);
\draw[](tid0) -- (tid2);
\draw[](tid0) -- (tid3);
\end{tikzpicture}
\nodepart{three}
\footnotesize{6.63281}
\nodepart{four}
\footnotesize{$38\:12\:30\:20$}
};
 \\ 
\\
};
\end{scope}
\begin{scope}[yshift=\leveltopII cm, anchor = center]
\matrix (line2)[row sep=0.5cm] {
\node[draw=black, rectangle split,  rectangle split parts=4] (sn0x22bcc60){
\footnotesize{37.5}
\nodepart{two}
\begin{tikzpicture}[scale=.2]
\node[circle, scale=0.75, fill] (tid0) at (4.5,1.5){};
\node[circle, scale=0.75, fill] (tid1) at (2.25,3){};
\node[circle, scale=0.75, fill, task_scheduled] (tid4) at (0.75,4.5){};
\node[circle, scale=0.75, fill] (tid5) at (2.25,4.5){};
\node[circle, scale=0.75, fill] (tid6) at (3.75,4.5){};
\draw[](tid1) -- (tid4);
\draw[](tid1) -- (tid5);
\draw[](tid1) -- (tid6);
\node[circle, scale=0.75, fill] (tid2) at (6,3){};
\node[circle, scale=0.75, fill] (tid7) at (5.25,4.5){};
\node[circle, scale=0.75, fill, task_scheduled] (tid9) at (5.25,6){};
\draw[](tid7) -- (tid9);
\node[circle, scale=0.75, fill] (tid8) at (6.75,4.5){};
\draw[](tid2) -- (tid7);
\draw[](tid2) -- (tid8);
\node[circle, scale=0.75, fill] (tid3) at (8.25,3){};
\draw[](tid0) -- (tid1);
\draw[](tid0) -- (tid2);
\draw[](tid0) -- (tid3);
\end{tikzpicture}
\nodepart{three}
\footnotesize{6.14062}
\nodepart{four}
\footnotesize{$33\:17\:25\:25$}
};
 \\ 
\node[draw=black, rectangle split,  rectangle split parts=4] (sn0x22bcd60){
\footnotesize{12.5}
\nodepart{two}
\begin{tikzpicture}[scale=.2]
\node[circle, scale=0.75, fill] (tid0) at (4.5,1.5){};
\node[circle, scale=0.75, fill] (tid1) at (2.25,3){};
\node[circle, scale=0.75, fill] (tid4) at (0.75,4.5){};
\node[circle, scale=0.75, fill] (tid5) at (2.25,4.5){};
\node[circle, scale=0.75, fill] (tid6) at (3.75,4.5){};
\draw[](tid1) -- (tid4);
\draw[](tid1) -- (tid5);
\draw[](tid1) -- (tid6);
\node[circle, scale=0.75, fill] (tid2) at (6,3){};
\node[circle, scale=0.75, fill] (tid7) at (5.25,4.5){};
\node[circle, scale=0.75, fill, task_scheduled] (tid9) at (5.25,6){};
\draw[](tid7) -- (tid9);
\node[circle, scale=0.75, fill, task_scheduled] (tid8) at (6.75,4.5){};
\draw[](tid2) -- (tid7);
\draw[](tid2) -- (tid8);
\node[circle, scale=0.75, fill] (tid3) at (8.25,3){};
\draw[](tid0) -- (tid1);
\draw[](tid0) -- (tid2);
\draw[](tid0) -- (tid3);
\end{tikzpicture}
\nodepart{three}
\footnotesize{6.14062}
\nodepart{four}
\footnotesize{$50\:38\:12$}
};
 \\ 
\node[draw=black, rectangle split,  rectangle split parts=4] (sn0x22b9170){
\footnotesize{30}
\nodepart{two}
\begin{tikzpicture}[scale=.2]
\node[circle, scale=0.75, fill] (tid0) at (4.5,1.5){};
\node[circle, scale=0.75, fill] (tid1) at (2.25,3){};
\node[circle, scale=0.75, fill, task_scheduled] (tid4) at (0.75,4.5){};
\node[circle, scale=0.75, fill] (tid5) at (2.25,4.5){};
\node[circle, scale=0.75, fill] (tid6) at (3.75,4.5){};
\draw[](tid1) -- (tid4);
\draw[](tid1) -- (tid5);
\draw[](tid1) -- (tid6);
\node[circle, scale=0.75, fill] (tid2) at (6,3){};
\node[circle, scale=0.75, fill] (tid7) at (5.25,4.5){};
\node[circle, scale=0.75, fill] (tid8) at (6.75,4.5){};
\draw[](tid2) -- (tid7);
\draw[](tid2) -- (tid8);
\node[circle, scale=0.75, fill] (tid3) at (8.25,3){};
\node[circle, scale=0.75, fill, task_scheduled] (tid9) at (8.25,4.5){};
\draw[](tid3) -- (tid9);
\draw[](tid0) -- (tid1);
\draw[](tid0) -- (tid2);
\draw[](tid0) -- (tid3);
\end{tikzpicture}
\nodepart{three}
\footnotesize{6.125}
\nodepart{four}
\footnotesize{$25\:25\:50$}
};
 \\ 
\node[draw=black, rectangle split,  rectangle split parts=4] (sn0x22ab500){
\footnotesize{20}
\nodepart{two}
\begin{tikzpicture}[scale=.2]
\node[circle, scale=0.75, fill] (tid0) at (4.5,1.5){};
\node[circle, scale=0.75, fill] (tid1) at (2.25,3){};
\node[circle, scale=0.75, fill] (tid4) at (0.75,4.5){};
\node[circle, scale=0.75, fill] (tid5) at (2.25,4.5){};
\node[circle, scale=0.75, fill] (tid6) at (3.75,4.5){};
\draw[](tid1) -- (tid4);
\draw[](tid1) -- (tid5);
\draw[](tid1) -- (tid6);
\node[circle, scale=0.75, fill] (tid2) at (6,3){};
\node[circle, scale=0.75, fill, task_scheduled] (tid7) at (5.25,4.5){};
\node[circle, scale=0.75, fill] (tid8) at (6.75,4.5){};
\draw[](tid2) -- (tid7);
\draw[](tid2) -- (tid8);
\node[circle, scale=0.75, fill] (tid3) at (8.25,3){};
\node[circle, scale=0.75, fill, task_scheduled] (tid9) at (8.25,4.5){};
\draw[](tid3) -- (tid9);
\draw[](tid0) -- (tid1);
\draw[](tid0) -- (tid2);
\draw[](tid0) -- (tid3);
\end{tikzpicture}
\nodepart{three}
\footnotesize{6.125}
\nodepart{four}
\footnotesize{$38\:12\:38\:12$}
};
 \\ 
\\
};
\end{scope}
\begin{scope}[yshift=\leveltopIII cm, anchor = center]
\matrix (line3)[row sep=0.5cm] {
\node[draw=black, rectangle split,  rectangle split parts=4] (sn0x22bc030){
\footnotesize{12.5}
\nodepart{two}
\begin{tikzpicture}[scale=.2]
\node[circle, scale=0.75, fill] (tid0) at (3.75,1.5){};
\node[circle, scale=0.75, fill] (tid1) at (1.5,3){};
\node[circle, scale=0.75, fill] (tid4) at (0.75,4.5){};
\node[circle, scale=0.75, fill, task_scheduled] (tid8) at (0.75,6){};
\draw[](tid4) -- (tid8);
\node[circle, scale=0.75, fill] (tid5) at (2.25,4.5){};
\draw[](tid1) -- (tid4);
\draw[](tid1) -- (tid5);
\node[circle, scale=0.75, fill] (tid2) at (4.5,3){};
\node[circle, scale=0.75, fill, task_scheduled] (tid6) at (3.75,4.5){};
\node[circle, scale=0.75, fill] (tid7) at (5.25,4.5){};
\draw[](tid2) -- (tid6);
\draw[](tid2) -- (tid7);
\node[circle, scale=0.75, fill] (tid3) at (6.75,3){};
\draw[](tid0) -- (tid1);
\draw[](tid0) -- (tid2);
\draw[](tid0) -- (tid3);
\end{tikzpicture}
\nodepart{three}
\footnotesize{5.65625}
\nodepart{four}
\footnotesize{$25\:25\:33\:17$}
};
 \\ 
\node[draw=black, rectangle split,  rectangle split parts=4] (sn0x22bb880){
\footnotesize{6.25}
\nodepart{two}
\begin{tikzpicture}[scale=.2]
\node[circle, scale=0.75, fill] (tid0) at (3.75,1.5){};
\node[circle, scale=0.75, fill] (tid1) at (1.5,3){};
\node[circle, scale=0.75, fill] (tid4) at (0.75,4.5){};
\node[circle, scale=0.75, fill, task_scheduled] (tid8) at (0.75,6){};
\draw[](tid4) -- (tid8);
\node[circle, scale=0.75, fill, task_scheduled] (tid5) at (2.25,4.5){};
\draw[](tid1) -- (tid4);
\draw[](tid1) -- (tid5);
\node[circle, scale=0.75, fill] (tid2) at (4.5,3){};
\node[circle, scale=0.75, fill] (tid6) at (3.75,4.5){};
\node[circle, scale=0.75, fill] (tid7) at (5.25,4.5){};
\draw[](tid2) -- (tid6);
\draw[](tid2) -- (tid7);
\node[circle, scale=0.75, fill] (tid3) at (6.75,3){};
\draw[](tid0) -- (tid1);
\draw[](tid0) -- (tid2);
\draw[](tid0) -- (tid3);
\end{tikzpicture}
\nodepart{three}
\footnotesize{5.65625}
\nodepart{four}
\footnotesize{$50\:17\:33$}
};
 \\ 
\node[draw=black, rectangle split,  rectangle split parts=4] (sn0x22b4a40){
\footnotesize{6.25}
\nodepart{two}
\begin{tikzpicture}[scale=.2]
\node[circle, scale=0.75, fill] (tid0) at (3.75,1.5){};
\node[circle, scale=0.75, fill] (tid1) at (2.25,3){};
\node[circle, scale=0.75, fill, task_scheduled] (tid4) at (0.75,4.5){};
\node[circle, scale=0.75, fill] (tid5) at (2.25,4.5){};
\node[circle, scale=0.75, fill] (tid6) at (3.75,4.5){};
\draw[](tid1) -- (tid4);
\draw[](tid1) -- (tid5);
\draw[](tid1) -- (tid6);
\node[circle, scale=0.75, fill] (tid2) at (5.25,3){};
\node[circle, scale=0.75, fill] (tid7) at (5.25,4.5){};
\node[circle, scale=0.75, fill, task_scheduled] (tid8) at (5.25,6){};
\draw[](tid7) -- (tid8);
\draw[](tid2) -- (tid7);
\node[circle, scale=0.75, fill] (tid3) at (6.75,3){};
\draw[](tid0) -- (tid1);
\draw[](tid0) -- (tid2);
\draw[](tid0) -- (tid3);
\end{tikzpicture}
\nodepart{three}
\footnotesize{5.65625}
\nodepart{four}
\footnotesize{$33\:17\:50$}
};
 \\ 
\node[draw=black, rectangle split,  rectangle split parts=4] (sn0x22bc5d0){
\footnotesize{16.875}
\nodepart{two}
\begin{tikzpicture}[scale=.2]
\node[circle, scale=0.75, fill] (tid0) at (4.5,1.5){};
\node[circle, scale=0.75, fill] (tid1) at (2.25,3){};
\node[circle, scale=0.75, fill, task_scheduled] (tid4) at (0.75,4.5){};
\node[circle, scale=0.75, fill, task_scheduled] (tid5) at (2.25,4.5){};
\node[circle, scale=0.75, fill] (tid6) at (3.75,4.5){};
\draw[](tid1) -- (tid4);
\draw[](tid1) -- (tid5);
\draw[](tid1) -- (tid6);
\node[circle, scale=0.75, fill] (tid2) at (6,3){};
\node[circle, scale=0.75, fill] (tid7) at (5.25,4.5){};
\node[circle, scale=0.75, fill] (tid8) at (6.75,4.5){};
\draw[](tid2) -- (tid7);
\draw[](tid2) -- (tid8);
\node[circle, scale=0.75, fill] (tid3) at (8.25,3){};
\draw[](tid0) -- (tid1);
\draw[](tid0) -- (tid2);
\draw[](tid0) -- (tid3);
\end{tikzpicture}
\nodepart{three}
\footnotesize{5.625}
\nodepart{four}
\footnotesize{$33\:67$}
};
 \\ 
\node[draw=black, rectangle split,  rectangle split parts=4] (sn0x22b7500){
\footnotesize{29.0625}
\nodepart{two}
\begin{tikzpicture}[scale=.2]
\node[circle, scale=0.75, fill] (tid0) at (4.5,1.5){};
\node[circle, scale=0.75, fill] (tid1) at (2.25,3){};
\node[circle, scale=0.75, fill, task_scheduled] (tid4) at (0.75,4.5){};
\node[circle, scale=0.75, fill] (tid5) at (2.25,4.5){};
\node[circle, scale=0.75, fill] (tid6) at (3.75,4.5){};
\draw[](tid1) -- (tid4);
\draw[](tid1) -- (tid5);
\draw[](tid1) -- (tid6);
\node[circle, scale=0.75, fill] (tid2) at (6,3){};
\node[circle, scale=0.75, fill, task_scheduled] (tid7) at (5.25,4.5){};
\node[circle, scale=0.75, fill] (tid8) at (6.75,4.5){};
\draw[](tid2) -- (tid7);
\draw[](tid2) -- (tid8);
\node[circle, scale=0.75, fill] (tid3) at (8.25,3){};
\draw[](tid0) -- (tid1);
\draw[](tid0) -- (tid2);
\draw[](tid0) -- (tid3);
\end{tikzpicture}
\nodepart{three}
\footnotesize{5.625}
\nodepart{four}
\footnotesize{$33\:17\:33\:17$}
};
 \\ 
\node[draw=black, rectangle split,  rectangle split parts=4] (sn0x22b7aa0){
\footnotesize{4.0625}
\nodepart{two}
\begin{tikzpicture}[scale=.2]
\node[circle, scale=0.75, fill] (tid0) at (4.5,1.5){};
\node[circle, scale=0.75, fill] (tid1) at (2.25,3){};
\node[circle, scale=0.75, fill] (tid4) at (0.75,4.5){};
\node[circle, scale=0.75, fill] (tid5) at (2.25,4.5){};
\node[circle, scale=0.75, fill] (tid6) at (3.75,4.5){};
\draw[](tid1) -- (tid4);
\draw[](tid1) -- (tid5);
\draw[](tid1) -- (tid6);
\node[circle, scale=0.75, fill] (tid2) at (6,3){};
\node[circle, scale=0.75, fill, task_scheduled] (tid7) at (5.25,4.5){};
\node[circle, scale=0.75, fill, task_scheduled] (tid8) at (6.75,4.5){};
\draw[](tid2) -- (tid7);
\draw[](tid2) -- (tid8);
\node[circle, scale=0.75, fill] (tid3) at (8.25,3){};
\draw[](tid0) -- (tid1);
\draw[](tid0) -- (tid2);
\draw[](tid0) -- (tid3);
\end{tikzpicture}
\nodepart{three}
\footnotesize{5.625}
\nodepart{four}
\footnotesize{$1$}
};
 \\ 
\node[draw=black, rectangle split,  rectangle split parts=4] (sn0x22ada30){
\footnotesize{7.5}
\nodepart{two}
\begin{tikzpicture}[scale=.2]
\node[circle, scale=0.75, fill] (tid0) at (3.75,1.5){};
\node[circle, scale=0.75, fill] (tid1) at (2.25,3){};
\node[circle, scale=0.75, fill, task_scheduled] (tid4) at (0.75,4.5){};
\node[circle, scale=0.75, fill] (tid5) at (2.25,4.5){};
\node[circle, scale=0.75, fill] (tid6) at (3.75,4.5){};
\draw[](tid1) -- (tid4);
\draw[](tid1) -- (tid5);
\draw[](tid1) -- (tid6);
\node[circle, scale=0.75, fill] (tid2) at (5.25,3){};
\node[circle, scale=0.75, fill, task_scheduled] (tid7) at (5.25,4.5){};
\draw[](tid2) -- (tid7);
\node[circle, scale=0.75, fill] (tid3) at (6.75,3){};
\node[circle, scale=0.75, fill] (tid8) at (6.75,4.5){};
\draw[](tid3) -- (tid8);
\draw[](tid0) -- (tid1);
\draw[](tid0) -- (tid2);
\draw[](tid0) -- (tid3);
\end{tikzpicture}
\nodepart{three}
\footnotesize{5.625}
\nodepart{four}
\footnotesize{$33\:17\:33\:17$}
};
 \\ 
\node[draw=black, rectangle split,  rectangle split parts=4] (sn0x22b4be0){
\footnotesize{2.5}
\nodepart{two}
\begin{tikzpicture}[scale=.2]
\node[circle, scale=0.75, fill] (tid0) at (3.75,1.5){};
\node[circle, scale=0.75, fill] (tid1) at (2.25,3){};
\node[circle, scale=0.75, fill] (tid4) at (0.75,4.5){};
\node[circle, scale=0.75, fill] (tid5) at (2.25,4.5){};
\node[circle, scale=0.75, fill] (tid6) at (3.75,4.5){};
\draw[](tid1) -- (tid4);
\draw[](tid1) -- (tid5);
\draw[](tid1) -- (tid6);
\node[circle, scale=0.75, fill] (tid2) at (5.25,3){};
\node[circle, scale=0.75, fill, task_scheduled] (tid7) at (5.25,4.5){};
\draw[](tid2) -- (tid7);
\node[circle, scale=0.75, fill] (tid3) at (6.75,3){};
\node[circle, scale=0.75, fill, task_scheduled] (tid8) at (6.75,4.5){};
\draw[](tid3) -- (tid8);
\draw[](tid0) -- (tid1);
\draw[](tid0) -- (tid2);
\draw[](tid0) -- (tid3);
\end{tikzpicture}
\nodepart{three}
\footnotesize{5.625}
\nodepart{four}
\footnotesize{$1$}
};
 \\ 
\node[draw=black, rectangle split,  rectangle split parts=4] (sn0x22b6220){
\footnotesize{15}
\nodepart{two}
\begin{tikzpicture}[scale=.2]
\node[circle, scale=0.75, fill] (tid0) at (3.75,1.5){};
\node[circle, scale=0.75, fill] (tid1) at (1.5,3){};
\node[circle, scale=0.75, fill, task_scheduled] (tid4) at (0.75,4.5){};
\node[circle, scale=0.75, fill] (tid5) at (2.25,4.5){};
\draw[](tid1) -- (tid4);
\draw[](tid1) -- (tid5);
\node[circle, scale=0.75, fill] (tid2) at (4.5,3){};
\node[circle, scale=0.75, fill] (tid6) at (3.75,4.5){};
\node[circle, scale=0.75, fill] (tid7) at (5.25,4.5){};
\draw[](tid2) -- (tid6);
\draw[](tid2) -- (tid7);
\node[circle, scale=0.75, fill] (tid3) at (6.75,3){};
\node[circle, scale=0.75, fill, task_scheduled] (tid8) at (6.75,4.5){};
\draw[](tid3) -- (tid8);
\draw[](tid0) -- (tid1);
\draw[](tid0) -- (tid2);
\draw[](tid0) -- (tid3);
\end{tikzpicture}
\nodepart{three}
\footnotesize{5.625}
\nodepart{four}
\footnotesize{$17\:33\:17\:33$}
};
 \\ 
\\
};
\end{scope}
\begin{scope}[yshift=\leveltopIIII cm, anchor = center]
\matrix (line4)[row sep=0.5cm] {
\node[draw=black, rectangle split,  rectangle split parts=4] (sn0x22baab0){
\footnotesize{3.125}
\nodepart{two}
\begin{tikzpicture}[scale=.2]
\node[circle, scale=0.75, fill] (tid0) at (3,1.5){};
\node[circle, scale=0.75, fill] (tid1) at (1.5,3){};
\node[circle, scale=0.75, fill] (tid4) at (0.75,4.5){};
\node[circle, scale=0.75, fill, task_scheduled] (tid7) at (0.75,6){};
\draw[](tid4) -- (tid7);
\node[circle, scale=0.75, fill, task_scheduled] (tid5) at (2.25,4.5){};
\draw[](tid1) -- (tid4);
\draw[](tid1) -- (tid5);
\node[circle, scale=0.75, fill] (tid2) at (3.75,3){};
\node[circle, scale=0.75, fill] (tid6) at (3.75,4.5){};
\draw[](tid2) -- (tid6);
\node[circle, scale=0.75, fill] (tid3) at (5.25,3){};
\draw[](tid0) -- (tid1);
\draw[](tid0) -- (tid2);
\draw[](tid0) -- (tid3);
\end{tikzpicture}
\nodepart{three}
\footnotesize{5.1875}
\nodepart{four}
\footnotesize{$50\:25\:25$}
};
 \\ 
\node[draw=black, rectangle split,  rectangle split parts=4] (sn0x22bafa0){
\footnotesize{3.125}
\nodepart{two}
\begin{tikzpicture}[scale=.2]
\node[circle, scale=0.75, fill] (tid0) at (3,1.5){};
\node[circle, scale=0.75, fill] (tid1) at (1.5,3){};
\node[circle, scale=0.75, fill] (tid4) at (0.75,4.5){};
\node[circle, scale=0.75, fill, task_scheduled] (tid7) at (0.75,6){};
\draw[](tid4) -- (tid7);
\node[circle, scale=0.75, fill] (tid5) at (2.25,4.5){};
\draw[](tid1) -- (tid4);
\draw[](tid1) -- (tid5);
\node[circle, scale=0.75, fill] (tid2) at (3.75,3){};
\node[circle, scale=0.75, fill, task_scheduled] (tid6) at (3.75,4.5){};
\draw[](tid2) -- (tid6);
\node[circle, scale=0.75, fill] (tid3) at (5.25,3){};
\draw[](tid0) -- (tid1);
\draw[](tid0) -- (tid2);
\draw[](tid0) -- (tid3);
\end{tikzpicture}
\nodepart{three}
\footnotesize{5.1875}
\nodepart{four}
\footnotesize{$50\:50$}
};
 \\ 
\node[draw=black, rectangle split,  rectangle split parts=4] (sn0x22b34f0){
\footnotesize{14.2708}
\nodepart{two}
\begin{tikzpicture}[scale=.2]
\node[circle, scale=0.75, fill] (tid0) at (3.75,1.5){};
\node[circle, scale=0.75, fill] (tid1) at (2.25,3){};
\node[circle, scale=0.75, fill, task_scheduled] (tid4) at (0.75,4.5){};
\node[circle, scale=0.75, fill, task_scheduled] (tid5) at (2.25,4.5){};
\node[circle, scale=0.75, fill] (tid6) at (3.75,4.5){};
\draw[](tid1) -- (tid4);
\draw[](tid1) -- (tid5);
\draw[](tid1) -- (tid6);
\node[circle, scale=0.75, fill] (tid2) at (5.25,3){};
\node[circle, scale=0.75, fill] (tid7) at (5.25,4.5){};
\draw[](tid2) -- (tid7);
\node[circle, scale=0.75, fill] (tid3) at (6.75,3){};
\draw[](tid0) -- (tid1);
\draw[](tid0) -- (tid2);
\draw[](tid0) -- (tid3);
\end{tikzpicture}
\nodepart{three}
\footnotesize{5.125}
\nodepart{four}
\footnotesize{$50\:50$}
};
 \\ 
\node[draw=black, rectangle split,  rectangle split parts=4] (sn0x22b3cf0){
\footnotesize{13.6979}
\nodepart{two}
\begin{tikzpicture}[scale=.2]
\node[circle, scale=0.75, fill] (tid0) at (3.75,1.5){};
\node[circle, scale=0.75, fill] (tid1) at (2.25,3){};
\node[circle, scale=0.75, fill, task_scheduled] (tid4) at (0.75,4.5){};
\node[circle, scale=0.75, fill] (tid5) at (2.25,4.5){};
\node[circle, scale=0.75, fill] (tid6) at (3.75,4.5){};
\draw[](tid1) -- (tid4);
\draw[](tid1) -- (tid5);
\draw[](tid1) -- (tid6);
\node[circle, scale=0.75, fill] (tid2) at (5.25,3){};
\node[circle, scale=0.75, fill, task_scheduled] (tid7) at (5.25,4.5){};
\draw[](tid2) -- (tid7);
\node[circle, scale=0.75, fill] (tid3) at (6.75,3){};
\draw[](tid0) -- (tid1);
\draw[](tid0) -- (tid2);
\draw[](tid0) -- (tid3);
\end{tikzpicture}
\nodepart{three}
\footnotesize{5.125}
\nodepart{four}
\footnotesize{$50\:50$}
};
 \\ 
\node[draw=black, rectangle split,  rectangle split parts=4] (sn0x22b2b90){
\footnotesize{6.25}
\nodepart{two}
\begin{tikzpicture}[scale=.2]
\node[circle, scale=0.75, fill] (tid0) at (3,1.5){};
\node[circle, scale=0.75, fill] (tid1) at (1.5,3){};
\node[circle, scale=0.75, fill, task_scheduled] (tid4) at (0.75,4.5){};
\node[circle, scale=0.75, fill] (tid5) at (2.25,4.5){};
\draw[](tid1) -- (tid4);
\draw[](tid1) -- (tid5);
\node[circle, scale=0.75, fill] (tid2) at (3.75,3){};
\node[circle, scale=0.75, fill] (tid6) at (3.75,4.5){};
\node[circle, scale=0.75, fill, task_scheduled] (tid7) at (3.75,6){};
\draw[](tid6) -- (tid7);
\draw[](tid2) -- (tid6);
\node[circle, scale=0.75, fill] (tid3) at (5.25,3){};
\draw[](tid0) -- (tid1);
\draw[](tid0) -- (tid2);
\draw[](tid0) -- (tid3);
\end{tikzpicture}
\nodepart{three}
\footnotesize{5.1875}
\nodepart{four}
\footnotesize{$50\:25\:25$}
};
 \\ 
\node[draw=black, rectangle split,  rectangle split parts=4] (sn0x22b6c20){
\footnotesize{32.1875}
\nodepart{two}
\begin{tikzpicture}[scale=.2]
\node[circle, scale=0.75, fill] (tid0) at (3.75,1.5){};
\node[circle, scale=0.75, fill] (tid1) at (1.5,3){};
\node[circle, scale=0.75, fill, task_scheduled] (tid4) at (0.75,4.5){};
\node[circle, scale=0.75, fill] (tid5) at (2.25,4.5){};
\draw[](tid1) -- (tid4);
\draw[](tid1) -- (tid5);
\node[circle, scale=0.75, fill] (tid2) at (4.5,3){};
\node[circle, scale=0.75, fill, task_scheduled] (tid6) at (3.75,4.5){};
\node[circle, scale=0.75, fill] (tid7) at (5.25,4.5){};
\draw[](tid2) -- (tid6);
\draw[](tid2) -- (tid7);
\node[circle, scale=0.75, fill] (tid3) at (6.75,3){};
\draw[](tid0) -- (tid1);
\draw[](tid0) -- (tid2);
\draw[](tid0) -- (tid3);
\end{tikzpicture}
\nodepart{three}
\footnotesize{5.125}
\nodepart{four}
\footnotesize{$50\:50$}
};
 \\ 
\node[draw=black, rectangle split,  rectangle split parts=4] (sn0x22b66d0){
\footnotesize{16.0938}
\nodepart{two}
\begin{tikzpicture}[scale=.2]
\node[circle, scale=0.75, fill] (tid0) at (3.75,1.5){};
\node[circle, scale=0.75, fill] (tid1) at (1.5,3){};
\node[circle, scale=0.75, fill, task_scheduled] (tid4) at (0.75,4.5){};
\node[circle, scale=0.75, fill, task_scheduled] (tid5) at (2.25,4.5){};
\draw[](tid1) -- (tid4);
\draw[](tid1) -- (tid5);
\node[circle, scale=0.75, fill] (tid2) at (4.5,3){};
\node[circle, scale=0.75, fill] (tid6) at (3.75,4.5){};
\node[circle, scale=0.75, fill] (tid7) at (5.25,4.5){};
\draw[](tid2) -- (tid6);
\draw[](tid2) -- (tid7);
\node[circle, scale=0.75, fill] (tid3) at (6.75,3){};
\draw[](tid0) -- (tid1);
\draw[](tid0) -- (tid2);
\draw[](tid0) -- (tid3);
\end{tikzpicture}
\nodepart{three}
\footnotesize{5.125}
\nodepart{four}
\footnotesize{$1$}
};
 \\ 
\node[draw=black, rectangle split,  rectangle split parts=4] (sn0x22b2d60){
\footnotesize{3.75}
\nodepart{two}
\begin{tikzpicture}[scale=.2]
\node[circle, scale=0.75, fill] (tid0) at (3,1.5){};
\node[circle, scale=0.75, fill] (tid1) at (1.5,3){};
\node[circle, scale=0.75, fill] (tid4) at (0.75,4.5){};
\node[circle, scale=0.75, fill] (tid5) at (2.25,4.5){};
\draw[](tid1) -- (tid4);
\draw[](tid1) -- (tid5);
\node[circle, scale=0.75, fill] (tid2) at (3.75,3){};
\node[circle, scale=0.75, fill, task_scheduled] (tid6) at (3.75,4.5){};
\draw[](tid2) -- (tid6);
\node[circle, scale=0.75, fill] (tid3) at (5.25,3){};
\node[circle, scale=0.75, fill, task_scheduled] (tid7) at (5.25,4.5){};
\draw[](tid3) -- (tid7);
\draw[](tid0) -- (tid1);
\draw[](tid0) -- (tid2);
\draw[](tid0) -- (tid3);
\end{tikzpicture}
\nodepart{three}
\footnotesize{5.125}
\nodepart{four}
\footnotesize{$1$}
};
 \\ 
\node[draw=black, rectangle split,  rectangle split parts=4] (sn0x22ae720){
\footnotesize{7.5}
\nodepart{two}
\begin{tikzpicture}[scale=.2]
\node[circle, scale=0.75, fill] (tid0) at (3,1.5){};
\node[circle, scale=0.75, fill] (tid1) at (1.5,3){};
\node[circle, scale=0.75, fill, task_scheduled] (tid4) at (0.75,4.5){};
\node[circle, scale=0.75, fill] (tid5) at (2.25,4.5){};
\draw[](tid1) -- (tid4);
\draw[](tid1) -- (tid5);
\node[circle, scale=0.75, fill] (tid2) at (3.75,3){};
\node[circle, scale=0.75, fill, task_scheduled] (tid6) at (3.75,4.5){};
\draw[](tid2) -- (tid6);
\node[circle, scale=0.75, fill] (tid3) at (5.25,3){};
\node[circle, scale=0.75, fill] (tid7) at (5.25,4.5){};
\draw[](tid3) -- (tid7);
\draw[](tid0) -- (tid1);
\draw[](tid0) -- (tid2);
\draw[](tid0) -- (tid3);
\end{tikzpicture}
\nodepart{three}
\footnotesize{5.125}
\nodepart{four}
\footnotesize{$25\:25\:50$}
};
 \\ 
\\
};
\end{scope}
\begin{scope}[yshift=\leveltopIIIII cm, anchor = center]
\matrix (line5)[row sep=0.5cm] {
\node[draw=black, rectangle split,  rectangle split parts=4] (sn0x22bb120){
\footnotesize{1.5625}
\nodepart{two}
\begin{tikzpicture}[scale=.2]
\node[circle, scale=0.75, fill] (tid0) at (3,1.5){};
\node[circle, scale=0.75, fill] (tid1) at (1.5,3){};
\node[circle, scale=0.75, fill] (tid4) at (0.75,4.5){};
\node[circle, scale=0.75, fill, task_scheduled] (tid6) at (0.75,6){};
\draw[](tid4) -- (tid6);
\node[circle, scale=0.75, fill, task_scheduled] (tid5) at (2.25,4.5){};
\draw[](tid1) -- (tid4);
\draw[](tid1) -- (tid5);
\node[circle, scale=0.75, fill] (tid2) at (3.75,3){};
\node[circle, scale=0.75, fill] (tid3) at (5.25,3){};
\draw[](tid0) -- (tid1);
\draw[](tid0) -- (tid2);
\draw[](tid0) -- (tid3);
\end{tikzpicture}
\nodepart{three}
\footnotesize{4.75}
\nodepart{four}
\footnotesize{$50\:50$}
};
 \\ 
\node[draw=black, rectangle split,  rectangle split parts=4] (sn0x22ae570){
\footnotesize{4.6875}
\nodepart{two}
\begin{tikzpicture}[scale=.2]
\node[circle, scale=0.75, fill] (tid0) at (2.25,1.5){};
\node[circle, scale=0.75, fill] (tid1) at (0.75,3){};
\node[circle, scale=0.75, fill] (tid4) at (0.75,4.5){};
\node[circle, scale=0.75, fill, task_scheduled] (tid6) at (0.75,6){};
\draw[](tid4) -- (tid6);
\draw[](tid1) -- (tid4);
\node[circle, scale=0.75, fill] (tid2) at (2.25,3){};
\node[circle, scale=0.75, fill, task_scheduled] (tid5) at (2.25,4.5){};
\draw[](tid2) -- (tid5);
\node[circle, scale=0.75, fill] (tid3) at (3.75,3){};
\draw[](tid0) -- (tid1);
\draw[](tid0) -- (tid2);
\draw[](tid0) -- (tid3);
\end{tikzpicture}
\nodepart{three}
\footnotesize{4.75}
\nodepart{four}
\footnotesize{$50\:50$}
};
 \\ 
\node[draw=black, rectangle split,  rectangle split parts=4] (sn0x22b40d0){
\footnotesize{6.84896}
\nodepart{two}
\begin{tikzpicture}[scale=.2]
\node[circle, scale=0.75, fill] (tid0) at (3.75,1.5){};
\node[circle, scale=0.75, fill] (tid1) at (2.25,3){};
\node[circle, scale=0.75, fill, task_scheduled] (tid4) at (0.75,4.5){};
\node[circle, scale=0.75, fill, task_scheduled] (tid5) at (2.25,4.5){};
\node[circle, scale=0.75, fill] (tid6) at (3.75,4.5){};
\draw[](tid1) -- (tid4);
\draw[](tid1) -- (tid5);
\draw[](tid1) -- (tid6);
\node[circle, scale=0.75, fill] (tid2) at (5.25,3){};
\node[circle, scale=0.75, fill] (tid3) at (6.75,3){};
\draw[](tid0) -- (tid1);
\draw[](tid0) -- (tid2);
\draw[](tid0) -- (tid3);
\end{tikzpicture}
\nodepart{three}
\footnotesize{4.625}
\nodepart{four}
\footnotesize{$1$}
};
 \\ 
\node[draw=black, rectangle split,  rectangle split parts=4] (sn0x22b14e0){
\footnotesize{27.4479}
\nodepart{two}
\begin{tikzpicture}[scale=.2]
\node[circle, scale=0.75, fill] (tid0) at (3,1.5){};
\node[circle, scale=0.75, fill] (tid1) at (1.5,3){};
\node[circle, scale=0.75, fill, task_scheduled] (tid4) at (0.75,4.5){};
\node[circle, scale=0.75, fill, task_scheduled] (tid5) at (2.25,4.5){};
\draw[](tid1) -- (tid4);
\draw[](tid1) -- (tid5);
\node[circle, scale=0.75, fill] (tid2) at (3.75,3){};
\node[circle, scale=0.75, fill] (tid6) at (3.75,4.5){};
\draw[](tid2) -- (tid6);
\node[circle, scale=0.75, fill] (tid3) at (5.25,3){};
\draw[](tid0) -- (tid1);
\draw[](tid0) -- (tid2);
\draw[](tid0) -- (tid3);
\end{tikzpicture}
\nodepart{three}
\footnotesize{4.625}
\nodepart{four}
\footnotesize{$1$}
};
 \\ 
\node[draw=black, rectangle split,  rectangle split parts=4] (sn0x22b1ca0){
\footnotesize{55.7031}
\nodepart{two}
\begin{tikzpicture}[scale=.2]
\node[circle, scale=0.75, fill] (tid0) at (3,1.5){};
\node[circle, scale=0.75, fill] (tid1) at (1.5,3){};
\node[circle, scale=0.75, fill, task_scheduled] (tid4) at (0.75,4.5){};
\node[circle, scale=0.75, fill] (tid5) at (2.25,4.5){};
\draw[](tid1) -- (tid4);
\draw[](tid1) -- (tid5);
\node[circle, scale=0.75, fill] (tid2) at (3.75,3){};
\node[circle, scale=0.75, fill, task_scheduled] (tid6) at (3.75,4.5){};
\draw[](tid2) -- (tid6);
\node[circle, scale=0.75, fill] (tid3) at (5.25,3){};
\draw[](tid0) -- (tid1);
\draw[](tid0) -- (tid2);
\draw[](tid0) -- (tid3);
\end{tikzpicture}
\nodepart{three}
\footnotesize{4.625}
\nodepart{four}
\footnotesize{$50\:50$}
};
 \\ 
\node[draw=black, rectangle split,  rectangle split parts=4] (sn0x22af010){
\footnotesize{3.75}
\nodepart{two}
\begin{tikzpicture}[scale=.2]
\node[circle, scale=0.75, fill] (tid0) at (2.25,1.5){};
\node[circle, scale=0.75, fill] (tid1) at (0.75,3){};
\node[circle, scale=0.75, fill, task_scheduled] (tid4) at (0.75,4.5){};
\draw[](tid1) -- (tid4);
\node[circle, scale=0.75, fill] (tid2) at (2.25,3){};
\node[circle, scale=0.75, fill, task_scheduled] (tid5) at (2.25,4.5){};
\draw[](tid2) -- (tid5);
\node[circle, scale=0.75, fill] (tid3) at (3.75,3){};
\node[circle, scale=0.75, fill] (tid6) at (3.75,4.5){};
\draw[](tid3) -- (tid6);
\draw[](tid0) -- (tid1);
\draw[](tid0) -- (tid2);
\draw[](tid0) -- (tid3);
\end{tikzpicture}
\nodepart{three}
\footnotesize{4.625}
\nodepart{four}
\footnotesize{$1$}
};
 \\ 
\\
};
\end{scope}
\begin{scope}[yshift=\leveltopIIIIII cm, anchor = center]
\matrix (line6)[row sep=0.5cm] {
\node[draw=black, rectangle split,  rectangle split parts=4] (sn0x22aecc0){
\footnotesize{3.125}
\nodepart{two}
\begin{tikzpicture}[scale=.2]
\node[circle, scale=0.75, fill] (tid0) at (2.25,1.5){};
\node[circle, scale=0.75, fill] (tid1) at (0.75,3){};
\node[circle, scale=0.75, fill] (tid4) at (0.75,4.5){};
\node[circle, scale=0.75, fill, task_scheduled] (tid5) at (0.75,6){};
\draw[](tid4) -- (tid5);
\draw[](tid1) -- (tid4);
\node[circle, scale=0.75, fill, task_scheduled] (tid2) at (2.25,3){};
\node[circle, scale=0.75, fill] (tid3) at (3.75,3){};
\draw[](tid0) -- (tid1);
\draw[](tid0) -- (tid2);
\draw[](tid0) -- (tid3);
\end{tikzpicture}
\nodepart{three}
\footnotesize{4.375}
\nodepart{four}
\footnotesize{$50\:50$}
};
 \\ 
\node[draw=black, rectangle split,  rectangle split parts=4] (sn0x22b21c0){
\footnotesize{35.4818}
\nodepart{two}
\begin{tikzpicture}[scale=.2]
\node[circle, scale=0.75, fill] (tid0) at (3,1.5){};
\node[circle, scale=0.75, fill] (tid1) at (1.5,3){};
\node[circle, scale=0.75, fill, task_scheduled] (tid4) at (0.75,4.5){};
\node[circle, scale=0.75, fill, task_scheduled] (tid5) at (2.25,4.5){};
\draw[](tid1) -- (tid4);
\draw[](tid1) -- (tid5);
\node[circle, scale=0.75, fill] (tid2) at (3.75,3){};
\node[circle, scale=0.75, fill] (tid3) at (5.25,3){};
\draw[](tid0) -- (tid1);
\draw[](tid0) -- (tid2);
\draw[](tid0) -- (tid3);
\end{tikzpicture}
\nodepart{three}
\footnotesize{4.125}
\nodepart{four}
\footnotesize{$1$}
};
 \\ 
\node[draw=black, rectangle split,  rectangle split parts=4] (sn0x22af160){
\footnotesize{61.3932}
\nodepart{two}
\begin{tikzpicture}[scale=.2]
\node[circle, scale=0.75, fill] (tid0) at (2.25,1.5){};
\node[circle, scale=0.75, fill] (tid1) at (0.75,3){};
\node[circle, scale=0.75, fill, task_scheduled] (tid4) at (0.75,4.5){};
\draw[](tid1) -- (tid4);
\node[circle, scale=0.75, fill] (tid2) at (2.25,3){};
\node[circle, scale=0.75, fill, task_scheduled] (tid5) at (2.25,4.5){};
\draw[](tid2) -- (tid5);
\node[circle, scale=0.75, fill] (tid3) at (3.75,3){};
\draw[](tid0) -- (tid1);
\draw[](tid0) -- (tid2);
\draw[](tid0) -- (tid3);
\end{tikzpicture}
\nodepart{three}
\footnotesize{4.125}
\nodepart{four}
\footnotesize{$1$}
};
 \\ 
\\
};
\end{scope}
\begin{scope}[yshift=\leveltopIIIIIII cm, anchor = center]
\matrix (line7)[row sep=0.5cm] {
\node[draw=black, rectangle split,  rectangle split parts=4] (sn0x22af6c0){
\footnotesize{1.5625}
\nodepart{two}
\begin{tikzpicture}[scale=.2]
\node[circle, scale=0.75, fill] (tid0) at (1.5,1.5){};
\node[circle, scale=0.75, fill] (tid1) at (0.75,3){};
\node[circle, scale=0.75, fill] (tid3) at (0.75,4.5){};
\node[circle, scale=0.75, fill, task_scheduled] (tid4) at (0.75,6){};
\draw[](tid3) -- (tid4);
\draw[](tid1) -- (tid3);
\node[circle, scale=0.75, fill, task_scheduled] (tid2) at (2.25,3){};
\draw[](tid0) -- (tid1);
\draw[](tid0) -- (tid2);
\end{tikzpicture}
\nodepart{three}
\footnotesize{4.125}
\nodepart{four}
\footnotesize{$50\:50$}
};
 \\ 
\node[draw=black, rectangle split,  rectangle split parts=4] (sn0x22afb50){
\footnotesize{98.4375}
\nodepart{two}
\begin{tikzpicture}[scale=.2]
\node[circle, scale=0.75, fill] (tid0) at (2.25,1.5){};
\node[circle, scale=0.75, fill] (tid1) at (0.75,3){};
\node[circle, scale=0.75, fill, task_scheduled] (tid4) at (0.75,4.5){};
\draw[](tid1) -- (tid4);
\node[circle, scale=0.75, fill, task_scheduled] (tid2) at (2.25,3){};
\node[circle, scale=0.75, fill] (tid3) at (3.75,3){};
\draw[](tid0) -- (tid1);
\draw[](tid0) -- (tid2);
\draw[](tid0) -- (tid3);
\end{tikzpicture}
\nodepart{three}
\footnotesize{3.625}
\nodepart{four}
\footnotesize{$50\:50$}
};
 \\ 
\\
};
\end{scope}
\draw (sn0x22aa490.east) -- (sn0x22bcc60.west);
\draw (sn0x22aa490.east) -- (sn0x22bcd60.west);
\draw (sn0x22aa490.east) -- (sn0x22b9170.west);
\draw (sn0x22aa490.east) -- (sn0x22ab500.west);
\draw (sn0x22bcc60.east) -- (sn0x22bc030.west);
\draw (sn0x22bcc60.east) -- (sn0x22bb880.west);
\draw (sn0x22bcc60.east) -- (sn0x22bc5d0.west);
\draw (sn0x22bcc60.east) -- (sn0x22b7500.west);
\draw (sn0x22bcd60.east) -- (sn0x22b4a40.west);
\draw (sn0x22bcd60.east) -- (sn0x22b7500.west);
\draw (sn0x22bcd60.east) -- (sn0x22b7aa0.west);
\draw (sn0x22b9170.east) -- (sn0x22b6220.west);
\draw (sn0x22b9170.east) -- (sn0x22bc5d0.west);
\draw (sn0x22b9170.east) -- (sn0x22b7500.west);
\draw (sn0x22ab500.east) -- (sn0x22ada30.west);
\draw (sn0x22ab500.east) -- (sn0x22b4be0.west);
\draw (sn0x22ab500.east) -- (sn0x22b7500.west);
\draw (sn0x22ab500.east) -- (sn0x22b7aa0.west);
\draw (sn0x22bc030.east) -- (sn0x22baab0.west);
\draw (sn0x22bc030.east) -- (sn0x22bafa0.west);
\draw (sn0x22bc030.east) -- (sn0x22b6c20.west);
\draw (sn0x22bc030.east) -- (sn0x22b66d0.west);
\draw (sn0x22bb880.east) -- (sn0x22b2b90.west);
\draw (sn0x22bb880.east) -- (sn0x22b66d0.west);
\draw (sn0x22bb880.east) -- (sn0x22b6c20.west);
\draw (sn0x22b4a40.east) -- (sn0x22b2b90.west);
\draw (sn0x22b4a40.east) -- (sn0x22b34f0.west);
\draw (sn0x22b4a40.east) -- (sn0x22b3cf0.west);
\draw (sn0x22bc5d0.east) -- (sn0x22b66d0.west);
\draw (sn0x22bc5d0.east) -- (sn0x22b6c20.west);
\draw (sn0x22b7500.east) -- (sn0x22b6c20.west);
\draw (sn0x22b7500.east) -- (sn0x22b66d0.west);
\draw (sn0x22b7500.east) -- (sn0x22b34f0.west);
\draw (sn0x22b7500.east) -- (sn0x22b3cf0.west);
\draw (sn0x22b7aa0.east) -- (sn0x22b3cf0.west);
\draw (sn0x22ada30.east) -- (sn0x22ae720.west);
\draw (sn0x22ada30.east) -- (sn0x22b2d60.west);
\draw (sn0x22ada30.east) -- (sn0x22b34f0.west);
\draw (sn0x22ada30.east) -- (sn0x22b3cf0.west);
\draw (sn0x22b4be0.east) -- (sn0x22b3cf0.west);
\draw (sn0x22b6220.east) -- (sn0x22b2d60.west);
\draw (sn0x22b6220.east) -- (sn0x22ae720.west);
\draw (sn0x22b6220.east) -- (sn0x22b66d0.west);
\draw (sn0x22b6220.east) -- (sn0x22b6c20.west);
\draw (sn0x22baab0.east) -- (sn0x22ae570.west);
\draw (sn0x22baab0.east) -- (sn0x22b14e0.west);
\draw (sn0x22baab0.east) -- (sn0x22b1ca0.west);
\draw (sn0x22bafa0.east) -- (sn0x22bb120.west);
\draw (sn0x22bafa0.east) -- (sn0x22b1ca0.west);
\draw (sn0x22b34f0.east) -- (sn0x22b14e0.west);
\draw (sn0x22b34f0.east) -- (sn0x22b1ca0.west);
\draw (sn0x22b3cf0.east) -- (sn0x22b1ca0.west);
\draw (sn0x22b3cf0.east) -- (sn0x22b40d0.west);
\draw (sn0x22b2b90.east) -- (sn0x22ae570.west);
\draw (sn0x22b2b90.east) -- (sn0x22b14e0.west);
\draw (sn0x22b2b90.east) -- (sn0x22b1ca0.west);
\draw (sn0x22b6c20.east) -- (sn0x22b1ca0.west);
\draw (sn0x22b6c20.east) -- (sn0x22b14e0.west);
\draw (sn0x22b66d0.east) -- (sn0x22b1ca0.west);
\draw (sn0x22b2d60.east) -- (sn0x22b1ca0.west);
\draw (sn0x22ae720.east) -- (sn0x22af010.west);
\draw (sn0x22ae720.east) -- (sn0x22b14e0.west);
\draw (sn0x22ae720.east) -- (sn0x22b1ca0.west);
\draw (sn0x22bb120.east) -- (sn0x22aecc0.west);
\draw (sn0x22bb120.east) -- (sn0x22b21c0.west);
\draw (sn0x22ae570.east) -- (sn0x22aecc0.west);
\draw (sn0x22ae570.east) -- (sn0x22af160.west);
\draw (sn0x22b40d0.east) -- (sn0x22b21c0.west);
\draw (sn0x22b14e0.east) -- (sn0x22af160.west);
\draw (sn0x22b1ca0.east) -- (sn0x22af160.west);
\draw (sn0x22b1ca0.east) -- (sn0x22b21c0.west);
\draw (sn0x22af010.east) -- (sn0x22af160.west);
\draw (sn0x22aecc0.east) -- (sn0x22af6c0.west);
\draw (sn0x22aecc0.east) -- (sn0x22afb50.west);
\draw (sn0x22b21c0.east) -- (sn0x22afb50.west);
\draw (sn0x22af160.east) -- (sn0x22afb50.west);
\end{tikzpicture}

%%% Local Variables:
%%% TeX-master: "thesis/thesis.tex"
%%% End: 

\input{../default_chain.tex}

% \input{../000111223hlf.tex}
% \renewcommand{\leveltopI}{-15cm + \leveltop}
\renewcommand{\leveltopII}{-15cm + \leveltopI}
\renewcommand{\leveltopIII}{-15cm + \leveltopII}
\renewcommand{\leveltopIIII}{-15cm + \leveltopIII}
\renewcommand{\leveltopIIIII}{-15cm + \leveltopIIII}
\renewcommand{\leveltopIIIIII}{-15cm + \leveltopIIIII}
\renewcommand{\leveltopIIIIIII}{-15cm + \leveltopIIIIII}
\renewcommand{\leveltopIIIIIIII}{-15cm + \leveltopIIIIIII}
\renewcommand{\leveltopIIIIIIIII}{-15cm + \leveltopIIIIIIII}
\renewcommand{\leveltopIIIIIIIIII}{-15cm + \leveltopIIIIIIIII}
\begin{tikzpicture}[scale=.2, anchor=south]
\begin{scope}[yshift=\leveltopI cm]
\matrix (line1) [column sep=1cm] {
\node[draw=black, rectangle split,  rectangle split parts=3] (sn0xd9fa30){
\begin{tikzpicture}[scale=.2]
\node[circle, scale=0.75, fill] (tid0) at (4.5,1.5){};
\node[circle, scale=0.75, fill] (tid1) at (2.25,3){};
\node[circle, scale=0.75, fill] (tid4) at (0.75,4.5){};
\node[circle, scale=0.75, fill] (tid5) at (2.25,4.5){};
\node[circle, scale=0.75, fill] (tid6) at (3.75,4.5){};
\draw[](tid1) -- (tid4);
\draw[](tid1) -- (tid5);
\draw[](tid1) -- (tid6);
\node[circle, scale=0.75, fill] (tid2) at (6,3){};
\node[circle, scale=0.75, fill, red] (tid7) at (5.25,4.5){};
\node[circle, scale=0.75, fill, red] (tid8) at (6.75,4.5){};
\draw[](tid2) -- (tid7);
\draw[](tid2) -- (tid8);
\node[circle, scale=0.75, fill] (tid3) at (8.25,3){};
\node[circle, scale=0.75, fill, red] (tid9) at (8.25,4.5){};
\draw[](tid3) -- (tid9);
\draw[](tid0) -- (tid1);
\draw[](tid0) -- (tid2);
\draw[](tid0) -- (tid3);
\end{tikzpicture}
\nodepart{two}
\footnotesize{5.20028}
\nodepart{three}
\footnotesize{$33\:67$}
};
 & 
\\
};
\end{scope}
\begin{scope}[yshift=\leveltopII cm]
\matrix (line2) [column sep=1cm] {
\node[draw=black, rectangle split,  rectangle split parts=3] (sn0xd9fb90){
\begin{tikzpicture}[scale=.2]
\node[circle, scale=0.75, fill] (tid0) at (4.5,1.5){};
\node[circle, scale=0.75, fill] (tid1) at (2.25,3){};
\node[circle, scale=0.75, fill, red] (tid4) at (0.75,4.5){};
\node[circle, scale=0.75, fill] (tid5) at (2.25,4.5){};
\node[circle, scale=0.75, fill] (tid6) at (3.75,4.5){};
\draw[](tid1) -- (tid4);
\draw[](tid1) -- (tid5);
\draw[](tid1) -- (tid6);
\node[circle, scale=0.75, fill] (tid2) at (6,3){};
\node[circle, scale=0.75, fill, red] (tid7) at (5.25,4.5){};
\node[circle, scale=0.75, fill, red] (tid8) at (6.75,4.5){};
\draw[](tid2) -- (tid7);
\draw[](tid2) -- (tid8);
\node[circle, scale=0.75, fill] (tid3) at (8.25,3){};
\draw[](tid0) -- (tid1);
\draw[](tid0) -- (tid2);
\draw[](tid0) -- (tid3);
\end{tikzpicture}
\nodepart{two}
\footnotesize{4.86626}
\nodepart{three}
\footnotesize{$67\:33$}
};
 & 
\node[draw=black, rectangle split,  rectangle split parts=3] (sn0xd9d000){
\begin{tikzpicture}[scale=.2]
\node[circle, scale=0.75, fill] (tid0) at (3.75,1.5){};
\node[circle, scale=0.75, fill] (tid1) at (2.25,3){};
\node[circle, scale=0.75, fill, red] (tid4) at (0.75,4.5){};
\node[circle, scale=0.75, fill] (tid5) at (2.25,4.5){};
\node[circle, scale=0.75, fill] (tid6) at (3.75,4.5){};
\draw[](tid1) -- (tid4);
\draw[](tid1) -- (tid5);
\draw[](tid1) -- (tid6);
\node[circle, scale=0.75, fill] (tid2) at (5.25,3){};
\node[circle, scale=0.75, fill, red] (tid7) at (5.25,4.5){};
\draw[](tid2) -- (tid7);
\node[circle, scale=0.75, fill] (tid3) at (6.75,3){};
\node[circle, scale=0.75, fill, red] (tid8) at (6.75,4.5){};
\draw[](tid3) -- (tid8);
\draw[](tid0) -- (tid1);
\draw[](tid0) -- (tid2);
\draw[](tid0) -- (tid3);
\end{tikzpicture}
\nodepart{two}
\footnotesize{4.86728}
\nodepart{three}
\footnotesize{$67\:33$}
};
 & 
\\
};
\end{scope}
\begin{scope}[yshift=\leveltopIII cm]
\matrix (line3) [column sep=1cm] {
\node[draw=black, rectangle split,  rectangle split parts=3] (sn0xd9cf30){
\begin{tikzpicture}[scale=.2]
\node[circle, scale=0.75, fill] (tid0) at (3.75,1.5){};
\node[circle, scale=0.75, fill] (tid1) at (2.25,3){};
\node[circle, scale=0.75, fill, red] (tid4) at (0.75,4.5){};
\node[circle, scale=0.75, fill, red] (tid5) at (2.25,4.5){};
\node[circle, scale=0.75, fill] (tid6) at (3.75,4.5){};
\draw[](tid1) -- (tid4);
\draw[](tid1) -- (tid5);
\draw[](tid1) -- (tid6);
\node[circle, scale=0.75, fill] (tid2) at (5.25,3){};
\node[circle, scale=0.75, fill, red] (tid7) at (5.25,4.5){};
\draw[](tid2) -- (tid7);
\node[circle, scale=0.75, fill] (tid3) at (6.75,3){};
\draw[](tid0) -- (tid1);
\draw[](tid0) -- (tid2);
\draw[](tid0) -- (tid3);
\end{tikzpicture}
\nodepart{two}
\footnotesize{4.53086}
\nodepart{three}
\footnotesize{$33\:67$}
};
 & 
\node[draw=black, rectangle split,  rectangle split parts=3] (sn0xd9f7a0){
\begin{tikzpicture}[scale=.2]
\node[circle, scale=0.75, fill] (tid0) at (3.75,1.5){};
\node[circle, scale=0.75, fill] (tid1) at (1.5,3){};
\node[circle, scale=0.75, fill, red] (tid4) at (0.75,4.5){};
\node[circle, scale=0.75, fill, red] (tid5) at (2.25,4.5){};
\draw[](tid1) -- (tid4);
\draw[](tid1) -- (tid5);
\node[circle, scale=0.75, fill] (tid2) at (4.5,3){};
\node[circle, scale=0.75, fill, red] (tid6) at (3.75,4.5){};
\node[circle, scale=0.75, fill] (tid7) at (5.25,4.5){};
\draw[](tid2) -- (tid6);
\draw[](tid2) -- (tid7);
\node[circle, scale=0.75, fill] (tid3) at (6.75,3){};
\draw[](tid0) -- (tid1);
\draw[](tid0) -- (tid2);
\draw[](tid0) -- (tid3);
\end{tikzpicture}
\nodepart{two}
\footnotesize{4.53704}
\nodepart{three}
\footnotesize{$1$}
};
 & 
\node[draw=black, rectangle split,  rectangle split parts=3] (sn0xd9a9c0){
\begin{tikzpicture}[scale=.2]
\node[circle, scale=0.75, fill] (tid0) at (3,1.5){};
\node[circle, scale=0.75, fill] (tid1) at (1.5,3){};
\node[circle, scale=0.75, fill, red] (tid4) at (0.75,4.5){};
\node[circle, scale=0.75, fill] (tid5) at (2.25,4.5){};
\draw[](tid1) -- (tid4);
\draw[](tid1) -- (tid5);
\node[circle, scale=0.75, fill] (tid2) at (3.75,3){};
\node[circle, scale=0.75, fill, red] (tid6) at (3.75,4.5){};
\draw[](tid2) -- (tid6);
\node[circle, scale=0.75, fill] (tid3) at (5.25,3){};
\node[circle, scale=0.75, fill, red] (tid7) at (5.25,4.5){};
\draw[](tid3) -- (tid7);
\draw[](tid0) -- (tid1);
\draw[](tid0) -- (tid2);
\draw[](tid0) -- (tid3);
\end{tikzpicture}
\nodepart{two}
\footnotesize{4.54012}
\nodepart{three}
\footnotesize{$67\:33$}
};
 & 
\\
};
\end{scope}
\begin{scope}[yshift=\leveltopIIII cm]
\matrix (line4) [column sep=1cm] {
\node[draw=black, rectangle split,  rectangle split parts=3] (sn0xd9d6c0){
\begin{tikzpicture}[scale=.2]
\node[circle, scale=0.75, fill] (tid0) at (3.75,1.5){};
\node[circle, scale=0.75, fill] (tid1) at (2.25,3){};
\node[circle, scale=0.75, fill, red] (tid4) at (0.75,4.5){};
\node[circle, scale=0.75, fill, red] (tid5) at (2.25,4.5){};
\node[circle, scale=0.75, fill, red] (tid6) at (3.75,4.5){};
\draw[](tid1) -- (tid4);
\draw[](tid1) -- (tid5);
\draw[](tid1) -- (tid6);
\node[circle, scale=0.75, fill] (tid2) at (5.25,3){};
\node[circle, scale=0.75, fill] (tid3) at (6.75,3){};
\draw[](tid0) -- (tid1);
\draw[](tid0) -- (tid2);
\draw[](tid0) -- (tid3);
\end{tikzpicture}
\nodepart{two}
\footnotesize{4.18519}
\nodepart{three}
\footnotesize{$1$}
};
 & 
\node[draw=black, rectangle split,  rectangle split parts=3] (sn0xd9ae50){
\begin{tikzpicture}[scale=.2]
\node[circle, scale=0.75, fill] (tid0) at (3,1.5){};
\node[circle, scale=0.75, fill] (tid1) at (1.5,3){};
\node[circle, scale=0.75, fill, red] (tid4) at (0.75,4.5){};
\node[circle, scale=0.75, fill, red] (tid5) at (2.25,4.5){};
\draw[](tid1) -- (tid4);
\draw[](tid1) -- (tid5);
\node[circle, scale=0.75, fill] (tid2) at (3.75,3){};
\node[circle, scale=0.75, fill, red] (tid6) at (3.75,4.5){};
\draw[](tid2) -- (tid6);
\node[circle, scale=0.75, fill] (tid3) at (5.25,3){};
\draw[](tid0) -- (tid1);
\draw[](tid0) -- (tid2);
\draw[](tid0) -- (tid3);
\end{tikzpicture}
\nodepart{two}
\footnotesize{4.2037}
\nodepart{three}
\footnotesize{$33\:67$}
};
 & 
\node[draw=black, rectangle split,  rectangle split parts=3] (sn0xd98db0){
\begin{tikzpicture}[scale=.2]
\node[circle, scale=0.75, fill] (tid0) at (2.25,1.5){};
\node[circle, scale=0.75, fill] (tid1) at (0.75,3){};
\node[circle, scale=0.75, fill, red] (tid4) at (0.75,4.5){};
\draw[](tid1) -- (tid4);
\node[circle, scale=0.75, fill] (tid2) at (2.25,3){};
\node[circle, scale=0.75, fill, red] (tid5) at (2.25,4.5){};
\draw[](tid2) -- (tid5);
\node[circle, scale=0.75, fill] (tid3) at (3.75,3){};
\node[circle, scale=0.75, fill, red] (tid6) at (3.75,4.5){};
\draw[](tid3) -- (tid6);
\draw[](tid0) -- (tid1);
\draw[](tid0) -- (tid2);
\draw[](tid0) -- (tid3);
\end{tikzpicture}
\nodepart{two}
\footnotesize{4.21296}
\nodepart{three}
\footnotesize{$1$}
};
 & 
\\
};
\end{scope}
\begin{scope}[yshift=\leveltopIIIII cm]
\matrix (line5) [column sep=1cm] {
\node[draw=black, rectangle split,  rectangle split parts=3] (sn0xd9a340){
\begin{tikzpicture}[scale=.2]
\node[circle, scale=0.75, fill] (tid0) at (3,1.5){};
\node[circle, scale=0.75, fill] (tid1) at (1.5,3){};
\node[circle, scale=0.75, fill, red] (tid4) at (0.75,4.5){};
\node[circle, scale=0.75, fill, red] (tid5) at (2.25,4.5){};
\draw[](tid1) -- (tid4);
\draw[](tid1) -- (tid5);
\node[circle, scale=0.75, fill, red] (tid2) at (3.75,3){};
\node[circle, scale=0.75, fill] (tid3) at (5.25,3){};
\draw[](tid0) -- (tid1);
\draw[](tid0) -- (tid2);
\draw[](tid0) -- (tid3);
\end{tikzpicture}
\nodepart{two}
\footnotesize{3.85185}
\nodepart{three}
\footnotesize{$33\:67$}
};
 & 
\node[draw=black, rectangle split,  rectangle split parts=3] (sn0xd98be0){
\begin{tikzpicture}[scale=.2]
\node[circle, scale=0.75, fill] (tid0) at (2.25,1.5){};
\node[circle, scale=0.75, fill] (tid1) at (0.75,3){};
\node[circle, scale=0.75, fill, red] (tid4) at (0.75,4.5){};
\draw[](tid1) -- (tid4);
\node[circle, scale=0.75, fill] (tid2) at (2.25,3){};
\node[circle, scale=0.75, fill, red] (tid5) at (2.25,4.5){};
\draw[](tid2) -- (tid5);
\node[circle, scale=0.75, fill, red] (tid3) at (3.75,3){};
\draw[](tid0) -- (tid1);
\draw[](tid0) -- (tid2);
\draw[](tid0) -- (tid3);
\end{tikzpicture}
\nodepart{two}
\footnotesize{3.87963}
\nodepart{three}
\footnotesize{$67\:33$}
};
 & 
\\
};
\end{scope}
\begin{scope}[yshift=\leveltopIIIIII cm]
\matrix (line6) [column sep=1cm] {
\node[draw=black, rectangle split,  rectangle split parts=3] (sn0xd99950){
\begin{tikzpicture}[scale=.2]
\node[circle, scale=0.75, fill] (tid0) at (2.25,1.5){};
\node[circle, scale=0.75, fill] (tid1) at (1.5,3){};
\node[circle, scale=0.75, fill, red] (tid3) at (0.75,4.5){};
\node[circle, scale=0.75, fill, red] (tid4) at (2.25,4.5){};
\draw[](tid1) -- (tid3);
\draw[](tid1) -- (tid4);
\node[circle, scale=0.75, fill, red] (tid2) at (3.75,3){};
\draw[](tid0) -- (tid1);
\draw[](tid0) -- (tid2);
\end{tikzpicture}
\nodepart{two}
\footnotesize{3.66667}
\nodepart{three}
\footnotesize{$33\:67$}
};
 & 
\node[draw=black, rectangle split,  rectangle split parts=3] (sn0xd98b10){
\begin{tikzpicture}[scale=.2]
\node[circle, scale=0.75, fill] (tid0) at (2.25,1.5){};
\node[circle, scale=0.75, fill] (tid1) at (0.75,3){};
\node[circle, scale=0.75, fill, red] (tid4) at (0.75,4.5){};
\draw[](tid1) -- (tid4);
\node[circle, scale=0.75, fill, red] (tid2) at (2.25,3){};
\node[circle, scale=0.75, fill, red] (tid3) at (3.75,3){};
\draw[](tid0) -- (tid1);
\draw[](tid0) -- (tid2);
\draw[](tid0) -- (tid3);
\end{tikzpicture}
\nodepart{two}
\footnotesize{3.44444}
\nodepart{three}
\footnotesize{$67\:33$}
};
 & 
\node[draw=black, rectangle split,  rectangle split parts=3] (sn0xd982f0){
\begin{tikzpicture}[scale=.2]
\node[circle, scale=0.75, fill] (tid0) at (1.5,1.5){};
\node[circle, scale=0.75, fill] (tid1) at (0.75,3){};
\node[circle, scale=0.75, fill, red] (tid3) at (0.75,4.5){};
\draw[](tid1) -- (tid3);
\node[circle, scale=0.75, fill] (tid2) at (2.25,3){};
\node[circle, scale=0.75, fill, red] (tid4) at (2.25,4.5){};
\draw[](tid2) -- (tid4);
\draw[](tid0) -- (tid1);
\draw[](tid0) -- (tid2);
\end{tikzpicture}
\nodepart{two}
\footnotesize{3.75}
\nodepart{three}
\footnotesize{$1$}
};
 & 
\\
};
\end{scope}
\begin{scope}[yshift=\leveltopIIIIIII cm]
\matrix (line7) [column sep=1cm] {
\node[draw=black, rectangle split,  rectangle split parts=3] (sn0xd99040){
\begin{tikzpicture}[scale=.2]
\node[circle, scale=0.75, fill] (tid0) at (1.5,1.5){};
\node[circle, scale=0.75, fill] (tid1) at (1.5,3){};
\node[circle, scale=0.75, fill, red] (tid2) at (0.75,4.5){};
\node[circle, scale=0.75, fill, red] (tid3) at (2.25,4.5){};
\draw[](tid1) -- (tid2);
\draw[](tid1) -- (tid3);
\draw[](tid0) -- (tid1);
\end{tikzpicture}
\nodepart{two}
\footnotesize{3.5}
\nodepart{three}
\footnotesize{$1$}
};
 & 
\node[draw=black, rectangle split,  rectangle split parts=3] (sn0xd981e0){
\begin{tikzpicture}[scale=.2]
\node[circle, scale=0.75, fill] (tid0) at (1.5,1.5){};
\node[circle, scale=0.75, fill] (tid1) at (0.75,3){};
\node[circle, scale=0.75, fill, red] (tid3) at (0.75,4.5){};
\draw[](tid1) -- (tid3);
\node[circle, scale=0.75, fill, red] (tid2) at (2.25,3){};
\draw[](tid0) -- (tid1);
\draw[](tid0) -- (tid2);
\end{tikzpicture}
\nodepart{two}
\footnotesize{3.25}
\nodepart{three}
\footnotesize{$50\:50$}
};
 & 
\node[draw=black, rectangle split,  rectangle split parts=3] (sn0xd985f0){
\begin{tikzpicture}[scale=.2]
\node[circle, scale=0.75, fill] (tid0) at (2.25,1.5){};
\node[circle, scale=0.75, fill, red] (tid1) at (0.75,3){};
\node[circle, scale=0.75, fill, red] (tid2) at (2.25,3){};
\node[circle, scale=0.75, fill, red] (tid3) at (3.75,3){};
\draw[](tid0) -- (tid1);
\draw[](tid0) -- (tid2);
\draw[](tid0) -- (tid3);
\end{tikzpicture}
\nodepart{two}
\footnotesize{2.83333}
\nodepart{three}
\footnotesize{$1$}
};
 & 
\\
};
\end{scope}
\begin{scope}[yshift=\leveltopIIIIIIII cm]
\matrix (line8) [column sep=1cm] {
\node[draw=black, rectangle split,  rectangle split parts=3] (sn0xd97cb0){
\begin{tikzpicture}[scale=.2]
\node[circle, scale=0.75, fill] (tid0) at (0.75,1.5){};
\node[circle, scale=0.75, fill] (tid1) at (0.75,3){};
\node[circle, scale=0.75, fill, red] (tid2) at (0.75,4.5){};
\draw[](tid1) -- (tid2);
\draw[](tid0) -- (tid1);
\end{tikzpicture}
\nodepart{two}
\footnotesize{3}
\nodepart{three}
\footnotesize{$1$}
};
 & 
\node[draw=black, rectangle split,  rectangle split parts=3] (sn0xd97ee0){
\begin{tikzpicture}[scale=.2]
\node[circle, scale=0.75, fill] (tid0) at (1.5,1.5){};
\node[circle, scale=0.75, fill, red] (tid1) at (0.75,3){};
\node[circle, scale=0.75, fill, red] (tid2) at (2.25,3){};
\draw[](tid0) -- (tid1);
\draw[](tid0) -- (tid2);
\end{tikzpicture}
\nodepart{two}
\footnotesize{2.5}
\nodepart{three}
\footnotesize{$1$}
};
 & 
\\
};
\end{scope}
\begin{scope}[yshift=\leveltopIIIIIIIII cm]
\matrix (line9) [column sep=1cm] {
\node[draw=black, rectangle split,  rectangle split parts=3] (sn0xd88140){
\begin{tikzpicture}[scale=.2]
\node[circle, scale=0.75, fill] (tid0) at (0.75,1.5){};
\node[circle, scale=0.75, fill, red] (tid1) at (0.75,3){};
\draw[](tid0) -- (tid1);
\end{tikzpicture}
\nodepart{two}
\footnotesize{2}
\nodepart{three}
\footnotesize{$1$}
};
 & 
\\
};
\end{scope}
\begin{scope}[yshift=\leveltopIIIIIIIIII cm]
\matrix (line10) [column sep=1cm] {
\node[draw=black, rectangle split,  rectangle split parts=3] (sn0xd88070){
\begin{tikzpicture}[scale=.2]
\node[circle, scale=0.75, fill, red] (tid0) at (0.75,1.5){};
\end{tikzpicture}
\nodepart{two}
\footnotesize{1}
\nodepart{three}
\footnotesize{$$}
};
 & 
\\
};
\end{scope}
\begin{scope}[yshift=\leveltopIIIIIIIIIII cm]
\matrix (line11) [column sep=1cm] {
\\
};
\end{scope}
\draw (sn0xd9fa30.south) -- (sn0xd9fb90.north);
\draw (sn0xd9fa30.south) -- (sn0xd9d000.north);
\draw (sn0xd9fb90.south) -- (sn0xd9cf30.north);
\draw (sn0xd9fb90.south) -- (sn0xd9f7a0.north);
\draw (sn0xd9d000.south) -- (sn0xd9cf30.north);
\draw (sn0xd9d000.south) -- (sn0xd9a9c0.north);
\draw (sn0xd9cf30.south) -- (sn0xd9d6c0.north);
\draw (sn0xd9cf30.south) -- (sn0xd9ae50.north);
\draw (sn0xd9f7a0.south) -- (sn0xd9ae50.north);
\draw (sn0xd9a9c0.south) -- (sn0xd9ae50.north);
\draw (sn0xd9a9c0.south) -- (sn0xd98db0.north);
\draw (sn0xd9d6c0.south) -- (sn0xd9a340.north);
\draw (sn0xd9ae50.south) -- (sn0xd9a340.north);
\draw (sn0xd9ae50.south) -- (sn0xd98be0.north);
\draw (sn0xd98db0.south) -- (sn0xd98be0.north);
\draw (sn0xd9a340.south) -- (sn0xd99950.north);
\draw (sn0xd9a340.south) -- (sn0xd98b10.north);
\draw (sn0xd98be0.south) -- (sn0xd982f0.north);
\draw (sn0xd98be0.south) -- (sn0xd98b10.north);
\draw (sn0xd99950.south) -- (sn0xd99040.north);
\draw (sn0xd99950.south) -- (sn0xd981e0.north);
\draw (sn0xd98b10.south) -- (sn0xd981e0.north);
\draw (sn0xd98b10.south) -- (sn0xd985f0.north);
\draw (sn0xd982f0.south) -- (sn0xd981e0.north);
\draw (sn0xd99040.south) -- (sn0xd97cb0.north);
\draw (sn0xd981e0.south) -- (sn0xd97cb0.north);
\draw (sn0xd981e0.south) -- (sn0xd97ee0.north);
\draw (sn0xd985f0.south) -- (sn0xd97ee0.north);
\draw (sn0xd97cb0.south) -- (sn0xd88140.north);
\draw (sn0xd97ee0.south) -- (sn0xd88140.north);
\draw (sn0xd88140.south) -- (sn0xd88070.north);
\end{tikzpicture}

%%% Local Variables:
%%% TeX-master: "thesis/thesis.tex"
%%% End: 
\renewcommand{\leveltopI}{-15cm + \leveltop}
\renewcommand{\leveltopII}{-15cm + \leveltopI}
\renewcommand{\leveltopIII}{-15cm + \leveltopII}
\renewcommand{\leveltopIIII}{-15cm + \leveltopIII}
\renewcommand{\leveltopIIIII}{-15cm + \leveltopIIII}
\renewcommand{\leveltopIIIIII}{-15cm + \leveltopIIIII}
\renewcommand{\leveltopIIIIIII}{-15cm + \leveltopIIIIII}
\renewcommand{\leveltopIIIIIIII}{-15cm + \leveltopIIIIIII}
\renewcommand{\leveltopIIIIIIIII}{-15cm + \leveltopIIIIIIII}
\renewcommand{\leveltopIIIIIIIIII}{-15cm + \leveltopIIIIIIIII}
\begin{tikzpicture}[scale=.2, anchor=south]
\begin{scope}[yshift=\leveltopI cm]
\matrix (line1) [column sep=1cm] {
\node[draw=black, rectangle split,  rectangle split parts=3] (sn0xda2a50){
\begin{tikzpicture}[scale=.2]
\node[circle, scale=0.75, fill] (tid0) at (4.5,1.5){};
\node[circle, scale=0.75, fill] (tid1) at (2.25,3){};
\node[circle, scale=0.75, fill, red] (tid4) at (0.75,4.5){};
\node[circle, scale=0.75, fill, red] (tid5) at (2.25,4.5){};
\node[circle, scale=0.75, fill] (tid6) at (3.75,4.5){};
\draw[](tid1) -- (tid4);
\draw[](tid1) -- (tid5);
\draw[](tid1) -- (tid6);
\node[circle, scale=0.75, fill] (tid2) at (6,3){};
\node[circle, scale=0.75, fill, red] (tid7) at (5.25,4.5){};
\node[circle, scale=0.75, fill] (tid8) at (6.75,4.5){};
\draw[](tid2) -- (tid7);
\draw[](tid2) -- (tid8);
\node[circle, scale=0.75, fill] (tid3) at (8.25,3){};
\node[circle, scale=0.75, fill] (tid9) at (8.25,4.5){};
\draw[](tid3) -- (tid9);
\draw[](tid0) -- (tid1);
\draw[](tid0) -- (tid2);
\draw[](tid0) -- (tid3);
\end{tikzpicture}
\nodepart{two}
\footnotesize{5.20508}
\nodepart{three}
\footnotesize{$33\:67$}
};
 & 
\\
};
\end{scope}
\begin{scope}[yshift=\leveltopII cm]
\matrix (line2) [column sep=1cm] {
\node[draw=black, rectangle split,  rectangle split parts=3] (sn0xda2090){
\begin{tikzpicture}[scale=.2]
\node[circle, scale=0.75, fill] (tid0) at (3.75,1.5){};
\node[circle, scale=0.75, fill] (tid1) at (2.25,3){};
\node[circle, scale=0.75, fill, red] (tid4) at (0.75,4.5){};
\node[circle, scale=0.75, fill, red] (tid5) at (2.25,4.5){};
\node[circle, scale=0.75, fill] (tid6) at (3.75,4.5){};
\draw[](tid1) -- (tid4);
\draw[](tid1) -- (tid5);
\draw[](tid1) -- (tid6);
\node[circle, scale=0.75, fill] (tid2) at (5.25,3){};
\node[circle, scale=0.75, fill, red] (tid7) at (5.25,4.5){};
\draw[](tid2) -- (tid7);
\node[circle, scale=0.75, fill] (tid3) at (6.75,3){};
\node[circle, scale=0.75, fill] (tid8) at (6.75,4.5){};
\draw[](tid3) -- (tid8);
\draw[](tid0) -- (tid1);
\draw[](tid0) -- (tid2);
\draw[](tid0) -- (tid3);
\end{tikzpicture}
\nodepart{two}
\footnotesize{4.87037}
\nodepart{three}
\footnotesize{$33\:67$}
};
 & 
\node[draw=black, rectangle split,  rectangle split parts=3] (sn0xda0d60){
\begin{tikzpicture}[scale=.2]
\node[circle, scale=0.75, fill] (tid0) at (3.75,1.5){};
\node[circle, scale=0.75, fill] (tid1) at (1.5,3){};
\node[circle, scale=0.75, fill, red] (tid4) at (0.75,4.5){};
\node[circle, scale=0.75, fill] (tid5) at (2.25,4.5){};
\draw[](tid1) -- (tid4);
\draw[](tid1) -- (tid5);
\node[circle, scale=0.75, fill] (tid2) at (4.5,3){};
\node[circle, scale=0.75, fill, red] (tid6) at (3.75,4.5){};
\node[circle, scale=0.75, fill] (tid7) at (5.25,4.5){};
\draw[](tid2) -- (tid6);
\draw[](tid2) -- (tid7);
\node[circle, scale=0.75, fill] (tid3) at (6.75,3){};
\node[circle, scale=0.75, fill, red] (tid8) at (6.75,4.5){};
\draw[](tid3) -- (tid8);
\draw[](tid0) -- (tid1);
\draw[](tid0) -- (tid2);
\draw[](tid0) -- (tid3);
\end{tikzpicture}
\nodepart{two}
\footnotesize{4.87243}
\nodepart{three}
\footnotesize{$67\:33$}
};
 & 
\\
};
\end{scope}
\begin{scope}[yshift=\leveltopIII cm]
\matrix (line3) [column sep=1cm] {
\node[draw=black, rectangle split,  rectangle split parts=3] (sn0xd9cf30){
\begin{tikzpicture}[scale=.2]
\node[circle, scale=0.75, fill] (tid0) at (3.75,1.5){};
\node[circle, scale=0.75, fill] (tid1) at (2.25,3){};
\node[circle, scale=0.75, fill, red] (tid4) at (0.75,4.5){};
\node[circle, scale=0.75, fill, red] (tid5) at (2.25,4.5){};
\node[circle, scale=0.75, fill] (tid6) at (3.75,4.5){};
\draw[](tid1) -- (tid4);
\draw[](tid1) -- (tid5);
\draw[](tid1) -- (tid6);
\node[circle, scale=0.75, fill] (tid2) at (5.25,3){};
\node[circle, scale=0.75, fill, red] (tid7) at (5.25,4.5){};
\draw[](tid2) -- (tid7);
\node[circle, scale=0.75, fill] (tid3) at (6.75,3){};
\draw[](tid0) -- (tid1);
\draw[](tid0) -- (tid2);
\draw[](tid0) -- (tid3);
\end{tikzpicture}
\nodepart{two}
\footnotesize{4.53086}
\nodepart{three}
\footnotesize{$33\:67$}
};
 & 
\node[draw=black, rectangle split,  rectangle split parts=3] (sn0xd9a9c0){
\begin{tikzpicture}[scale=.2]
\node[circle, scale=0.75, fill] (tid0) at (3,1.5){};
\node[circle, scale=0.75, fill] (tid1) at (1.5,3){};
\node[circle, scale=0.75, fill, red] (tid4) at (0.75,4.5){};
\node[circle, scale=0.75, fill] (tid5) at (2.25,4.5){};
\draw[](tid1) -- (tid4);
\draw[](tid1) -- (tid5);
\node[circle, scale=0.75, fill] (tid2) at (3.75,3){};
\node[circle, scale=0.75, fill, red] (tid6) at (3.75,4.5){};
\draw[](tid2) -- (tid6);
\node[circle, scale=0.75, fill] (tid3) at (5.25,3){};
\node[circle, scale=0.75, fill, red] (tid7) at (5.25,4.5){};
\draw[](tid3) -- (tid7);
\draw[](tid0) -- (tid1);
\draw[](tid0) -- (tid2);
\draw[](tid0) -- (tid3);
\end{tikzpicture}
\nodepart{two}
\footnotesize{4.54012}
\nodepart{three}
\footnotesize{$67\:33$}
};
 & 
\node[draw=black, rectangle split,  rectangle split parts=3] (sn0xd9f7a0){
\begin{tikzpicture}[scale=.2]
\node[circle, scale=0.75, fill] (tid0) at (3.75,1.5){};
\node[circle, scale=0.75, fill] (tid1) at (1.5,3){};
\node[circle, scale=0.75, fill, red] (tid4) at (0.75,4.5){};
\node[circle, scale=0.75, fill, red] (tid5) at (2.25,4.5){};
\draw[](tid1) -- (tid4);
\draw[](tid1) -- (tid5);
\node[circle, scale=0.75, fill] (tid2) at (4.5,3){};
\node[circle, scale=0.75, fill, red] (tid6) at (3.75,4.5){};
\node[circle, scale=0.75, fill] (tid7) at (5.25,4.5){};
\draw[](tid2) -- (tid6);
\draw[](tid2) -- (tid7);
\node[circle, scale=0.75, fill] (tid3) at (6.75,3){};
\draw[](tid0) -- (tid1);
\draw[](tid0) -- (tid2);
\draw[](tid0) -- (tid3);
\end{tikzpicture}
\nodepart{two}
\footnotesize{4.53704}
\nodepart{three}
\footnotesize{$1$}
};
 & 
\\
};
\end{scope}
\begin{scope}[yshift=\leveltopIIII cm]
\matrix (line4) [column sep=1cm] {
\node[draw=black, rectangle split,  rectangle split parts=3] (sn0xd9d6c0){
\begin{tikzpicture}[scale=.2]
\node[circle, scale=0.75, fill] (tid0) at (3.75,1.5){};
\node[circle, scale=0.75, fill] (tid1) at (2.25,3){};
\node[circle, scale=0.75, fill, red] (tid4) at (0.75,4.5){};
\node[circle, scale=0.75, fill, red] (tid5) at (2.25,4.5){};
\node[circle, scale=0.75, fill, red] (tid6) at (3.75,4.5){};
\draw[](tid1) -- (tid4);
\draw[](tid1) -- (tid5);
\draw[](tid1) -- (tid6);
\node[circle, scale=0.75, fill] (tid2) at (5.25,3){};
\node[circle, scale=0.75, fill] (tid3) at (6.75,3){};
\draw[](tid0) -- (tid1);
\draw[](tid0) -- (tid2);
\draw[](tid0) -- (tid3);
\end{tikzpicture}
\nodepart{two}
\footnotesize{4.18519}
\nodepart{three}
\footnotesize{$1$}
};
 & 
\node[draw=black, rectangle split,  rectangle split parts=3] (sn0xd9ae50){
\begin{tikzpicture}[scale=.2]
\node[circle, scale=0.75, fill] (tid0) at (3,1.5){};
\node[circle, scale=0.75, fill] (tid1) at (1.5,3){};
\node[circle, scale=0.75, fill, red] (tid4) at (0.75,4.5){};
\node[circle, scale=0.75, fill, red] (tid5) at (2.25,4.5){};
\draw[](tid1) -- (tid4);
\draw[](tid1) -- (tid5);
\node[circle, scale=0.75, fill] (tid2) at (3.75,3){};
\node[circle, scale=0.75, fill, red] (tid6) at (3.75,4.5){};
\draw[](tid2) -- (tid6);
\node[circle, scale=0.75, fill] (tid3) at (5.25,3){};
\draw[](tid0) -- (tid1);
\draw[](tid0) -- (tid2);
\draw[](tid0) -- (tid3);
\end{tikzpicture}
\nodepart{two}
\footnotesize{4.2037}
\nodepart{three}
\footnotesize{$33\:67$}
};
 & 
\node[draw=black, rectangle split,  rectangle split parts=3] (sn0xd98db0){
\begin{tikzpicture}[scale=.2]
\node[circle, scale=0.75, fill] (tid0) at (2.25,1.5){};
\node[circle, scale=0.75, fill] (tid1) at (0.75,3){};
\node[circle, scale=0.75, fill, red] (tid4) at (0.75,4.5){};
\draw[](tid1) -- (tid4);
\node[circle, scale=0.75, fill] (tid2) at (2.25,3){};
\node[circle, scale=0.75, fill, red] (tid5) at (2.25,4.5){};
\draw[](tid2) -- (tid5);
\node[circle, scale=0.75, fill] (tid3) at (3.75,3){};
\node[circle, scale=0.75, fill, red] (tid6) at (3.75,4.5){};
\draw[](tid3) -- (tid6);
\draw[](tid0) -- (tid1);
\draw[](tid0) -- (tid2);
\draw[](tid0) -- (tid3);
\end{tikzpicture}
\nodepart{two}
\footnotesize{4.21296}
\nodepart{three}
\footnotesize{$1$}
};
 & 
\\
};
\end{scope}
\begin{scope}[yshift=\leveltopIIIII cm]
\matrix (line5) [column sep=1cm] {
\node[draw=black, rectangle split,  rectangle split parts=3] (sn0xd9a340){
\begin{tikzpicture}[scale=.2]
\node[circle, scale=0.75, fill] (tid0) at (3,1.5){};
\node[circle, scale=0.75, fill] (tid1) at (1.5,3){};
\node[circle, scale=0.75, fill, red] (tid4) at (0.75,4.5){};
\node[circle, scale=0.75, fill, red] (tid5) at (2.25,4.5){};
\draw[](tid1) -- (tid4);
\draw[](tid1) -- (tid5);
\node[circle, scale=0.75, fill, red] (tid2) at (3.75,3){};
\node[circle, scale=0.75, fill] (tid3) at (5.25,3){};
\draw[](tid0) -- (tid1);
\draw[](tid0) -- (tid2);
\draw[](tid0) -- (tid3);
\end{tikzpicture}
\nodepart{two}
\footnotesize{3.85185}
\nodepart{three}
\footnotesize{$33\:67$}
};
 & 
\node[draw=black, rectangle split,  rectangle split parts=3] (sn0xd98be0){
\begin{tikzpicture}[scale=.2]
\node[circle, scale=0.75, fill] (tid0) at (2.25,1.5){};
\node[circle, scale=0.75, fill] (tid1) at (0.75,3){};
\node[circle, scale=0.75, fill, red] (tid4) at (0.75,4.5){};
\draw[](tid1) -- (tid4);
\node[circle, scale=0.75, fill] (tid2) at (2.25,3){};
\node[circle, scale=0.75, fill, red] (tid5) at (2.25,4.5){};
\draw[](tid2) -- (tid5);
\node[circle, scale=0.75, fill, red] (tid3) at (3.75,3){};
\draw[](tid0) -- (tid1);
\draw[](tid0) -- (tid2);
\draw[](tid0) -- (tid3);
\end{tikzpicture}
\nodepart{two}
\footnotesize{3.87963}
\nodepart{three}
\footnotesize{$67\:33$}
};
 & 
\\
};
\end{scope}
\begin{scope}[yshift=\leveltopIIIIII cm]
\matrix (line6) [column sep=1cm] {
\node[draw=black, rectangle split,  rectangle split parts=3] (sn0xd99950){
\begin{tikzpicture}[scale=.2]
\node[circle, scale=0.75, fill] (tid0) at (2.25,1.5){};
\node[circle, scale=0.75, fill] (tid1) at (1.5,3){};
\node[circle, scale=0.75, fill, red] (tid3) at (0.75,4.5){};
\node[circle, scale=0.75, fill, red] (tid4) at (2.25,4.5){};
\draw[](tid1) -- (tid3);
\draw[](tid1) -- (tid4);
\node[circle, scale=0.75, fill, red] (tid2) at (3.75,3){};
\draw[](tid0) -- (tid1);
\draw[](tid0) -- (tid2);
\end{tikzpicture}
\nodepart{two}
\footnotesize{3.66667}
\nodepart{three}
\footnotesize{$33\:67$}
};
 & 
\node[draw=black, rectangle split,  rectangle split parts=3] (sn0xd98b10){
\begin{tikzpicture}[scale=.2]
\node[circle, scale=0.75, fill] (tid0) at (2.25,1.5){};
\node[circle, scale=0.75, fill] (tid1) at (0.75,3){};
\node[circle, scale=0.75, fill, red] (tid4) at (0.75,4.5){};
\draw[](tid1) -- (tid4);
\node[circle, scale=0.75, fill, red] (tid2) at (2.25,3){};
\node[circle, scale=0.75, fill, red] (tid3) at (3.75,3){};
\draw[](tid0) -- (tid1);
\draw[](tid0) -- (tid2);
\draw[](tid0) -- (tid3);
\end{tikzpicture}
\nodepart{two}
\footnotesize{3.44444}
\nodepart{three}
\footnotesize{$67\:33$}
};
 & 
\node[draw=black, rectangle split,  rectangle split parts=3] (sn0xd982f0){
\begin{tikzpicture}[scale=.2]
\node[circle, scale=0.75, fill] (tid0) at (1.5,1.5){};
\node[circle, scale=0.75, fill] (tid1) at (0.75,3){};
\node[circle, scale=0.75, fill, red] (tid3) at (0.75,4.5){};
\draw[](tid1) -- (tid3);
\node[circle, scale=0.75, fill] (tid2) at (2.25,3){};
\node[circle, scale=0.75, fill, red] (tid4) at (2.25,4.5){};
\draw[](tid2) -- (tid4);
\draw[](tid0) -- (tid1);
\draw[](tid0) -- (tid2);
\end{tikzpicture}
\nodepart{two}
\footnotesize{3.75}
\nodepart{three}
\footnotesize{$1$}
};
 & 
\\
};
\end{scope}
\begin{scope}[yshift=\leveltopIIIIIII cm]
\matrix (line7) [column sep=1cm] {
\node[draw=black, rectangle split,  rectangle split parts=3] (sn0xd99040){
\begin{tikzpicture}[scale=.2]
\node[circle, scale=0.75, fill] (tid0) at (1.5,1.5){};
\node[circle, scale=0.75, fill] (tid1) at (1.5,3){};
\node[circle, scale=0.75, fill, red] (tid2) at (0.75,4.5){};
\node[circle, scale=0.75, fill, red] (tid3) at (2.25,4.5){};
\draw[](tid1) -- (tid2);
\draw[](tid1) -- (tid3);
\draw[](tid0) -- (tid1);
\end{tikzpicture}
\nodepart{two}
\footnotesize{3.5}
\nodepart{three}
\footnotesize{$1$}
};
 & 
\node[draw=black, rectangle split,  rectangle split parts=3] (sn0xd981e0){
\begin{tikzpicture}[scale=.2]
\node[circle, scale=0.75, fill] (tid0) at (1.5,1.5){};
\node[circle, scale=0.75, fill] (tid1) at (0.75,3){};
\node[circle, scale=0.75, fill, red] (tid3) at (0.75,4.5){};
\draw[](tid1) -- (tid3);
\node[circle, scale=0.75, fill, red] (tid2) at (2.25,3){};
\draw[](tid0) -- (tid1);
\draw[](tid0) -- (tid2);
\end{tikzpicture}
\nodepart{two}
\footnotesize{3.25}
\nodepart{three}
\footnotesize{$50\:50$}
};
 & 
\node[draw=black, rectangle split,  rectangle split parts=3] (sn0xd985f0){
\begin{tikzpicture}[scale=.2]
\node[circle, scale=0.75, fill] (tid0) at (2.25,1.5){};
\node[circle, scale=0.75, fill, red] (tid1) at (0.75,3){};
\node[circle, scale=0.75, fill, red] (tid2) at (2.25,3){};
\node[circle, scale=0.75, fill, red] (tid3) at (3.75,3){};
\draw[](tid0) -- (tid1);
\draw[](tid0) -- (tid2);
\draw[](tid0) -- (tid3);
\end{tikzpicture}
\nodepart{two}
\footnotesize{2.83333}
\nodepart{three}
\footnotesize{$1$}
};
 & 
\\
};
\end{scope}
\begin{scope}[yshift=\leveltopIIIIIIII cm]
\matrix (line8) [column sep=1cm] {
\node[draw=black, rectangle split,  rectangle split parts=3] (sn0xd97cb0){
\begin{tikzpicture}[scale=.2]
\node[circle, scale=0.75, fill] (tid0) at (0.75,1.5){};
\node[circle, scale=0.75, fill] (tid1) at (0.75,3){};
\node[circle, scale=0.75, fill, red] (tid2) at (0.75,4.5){};
\draw[](tid1) -- (tid2);
\draw[](tid0) -- (tid1);
\end{tikzpicture}
\nodepart{two}
\footnotesize{3}
\nodepart{three}
\footnotesize{$1$}
};
 & 
\node[draw=black, rectangle split,  rectangle split parts=3] (sn0xd97ee0){
\begin{tikzpicture}[scale=.2]
\node[circle, scale=0.75, fill] (tid0) at (1.5,1.5){};
\node[circle, scale=0.75, fill, red] (tid1) at (0.75,3){};
\node[circle, scale=0.75, fill, red] (tid2) at (2.25,3){};
\draw[](tid0) -- (tid1);
\draw[](tid0) -- (tid2);
\end{tikzpicture}
\nodepart{two}
\footnotesize{2.5}
\nodepart{three}
\footnotesize{$1$}
};
 & 
\\
};
\end{scope}
\begin{scope}[yshift=\leveltopIIIIIIIII cm]
\matrix (line9) [column sep=1cm] {
\node[draw=black, rectangle split,  rectangle split parts=3] (sn0xd88140){
\begin{tikzpicture}[scale=.2]
\node[circle, scale=0.75, fill] (tid0) at (0.75,1.5){};
\node[circle, scale=0.75, fill, red] (tid1) at (0.75,3){};
\draw[](tid0) -- (tid1);
\end{tikzpicture}
\nodepart{two}
\footnotesize{2}
\nodepart{three}
\footnotesize{$1$}
};
 & 
\\
};
\end{scope}
\begin{scope}[yshift=\leveltopIIIIIIIIII cm]
\matrix (line10) [column sep=1cm] {
\node[draw=black, rectangle split,  rectangle split parts=3] (sn0xd88070){
\begin{tikzpicture}[scale=.2]
\node[circle, scale=0.75, fill, red] (tid0) at (0.75,1.5){};
\end{tikzpicture}
\nodepart{two}
\footnotesize{1}
\nodepart{three}
\footnotesize{$$}
};
 & 
\\
};
\end{scope}
\begin{scope}[yshift=\leveltopIIIIIIIIIII cm]
\matrix (line11) [column sep=1cm] {
\\
};
\end{scope}
\draw (sn0xda2a50.south) -- (sn0xda2090.north);
\draw (sn0xda2a50.south) -- (sn0xda0d60.north);
\draw (sn0xda2090.south) -- (sn0xd9cf30.north);
\draw (sn0xda2090.south) -- (sn0xd9a9c0.north);
\draw (sn0xda0d60.south) -- (sn0xd9f7a0.north);
\draw (sn0xda0d60.south) -- (sn0xd9a9c0.north);
\draw (sn0xd9cf30.south) -- (sn0xd9d6c0.north);
\draw (sn0xd9cf30.south) -- (sn0xd9ae50.north);
\draw (sn0xd9a9c0.south) -- (sn0xd9ae50.north);
\draw (sn0xd9a9c0.south) -- (sn0xd98db0.north);
\draw (sn0xd9f7a0.south) -- (sn0xd9ae50.north);
\draw (sn0xd9d6c0.south) -- (sn0xd9a340.north);
\draw (sn0xd9ae50.south) -- (sn0xd9a340.north);
\draw (sn0xd9ae50.south) -- (sn0xd98be0.north);
\draw (sn0xd98db0.south) -- (sn0xd98be0.north);
\draw (sn0xd9a340.south) -- (sn0xd99950.north);
\draw (sn0xd9a340.south) -- (sn0xd98b10.north);
\draw (sn0xd98be0.south) -- (sn0xd982f0.north);
\draw (sn0xd98be0.south) -- (sn0xd98b10.north);
\draw (sn0xd99950.south) -- (sn0xd99040.north);
\draw (sn0xd99950.south) -- (sn0xd981e0.north);
\draw (sn0xd98b10.south) -- (sn0xd981e0.north);
\draw (sn0xd98b10.south) -- (sn0xd985f0.north);
\draw (sn0xd982f0.south) -- (sn0xd981e0.north);
\draw (sn0xd99040.south) -- (sn0xd97cb0.north);
\draw (sn0xd981e0.south) -- (sn0xd97cb0.north);
\draw (sn0xd981e0.south) -- (sn0xd97ee0.north);
\draw (sn0xd985f0.south) -- (sn0xd97ee0.north);
\draw (sn0xd97cb0.south) -- (sn0xd88140.north);
\draw (sn0xd97ee0.south) -- (sn0xd88140.north);
\draw (sn0xd88140.south) -- (sn0xd88070.north);
\end{tikzpicture}

%%% Local Variables:
%%% TeX-master: "thesis/thesis.tex"
%%% End: 
\renewcommand{\leveltopI}{-15cm + \leveltop}
\renewcommand{\leveltopII}{-15cm + \leveltopI}
\renewcommand{\leveltopIII}{-15cm + \leveltopII}
\renewcommand{\leveltopIIII}{-15cm + \leveltopIII}
\renewcommand{\leveltopIIIII}{-15cm + \leveltopIIII}
\renewcommand{\leveltopIIIIII}{-15cm + \leveltopIIIII}
\renewcommand{\leveltopIIIIIII}{-15cm + \leveltopIIIIII}
\renewcommand{\leveltopIIIIIIII}{-15cm + \leveltopIIIIIII}
\renewcommand{\leveltopIIIIIIIII}{-15cm + \leveltopIIIIIIII}
\renewcommand{\leveltopIIIIIIIIII}{-15cm + \leveltopIIIIIIIII}
\begin{tikzpicture}[scale=.2, anchor=south]
\begin{scope}[yshift=\leveltopI cm]
\matrix (line1) [column sep=1cm] {
\node[draw=black, rectangle split,  rectangle split parts=3] (sn0xda3200){
\begin{tikzpicture}[scale=.2]
\node[circle, scale=0.75, fill] (tid0) at (4.5,1.5){};
\node[circle, scale=0.75, fill] (tid1) at (2.25,3){};
\node[circle, scale=0.75, fill, red] (tid4) at (0.75,4.5){};
\node[circle, scale=0.75, fill] (tid5) at (2.25,4.5){};
\node[circle, scale=0.75, fill] (tid6) at (3.75,4.5){};
\draw[](tid1) -- (tid4);
\draw[](tid1) -- (tid5);
\draw[](tid1) -- (tid6);
\node[circle, scale=0.75, fill] (tid2) at (6,3){};
\node[circle, scale=0.75, fill, red] (tid7) at (5.25,4.5){};
\node[circle, scale=0.75, fill, red] (tid8) at (6.75,4.5){};
\draw[](tid2) -- (tid7);
\draw[](tid2) -- (tid8);
\node[circle, scale=0.75, fill] (tid3) at (8.25,3){};
\node[circle, scale=0.75, fill] (tid9) at (8.25,4.5){};
\draw[](tid3) -- (tid9);
\draw[](tid0) -- (tid1);
\draw[](tid0) -- (tid2);
\draw[](tid0) -- (tid3);
\end{tikzpicture}
\nodepart{two}
\footnotesize{5.20233}
\nodepart{three}
\footnotesize{$67\:33$}
};
 & 
\\
};
\end{scope}
\begin{scope}[yshift=\leveltopII cm]
\matrix (line2) [column sep=1cm] {
\node[draw=black, rectangle split,  rectangle split parts=3] (sn0xd9d000){
\begin{tikzpicture}[scale=.2]
\node[circle, scale=0.75, fill] (tid0) at (3.75,1.5){};
\node[circle, scale=0.75, fill] (tid1) at (2.25,3){};
\node[circle, scale=0.75, fill, red] (tid4) at (0.75,4.5){};
\node[circle, scale=0.75, fill] (tid5) at (2.25,4.5){};
\node[circle, scale=0.75, fill] (tid6) at (3.75,4.5){};
\draw[](tid1) -- (tid4);
\draw[](tid1) -- (tid5);
\draw[](tid1) -- (tid6);
\node[circle, scale=0.75, fill] (tid2) at (5.25,3){};
\node[circle, scale=0.75, fill, red] (tid7) at (5.25,4.5){};
\draw[](tid2) -- (tid7);
\node[circle, scale=0.75, fill] (tid3) at (6.75,3){};
\node[circle, scale=0.75, fill, red] (tid8) at (6.75,4.5){};
\draw[](tid3) -- (tid8);
\draw[](tid0) -- (tid1);
\draw[](tid0) -- (tid2);
\draw[](tid0) -- (tid3);
\end{tikzpicture}
\nodepart{two}
\footnotesize{4.86728}
\nodepart{three}
\footnotesize{$67\:33$}
};
 & 
\node[draw=black, rectangle split,  rectangle split parts=3] (sn0xda2f30){
\begin{tikzpicture}[scale=.2]
\node[circle, scale=0.75, fill] (tid0) at (3.75,1.5){};
\node[circle, scale=0.75, fill] (tid1) at (1.5,3){};
\node[circle, scale=0.75, fill, red] (tid4) at (0.75,4.5){};
\node[circle, scale=0.75, fill, red] (tid5) at (2.25,4.5){};
\draw[](tid1) -- (tid4);
\draw[](tid1) -- (tid5);
\node[circle, scale=0.75, fill] (tid2) at (4.5,3){};
\node[circle, scale=0.75, fill] (tid6) at (3.75,4.5){};
\node[circle, scale=0.75, fill] (tid7) at (5.25,4.5){};
\draw[](tid2) -- (tid6);
\draw[](tid2) -- (tid7);
\node[circle, scale=0.75, fill] (tid3) at (6.75,3){};
\node[circle, scale=0.75, fill, red] (tid8) at (6.75,4.5){};
\draw[](tid3) -- (tid8);
\draw[](tid0) -- (tid1);
\draw[](tid0) -- (tid2);
\draw[](tid0) -- (tid3);
\end{tikzpicture}
\nodepart{two}
\footnotesize{4.87243}
\nodepart{three}
\footnotesize{$67\:33$}
};
 & 
\\
};
\end{scope}
\begin{scope}[yshift=\leveltopIII cm]
\matrix (line3) [column sep=1cm] {
\node[draw=black, rectangle split,  rectangle split parts=3] (sn0xd9cf30){
\begin{tikzpicture}[scale=.2]
\node[circle, scale=0.75, fill] (tid0) at (3.75,1.5){};
\node[circle, scale=0.75, fill] (tid1) at (2.25,3){};
\node[circle, scale=0.75, fill, red] (tid4) at (0.75,4.5){};
\node[circle, scale=0.75, fill, red] (tid5) at (2.25,4.5){};
\node[circle, scale=0.75, fill] (tid6) at (3.75,4.5){};
\draw[](tid1) -- (tid4);
\draw[](tid1) -- (tid5);
\draw[](tid1) -- (tid6);
\node[circle, scale=0.75, fill] (tid2) at (5.25,3){};
\node[circle, scale=0.75, fill, red] (tid7) at (5.25,4.5){};
\draw[](tid2) -- (tid7);
\node[circle, scale=0.75, fill] (tid3) at (6.75,3){};
\draw[](tid0) -- (tid1);
\draw[](tid0) -- (tid2);
\draw[](tid0) -- (tid3);
\end{tikzpicture}
\nodepart{two}
\footnotesize{4.53086}
\nodepart{three}
\footnotesize{$33\:67$}
};
 & 
\node[draw=black, rectangle split,  rectangle split parts=3] (sn0xd9a9c0){
\begin{tikzpicture}[scale=.2]
\node[circle, scale=0.75, fill] (tid0) at (3,1.5){};
\node[circle, scale=0.75, fill] (tid1) at (1.5,3){};
\node[circle, scale=0.75, fill, red] (tid4) at (0.75,4.5){};
\node[circle, scale=0.75, fill] (tid5) at (2.25,4.5){};
\draw[](tid1) -- (tid4);
\draw[](tid1) -- (tid5);
\node[circle, scale=0.75, fill] (tid2) at (3.75,3){};
\node[circle, scale=0.75, fill, red] (tid6) at (3.75,4.5){};
\draw[](tid2) -- (tid6);
\node[circle, scale=0.75, fill] (tid3) at (5.25,3){};
\node[circle, scale=0.75, fill, red] (tid7) at (5.25,4.5){};
\draw[](tid3) -- (tid7);
\draw[](tid0) -- (tid1);
\draw[](tid0) -- (tid2);
\draw[](tid0) -- (tid3);
\end{tikzpicture}
\nodepart{two}
\footnotesize{4.54012}
\nodepart{three}
\footnotesize{$67\:33$}
};
 & 
\node[draw=black, rectangle split,  rectangle split parts=3] (sn0xd9f7a0){
\begin{tikzpicture}[scale=.2]
\node[circle, scale=0.75, fill] (tid0) at (3.75,1.5){};
\node[circle, scale=0.75, fill] (tid1) at (1.5,3){};
\node[circle, scale=0.75, fill, red] (tid4) at (0.75,4.5){};
\node[circle, scale=0.75, fill, red] (tid5) at (2.25,4.5){};
\draw[](tid1) -- (tid4);
\draw[](tid1) -- (tid5);
\node[circle, scale=0.75, fill] (tid2) at (4.5,3){};
\node[circle, scale=0.75, fill, red] (tid6) at (3.75,4.5){};
\node[circle, scale=0.75, fill] (tid7) at (5.25,4.5){};
\draw[](tid2) -- (tid6);
\draw[](tid2) -- (tid7);
\node[circle, scale=0.75, fill] (tid3) at (6.75,3){};
\draw[](tid0) -- (tid1);
\draw[](tid0) -- (tid2);
\draw[](tid0) -- (tid3);
\end{tikzpicture}
\nodepart{two}
\footnotesize{4.53704}
\nodepart{three}
\footnotesize{$1$}
};
 & 
\\
};
\end{scope}
\begin{scope}[yshift=\leveltopIIII cm]
\matrix (line4) [column sep=1cm] {
\node[draw=black, rectangle split,  rectangle split parts=3] (sn0xd9d6c0){
\begin{tikzpicture}[scale=.2]
\node[circle, scale=0.75, fill] (tid0) at (3.75,1.5){};
\node[circle, scale=0.75, fill] (tid1) at (2.25,3){};
\node[circle, scale=0.75, fill, red] (tid4) at (0.75,4.5){};
\node[circle, scale=0.75, fill, red] (tid5) at (2.25,4.5){};
\node[circle, scale=0.75, fill, red] (tid6) at (3.75,4.5){};
\draw[](tid1) -- (tid4);
\draw[](tid1) -- (tid5);
\draw[](tid1) -- (tid6);
\node[circle, scale=0.75, fill] (tid2) at (5.25,3){};
\node[circle, scale=0.75, fill] (tid3) at (6.75,3){};
\draw[](tid0) -- (tid1);
\draw[](tid0) -- (tid2);
\draw[](tid0) -- (tid3);
\end{tikzpicture}
\nodepart{two}
\footnotesize{4.18519}
\nodepart{three}
\footnotesize{$1$}
};
 & 
\node[draw=black, rectangle split,  rectangle split parts=3] (sn0xd9ae50){
\begin{tikzpicture}[scale=.2]
\node[circle, scale=0.75, fill] (tid0) at (3,1.5){};
\node[circle, scale=0.75, fill] (tid1) at (1.5,3){};
\node[circle, scale=0.75, fill, red] (tid4) at (0.75,4.5){};
\node[circle, scale=0.75, fill, red] (tid5) at (2.25,4.5){};
\draw[](tid1) -- (tid4);
\draw[](tid1) -- (tid5);
\node[circle, scale=0.75, fill] (tid2) at (3.75,3){};
\node[circle, scale=0.75, fill, red] (tid6) at (3.75,4.5){};
\draw[](tid2) -- (tid6);
\node[circle, scale=0.75, fill] (tid3) at (5.25,3){};
\draw[](tid0) -- (tid1);
\draw[](tid0) -- (tid2);
\draw[](tid0) -- (tid3);
\end{tikzpicture}
\nodepart{two}
\footnotesize{4.2037}
\nodepart{three}
\footnotesize{$33\:67$}
};
 & 
\node[draw=black, rectangle split,  rectangle split parts=3] (sn0xd98db0){
\begin{tikzpicture}[scale=.2]
\node[circle, scale=0.75, fill] (tid0) at (2.25,1.5){};
\node[circle, scale=0.75, fill] (tid1) at (0.75,3){};
\node[circle, scale=0.75, fill, red] (tid4) at (0.75,4.5){};
\draw[](tid1) -- (tid4);
\node[circle, scale=0.75, fill] (tid2) at (2.25,3){};
\node[circle, scale=0.75, fill, red] (tid5) at (2.25,4.5){};
\draw[](tid2) -- (tid5);
\node[circle, scale=0.75, fill] (tid3) at (3.75,3){};
\node[circle, scale=0.75, fill, red] (tid6) at (3.75,4.5){};
\draw[](tid3) -- (tid6);
\draw[](tid0) -- (tid1);
\draw[](tid0) -- (tid2);
\draw[](tid0) -- (tid3);
\end{tikzpicture}
\nodepart{two}
\footnotesize{4.21296}
\nodepart{three}
\footnotesize{$1$}
};
 & 
\\
};
\end{scope}
\begin{scope}[yshift=\leveltopIIIII cm]
\matrix (line5) [column sep=1cm] {
\node[draw=black, rectangle split,  rectangle split parts=3] (sn0xd9a340){
\begin{tikzpicture}[scale=.2]
\node[circle, scale=0.75, fill] (tid0) at (3,1.5){};
\node[circle, scale=0.75, fill] (tid1) at (1.5,3){};
\node[circle, scale=0.75, fill, red] (tid4) at (0.75,4.5){};
\node[circle, scale=0.75, fill, red] (tid5) at (2.25,4.5){};
\draw[](tid1) -- (tid4);
\draw[](tid1) -- (tid5);
\node[circle, scale=0.75, fill, red] (tid2) at (3.75,3){};
\node[circle, scale=0.75, fill] (tid3) at (5.25,3){};
\draw[](tid0) -- (tid1);
\draw[](tid0) -- (tid2);
\draw[](tid0) -- (tid3);
\end{tikzpicture}
\nodepart{two}
\footnotesize{3.85185}
\nodepart{three}
\footnotesize{$33\:67$}
};
 & 
\node[draw=black, rectangle split,  rectangle split parts=3] (sn0xd98be0){
\begin{tikzpicture}[scale=.2]
\node[circle, scale=0.75, fill] (tid0) at (2.25,1.5){};
\node[circle, scale=0.75, fill] (tid1) at (0.75,3){};
\node[circle, scale=0.75, fill, red] (tid4) at (0.75,4.5){};
\draw[](tid1) -- (tid4);
\node[circle, scale=0.75, fill] (tid2) at (2.25,3){};
\node[circle, scale=0.75, fill, red] (tid5) at (2.25,4.5){};
\draw[](tid2) -- (tid5);
\node[circle, scale=0.75, fill, red] (tid3) at (3.75,3){};
\draw[](tid0) -- (tid1);
\draw[](tid0) -- (tid2);
\draw[](tid0) -- (tid3);
\end{tikzpicture}
\nodepart{two}
\footnotesize{3.87963}
\nodepart{three}
\footnotesize{$67\:33$}
};
 & 
\\
};
\end{scope}
\begin{scope}[yshift=\leveltopIIIIII cm]
\matrix (line6) [column sep=1cm] {
\node[draw=black, rectangle split,  rectangle split parts=3] (sn0xd99950){
\begin{tikzpicture}[scale=.2]
\node[circle, scale=0.75, fill] (tid0) at (2.25,1.5){};
\node[circle, scale=0.75, fill] (tid1) at (1.5,3){};
\node[circle, scale=0.75, fill, red] (tid3) at (0.75,4.5){};
\node[circle, scale=0.75, fill, red] (tid4) at (2.25,4.5){};
\draw[](tid1) -- (tid3);
\draw[](tid1) -- (tid4);
\node[circle, scale=0.75, fill, red] (tid2) at (3.75,3){};
\draw[](tid0) -- (tid1);
\draw[](tid0) -- (tid2);
\end{tikzpicture}
\nodepart{two}
\footnotesize{3.66667}
\nodepart{three}
\footnotesize{$33\:67$}
};
 & 
\node[draw=black, rectangle split,  rectangle split parts=3] (sn0xd98b10){
\begin{tikzpicture}[scale=.2]
\node[circle, scale=0.75, fill] (tid0) at (2.25,1.5){};
\node[circle, scale=0.75, fill] (tid1) at (0.75,3){};
\node[circle, scale=0.75, fill, red] (tid4) at (0.75,4.5){};
\draw[](tid1) -- (tid4);
\node[circle, scale=0.75, fill, red] (tid2) at (2.25,3){};
\node[circle, scale=0.75, fill, red] (tid3) at (3.75,3){};
\draw[](tid0) -- (tid1);
\draw[](tid0) -- (tid2);
\draw[](tid0) -- (tid3);
\end{tikzpicture}
\nodepart{two}
\footnotesize{3.44444}
\nodepart{three}
\footnotesize{$67\:33$}
};
 & 
\node[draw=black, rectangle split,  rectangle split parts=3] (sn0xd982f0){
\begin{tikzpicture}[scale=.2]
\node[circle, scale=0.75, fill] (tid0) at (1.5,1.5){};
\node[circle, scale=0.75, fill] (tid1) at (0.75,3){};
\node[circle, scale=0.75, fill, red] (tid3) at (0.75,4.5){};
\draw[](tid1) -- (tid3);
\node[circle, scale=0.75, fill] (tid2) at (2.25,3){};
\node[circle, scale=0.75, fill, red] (tid4) at (2.25,4.5){};
\draw[](tid2) -- (tid4);
\draw[](tid0) -- (tid1);
\draw[](tid0) -- (tid2);
\end{tikzpicture}
\nodepart{two}
\footnotesize{3.75}
\nodepart{three}
\footnotesize{$1$}
};
 & 
\\
};
\end{scope}
\begin{scope}[yshift=\leveltopIIIIIII cm]
\matrix (line7) [column sep=1cm] {
\node[draw=black, rectangle split,  rectangle split parts=3] (sn0xd99040){
\begin{tikzpicture}[scale=.2]
\node[circle, scale=0.75, fill] (tid0) at (1.5,1.5){};
\node[circle, scale=0.75, fill] (tid1) at (1.5,3){};
\node[circle, scale=0.75, fill, red] (tid2) at (0.75,4.5){};
\node[circle, scale=0.75, fill, red] (tid3) at (2.25,4.5){};
\draw[](tid1) -- (tid2);
\draw[](tid1) -- (tid3);
\draw[](tid0) -- (tid1);
\end{tikzpicture}
\nodepart{two}
\footnotesize{3.5}
\nodepart{three}
\footnotesize{$1$}
};
 & 
\node[draw=black, rectangle split,  rectangle split parts=3] (sn0xd981e0){
\begin{tikzpicture}[scale=.2]
\node[circle, scale=0.75, fill] (tid0) at (1.5,1.5){};
\node[circle, scale=0.75, fill] (tid1) at (0.75,3){};
\node[circle, scale=0.75, fill, red] (tid3) at (0.75,4.5){};
\draw[](tid1) -- (tid3);
\node[circle, scale=0.75, fill, red] (tid2) at (2.25,3){};
\draw[](tid0) -- (tid1);
\draw[](tid0) -- (tid2);
\end{tikzpicture}
\nodepart{two}
\footnotesize{3.25}
\nodepart{three}
\footnotesize{$50\:50$}
};
 & 
\node[draw=black, rectangle split,  rectangle split parts=3] (sn0xd985f0){
\begin{tikzpicture}[scale=.2]
\node[circle, scale=0.75, fill] (tid0) at (2.25,1.5){};
\node[circle, scale=0.75, fill, red] (tid1) at (0.75,3){};
\node[circle, scale=0.75, fill, red] (tid2) at (2.25,3){};
\node[circle, scale=0.75, fill, red] (tid3) at (3.75,3){};
\draw[](tid0) -- (tid1);
\draw[](tid0) -- (tid2);
\draw[](tid0) -- (tid3);
\end{tikzpicture}
\nodepart{two}
\footnotesize{2.83333}
\nodepart{three}
\footnotesize{$1$}
};
 & 
\\
};
\end{scope}
\begin{scope}[yshift=\leveltopIIIIIIII cm]
\matrix (line8) [column sep=1cm] {
\node[draw=black, rectangle split,  rectangle split parts=3] (sn0xd97cb0){
\begin{tikzpicture}[scale=.2]
\node[circle, scale=0.75, fill] (tid0) at (0.75,1.5){};
\node[circle, scale=0.75, fill] (tid1) at (0.75,3){};
\node[circle, scale=0.75, fill, red] (tid2) at (0.75,4.5){};
\draw[](tid1) -- (tid2);
\draw[](tid0) -- (tid1);
\end{tikzpicture}
\nodepart{two}
\footnotesize{3}
\nodepart{three}
\footnotesize{$1$}
};
 & 
\node[draw=black, rectangle split,  rectangle split parts=3] (sn0xd97ee0){
\begin{tikzpicture}[scale=.2]
\node[circle, scale=0.75, fill] (tid0) at (1.5,1.5){};
\node[circle, scale=0.75, fill, red] (tid1) at (0.75,3){};
\node[circle, scale=0.75, fill, red] (tid2) at (2.25,3){};
\draw[](tid0) -- (tid1);
\draw[](tid0) -- (tid2);
\end{tikzpicture}
\nodepart{two}
\footnotesize{2.5}
\nodepart{three}
\footnotesize{$1$}
};
 & 
\\
};
\end{scope}
\begin{scope}[yshift=\leveltopIIIIIIIII cm]
\matrix (line9) [column sep=1cm] {
\node[draw=black, rectangle split,  rectangle split parts=3] (sn0xd88140){
\begin{tikzpicture}[scale=.2]
\node[circle, scale=0.75, fill] (tid0) at (0.75,1.5){};
\node[circle, scale=0.75, fill, red] (tid1) at (0.75,3){};
\draw[](tid0) -- (tid1);
\end{tikzpicture}
\nodepart{two}
\footnotesize{2}
\nodepart{three}
\footnotesize{$1$}
};
 & 
\\
};
\end{scope}
\begin{scope}[yshift=\leveltopIIIIIIIIII cm]
\matrix (line10) [column sep=1cm] {
\node[draw=black, rectangle split,  rectangle split parts=3] (sn0xd88070){
\begin{tikzpicture}[scale=.2]
\node[circle, scale=0.75, fill, red] (tid0) at (0.75,1.5){};
\end{tikzpicture}
\nodepart{two}
\footnotesize{1}
\nodepart{three}
\footnotesize{$$}
};
 & 
\\
};
\end{scope}
\begin{scope}[yshift=\leveltopIIIIIIIIIII cm]
\matrix (line11) [column sep=1cm] {
\\
};
\end{scope}
\draw (sn0xda3200.south) -- (sn0xd9d000.north);
\draw (sn0xda3200.south) -- (sn0xda2f30.north);
\draw (sn0xd9d000.south) -- (sn0xd9cf30.north);
\draw (sn0xd9d000.south) -- (sn0xd9a9c0.north);
\draw (sn0xda2f30.south) -- (sn0xd9f7a0.north);
\draw (sn0xda2f30.south) -- (sn0xd9a9c0.north);
\draw (sn0xd9cf30.south) -- (sn0xd9d6c0.north);
\draw (sn0xd9cf30.south) -- (sn0xd9ae50.north);
\draw (sn0xd9a9c0.south) -- (sn0xd9ae50.north);
\draw (sn0xd9a9c0.south) -- (sn0xd98db0.north);
\draw (sn0xd9f7a0.south) -- (sn0xd9ae50.north);
\draw (sn0xd9d6c0.south) -- (sn0xd9a340.north);
\draw (sn0xd9ae50.south) -- (sn0xd9a340.north);
\draw (sn0xd9ae50.south) -- (sn0xd98be0.north);
\draw (sn0xd98db0.south) -- (sn0xd98be0.north);
\draw (sn0xd9a340.south) -- (sn0xd99950.north);
\draw (sn0xd9a340.south) -- (sn0xd98b10.north);
\draw (sn0xd98be0.south) -- (sn0xd982f0.north);
\draw (sn0xd98be0.south) -- (sn0xd98b10.north);
\draw (sn0xd99950.south) -- (sn0xd99040.north);
\draw (sn0xd99950.south) -- (sn0xd981e0.north);
\draw (sn0xd98b10.south) -- (sn0xd981e0.north);
\draw (sn0xd98b10.south) -- (sn0xd985f0.north);
\draw (sn0xd982f0.south) -- (sn0xd981e0.north);
\draw (sn0xd99040.south) -- (sn0xd97cb0.north);
\draw (sn0xd981e0.south) -- (sn0xd97cb0.north);
\draw (sn0xd981e0.south) -- (sn0xd97ee0.north);
\draw (sn0xd985f0.south) -- (sn0xd97ee0.north);
\draw (sn0xd97cb0.south) -- (sn0xd88140.north);
\draw (sn0xd97ee0.south) -- (sn0xd88140.north);
\draw (sn0xd88140.south) -- (sn0xd88070.north);
\end{tikzpicture}

%%% Local Variables:
%%% TeX-master: "thesis/thesis.tex"
%%% End: 
\renewcommand{\leveltopI}{-15cm + \leveltop}
\renewcommand{\leveltopII}{-15cm + \leveltopI}
\renewcommand{\leveltopIII}{-15cm + \leveltopII}
\renewcommand{\leveltopIIII}{-15cm + \leveltopIII}
\renewcommand{\leveltopIIIII}{-15cm + \leveltopIIII}
\renewcommand{\leveltopIIIIII}{-15cm + \leveltopIIIII}
\renewcommand{\leveltopIIIIIII}{-15cm + \leveltopIIIIII}
\renewcommand{\leveltopIIIIIIII}{-15cm + \leveltopIIIIIII}
\renewcommand{\leveltopIIIIIIIII}{-15cm + \leveltopIIIIIIII}
\renewcommand{\leveltopIIIIIIIIII}{-15cm + \leveltopIIIIIIIII}
\begin{tikzpicture}[scale=.2, anchor=south]
\begin{scope}[yshift=\leveltopI cm]
\matrix (line1) [column sep=1cm] {
\node[draw=black, rectangle split,  rectangle split parts=3] (sn0xda3ca0){
\begin{tikzpicture}[scale=.2]
\node[circle, scale=0.75, fill] (tid0) at (4.5,1.5){};
\node[circle, scale=0.75, fill] (tid1) at (2.25,3){};
\node[circle, scale=0.75, fill, red] (tid4) at (0.75,4.5){};
\node[circle, scale=0.75, fill, red] (tid5) at (2.25,4.5){};
\node[circle, scale=0.75, fill, red] (tid6) at (3.75,4.5){};
\draw[](tid1) -- (tid4);
\draw[](tid1) -- (tid5);
\draw[](tid1) -- (tid6);
\node[circle, scale=0.75, fill] (tid2) at (6,3){};
\node[circle, scale=0.75, fill] (tid7) at (5.25,4.5){};
\node[circle, scale=0.75, fill] (tid8) at (6.75,4.5){};
\draw[](tid2) -- (tid7);
\draw[](tid2) -- (tid8);
\node[circle, scale=0.75, fill] (tid3) at (8.25,3){};
\node[circle, scale=0.75, fill] (tid9) at (8.25,4.5){};
\draw[](tid3) -- (tid9);
\draw[](tid0) -- (tid1);
\draw[](tid0) -- (tid2);
\draw[](tid0) -- (tid3);
\end{tikzpicture}
\nodepart{two}
\footnotesize{5.20576}
\nodepart{three}
\footnotesize{$1$}
};
 & 
\\
};
\end{scope}
\begin{scope}[yshift=\leveltopII cm]
\matrix (line2) [column sep=1cm] {
\node[draw=black, rectangle split,  rectangle split parts=3] (sn0xda2f30){
\begin{tikzpicture}[scale=.2]
\node[circle, scale=0.75, fill] (tid0) at (3.75,1.5){};
\node[circle, scale=0.75, fill] (tid1) at (1.5,3){};
\node[circle, scale=0.75, fill, red] (tid4) at (0.75,4.5){};
\node[circle, scale=0.75, fill, red] (tid5) at (2.25,4.5){};
\draw[](tid1) -- (tid4);
\draw[](tid1) -- (tid5);
\node[circle, scale=0.75, fill] (tid2) at (4.5,3){};
\node[circle, scale=0.75, fill] (tid6) at (3.75,4.5){};
\node[circle, scale=0.75, fill] (tid7) at (5.25,4.5){};
\draw[](tid2) -- (tid6);
\draw[](tid2) -- (tid7);
\node[circle, scale=0.75, fill] (tid3) at (6.75,3){};
\node[circle, scale=0.75, fill, red] (tid8) at (6.75,4.5){};
\draw[](tid3) -- (tid8);
\draw[](tid0) -- (tid1);
\draw[](tid0) -- (tid2);
\draw[](tid0) -- (tid3);
\end{tikzpicture}
\nodepart{two}
\footnotesize{4.87243}
\nodepart{three}
\footnotesize{$33\:67$}
};
 & 
\\
};
\end{scope}
\begin{scope}[yshift=\leveltopIII cm]
\matrix (line3) [column sep=1cm] {
\node[draw=black, rectangle split,  rectangle split parts=3] (sn0xd9f7a0){
\begin{tikzpicture}[scale=.2]
\node[circle, scale=0.75, fill] (tid0) at (3.75,1.5){};
\node[circle, scale=0.75, fill] (tid1) at (1.5,3){};
\node[circle, scale=0.75, fill, red] (tid4) at (0.75,4.5){};
\node[circle, scale=0.75, fill, red] (tid5) at (2.25,4.5){};
\draw[](tid1) -- (tid4);
\draw[](tid1) -- (tid5);
\node[circle, scale=0.75, fill] (tid2) at (4.5,3){};
\node[circle, scale=0.75, fill, red] (tid6) at (3.75,4.5){};
\node[circle, scale=0.75, fill] (tid7) at (5.25,4.5){};
\draw[](tid2) -- (tid6);
\draw[](tid2) -- (tid7);
\node[circle, scale=0.75, fill] (tid3) at (6.75,3){};
\draw[](tid0) -- (tid1);
\draw[](tid0) -- (tid2);
\draw[](tid0) -- (tid3);
\end{tikzpicture}
\nodepart{two}
\footnotesize{4.53704}
\nodepart{three}
\footnotesize{$1$}
};
 & 
\node[draw=black, rectangle split,  rectangle split parts=3] (sn0xd9a9c0){
\begin{tikzpicture}[scale=.2]
\node[circle, scale=0.75, fill] (tid0) at (3,1.5){};
\node[circle, scale=0.75, fill] (tid1) at (1.5,3){};
\node[circle, scale=0.75, fill, red] (tid4) at (0.75,4.5){};
\node[circle, scale=0.75, fill] (tid5) at (2.25,4.5){};
\draw[](tid1) -- (tid4);
\draw[](tid1) -- (tid5);
\node[circle, scale=0.75, fill] (tid2) at (3.75,3){};
\node[circle, scale=0.75, fill, red] (tid6) at (3.75,4.5){};
\draw[](tid2) -- (tid6);
\node[circle, scale=0.75, fill] (tid3) at (5.25,3){};
\node[circle, scale=0.75, fill, red] (tid7) at (5.25,4.5){};
\draw[](tid3) -- (tid7);
\draw[](tid0) -- (tid1);
\draw[](tid0) -- (tid2);
\draw[](tid0) -- (tid3);
\end{tikzpicture}
\nodepart{two}
\footnotesize{4.54012}
\nodepart{three}
\footnotesize{$67\:33$}
};
 & 
\\
};
\end{scope}
\begin{scope}[yshift=\leveltopIIII cm]
\matrix (line4) [column sep=1cm] {
\node[draw=black, rectangle split,  rectangle split parts=3] (sn0xd9ae50){
\begin{tikzpicture}[scale=.2]
\node[circle, scale=0.75, fill] (tid0) at (3,1.5){};
\node[circle, scale=0.75, fill] (tid1) at (1.5,3){};
\node[circle, scale=0.75, fill, red] (tid4) at (0.75,4.5){};
\node[circle, scale=0.75, fill, red] (tid5) at (2.25,4.5){};
\draw[](tid1) -- (tid4);
\draw[](tid1) -- (tid5);
\node[circle, scale=0.75, fill] (tid2) at (3.75,3){};
\node[circle, scale=0.75, fill, red] (tid6) at (3.75,4.5){};
\draw[](tid2) -- (tid6);
\node[circle, scale=0.75, fill] (tid3) at (5.25,3){};
\draw[](tid0) -- (tid1);
\draw[](tid0) -- (tid2);
\draw[](tid0) -- (tid3);
\end{tikzpicture}
\nodepart{two}
\footnotesize{4.2037}
\nodepart{three}
\footnotesize{$33\:67$}
};
 & 
\node[draw=black, rectangle split,  rectangle split parts=3] (sn0xd98db0){
\begin{tikzpicture}[scale=.2]
\node[circle, scale=0.75, fill] (tid0) at (2.25,1.5){};
\node[circle, scale=0.75, fill] (tid1) at (0.75,3){};
\node[circle, scale=0.75, fill, red] (tid4) at (0.75,4.5){};
\draw[](tid1) -- (tid4);
\node[circle, scale=0.75, fill] (tid2) at (2.25,3){};
\node[circle, scale=0.75, fill, red] (tid5) at (2.25,4.5){};
\draw[](tid2) -- (tid5);
\node[circle, scale=0.75, fill] (tid3) at (3.75,3){};
\node[circle, scale=0.75, fill, red] (tid6) at (3.75,4.5){};
\draw[](tid3) -- (tid6);
\draw[](tid0) -- (tid1);
\draw[](tid0) -- (tid2);
\draw[](tid0) -- (tid3);
\end{tikzpicture}
\nodepart{two}
\footnotesize{4.21296}
\nodepart{three}
\footnotesize{$1$}
};
 & 
\\
};
\end{scope}
\begin{scope}[yshift=\leveltopIIIII cm]
\matrix (line5) [column sep=1cm] {
\node[draw=black, rectangle split,  rectangle split parts=3] (sn0xd9a340){
\begin{tikzpicture}[scale=.2]
\node[circle, scale=0.75, fill] (tid0) at (3,1.5){};
\node[circle, scale=0.75, fill] (tid1) at (1.5,3){};
\node[circle, scale=0.75, fill, red] (tid4) at (0.75,4.5){};
\node[circle, scale=0.75, fill, red] (tid5) at (2.25,4.5){};
\draw[](tid1) -- (tid4);
\draw[](tid1) -- (tid5);
\node[circle, scale=0.75, fill, red] (tid2) at (3.75,3){};
\node[circle, scale=0.75, fill] (tid3) at (5.25,3){};
\draw[](tid0) -- (tid1);
\draw[](tid0) -- (tid2);
\draw[](tid0) -- (tid3);
\end{tikzpicture}
\nodepart{two}
\footnotesize{3.85185}
\nodepart{three}
\footnotesize{$33\:67$}
};
 & 
\node[draw=black, rectangle split,  rectangle split parts=3] (sn0xd98be0){
\begin{tikzpicture}[scale=.2]
\node[circle, scale=0.75, fill] (tid0) at (2.25,1.5){};
\node[circle, scale=0.75, fill] (tid1) at (0.75,3){};
\node[circle, scale=0.75, fill, red] (tid4) at (0.75,4.5){};
\draw[](tid1) -- (tid4);
\node[circle, scale=0.75, fill] (tid2) at (2.25,3){};
\node[circle, scale=0.75, fill, red] (tid5) at (2.25,4.5){};
\draw[](tid2) -- (tid5);
\node[circle, scale=0.75, fill, red] (tid3) at (3.75,3){};
\draw[](tid0) -- (tid1);
\draw[](tid0) -- (tid2);
\draw[](tid0) -- (tid3);
\end{tikzpicture}
\nodepart{two}
\footnotesize{3.87963}
\nodepart{three}
\footnotesize{$67\:33$}
};
 & 
\\
};
\end{scope}
\begin{scope}[yshift=\leveltopIIIIII cm]
\matrix (line6) [column sep=1cm] {
\node[draw=black, rectangle split,  rectangle split parts=3] (sn0xd99950){
\begin{tikzpicture}[scale=.2]
\node[circle, scale=0.75, fill] (tid0) at (2.25,1.5){};
\node[circle, scale=0.75, fill] (tid1) at (1.5,3){};
\node[circle, scale=0.75, fill, red] (tid3) at (0.75,4.5){};
\node[circle, scale=0.75, fill, red] (tid4) at (2.25,4.5){};
\draw[](tid1) -- (tid3);
\draw[](tid1) -- (tid4);
\node[circle, scale=0.75, fill, red] (tid2) at (3.75,3){};
\draw[](tid0) -- (tid1);
\draw[](tid0) -- (tid2);
\end{tikzpicture}
\nodepart{two}
\footnotesize{3.66667}
\nodepart{three}
\footnotesize{$33\:67$}
};
 & 
\node[draw=black, rectangle split,  rectangle split parts=3] (sn0xd98b10){
\begin{tikzpicture}[scale=.2]
\node[circle, scale=0.75, fill] (tid0) at (2.25,1.5){};
\node[circle, scale=0.75, fill] (tid1) at (0.75,3){};
\node[circle, scale=0.75, fill, red] (tid4) at (0.75,4.5){};
\draw[](tid1) -- (tid4);
\node[circle, scale=0.75, fill, red] (tid2) at (2.25,3){};
\node[circle, scale=0.75, fill, red] (tid3) at (3.75,3){};
\draw[](tid0) -- (tid1);
\draw[](tid0) -- (tid2);
\draw[](tid0) -- (tid3);
\end{tikzpicture}
\nodepart{two}
\footnotesize{3.44444}
\nodepart{three}
\footnotesize{$67\:33$}
};
 & 
\node[draw=black, rectangle split,  rectangle split parts=3] (sn0xd982f0){
\begin{tikzpicture}[scale=.2]
\node[circle, scale=0.75, fill] (tid0) at (1.5,1.5){};
\node[circle, scale=0.75, fill] (tid1) at (0.75,3){};
\node[circle, scale=0.75, fill, red] (tid3) at (0.75,4.5){};
\draw[](tid1) -- (tid3);
\node[circle, scale=0.75, fill] (tid2) at (2.25,3){};
\node[circle, scale=0.75, fill, red] (tid4) at (2.25,4.5){};
\draw[](tid2) -- (tid4);
\draw[](tid0) -- (tid1);
\draw[](tid0) -- (tid2);
\end{tikzpicture}
\nodepart{two}
\footnotesize{3.75}
\nodepart{three}
\footnotesize{$1$}
};
 & 
\\
};
\end{scope}
\begin{scope}[yshift=\leveltopIIIIIII cm]
\matrix (line7) [column sep=1cm] {
\node[draw=black, rectangle split,  rectangle split parts=3] (sn0xd99040){
\begin{tikzpicture}[scale=.2]
\node[circle, scale=0.75, fill] (tid0) at (1.5,1.5){};
\node[circle, scale=0.75, fill] (tid1) at (1.5,3){};
\node[circle, scale=0.75, fill, red] (tid2) at (0.75,4.5){};
\node[circle, scale=0.75, fill, red] (tid3) at (2.25,4.5){};
\draw[](tid1) -- (tid2);
\draw[](tid1) -- (tid3);
\draw[](tid0) -- (tid1);
\end{tikzpicture}
\nodepart{two}
\footnotesize{3.5}
\nodepart{three}
\footnotesize{$1$}
};
 & 
\node[draw=black, rectangle split,  rectangle split parts=3] (sn0xd981e0){
\begin{tikzpicture}[scale=.2]
\node[circle, scale=0.75, fill] (tid0) at (1.5,1.5){};
\node[circle, scale=0.75, fill] (tid1) at (0.75,3){};
\node[circle, scale=0.75, fill, red] (tid3) at (0.75,4.5){};
\draw[](tid1) -- (tid3);
\node[circle, scale=0.75, fill, red] (tid2) at (2.25,3){};
\draw[](tid0) -- (tid1);
\draw[](tid0) -- (tid2);
\end{tikzpicture}
\nodepart{two}
\footnotesize{3.25}
\nodepart{three}
\footnotesize{$50\:50$}
};
 & 
\node[draw=black, rectangle split,  rectangle split parts=3] (sn0xd985f0){
\begin{tikzpicture}[scale=.2]
\node[circle, scale=0.75, fill] (tid0) at (2.25,1.5){};
\node[circle, scale=0.75, fill, red] (tid1) at (0.75,3){};
\node[circle, scale=0.75, fill, red] (tid2) at (2.25,3){};
\node[circle, scale=0.75, fill, red] (tid3) at (3.75,3){};
\draw[](tid0) -- (tid1);
\draw[](tid0) -- (tid2);
\draw[](tid0) -- (tid3);
\end{tikzpicture}
\nodepart{two}
\footnotesize{2.83333}
\nodepart{three}
\footnotesize{$1$}
};
 & 
\\
};
\end{scope}
\begin{scope}[yshift=\leveltopIIIIIIII cm]
\matrix (line8) [column sep=1cm] {
\node[draw=black, rectangle split,  rectangle split parts=3] (sn0xd97cb0){
\begin{tikzpicture}[scale=.2]
\node[circle, scale=0.75, fill] (tid0) at (0.75,1.5){};
\node[circle, scale=0.75, fill] (tid1) at (0.75,3){};
\node[circle, scale=0.75, fill, red] (tid2) at (0.75,4.5){};
\draw[](tid1) -- (tid2);
\draw[](tid0) -- (tid1);
\end{tikzpicture}
\nodepart{two}
\footnotesize{3}
\nodepart{three}
\footnotesize{$1$}
};
 & 
\node[draw=black, rectangle split,  rectangle split parts=3] (sn0xd97ee0){
\begin{tikzpicture}[scale=.2]
\node[circle, scale=0.75, fill] (tid0) at (1.5,1.5){};
\node[circle, scale=0.75, fill, red] (tid1) at (0.75,3){};
\node[circle, scale=0.75, fill, red] (tid2) at (2.25,3){};
\draw[](tid0) -- (tid1);
\draw[](tid0) -- (tid2);
\end{tikzpicture}
\nodepart{two}
\footnotesize{2.5}
\nodepart{three}
\footnotesize{$1$}
};
 & 
\\
};
\end{scope}
\begin{scope}[yshift=\leveltopIIIIIIIII cm]
\matrix (line9) [column sep=1cm] {
\node[draw=black, rectangle split,  rectangle split parts=3] (sn0xd88140){
\begin{tikzpicture}[scale=.2]
\node[circle, scale=0.75, fill] (tid0) at (0.75,1.5){};
\node[circle, scale=0.75, fill, red] (tid1) at (0.75,3){};
\draw[](tid0) -- (tid1);
\end{tikzpicture}
\nodepart{two}
\footnotesize{2}
\nodepart{three}
\footnotesize{$1$}
};
 & 
\\
};
\end{scope}
\begin{scope}[yshift=\leveltopIIIIIIIIII cm]
\matrix (line10) [column sep=1cm] {
\node[draw=black, rectangle split,  rectangle split parts=3] (sn0xd88070){
\begin{tikzpicture}[scale=.2]
\node[circle, scale=0.75, fill, red] (tid0) at (0.75,1.5){};
\end{tikzpicture}
\nodepart{two}
\footnotesize{1}
\nodepart{three}
\footnotesize{$$}
};
 & 
\\
};
\end{scope}
\begin{scope}[yshift=\leveltopIIIIIIIIIII cm]
\matrix (line11) [column sep=1cm] {
\\
};
\end{scope}
\draw (sn0xda3ca0.south) -- (sn0xda2f30.north);
\draw (sn0xda2f30.south) -- (sn0xd9f7a0.north);
\draw (sn0xda2f30.south) -- (sn0xd9a9c0.north);
\draw (sn0xd9f7a0.south) -- (sn0xd9ae50.north);
\draw (sn0xd9a9c0.south) -- (sn0xd9ae50.north);
\draw (sn0xd9a9c0.south) -- (sn0xd98db0.north);
\draw (sn0xd9ae50.south) -- (sn0xd9a340.north);
\draw (sn0xd9ae50.south) -- (sn0xd98be0.north);
\draw (sn0xd98db0.south) -- (sn0xd98be0.north);
\draw (sn0xd9a340.south) -- (sn0xd99950.north);
\draw (sn0xd9a340.south) -- (sn0xd98b10.north);
\draw (sn0xd98be0.south) -- (sn0xd982f0.north);
\draw (sn0xd98be0.south) -- (sn0xd98b10.north);
\draw (sn0xd99950.south) -- (sn0xd99040.north);
\draw (sn0xd99950.south) -- (sn0xd981e0.north);
\draw (sn0xd98b10.south) -- (sn0xd981e0.north);
\draw (sn0xd98b10.south) -- (sn0xd985f0.north);
\draw (sn0xd982f0.south) -- (sn0xd981e0.north);
\draw (sn0xd99040.south) -- (sn0xd97cb0.north);
\draw (sn0xd981e0.south) -- (sn0xd97cb0.north);
\draw (sn0xd981e0.south) -- (sn0xd97ee0.north);
\draw (sn0xd985f0.south) -- (sn0xd97ee0.north);
\draw (sn0xd97cb0.south) -- (sn0xd88140.north);
\draw (sn0xd97ee0.south) -- (sn0xd88140.north);
\draw (sn0xd88140.south) -- (sn0xd88070.north);
\end{tikzpicture}

%%% Local Variables:
%%% TeX-master: "thesis/thesis.tex"
%%% End: 
\renewcommand{\leveltopI}{-15cm + \leveltop}
\renewcommand{\leveltopII}{-15cm + \leveltopI}
\renewcommand{\leveltopIII}{-15cm + \leveltopII}
\renewcommand{\leveltopIIII}{-15cm + \leveltopIII}
\renewcommand{\leveltopIIIII}{-15cm + \leveltopIIII}
\renewcommand{\leveltopIIIIII}{-15cm + \leveltopIIIII}
\renewcommand{\leveltopIIIIIII}{-15cm + \leveltopIIIIII}
\renewcommand{\leveltopIIIIIIII}{-15cm + \leveltopIIIIIII}
\renewcommand{\leveltopIIIIIIIII}{-15cm + \leveltopIIIIIIII}
\renewcommand{\leveltopIIIIIIIIII}{-15cm + \leveltopIIIIIIIII}
\begin{tikzpicture}[scale=.2, anchor=south]
\begin{scope}[yshift=\leveltopI cm]
\matrix (line1) [column sep=1cm] {
\node[draw=black, rectangle split,  rectangle split parts=3] (sn0xda5120){
\begin{tikzpicture}[scale=.2]
\node[circle, scale=0.75, fill] (tid0) at (4.5,1.5){};
\node[circle, scale=0.75, fill] (tid1) at (2.25,3){};
\node[circle, scale=0.75, fill, red] (tid4) at (0.75,4.5){};
\node[circle, scale=0.75, fill, red] (tid5) at (2.25,4.5){};
\node[circle, scale=0.75, fill] (tid6) at (3.75,4.5){};
\draw[](tid1) -- (tid4);
\draw[](tid1) -- (tid5);
\draw[](tid1) -- (tid6);
\node[circle, scale=0.75, fill] (tid2) at (6,3){};
\node[circle, scale=0.75, fill] (tid7) at (5.25,4.5){};
\node[circle, scale=0.75, fill] (tid8) at (6.75,4.5){};
\draw[](tid2) -- (tid7);
\draw[](tid2) -- (tid8);
\node[circle, scale=0.75, fill] (tid3) at (8.25,3){};
\node[circle, scale=0.75, fill, red] (tid9) at (8.25,4.5){};
\draw[](tid3) -- (tid9);
\draw[](tid0) -- (tid1);
\draw[](tid0) -- (tid2);
\draw[](tid0) -- (tid3);
\end{tikzpicture}
\nodepart{two}
\footnotesize{5.20439}
\nodepart{three}
\footnotesize{$33\:22\:44$}
};
 & 
\\
};
\end{scope}
\begin{scope}[yshift=\leveltopII cm]
\matrix (line2) [column sep=1cm] {
\node[draw=black, rectangle split,  rectangle split parts=3] (sn0xda6020){
\begin{tikzpicture}[scale=.2]
\node[circle, scale=0.75, fill] (tid0) at (4.5,1.5){};
\node[circle, scale=0.75, fill] (tid1) at (2.25,3){};
\node[circle, scale=0.75, fill, red] (tid4) at (0.75,4.5){};
\node[circle, scale=0.75, fill, red] (tid5) at (2.25,4.5){};
\node[circle, scale=0.75, fill] (tid6) at (3.75,4.5){};
\draw[](tid1) -- (tid4);
\draw[](tid1) -- (tid5);
\draw[](tid1) -- (tid6);
\node[circle, scale=0.75, fill] (tid2) at (6,3){};
\node[circle, scale=0.75, fill, red] (tid7) at (5.25,4.5){};
\node[circle, scale=0.75, fill] (tid8) at (6.75,4.5){};
\draw[](tid2) -- (tid7);
\draw[](tid2) -- (tid8);
\node[circle, scale=0.75, fill] (tid3) at (8.25,3){};
\draw[](tid0) -- (tid1);
\draw[](tid0) -- (tid2);
\draw[](tid0) -- (tid3);
\end{tikzpicture}
\nodepart{two}
\footnotesize{4.86831}
\nodepart{three}
\footnotesize{$33\:67$}
};
 & 
\node[draw=black, rectangle split,  rectangle split parts=3] (sn0xda2f30){
\begin{tikzpicture}[scale=.2]
\node[circle, scale=0.75, fill] (tid0) at (3.75,1.5){};
\node[circle, scale=0.75, fill] (tid1) at (1.5,3){};
\node[circle, scale=0.75, fill, red] (tid4) at (0.75,4.5){};
\node[circle, scale=0.75, fill, red] (tid5) at (2.25,4.5){};
\draw[](tid1) -- (tid4);
\draw[](tid1) -- (tid5);
\node[circle, scale=0.75, fill] (tid2) at (4.5,3){};
\node[circle, scale=0.75, fill] (tid6) at (3.75,4.5){};
\node[circle, scale=0.75, fill] (tid7) at (5.25,4.5){};
\draw[](tid2) -- (tid6);
\draw[](tid2) -- (tid7);
\node[circle, scale=0.75, fill] (tid3) at (6.75,3){};
\node[circle, scale=0.75, fill, red] (tid8) at (6.75,4.5){};
\draw[](tid3) -- (tid8);
\draw[](tid0) -- (tid1);
\draw[](tid0) -- (tid2);
\draw[](tid0) -- (tid3);
\end{tikzpicture}
\nodepart{two}
\footnotesize{4.87243}
\nodepart{three}
\footnotesize{$33\:67$}
};
 & 
\node[draw=black, rectangle split,  rectangle split parts=3] (sn0xda0d60){
\begin{tikzpicture}[scale=.2]
\node[circle, scale=0.75, fill] (tid0) at (3.75,1.5){};
\node[circle, scale=0.75, fill] (tid1) at (1.5,3){};
\node[circle, scale=0.75, fill, red] (tid4) at (0.75,4.5){};
\node[circle, scale=0.75, fill] (tid5) at (2.25,4.5){};
\draw[](tid1) -- (tid4);
\draw[](tid1) -- (tid5);
\node[circle, scale=0.75, fill] (tid2) at (4.5,3){};
\node[circle, scale=0.75, fill, red] (tid6) at (3.75,4.5){};
\node[circle, scale=0.75, fill] (tid7) at (5.25,4.5){};
\draw[](tid2) -- (tid6);
\draw[](tid2) -- (tid7);
\node[circle, scale=0.75, fill] (tid3) at (6.75,3){};
\node[circle, scale=0.75, fill, red] (tid8) at (6.75,4.5){};
\draw[](tid3) -- (tid8);
\draw[](tid0) -- (tid1);
\draw[](tid0) -- (tid2);
\draw[](tid0) -- (tid3);
\end{tikzpicture}
\nodepart{two}
\footnotesize{4.87243}
\nodepart{three}
\footnotesize{$33\:67$}
};
 & 
\\
};
\end{scope}
\begin{scope}[yshift=\leveltopIII cm]
\matrix (line3) [column sep=1cm] {
\node[draw=black, rectangle split,  rectangle split parts=3] (sn0xd9cf30){
\begin{tikzpicture}[scale=.2]
\node[circle, scale=0.75, fill] (tid0) at (3.75,1.5){};
\node[circle, scale=0.75, fill] (tid1) at (2.25,3){};
\node[circle, scale=0.75, fill, red] (tid4) at (0.75,4.5){};
\node[circle, scale=0.75, fill, red] (tid5) at (2.25,4.5){};
\node[circle, scale=0.75, fill] (tid6) at (3.75,4.5){};
\draw[](tid1) -- (tid4);
\draw[](tid1) -- (tid5);
\draw[](tid1) -- (tid6);
\node[circle, scale=0.75, fill] (tid2) at (5.25,3){};
\node[circle, scale=0.75, fill, red] (tid7) at (5.25,4.5){};
\draw[](tid2) -- (tid7);
\node[circle, scale=0.75, fill] (tid3) at (6.75,3){};
\draw[](tid0) -- (tid1);
\draw[](tid0) -- (tid2);
\draw[](tid0) -- (tid3);
\end{tikzpicture}
\nodepart{two}
\footnotesize{4.53086}
\nodepart{three}
\footnotesize{$33\:67$}
};
 & 
\node[draw=black, rectangle split,  rectangle split parts=3] (sn0xd9f7a0){
\begin{tikzpicture}[scale=.2]
\node[circle, scale=0.75, fill] (tid0) at (3.75,1.5){};
\node[circle, scale=0.75, fill] (tid1) at (1.5,3){};
\node[circle, scale=0.75, fill, red] (tid4) at (0.75,4.5){};
\node[circle, scale=0.75, fill, red] (tid5) at (2.25,4.5){};
\draw[](tid1) -- (tid4);
\draw[](tid1) -- (tid5);
\node[circle, scale=0.75, fill] (tid2) at (4.5,3){};
\node[circle, scale=0.75, fill, red] (tid6) at (3.75,4.5){};
\node[circle, scale=0.75, fill] (tid7) at (5.25,4.5){};
\draw[](tid2) -- (tid6);
\draw[](tid2) -- (tid7);
\node[circle, scale=0.75, fill] (tid3) at (6.75,3){};
\draw[](tid0) -- (tid1);
\draw[](tid0) -- (tid2);
\draw[](tid0) -- (tid3);
\end{tikzpicture}
\nodepart{two}
\footnotesize{4.53704}
\nodepart{three}
\footnotesize{$1$}
};
 & 
\node[draw=black, rectangle split,  rectangle split parts=3] (sn0xd9a9c0){
\begin{tikzpicture}[scale=.2]
\node[circle, scale=0.75, fill] (tid0) at (3,1.5){};
\node[circle, scale=0.75, fill] (tid1) at (1.5,3){};
\node[circle, scale=0.75, fill, red] (tid4) at (0.75,4.5){};
\node[circle, scale=0.75, fill] (tid5) at (2.25,4.5){};
\draw[](tid1) -- (tid4);
\draw[](tid1) -- (tid5);
\node[circle, scale=0.75, fill] (tid2) at (3.75,3){};
\node[circle, scale=0.75, fill, red] (tid6) at (3.75,4.5){};
\draw[](tid2) -- (tid6);
\node[circle, scale=0.75, fill] (tid3) at (5.25,3){};
\node[circle, scale=0.75, fill, red] (tid7) at (5.25,4.5){};
\draw[](tid3) -- (tid7);
\draw[](tid0) -- (tid1);
\draw[](tid0) -- (tid2);
\draw[](tid0) -- (tid3);
\end{tikzpicture}
\nodepart{two}
\footnotesize{4.54012}
\nodepart{three}
\footnotesize{$67\:33$}
};
 & 
\\
};
\end{scope}
\begin{scope}[yshift=\leveltopIIII cm]
\matrix (line4) [column sep=1cm] {
\node[draw=black, rectangle split,  rectangle split parts=3] (sn0xd9d6c0){
\begin{tikzpicture}[scale=.2]
\node[circle, scale=0.75, fill] (tid0) at (3.75,1.5){};
\node[circle, scale=0.75, fill] (tid1) at (2.25,3){};
\node[circle, scale=0.75, fill, red] (tid4) at (0.75,4.5){};
\node[circle, scale=0.75, fill, red] (tid5) at (2.25,4.5){};
\node[circle, scale=0.75, fill, red] (tid6) at (3.75,4.5){};
\draw[](tid1) -- (tid4);
\draw[](tid1) -- (tid5);
\draw[](tid1) -- (tid6);
\node[circle, scale=0.75, fill] (tid2) at (5.25,3){};
\node[circle, scale=0.75, fill] (tid3) at (6.75,3){};
\draw[](tid0) -- (tid1);
\draw[](tid0) -- (tid2);
\draw[](tid0) -- (tid3);
\end{tikzpicture}
\nodepart{two}
\footnotesize{4.18519}
\nodepart{three}
\footnotesize{$1$}
};
 & 
\node[draw=black, rectangle split,  rectangle split parts=3] (sn0xd9ae50){
\begin{tikzpicture}[scale=.2]
\node[circle, scale=0.75, fill] (tid0) at (3,1.5){};
\node[circle, scale=0.75, fill] (tid1) at (1.5,3){};
\node[circle, scale=0.75, fill, red] (tid4) at (0.75,4.5){};
\node[circle, scale=0.75, fill, red] (tid5) at (2.25,4.5){};
\draw[](tid1) -- (tid4);
\draw[](tid1) -- (tid5);
\node[circle, scale=0.75, fill] (tid2) at (3.75,3){};
\node[circle, scale=0.75, fill, red] (tid6) at (3.75,4.5){};
\draw[](tid2) -- (tid6);
\node[circle, scale=0.75, fill] (tid3) at (5.25,3){};
\draw[](tid0) -- (tid1);
\draw[](tid0) -- (tid2);
\draw[](tid0) -- (tid3);
\end{tikzpicture}
\nodepart{two}
\footnotesize{4.2037}
\nodepart{three}
\footnotesize{$33\:67$}
};
 & 
\node[draw=black, rectangle split,  rectangle split parts=3] (sn0xd98db0){
\begin{tikzpicture}[scale=.2]
\node[circle, scale=0.75, fill] (tid0) at (2.25,1.5){};
\node[circle, scale=0.75, fill] (tid1) at (0.75,3){};
\node[circle, scale=0.75, fill, red] (tid4) at (0.75,4.5){};
\draw[](tid1) -- (tid4);
\node[circle, scale=0.75, fill] (tid2) at (2.25,3){};
\node[circle, scale=0.75, fill, red] (tid5) at (2.25,4.5){};
\draw[](tid2) -- (tid5);
\node[circle, scale=0.75, fill] (tid3) at (3.75,3){};
\node[circle, scale=0.75, fill, red] (tid6) at (3.75,4.5){};
\draw[](tid3) -- (tid6);
\draw[](tid0) -- (tid1);
\draw[](tid0) -- (tid2);
\draw[](tid0) -- (tid3);
\end{tikzpicture}
\nodepart{two}
\footnotesize{4.21296}
\nodepart{three}
\footnotesize{$1$}
};
 & 
\\
};
\end{scope}
\begin{scope}[yshift=\leveltopIIIII cm]
\matrix (line5) [column sep=1cm] {
\node[draw=black, rectangle split,  rectangle split parts=3] (sn0xd9a340){
\begin{tikzpicture}[scale=.2]
\node[circle, scale=0.75, fill] (tid0) at (3,1.5){};
\node[circle, scale=0.75, fill] (tid1) at (1.5,3){};
\node[circle, scale=0.75, fill, red] (tid4) at (0.75,4.5){};
\node[circle, scale=0.75, fill, red] (tid5) at (2.25,4.5){};
\draw[](tid1) -- (tid4);
\draw[](tid1) -- (tid5);
\node[circle, scale=0.75, fill, red] (tid2) at (3.75,3){};
\node[circle, scale=0.75, fill] (tid3) at (5.25,3){};
\draw[](tid0) -- (tid1);
\draw[](tid0) -- (tid2);
\draw[](tid0) -- (tid3);
\end{tikzpicture}
\nodepart{two}
\footnotesize{3.85185}
\nodepart{three}
\footnotesize{$33\:67$}
};
 & 
\node[draw=black, rectangle split,  rectangle split parts=3] (sn0xd98be0){
\begin{tikzpicture}[scale=.2]
\node[circle, scale=0.75, fill] (tid0) at (2.25,1.5){};
\node[circle, scale=0.75, fill] (tid1) at (0.75,3){};
\node[circle, scale=0.75, fill, red] (tid4) at (0.75,4.5){};
\draw[](tid1) -- (tid4);
\node[circle, scale=0.75, fill] (tid2) at (2.25,3){};
\node[circle, scale=0.75, fill, red] (tid5) at (2.25,4.5){};
\draw[](tid2) -- (tid5);
\node[circle, scale=0.75, fill, red] (tid3) at (3.75,3){};
\draw[](tid0) -- (tid1);
\draw[](tid0) -- (tid2);
\draw[](tid0) -- (tid3);
\end{tikzpicture}
\nodepart{two}
\footnotesize{3.87963}
\nodepart{three}
\footnotesize{$67\:33$}
};
 & 
\\
};
\end{scope}
\begin{scope}[yshift=\leveltopIIIIII cm]
\matrix (line6) [column sep=1cm] {
\node[draw=black, rectangle split,  rectangle split parts=3] (sn0xd99950){
\begin{tikzpicture}[scale=.2]
\node[circle, scale=0.75, fill] (tid0) at (2.25,1.5){};
\node[circle, scale=0.75, fill] (tid1) at (1.5,3){};
\node[circle, scale=0.75, fill, red] (tid3) at (0.75,4.5){};
\node[circle, scale=0.75, fill, red] (tid4) at (2.25,4.5){};
\draw[](tid1) -- (tid3);
\draw[](tid1) -- (tid4);
\node[circle, scale=0.75, fill, red] (tid2) at (3.75,3){};
\draw[](tid0) -- (tid1);
\draw[](tid0) -- (tid2);
\end{tikzpicture}
\nodepart{two}
\footnotesize{3.66667}
\nodepart{three}
\footnotesize{$33\:67$}
};
 & 
\node[draw=black, rectangle split,  rectangle split parts=3] (sn0xd98b10){
\begin{tikzpicture}[scale=.2]
\node[circle, scale=0.75, fill] (tid0) at (2.25,1.5){};
\node[circle, scale=0.75, fill] (tid1) at (0.75,3){};
\node[circle, scale=0.75, fill, red] (tid4) at (0.75,4.5){};
\draw[](tid1) -- (tid4);
\node[circle, scale=0.75, fill, red] (tid2) at (2.25,3){};
\node[circle, scale=0.75, fill, red] (tid3) at (3.75,3){};
\draw[](tid0) -- (tid1);
\draw[](tid0) -- (tid2);
\draw[](tid0) -- (tid3);
\end{tikzpicture}
\nodepart{two}
\footnotesize{3.44444}
\nodepart{three}
\footnotesize{$67\:33$}
};
 & 
\node[draw=black, rectangle split,  rectangle split parts=3] (sn0xd982f0){
\begin{tikzpicture}[scale=.2]
\node[circle, scale=0.75, fill] (tid0) at (1.5,1.5){};
\node[circle, scale=0.75, fill] (tid1) at (0.75,3){};
\node[circle, scale=0.75, fill, red] (tid3) at (0.75,4.5){};
\draw[](tid1) -- (tid3);
\node[circle, scale=0.75, fill] (tid2) at (2.25,3){};
\node[circle, scale=0.75, fill, red] (tid4) at (2.25,4.5){};
\draw[](tid2) -- (tid4);
\draw[](tid0) -- (tid1);
\draw[](tid0) -- (tid2);
\end{tikzpicture}
\nodepart{two}
\footnotesize{3.75}
\nodepart{three}
\footnotesize{$1$}
};
 & 
\\
};
\end{scope}
\begin{scope}[yshift=\leveltopIIIIIII cm]
\matrix (line7) [column sep=1cm] {
\node[draw=black, rectangle split,  rectangle split parts=3] (sn0xd99040){
\begin{tikzpicture}[scale=.2]
\node[circle, scale=0.75, fill] (tid0) at (1.5,1.5){};
\node[circle, scale=0.75, fill] (tid1) at (1.5,3){};
\node[circle, scale=0.75, fill, red] (tid2) at (0.75,4.5){};
\node[circle, scale=0.75, fill, red] (tid3) at (2.25,4.5){};
\draw[](tid1) -- (tid2);
\draw[](tid1) -- (tid3);
\draw[](tid0) -- (tid1);
\end{tikzpicture}
\nodepart{two}
\footnotesize{3.5}
\nodepart{three}
\footnotesize{$1$}
};
 & 
\node[draw=black, rectangle split,  rectangle split parts=3] (sn0xd981e0){
\begin{tikzpicture}[scale=.2]
\node[circle, scale=0.75, fill] (tid0) at (1.5,1.5){};
\node[circle, scale=0.75, fill] (tid1) at (0.75,3){};
\node[circle, scale=0.75, fill, red] (tid3) at (0.75,4.5){};
\draw[](tid1) -- (tid3);
\node[circle, scale=0.75, fill, red] (tid2) at (2.25,3){};
\draw[](tid0) -- (tid1);
\draw[](tid0) -- (tid2);
\end{tikzpicture}
\nodepart{two}
\footnotesize{3.25}
\nodepart{three}
\footnotesize{$50\:50$}
};
 & 
\node[draw=black, rectangle split,  rectangle split parts=3] (sn0xd985f0){
\begin{tikzpicture}[scale=.2]
\node[circle, scale=0.75, fill] (tid0) at (2.25,1.5){};
\node[circle, scale=0.75, fill, red] (tid1) at (0.75,3){};
\node[circle, scale=0.75, fill, red] (tid2) at (2.25,3){};
\node[circle, scale=0.75, fill, red] (tid3) at (3.75,3){};
\draw[](tid0) -- (tid1);
\draw[](tid0) -- (tid2);
\draw[](tid0) -- (tid3);
\end{tikzpicture}
\nodepart{two}
\footnotesize{2.83333}
\nodepart{three}
\footnotesize{$1$}
};
 & 
\\
};
\end{scope}
\begin{scope}[yshift=\leveltopIIIIIIII cm]
\matrix (line8) [column sep=1cm] {
\node[draw=black, rectangle split,  rectangle split parts=3] (sn0xd97cb0){
\begin{tikzpicture}[scale=.2]
\node[circle, scale=0.75, fill] (tid0) at (0.75,1.5){};
\node[circle, scale=0.75, fill] (tid1) at (0.75,3){};
\node[circle, scale=0.75, fill, red] (tid2) at (0.75,4.5){};
\draw[](tid1) -- (tid2);
\draw[](tid0) -- (tid1);
\end{tikzpicture}
\nodepart{two}
\footnotesize{3}
\nodepart{three}
\footnotesize{$1$}
};
 & 
\node[draw=black, rectangle split,  rectangle split parts=3] (sn0xd97ee0){
\begin{tikzpicture}[scale=.2]
\node[circle, scale=0.75, fill] (tid0) at (1.5,1.5){};
\node[circle, scale=0.75, fill, red] (tid1) at (0.75,3){};
\node[circle, scale=0.75, fill, red] (tid2) at (2.25,3){};
\draw[](tid0) -- (tid1);
\draw[](tid0) -- (tid2);
\end{tikzpicture}
\nodepart{two}
\footnotesize{2.5}
\nodepart{three}
\footnotesize{$1$}
};
 & 
\\
};
\end{scope}
\begin{scope}[yshift=\leveltopIIIIIIIII cm]
\matrix (line9) [column sep=1cm] {
\node[draw=black, rectangle split,  rectangle split parts=3] (sn0xd88140){
\begin{tikzpicture}[scale=.2]
\node[circle, scale=0.75, fill] (tid0) at (0.75,1.5){};
\node[circle, scale=0.75, fill, red] (tid1) at (0.75,3){};
\draw[](tid0) -- (tid1);
\end{tikzpicture}
\nodepart{two}
\footnotesize{2}
\nodepart{three}
\footnotesize{$1$}
};
 & 
\\
};
\end{scope}
\begin{scope}[yshift=\leveltopIIIIIIIIII cm]
\matrix (line10) [column sep=1cm] {
\node[draw=black, rectangle split,  rectangle split parts=3] (sn0xd88070){
\begin{tikzpicture}[scale=.2]
\node[circle, scale=0.75, fill, red] (tid0) at (0.75,1.5){};
\end{tikzpicture}
\nodepart{two}
\footnotesize{1}
\nodepart{three}
\footnotesize{$$}
};
 & 
\\
};
\end{scope}
\begin{scope}[yshift=\leveltopIIIIIIIIIII cm]
\matrix (line11) [column sep=1cm] {
\\
};
\end{scope}
\draw (sn0xda5120.south) -- (sn0xda6020.north);
\draw (sn0xda5120.south) -- (sn0xda2f30.north);
\draw (sn0xda5120.south) -- (sn0xda0d60.north);
\draw (sn0xda6020.south) -- (sn0xd9cf30.north);
\draw (sn0xda6020.south) -- (sn0xd9f7a0.north);
\draw (sn0xda2f30.south) -- (sn0xd9f7a0.north);
\draw (sn0xda2f30.south) -- (sn0xd9a9c0.north);
\draw (sn0xda0d60.south) -- (sn0xd9f7a0.north);
\draw (sn0xda0d60.south) -- (sn0xd9a9c0.north);
\draw (sn0xd9cf30.south) -- (sn0xd9d6c0.north);
\draw (sn0xd9cf30.south) -- (sn0xd9ae50.north);
\draw (sn0xd9f7a0.south) -- (sn0xd9ae50.north);
\draw (sn0xd9a9c0.south) -- (sn0xd9ae50.north);
\draw (sn0xd9a9c0.south) -- (sn0xd98db0.north);
\draw (sn0xd9d6c0.south) -- (sn0xd9a340.north);
\draw (sn0xd9ae50.south) -- (sn0xd9a340.north);
\draw (sn0xd9ae50.south) -- (sn0xd98be0.north);
\draw (sn0xd98db0.south) -- (sn0xd98be0.north);
\draw (sn0xd9a340.south) -- (sn0xd99950.north);
\draw (sn0xd9a340.south) -- (sn0xd98b10.north);
\draw (sn0xd98be0.south) -- (sn0xd982f0.north);
\draw (sn0xd98be0.south) -- (sn0xd98b10.north);
\draw (sn0xd99950.south) -- (sn0xd99040.north);
\draw (sn0xd99950.south) -- (sn0xd981e0.north);
\draw (sn0xd98b10.south) -- (sn0xd981e0.north);
\draw (sn0xd98b10.south) -- (sn0xd985f0.north);
\draw (sn0xd982f0.south) -- (sn0xd981e0.north);
\draw (sn0xd99040.south) -- (sn0xd97cb0.north);
\draw (sn0xd981e0.south) -- (sn0xd97cb0.north);
\draw (sn0xd981e0.south) -- (sn0xd97ee0.north);
\draw (sn0xd985f0.south) -- (sn0xd97ee0.north);
\draw (sn0xd97cb0.south) -- (sn0xd88140.north);
\draw (sn0xd97ee0.south) -- (sn0xd88140.north);
\draw (sn0xd88140.south) -- (sn0xd88070.north);
\end{tikzpicture}

%%% Local Variables:
%%% TeX-master: "thesis/thesis.tex"
%%% End: 
\renewcommand{\leveltopI}{-15cm + \leveltop}
\renewcommand{\leveltopII}{-15cm + \leveltopI}
\renewcommand{\leveltopIII}{-15cm + \leveltopII}
\renewcommand{\leveltopIIII}{-15cm + \leveltopIII}
\renewcommand{\leveltopIIIII}{-15cm + \leveltopIIII}
\renewcommand{\leveltopIIIIII}{-15cm + \leveltopIIIII}
\renewcommand{\leveltopIIIIIII}{-15cm + \leveltopIIIIII}
\renewcommand{\leveltopIIIIIIII}{-15cm + \leveltopIIIIIII}
\renewcommand{\leveltopIIIIIIIII}{-15cm + \leveltopIIIIIIII}
\renewcommand{\leveltopIIIIIIIIII}{-15cm + \leveltopIIIIIIIII}
\begin{tikzpicture}[scale=.2, anchor=south]
\begin{scope}[yshift=\leveltopI cm]
\matrix (line1) [column sep=1cm] {
\node[draw=black, rectangle split,  rectangle split parts=3] (sn0xda6f40){
\begin{tikzpicture}[scale=.2]
\node[circle, scale=0.75, fill] (tid0) at (4.5,1.5){};
\node[circle, scale=0.75, fill] (tid1) at (2.25,3){};
\node[circle, scale=0.75, fill, red] (tid4) at (0.75,4.5){};
\node[circle, scale=0.75, fill] (tid5) at (2.25,4.5){};
\node[circle, scale=0.75, fill] (tid6) at (3.75,4.5){};
\draw[](tid1) -- (tid4);
\draw[](tid1) -- (tid5);
\draw[](tid1) -- (tid6);
\node[circle, scale=0.75, fill] (tid2) at (6,3){};
\node[circle, scale=0.75, fill, red] (tid7) at (5.25,4.5){};
\node[circle, scale=0.75, fill] (tid8) at (6.75,4.5){};
\draw[](tid2) -- (tid7);
\draw[](tid2) -- (tid8);
\node[circle, scale=0.75, fill] (tid3) at (8.25,3){};
\node[circle, scale=0.75, fill, red] (tid9) at (8.25,4.5){};
\draw[](tid3) -- (tid9);
\draw[](tid0) -- (tid1);
\draw[](tid0) -- (tid2);
\draw[](tid0) -- (tid3);
\end{tikzpicture}
\nodepart{two}
\footnotesize{5.20199}
\nodepart{three}
\footnotesize{$33\:33\:22\:11$}
};
 & 
\\
};
\end{scope}
\begin{scope}[yshift=\leveltopII cm]
\matrix (line2) [column sep=1cm] {
\node[draw=black, rectangle split,  rectangle split parts=3] (sn0xd9fb90){
\begin{tikzpicture}[scale=.2]
\node[circle, scale=0.75, fill] (tid0) at (4.5,1.5){};
\node[circle, scale=0.75, fill] (tid1) at (2.25,3){};
\node[circle, scale=0.75, fill, red] (tid4) at (0.75,4.5){};
\node[circle, scale=0.75, fill] (tid5) at (2.25,4.5){};
\node[circle, scale=0.75, fill] (tid6) at (3.75,4.5){};
\draw[](tid1) -- (tid4);
\draw[](tid1) -- (tid5);
\draw[](tid1) -- (tid6);
\node[circle, scale=0.75, fill] (tid2) at (6,3){};
\node[circle, scale=0.75, fill, red] (tid7) at (5.25,4.5){};
\node[circle, scale=0.75, fill, red] (tid8) at (6.75,4.5){};
\draw[](tid2) -- (tid7);
\draw[](tid2) -- (tid8);
\node[circle, scale=0.75, fill] (tid3) at (8.25,3){};
\draw[](tid0) -- (tid1);
\draw[](tid0) -- (tid2);
\draw[](tid0) -- (tid3);
\end{tikzpicture}
\nodepart{two}
\footnotesize{4.86626}
\nodepart{three}
\footnotesize{$67\:33$}
};
 & 
\node[draw=black, rectangle split,  rectangle split parts=3] (sn0xd9d000){
\begin{tikzpicture}[scale=.2]
\node[circle, scale=0.75, fill] (tid0) at (3.75,1.5){};
\node[circle, scale=0.75, fill] (tid1) at (2.25,3){};
\node[circle, scale=0.75, fill, red] (tid4) at (0.75,4.5){};
\node[circle, scale=0.75, fill] (tid5) at (2.25,4.5){};
\node[circle, scale=0.75, fill] (tid6) at (3.75,4.5){};
\draw[](tid1) -- (tid4);
\draw[](tid1) -- (tid5);
\draw[](tid1) -- (tid6);
\node[circle, scale=0.75, fill] (tid2) at (5.25,3){};
\node[circle, scale=0.75, fill, red] (tid7) at (5.25,4.5){};
\draw[](tid2) -- (tid7);
\node[circle, scale=0.75, fill] (tid3) at (6.75,3){};
\node[circle, scale=0.75, fill, red] (tid8) at (6.75,4.5){};
\draw[](tid3) -- (tid8);
\draw[](tid0) -- (tid1);
\draw[](tid0) -- (tid2);
\draw[](tid0) -- (tid3);
\end{tikzpicture}
\nodepart{two}
\footnotesize{4.86728}
\nodepart{three}
\footnotesize{$67\:33$}
};
 & 
\node[draw=black, rectangle split,  rectangle split parts=3] (sn0xda0d60){
\begin{tikzpicture}[scale=.2]
\node[circle, scale=0.75, fill] (tid0) at (3.75,1.5){};
\node[circle, scale=0.75, fill] (tid1) at (1.5,3){};
\node[circle, scale=0.75, fill, red] (tid4) at (0.75,4.5){};
\node[circle, scale=0.75, fill] (tid5) at (2.25,4.5){};
\draw[](tid1) -- (tid4);
\draw[](tid1) -- (tid5);
\node[circle, scale=0.75, fill] (tid2) at (4.5,3){};
\node[circle, scale=0.75, fill, red] (tid6) at (3.75,4.5){};
\node[circle, scale=0.75, fill] (tid7) at (5.25,4.5){};
\draw[](tid2) -- (tid6);
\draw[](tid2) -- (tid7);
\node[circle, scale=0.75, fill] (tid3) at (6.75,3){};
\node[circle, scale=0.75, fill, red] (tid8) at (6.75,4.5){};
\draw[](tid3) -- (tid8);
\draw[](tid0) -- (tid1);
\draw[](tid0) -- (tid2);
\draw[](tid0) -- (tid3);
\end{tikzpicture}
\nodepart{two}
\footnotesize{4.87243}
\nodepart{three}
\footnotesize{$33\:67$}
};
 & 
\node[draw=black, rectangle split,  rectangle split parts=3] (sn0xda2f30){
\begin{tikzpicture}[scale=.2]
\node[circle, scale=0.75, fill] (tid0) at (3.75,1.5){};
\node[circle, scale=0.75, fill] (tid1) at (1.5,3){};
\node[circle, scale=0.75, fill, red] (tid4) at (0.75,4.5){};
\node[circle, scale=0.75, fill, red] (tid5) at (2.25,4.5){};
\draw[](tid1) -- (tid4);
\draw[](tid1) -- (tid5);
\node[circle, scale=0.75, fill] (tid2) at (4.5,3){};
\node[circle, scale=0.75, fill] (tid6) at (3.75,4.5){};
\node[circle, scale=0.75, fill] (tid7) at (5.25,4.5){};
\draw[](tid2) -- (tid6);
\draw[](tid2) -- (tid7);
\node[circle, scale=0.75, fill] (tid3) at (6.75,3){};
\node[circle, scale=0.75, fill, red] (tid8) at (6.75,4.5){};
\draw[](tid3) -- (tid8);
\draw[](tid0) -- (tid1);
\draw[](tid0) -- (tid2);
\draw[](tid0) -- (tid3);
\end{tikzpicture}
\nodepart{two}
\footnotesize{4.87243}
\nodepart{three}
\footnotesize{$33\:67$}
};
 & 
\\
};
\end{scope}
\begin{scope}[yshift=\leveltopIII cm]
\matrix (line3) [column sep=1cm] {
\node[draw=black, rectangle split,  rectangle split parts=3] (sn0xd9cf30){
\begin{tikzpicture}[scale=.2]
\node[circle, scale=0.75, fill] (tid0) at (3.75,1.5){};
\node[circle, scale=0.75, fill] (tid1) at (2.25,3){};
\node[circle, scale=0.75, fill, red] (tid4) at (0.75,4.5){};
\node[circle, scale=0.75, fill, red] (tid5) at (2.25,4.5){};
\node[circle, scale=0.75, fill] (tid6) at (3.75,4.5){};
\draw[](tid1) -- (tid4);
\draw[](tid1) -- (tid5);
\draw[](tid1) -- (tid6);
\node[circle, scale=0.75, fill] (tid2) at (5.25,3){};
\node[circle, scale=0.75, fill, red] (tid7) at (5.25,4.5){};
\draw[](tid2) -- (tid7);
\node[circle, scale=0.75, fill] (tid3) at (6.75,3){};
\draw[](tid0) -- (tid1);
\draw[](tid0) -- (tid2);
\draw[](tid0) -- (tid3);
\end{tikzpicture}
\nodepart{two}
\footnotesize{4.53086}
\nodepart{three}
\footnotesize{$33\:67$}
};
 & 
\node[draw=black, rectangle split,  rectangle split parts=3] (sn0xd9f7a0){
\begin{tikzpicture}[scale=.2]
\node[circle, scale=0.75, fill] (tid0) at (3.75,1.5){};
\node[circle, scale=0.75, fill] (tid1) at (1.5,3){};
\node[circle, scale=0.75, fill, red] (tid4) at (0.75,4.5){};
\node[circle, scale=0.75, fill, red] (tid5) at (2.25,4.5){};
\draw[](tid1) -- (tid4);
\draw[](tid1) -- (tid5);
\node[circle, scale=0.75, fill] (tid2) at (4.5,3){};
\node[circle, scale=0.75, fill, red] (tid6) at (3.75,4.5){};
\node[circle, scale=0.75, fill] (tid7) at (5.25,4.5){};
\draw[](tid2) -- (tid6);
\draw[](tid2) -- (tid7);
\node[circle, scale=0.75, fill] (tid3) at (6.75,3){};
\draw[](tid0) -- (tid1);
\draw[](tid0) -- (tid2);
\draw[](tid0) -- (tid3);
\end{tikzpicture}
\nodepart{two}
\footnotesize{4.53704}
\nodepart{three}
\footnotesize{$1$}
};
 & 
\node[draw=black, rectangle split,  rectangle split parts=3] (sn0xd9a9c0){
\begin{tikzpicture}[scale=.2]
\node[circle, scale=0.75, fill] (tid0) at (3,1.5){};
\node[circle, scale=0.75, fill] (tid1) at (1.5,3){};
\node[circle, scale=0.75, fill, red] (tid4) at (0.75,4.5){};
\node[circle, scale=0.75, fill] (tid5) at (2.25,4.5){};
\draw[](tid1) -- (tid4);
\draw[](tid1) -- (tid5);
\node[circle, scale=0.75, fill] (tid2) at (3.75,3){};
\node[circle, scale=0.75, fill, red] (tid6) at (3.75,4.5){};
\draw[](tid2) -- (tid6);
\node[circle, scale=0.75, fill] (tid3) at (5.25,3){};
\node[circle, scale=0.75, fill, red] (tid7) at (5.25,4.5){};
\draw[](tid3) -- (tid7);
\draw[](tid0) -- (tid1);
\draw[](tid0) -- (tid2);
\draw[](tid0) -- (tid3);
\end{tikzpicture}
\nodepart{two}
\footnotesize{4.54012}
\nodepart{three}
\footnotesize{$67\:33$}
};
 & 
\\
};
\end{scope}
\begin{scope}[yshift=\leveltopIIII cm]
\matrix (line4) [column sep=1cm] {
\node[draw=black, rectangle split,  rectangle split parts=3] (sn0xd9d6c0){
\begin{tikzpicture}[scale=.2]
\node[circle, scale=0.75, fill] (tid0) at (3.75,1.5){};
\node[circle, scale=0.75, fill] (tid1) at (2.25,3){};
\node[circle, scale=0.75, fill, red] (tid4) at (0.75,4.5){};
\node[circle, scale=0.75, fill, red] (tid5) at (2.25,4.5){};
\node[circle, scale=0.75, fill, red] (tid6) at (3.75,4.5){};
\draw[](tid1) -- (tid4);
\draw[](tid1) -- (tid5);
\draw[](tid1) -- (tid6);
\node[circle, scale=0.75, fill] (tid2) at (5.25,3){};
\node[circle, scale=0.75, fill] (tid3) at (6.75,3){};
\draw[](tid0) -- (tid1);
\draw[](tid0) -- (tid2);
\draw[](tid0) -- (tid3);
\end{tikzpicture}
\nodepart{two}
\footnotesize{4.18519}
\nodepart{three}
\footnotesize{$1$}
};
 & 
\node[draw=black, rectangle split,  rectangle split parts=3] (sn0xd9ae50){
\begin{tikzpicture}[scale=.2]
\node[circle, scale=0.75, fill] (tid0) at (3,1.5){};
\node[circle, scale=0.75, fill] (tid1) at (1.5,3){};
\node[circle, scale=0.75, fill, red] (tid4) at (0.75,4.5){};
\node[circle, scale=0.75, fill, red] (tid5) at (2.25,4.5){};
\draw[](tid1) -- (tid4);
\draw[](tid1) -- (tid5);
\node[circle, scale=0.75, fill] (tid2) at (3.75,3){};
\node[circle, scale=0.75, fill, red] (tid6) at (3.75,4.5){};
\draw[](tid2) -- (tid6);
\node[circle, scale=0.75, fill] (tid3) at (5.25,3){};
\draw[](tid0) -- (tid1);
\draw[](tid0) -- (tid2);
\draw[](tid0) -- (tid3);
\end{tikzpicture}
\nodepart{two}
\footnotesize{4.2037}
\nodepart{three}
\footnotesize{$33\:67$}
};
 & 
\node[draw=black, rectangle split,  rectangle split parts=3] (sn0xd98db0){
\begin{tikzpicture}[scale=.2]
\node[circle, scale=0.75, fill] (tid0) at (2.25,1.5){};
\node[circle, scale=0.75, fill] (tid1) at (0.75,3){};
\node[circle, scale=0.75, fill, red] (tid4) at (0.75,4.5){};
\draw[](tid1) -- (tid4);
\node[circle, scale=0.75, fill] (tid2) at (2.25,3){};
\node[circle, scale=0.75, fill, red] (tid5) at (2.25,4.5){};
\draw[](tid2) -- (tid5);
\node[circle, scale=0.75, fill] (tid3) at (3.75,3){};
\node[circle, scale=0.75, fill, red] (tid6) at (3.75,4.5){};
\draw[](tid3) -- (tid6);
\draw[](tid0) -- (tid1);
\draw[](tid0) -- (tid2);
\draw[](tid0) -- (tid3);
\end{tikzpicture}
\nodepart{two}
\footnotesize{4.21296}
\nodepart{three}
\footnotesize{$1$}
};
 & 
\\
};
\end{scope}
\begin{scope}[yshift=\leveltopIIIII cm]
\matrix (line5) [column sep=1cm] {
\node[draw=black, rectangle split,  rectangle split parts=3] (sn0xd9a340){
\begin{tikzpicture}[scale=.2]
\node[circle, scale=0.75, fill] (tid0) at (3,1.5){};
\node[circle, scale=0.75, fill] (tid1) at (1.5,3){};
\node[circle, scale=0.75, fill, red] (tid4) at (0.75,4.5){};
\node[circle, scale=0.75, fill, red] (tid5) at (2.25,4.5){};
\draw[](tid1) -- (tid4);
\draw[](tid1) -- (tid5);
\node[circle, scale=0.75, fill, red] (tid2) at (3.75,3){};
\node[circle, scale=0.75, fill] (tid3) at (5.25,3){};
\draw[](tid0) -- (tid1);
\draw[](tid0) -- (tid2);
\draw[](tid0) -- (tid3);
\end{tikzpicture}
\nodepart{two}
\footnotesize{3.85185}
\nodepart{three}
\footnotesize{$33\:67$}
};
 & 
\node[draw=black, rectangle split,  rectangle split parts=3] (sn0xd98be0){
\begin{tikzpicture}[scale=.2]
\node[circle, scale=0.75, fill] (tid0) at (2.25,1.5){};
\node[circle, scale=0.75, fill] (tid1) at (0.75,3){};
\node[circle, scale=0.75, fill, red] (tid4) at (0.75,4.5){};
\draw[](tid1) -- (tid4);
\node[circle, scale=0.75, fill] (tid2) at (2.25,3){};
\node[circle, scale=0.75, fill, red] (tid5) at (2.25,4.5){};
\draw[](tid2) -- (tid5);
\node[circle, scale=0.75, fill, red] (tid3) at (3.75,3){};
\draw[](tid0) -- (tid1);
\draw[](tid0) -- (tid2);
\draw[](tid0) -- (tid3);
\end{tikzpicture}
\nodepart{two}
\footnotesize{3.87963}
\nodepart{three}
\footnotesize{$67\:33$}
};
 & 
\\
};
\end{scope}
\begin{scope}[yshift=\leveltopIIIIII cm]
\matrix (line6) [column sep=1cm] {
\node[draw=black, rectangle split,  rectangle split parts=3] (sn0xd99950){
\begin{tikzpicture}[scale=.2]
\node[circle, scale=0.75, fill] (tid0) at (2.25,1.5){};
\node[circle, scale=0.75, fill] (tid1) at (1.5,3){};
\node[circle, scale=0.75, fill, red] (tid3) at (0.75,4.5){};
\node[circle, scale=0.75, fill, red] (tid4) at (2.25,4.5){};
\draw[](tid1) -- (tid3);
\draw[](tid1) -- (tid4);
\node[circle, scale=0.75, fill, red] (tid2) at (3.75,3){};
\draw[](tid0) -- (tid1);
\draw[](tid0) -- (tid2);
\end{tikzpicture}
\nodepart{two}
\footnotesize{3.66667}
\nodepart{three}
\footnotesize{$33\:67$}
};
 & 
\node[draw=black, rectangle split,  rectangle split parts=3] (sn0xd98b10){
\begin{tikzpicture}[scale=.2]
\node[circle, scale=0.75, fill] (tid0) at (2.25,1.5){};
\node[circle, scale=0.75, fill] (tid1) at (0.75,3){};
\node[circle, scale=0.75, fill, red] (tid4) at (0.75,4.5){};
\draw[](tid1) -- (tid4);
\node[circle, scale=0.75, fill, red] (tid2) at (2.25,3){};
\node[circle, scale=0.75, fill, red] (tid3) at (3.75,3){};
\draw[](tid0) -- (tid1);
\draw[](tid0) -- (tid2);
\draw[](tid0) -- (tid3);
\end{tikzpicture}
\nodepart{two}
\footnotesize{3.44444}
\nodepart{three}
\footnotesize{$67\:33$}
};
 & 
\node[draw=black, rectangle split,  rectangle split parts=3] (sn0xd982f0){
\begin{tikzpicture}[scale=.2]
\node[circle, scale=0.75, fill] (tid0) at (1.5,1.5){};
\node[circle, scale=0.75, fill] (tid1) at (0.75,3){};
\node[circle, scale=0.75, fill, red] (tid3) at (0.75,4.5){};
\draw[](tid1) -- (tid3);
\node[circle, scale=0.75, fill] (tid2) at (2.25,3){};
\node[circle, scale=0.75, fill, red] (tid4) at (2.25,4.5){};
\draw[](tid2) -- (tid4);
\draw[](tid0) -- (tid1);
\draw[](tid0) -- (tid2);
\end{tikzpicture}
\nodepart{two}
\footnotesize{3.75}
\nodepart{three}
\footnotesize{$1$}
};
 & 
\\
};
\end{scope}
\begin{scope}[yshift=\leveltopIIIIIII cm]
\matrix (line7) [column sep=1cm] {
\node[draw=black, rectangle split,  rectangle split parts=3] (sn0xd99040){
\begin{tikzpicture}[scale=.2]
\node[circle, scale=0.75, fill] (tid0) at (1.5,1.5){};
\node[circle, scale=0.75, fill] (tid1) at (1.5,3){};
\node[circle, scale=0.75, fill, red] (tid2) at (0.75,4.5){};
\node[circle, scale=0.75, fill, red] (tid3) at (2.25,4.5){};
\draw[](tid1) -- (tid2);
\draw[](tid1) -- (tid3);
\draw[](tid0) -- (tid1);
\end{tikzpicture}
\nodepart{two}
\footnotesize{3.5}
\nodepart{three}
\footnotesize{$1$}
};
 & 
\node[draw=black, rectangle split,  rectangle split parts=3] (sn0xd981e0){
\begin{tikzpicture}[scale=.2]
\node[circle, scale=0.75, fill] (tid0) at (1.5,1.5){};
\node[circle, scale=0.75, fill] (tid1) at (0.75,3){};
\node[circle, scale=0.75, fill, red] (tid3) at (0.75,4.5){};
\draw[](tid1) -- (tid3);
\node[circle, scale=0.75, fill, red] (tid2) at (2.25,3){};
\draw[](tid0) -- (tid1);
\draw[](tid0) -- (tid2);
\end{tikzpicture}
\nodepart{two}
\footnotesize{3.25}
\nodepart{three}
\footnotesize{$50\:50$}
};
 & 
\node[draw=black, rectangle split,  rectangle split parts=3] (sn0xd985f0){
\begin{tikzpicture}[scale=.2]
\node[circle, scale=0.75, fill] (tid0) at (2.25,1.5){};
\node[circle, scale=0.75, fill, red] (tid1) at (0.75,3){};
\node[circle, scale=0.75, fill, red] (tid2) at (2.25,3){};
\node[circle, scale=0.75, fill, red] (tid3) at (3.75,3){};
\draw[](tid0) -- (tid1);
\draw[](tid0) -- (tid2);
\draw[](tid0) -- (tid3);
\end{tikzpicture}
\nodepart{two}
\footnotesize{2.83333}
\nodepart{three}
\footnotesize{$1$}
};
 & 
\\
};
\end{scope}
\begin{scope}[yshift=\leveltopIIIIIIII cm]
\matrix (line8) [column sep=1cm] {
\node[draw=black, rectangle split,  rectangle split parts=3] (sn0xd97cb0){
\begin{tikzpicture}[scale=.2]
\node[circle, scale=0.75, fill] (tid0) at (0.75,1.5){};
\node[circle, scale=0.75, fill] (tid1) at (0.75,3){};
\node[circle, scale=0.75, fill, red] (tid2) at (0.75,4.5){};
\draw[](tid1) -- (tid2);
\draw[](tid0) -- (tid1);
\end{tikzpicture}
\nodepart{two}
\footnotesize{3}
\nodepart{three}
\footnotesize{$1$}
};
 & 
\node[draw=black, rectangle split,  rectangle split parts=3] (sn0xd97ee0){
\begin{tikzpicture}[scale=.2]
\node[circle, scale=0.75, fill] (tid0) at (1.5,1.5){};
\node[circle, scale=0.75, fill, red] (tid1) at (0.75,3){};
\node[circle, scale=0.75, fill, red] (tid2) at (2.25,3){};
\draw[](tid0) -- (tid1);
\draw[](tid0) -- (tid2);
\end{tikzpicture}
\nodepart{two}
\footnotesize{2.5}
\nodepart{three}
\footnotesize{$1$}
};
 & 
\\
};
\end{scope}
\begin{scope}[yshift=\leveltopIIIIIIIII cm]
\matrix (line9) [column sep=1cm] {
\node[draw=black, rectangle split,  rectangle split parts=3] (sn0xd88140){
\begin{tikzpicture}[scale=.2]
\node[circle, scale=0.75, fill] (tid0) at (0.75,1.5){};
\node[circle, scale=0.75, fill, red] (tid1) at (0.75,3){};
\draw[](tid0) -- (tid1);
\end{tikzpicture}
\nodepart{two}
\footnotesize{2}
\nodepart{three}
\footnotesize{$1$}
};
 & 
\\
};
\end{scope}
\begin{scope}[yshift=\leveltopIIIIIIIIII cm]
\matrix (line10) [column sep=1cm] {
\node[draw=black, rectangle split,  rectangle split parts=3] (sn0xd88070){
\begin{tikzpicture}[scale=.2]
\node[circle, scale=0.75, fill, red] (tid0) at (0.75,1.5){};
\end{tikzpicture}
\nodepart{two}
\footnotesize{1}
\nodepart{three}
\footnotesize{$$}
};
 & 
\\
};
\end{scope}
\begin{scope}[yshift=\leveltopIIIIIIIIIII cm]
\matrix (line11) [column sep=1cm] {
\\
};
\end{scope}
\draw (sn0xda6f40.south) -- (sn0xd9fb90.north);
\draw (sn0xda6f40.south) -- (sn0xd9d000.north);
\draw (sn0xda6f40.south) -- (sn0xda0d60.north);
\draw (sn0xda6f40.south) -- (sn0xda2f30.north);
\draw (sn0xd9fb90.south) -- (sn0xd9cf30.north);
\draw (sn0xd9fb90.south) -- (sn0xd9f7a0.north);
\draw (sn0xd9d000.south) -- (sn0xd9cf30.north);
\draw (sn0xd9d000.south) -- (sn0xd9a9c0.north);
\draw (sn0xda0d60.south) -- (sn0xd9f7a0.north);
\draw (sn0xda0d60.south) -- (sn0xd9a9c0.north);
\draw (sn0xda2f30.south) -- (sn0xd9f7a0.north);
\draw (sn0xda2f30.south) -- (sn0xd9a9c0.north);
\draw (sn0xd9cf30.south) -- (sn0xd9d6c0.north);
\draw (sn0xd9cf30.south) -- (sn0xd9ae50.north);
\draw (sn0xd9f7a0.south) -- (sn0xd9ae50.north);
\draw (sn0xd9a9c0.south) -- (sn0xd9ae50.north);
\draw (sn0xd9a9c0.south) -- (sn0xd98db0.north);
\draw (sn0xd9d6c0.south) -- (sn0xd9a340.north);
\draw (sn0xd9ae50.south) -- (sn0xd9a340.north);
\draw (sn0xd9ae50.south) -- (sn0xd98be0.north);
\draw (sn0xd98db0.south) -- (sn0xd98be0.north);
\draw (sn0xd9a340.south) -- (sn0xd99950.north);
\draw (sn0xd9a340.south) -- (sn0xd98b10.north);
\draw (sn0xd98be0.south) -- (sn0xd982f0.north);
\draw (sn0xd98be0.south) -- (sn0xd98b10.north);
\draw (sn0xd99950.south) -- (sn0xd99040.north);
\draw (sn0xd99950.south) -- (sn0xd981e0.north);
\draw (sn0xd98b10.south) -- (sn0xd981e0.north);
\draw (sn0xd98b10.south) -- (sn0xd985f0.north);
\draw (sn0xd982f0.south) -- (sn0xd981e0.north);
\draw (sn0xd99040.south) -- (sn0xd97cb0.north);
\draw (sn0xd981e0.south) -- (sn0xd97cb0.north);
\draw (sn0xd981e0.south) -- (sn0xd97ee0.north);
\draw (sn0xd985f0.south) -- (sn0xd97ee0.north);
\draw (sn0xd97cb0.south) -- (sn0xd88140.north);
\draw (sn0xd97ee0.south) -- (sn0xd88140.north);
\draw (sn0xd88140.south) -- (sn0xd88070.north);
\end{tikzpicture}

%%% Local Variables:
%%% TeX-master: "thesis/thesis.tex"
%%% End: 


\end{document}
