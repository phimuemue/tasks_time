%%% hugegraph.tex --- 

%% Author: philipp@pmpc
%% Version: $Id: thesis.tex,v 0.0 2013/04/08 12:19:13 philipp Exp$

%%\revision$Header: /home/philipp/Documents/Uni/masterarbeit/thesis/thesis.tex,v 0.0 2013/04/08 12:19:13 philipp Exp$

\documentclass{report}

\usepackage[english]{babel}
\usepackage{lmodern}
\usepackage[utf8]{inputenc}
\usepackage{hyperref}
\usepackage{amsmath}
\usepackage{amssymb}
\usepackage{amsthm}
\usepackage{graphicx}
\usepackage{tikz}
\usepackage{suffix}
\usepackage{multicol}
\usepackage[left=3.5cm,right=3.5cm,a0paper]{geometry}

\usetikzlibrary{shapes.multipart,chains}
\usetikzlibrary{positioning}
\usetikzlibrary{matrix}
\usetikzlibrary{external} 
%\tikzexternalize


\tikzstyle{task_cross}=[
    {path picture={ 
        \draw[black]
        (path picture bounding box.south east) -- 
        (path picture bounding box.north west) 
        (path picture bounding box.south west) -- 
        (path picture bounding box.north east);
      }
    }
]

\tikzstyle{task_scheduled}=[fill=white, draw=black, task_cross]

\newtheorem{definition}{Definition}[chapter]
\newtheorem{theorem}{Theorem}[chapter]

\newcommand{\p}[1]{Pr\left[#1\right]}
\newcommand{\alltasks}{{\mathbb T}}
\newcommand{\neededfor}{\rightarrow}
\WithSuffix\newcommand\neededfor*{\stackrel{*}{\rightarrow}}

% \getwidthofnode will measure the width of the node given as its second
% parameter and store it into the first parameter.
\makeatletter
\newcommand\getwidthofnode[2]{%
    \pgfextractx{#1}{\pgfpointanchor{#2}{east}}%
    \pgfextractx{\pgf@xa}{\pgfpointanchor{#2}{west}}% \pgf@xa is a length defined by PGF for temporary storage. No need to create a new temporary length.
    \addtolength{#1}{-\pgf@xa}%
}
\makeatother

\begin{document}

% stuff to draw diagrams levelwise
\newcommand{\leveltop}{0}
\newcommand{\leveltopI}{0}
\newcommand{\leveltopII}{0}
\newcommand{\leveltopIII}{0}
\newcommand{\leveltopIIII}{0}
\newcommand{\leveltopIIIII}{0}
\newcommand{\leveltopIIIIII}{0}
\newcommand{\leveltopIIIIIII}{0}
\newcommand{\leveltopIIIIIIII}{0}
\newcommand{\leveltopIIIIIIIII}{0}
\newcommand{\leveltopIIIIIIIIII}{0}
\newcommand{\leveltopIIIIIIIIIII}{0}
\newcommand{\leveltopIIIIIIIIIIII}{0}
\newcommand{\leveltopIIIIIIIIIIIII}{0}
\newcommand{\leveltopIIIIIIIIIIIIII}{0}
\newcommand{\leveltopIIIIIIIIIIIIIII}{0}
\newcommand{\leveltopIIIIIIIIIIIIIIII}{0}
\newcommand{\leveltopIIIIIIIIIIIIIIIII}{0}
\newcommand{\leveltopIIIIIIIIIIIIIIIIII}{0}
\newcommand{\leveltopIIIIIIIIIIIIIIIIIII}{0}
\newcommand{\leveltopIIIIIIIIIIIIIIIIIIII}{0}
\newcommand{\leveltopIIIIIIIIIIIIIIIIIIIII}{0}
\newcommand{\leveltopIIIIIIIIIIIIIIIIIIIIII}{0}
\newcommand{\leveltopIIIIIIIIIIIIIIIIIIIIIII}{0}
\newcommand{\leveltopIIIIIIIIIIIIIIIIIIIIIIII}{0}
\newcommand{\leveltopIIIIIIIIIIIIIIIIIIIIIIIII}{0}
\newcommand{\leveltopIIIIIIIIIIIIIIIIIIIIIIIIII}{0}
\newcommand{\leveltopIIIIIIIIIIIIIIIIIIIIIIIIIII}{0}
\newcommand{\leveltopIIIIIIIIIIIIIIIIIIIIIIIIIIII}{0}
\newcommand{\leveltopIIIIIIIIIIIIIIIIIIIIIIIIIIIII}{0}

\section{P2: A whole intree-DAG and its condensed counterpart}

\renewcommand{\leveltopI}{-15cm + \leveltop}
\renewcommand{\leveltopII}{-15cm + \leveltopI}
\renewcommand{\leveltopIII}{-15cm + \leveltopII}
\renewcommand{\leveltopIIII}{-15cm + \leveltopIII}
\renewcommand{\leveltopIIIII}{-15cm + \leveltopIIII}
\renewcommand{\leveltopIIIIII}{-15cm + \leveltopIIIII}
\renewcommand{\leveltopIIIIIII}{-15cm + \leveltopIIIIII}
\renewcommand{\leveltopIIIIIIII}{-15cm + \leveltopIIIIIII}
\renewcommand{\leveltopIIIIIIIII}{-15cm + \leveltopIIIIIIII}
\renewcommand{\leveltopIIIIIIIIII}{-15cm + \leveltopIIIIIIIII}
\renewcommand{\leveltopIIIIIIIIIII}{-15cm + \leveltopIIIIIIIIII}
\begin{tikzpicture}[scale=.2, anchor=south]
  \begin{scope}[yshift=\leveltopI cm]
    \matrix (line1) [column sep=1cm] {
      \node[draw=black, rectangle split,  rectangle split parts=3] (sn0x104f980){
        \begin{tikzpicture}[scale=.2]
          \node[circle, scale=0.75, fill] (tid0) at (3.75,1.5){};
          \node[circle, scale=0.75, fill] (tid1) at (2.25,3){};
          \node[circle, scale=0.75, fill] (tid3) at (0.75,4.5){};
          \node[circle, scale=0.75, fill] (tid7) at (0.75,6){};
          \draw[](tid3) -- (tid7);
          \node[circle, scale=0.75, fill] (tid4) at (2.25,4.5){};
          \node[circle, scale=0.75, fill] (tid5) at (3.75,4.5){};
          \draw[](tid1) -- (tid3);
          \draw[](tid1) -- (tid4);
          \draw[](tid1) -- (tid5);
          \node[circle, scale=0.75, fill] (tid2) at (6,3){};
          \node[circle, scale=0.75, fill] (tid6) at (6,4.5){};
          \node[circle, scale=0.75, fill] (tid8) at (5.25,6){};
          \node[circle, scale=0.75, fill, red] (tid10) at (5.25,7.5){};
          \draw[](tid8) -- (tid10);
          \node[circle, scale=0.75, fill, red] (tid9) at (6.75,6){};
          \draw[](tid6) -- (tid8);
          \draw[](tid6) -- (tid9);
          \draw[](tid2) -- (tid6);
          \draw[](tid0) -- (tid1);
          \draw[](tid0) -- (tid2);
        \end{tikzpicture}
        \nodepart{two}
        \footnotesize{6.82812}
        \nodepart{three}
        \footnotesize{$50\:25\:25$}
      };
      & 
      \\
    };
  \end{scope}
  \begin{scope}[yshift=\leveltopII cm]
    \matrix (line2) [column sep=1cm] {
      \node[draw=black, rectangle split,  rectangle split parts=3] (sn0x1050190){
        \begin{tikzpicture}[scale=.2]
          \node[circle, scale=0.75, fill] (tid0) at (3,1.5){};
          \node[circle, scale=0.75, fill] (tid1) at (2.25,3){};
          \node[circle, scale=0.75, fill] (tid3) at (0.75,4.5){};
          \node[circle, scale=0.75, fill, red] (tid7) at (0.75,6){};
          \draw[](tid3) -- (tid7);
          \node[circle, scale=0.75, fill] (tid4) at (2.25,4.5){};
          \node[circle, scale=0.75, fill] (tid5) at (3.75,4.5){};
          \draw[](tid1) -- (tid3);
          \draw[](tid1) -- (tid4);
          \draw[](tid1) -- (tid5);
          \node[circle, scale=0.75, fill] (tid2) at (5.25,3){};
          \node[circle, scale=0.75, fill] (tid6) at (5.25,4.5){};
          \node[circle, scale=0.75, fill] (tid8) at (5.25,6){};
          \node[circle, scale=0.75, fill, red] (tid9) at (5.25,7.5){};
          \draw[](tid8) -- (tid9);
          \draw[](tid6) -- (tid8);
          \draw[](tid2) -- (tid6);
          \draw[](tid0) -- (tid1);
          \draw[](tid0) -- (tid2);
        \end{tikzpicture}
        \nodepart{two}
        \footnotesize{6.35938}
        \nodepart{three}
        \footnotesize{$50\:50$}
      };
      & 
      \node[draw=black, rectangle split,  rectangle split parts=3] (sn0x104cb60){
        \begin{tikzpicture}[scale=.2]
          \node[circle, scale=0.75, fill] (tid0) at (3.75,1.5){};
          \node[circle, scale=0.75, fill] (tid1) at (2.25,3){};
          \node[circle, scale=0.75, fill] (tid3) at (0.75,4.5){};
          \node[circle, scale=0.75, fill, red] (tid7) at (0.75,6){};
          \draw[](tid3) -- (tid7);
          \node[circle, scale=0.75, fill] (tid4) at (2.25,4.5){};
          \node[circle, scale=0.75, fill] (tid5) at (3.75,4.5){};
          \draw[](tid1) -- (tid3);
          \draw[](tid1) -- (tid4);
          \draw[](tid1) -- (tid5);
          \node[circle, scale=0.75, fill] (tid2) at (6,3){};
          \node[circle, scale=0.75, fill] (tid6) at (6,4.5){};
          \node[circle, scale=0.75, fill, red] (tid8) at (5.25,6){};
          \node[circle, scale=0.75, fill] (tid9) at (6.75,6){};
          \draw[](tid6) -- (tid8);
          \draw[](tid6) -- (tid9);
          \draw[](tid2) -- (tid6);
          \draw[](tid0) -- (tid1);
          \draw[](tid0) -- (tid2);
        \end{tikzpicture}
        \nodepart{two}
        \footnotesize{6.29688}
        \nodepart{three}
        \footnotesize{$50\:50$}
      };
      & 
      \node[draw=black, rectangle split,  rectangle split parts=3] (sn0x104dd20){
        \begin{tikzpicture}[scale=.2]
          \node[circle, scale=0.75, fill] (tid0) at (3.75,1.5){};
          \node[circle, scale=0.75, fill] (tid1) at (2.25,3){};
          \node[circle, scale=0.75, fill] (tid3) at (0.75,4.5){};
          \node[circle, scale=0.75, fill] (tid7) at (0.75,6){};
          \draw[](tid3) -- (tid7);
          \node[circle, scale=0.75, fill] (tid4) at (2.25,4.5){};
          \node[circle, scale=0.75, fill] (tid5) at (3.75,4.5){};
          \draw[](tid1) -- (tid3);
          \draw[](tid1) -- (tid4);
          \draw[](tid1) -- (tid5);
          \node[circle, scale=0.75, fill] (tid2) at (6,3){};
          \node[circle, scale=0.75, fill] (tid6) at (6,4.5){};
          \node[circle, scale=0.75, fill, red] (tid8) at (5.25,6){};
          \node[circle, scale=0.75, fill, red] (tid9) at (6.75,6){};
          \draw[](tid6) -- (tid8);
          \draw[](tid6) -- (tid9);
          \draw[](tid2) -- (tid6);
          \draw[](tid0) -- (tid1);
          \draw[](tid0) -- (tid2);
        \end{tikzpicture}
        \nodepart{two}
        \footnotesize{6.29688}
        \nodepart{three}
        \footnotesize{$1$}
      };
      & 
      \\
    };
  \end{scope}
  \begin{scope}[yshift=\leveltopIII cm]
    \matrix (line3) [column sep=1cm] {
      \node[draw=black, rectangle split,  rectangle split parts=3] (sn0x10519d0){
        \begin{tikzpicture}[scale=.2]
          \node[circle, scale=0.75, fill] (tid0) at (3,1.5){};
          \node[circle, scale=0.75, fill] (tid1) at (2.25,3){};
          \node[circle, scale=0.75, fill, red] (tid3) at (0.75,4.5){};
          \node[circle, scale=0.75, fill] (tid4) at (2.25,4.5){};
          \node[circle, scale=0.75, fill] (tid5) at (3.75,4.5){};
          \draw[](tid1) -- (tid3);
          \draw[](tid1) -- (tid4);
          \draw[](tid1) -- (tid5);
          \node[circle, scale=0.75, fill] (tid2) at (5.25,3){};
          \node[circle, scale=0.75, fill] (tid6) at (5.25,4.5){};
          \node[circle, scale=0.75, fill] (tid7) at (5.25,6){};
          \node[circle, scale=0.75, fill, red] (tid8) at (5.25,7.5){};
          \draw[](tid7) -- (tid8);
          \draw[](tid6) -- (tid7);
          \draw[](tid2) -- (tid6);
          \draw[](tid0) -- (tid1);
          \draw[](tid0) -- (tid2);
        \end{tikzpicture}
        \nodepart{two}
        \footnotesize{5.92188}
        \nodepart{three}
        \footnotesize{$50\:50$}
      };
      & 
      \node[draw=black, rectangle split,  rectangle split parts=3] (sn0x104fbd0){
        \begin{tikzpicture}[scale=.2]
          \node[circle, scale=0.75, fill] (tid0) at (3,1.5){};
          \node[circle, scale=0.75, fill] (tid1) at (2.25,3){};
          \node[circle, scale=0.75, fill] (tid3) at (0.75,4.5){};
          \node[circle, scale=0.75, fill, red] (tid7) at (0.75,6){};
          \draw[](tid3) -- (tid7);
          \node[circle, scale=0.75, fill] (tid4) at (2.25,4.5){};
          \node[circle, scale=0.75, fill] (tid5) at (3.75,4.5){};
          \draw[](tid1) -- (tid3);
          \draw[](tid1) -- (tid4);
          \draw[](tid1) -- (tid5);
          \node[circle, scale=0.75, fill] (tid2) at (5.25,3){};
          \node[circle, scale=0.75, fill] (tid6) at (5.25,4.5){};
          \node[circle, scale=0.75, fill, red] (tid8) at (5.25,6){};
          \draw[](tid6) -- (tid8);
          \draw[](tid2) -- (tid6);
          \draw[](tid0) -- (tid1);
          \draw[](tid0) -- (tid2);
        \end{tikzpicture}
        \nodepart{two}
        \footnotesize{5.79688}
        \nodepart{three}
        \footnotesize{$50\:33\:17$}
      };
      & 
      \node[draw=black, rectangle split,  rectangle split parts=3] (sn0x105a080){
        \begin{tikzpicture}[scale=.2]
          \node[circle, scale=0.75, fill] (tid0) at (3.75,1.5){};
          \node[circle, scale=0.75, fill] (tid1) at (2.25,3){};
          \node[circle, scale=0.75, fill] (tid3) at (0.75,4.5){};
          \node[circle, scale=0.75, fill] (tid4) at (2.25,4.5){};
          \node[circle, scale=0.75, fill] (tid5) at (3.75,4.5){};
          \draw[](tid1) -- (tid3);
          \draw[](tid1) -- (tid4);
          \draw[](tid1) -- (tid5);
          \node[circle, scale=0.75, fill] (tid2) at (6,3){};
          \node[circle, scale=0.75, fill] (tid6) at (6,4.5){};
          \node[circle, scale=0.75, fill, red] (tid7) at (5.25,6){};
          \node[circle, scale=0.75, fill, red] (tid8) at (6.75,6){};
          \draw[](tid6) -- (tid7);
          \draw[](tid6) -- (tid8);
          \draw[](tid2) -- (tid6);
          \draw[](tid0) -- (tid1);
          \draw[](tid0) -- (tid2);
        \end{tikzpicture}
        \nodepart{two}
        \footnotesize{5.79688}
        \nodepart{three}
        \footnotesize{$1$}
      };
      & 
      \\
    };
  \end{scope}
  \begin{scope}[yshift=\leveltopIIII cm]
    \matrix (line4) [column sep=1cm] {
      \node[draw=black, rectangle split,  rectangle split parts=3] (sn0x1052250){
        \begin{tikzpicture}[scale=.2]
          \node[circle, scale=0.75, fill] (tid0) at (2.25,1.5){};
          \node[circle, scale=0.75, fill] (tid1) at (0.75,3){};
          \node[circle, scale=0.75, fill] (tid3) at (0.75,4.5){};
          \node[circle, scale=0.75, fill] (tid6) at (0.75,6){};
          \node[circle, scale=0.75, fill, red] (tid7) at (0.75,7.5){};
          \draw[](tid6) -- (tid7);
          \draw[](tid3) -- (tid6);
          \draw[](tid1) -- (tid3);
          \node[circle, scale=0.75, fill] (tid2) at (3,3){};
          \node[circle, scale=0.75, fill, red] (tid4) at (2.25,4.5){};
          \node[circle, scale=0.75, fill] (tid5) at (3.75,4.5){};
          \draw[](tid2) -- (tid4);
          \draw[](tid2) -- (tid5);
          \draw[](tid0) -- (tid1);
          \draw[](tid0) -- (tid2);
        \end{tikzpicture}
        \nodepart{two}
        \footnotesize{5.54688}
        \nodepart{three}
        \footnotesize{$50\:50$}
      };
      & 
      \node[draw=black, rectangle split,  rectangle split parts=3] (sn0x1052960){
        \begin{tikzpicture}[scale=.2]
          \node[circle, scale=0.75, fill] (tid0) at (3,1.5){};
          \node[circle, scale=0.75, fill] (tid1) at (2.25,3){};
          \node[circle, scale=0.75, fill, red] (tid3) at (0.75,4.5){};
          \node[circle, scale=0.75, fill] (tid4) at (2.25,4.5){};
          \node[circle, scale=0.75, fill] (tid5) at (3.75,4.5){};
          \draw[](tid1) -- (tid3);
          \draw[](tid1) -- (tid4);
          \draw[](tid1) -- (tid5);
          \node[circle, scale=0.75, fill] (tid2) at (5.25,3){};
          \node[circle, scale=0.75, fill] (tid6) at (5.25,4.5){};
          \node[circle, scale=0.75, fill, red] (tid7) at (5.25,6){};
          \draw[](tid6) -- (tid7);
          \draw[](tid2) -- (tid6);
          \draw[](tid0) -- (tid1);
          \draw[](tid0) -- (tid2);
        \end{tikzpicture}
        \nodepart{two}
        \footnotesize{5.29688}
        \nodepart{three}
        \footnotesize{$50\:33\:17$}
      };
      & 
      \node[draw=black, rectangle split,  rectangle split parts=3] (sn0x10581e0){
        \begin{tikzpicture}[scale=.2]
          \node[circle, scale=0.75, fill] (tid0) at (3,1.5){};
          \node[circle, scale=0.75, fill] (tid1) at (2.25,3){};
          \node[circle, scale=0.75, fill] (tid3) at (0.75,4.5){};
          \node[circle, scale=0.75, fill, red] (tid7) at (0.75,6){};
          \draw[](tid3) -- (tid7);
          \node[circle, scale=0.75, fill, red] (tid4) at (2.25,4.5){};
          \node[circle, scale=0.75, fill] (tid5) at (3.75,4.5){};
          \draw[](tid1) -- (tid3);
          \draw[](tid1) -- (tid4);
          \draw[](tid1) -- (tid5);
          \node[circle, scale=0.75, fill] (tid2) at (5.25,3){};
          \node[circle, scale=0.75, fill] (tid6) at (5.25,4.5){};
          \draw[](tid2) -- (tid6);
          \draw[](tid0) -- (tid1);
          \draw[](tid0) -- (tid2);
        \end{tikzpicture}
        \nodepart{two}
        \footnotesize{5.29688}
        \nodepart{three}
        \footnotesize{$33\:17\:25\:25$}
      };
      & 
      \node[draw=black, rectangle split,  rectangle split parts=3] (sn0x1058550){
        \begin{tikzpicture}[scale=.2]
          \node[circle, scale=0.75, fill] (tid0) at (3,1.5){};
          \node[circle, scale=0.75, fill] (tid1) at (2.25,3){};
          \node[circle, scale=0.75, fill] (tid3) at (0.75,4.5){};
          \node[circle, scale=0.75, fill, red] (tid7) at (0.75,6){};
          \draw[](tid3) -- (tid7);
          \node[circle, scale=0.75, fill] (tid4) at (2.25,4.5){};
          \node[circle, scale=0.75, fill] (tid5) at (3.75,4.5){};
          \draw[](tid1) -- (tid3);
          \draw[](tid1) -- (tid4);
          \draw[](tid1) -- (tid5);
          \node[circle, scale=0.75, fill] (tid2) at (5.25,3){};
          \node[circle, scale=0.75, fill, red] (tid6) at (5.25,4.5){};
          \draw[](tid2) -- (tid6);
          \draw[](tid0) -- (tid1);
          \draw[](tid0) -- (tid2);
        \end{tikzpicture}
        \nodepart{two}
        \footnotesize{5.29688}
        \nodepart{three}
        \footnotesize{$50\:50$}
      };
      & 
      \\
    };
  \end{scope}
  \begin{scope}[yshift=\leveltopIIIII cm]
    \matrix (line5) [column sep=1cm] {
      \node[draw=black, rectangle split,  rectangle split parts=3] (sn0x10525f0){
        \begin{tikzpicture}[scale=.2]
          \node[circle, scale=0.75, fill] (tid0) at (1.5,1.5){};
          \node[circle, scale=0.75, fill] (tid1) at (0.75,3){};
          \node[circle, scale=0.75, fill] (tid3) at (0.75,4.5){};
          \node[circle, scale=0.75, fill] (tid5) at (0.75,6){};
          \node[circle, scale=0.75, fill, red] (tid6) at (0.75,7.5){};
          \draw[](tid5) -- (tid6);
          \draw[](tid3) -- (tid5);
          \draw[](tid1) -- (tid3);
          \node[circle, scale=0.75, fill] (tid2) at (2.25,3){};
          \node[circle, scale=0.75, fill, red] (tid4) at (2.25,4.5){};
          \draw[](tid2) -- (tid4);
          \draw[](tid0) -- (tid1);
          \draw[](tid0) -- (tid2);
        \end{tikzpicture}
        \nodepart{two}
        \footnotesize{5.25}
        \nodepart{three}
        \footnotesize{$50\:50$}
      };
      & 
      \node[draw=black, rectangle split,  rectangle split parts=3] (sn0x1053850){
        \begin{tikzpicture}[scale=.2]
          \node[circle, scale=0.75, fill] (tid0) at (2.25,1.5){};
          \node[circle, scale=0.75, fill] (tid1) at (1.5,3){};
          \node[circle, scale=0.75, fill, red] (tid3) at (0.75,4.5){};
          \node[circle, scale=0.75, fill] (tid4) at (2.25,4.5){};
          \draw[](tid1) -- (tid3);
          \draw[](tid1) -- (tid4);
          \node[circle, scale=0.75, fill] (tid2) at (3.75,3){};
          \node[circle, scale=0.75, fill] (tid5) at (3.75,4.5){};
          \node[circle, scale=0.75, fill, red] (tid6) at (3.75,6){};
          \draw[](tid5) -- (tid6);
          \draw[](tid2) -- (tid5);
          \draw[](tid0) -- (tid1);
          \draw[](tid0) -- (tid2);
        \end{tikzpicture}
        \nodepart{two}
        \footnotesize{4.84375}
        \nodepart{three}
        \footnotesize{$50\:25\:25$}
      };
      & 
      \node[draw=black, rectangle split,  rectangle split parts=3] (sn0x1056b00){
        \begin{tikzpicture}[scale=.2]
          \node[circle, scale=0.75, fill] (tid0) at (3,1.5){};
          \node[circle, scale=0.75, fill] (tid1) at (2.25,3){};
          \node[circle, scale=0.75, fill, red] (tid3) at (0.75,4.5){};
          \node[circle, scale=0.75, fill, red] (tid4) at (2.25,4.5){};
          \node[circle, scale=0.75, fill] (tid5) at (3.75,4.5){};
          \draw[](tid1) -- (tid3);
          \draw[](tid1) -- (tid4);
          \draw[](tid1) -- (tid5);
          \node[circle, scale=0.75, fill] (tid2) at (5.25,3){};
          \node[circle, scale=0.75, fill] (tid6) at (5.25,4.5){};
          \draw[](tid2) -- (tid6);
          \draw[](tid0) -- (tid1);
          \draw[](tid0) -- (tid2);
        \end{tikzpicture}
        \nodepart{two}
        \footnotesize{4.75}
        \nodepart{three}
        \footnotesize{$50\:50$}
      };
      & 
      \node[draw=black, rectangle split,  rectangle split parts=3] (sn0x1056fb0){
        \begin{tikzpicture}[scale=.2]
          \node[circle, scale=0.75, fill] (tid0) at (3,1.5){};
          \node[circle, scale=0.75, fill] (tid1) at (2.25,3){};
          \node[circle, scale=0.75, fill, red] (tid3) at (0.75,4.5){};
          \node[circle, scale=0.75, fill] (tid4) at (2.25,4.5){};
          \node[circle, scale=0.75, fill] (tid5) at (3.75,4.5){};
          \draw[](tid1) -- (tid3);
          \draw[](tid1) -- (tid4);
          \draw[](tid1) -- (tid5);
          \node[circle, scale=0.75, fill] (tid2) at (5.25,3){};
          \node[circle, scale=0.75, fill, red] (tid6) at (5.25,4.5){};
          \draw[](tid2) -- (tid6);
          \draw[](tid0) -- (tid1);
          \draw[](tid0) -- (tid2);
        \end{tikzpicture}
        \nodepart{two}
        \footnotesize{4.75}
        \nodepart{three}
        \footnotesize{$50\:50$}
      };
      & 
      \node[draw=black, rectangle split,  rectangle split parts=3] (sn0x1058f50){
        \begin{tikzpicture}[scale=.2]
          \node[circle, scale=0.75, fill] (tid0) at (2.25,1.5){};
          \node[circle, scale=0.75, fill] (tid1) at (1.5,3){};
          \node[circle, scale=0.75, fill] (tid3) at (0.75,4.5){};
          \node[circle, scale=0.75, fill, red] (tid6) at (0.75,6){};
          \draw[](tid3) -- (tid6);
          \node[circle, scale=0.75, fill, red] (tid4) at (2.25,4.5){};
          \draw[](tid1) -- (tid3);
          \draw[](tid1) -- (tid4);
          \node[circle, scale=0.75, fill] (tid2) at (3.75,3){};
          \node[circle, scale=0.75, fill] (tid5) at (3.75,4.5){};
          \draw[](tid2) -- (tid5);
          \draw[](tid0) -- (tid1);
          \draw[](tid0) -- (tid2);
        \end{tikzpicture}
        \nodepart{two}
        \footnotesize{4.84375}
        \nodepart{three}
        \footnotesize{$50\:25\:25$}
      };
      & 
      \node[draw=black, rectangle split,  rectangle split parts=3] (sn0x1058a50){
        \begin{tikzpicture}[scale=.2]
          \node[circle, scale=0.75, fill] (tid0) at (2.25,1.5){};
          \node[circle, scale=0.75, fill] (tid1) at (1.5,3){};
          \node[circle, scale=0.75, fill] (tid3) at (0.75,4.5){};
          \node[circle, scale=0.75, fill, red] (tid6) at (0.75,6){};
          \draw[](tid3) -- (tid6);
          \node[circle, scale=0.75, fill] (tid4) at (2.25,4.5){};
          \draw[](tid1) -- (tid3);
          \draw[](tid1) -- (tid4);
          \node[circle, scale=0.75, fill] (tid2) at (3.75,3){};
          \node[circle, scale=0.75, fill, red] (tid5) at (3.75,4.5){};
          \draw[](tid2) -- (tid5);
          \draw[](tid0) -- (tid1);
          \draw[](tid0) -- (tid2);
        \end{tikzpicture}
        \nodepart{two}
        \footnotesize{4.84375}
        \nodepart{three}
        \footnotesize{$50\:50$}
      };
      & 
      \node[draw=black, rectangle split,  rectangle split parts=3] (sn0x10597a0){
        \begin{tikzpicture}[scale=.2]
          \node[circle, scale=0.75, fill] (tid0) at (3,1.5){};
          \node[circle, scale=0.75, fill] (tid1) at (2.25,3){};
          \node[circle, scale=0.75, fill] (tid3) at (0.75,4.5){};
          \node[circle, scale=0.75, fill, red] (tid6) at (0.75,6){};
          \draw[](tid3) -- (tid6);
          \node[circle, scale=0.75, fill, red] (tid4) at (2.25,4.5){};
          \node[circle, scale=0.75, fill] (tid5) at (3.75,4.5){};
          \draw[](tid1) -- (tid3);
          \draw[](tid1) -- (tid4);
          \draw[](tid1) -- (tid5);
          \node[circle, scale=0.75, fill] (tid2) at (5.25,3){};
          \draw[](tid0) -- (tid1);
          \draw[](tid0) -- (tid2);
        \end{tikzpicture}
        \nodepart{two}
        \footnotesize{4.84375}
        \nodepart{three}
        \footnotesize{$50\:50$}
      };
      & 
      \\
    };
  \end{scope}
  \begin{scope}[yshift=\leveltopIIIIII cm]
    \matrix (line6) [column sep=1cm] {
      \node[draw=black, rectangle split,  rectangle split parts=3] (sn0x1053920){
        \begin{tikzpicture}[scale=.2]
          \node[circle, scale=0.75, fill] (tid0) at (1.5,1.5){};
          \node[circle, scale=0.75, fill] (tid1) at (0.75,3){};
          \node[circle, scale=0.75, fill] (tid3) at (0.75,4.5){};
          \node[circle, scale=0.75, fill] (tid4) at (0.75,6){};
          \node[circle, scale=0.75, fill, red] (tid5) at (0.75,7.5){};
          \draw[](tid4) -- (tid5);
          \draw[](tid3) -- (tid4);
          \draw[](tid1) -- (tid3);
          \node[circle, scale=0.75, fill, red] (tid2) at (2.25,3){};
          \draw[](tid0) -- (tid1);
          \draw[](tid0) -- (tid2);
        \end{tikzpicture}
        \nodepart{two}
        \footnotesize{5.0625}
        \nodepart{three}
        \footnotesize{$50\:50$}
      };
      & 
      \node[draw=black, rectangle split,  rectangle split parts=3] (sn0x1053bc0){
        \begin{tikzpicture}[scale=.2]
          \node[circle, scale=0.75, fill] (tid0) at (1.5,1.5){};
          \node[circle, scale=0.75, fill] (tid1) at (0.75,3){};
          \node[circle, scale=0.75, fill] (tid3) at (0.75,4.5){};
          \node[circle, scale=0.75, fill, red] (tid5) at (0.75,6){};
          \draw[](tid3) -- (tid5);
          \draw[](tid1) -- (tid3);
          \node[circle, scale=0.75, fill] (tid2) at (2.25,3){};
          \node[circle, scale=0.75, fill, red] (tid4) at (2.25,4.5){};
          \draw[](tid2) -- (tid4);
          \draw[](tid0) -- (tid1);
          \draw[](tid0) -- (tid2);
        \end{tikzpicture}
        \nodepart{two}
        \footnotesize{4.4375}
        \nodepart{three}
        \footnotesize{$50\:50$}
      };
      & 
      \node[draw=black, rectangle split,  rectangle split parts=3] (sn0x1056090){
        \begin{tikzpicture}[scale=.2]
          \node[circle, scale=0.75, fill] (tid0) at (2.25,1.5){};
          \node[circle, scale=0.75, fill] (tid1) at (1.5,3){};
          \node[circle, scale=0.75, fill, red] (tid3) at (0.75,4.5){};
          \node[circle, scale=0.75, fill, red] (tid4) at (2.25,4.5){};
          \draw[](tid1) -- (tid3);
          \draw[](tid1) -- (tid4);
          \node[circle, scale=0.75, fill] (tid2) at (3.75,3){};
          \node[circle, scale=0.75, fill] (tid5) at (3.75,4.5){};
          \draw[](tid2) -- (tid5);
          \draw[](tid0) -- (tid1);
          \draw[](tid0) -- (tid2);
        \end{tikzpicture}
        \nodepart{two}
        \footnotesize{4.25}
        \nodepart{three}
        \footnotesize{$1$}
      };
      & 
      \node[draw=black, rectangle split,  rectangle split parts=3] (sn0x1056160){
        \begin{tikzpicture}[scale=.2]
          \node[circle, scale=0.75, fill] (tid0) at (2.25,1.5){};
          \node[circle, scale=0.75, fill] (tid1) at (1.5,3){};
          \node[circle, scale=0.75, fill, red] (tid3) at (0.75,4.5){};
          \node[circle, scale=0.75, fill] (tid4) at (2.25,4.5){};
          \draw[](tid1) -- (tid3);
          \draw[](tid1) -- (tid4);
          \node[circle, scale=0.75, fill] (tid2) at (3.75,3){};
          \node[circle, scale=0.75, fill, red] (tid5) at (3.75,4.5){};
          \draw[](tid2) -- (tid5);
          \draw[](tid0) -- (tid1);
          \draw[](tid0) -- (tid2);
        \end{tikzpicture}
        \nodepart{two}
        \footnotesize{4.25}
        \nodepart{three}
        \footnotesize{$50\:50$}
      };
      & 
      \node[draw=black, rectangle split,  rectangle split parts=3] (sn0x1057630){
        \begin{tikzpicture}[scale=.2]
          \node[circle, scale=0.75, fill] (tid0) at (3,1.5){};
          \node[circle, scale=0.75, fill] (tid1) at (2.25,3){};
          \node[circle, scale=0.75, fill, red] (tid3) at (0.75,4.5){};
          \node[circle, scale=0.75, fill, red] (tid4) at (2.25,4.5){};
          \node[circle, scale=0.75, fill] (tid5) at (3.75,4.5){};
          \draw[](tid1) -- (tid3);
          \draw[](tid1) -- (tid4);
          \draw[](tid1) -- (tid5);
          \node[circle, scale=0.75, fill] (tid2) at (5.25,3){};
          \draw[](tid0) -- (tid1);
          \draw[](tid0) -- (tid2);
        \end{tikzpicture}
        \nodepart{two}
        \footnotesize{4.25}
        \nodepart{three}
        \footnotesize{$1$}
      };
      & 
      \node[draw=black, rectangle split,  rectangle split parts=3] (sn0x1058b20){
        \begin{tikzpicture}[scale=.2]
          \node[circle, scale=0.75, fill] (tid0) at (2.25,1.5){};
          \node[circle, scale=0.75, fill] (tid1) at (1.5,3){};
          \node[circle, scale=0.75, fill] (tid3) at (0.75,4.5){};
          \node[circle, scale=0.75, fill, red] (tid5) at (0.75,6){};
          \draw[](tid3) -- (tid5);
          \node[circle, scale=0.75, fill, red] (tid4) at (2.25,4.5){};
          \draw[](tid1) -- (tid3);
          \draw[](tid1) -- (tid4);
          \node[circle, scale=0.75, fill] (tid2) at (3.75,3){};
          \draw[](tid0) -- (tid1);
          \draw[](tid0) -- (tid2);
        \end{tikzpicture}
        \nodepart{two}
        \footnotesize{4.4375}
        \nodepart{three}
        \footnotesize{$50\:50$}
      };
      & 
      \\
    };
  \end{scope}
  \begin{scope}[yshift=\leveltopIIIIIII cm]
    \matrix (line7) [column sep=1cm] {
      \node[draw=black, rectangle split,  rectangle split parts=3] (sn0x10540d0){
        \begin{tikzpicture}[scale=.2]
          \node[circle, scale=0.75, fill] (tid0) at (0.75,1.5){};
          \node[circle, scale=0.75, fill] (tid1) at (0.75,3){};
          \node[circle, scale=0.75, fill] (tid2) at (0.75,4.5){};
          \node[circle, scale=0.75, fill] (tid3) at (0.75,6){};
          \node[circle, scale=0.75, fill, red] (tid4) at (0.75,7.5){};
          \draw[](tid3) -- (tid4);
          \draw[](tid2) -- (tid3);
          \draw[](tid1) -- (tid2);
          \draw[](tid0) -- (tid1);
        \end{tikzpicture}
        \nodepart{two}
        \footnotesize{5}
        \nodepart{three}
        \footnotesize{$1$}
      };
      & 
      \node[draw=black, rectangle split,  rectangle split parts=3] (sn0x1054480){
        \begin{tikzpicture}[scale=.2]
          \node[circle, scale=0.75, fill] (tid0) at (1.5,1.5){};
          \node[circle, scale=0.75, fill] (tid1) at (0.75,3){};
          \node[circle, scale=0.75, fill] (tid3) at (0.75,4.5){};
          \node[circle, scale=0.75, fill, red] (tid4) at (0.75,6){};
          \draw[](tid3) -- (tid4);
          \draw[](tid1) -- (tid3);
          \node[circle, scale=0.75, fill, red] (tid2) at (2.25,3){};
          \draw[](tid0) -- (tid1);
          \draw[](tid0) -- (tid2);
        \end{tikzpicture}
        \nodepart{two}
        \footnotesize{4.125}
        \nodepart{three}
        \footnotesize{$50\:50$}
      };
      & 
      \node[draw=black, rectangle split,  rectangle split parts=3] (sn0x1055dd0){
        \begin{tikzpicture}[scale=.2]
          \node[circle, scale=0.75, fill] (tid0) at (1.5,1.5){};
          \node[circle, scale=0.75, fill] (tid1) at (0.75,3){};
          \node[circle, scale=0.75, fill, red] (tid3) at (0.75,4.5){};
          \draw[](tid1) -- (tid3);
          \node[circle, scale=0.75, fill] (tid2) at (2.25,3){};
          \node[circle, scale=0.75, fill, red] (tid4) at (2.25,4.5){};
          \draw[](tid2) -- (tid4);
          \draw[](tid0) -- (tid1);
          \draw[](tid0) -- (tid2);
        \end{tikzpicture}
        \nodepart{two}
        \footnotesize{3.75}
        \nodepart{three}
        \footnotesize{$1$}
      };
      & 
      \node[draw=black, rectangle split,  rectangle split parts=3] (sn0x10568c0){
        \begin{tikzpicture}[scale=.2]
          \node[circle, scale=0.75, fill] (tid0) at (2.25,1.5){};
          \node[circle, scale=0.75, fill] (tid1) at (1.5,3){};
          \node[circle, scale=0.75, fill, red] (tid3) at (0.75,4.5){};
          \node[circle, scale=0.75, fill, red] (tid4) at (2.25,4.5){};
          \draw[](tid1) -- (tid3);
          \draw[](tid1) -- (tid4);
          \node[circle, scale=0.75, fill] (tid2) at (3.75,3){};
          \draw[](tid0) -- (tid1);
          \draw[](tid0) -- (tid2);
        \end{tikzpicture}
        \nodepart{two}
        \footnotesize{3.75}
        \nodepart{three}
        \footnotesize{$1$}
      };
      & 
      \\
    };
  \end{scope}
  \begin{scope}[yshift=\leveltopIIIIIIII cm]
    \matrix (line8) [column sep=1cm] {
      \node[draw=black, rectangle split,  rectangle split parts=3] (sn0x1054550){
        \begin{tikzpicture}[scale=.2]
          \node[circle, scale=0.75, fill] (tid0) at (0.75,1.5){};
          \node[circle, scale=0.75, fill] (tid1) at (0.75,3){};
          \node[circle, scale=0.75, fill] (tid2) at (0.75,4.5){};
          \node[circle, scale=0.75, fill, red] (tid3) at (0.75,6){};
          \draw[](tid2) -- (tid3);
          \draw[](tid1) -- (tid2);
          \draw[](tid0) -- (tid1);
        \end{tikzpicture}
        \nodepart{two}
        \footnotesize{4}
        \nodepart{three}
        \footnotesize{$1$}
      };
      & 
      \node[draw=black, rectangle split,  rectangle split parts=3] (sn0x1055270){
        \begin{tikzpicture}[scale=.2]
          \node[circle, scale=0.75, fill] (tid0) at (1.5,1.5){};
          \node[circle, scale=0.75, fill] (tid1) at (0.75,3){};
          \node[circle, scale=0.75, fill, red] (tid3) at (0.75,4.5){};
          \draw[](tid1) -- (tid3);
          \node[circle, scale=0.75, fill, red] (tid2) at (2.25,3){};
          \draw[](tid0) -- (tid1);
          \draw[](tid0) -- (tid2);
        \end{tikzpicture}
        \nodepart{two}
        \footnotesize{3.25}
        \nodepart{three}
        \footnotesize{$50\:50$}
      };
      & 
      \\
    };
  \end{scope}
  \begin{scope}[yshift=\leveltopIIIIIIIII cm]
    \matrix (line9) [column sep=1cm] {
      \node[draw=black, rectangle split,  rectangle split parts=3] (sn0x1054a50){
        \begin{tikzpicture}[scale=.2]
          \node[circle, scale=0.75, fill] (tid0) at (0.75,1.5){};
          \node[circle, scale=0.75, fill] (tid1) at (0.75,3){};
          \node[circle, scale=0.75, fill, red] (tid2) at (0.75,4.5){};
          \draw[](tid1) -- (tid2);
          \draw[](tid0) -- (tid1);
        \end{tikzpicture}
        \nodepart{two}
        \footnotesize{3}
        \nodepart{three}
        \footnotesize{$1$}
      };
      & 
      \node[draw=black, rectangle split,  rectangle split parts=3] (sn0x1054cb0){
        \begin{tikzpicture}[scale=.2]
          \node[circle, scale=0.75, fill] (tid0) at (1.5,1.5){};
          \node[circle, scale=0.75, fill, red] (tid1) at (0.75,3){};
          \node[circle, scale=0.75, fill, red] (tid2) at (2.25,3){};
          \draw[](tid0) -- (tid1);
          \draw[](tid0) -- (tid2);
        \end{tikzpicture}
        \nodepart{two}
        \footnotesize{2.5}
        \nodepart{three}
        \footnotesize{$1$}
      };
      & 
      \\
    };
  \end{scope}
  \begin{scope}[yshift=\leveltopIIIIIIIIII cm]
    \matrix (line10) [column sep=1cm] {
      \node[draw=black, rectangle split,  rectangle split parts=3] (sn0x1054b20){
        \begin{tikzpicture}[scale=.2]
          \node[circle, scale=0.75, fill] (tid0) at (0.75,1.5){};
          \node[circle, scale=0.75, fill, red] (tid1) at (0.75,3){};
          \draw[](tid0) -- (tid1);
        \end{tikzpicture}
        \nodepart{two}
        \footnotesize{2}
        \nodepart{three}
        \footnotesize{$1$}
      };
      & 
      \\
    };
  \end{scope}
  \begin{scope}[yshift=\leveltopIIIIIIIIIII cm]
    \matrix (line11) [column sep=1cm] {
      \node[draw=black, rectangle split,  rectangle split parts=3] (sn0x10547e0){
        \begin{tikzpicture}[scale=.2]
          \node[circle, scale=0.75, fill, red] (tid0) at (0.75,1.5){};
        \end{tikzpicture}
        \nodepart{two}
        \footnotesize{1}
        \nodepart{three}
        \footnotesize{$$}
      };
      & 
      \\
    };
  \end{scope}
  \begin{scope}[yshift=\leveltopIIIIIIIIIIII cm]
    \matrix (line12) [column sep=1cm] {
      \\
    };
  \end{scope}
  \draw (sn0x104f980.south) -- (sn0x1050190.north);
  \draw (sn0x104f980.south) -- (sn0x104cb60.north);
  \draw (sn0x104f980.south) -- (sn0x104dd20.north);
  \draw (sn0x1050190.south) -- (sn0x10519d0.north);
  \draw (sn0x1050190.south) -- (sn0x104fbd0.north);
  \draw (sn0x104cb60.south) -- (sn0x105a080.north);
  \draw (sn0x104cb60.south) -- (sn0x104fbd0.north);
  \draw (sn0x104dd20.south) -- (sn0x104fbd0.north);
  \draw (sn0x10519d0.south) -- (sn0x1052250.north);
  \draw (sn0x10519d0.south) -- (sn0x1052960.north);
  \draw (sn0x104fbd0.south) -- (sn0x1052960.north);
  \draw (sn0x104fbd0.south) -- (sn0x10581e0.north);
  \draw (sn0x104fbd0.south) -- (sn0x1058550.north);
  \draw (sn0x105a080.south) -- (sn0x1052960.north);
  \draw (sn0x1052250.south) -- (sn0x10525f0.north);
  \draw (sn0x1052250.south) -- (sn0x1053850.north);
  \draw (sn0x1052960.south) -- (sn0x1053850.north);
  \draw (sn0x1052960.south) -- (sn0x1056b00.north);
  \draw (sn0x1052960.south) -- (sn0x1056fb0.north);
  \draw (sn0x10581e0.south) -- (sn0x1058f50.north);
  \draw (sn0x10581e0.south) -- (sn0x1058a50.north);
  \draw (sn0x10581e0.south) -- (sn0x1056b00.north);
  \draw (sn0x10581e0.south) -- (sn0x1056fb0.north);
  \draw (sn0x1058550.south) -- (sn0x10597a0.north);
  \draw (sn0x1058550.south) -- (sn0x1056fb0.north);
  \draw (sn0x10525f0.south) -- (sn0x1053920.north);
  \draw (sn0x10525f0.south) -- (sn0x1053bc0.north);
  \draw (sn0x1053850.south) -- (sn0x1053bc0.north);
  \draw (sn0x1053850.south) -- (sn0x1056090.north);
  \draw (sn0x1053850.south) -- (sn0x1056160.north);
  \draw (sn0x1056b00.south) -- (sn0x1056090.north);
  \draw (sn0x1056b00.south) -- (sn0x1056160.north);
  \draw (sn0x1056fb0.south) -- (sn0x1056160.north);
  \draw (sn0x1056fb0.south) -- (sn0x1057630.north);
  \draw (sn0x1058f50.south) -- (sn0x1053bc0.north);
  \draw (sn0x1058f50.south) -- (sn0x1056090.north);
  \draw (sn0x1058f50.south) -- (sn0x1056160.north);
  \draw (sn0x1058a50.south) -- (sn0x1058b20.north);
  \draw (sn0x1058a50.south) -- (sn0x1056160.north);
  \draw (sn0x10597a0.south) -- (sn0x1058b20.north);
  \draw (sn0x10597a0.south) -- (sn0x1057630.north);
  \draw (sn0x1053920.south) -- (sn0x10540d0.north);
  \draw (sn0x1053920.south) -- (sn0x1054480.north);
  \draw (sn0x1053bc0.south) -- (sn0x1054480.north);
  \draw (sn0x1053bc0.south) -- (sn0x1055dd0.north);
  \draw (sn0x1056090.south) -- (sn0x1055dd0.north);
  \draw (sn0x1056160.south) -- (sn0x1055dd0.north);
  \draw (sn0x1056160.south) -- (sn0x10568c0.north);
  \draw (sn0x1057630.south) -- (sn0x10568c0.north);
  \draw (sn0x1058b20.south) -- (sn0x1054480.north);
  \draw (sn0x1058b20.south) -- (sn0x10568c0.north);
  \draw (sn0x10540d0.south) -- (sn0x1054550.north);
  \draw (sn0x1054480.south) -- (sn0x1054550.north);
  \draw (sn0x1054480.south) -- (sn0x1055270.north);
  \draw (sn0x1055dd0.south) -- (sn0x1055270.north);
  \draw (sn0x10568c0.south) -- (sn0x1055270.north);
  \draw (sn0x1054550.south) -- (sn0x1054a50.north);
  \draw (sn0x1055270.south) -- (sn0x1054a50.north);
  \draw (sn0x1055270.south) -- (sn0x1054cb0.north);
  \draw (sn0x1054a50.south) -- (sn0x1054b20.north);
  \draw (sn0x1054cb0.south) -- (sn0x1054b20.north);
  \draw (sn0x1054b20.south) -- (sn0x10547e0.north);

  \newcommand{\nd}[4]{
    \node[draw=black, rectangle split, rectangle split parts=3] (n#1#2) {
      $#1/#2$
      \nodepart{two}
      #3
      \nodepart{three}
      #4
    };
  }

  \begin{scope}[yshift=\leveltopI, xshift=70cm, rectangle, draw=black,anchor=south]
    \matrix (test) [column sep=1cm] {
      \nd{5}{6}{6.82812}{50 50};
      \\
    };
  \end{scope}

  \begin{scope}[yshift=\leveltopII, xshift=70cm, rectangle, draw=black,anchor=south]
    \matrix (test) [column sep=1cm] {
      \nd{5}{5}{6.35938}{50 50};
      &
      \nd{4}{6}{6.29688}{1};
      \\
      };
    \end{scope}

    \begin{scope}[yshift=\leveltopIII, xshift=70cm, rectangle, draw=black,anchor=south]
      \matrix (test) [column sep=1cm] {
        \nd{5}{4}{5.92188}{50 50};
        &
        \nd{4}{5}{5.79688}{1};
        \\
      };
    \end{scope}

    \begin{scope}[yshift=\leveltopIIII, xshift=70cm, rectangle, draw=black,anchor=south]
      \matrix (test) [column sep=1cm] {
        \nd{5}{3}{5.54688}{50 50};
        &
        \nd{4}{4}{5.29688}{50 50};
        \\
      };
    \end{scope}

    \begin{scope}[yshift=\leveltopIIIII, xshift=70cm, rectangle, draw=black,anchor=south]
      \matrix (test) [column sep=1cm] {
        \nd{5}{2}{5.25}{50 50};
        &
        \nd{4}{3}{4.84375}{50 50};
        &
        \nd{3}{4}{5.29688}{1};
        \\
      };
    \end{scope}

    \begin{scope}[yshift=\leveltopIIIIII, xshift=70cm, rectangle, draw=black,anchor=south]
      \matrix (test) [column sep=1cm] {
        \nd{5}{1}{5.0625}{50 50};
        &
        \nd{4}{2}{4.4375}{50 50};
        &
        \nd{3}{3}{5.75}{1};
        \\
      };
    \end{scope}

    \begin{scope}[yshift=\leveltopIIIIIII, xshift=70cm, rectangle, draw=black,anchor=south]
      \matrix (test) [column sep=1cm] {
        \nd{5}{0}{5}{50 50};
        &
        \nd{4}{1}{4.125}{50 50};
        &
        \nd{3}{2}{5.25}{1};
        \\
      };
    \end{scope}
    
    \begin{scope}[yshift=\leveltopIIIIIIII, xshift=70cm, rectangle, draw=black,anchor=south]
      \matrix (test) [column sep=1cm] {
        \nd{4}{0}{4}{1};
        &
        \nd{3}{1}{3.25}{50 50};
        \\
      };
    \end{scope}

    \begin{scope}[yshift=\leveltopIIIIIIIII, xshift=70cm, rectangle, draw=black,anchor=south]
      \matrix (test) [column sep=1cm] {
        \nd{3}{0}{3}{1};
        &
        \nd{2}{1}{2.5}{1};
        \\
      };
    \end{scope}

    \begin{scope}[yshift=\leveltopIIIIIIIIII, xshift=70cm, rectangle, draw=black,anchor=south]
      \matrix (test) [column sep=1cm] {
        \nd{2}{0}{2}{1};
        \\
      };
    \end{scope}
    
    \begin{scope}[yshift=\leveltopIIIIIIIIIII, xshift=70cm, rectangle, draw=black,anchor=south]
      \matrix (test) [column sep=1cm] {
        \nd{1}{0}{1}{1};
        \\
      };
    \end{scope}

    \draw (n56.south) -- (n55.north);
    \draw (n56.south) -- (n46.north);
    \draw (n55.south) -- (n54.north);
    \draw (n55.south) -- (n45.north);
    \draw (n46.south) -- (n45.north);
    \draw (n54.south) -- (n53.north);
    \draw (n54.south) -- (n44.north);
    \draw (n45.south) -- (n44.north);
    \draw (n53.south) -- (n52.north);
    \draw (n53.south) -- (n43.north);
    \draw (n44.south) -- (n43.north);
    \draw (n44.south) -- (n34.north);
    \draw (n52.south) -- (n51.north);
    \draw (n52.south) -- (n42.north);
    \draw (n43.south) -- (n42.north);
    \draw (n43.south) -- (n33.north);
    \draw (n34.south) -- (n33.north);
    \draw (n51.south) -- (n50.north);
    \draw (n51.south) -- (n41.north);
    \draw (n42.south) -- (n41.north);
    \draw (n42.south) -- (n32.north);
    \draw (n33.south) -- (n32.north);
    \draw (n32.south) -- (n31.north);
    \draw (n50.south) -- (n40.north);
    \draw (n41.south) -- (n40.north);
    \draw (n41.south) -- (n31.north);
    \draw (n40.south) -- (n30.north);
    \draw (n31.south) -- (n30.north);
    \draw (n31.south) -- (n21.north);
    \draw (n30.south) -- (n20.north);
    \draw (n21.south) -- (n20.north);
    \draw (n20.south) -- (n10.north);

\end{tikzpicture}

%%% Local Variables:
%%% TeX-master: "thesis/thesis.tex"
%%% End: 
\renewcommand{\leveltopI}{-15cm + \leveltop}
\renewcommand{\leveltopII}{-15cm + \leveltopI}
\renewcommand{\leveltopIII}{-15cm + \leveltopII}
\renewcommand{\leveltopIIII}{-15cm + \leveltopIII}
\renewcommand{\leveltopIIIII}{-15cm + \leveltopIIII}
\renewcommand{\leveltopIIIIII}{-15cm + \leveltopIIIII}
\renewcommand{\leveltopIIIIIII}{-15cm + \leveltopIIIIII}
\renewcommand{\leveltopIIIIIIII}{-15cm + \leveltopIIIIIII}
\renewcommand{\leveltopIIIIIIIII}{-15cm + \leveltopIIIIIIII}
\renewcommand{\leveltopIIIIIIIIII}{-15cm + \leveltopIIIIIIIII}
\renewcommand{\leveltopIIIIIIIIIII}{-15cm + \leveltopIIIIIIIIII}
% \begin{tikzpicture}[scale=.2, anchor=south]
%   \begin{scope}[yshift=\leveltopI cm]
%     \matrix (line1) [column sep=1cm] {
%       \node[draw=black, rectangle split,  rectangle split parts=3] (sn0x1050af0){
%         \begin{tikzpicture}[scale=.2]
%           \node[circle, scale=0.75, fill] (tid0) at (3.75,1.5){};
%           \node[circle, scale=0.75, fill] (tid1) at (2.25,3){};
%           \node[circle, scale=0.75, fill] (tid3) at (0.75,4.5){};
%           \node[circle, scale=0.75, fill, red] (tid7) at (0.75,6){};
%           \draw[](tid3) -- (tid7);
%           \node[circle, scale=0.75, fill] (tid4) at (2.25,4.5){};
%           \node[circle, scale=0.75, fill] (tid5) at (3.75,4.5){};
%           \draw[](tid1) -- (tid3);
%           \draw[](tid1) -- (tid4);
%           \draw[](tid1) -- (tid5);
%           \node[circle, scale=0.75, fill] (tid2) at (6,3){};
%           \node[circle, scale=0.75, fill] (tid6) at (6,4.5){};
%           \node[circle, scale=0.75, fill] (tid8) at (5.25,6){};
%           \node[circle, scale=0.75, fill, red] (tid10) at (5.25,7.5){};
%           \draw[](tid8) -- (tid10);
%           \node[circle, scale=0.75, fill] (tid9) at (6.75,6){};
%           \draw[](tid6) -- (tid8);
%           \draw[](tid6) -- (tid9);
%           \draw[](tid2) -- (tid6);
%           \draw[](tid0) -- (tid1);
%           \draw[](tid0) -- (tid2);
%         \end{tikzpicture}
%         \nodepart{two}
%         \footnotesize{6.82812}
%         \nodepart{three}
%         \footnotesize{$50\:50$}
%       };
%       & 
%       \\
%     };
%   \end{scope}
%   \begin{scope}[yshift=\leveltopII cm]
%     \matrix (line2) [column sep=1cm] {
%       \node[draw=black, rectangle split,  rectangle split parts=3] (sn0x105a150){
%         \begin{tikzpicture}[scale=.2]
%           \node[circle, scale=0.75, fill] (tid0) at (3.75,1.5){};
%           \node[circle, scale=0.75, fill] (tid1) at (1.5,3){};
%           \node[circle, scale=0.75, fill] (tid3) at (1.5,4.5){};
%           \node[circle, scale=0.75, fill] (tid7) at (0.75,6){};
%           \node[circle, scale=0.75, fill, red] (tid9) at (0.75,7.5){};
%           \draw[](tid7) -- (tid9);
%           \node[circle, scale=0.75, fill, red] (tid8) at (2.25,6){};
%           \draw[](tid3) -- (tid7);
%           \draw[](tid3) -- (tid8);
%           \draw[](tid1) -- (tid3);
%           \node[circle, scale=0.75, fill] (tid2) at (5.25,3){};
%           \node[circle, scale=0.75, fill] (tid4) at (3.75,4.5){};
%           \node[circle, scale=0.75, fill] (tid5) at (5.25,4.5){};
%           \node[circle, scale=0.75, fill] (tid6) at (6.75,4.5){};
%           \draw[](tid2) -- (tid4);
%           \draw[](tid2) -- (tid5);
%           \draw[](tid2) -- (tid6);
%           \draw[](tid0) -- (tid1);
%           \draw[](tid0) -- (tid2);
%         \end{tikzpicture}
%         \nodepart{two}
%         \footnotesize{6.35938}
%         \nodepart{three}
%         \footnotesize{$50\:50$}
%       };
%       & 
%       \node[draw=black, rectangle split,  rectangle split parts=3] (sn0x104cb60){
%         \begin{tikzpicture}[scale=.2]
%           \node[circle, scale=0.75, fill] (tid0) at (3.75,1.5){};
%           \node[circle, scale=0.75, fill] (tid1) at (2.25,3){};
%           \node[circle, scale=0.75, fill] (tid3) at (0.75,4.5){};
%           \node[circle, scale=0.75, fill, red] (tid7) at (0.75,6){};
%           \draw[](tid3) -- (tid7);
%           \node[circle, scale=0.75, fill] (tid4) at (2.25,4.5){};
%           \node[circle, scale=0.75, fill] (tid5) at (3.75,4.5){};
%           \draw[](tid1) -- (tid3);
%           \draw[](tid1) -- (tid4);
%           \draw[](tid1) -- (tid5);
%           \node[circle, scale=0.75, fill] (tid2) at (6,3){};
%           \node[circle, scale=0.75, fill] (tid6) at (6,4.5){};
%           \node[circle, scale=0.75, fill, red] (tid8) at (5.25,6){};
%           \node[circle, scale=0.75, fill] (tid9) at (6.75,6){};
%           \draw[](tid6) -- (tid8);
%           \draw[](tid6) -- (tid9);
%           \draw[](tid2) -- (tid6);
%           \draw[](tid0) -- (tid1);
%           \draw[](tid0) -- (tid2);
%         \end{tikzpicture}
%         \nodepart{two}
%         \footnotesize{6.29688}
%         \nodepart{three}
%         \footnotesize{$50\:50$}
%       };
%       & 
%       \\
%     };
%   \end{scope}
%   \begin{scope}[yshift=\leveltopIII cm]
%     \matrix (line3) [column sep=1cm] {
%       \node[draw=black, rectangle split,  rectangle split parts=3] (sn0x10519d0){
%         \begin{tikzpicture}[scale=.2]
%           \node[circle, scale=0.75, fill] (tid0) at (3,1.5){};
%           \node[circle, scale=0.75, fill] (tid1) at (2.25,3){};
%           \node[circle, scale=0.75, fill, red] (tid3) at (0.75,4.5){};
%           \node[circle, scale=0.75, fill] (tid4) at (2.25,4.5){};
%           \node[circle, scale=0.75, fill] (tid5) at (3.75,4.5){};
%           \draw[](tid1) -- (tid3);
%           \draw[](tid1) -- (tid4);
%           \draw[](tid1) -- (tid5);
%           \node[circle, scale=0.75, fill] (tid2) at (5.25,3){};
%           \node[circle, scale=0.75, fill] (tid6) at (5.25,4.5){};
%           \node[circle, scale=0.75, fill] (tid7) at (5.25,6){};
%           \node[circle, scale=0.75, fill, red] (tid8) at (5.25,7.5){};
%           \draw[](tid7) -- (tid8);
%           \draw[](tid6) -- (tid7);
%           \draw[](tid2) -- (tid6);
%           \draw[](tid0) -- (tid1);
%           \draw[](tid0) -- (tid2);
%         \end{tikzpicture}
%         \nodepart{two}
%         \footnotesize{5.92188}
%         \nodepart{three}
%         \footnotesize{$50\:50$}
%       };
%       & 
%       \node[draw=black, rectangle split,  rectangle split parts=3] (sn0x105a080){
%         \begin{tikzpicture}[scale=.2]
%           \node[circle, scale=0.75, fill] (tid0) at (3.75,1.5){};
%           \node[circle, scale=0.75, fill] (tid1) at (2.25,3){};
%           \node[circle, scale=0.75, fill] (tid3) at (0.75,4.5){};
%           \node[circle, scale=0.75, fill] (tid4) at (2.25,4.5){};
%           \node[circle, scale=0.75, fill] (tid5) at (3.75,4.5){};
%           \draw[](tid1) -- (tid3);
%           \draw[](tid1) -- (tid4);
%           \draw[](tid1) -- (tid5);
%           \node[circle, scale=0.75, fill] (tid2) at (6,3){};
%           \node[circle, scale=0.75, fill] (tid6) at (6,4.5){};
%           \node[circle, scale=0.75, fill, red] (tid7) at (5.25,6){};
%           \node[circle, scale=0.75, fill, red] (tid8) at (6.75,6){};
%           \draw[](tid6) -- (tid7);
%           \draw[](tid6) -- (tid8);
%           \draw[](tid2) -- (tid6);
%           \draw[](tid0) -- (tid1);
%           \draw[](tid0) -- (tid2);
%         \end{tikzpicture}
%         \nodepart{two}
%         \footnotesize{5.79688}
%         \nodepart{three}
%         \footnotesize{$1$}
%       };
%       & 
%       \node[draw=black, rectangle split,  rectangle split parts=3] (sn0x104fbd0){
%         \begin{tikzpicture}[scale=.2]
%           \node[circle, scale=0.75, fill] (tid0) at (3,1.5){};
%           \node[circle, scale=0.75, fill] (tid1) at (2.25,3){};
%           \node[circle, scale=0.75, fill] (tid3) at (0.75,4.5){};
%           \node[circle, scale=0.75, fill, red] (tid7) at (0.75,6){};
%           \draw[](tid3) -- (tid7);
%           \node[circle, scale=0.75, fill] (tid4) at (2.25,4.5){};
%           \node[circle, scale=0.75, fill] (tid5) at (3.75,4.5){};
%           \draw[](tid1) -- (tid3);
%           \draw[](tid1) -- (tid4);
%           \draw[](tid1) -- (tid5);
%           \node[circle, scale=0.75, fill] (tid2) at (5.25,3){};
%           \node[circle, scale=0.75, fill] (tid6) at (5.25,4.5){};
%           \node[circle, scale=0.75, fill, red] (tid8) at (5.25,6){};
%           \draw[](tid6) -- (tid8);
%           \draw[](tid2) -- (tid6);
%           \draw[](tid0) -- (tid1);
%           \draw[](tid0) -- (tid2);
%         \end{tikzpicture}
%         \nodepart{two}
%         \footnotesize{5.79688}
%         \nodepart{three}
%         \footnotesize{$50\:33\:17$}
%       };
%       & 
%       \\
%     };
%   \end{scope}
%   \begin{scope}[yshift=\leveltopIIII cm]
%     \matrix (line4) [column sep=1cm] {
%       \node[draw=black, rectangle split,  rectangle split parts=3] (sn0x1052250){
%         \begin{tikzpicture}[scale=.2]
%           \node[circle, scale=0.75, fill] (tid0) at (2.25,1.5){};
%           \node[circle, scale=0.75, fill] (tid1) at (0.75,3){};
%           \node[circle, scale=0.75, fill] (tid3) at (0.75,4.5){};
%           \node[circle, scale=0.75, fill] (tid6) at (0.75,6){};
%           \node[circle, scale=0.75, fill, red] (tid7) at (0.75,7.5){};
%           \draw[](tid6) -- (tid7);
%           \draw[](tid3) -- (tid6);
%           \draw[](tid1) -- (tid3);
%           \node[circle, scale=0.75, fill] (tid2) at (3,3){};
%           \node[circle, scale=0.75, fill, red] (tid4) at (2.25,4.5){};
%           \node[circle, scale=0.75, fill] (tid5) at (3.75,4.5){};
%           \draw[](tid2) -- (tid4);
%           \draw[](tid2) -- (tid5);
%           \draw[](tid0) -- (tid1);
%           \draw[](tid0) -- (tid2);
%         \end{tikzpicture}
%         \nodepart{two}
%         \footnotesize{5.54688}
%         \nodepart{three}
%         \footnotesize{$50\:50$}
%       };
%       & 
%       \node[draw=black, rectangle split,  rectangle split parts=3] (sn0x1052960){
%         \begin{tikzpicture}[scale=.2]
%           \node[circle, scale=0.75, fill] (tid0) at (3,1.5){};
%           \node[circle, scale=0.75, fill] (tid1) at (2.25,3){};
%           \node[circle, scale=0.75, fill, red] (tid3) at (0.75,4.5){};
%           \node[circle, scale=0.75, fill] (tid4) at (2.25,4.5){};
%           \node[circle, scale=0.75, fill] (tid5) at (3.75,4.5){};
%           \draw[](tid1) -- (tid3);
%           \draw[](tid1) -- (tid4);
%           \draw[](tid1) -- (tid5);
%           \node[circle, scale=0.75, fill] (tid2) at (5.25,3){};
%           \node[circle, scale=0.75, fill] (tid6) at (5.25,4.5){};
%           \node[circle, scale=0.75, fill, red] (tid7) at (5.25,6){};
%           \draw[](tid6) -- (tid7);
%           \draw[](tid2) -- (tid6);
%           \draw[](tid0) -- (tid1);
%           \draw[](tid0) -- (tid2);
%         \end{tikzpicture}
%         \nodepart{two}
%         \footnotesize{5.29688}
%         \nodepart{three}
%         \footnotesize{$50\:33\:17$}
%       };
%       & 
%       \node[draw=black, rectangle split,  rectangle split parts=3] (sn0x10581e0){
%         \begin{tikzpicture}[scale=.2]
%           \node[circle, scale=0.75, fill] (tid0) at (3,1.5){};
%           \node[circle, scale=0.75, fill] (tid1) at (2.25,3){};
%           \node[circle, scale=0.75, fill] (tid3) at (0.75,4.5){};
%           \node[circle, scale=0.75, fill, red] (tid7) at (0.75,6){};
%           \draw[](tid3) -- (tid7);
%           \node[circle, scale=0.75, fill, red] (tid4) at (2.25,4.5){};
%           \node[circle, scale=0.75, fill] (tid5) at (3.75,4.5){};
%           \draw[](tid1) -- (tid3);
%           \draw[](tid1) -- (tid4);
%           \draw[](tid1) -- (tid5);
%           \node[circle, scale=0.75, fill] (tid2) at (5.25,3){};
%           \node[circle, scale=0.75, fill] (tid6) at (5.25,4.5){};
%           \draw[](tid2) -- (tid6);
%           \draw[](tid0) -- (tid1);
%           \draw[](tid0) -- (tid2);
%         \end{tikzpicture}
%         \nodepart{two}
%         \footnotesize{5.29688}
%         \nodepart{three}
%         \footnotesize{$33\:17\:25\:25$}
%       };
%       & 
%       \node[draw=black, rectangle split,  rectangle split parts=3] (sn0x1058550){
%         \begin{tikzpicture}[scale=.2]
%           \node[circle, scale=0.75, fill] (tid0) at (3,1.5){};
%           \node[circle, scale=0.75, fill] (tid1) at (2.25,3){};
%           \node[circle, scale=0.75, fill] (tid3) at (0.75,4.5){};
%           \node[circle, scale=0.75, fill, red] (tid7) at (0.75,6){};
%           \draw[](tid3) -- (tid7);
%           \node[circle, scale=0.75, fill] (tid4) at (2.25,4.5){};
%           \node[circle, scale=0.75, fill] (tid5) at (3.75,4.5){};
%           \draw[](tid1) -- (tid3);
%           \draw[](tid1) -- (tid4);
%           \draw[](tid1) -- (tid5);
%           \node[circle, scale=0.75, fill] (tid2) at (5.25,3){};
%           \node[circle, scale=0.75, fill, red] (tid6) at (5.25,4.5){};
%           \draw[](tid2) -- (tid6);
%           \draw[](tid0) -- (tid1);
%           \draw[](tid0) -- (tid2);
%         \end{tikzpicture}
%         \nodepart{two}
%         \footnotesize{5.29688}
%         \nodepart{three}
%         \footnotesize{$50\:50$}
%       };
%       & 
%       \\
%     };
%   \end{scope}
%   \begin{scope}[yshift=\leveltopIIIII cm]
%     \matrix (line5) [column sep=1cm] {
%       \node[draw=black, rectangle split,  rectangle split parts=3] (sn0x10525f0){
%         \begin{tikzpicture}[scale=.2]
%           \node[circle, scale=0.75, fill] (tid0) at (1.5,1.5){};
%           \node[circle, scale=0.75, fill] (tid1) at (0.75,3){};
%           \node[circle, scale=0.75, fill] (tid3) at (0.75,4.5){};
%           \node[circle, scale=0.75, fill] (tid5) at (0.75,6){};
%           \node[circle, scale=0.75, fill, red] (tid6) at (0.75,7.5){};
%           \draw[](tid5) -- (tid6);
%           \draw[](tid3) -- (tid5);
%           \draw[](tid1) -- (tid3);
%           \node[circle, scale=0.75, fill] (tid2) at (2.25,3){};
%           \node[circle, scale=0.75, fill, red] (tid4) at (2.25,4.5){};
%           \draw[](tid2) -- (tid4);
%           \draw[](tid0) -- (tid1);
%           \draw[](tid0) -- (tid2);
%         \end{tikzpicture}
%         \nodepart{two}
%         \footnotesize{5.25}
%         \nodepart{three}
%         \footnotesize{$50\:50$}
%       };
%       & 
%       \node[draw=black, rectangle split,  rectangle split parts=3] (sn0x1053850){
%         \begin{tikzpicture}[scale=.2]
%           \node[circle, scale=0.75, fill] (tid0) at (2.25,1.5){};
%           \node[circle, scale=0.75, fill] (tid1) at (1.5,3){};
%           \node[circle, scale=0.75, fill, red] (tid3) at (0.75,4.5){};
%           \node[circle, scale=0.75, fill] (tid4) at (2.25,4.5){};
%           \draw[](tid1) -- (tid3);
%           \draw[](tid1) -- (tid4);
%           \node[circle, scale=0.75, fill] (tid2) at (3.75,3){};
%           \node[circle, scale=0.75, fill] (tid5) at (3.75,4.5){};
%           \node[circle, scale=0.75, fill, red] (tid6) at (3.75,6){};
%           \draw[](tid5) -- (tid6);
%           \draw[](tid2) -- (tid5);
%           \draw[](tid0) -- (tid1);
%           \draw[](tid0) -- (tid2);
%         \end{tikzpicture}
%         \nodepart{two}
%         \footnotesize{4.84375}
%         \nodepart{three}
%         \footnotesize{$50\:25\:25$}
%       };
%       & 
%       \node[draw=black, rectangle split,  rectangle split parts=3] (sn0x1056b00){
%         \begin{tikzpicture}[scale=.2]
%           \node[circle, scale=0.75, fill] (tid0) at (3,1.5){};
%           \node[circle, scale=0.75, fill] (tid1) at (2.25,3){};
%           \node[circle, scale=0.75, fill, red] (tid3) at (0.75,4.5){};
%           \node[circle, scale=0.75, fill, red] (tid4) at (2.25,4.5){};
%           \node[circle, scale=0.75, fill] (tid5) at (3.75,4.5){};
%           \draw[](tid1) -- (tid3);
%           \draw[](tid1) -- (tid4);
%           \draw[](tid1) -- (tid5);
%           \node[circle, scale=0.75, fill] (tid2) at (5.25,3){};
%           \node[circle, scale=0.75, fill] (tid6) at (5.25,4.5){};
%           \draw[](tid2) -- (tid6);
%           \draw[](tid0) -- (tid1);
%           \draw[](tid0) -- (tid2);
%         \end{tikzpicture}
%         \nodepart{two}
%         \footnotesize{4.75}
%         \nodepart{three}
%         \footnotesize{$50\:50$}
%       };
%       & 
%       \node[draw=black, rectangle split,  rectangle split parts=3] (sn0x1056fb0){
%         \begin{tikzpicture}[scale=.2]
%           \node[circle, scale=0.75, fill] (tid0) at (3,1.5){};
%           \node[circle, scale=0.75, fill] (tid1) at (2.25,3){};
%           \node[circle, scale=0.75, fill, red] (tid3) at (0.75,4.5){};
%           \node[circle, scale=0.75, fill] (tid4) at (2.25,4.5){};
%           \node[circle, scale=0.75, fill] (tid5) at (3.75,4.5){};
%           \draw[](tid1) -- (tid3);
%           \draw[](tid1) -- (tid4);
%           \draw[](tid1) -- (tid5);
%           \node[circle, scale=0.75, fill] (tid2) at (5.25,3){};
%           \node[circle, scale=0.75, fill, red] (tid6) at (5.25,4.5){};
%           \draw[](tid2) -- (tid6);
%           \draw[](tid0) -- (tid1);
%           \draw[](tid0) -- (tid2);
%         \end{tikzpicture}
%         \nodepart{two}
%         \footnotesize{4.75}
%         \nodepart{three}
%         \footnotesize{$50\:50$}
%       };
%       & 
%       \node[draw=black, rectangle split,  rectangle split parts=3] (sn0x1058f50){
%         \begin{tikzpicture}[scale=.2]
%           \node[circle, scale=0.75, fill] (tid0) at (2.25,1.5){};
%           \node[circle, scale=0.75, fill] (tid1) at (1.5,3){};
%           \node[circle, scale=0.75, fill] (tid3) at (0.75,4.5){};
%           \node[circle, scale=0.75, fill, red] (tid6) at (0.75,6){};
%           \draw[](tid3) -- (tid6);
%           \node[circle, scale=0.75, fill, red] (tid4) at (2.25,4.5){};
%           \draw[](tid1) -- (tid3);
%           \draw[](tid1) -- (tid4);
%           \node[circle, scale=0.75, fill] (tid2) at (3.75,3){};
%           \node[circle, scale=0.75, fill] (tid5) at (3.75,4.5){};
%           \draw[](tid2) -- (tid5);
%           \draw[](tid0) -- (tid1);
%           \draw[](tid0) -- (tid2);
%         \end{tikzpicture}
%         \nodepart{two}
%         \footnotesize{4.84375}
%         \nodepart{three}
%         \footnotesize{$50\:25\:25$}
%       };
%       & 
%       \node[draw=black, rectangle split,  rectangle split parts=3] (sn0x1058a50){
%         \begin{tikzpicture}[scale=.2]
%           \node[circle, scale=0.75, fill] (tid0) at (2.25,1.5){};
%           \node[circle, scale=0.75, fill] (tid1) at (1.5,3){};
%           \node[circle, scale=0.75, fill] (tid3) at (0.75,4.5){};
%           \node[circle, scale=0.75, fill, red] (tid6) at (0.75,6){};
%           \draw[](tid3) -- (tid6);
%           \node[circle, scale=0.75, fill] (tid4) at (2.25,4.5){};
%           \draw[](tid1) -- (tid3);
%           \draw[](tid1) -- (tid4);
%           \node[circle, scale=0.75, fill] (tid2) at (3.75,3){};
%           \node[circle, scale=0.75, fill, red] (tid5) at (3.75,4.5){};
%           \draw[](tid2) -- (tid5);
%           \draw[](tid0) -- (tid1);
%           \draw[](tid0) -- (tid2);
%         \end{tikzpicture}
%         \nodepart{two}
%         \footnotesize{4.84375}
%         \nodepart{three}
%         \footnotesize{$50\:50$}
%       };
%       & 
%       \node[draw=black, rectangle split,  rectangle split parts=3] (sn0x10597a0){
%         \begin{tikzpicture}[scale=.2]
%           \node[circle, scale=0.75, fill] (tid0) at (3,1.5){};
%           \node[circle, scale=0.75, fill] (tid1) at (2.25,3){};
%           \node[circle, scale=0.75, fill] (tid3) at (0.75,4.5){};
%           \node[circle, scale=0.75, fill, red] (tid6) at (0.75,6){};
%           \draw[](tid3) -- (tid6);
%           \node[circle, scale=0.75, fill, red] (tid4) at (2.25,4.5){};
%           \node[circle, scale=0.75, fill] (tid5) at (3.75,4.5){};
%           \draw[](tid1) -- (tid3);
%           \draw[](tid1) -- (tid4);
%           \draw[](tid1) -- (tid5);
%           \node[circle, scale=0.75, fill] (tid2) at (5.25,3){};
%           \draw[](tid0) -- (tid1);
%           \draw[](tid0) -- (tid2);
%         \end{tikzpicture}
%         \nodepart{two}
%         \footnotesize{4.84375}
%         \nodepart{three}
%         \footnotesize{$50\:50$}
%       };
%       & 
%       \\
%     };
%   \end{scope}
%   \begin{scope}[yshift=\leveltopIIIIII cm]
%     \matrix (line6) [column sep=1cm] {
%       \node[draw=black, rectangle split,  rectangle split parts=3] (sn0x1053920){
%         \begin{tikzpicture}[scale=.2]
%           \node[circle, scale=0.75, fill] (tid0) at (1.5,1.5){};
%           \node[circle, scale=0.75, fill] (tid1) at (0.75,3){};
%           \node[circle, scale=0.75, fill] (tid3) at (0.75,4.5){};
%           \node[circle, scale=0.75, fill] (tid4) at (0.75,6){};
%           \node[circle, scale=0.75, fill, red] (tid5) at (0.75,7.5){};
%           \draw[](tid4) -- (tid5);
%           \draw[](tid3) -- (tid4);
%           \draw[](tid1) -- (tid3);
%           \node[circle, scale=0.75, fill, red] (tid2) at (2.25,3){};
%           \draw[](tid0) -- (tid1);
%           \draw[](tid0) -- (tid2);
%         \end{tikzpicture}
%         \nodepart{two}
%         \footnotesize{5.0625}
%         \nodepart{three}
%         \footnotesize{$50\:50$}
%       };
%       & 
%       \node[draw=black, rectangle split,  rectangle split parts=3] (sn0x1053bc0){
%         \begin{tikzpicture}[scale=.2]
%           \node[circle, scale=0.75, fill] (tid0) at (1.5,1.5){};
%           \node[circle, scale=0.75, fill] (tid1) at (0.75,3){};
%           \node[circle, scale=0.75, fill] (tid3) at (0.75,4.5){};
%           \node[circle, scale=0.75, fill, red] (tid5) at (0.75,6){};
%           \draw[](tid3) -- (tid5);
%           \draw[](tid1) -- (tid3);
%           \node[circle, scale=0.75, fill] (tid2) at (2.25,3){};
%           \node[circle, scale=0.75, fill, red] (tid4) at (2.25,4.5){};
%           \draw[](tid2) -- (tid4);
%           \draw[](tid0) -- (tid1);
%           \draw[](tid0) -- (tid2);
%         \end{tikzpicture}
%         \nodepart{two}
%         \footnotesize{4.4375}
%         \nodepart{three}
%         \footnotesize{$50\:50$}
%       };
%       & 
%       \node[draw=black, rectangle split,  rectangle split parts=3] (sn0x1056090){
%         \begin{tikzpicture}[scale=.2]
%           \node[circle, scale=0.75, fill] (tid0) at (2.25,1.5){};
%           \node[circle, scale=0.75, fill] (tid1) at (1.5,3){};
%           \node[circle, scale=0.75, fill, red] (tid3) at (0.75,4.5){};
%           \node[circle, scale=0.75, fill, red] (tid4) at (2.25,4.5){};
%           \draw[](tid1) -- (tid3);
%           \draw[](tid1) -- (tid4);
%           \node[circle, scale=0.75, fill] (tid2) at (3.75,3){};
%           \node[circle, scale=0.75, fill] (tid5) at (3.75,4.5){};
%           \draw[](tid2) -- (tid5);
%           \draw[](tid0) -- (tid1);
%           \draw[](tid0) -- (tid2);
%         \end{tikzpicture}
%         \nodepart{two}
%         \footnotesize{4.25}
%         \nodepart{three}
%         \footnotesize{$1$}
%       };
%       & 
%       \node[draw=black, rectangle split,  rectangle split parts=3] (sn0x1056160){
%         \begin{tikzpicture}[scale=.2]
%           \node[circle, scale=0.75, fill] (tid0) at (2.25,1.5){};
%           \node[circle, scale=0.75, fill] (tid1) at (1.5,3){};
%           \node[circle, scale=0.75, fill, red] (tid3) at (0.75,4.5){};
%           \node[circle, scale=0.75, fill] (tid4) at (2.25,4.5){};
%           \draw[](tid1) -- (tid3);
%           \draw[](tid1) -- (tid4);
%           \node[circle, scale=0.75, fill] (tid2) at (3.75,3){};
%           \node[circle, scale=0.75, fill, red] (tid5) at (3.75,4.5){};
%           \draw[](tid2) -- (tid5);
%           \draw[](tid0) -- (tid1);
%           \draw[](tid0) -- (tid2);
%         \end{tikzpicture}
%         \nodepart{two}
%         \footnotesize{4.25}
%         \nodepart{three}
%         \footnotesize{$50\:50$}
%       };
%       & 
%       \node[draw=black, rectangle split,  rectangle split parts=3] (sn0x1057630){
%         \begin{tikzpicture}[scale=.2]
%           \node[circle, scale=0.75, fill] (tid0) at (3,1.5){};
%           \node[circle, scale=0.75, fill] (tid1) at (2.25,3){};
%           \node[circle, scale=0.75, fill, red] (tid3) at (0.75,4.5){};
%           \node[circle, scale=0.75, fill, red] (tid4) at (2.25,4.5){};
%           \node[circle, scale=0.75, fill] (tid5) at (3.75,4.5){};
%           \draw[](tid1) -- (tid3);
%           \draw[](tid1) -- (tid4);
%           \draw[](tid1) -- (tid5);
%           \node[circle, scale=0.75, fill] (tid2) at (5.25,3){};
%           \draw[](tid0) -- (tid1);
%           \draw[](tid0) -- (tid2);
%         \end{tikzpicture}
%         \nodepart{two}
%         \footnotesize{4.25}
%         \nodepart{three}
%         \footnotesize{$1$}
%       };
%       & 
%       \node[draw=black, rectangle split,  rectangle split parts=3] (sn0x1058b20){
%         \begin{tikzpicture}[scale=.2]
%           \node[circle, scale=0.75, fill] (tid0) at (2.25,1.5){};
%           \node[circle, scale=0.75, fill] (tid1) at (1.5,3){};
%           \node[circle, scale=0.75, fill] (tid3) at (0.75,4.5){};
%           \node[circle, scale=0.75, fill, red] (tid5) at (0.75,6){};
%           \draw[](tid3) -- (tid5);
%           \node[circle, scale=0.75, fill, red] (tid4) at (2.25,4.5){};
%           \draw[](tid1) -- (tid3);
%           \draw[](tid1) -- (tid4);
%           \node[circle, scale=0.75, fill] (tid2) at (3.75,3){};
%           \draw[](tid0) -- (tid1);
%           \draw[](tid0) -- (tid2);
%         \end{tikzpicture}
%         \nodepart{two}
%         \footnotesize{4.4375}
%         \nodepart{three}
%         \footnotesize{$50\:50$}
%       };
%       & 
%       \\
%     };
%   \end{scope}
%   \begin{scope}[yshift=\leveltopIIIIIII cm]
%     \matrix (line7) [column sep=1cm] {
%       \node[draw=black, rectangle split,  rectangle split parts=3] (sn0x10540d0){
%         \begin{tikzpicture}[scale=.2]
%           \node[circle, scale=0.75, fill] (tid0) at (0.75,1.5){};
%           \node[circle, scale=0.75, fill] (tid1) at (0.75,3){};
%           \node[circle, scale=0.75, fill] (tid2) at (0.75,4.5){};
%           \node[circle, scale=0.75, fill] (tid3) at (0.75,6){};
%           \node[circle, scale=0.75, fill, red] (tid4) at (0.75,7.5){};
%           \draw[](tid3) -- (tid4);
%           \draw[](tid2) -- (tid3);
%           \draw[](tid1) -- (tid2);
%           \draw[](tid0) -- (tid1);
%         \end{tikzpicture}
%         \nodepart{two}
%         \footnotesize{5}
%         \nodepart{three}
%         \footnotesize{$1$}
%       };
%       & 
%       \node[draw=black, rectangle split,  rectangle split parts=3] (sn0x1054480){
%         \begin{tikzpicture}[scale=.2]
%           \node[circle, scale=0.75, fill] (tid0) at (1.5,1.5){};
%           \node[circle, scale=0.75, fill] (tid1) at (0.75,3){};
%           \node[circle, scale=0.75, fill] (tid3) at (0.75,4.5){};
%           \node[circle, scale=0.75, fill, red] (tid4) at (0.75,6){};
%           \draw[](tid3) -- (tid4);
%           \draw[](tid1) -- (tid3);
%           \node[circle, scale=0.75, fill, red] (tid2) at (2.25,3){};
%           \draw[](tid0) -- (tid1);
%           \draw[](tid0) -- (tid2);
%         \end{tikzpicture}
%         \nodepart{two}
%         \footnotesize{4.125}
%         \nodepart{three}
%         \footnotesize{$50\:50$}
%       };
%       & 
%       \node[draw=black, rectangle split,  rectangle split parts=3] (sn0x1055dd0){
%         \begin{tikzpicture}[scale=.2]
%           \node[circle, scale=0.75, fill] (tid0) at (1.5,1.5){};
%           \node[circle, scale=0.75, fill] (tid1) at (0.75,3){};
%           \node[circle, scale=0.75, fill, red] (tid3) at (0.75,4.5){};
%           \draw[](tid1) -- (tid3);
%           \node[circle, scale=0.75, fill] (tid2) at (2.25,3){};
%           \node[circle, scale=0.75, fill, red] (tid4) at (2.25,4.5){};
%           \draw[](tid2) -- (tid4);
%           \draw[](tid0) -- (tid1);
%           \draw[](tid0) -- (tid2);
%         \end{tikzpicture}
%         \nodepart{two}
%         \footnotesize{3.75}
%         \nodepart{three}
%         \footnotesize{$1$}
%       };
%       & 
%       \node[draw=black, rectangle split,  rectangle split parts=3] (sn0x10568c0){
%         \begin{tikzpicture}[scale=.2]
%           \node[circle, scale=0.75, fill] (tid0) at (2.25,1.5){};
%           \node[circle, scale=0.75, fill] (tid1) at (1.5,3){};
%           \node[circle, scale=0.75, fill, red] (tid3) at (0.75,4.5){};
%           \node[circle, scale=0.75, fill, red] (tid4) at (2.25,4.5){};
%           \draw[](tid1) -- (tid3);
%           \draw[](tid1) -- (tid4);
%           \node[circle, scale=0.75, fill] (tid2) at (3.75,3){};
%           \draw[](tid0) -- (tid1);
%           \draw[](tid0) -- (tid2);
%         \end{tikzpicture}
%         \nodepart{two}
%         \footnotesize{3.75}
%         \nodepart{three}
%         \footnotesize{$1$}
%       };
%       & 
%       \\
%     };
%   \end{scope}
%   \begin{scope}[yshift=\leveltopIIIIIIII cm]
%     \matrix (line8) [column sep=1cm] {
%       \node[draw=black, rectangle split,  rectangle split parts=3] (sn0x1054550){
%         \begin{tikzpicture}[scale=.2]
%           \node[circle, scale=0.75, fill] (tid0) at (0.75,1.5){};
%           \node[circle, scale=0.75, fill] (tid1) at (0.75,3){};
%           \node[circle, scale=0.75, fill] (tid2) at (0.75,4.5){};
%           \node[circle, scale=0.75, fill, red] (tid3) at (0.75,6){};
%           \draw[](tid2) -- (tid3);
%           \draw[](tid1) -- (tid2);
%           \draw[](tid0) -- (tid1);
%         \end{tikzpicture}
%         \nodepart{two}
%         \footnotesize{4}
%         \nodepart{three}
%         \footnotesize{$1$}
%       };
%       & 
%       \node[draw=black, rectangle split,  rectangle split parts=3] (sn0x1055270){
%         \begin{tikzpicture}[scale=.2]
%           \node[circle, scale=0.75, fill] (tid0) at (1.5,1.5){};
%           \node[circle, scale=0.75, fill] (tid1) at (0.75,3){};
%           \node[circle, scale=0.75, fill, red] (tid3) at (0.75,4.5){};
%           \draw[](tid1) -- (tid3);
%           \node[circle, scale=0.75, fill, red] (tid2) at (2.25,3){};
%           \draw[](tid0) -- (tid1);
%           \draw[](tid0) -- (tid2);
%         \end{tikzpicture}
%         \nodepart{two}
%         \footnotesize{3.25}
%         \nodepart{three}
%         \footnotesize{$50\:50$}
%       };
%       & 
%       \\
%     };
%   \end{scope}
%   \begin{scope}[yshift=\leveltopIIIIIIIII cm]
%     \matrix (line9) [column sep=1cm] {
%       \node[draw=black, rectangle split,  rectangle split parts=3] (sn0x1054a50){
%         \begin{tikzpicture}[scale=.2]
%           \node[circle, scale=0.75, fill] (tid0) at (0.75,1.5){};
%           \node[circle, scale=0.75, fill] (tid1) at (0.75,3){};
%           \node[circle, scale=0.75, fill, red] (tid2) at (0.75,4.5){};
%           \draw[](tid1) -- (tid2);
%           \draw[](tid0) -- (tid1);
%         \end{tikzpicture}
%         \nodepart{two}
%         \footnotesize{3}
%         \nodepart{three}
%         \footnotesize{$1$}
%       };
%       & 
%       \node[draw=black, rectangle split,  rectangle split parts=3] (sn0x1054cb0){
%         \begin{tikzpicture}[scale=.2]
%           \node[circle, scale=0.75, fill] (tid0) at (1.5,1.5){};
%           \node[circle, scale=0.75, fill, red] (tid1) at (0.75,3){};
%           \node[circle, scale=0.75, fill, red] (tid2) at (2.25,3){};
%           \draw[](tid0) -- (tid1);
%           \draw[](tid0) -- (tid2);
%         \end{tikzpicture}
%         \nodepart{two}
%         \footnotesize{2.5}
%         \nodepart{three}
%         \footnotesize{$1$}
%       };
%       & 
%       \\
%     };
%   \end{scope}
%   \begin{scope}[yshift=\leveltopIIIIIIIIII cm]
%     \matrix (line10) [column sep=1cm] {
%       \node[draw=black, rectangle split,  rectangle split parts=3] (sn0x1054b20){
%         \begin{tikzpicture}[scale=.2]
%           \node[circle, scale=0.75, fill] (tid0) at (0.75,1.5){};
%           \node[circle, scale=0.75, fill, red] (tid1) at (0.75,3){};
%           \draw[](tid0) -- (tid1);
%         \end{tikzpicture}
%         \nodepart{two}
%         \footnotesize{2}
%         \nodepart{three}
%         \footnotesize{$1$}
%       };
%       & 
%       \\
%     };
%   \end{scope}
%   \begin{scope}[yshift=\leveltopIIIIIIIIIII cm]
%     \matrix (line11) [column sep=1cm] {
%       \node[draw=black, rectangle split,  rectangle split parts=3] (sn0x10547e0){
%         \begin{tikzpicture}[scale=.2]
%           \node[circle, scale=0.75, fill, red] (tid0) at (0.75,1.5){};
%         \end{tikzpicture}
%         \nodepart{two}
%         \footnotesize{1}
%         \nodepart{three}
%         \footnotesize{$$}
%       };
%       & 
%       \\
%     };
%   \end{scope}
%   \begin{scope}[yshift=\leveltopIIIIIIIIIIII cm]
%     \matrix (line12) [column sep=1cm] {
%       \\
%     };
%   \end{scope}
%   \draw (sn0x1050af0.south) -- (sn0x105a150.north);
%   \draw (sn0x1050af0.south) -- (sn0x104cb60.north);
%   \draw (sn0x105a150.south) -- (sn0x10519d0.north);
%   \draw (sn0x105a150.south) -- (sn0x105a080.north);
%   \draw (sn0x104cb60.south) -- (sn0x105a080.north);
%   \draw (sn0x104cb60.south) -- (sn0x104fbd0.north);
%   \draw (sn0x10519d0.south) -- (sn0x1052250.north);
%   \draw (sn0x10519d0.south) -- (sn0x1052960.north);
%   \draw (sn0x105a080.south) -- (sn0x1052960.north);
%   \draw (sn0x104fbd0.south) -- (sn0x1052960.north);
%   \draw (sn0x104fbd0.south) -- (sn0x10581e0.north);
%   \draw (sn0x104fbd0.south) -- (sn0x1058550.north);
%   \draw (sn0x1052250.south) -- (sn0x10525f0.north);
%   \draw (sn0x1052250.south) -- (sn0x1053850.north);
%   \draw (sn0x1052960.south) -- (sn0x1053850.north);
%   \draw (sn0x1052960.south) -- (sn0x1056b00.north);
%   \draw (sn0x1052960.south) -- (sn0x1056fb0.north);
%   \draw (sn0x10581e0.south) -- (sn0x1058f50.north);
%   \draw (sn0x10581e0.south) -- (sn0x1058a50.north);
%   \draw (sn0x10581e0.south) -- (sn0x1056b00.north);
%   \draw (sn0x10581e0.south) -- (sn0x1056fb0.north);
%   \draw (sn0x1058550.south) -- (sn0x10597a0.north);
%   \draw (sn0x1058550.south) -- (sn0x1056fb0.north);
%   \draw (sn0x10525f0.south) -- (sn0x1053920.north);
%   \draw (sn0x10525f0.south) -- (sn0x1053bc0.north);
%   \draw (sn0x1053850.south) -- (sn0x1053bc0.north);
%   \draw (sn0x1053850.south) -- (sn0x1056090.north);
%   \draw (sn0x1053850.south) -- (sn0x1056160.north);
%   \draw (sn0x1056b00.south) -- (sn0x1056090.north);
%   \draw (sn0x1056b00.south) -- (sn0x1056160.north);
%   \draw (sn0x1056fb0.south) -- (sn0x1056160.north);
%   \draw (sn0x1056fb0.south) -- (sn0x1057630.north);
%   \draw (sn0x1058f50.south) -- (sn0x1053bc0.north);
%   \draw (sn0x1058f50.south) -- (sn0x1056090.north);
%   \draw (sn0x1058f50.south) -- (sn0x1056160.north);
%   \draw (sn0x1058a50.south) -- (sn0x1058b20.north);
%   \draw (sn0x1058a50.south) -- (sn0x1056160.north);
%   \draw (sn0x10597a0.south) -- (sn0x1058b20.north);
%   \draw (sn0x10597a0.south) -- (sn0x1057630.north);
%   \draw (sn0x1053920.south) -- (sn0x10540d0.north);
%   \draw (sn0x1053920.south) -- (sn0x1054480.north);
%   \draw (sn0x1053bc0.south) -- (sn0x1054480.north);
%   \draw (sn0x1053bc0.south) -- (sn0x1055dd0.north);
%   \draw (sn0x1056090.south) -- (sn0x1055dd0.north);
%   \draw (sn0x1056160.south) -- (sn0x1055dd0.north);
%   \draw (sn0x1056160.south) -- (sn0x10568c0.north);
%   \draw (sn0x1057630.south) -- (sn0x10568c0.north);
%   \draw (sn0x1058b20.south) -- (sn0x1054480.north);
%   \draw (sn0x1058b20.south) -- (sn0x10568c0.north);
%   \draw (sn0x10540d0.south) -- (sn0x1054550.north);
%   \draw (sn0x1054480.south) -- (sn0x1054550.north);
%   \draw (sn0x1054480.south) -- (sn0x1055270.north);
%   \draw (sn0x1055dd0.south) -- (sn0x1055270.north);
%   \draw (sn0x10568c0.south) -- (sn0x1055270.north);
%   \draw (sn0x1054550.south) -- (sn0x1054a50.north);
%   \draw (sn0x1055270.south) -- (sn0x1054a50.north);
%   \draw (sn0x1055270.south) -- (sn0x1054cb0.north);
%   \draw (sn0x1054a50.south) -- (sn0x1054b20.north);
%   \draw (sn0x1054cb0.south) -- (sn0x1054b20.north);
%   \draw (sn0x1054b20.south) -- (sn0x10547e0.north);
% \end{tikzpicture}

%%% Local Variables:
%%% TeX-master: "thesis/thesis.tex"
%%% End: 


\hrule



\section{Several proofs that HLF is not optimal for P3}
\frame{
  \begin{minipage}{.25\textwidth}
    \subsection{Different runs of P3-HLF yield diffe\-rent results on 001112}
    \centering{}
    \renewcommand{\leveltopI}{-15cm + \leveltop}
\renewcommand{\leveltopII}{-15cm + \leveltopI}
\renewcommand{\leveltopIII}{-15cm + \leveltopII}
\renewcommand{\leveltopIIII}{-15cm + \leveltopIII}
\renewcommand{\leveltopIIIII}{-15cm + \leveltopIIII}
\renewcommand{\leveltopIIIIII}{-15cm + \leveltopIIIII}
\renewcommand{\leveltopIIIIIII}{-15cm + \leveltopIIIIII}
\begin{tikzpicture}[scale=.2, anchor=south]
\begin{scope}[yshift=\leveltopI cm]
\matrix (line1)[column sep=0.5cm] {
\node[draw=black, rectangle split,  rectangle split parts=4] (sn0x8968ef8){
\footnotesize{100}
\nodepart{two}
\begin{tikzpicture}[scale=.2]
\node[circle, scale=0.75, fill] (tid0) at (3,1.5){};
\node[circle, scale=0.75, fill] (tid1) at (2.25,3){};
\node[circle, scale=0.75, fill, red] (tid3) at (0.75,4.5){};
\node[circle, scale=0.75, fill, red] (tid4) at (2.25,4.5){};
\node[circle, scale=0.75, fill, red] (tid5) at (3.75,4.5){};
\draw[](tid1) -- (tid3);
\draw[](tid1) -- (tid4);
\draw[](tid1) -- (tid5);
\node[circle, scale=0.75, fill] (tid2) at (5.25,3){};
\node[circle, scale=0.75, fill] (tid6) at (5.25,4.5){};
\draw[](tid2) -- (tid6);
\draw[](tid0) -- (tid1);
\draw[](tid0) -- (tid2);
\end{tikzpicture}
\nodepart{three}
\footnotesize{4.38889}
\nodepart{four}
\footnotesize{$1$}
};
 & 
\\
};
\end{scope}
\begin{scope}[yshift=\leveltopII cm]
\matrix (line2)[column sep=0.5cm] {
\node[draw=black, rectangle split,  rectangle split parts=4] (sn0x8969f98){
\footnotesize{100}
\nodepart{two}
\begin{tikzpicture}[scale=.2]
\node[circle, scale=0.75, fill] (tid0) at (2.25,1.5){};
\node[circle, scale=0.75, fill] (tid1) at (1.5,3){};
\node[circle, scale=0.75, fill, red] (tid3) at (0.75,4.5){};
\node[circle, scale=0.75, fill, red] (tid4) at (2.25,4.5){};
\draw[](tid1) -- (tid3);
\draw[](tid1) -- (tid4);
\node[circle, scale=0.75, fill] (tid2) at (3.75,3){};
\node[circle, scale=0.75, fill, red] (tid5) at (3.75,4.5){};
\draw[](tid2) -- (tid5);
\draw[](tid0) -- (tid1);
\draw[](tid0) -- (tid2);
\end{tikzpicture}
\nodepart{three}
\footnotesize{4.05556}
\nodepart{four}
\footnotesize{$67\:33$}
};
 & 
\\
};
\end{scope}
\begin{scope}[yshift=\leveltopIII cm]
\matrix (line3)[column sep=0.5cm] {
\node[draw=black, rectangle split,  rectangle split parts=4] (sn0x8969bb0){
\footnotesize{66.6667}
\nodepart{two}
\begin{tikzpicture}[scale=.2]
\node[circle, scale=0.75, fill] (tid0) at (1.5,1.5){};
\node[circle, scale=0.75, fill] (tid1) at (0.75,3){};
\node[circle, scale=0.75, fill, red] (tid3) at (0.75,4.5){};
\draw[](tid1) -- (tid3);
\node[circle, scale=0.75, fill] (tid2) at (2.25,3){};
\node[circle, scale=0.75, fill, red] (tid4) at (2.25,4.5){};
\draw[](tid2) -- (tid4);
\draw[](tid0) -- (tid1);
\draw[](tid0) -- (tid2);
\end{tikzpicture}
\nodepart{three}
\footnotesize{3.75}
\nodepart{four}
\footnotesize{$1$}
};
 & 
\node[draw=black, rectangle split,  rectangle split parts=4] (sn0x896a3b8){
\footnotesize{33.3333}
\nodepart{two}
\begin{tikzpicture}[scale=.2]
\node[circle, scale=0.75, fill] (tid0) at (2.25,1.5){};
\node[circle, scale=0.75, fill] (tid1) at (1.5,3){};
\node[circle, scale=0.75, fill, red] (tid3) at (0.75,4.5){};
\node[circle, scale=0.75, fill, red] (tid4) at (2.25,4.5){};
\draw[](tid1) -- (tid3);
\draw[](tid1) -- (tid4);
\node[circle, scale=0.75, fill, red] (tid2) at (3.75,3){};
\draw[](tid0) -- (tid1);
\draw[](tid0) -- (tid2);
\end{tikzpicture}
\nodepart{three}
\footnotesize{3.66667}
\nodepart{four}
\footnotesize{$67\:33$}
};
 & 
\\
};
\end{scope}
\draw (sn0x8968ef8.south) -- (sn0x8969f98.north);
\draw (sn0x8969f98.south) -- (sn0x8969bb0.north);
\draw (sn0x8969f98.south) -- (sn0x896a3b8.north);
\end{tikzpicture}
\renewcommand{\leveltopI}{-15cm + \leveltop}
\renewcommand{\leveltopII}{-15cm + \leveltopI}
\renewcommand{\leveltopIII}{-15cm + \leveltopII}
\renewcommand{\leveltopIIII}{-15cm + \leveltopIII}
\renewcommand{\leveltopIIIII}{-15cm + \leveltopIIII}
\renewcommand{\leveltopIIIIII}{-15cm + \leveltopIIIII}
\renewcommand{\leveltopIIIIIII}{-15cm + \leveltopIIIIII}
\begin{tikzpicture}[scale=.2, anchor=south]
\begin{scope}[yshift=\leveltopI cm]
\matrix (line1)[column sep=0.5cm] {
\node[draw=black, rectangle split,  rectangle split parts=4] (sn0x8969e68){
\footnotesize{100}
\nodepart{two}
\begin{tikzpicture}[scale=.2]
\node[circle, scale=0.75, fill] (tid0) at (3,1.5){};
\node[circle, scale=0.75, fill] (tid1) at (2.25,3){};
\node[circle, scale=0.75, fill, red] (tid3) at (0.75,4.5){};
\node[circle, scale=0.75, fill, red] (tid4) at (2.25,4.5){};
\node[circle, scale=0.75, fill] (tid5) at (3.75,4.5){};
\draw[](tid1) -- (tid3);
\draw[](tid1) -- (tid4);
\draw[](tid1) -- (tid5);
\node[circle, scale=0.75, fill] (tid2) at (5.25,3){};
\node[circle, scale=0.75, fill, red] (tid6) at (5.25,4.5){};
\draw[](tid2) -- (tid6);
\draw[](tid0) -- (tid1);
\draw[](tid0) -- (tid2);
\end{tikzpicture}
\nodepart{three}
\footnotesize{4.37037}
\nodepart{four}
\footnotesize{$67\:33$}
};
 & 
\\
};
\end{scope}
\begin{scope}[yshift=\leveltopII cm]
\matrix (line2)[column sep=0.5cm] {
\node[draw=black, rectangle split,  rectangle split parts=4] (sn0x8969f98){
\footnotesize{66.6667}
\nodepart{two}
\begin{tikzpicture}[scale=.2]
\node[circle, scale=0.75, fill] (tid0) at (2.25,1.5){};
\node[circle, scale=0.75, fill] (tid1) at (1.5,3){};
\node[circle, scale=0.75, fill, red] (tid3) at (0.75,4.5){};
\node[circle, scale=0.75, fill, red] (tid4) at (2.25,4.5){};
\draw[](tid1) -- (tid3);
\draw[](tid1) -- (tid4);
\node[circle, scale=0.75, fill] (tid2) at (3.75,3){};
\node[circle, scale=0.75, fill, red] (tid5) at (3.75,4.5){};
\draw[](tid2) -- (tid5);
\draw[](tid0) -- (tid1);
\draw[](tid0) -- (tid2);
\end{tikzpicture}
\nodepart{three}
\footnotesize{4.05556}
\nodepart{four}
\footnotesize{$67\:33$}
};
 & 
\node[draw=black, rectangle split,  rectangle split parts=4] (sn0x896a940){
\footnotesize{33.3333}
\nodepart{two}
\begin{tikzpicture}[scale=.2]
\node[circle, scale=0.75, fill] (tid0) at (3,1.5){};
\node[circle, scale=0.75, fill] (tid1) at (2.25,3){};
\node[circle, scale=0.75, fill, red] (tid3) at (0.75,4.5){};
\node[circle, scale=0.75, fill, red] (tid4) at (2.25,4.5){};
\node[circle, scale=0.75, fill, red] (tid5) at (3.75,4.5){};
\draw[](tid1) -- (tid3);
\draw[](tid1) -- (tid4);
\draw[](tid1) -- (tid5);
\node[circle, scale=0.75, fill] (tid2) at (5.25,3){};
\draw[](tid0) -- (tid1);
\draw[](tid0) -- (tid2);
\end{tikzpicture}
\nodepart{three}
\footnotesize{4}
\nodepart{four}
\footnotesize{$1$}
};
 & 
\\
};
\end{scope}
\begin{scope}[yshift=\leveltopIII cm]
\matrix (line3)[column sep=0.5cm] {
\node[draw=black, rectangle split,  rectangle split parts=4] (sn0x8969bb0){
\footnotesize{44.4444}
\nodepart{two}
\begin{tikzpicture}[scale=.2]
\node[circle, scale=0.75, fill] (tid0) at (1.5,1.5){};
\node[circle, scale=0.75, fill] (tid1) at (0.75,3){};
\node[circle, scale=0.75, fill, red] (tid3) at (0.75,4.5){};
\draw[](tid1) -- (tid3);
\node[circle, scale=0.75, fill] (tid2) at (2.25,3){};
\node[circle, scale=0.75, fill, red] (tid4) at (2.25,4.5){};
\draw[](tid2) -- (tid4);
\draw[](tid0) -- (tid1);
\draw[](tid0) -- (tid2);
\end{tikzpicture}
\nodepart{three}
\footnotesize{3.75}
\nodepart{four}
\footnotesize{$1$}
};
 & 
\node[draw=black, rectangle split,  rectangle split parts=4] (sn0x896a3b8){
\footnotesize{55.5556}
\nodepart{two}
\begin{tikzpicture}[scale=.2]
\node[circle, scale=0.75, fill] (tid0) at (2.25,1.5){};
\node[circle, scale=0.75, fill] (tid1) at (1.5,3){};
\node[circle, scale=0.75, fill, red] (tid3) at (0.75,4.5){};
\node[circle, scale=0.75, fill, red] (tid4) at (2.25,4.5){};
\draw[](tid1) -- (tid3);
\draw[](tid1) -- (tid4);
\node[circle, scale=0.75, fill, red] (tid2) at (3.75,3){};
\draw[](tid0) -- (tid1);
\draw[](tid0) -- (tid2);
\end{tikzpicture}
\nodepart{three}
\footnotesize{3.66667}
\nodepart{four}
\footnotesize{$67\:33$}
};
 & 
\\
};
\end{scope}
\draw (sn0x8969e68.south) -- (sn0x8969f98.north);
\draw (sn0x8969e68.south) -- (sn0x896a940.north);
\draw (sn0x8969f98.south) -- (sn0x8969bb0.north);
\draw (sn0x8969f98.south) -- (sn0x896a3b8.north);
\draw (sn0x896a940.south) -- (sn0x896a3b8.north);
\end{tikzpicture}
%%% Local Variables:
%%% TeX-master: "thesis/thesis.tex"
%%% End: 
  \end{minipage}
}
\frame{
  \begin{minipage}{.25\textwidth}
    \subsection{P3-HLF not optimal for 0012346688}
    \renewcommand{\leveltopI}{-15cm + \leveltop}
\renewcommand{\leveltopII}{-15cm + \leveltopI}
\renewcommand{\leveltopIII}{-15cm + \leveltopII}
\renewcommand{\leveltopIIII}{-15cm + \leveltopIII}
\renewcommand{\leveltopIIIII}{-15cm + \leveltopIIII}
\renewcommand{\leveltopIIIIII}{-15cm + \leveltopIIIII}
\renewcommand{\leveltopIIIIIII}{-15cm + \leveltopIIIIII}
\renewcommand{\leveltopIIIIIIII}{-15cm + \leveltopIIIIIII}
\renewcommand{\leveltopIIIIIIIII}{-15cm + \leveltopIIIIIIII}
\renewcommand{\leveltopIIIIIIIIII}{-15cm + \leveltopIIIIIIIII}
\renewcommand{\leveltopIIIIIIIIIII}{-15cm + \leveltopIIIIIIIIII}
\begin{tikzpicture}[scale=.2, anchor=south]
\begin{scope}[yshift=\leveltopI cm]
\matrix (line1) [column sep=1cm] {
\node[draw=black, rectangle split,  rectangle split parts=3] (sn0x14639f0){
\begin{tikzpicture}[scale=.2]
\node[circle, scale=0.75, fill] (tid0) at (3,1.5){};
\node[circle, scale=0.75, fill] (tid1) at (2.25,3){};
\node[circle, scale=0.75, fill] (tid3) at (2.25,4.5){};
\node[circle, scale=0.75, fill] (tid5) at (2.25,6){};
\node[circle, scale=0.75, fill] (tid7) at (1.5,7.5){};
\node[circle, scale=0.75, fill, red] (tid9) at (0.75,9){};
\node[circle, scale=0.75, fill, red] (tid10) at (2.25,9){};
\draw[](tid7) -- (tid9);
\draw[](tid7) -- (tid10);
\node[circle, scale=0.75, fill, red] (tid8) at (3.75,7.5){};
\draw[](tid5) -- (tid7);
\draw[](tid5) -- (tid8);
\draw[](tid3) -- (tid5);
\draw[](tid1) -- (tid3);
\node[circle, scale=0.75, fill] (tid2) at (5.25,3){};
\node[circle, scale=0.75, fill] (tid4) at (5.25,4.5){};
\node[circle, scale=0.75, fill] (tid6) at (5.25,6){};
\draw[](tid4) -- (tid6);
\draw[](tid2) -- (tid4);
\draw[](tid0) -- (tid1);
\draw[](tid0) -- (tid2);
\end{tikzpicture}
\nodepart{two}
\footnotesize{6.96798}
\nodepart{three}
\footnotesize{$33\:67$}
};
 & 
\\
};
\end{scope}
\begin{scope}[yshift=\leveltopII cm]
\matrix (line2) [column sep=1cm] {
\node[draw=black, rectangle split,  rectangle split parts=3] (sn0x1460d30){
\begin{tikzpicture}[scale=.2]
\node[circle, scale=0.75, fill] (tid0) at (2.25,1.5){};
\node[circle, scale=0.75, fill] (tid1) at (1.5,3){};
\node[circle, scale=0.75, fill] (tid3) at (1.5,4.5){};
\node[circle, scale=0.75, fill] (tid5) at (1.5,6){};
\node[circle, scale=0.75, fill] (tid7) at (1.5,7.5){};
\node[circle, scale=0.75, fill, red] (tid8) at (0.75,9){};
\node[circle, scale=0.75, fill, red] (tid9) at (2.25,9){};
\draw[](tid7) -- (tid8);
\draw[](tid7) -- (tid9);
\draw[](tid5) -- (tid7);
\draw[](tid3) -- (tid5);
\draw[](tid1) -- (tid3);
\node[circle, scale=0.75, fill] (tid2) at (3.75,3){};
\node[circle, scale=0.75, fill] (tid4) at (3.75,4.5){};
\node[circle, scale=0.75, fill, red] (tid6) at (3.75,6){};
\draw[](tid4) -- (tid6);
\draw[](tid2) -- (tid4);
\draw[](tid0) -- (tid1);
\draw[](tid0) -- (tid2);
\end{tikzpicture}
\nodepart{two}
\footnotesize{6.77836}
\nodepart{three}
\footnotesize{$33\:67$}
};
 & 
\node[draw=black, rectangle split,  rectangle split parts=3] (sn0x14633d0){
\begin{tikzpicture}[scale=.2]
\node[circle, scale=0.75, fill] (tid0) at (2.25,1.5){};
\node[circle, scale=0.75, fill] (tid1) at (1.5,3){};
\node[circle, scale=0.75, fill] (tid3) at (1.5,4.5){};
\node[circle, scale=0.75, fill] (tid5) at (1.5,6){};
\node[circle, scale=0.75, fill] (tid7) at (0.75,7.5){};
\node[circle, scale=0.75, fill, red] (tid9) at (0.75,9){};
\draw[](tid7) -- (tid9);
\node[circle, scale=0.75, fill, red] (tid8) at (2.25,7.5){};
\draw[](tid5) -- (tid7);
\draw[](tid5) -- (tid8);
\draw[](tid3) -- (tid5);
\draw[](tid1) -- (tid3);
\node[circle, scale=0.75, fill] (tid2) at (3.75,3){};
\node[circle, scale=0.75, fill] (tid4) at (3.75,4.5){};
\node[circle, scale=0.75, fill, red] (tid6) at (3.75,6){};
\draw[](tid4) -- (tid6);
\draw[](tid2) -- (tid4);
\draw[](tid0) -- (tid1);
\draw[](tid0) -- (tid2);
\end{tikzpicture}
\nodepart{two}
\footnotesize{6.56279}
\nodepart{three}
\footnotesize{$33\:33\:33$}
};
 & 
\\
};
\end{scope}
\begin{scope}[yshift=\leveltopIII cm]
\matrix (line3) [column sep=1cm] {
\node[draw=black, rectangle split,  rectangle split parts=3] (sn0x145fad0){
\begin{tikzpicture}[scale=.2]
\node[circle, scale=0.75, fill] (tid0) at (2.25,1.5){};
\node[circle, scale=0.75, fill] (tid1) at (1.5,3){};
\node[circle, scale=0.75, fill] (tid3) at (1.5,4.5){};
\node[circle, scale=0.75, fill] (tid5) at (1.5,6){};
\node[circle, scale=0.75, fill] (tid6) at (1.5,7.5){};
\node[circle, scale=0.75, fill, red] (tid7) at (0.75,9){};
\node[circle, scale=0.75, fill, red] (tid8) at (2.25,9){};
\draw[](tid6) -- (tid7);
\draw[](tid6) -- (tid8);
\draw[](tid5) -- (tid6);
\draw[](tid3) -- (tid5);
\draw[](tid1) -- (tid3);
\node[circle, scale=0.75, fill] (tid2) at (3.75,3){};
\node[circle, scale=0.75, fill, red] (tid4) at (3.75,4.5){};
\draw[](tid2) -- (tid4);
\draw[](tid0) -- (tid1);
\draw[](tid0) -- (tid2);
\end{tikzpicture}
\nodepart{two}
\footnotesize{6.60069}
\nodepart{three}
\footnotesize{$33\:67$}
};
 & 
\node[draw=black, rectangle split,  rectangle split parts=3] (sn0x1460050){
\begin{tikzpicture}[scale=.2]
\node[circle, scale=0.75, fill] (tid0) at (1.5,1.5){};
\node[circle, scale=0.75, fill] (tid1) at (0.75,3){};
\node[circle, scale=0.75, fill] (tid3) at (0.75,4.5){};
\node[circle, scale=0.75, fill] (tid5) at (0.75,6){};
\node[circle, scale=0.75, fill] (tid7) at (0.75,7.5){};
\node[circle, scale=0.75, fill, red] (tid8) at (0.75,9){};
\draw[](tid7) -- (tid8);
\draw[](tid5) -- (tid7);
\draw[](tid3) -- (tid5);
\draw[](tid1) -- (tid3);
\node[circle, scale=0.75, fill] (tid2) at (2.25,3){};
\node[circle, scale=0.75, fill] (tid4) at (2.25,4.5){};
\node[circle, scale=0.75, fill, red] (tid6) at (2.25,6){};
\draw[](tid4) -- (tid6);
\draw[](tid2) -- (tid4);
\draw[](tid0) -- (tid1);
\draw[](tid0) -- (tid2);
\end{tikzpicture}
\nodepart{two}
\footnotesize{6.36719}
\nodepart{three}
\footnotesize{$50\:50$}
};
 & 
\node[draw=black, rectangle split,  rectangle split parts=3] (sn0x1462fe0){
\begin{tikzpicture}[scale=.2]
\node[circle, scale=0.75, fill] (tid0) at (2.25,1.5){};
\node[circle, scale=0.75, fill] (tid1) at (1.5,3){};
\node[circle, scale=0.75, fill] (tid3) at (1.5,4.5){};
\node[circle, scale=0.75, fill] (tid5) at (1.5,6){};
\node[circle, scale=0.75, fill] (tid6) at (0.75,7.5){};
\node[circle, scale=0.75, fill, red] (tid8) at (0.75,9){};
\draw[](tid6) -- (tid8);
\node[circle, scale=0.75, fill, red] (tid7) at (2.25,7.5){};
\draw[](tid5) -- (tid6);
\draw[](tid5) -- (tid7);
\draw[](tid3) -- (tid5);
\draw[](tid1) -- (tid3);
\node[circle, scale=0.75, fill] (tid2) at (3.75,3){};
\node[circle, scale=0.75, fill, red] (tid4) at (3.75,4.5){};
\draw[](tid2) -- (tid4);
\draw[](tid0) -- (tid1);
\draw[](tid0) -- (tid2);
\end{tikzpicture}
\nodepart{two}
\footnotesize{6.36516}
\nodepart{three}
\footnotesize{$33\:33\:33$}
};
 & 
\node[draw=black, rectangle split,  rectangle split parts=3] (sn0x1462bf0){
\begin{tikzpicture}[scale=.2]
\node[circle, scale=0.75, fill] (tid0) at (2.25,1.5){};
\node[circle, scale=0.75, fill] (tid1) at (1.5,3){};
\node[circle, scale=0.75, fill] (tid3) at (1.5,4.5){};
\node[circle, scale=0.75, fill] (tid5) at (1.5,6){};
\node[circle, scale=0.75, fill, red] (tid7) at (0.75,7.5){};
\node[circle, scale=0.75, fill, red] (tid8) at (2.25,7.5){};
\draw[](tid5) -- (tid7);
\draw[](tid5) -- (tid8);
\draw[](tid3) -- (tid5);
\draw[](tid1) -- (tid3);
\node[circle, scale=0.75, fill] (tid2) at (3.75,3){};
\node[circle, scale=0.75, fill] (tid4) at (3.75,4.5){};
\node[circle, scale=0.75, fill, red] (tid6) at (3.75,6){};
\draw[](tid4) -- (tid6);
\draw[](tid2) -- (tid4);
\draw[](tid0) -- (tid1);
\draw[](tid0) -- (tid2);
\end{tikzpicture}
\nodepart{two}
\footnotesize{5.95602}
\nodepart{three}
\footnotesize{$67\:33$}
};
 & 
\\
};
\end{scope}
\begin{scope}[yshift=\leveltopIIII cm]
\matrix (line4) [column sep=1cm] {
\node[draw=black, rectangle split,  rectangle split parts=3] (sn0x145e900){
\begin{tikzpicture}[scale=.2]
\node[circle, scale=0.75, fill] (tid0) at (2.25,1.5){};
\node[circle, scale=0.75, fill] (tid1) at (1.5,3){};
\node[circle, scale=0.75, fill] (tid3) at (1.5,4.5){};
\node[circle, scale=0.75, fill] (tid4) at (1.5,6){};
\node[circle, scale=0.75, fill] (tid5) at (1.5,7.5){};
\node[circle, scale=0.75, fill, red] (tid6) at (0.75,9){};
\node[circle, scale=0.75, fill, red] (tid7) at (2.25,9){};
\draw[](tid5) -- (tid6);
\draw[](tid5) -- (tid7);
\draw[](tid4) -- (tid5);
\draw[](tid3) -- (tid4);
\draw[](tid1) -- (tid3);
\node[circle, scale=0.75, fill, red] (tid2) at (3.75,3){};
\draw[](tid0) -- (tid1);
\draw[](tid0) -- (tid2);
\end{tikzpicture}
\nodepart{two}
\footnotesize{6.52083}
\nodepart{three}
\footnotesize{$33\:67$}
};
 & 
\node[draw=black, rectangle split,  rectangle split parts=3] (sn0x145fa00){
\begin{tikzpicture}[scale=.2]
\node[circle, scale=0.75, fill] (tid0) at (1.5,1.5){};
\node[circle, scale=0.75, fill] (tid1) at (0.75,3){};
\node[circle, scale=0.75, fill] (tid3) at (0.75,4.5){};
\node[circle, scale=0.75, fill] (tid5) at (0.75,6){};
\node[circle, scale=0.75, fill] (tid6) at (0.75,7.5){};
\node[circle, scale=0.75, fill, red] (tid7) at (0.75,9){};
\draw[](tid6) -- (tid7);
\draw[](tid5) -- (tid6);
\draw[](tid3) -- (tid5);
\draw[](tid1) -- (tid3);
\node[circle, scale=0.75, fill] (tid2) at (2.25,3){};
\node[circle, scale=0.75, fill, red] (tid4) at (2.25,4.5){};
\draw[](tid2) -- (tid4);
\draw[](tid0) -- (tid1);
\draw[](tid0) -- (tid2);
\end{tikzpicture}
\nodepart{two}
\footnotesize{6.14062}
\nodepart{three}
\footnotesize{$50\:50$}
};
 & 
\node[draw=black, rectangle split,  rectangle split parts=3] (sn0x145ff80){
\begin{tikzpicture}[scale=.2]
\node[circle, scale=0.75, fill] (tid0) at (1.5,1.5){};
\node[circle, scale=0.75, fill] (tid1) at (0.75,3){};
\node[circle, scale=0.75, fill] (tid3) at (0.75,4.5){};
\node[circle, scale=0.75, fill] (tid5) at (0.75,6){};
\node[circle, scale=0.75, fill, red] (tid7) at (0.75,7.5){};
\draw[](tid5) -- (tid7);
\draw[](tid3) -- (tid5);
\draw[](tid1) -- (tid3);
\node[circle, scale=0.75, fill] (tid2) at (2.25,3){};
\node[circle, scale=0.75, fill] (tid4) at (2.25,4.5){};
\node[circle, scale=0.75, fill, red] (tid6) at (2.25,6){};
\draw[](tid4) -- (tid6);
\draw[](tid2) -- (tid4);
\draw[](tid0) -- (tid1);
\draw[](tid0) -- (tid2);
\end{tikzpicture}
\nodepart{two}
\footnotesize{5.59375}
\nodepart{three}
\footnotesize{$50\:50$}
};
 & 
\node[draw=black, rectangle split,  rectangle split parts=3] (sn0x1462420){
\begin{tikzpicture}[scale=.2]
\node[circle, scale=0.75, fill] (tid0) at (2.25,1.5){};
\node[circle, scale=0.75, fill] (tid1) at (1.5,3){};
\node[circle, scale=0.75, fill] (tid3) at (1.5,4.5){};
\node[circle, scale=0.75, fill] (tid4) at (1.5,6){};
\node[circle, scale=0.75, fill] (tid5) at (0.75,7.5){};
\node[circle, scale=0.75, fill, red] (tid7) at (0.75,9){};
\draw[](tid5) -- (tid7);
\node[circle, scale=0.75, fill, red] (tid6) at (2.25,7.5){};
\draw[](tid4) -- (tid5);
\draw[](tid4) -- (tid6);
\draw[](tid3) -- (tid4);
\draw[](tid1) -- (tid3);
\node[circle, scale=0.75, fill, red] (tid2) at (3.75,3){};
\draw[](tid0) -- (tid1);
\draw[](tid0) -- (tid2);
\end{tikzpicture}
\nodepart{two}
\footnotesize{6.27431}
\nodepart{three}
\footnotesize{$33\:33\:33$}
};
 & 
\node[draw=black, rectangle split,  rectangle split parts=3] (sn0x14624f0){
\begin{tikzpicture}[scale=.2]
\node[circle, scale=0.75, fill] (tid0) at (2.25,1.5){};
\node[circle, scale=0.75, fill] (tid1) at (1.5,3){};
\node[circle, scale=0.75, fill] (tid3) at (1.5,4.5){};
\node[circle, scale=0.75, fill] (tid5) at (1.5,6){};
\node[circle, scale=0.75, fill, red] (tid6) at (0.75,7.5){};
\node[circle, scale=0.75, fill, red] (tid7) at (2.25,7.5){};
\draw[](tid5) -- (tid6);
\draw[](tid5) -- (tid7);
\draw[](tid3) -- (tid5);
\draw[](tid1) -- (tid3);
\node[circle, scale=0.75, fill] (tid2) at (3.75,3){};
\node[circle, scale=0.75, fill, red] (tid4) at (3.75,4.5){};
\draw[](tid2) -- (tid4);
\draw[](tid0) -- (tid1);
\draw[](tid0) -- (tid2);
\end{tikzpicture}
\nodepart{two}
\footnotesize{5.68056}
\nodepart{three}
\footnotesize{$67\:33$}
};
 & 
\\
};
\end{scope}
\begin{scope}[yshift=\leveltopIIIII cm]
\matrix (line5) [column sep=1cm] {
\node[draw=black, rectangle split,  rectangle split parts=3] (sn0x145d480){
\begin{tikzpicture}[scale=.2]
\node[circle, scale=0.75, fill] (tid0) at (1.5,1.5){};
\node[circle, scale=0.75, fill] (tid1) at (1.5,3){};
\node[circle, scale=0.75, fill] (tid2) at (1.5,4.5){};
\node[circle, scale=0.75, fill] (tid3) at (1.5,6){};
\node[circle, scale=0.75, fill] (tid4) at (1.5,7.5){};
\node[circle, scale=0.75, fill, red] (tid5) at (0.75,9){};
\node[circle, scale=0.75, fill, red] (tid6) at (2.25,9){};
\draw[](tid4) -- (tid5);
\draw[](tid4) -- (tid6);
\draw[](tid3) -- (tid4);
\draw[](tid2) -- (tid3);
\draw[](tid1) -- (tid2);
\draw[](tid0) -- (tid1);
\end{tikzpicture}
\nodepart{two}
\footnotesize{6.5}
\nodepart{three}
\footnotesize{$1$}
};
 & 
\node[draw=black, rectangle split,  rectangle split parts=3] (sn0x145e670){
\begin{tikzpicture}[scale=.2]
\node[circle, scale=0.75, fill] (tid0) at (1.5,1.5){};
\node[circle, scale=0.75, fill] (tid1) at (0.75,3){};
\node[circle, scale=0.75, fill] (tid3) at (0.75,4.5){};
\node[circle, scale=0.75, fill] (tid4) at (0.75,6){};
\node[circle, scale=0.75, fill] (tid5) at (0.75,7.5){};
\node[circle, scale=0.75, fill, red] (tid6) at (0.75,9){};
\draw[](tid5) -- (tid6);
\draw[](tid4) -- (tid5);
\draw[](tid3) -- (tid4);
\draw[](tid1) -- (tid3);
\node[circle, scale=0.75, fill, red] (tid2) at (2.25,3){};
\draw[](tid0) -- (tid1);
\draw[](tid0) -- (tid2);
\end{tikzpicture}
\nodepart{two}
\footnotesize{6.03125}
\nodepart{three}
\footnotesize{$50\:50$}
};
 & 
\node[draw=black, rectangle split,  rectangle split parts=3] (sn0x145f770){
\begin{tikzpicture}[scale=.2]
\node[circle, scale=0.75, fill] (tid0) at (1.5,1.5){};
\node[circle, scale=0.75, fill] (tid1) at (0.75,3){};
\node[circle, scale=0.75, fill] (tid3) at (0.75,4.5){};
\node[circle, scale=0.75, fill] (tid5) at (0.75,6){};
\node[circle, scale=0.75, fill, red] (tid6) at (0.75,7.5){};
\draw[](tid5) -- (tid6);
\draw[](tid3) -- (tid5);
\draw[](tid1) -- (tid3);
\node[circle, scale=0.75, fill] (tid2) at (2.25,3){};
\node[circle, scale=0.75, fill, red] (tid4) at (2.25,4.5){};
\draw[](tid2) -- (tid4);
\draw[](tid0) -- (tid1);
\draw[](tid0) -- (tid2);
\end{tikzpicture}
\nodepart{two}
\footnotesize{5.25}
\nodepart{three}
\footnotesize{$50\:50$}
};
 & 
\node[draw=black, rectangle split,  rectangle split parts=3] (sn0x1460b00){
\begin{tikzpicture}[scale=.2]
\node[circle, scale=0.75, fill] (tid0) at (1.5,1.5){};
\node[circle, scale=0.75, fill] (tid1) at (0.75,3){};
\node[circle, scale=0.75, fill] (tid3) at (0.75,4.5){};
\node[circle, scale=0.75, fill, red] (tid5) at (0.75,6){};
\draw[](tid3) -- (tid5);
\draw[](tid1) -- (tid3);
\node[circle, scale=0.75, fill] (tid2) at (2.25,3){};
\node[circle, scale=0.75, fill] (tid4) at (2.25,4.5){};
\node[circle, scale=0.75, fill, red] (tid6) at (2.25,6){};
\draw[](tid4) -- (tid6);
\draw[](tid2) -- (tid4);
\draw[](tid0) -- (tid1);
\draw[](tid0) -- (tid2);
\end{tikzpicture}
\nodepart{two}
\footnotesize{4.9375}
\nodepart{three}
\footnotesize{$1$}
};
 & 
\node[draw=black, rectangle split,  rectangle split parts=3] (sn0x1462350){
\begin{tikzpicture}[scale=.2]
\node[circle, scale=0.75, fill] (tid0) at (1.5,1.5){};
\node[circle, scale=0.75, fill] (tid1) at (1.5,3){};
\node[circle, scale=0.75, fill] (tid2) at (1.5,4.5){};
\node[circle, scale=0.75, fill] (tid3) at (1.5,6){};
\node[circle, scale=0.75, fill] (tid4) at (0.75,7.5){};
\node[circle, scale=0.75, fill, red] (tid6) at (0.75,9){};
\draw[](tid4) -- (tid6);
\node[circle, scale=0.75, fill, red] (tid5) at (2.25,7.5){};
\draw[](tid3) -- (tid4);
\draw[](tid3) -- (tid5);
\draw[](tid2) -- (tid3);
\draw[](tid1) -- (tid2);
\draw[](tid0) -- (tid1);
\end{tikzpicture}
\nodepart{two}
\footnotesize{6.25}
\nodepart{three}
\footnotesize{$50\:50$}
};
 & 
\node[draw=black, rectangle split,  rectangle split parts=3] (sn0x14628e0){
\begin{tikzpicture}[scale=.2]
\node[circle, scale=0.75, fill] (tid0) at (2.25,1.5){};
\node[circle, scale=0.75, fill] (tid1) at (1.5,3){};
\node[circle, scale=0.75, fill] (tid3) at (1.5,4.5){};
\node[circle, scale=0.75, fill] (tid4) at (1.5,6){};
\node[circle, scale=0.75, fill, red] (tid5) at (0.75,7.5){};
\node[circle, scale=0.75, fill, red] (tid6) at (2.25,7.5){};
\draw[](tid4) -- (tid5);
\draw[](tid4) -- (tid6);
\draw[](tid3) -- (tid4);
\draw[](tid1) -- (tid3);
\node[circle, scale=0.75, fill, red] (tid2) at (3.75,3){};
\draw[](tid0) -- (tid1);
\draw[](tid0) -- (tid2);
\end{tikzpicture}
\nodepart{two}
\footnotesize{5.54167}
\nodepart{three}
\footnotesize{$67\:33$}
};
 & 
\\
};
\end{scope}
\begin{scope}[yshift=\leveltopIIIIII cm]
\matrix (line6) [column sep=1cm] {
\node[draw=black, rectangle split,  rectangle split parts=3] (sn0x145d0f0){
\begin{tikzpicture}[scale=.2]
\node[circle, scale=0.75, fill] (tid0) at (0.75,1.5){};
\node[circle, scale=0.75, fill] (tid1) at (0.75,3){};
\node[circle, scale=0.75, fill] (tid2) at (0.75,4.5){};
\node[circle, scale=0.75, fill] (tid3) at (0.75,6){};
\node[circle, scale=0.75, fill] (tid4) at (0.75,7.5){};
\node[circle, scale=0.75, fill, red] (tid5) at (0.75,9){};
\draw[](tid4) -- (tid5);
\draw[](tid3) -- (tid4);
\draw[](tid2) -- (tid3);
\draw[](tid1) -- (tid2);
\draw[](tid0) -- (tid1);
\end{tikzpicture}
\nodepart{two}
\footnotesize{6}
\nodepart{three}
\footnotesize{$1$}
};
 & 
\node[draw=black, rectangle split,  rectangle split parts=3] (sn0x145e380){
\begin{tikzpicture}[scale=.2]
\node[circle, scale=0.75, fill] (tid0) at (1.5,1.5){};
\node[circle, scale=0.75, fill] (tid1) at (0.75,3){};
\node[circle, scale=0.75, fill] (tid3) at (0.75,4.5){};
\node[circle, scale=0.75, fill] (tid4) at (0.75,6){};
\node[circle, scale=0.75, fill, red] (tid5) at (0.75,7.5){};
\draw[](tid4) -- (tid5);
\draw[](tid3) -- (tid4);
\draw[](tid1) -- (tid3);
\node[circle, scale=0.75, fill, red] (tid2) at (2.25,3){};
\draw[](tid0) -- (tid1);
\draw[](tid0) -- (tid2);
\end{tikzpicture}
\nodepart{two}
\footnotesize{5.0625}
\nodepart{three}
\footnotesize{$50\:50$}
};
 & 
\node[draw=black, rectangle split,  rectangle split parts=3] (sn0x145ec40){
\begin{tikzpicture}[scale=.2]
\node[circle, scale=0.75, fill] (tid0) at (1.5,1.5){};
\node[circle, scale=0.75, fill] (tid1) at (0.75,3){};
\node[circle, scale=0.75, fill] (tid3) at (0.75,4.5){};
\node[circle, scale=0.75, fill, red] (tid5) at (0.75,6){};
\draw[](tid3) -- (tid5);
\draw[](tid1) -- (tid3);
\node[circle, scale=0.75, fill] (tid2) at (2.25,3){};
\node[circle, scale=0.75, fill, red] (tid4) at (2.25,4.5){};
\draw[](tid2) -- (tid4);
\draw[](tid0) -- (tid1);
\draw[](tid0) -- (tid2);
\end{tikzpicture}
\nodepart{two}
\footnotesize{4.4375}
\nodepart{three}
\footnotesize{$50\:50$}
};
 & 
\node[draw=black, rectangle split,  rectangle split parts=3] (sn0x1461150){
\begin{tikzpicture}[scale=.2]
\node[circle, scale=0.75, fill] (tid0) at (1.5,1.5){};
\node[circle, scale=0.75, fill] (tid1) at (1.5,3){};
\node[circle, scale=0.75, fill] (tid2) at (1.5,4.5){};
\node[circle, scale=0.75, fill] (tid3) at (1.5,6){};
\node[circle, scale=0.75, fill, red] (tid4) at (0.75,7.5){};
\node[circle, scale=0.75, fill, red] (tid5) at (2.25,7.5){};
\draw[](tid3) -- (tid4);
\draw[](tid3) -- (tid5);
\draw[](tid2) -- (tid3);
\draw[](tid1) -- (tid2);
\draw[](tid0) -- (tid1);
\end{tikzpicture}
\nodepart{two}
\footnotesize{5.5}
\nodepart{three}
\footnotesize{$1$}
};
 & 
\\
};
\end{scope}
\begin{scope}[yshift=\leveltopIIIIIII cm]
\matrix (line7) [column sep=1cm] {
\node[draw=black, rectangle split,  rectangle split parts=3] (sn0x145cdf0){
\begin{tikzpicture}[scale=.2]
\node[circle, scale=0.75, fill] (tid0) at (0.75,1.5){};
\node[circle, scale=0.75, fill] (tid1) at (0.75,3){};
\node[circle, scale=0.75, fill] (tid2) at (0.75,4.5){};
\node[circle, scale=0.75, fill] (tid3) at (0.75,6){};
\node[circle, scale=0.75, fill, red] (tid4) at (0.75,7.5){};
\draw[](tid3) -- (tid4);
\draw[](tid2) -- (tid3);
\draw[](tid1) -- (tid2);
\draw[](tid0) -- (tid1);
\end{tikzpicture}
\nodepart{two}
\footnotesize{5}
\nodepart{three}
\footnotesize{$1$}
};
 & 
\node[draw=black, rectangle split,  rectangle split parts=3] (sn0x145e0d0){
\begin{tikzpicture}[scale=.2]
\node[circle, scale=0.75, fill] (tid0) at (1.5,1.5){};
\node[circle, scale=0.75, fill] (tid1) at (0.75,3){};
\node[circle, scale=0.75, fill] (tid3) at (0.75,4.5){};
\node[circle, scale=0.75, fill, red] (tid4) at (0.75,6){};
\draw[](tid3) -- (tid4);
\draw[](tid1) -- (tid3);
\node[circle, scale=0.75, fill, red] (tid2) at (2.25,3){};
\draw[](tid0) -- (tid1);
\draw[](tid0) -- (tid2);
\end{tikzpicture}
\nodepart{two}
\footnotesize{4.125}
\nodepart{three}
\footnotesize{$50\:50$}
};
 & 
\node[draw=black, rectangle split,  rectangle split parts=3] (sn0x145eb50){
\begin{tikzpicture}[scale=.2]
\node[circle, scale=0.75, fill] (tid0) at (1.5,1.5){};
\node[circle, scale=0.75, fill] (tid1) at (0.75,3){};
\node[circle, scale=0.75, fill, red] (tid3) at (0.75,4.5){};
\draw[](tid1) -- (tid3);
\node[circle, scale=0.75, fill] (tid2) at (2.25,3){};
\node[circle, scale=0.75, fill, red] (tid4) at (2.25,4.5){};
\draw[](tid2) -- (tid4);
\draw[](tid0) -- (tid1);
\draw[](tid0) -- (tid2);
\end{tikzpicture}
\nodepart{two}
\footnotesize{3.75}
\nodepart{three}
\footnotesize{$1$}
};
 & 
\\
};
\end{scope}
\begin{scope}[yshift=\leveltopIIIIIIII cm]
\matrix (line8) [column sep=1cm] {
\node[draw=black, rectangle split,  rectangle split parts=3] (sn0x145cd20){
\begin{tikzpicture}[scale=.2]
\node[circle, scale=0.75, fill] (tid0) at (0.75,1.5){};
\node[circle, scale=0.75, fill] (tid1) at (0.75,3){};
\node[circle, scale=0.75, fill] (tid2) at (0.75,4.5){};
\node[circle, scale=0.75, fill, red] (tid3) at (0.75,6){};
\draw[](tid2) -- (tid3);
\draw[](tid1) -- (tid2);
\draw[](tid0) -- (tid1);
\end{tikzpicture}
\nodepart{two}
\footnotesize{4}
\nodepart{three}
\footnotesize{$1$}
};
 & 
\node[draw=black, rectangle split,  rectangle split parts=3] (sn0x145d820){
\begin{tikzpicture}[scale=.2]
\node[circle, scale=0.75, fill] (tid0) at (1.5,1.5){};
\node[circle, scale=0.75, fill] (tid1) at (0.75,3){};
\node[circle, scale=0.75, fill, red] (tid3) at (0.75,4.5){};
\draw[](tid1) -- (tid3);
\node[circle, scale=0.75, fill, red] (tid2) at (2.25,3){};
\draw[](tid0) -- (tid1);
\draw[](tid0) -- (tid2);
\end{tikzpicture}
\nodepart{two}
\footnotesize{3.25}
\nodepart{three}
\footnotesize{$50\:50$}
};
 & 
\\
};
\end{scope}
\begin{scope}[yshift=\leveltopIIIIIIIII cm]
\matrix (line9) [column sep=1cm] {
\node[draw=black, rectangle split,  rectangle split parts=3] (sn0x145b6b0){
\begin{tikzpicture}[scale=.2]
\node[circle, scale=0.75, fill] (tid0) at (0.75,1.5){};
\node[circle, scale=0.75, fill] (tid1) at (0.75,3){};
\node[circle, scale=0.75, fill, red] (tid2) at (0.75,4.5){};
\draw[](tid1) -- (tid2);
\draw[](tid0) -- (tid1);
\end{tikzpicture}
\nodepart{two}
\footnotesize{3}
\nodepart{three}
\footnotesize{$1$}
};
 & 
\node[draw=black, rectangle split,  rectangle split parts=3] (sn0x145d710){
\begin{tikzpicture}[scale=.2]
\node[circle, scale=0.75, fill] (tid0) at (1.5,1.5){};
\node[circle, scale=0.75, fill, red] (tid1) at (0.75,3){};
\node[circle, scale=0.75, fill, red] (tid2) at (2.25,3){};
\draw[](tid0) -- (tid1);
\draw[](tid0) -- (tid2);
\end{tikzpicture}
\nodepart{two}
\footnotesize{2.5}
\nodepart{three}
\footnotesize{$1$}
};
 & 
\\
};
\end{scope}
\begin{scope}[yshift=\leveltopIIIIIIIIII cm]
\matrix (line10) [column sep=1cm] {
\node[draw=black, rectangle split,  rectangle split parts=3] (sn0x145aea0){
\begin{tikzpicture}[scale=.2]
\node[circle, scale=0.75, fill] (tid0) at (0.75,1.5){};
\node[circle, scale=0.75, fill, red] (tid1) at (0.75,3){};
\draw[](tid0) -- (tid1);
\end{tikzpicture}
\nodepart{two}
\footnotesize{2}
\nodepart{three}
\footnotesize{$1$}
};
 & 
\\
};
\end{scope}
\begin{scope}[yshift=\leveltopIIIIIIIIIII cm]
\matrix (line11) [column sep=1cm] {
\node[draw=black, rectangle split,  rectangle split parts=3] (sn0x145b5e0){
\begin{tikzpicture}[scale=.2]
\node[circle, scale=0.75, fill, red] (tid0) at (0.75,1.5){};
\end{tikzpicture}
\nodepart{two}
\footnotesize{1}
\nodepart{three}
\footnotesize{$$}
};
 & 
\\
};
\end{scope}
\begin{scope}[yshift=\leveltopIIIIIIIIIIII cm]
\matrix (line12) [column sep=1cm] {
\\
};
\end{scope}
\draw (sn0x14639f0.south) -- (sn0x1460d30.north);
\draw (sn0x14639f0.south) -- (sn0x14633d0.north);
\draw (sn0x1460d30.south) -- (sn0x145fad0.north);
\draw (sn0x1460d30.south) -- (sn0x1460050.north);
\draw (sn0x14633d0.south) -- (sn0x1462fe0.north);
\draw (sn0x14633d0.south) -- (sn0x1460050.north);
\draw (sn0x14633d0.south) -- (sn0x1462bf0.north);
\draw (sn0x145fad0.south) -- (sn0x145e900.north);
\draw (sn0x145fad0.south) -- (sn0x145fa00.north);
\draw (sn0x1460050.south) -- (sn0x145fa00.north);
\draw (sn0x1460050.south) -- (sn0x145ff80.north);
\draw (sn0x1462fe0.south) -- (sn0x1462420.north);
\draw (sn0x1462fe0.south) -- (sn0x145fa00.north);
\draw (sn0x1462fe0.south) -- (sn0x14624f0.north);
\draw (sn0x1462bf0.south) -- (sn0x14624f0.north);
\draw (sn0x1462bf0.south) -- (sn0x145ff80.north);
\draw (sn0x145e900.south) -- (sn0x145d480.north);
\draw (sn0x145e900.south) -- (sn0x145e670.north);
\draw (sn0x145fa00.south) -- (sn0x145e670.north);
\draw (sn0x145fa00.south) -- (sn0x145f770.north);
\draw (sn0x145ff80.south) -- (sn0x145f770.north);
\draw (sn0x145ff80.south) -- (sn0x1460b00.north);
\draw (sn0x1462420.south) -- (sn0x1462350.north);
\draw (sn0x1462420.south) -- (sn0x145e670.north);
\draw (sn0x1462420.south) -- (sn0x14628e0.north);
\draw (sn0x14624f0.south) -- (sn0x14628e0.north);
\draw (sn0x14624f0.south) -- (sn0x145f770.north);
\draw (sn0x145d480.south) -- (sn0x145d0f0.north);
\draw (sn0x145e670.south) -- (sn0x145d0f0.north);
\draw (sn0x145e670.south) -- (sn0x145e380.north);
\draw (sn0x145f770.south) -- (sn0x145e380.north);
\draw (sn0x145f770.south) -- (sn0x145ec40.north);
\draw (sn0x1460b00.south) -- (sn0x145ec40.north);
\draw (sn0x1462350.south) -- (sn0x145d0f0.north);
\draw (sn0x1462350.south) -- (sn0x1461150.north);
\draw (sn0x14628e0.south) -- (sn0x1461150.north);
\draw (sn0x14628e0.south) -- (sn0x145e380.north);
\draw (sn0x145d0f0.south) -- (sn0x145cdf0.north);
\draw (sn0x145e380.south) -- (sn0x145cdf0.north);
\draw (sn0x145e380.south) -- (sn0x145e0d0.north);
\draw (sn0x145ec40.south) -- (sn0x145e0d0.north);
\draw (sn0x145ec40.south) -- (sn0x145eb50.north);
\draw (sn0x1461150.south) -- (sn0x145cdf0.north);
\draw (sn0x145cdf0.south) -- (sn0x145cd20.north);
\draw (sn0x145e0d0.south) -- (sn0x145cd20.north);
\draw (sn0x145e0d0.south) -- (sn0x145d820.north);
\draw (sn0x145eb50.south) -- (sn0x145d820.north);
\draw (sn0x145cd20.south) -- (sn0x145b6b0.north);
\draw (sn0x145d820.south) -- (sn0x145b6b0.north);
\draw (sn0x145d820.south) -- (sn0x145d710.north);
\draw (sn0x145b6b0.south) -- (sn0x145aea0.north);
\draw (sn0x145d710.south) -- (sn0x145aea0.north);
\draw (sn0x145aea0.south) -- (sn0x145b5e0.north);
\end{tikzpicture}

%%% Local Variables:
%%% TeX-master: "thesis/thesis.tex"
%%% End: 

    \renewcommand{\leveltopI}{-15cm + \leveltop}
\renewcommand{\leveltopII}{-15cm + \leveltopI}
\renewcommand{\leveltopIII}{-15cm + \leveltopII}
\renewcommand{\leveltopIIII}{-15cm + \leveltopIII}
\renewcommand{\leveltopIIIII}{-15cm + \leveltopIIII}
\renewcommand{\leveltopIIIIII}{-15cm + \leveltopIIIII}
\renewcommand{\leveltopIIIIIII}{-15cm + \leveltopIIIIII}
\renewcommand{\leveltopIIIIIIII}{-15cm + \leveltopIIIIIII}
\renewcommand{\leveltopIIIIIIIII}{-15cm + \leveltopIIIIIIII}
\renewcommand{\leveltopIIIIIIIIII}{-15cm + \leveltopIIIIIIIII}
\renewcommand{\leveltopIIIIIIIIIII}{-15cm + \leveltopIIIIIIIIII}
\begin{tikzpicture}[scale=.2, anchor=south]
\begin{scope}[yshift=\leveltopI cm]
\matrix (line1) [column sep=1cm] {
\node[draw=black, rectangle split,  rectangle split parts=4] (sn0x18908d0){
\footnotesize{1}
\nodepart{two}
\begin{tikzpicture}[scale=.2]
\node[circle, scale=0.75, fill] (tid0) at (3,1.5){};
\node[circle, scale=0.75, fill] (tid1) at (2.25,3){};
\node[circle, scale=0.75, fill] (tid3) at (2.25,4.5){};
\node[circle, scale=0.75, fill] (tid5) at (2.25,6){};
\node[circle, scale=0.75, fill] (tid7) at (1.5,7.5){};
\node[circle, scale=0.75, fill, red] (tid9) at (0.75,9){};
\node[circle, scale=0.75, fill, red] (tid10) at (2.25,9){};
\draw[](tid7) -- (tid9);
\draw[](tid7) -- (tid10);
\node[circle, scale=0.75, fill] (tid8) at (3.75,7.5){};
\draw[](tid5) -- (tid7);
\draw[](tid5) -- (tid8);
\draw[](tid3) -- (tid5);
\draw[](tid1) -- (tid3);
\node[circle, scale=0.75, fill] (tid2) at (5.25,3){};
\node[circle, scale=0.75, fill] (tid4) at (5.25,4.5){};
\node[circle, scale=0.75, fill, red] (tid6) at (5.25,6){};
\draw[](tid4) -- (tid6);
\draw[](tid2) -- (tid4);
\draw[](tid0) -- (tid1);
\draw[](tid0) -- (tid2);
\end{tikzpicture}
\nodepart{three}
\footnotesize{6.96753}
\nodepart{four}
\footnotesize{$33\:67$}
};
 & 
\\
};
\end{scope}
\begin{scope}[yshift=\leveltopII cm]
\matrix (line2) [column sep=1cm] {
\node[draw=black, rectangle split,  rectangle split parts=4] (sn0x188ec30){
\footnotesize{0.333333}
\nodepart{two}
\begin{tikzpicture}[scale=.2]
\node[circle, scale=0.75, fill] (tid0) at (3,1.5){};
\node[circle, scale=0.75, fill] (tid1) at (2.25,3){};
\node[circle, scale=0.75, fill] (tid3) at (2.25,4.5){};
\node[circle, scale=0.75, fill] (tid5) at (2.25,6){};
\node[circle, scale=0.75, fill] (tid6) at (1.5,7.5){};
\node[circle, scale=0.75, fill, red] (tid8) at (0.75,9){};
\node[circle, scale=0.75, fill, red] (tid9) at (2.25,9){};
\draw[](tid6) -- (tid8);
\draw[](tid6) -- (tid9);
\node[circle, scale=0.75, fill, red] (tid7) at (3.75,7.5){};
\draw[](tid5) -- (tid6);
\draw[](tid5) -- (tid7);
\draw[](tid3) -- (tid5);
\draw[](tid1) -- (tid3);
\node[circle, scale=0.75, fill] (tid2) at (5.25,3){};
\node[circle, scale=0.75, fill] (tid4) at (5.25,4.5){};
\draw[](tid2) -- (tid4);
\draw[](tid0) -- (tid1);
\draw[](tid0) -- (tid2);
\end{tikzpicture}
\nodepart{three}
\footnotesize{6.77701}
\nodepart{four}
\footnotesize{$33\:67$}
};
 & 
\node[draw=black, rectangle split,  rectangle split parts=4] (sn0x18906a0){
\footnotesize{0.666667}
\nodepart{two}
\begin{tikzpicture}[scale=.2]
\node[circle, scale=0.75, fill] (tid0) at (2.25,1.5){};
\node[circle, scale=0.75, fill] (tid1) at (1.5,3){};
\node[circle, scale=0.75, fill] (tid3) at (1.5,4.5){};
\node[circle, scale=0.75, fill] (tid5) at (1.5,6){};
\node[circle, scale=0.75, fill] (tid7) at (0.75,7.5){};
\node[circle, scale=0.75, fill, red] (tid9) at (0.75,9){};
\draw[](tid7) -- (tid9);
\node[circle, scale=0.75, fill, red] (tid8) at (2.25,7.5){};
\draw[](tid5) -- (tid7);
\draw[](tid5) -- (tid8);
\draw[](tid3) -- (tid5);
\draw[](tid1) -- (tid3);
\node[circle, scale=0.75, fill] (tid2) at (3.75,3){};
\node[circle, scale=0.75, fill] (tid4) at (3.75,4.5){};
\node[circle, scale=0.75, fill, red] (tid6) at (3.75,6){};
\draw[](tid4) -- (tid6);
\draw[](tid2) -- (tid4);
\draw[](tid0) -- (tid1);
\draw[](tid0) -- (tid2);
\end{tikzpicture}
\nodepart{three}
\footnotesize{6.56279}
\nodepart{four}
\footnotesize{$33\:33\:33$}
};
 & 
\\
};
\end{scope}
\begin{scope}[yshift=\leveltopIII cm]
\matrix (line3) [column sep=1cm] {
\node[draw=black, rectangle split,  rectangle split parts=4] (sn0x188e220){
\footnotesize{0.111111}
\nodepart{two}
\begin{tikzpicture}[scale=.2]
\node[circle, scale=0.75, fill] (tid0) at (2.25,1.5){};
\node[circle, scale=0.75, fill] (tid1) at (1.5,3){};
\node[circle, scale=0.75, fill] (tid3) at (1.5,4.5){};
\node[circle, scale=0.75, fill] (tid5) at (1.5,6){};
\node[circle, scale=0.75, fill] (tid6) at (1.5,7.5){};
\node[circle, scale=0.75, fill, red] (tid7) at (0.75,9){};
\node[circle, scale=0.75, fill, red] (tid8) at (2.25,9){};
\draw[](tid6) -- (tid7);
\draw[](tid6) -- (tid8);
\draw[](tid5) -- (tid6);
\draw[](tid3) -- (tid5);
\draw[](tid1) -- (tid3);
\node[circle, scale=0.75, fill] (tid2) at (3.75,3){};
\node[circle, scale=0.75, fill, red] (tid4) at (3.75,4.5){};
\draw[](tid2) -- (tid4);
\draw[](tid0) -- (tid1);
\draw[](tid0) -- (tid2);
\end{tikzpicture}
\nodepart{three}
\footnotesize{6.60069}
\nodepart{four}
\footnotesize{$33\:67$}
};
 & 
\node[draw=black, rectangle split,  rectangle split parts=4] (sn0x188d8a0){
\footnotesize{0.444444}
\nodepart{two}
\begin{tikzpicture}[scale=.2]
\node[circle, scale=0.75, fill] (tid0) at (2.25,1.5){};
\node[circle, scale=0.75, fill] (tid1) at (1.5,3){};
\node[circle, scale=0.75, fill] (tid3) at (1.5,4.5){};
\node[circle, scale=0.75, fill] (tid5) at (1.5,6){};
\node[circle, scale=0.75, fill] (tid6) at (0.75,7.5){};
\node[circle, scale=0.75, fill, red] (tid8) at (0.75,9){};
\draw[](tid6) -- (tid8);
\node[circle, scale=0.75, fill, red] (tid7) at (2.25,7.5){};
\draw[](tid5) -- (tid6);
\draw[](tid5) -- (tid7);
\draw[](tid3) -- (tid5);
\draw[](tid1) -- (tid3);
\node[circle, scale=0.75, fill] (tid2) at (3.75,3){};
\node[circle, scale=0.75, fill, red] (tid4) at (3.75,4.5){};
\draw[](tid2) -- (tid4);
\draw[](tid0) -- (tid1);
\draw[](tid0) -- (tid2);
\end{tikzpicture}
\nodepart{three}
\footnotesize{6.36516}
\nodepart{four}
\footnotesize{$33\:33\:33$}
};
 & 
\node[draw=black, rectangle split,  rectangle split parts=4] (sn0x188f0f0){
\footnotesize{0.222222}
\nodepart{two}
\begin{tikzpicture}[scale=.2]
\node[circle, scale=0.75, fill] (tid0) at (1.5,1.5){};
\node[circle, scale=0.75, fill] (tid1) at (0.75,3){};
\node[circle, scale=0.75, fill] (tid3) at (0.75,4.5){};
\node[circle, scale=0.75, fill] (tid5) at (0.75,6){};
\node[circle, scale=0.75, fill] (tid7) at (0.75,7.5){};
\node[circle, scale=0.75, fill, red] (tid8) at (0.75,9){};
\draw[](tid7) -- (tid8);
\draw[](tid5) -- (tid7);
\draw[](tid3) -- (tid5);
\draw[](tid1) -- (tid3);
\node[circle, scale=0.75, fill] (tid2) at (2.25,3){};
\node[circle, scale=0.75, fill] (tid4) at (2.25,4.5){};
\node[circle, scale=0.75, fill, red] (tid6) at (2.25,6){};
\draw[](tid4) -- (tid6);
\draw[](tid2) -- (tid4);
\draw[](tid0) -- (tid1);
\draw[](tid0) -- (tid2);
\end{tikzpicture}
\nodepart{three}
\footnotesize{6.36719}
\nodepart{four}
\footnotesize{$50\:50$}
};
 & 
\node[draw=black, rectangle split,  rectangle split parts=4] (sn0x188ff90){
\footnotesize{0.222222}
\nodepart{two}
\begin{tikzpicture}[scale=.2]
\node[circle, scale=0.75, fill] (tid0) at (2.25,1.5){};
\node[circle, scale=0.75, fill] (tid1) at (1.5,3){};
\node[circle, scale=0.75, fill] (tid3) at (1.5,4.5){};
\node[circle, scale=0.75, fill] (tid5) at (1.5,6){};
\node[circle, scale=0.75, fill, red] (tid7) at (0.75,7.5){};
\node[circle, scale=0.75, fill, red] (tid8) at (2.25,7.5){};
\draw[](tid5) -- (tid7);
\draw[](tid5) -- (tid8);
\draw[](tid3) -- (tid5);
\draw[](tid1) -- (tid3);
\node[circle, scale=0.75, fill] (tid2) at (3.75,3){};
\node[circle, scale=0.75, fill] (tid4) at (3.75,4.5){};
\node[circle, scale=0.75, fill, red] (tid6) at (3.75,6){};
\draw[](tid4) -- (tid6);
\draw[](tid2) -- (tid4);
\draw[](tid0) -- (tid1);
\draw[](tid0) -- (tid2);
\end{tikzpicture}
\nodepart{three}
\footnotesize{5.95602}
\nodepart{four}
\footnotesize{$33\:67$}
};
 & 
\\
};
\end{scope}
\begin{scope}[yshift=\leveltopIIII cm]
\matrix (line4) [column sep=1cm] {
\node[draw=black, rectangle split,  rectangle split parts=4] (sn0x188c370){
\footnotesize{0.037037}
\nodepart{two}
\begin{tikzpicture}[scale=.2]
\node[circle, scale=0.75, fill] (tid0) at (2.25,1.5){};
\node[circle, scale=0.75, fill] (tid1) at (1.5,3){};
\node[circle, scale=0.75, fill] (tid3) at (1.5,4.5){};
\node[circle, scale=0.75, fill] (tid4) at (1.5,6){};
\node[circle, scale=0.75, fill] (tid5) at (1.5,7.5){};
\node[circle, scale=0.75, fill, red] (tid6) at (0.75,9){};
\node[circle, scale=0.75, fill, red] (tid7) at (2.25,9){};
\draw[](tid5) -- (tid6);
\draw[](tid5) -- (tid7);
\draw[](tid4) -- (tid5);
\draw[](tid3) -- (tid4);
\draw[](tid1) -- (tid3);
\node[circle, scale=0.75, fill, red] (tid2) at (3.75,3){};
\draw[](tid0) -- (tid1);
\draw[](tid0) -- (tid2);
\end{tikzpicture}
\nodepart{three}
\footnotesize{6.52083}
\nodepart{four}
\footnotesize{$33\:67$}
};
 & 
\node[draw=black, rectangle split,  rectangle split parts=4] (sn0x188d690){
\footnotesize{0.333333}
\nodepart{two}
\begin{tikzpicture}[scale=.2]
\node[circle, scale=0.75, fill] (tid0) at (1.5,1.5){};
\node[circle, scale=0.75, fill] (tid1) at (0.75,3){};
\node[circle, scale=0.75, fill] (tid3) at (0.75,4.5){};
\node[circle, scale=0.75, fill] (tid5) at (0.75,6){};
\node[circle, scale=0.75, fill] (tid6) at (0.75,7.5){};
\node[circle, scale=0.75, fill, red] (tid7) at (0.75,9){};
\draw[](tid6) -- (tid7);
\draw[](tid5) -- (tid6);
\draw[](tid3) -- (tid5);
\draw[](tid1) -- (tid3);
\node[circle, scale=0.75, fill] (tid2) at (2.25,3){};
\node[circle, scale=0.75, fill, red] (tid4) at (2.25,4.5){};
\draw[](tid2) -- (tid4);
\draw[](tid0) -- (tid1);
\draw[](tid0) -- (tid2);
\end{tikzpicture}
\nodepart{three}
\footnotesize{6.14062}
\nodepart{four}
\footnotesize{$50\:50$}
};
 & 
\node[draw=black, rectangle split,  rectangle split parts=4] (sn0x188b180){
\footnotesize{0.148148}
\nodepart{two}
\begin{tikzpicture}[scale=.2]
\node[circle, scale=0.75, fill] (tid0) at (2.25,1.5){};
\node[circle, scale=0.75, fill] (tid1) at (1.5,3){};
\node[circle, scale=0.75, fill] (tid3) at (1.5,4.5){};
\node[circle, scale=0.75, fill] (tid4) at (1.5,6){};
\node[circle, scale=0.75, fill] (tid5) at (0.75,7.5){};
\node[circle, scale=0.75, fill, red] (tid7) at (0.75,9){};
\draw[](tid5) -- (tid7);
\node[circle, scale=0.75, fill, red] (tid6) at (2.25,7.5){};
\draw[](tid4) -- (tid5);
\draw[](tid4) -- (tid6);
\draw[](tid3) -- (tid4);
\draw[](tid1) -- (tid3);
\node[circle, scale=0.75, fill, red] (tid2) at (3.75,3){};
\draw[](tid0) -- (tid1);
\draw[](tid0) -- (tid2);
\end{tikzpicture}
\nodepart{three}
\footnotesize{6.27431}
\nodepart{four}
\footnotesize{$33\:33\:33$}
};
 & 
\node[draw=black, rectangle split,  rectangle split parts=4] (sn0x188de90){
\footnotesize{0.222222}
\nodepart{two}
\begin{tikzpicture}[scale=.2]
\node[circle, scale=0.75, fill] (tid0) at (2.25,1.5){};
\node[circle, scale=0.75, fill] (tid1) at (1.5,3){};
\node[circle, scale=0.75, fill] (tid3) at (1.5,4.5){};
\node[circle, scale=0.75, fill] (tid5) at (1.5,6){};
\node[circle, scale=0.75, fill, red] (tid6) at (0.75,7.5){};
\node[circle, scale=0.75, fill, red] (tid7) at (2.25,7.5){};
\draw[](tid5) -- (tid6);
\draw[](tid5) -- (tid7);
\draw[](tid3) -- (tid5);
\draw[](tid1) -- (tid3);
\node[circle, scale=0.75, fill] (tid2) at (3.75,3){};
\node[circle, scale=0.75, fill, red] (tid4) at (3.75,4.5){};
\draw[](tid2) -- (tid4);
\draw[](tid0) -- (tid1);
\draw[](tid0) -- (tid2);
\end{tikzpicture}
\nodepart{three}
\footnotesize{5.68056}
\nodepart{four}
\footnotesize{$67\:33$}
};
 & 
\node[draw=black, rectangle split,  rectangle split parts=4] (sn0x188f020){
\footnotesize{0.259259}
\nodepart{two}
\begin{tikzpicture}[scale=.2]
\node[circle, scale=0.75, fill] (tid0) at (1.5,1.5){};
\node[circle, scale=0.75, fill] (tid1) at (0.75,3){};
\node[circle, scale=0.75, fill] (tid3) at (0.75,4.5){};
\node[circle, scale=0.75, fill] (tid5) at (0.75,6){};
\node[circle, scale=0.75, fill, red] (tid7) at (0.75,7.5){};
\draw[](tid5) -- (tid7);
\draw[](tid3) -- (tid5);
\draw[](tid1) -- (tid3);
\node[circle, scale=0.75, fill] (tid2) at (2.25,3){};
\node[circle, scale=0.75, fill] (tid4) at (2.25,4.5){};
\node[circle, scale=0.75, fill, red] (tid6) at (2.25,6){};
\draw[](tid4) -- (tid6);
\draw[](tid2) -- (tid4);
\draw[](tid0) -- (tid1);
\draw[](tid0) -- (tid2);
\end{tikzpicture}
\nodepart{three}
\footnotesize{5.59375}
\nodepart{four}
\footnotesize{$50\:50$}
};
 & 
\\
};
\end{scope}
\begin{scope}[yshift=\leveltopIIIII cm]
\matrix (line5) [column sep=1cm] {
\node[draw=black, rectangle split,  rectangle split parts=4] (sn0x1888fa0){
\footnotesize{0.0123457}
\nodepart{two}
\begin{tikzpicture}[scale=.2]
\node[circle, scale=0.75, fill] (tid0) at (1.5,1.5){};
\node[circle, scale=0.75, fill] (tid1) at (1.5,3){};
\node[circle, scale=0.75, fill] (tid2) at (1.5,4.5){};
\node[circle, scale=0.75, fill] (tid3) at (1.5,6){};
\node[circle, scale=0.75, fill] (tid4) at (1.5,7.5){};
\node[circle, scale=0.75, fill, red] (tid5) at (0.75,9){};
\node[circle, scale=0.75, fill, red] (tid6) at (2.25,9){};
\draw[](tid4) -- (tid5);
\draw[](tid4) -- (tid6);
\draw[](tid3) -- (tid4);
\draw[](tid2) -- (tid3);
\draw[](tid1) -- (tid2);
\draw[](tid0) -- (tid1);
\end{tikzpicture}
\nodepart{three}
\footnotesize{6.5}
\nodepart{four}
\footnotesize{$1$}
};
 & 
\node[draw=black, rectangle split,  rectangle split parts=4] (sn0x188ae00){
\footnotesize{0.240741}
\nodepart{two}
\begin{tikzpicture}[scale=.2]
\node[circle, scale=0.75, fill] (tid0) at (1.5,1.5){};
\node[circle, scale=0.75, fill] (tid1) at (0.75,3){};
\node[circle, scale=0.75, fill] (tid3) at (0.75,4.5){};
\node[circle, scale=0.75, fill] (tid4) at (0.75,6){};
\node[circle, scale=0.75, fill] (tid5) at (0.75,7.5){};
\node[circle, scale=0.75, fill, red] (tid6) at (0.75,9){};
\draw[](tid5) -- (tid6);
\draw[](tid4) -- (tid5);
\draw[](tid3) -- (tid4);
\draw[](tid1) -- (tid3);
\node[circle, scale=0.75, fill, red] (tid2) at (2.25,3){};
\draw[](tid0) -- (tid1);
\draw[](tid0) -- (tid2);
\end{tikzpicture}
\nodepart{three}
\footnotesize{6.03125}
\nodepart{four}
\footnotesize{$50\:50$}
};
 & 
\node[draw=black, rectangle split,  rectangle split parts=4] (sn0x188d460){
\footnotesize{0.444444}
\nodepart{two}
\begin{tikzpicture}[scale=.2]
\node[circle, scale=0.75, fill] (tid0) at (1.5,1.5){};
\node[circle, scale=0.75, fill] (tid1) at (0.75,3){};
\node[circle, scale=0.75, fill] (tid3) at (0.75,4.5){};
\node[circle, scale=0.75, fill] (tid5) at (0.75,6){};
\node[circle, scale=0.75, fill, red] (tid6) at (0.75,7.5){};
\draw[](tid5) -- (tid6);
\draw[](tid3) -- (tid5);
\draw[](tid1) -- (tid3);
\node[circle, scale=0.75, fill] (tid2) at (2.25,3){};
\node[circle, scale=0.75, fill, red] (tid4) at (2.25,4.5){};
\draw[](tid2) -- (tid4);
\draw[](tid0) -- (tid1);
\draw[](tid0) -- (tid2);
\end{tikzpicture}
\nodepart{three}
\footnotesize{5.25}
\nodepart{four}
\footnotesize{$50\:50$}
};
 & 
\node[draw=black, rectangle split,  rectangle split parts=4] (sn0x1889a70){
\footnotesize{0.0493827}
\nodepart{two}
\begin{tikzpicture}[scale=.2]
\node[circle, scale=0.75, fill] (tid0) at (1.5,1.5){};
\node[circle, scale=0.75, fill] (tid1) at (1.5,3){};
\node[circle, scale=0.75, fill] (tid2) at (1.5,4.5){};
\node[circle, scale=0.75, fill] (tid3) at (1.5,6){};
\node[circle, scale=0.75, fill] (tid4) at (0.75,7.5){};
\node[circle, scale=0.75, fill, red] (tid6) at (0.75,9){};
\draw[](tid4) -- (tid6);
\node[circle, scale=0.75, fill, red] (tid5) at (2.25,7.5){};
\draw[](tid3) -- (tid4);
\draw[](tid3) -- (tid5);
\draw[](tid2) -- (tid3);
\draw[](tid1) -- (tid2);
\draw[](tid0) -- (tid1);
\end{tikzpicture}
\nodepart{three}
\footnotesize{6.25}
\nodepart{four}
\footnotesize{$50\:50$}
};
 & 
\node[draw=black, rectangle split,  rectangle split parts=4] (sn0x188b630){
\footnotesize{0.123457}
\nodepart{two}
\begin{tikzpicture}[scale=.2]
\node[circle, scale=0.75, fill] (tid0) at (2.25,1.5){};
\node[circle, scale=0.75, fill] (tid1) at (1.5,3){};
\node[circle, scale=0.75, fill] (tid3) at (1.5,4.5){};
\node[circle, scale=0.75, fill] (tid4) at (1.5,6){};
\node[circle, scale=0.75, fill, red] (tid5) at (0.75,7.5){};
\node[circle, scale=0.75, fill, red] (tid6) at (2.25,7.5){};
\draw[](tid4) -- (tid5);
\draw[](tid4) -- (tid6);
\draw[](tid3) -- (tid4);
\draw[](tid1) -- (tid3);
\node[circle, scale=0.75, fill, red] (tid2) at (3.75,3){};
\draw[](tid0) -- (tid1);
\draw[](tid0) -- (tid2);
\end{tikzpicture}
\nodepart{three}
\footnotesize{5.54167}
\nodepart{four}
\footnotesize{$67\:33$}
};
 & 
\node[draw=black, rectangle split,  rectangle split parts=4] (sn0x188fd60){
\footnotesize{0.12963}
\nodepart{two}
\begin{tikzpicture}[scale=.2]
\node[circle, scale=0.75, fill] (tid0) at (1.5,1.5){};
\node[circle, scale=0.75, fill] (tid1) at (0.75,3){};
\node[circle, scale=0.75, fill] (tid3) at (0.75,4.5){};
\node[circle, scale=0.75, fill, red] (tid5) at (0.75,6){};
\draw[](tid3) -- (tid5);
\draw[](tid1) -- (tid3);
\node[circle, scale=0.75, fill] (tid2) at (2.25,3){};
\node[circle, scale=0.75, fill] (tid4) at (2.25,4.5){};
\node[circle, scale=0.75, fill, red] (tid6) at (2.25,6){};
\draw[](tid4) -- (tid6);
\draw[](tid2) -- (tid4);
\draw[](tid0) -- (tid1);
\draw[](tid0) -- (tid2);
\end{tikzpicture}
\nodepart{three}
\footnotesize{4.9375}
\nodepart{four}
\footnotesize{$1$}
};
 & 
\\
};
\end{scope}
\begin{scope}[yshift=\leveltopIIIIII cm]
\matrix (line6) [column sep=1cm] {
\node[draw=black, rectangle split,  rectangle split parts=4] (sn0x1888e50){
\footnotesize{0.157407}
\nodepart{two}
\begin{tikzpicture}[scale=.2]
\node[circle, scale=0.75, fill] (tid0) at (0.75,1.5){};
\node[circle, scale=0.75, fill] (tid1) at (0.75,3){};
\node[circle, scale=0.75, fill] (tid2) at (0.75,4.5){};
\node[circle, scale=0.75, fill] (tid3) at (0.75,6){};
\node[circle, scale=0.75, fill] (tid4) at (0.75,7.5){};
\node[circle, scale=0.75, fill, red] (tid5) at (0.75,9){};
\draw[](tid4) -- (tid5);
\draw[](tid3) -- (tid4);
\draw[](tid2) -- (tid3);
\draw[](tid1) -- (tid2);
\draw[](tid0) -- (tid1);
\end{tikzpicture}
\nodepart{three}
\footnotesize{6}
\nodepart{four}
\footnotesize{$1$}
};
 & 
\node[draw=black, rectangle split,  rectangle split parts=4] (sn0x188acf0){
\footnotesize{0.424897}
\nodepart{two}
\begin{tikzpicture}[scale=.2]
\node[circle, scale=0.75, fill] (tid0) at (1.5,1.5){};
\node[circle, scale=0.75, fill] (tid1) at (0.75,3){};
\node[circle, scale=0.75, fill] (tid3) at (0.75,4.5){};
\node[circle, scale=0.75, fill] (tid4) at (0.75,6){};
\node[circle, scale=0.75, fill, red] (tid5) at (0.75,7.5){};
\draw[](tid4) -- (tid5);
\draw[](tid3) -- (tid4);
\draw[](tid1) -- (tid3);
\node[circle, scale=0.75, fill, red] (tid2) at (2.25,3){};
\draw[](tid0) -- (tid1);
\draw[](tid0) -- (tid2);
\end{tikzpicture}
\nodepart{three}
\footnotesize{5.0625}
\nodepart{four}
\footnotesize{$50\:50$}
};
 & 
\node[draw=black, rectangle split,  rectangle split parts=4] (sn0x188c6a0){
\footnotesize{0.351852}
\nodepart{two}
\begin{tikzpicture}[scale=.2]
\node[circle, scale=0.75, fill] (tid0) at (1.5,1.5){};
\node[circle, scale=0.75, fill] (tid1) at (0.75,3){};
\node[circle, scale=0.75, fill] (tid3) at (0.75,4.5){};
\node[circle, scale=0.75, fill, red] (tid5) at (0.75,6){};
\draw[](tid3) -- (tid5);
\draw[](tid1) -- (tid3);
\node[circle, scale=0.75, fill] (tid2) at (2.25,3){};
\node[circle, scale=0.75, fill, red] (tid4) at (2.25,4.5){};
\draw[](tid2) -- (tid4);
\draw[](tid0) -- (tid1);
\draw[](tid0) -- (tid2);
\end{tikzpicture}
\nodepart{three}
\footnotesize{4.4375}
\nodepart{four}
\footnotesize{$50\:50$}
};
 & 
\node[draw=black, rectangle split,  rectangle split parts=4] (sn0x1889230){
\footnotesize{0.0658436}
\nodepart{two}
\begin{tikzpicture}[scale=.2]
\node[circle, scale=0.75, fill] (tid0) at (1.5,1.5){};
\node[circle, scale=0.75, fill] (tid1) at (1.5,3){};
\node[circle, scale=0.75, fill] (tid2) at (1.5,4.5){};
\node[circle, scale=0.75, fill] (tid3) at (1.5,6){};
\node[circle, scale=0.75, fill, red] (tid4) at (0.75,7.5){};
\node[circle, scale=0.75, fill, red] (tid5) at (2.25,7.5){};
\draw[](tid3) -- (tid4);
\draw[](tid3) -- (tid5);
\draw[](tid2) -- (tid3);
\draw[](tid1) -- (tid2);
\draw[](tid0) -- (tid1);
\end{tikzpicture}
\nodepart{three}
\footnotesize{5.5}
\nodepart{four}
\footnotesize{$1$}
};
 & 
\\
};
\end{scope}
\begin{scope}[yshift=\leveltopIIIIIII cm]
\matrix (line7) [column sep=1cm] {
\node[draw=black, rectangle split,  rectangle split parts=4] (sn0x1888b50){
\footnotesize{0.4357}
\nodepart{two}
\begin{tikzpicture}[scale=.2]
\node[circle, scale=0.75, fill] (tid0) at (0.75,1.5){};
\node[circle, scale=0.75, fill] (tid1) at (0.75,3){};
\node[circle, scale=0.75, fill] (tid2) at (0.75,4.5){};
\node[circle, scale=0.75, fill] (tid3) at (0.75,6){};
\node[circle, scale=0.75, fill, red] (tid4) at (0.75,7.5){};
\draw[](tid3) -- (tid4);
\draw[](tid2) -- (tid3);
\draw[](tid1) -- (tid2);
\draw[](tid0) -- (tid1);
\end{tikzpicture}
\nodepart{three}
\footnotesize{5}
\nodepart{four}
\footnotesize{$1$}
};
 & 
\node[draw=black, rectangle split,  rectangle split parts=4] (sn0x188a9f0){
\footnotesize{0.388375}
\nodepart{two}
\begin{tikzpicture}[scale=.2]
\node[circle, scale=0.75, fill] (tid0) at (1.5,1.5){};
\node[circle, scale=0.75, fill] (tid1) at (0.75,3){};
\node[circle, scale=0.75, fill] (tid3) at (0.75,4.5){};
\node[circle, scale=0.75, fill, red] (tid4) at (0.75,6){};
\draw[](tid3) -- (tid4);
\draw[](tid1) -- (tid3);
\node[circle, scale=0.75, fill, red] (tid2) at (2.25,3){};
\draw[](tid0) -- (tid1);
\draw[](tid0) -- (tid2);
\end{tikzpicture}
\nodepart{three}
\footnotesize{4.125}
\nodepart{four}
\footnotesize{$50\:50$}
};
 & 
\node[draw=black, rectangle split,  rectangle split parts=4] (sn0x188c5b0){
\footnotesize{0.175926}
\nodepart{two}
\begin{tikzpicture}[scale=.2]
\node[circle, scale=0.75, fill] (tid0) at (1.5,1.5){};
\node[circle, scale=0.75, fill] (tid1) at (0.75,3){};
\node[circle, scale=0.75, fill, red] (tid3) at (0.75,4.5){};
\draw[](tid1) -- (tid3);
\node[circle, scale=0.75, fill] (tid2) at (2.25,3){};
\node[circle, scale=0.75, fill, red] (tid4) at (2.25,4.5){};
\draw[](tid2) -- (tid4);
\draw[](tid0) -- (tid1);
\draw[](tid0) -- (tid2);
\end{tikzpicture}
\nodepart{three}
\footnotesize{3.75}
\nodepart{four}
\footnotesize{$1$}
};
 & 
\\
};
\end{scope}
\begin{scope}[yshift=\leveltopIIIIIIII cm]
\matrix (line8) [column sep=1cm] {
\node[draw=black, rectangle split,  rectangle split parts=4] (sn0x1888900){
\footnotesize{0.629887}
\nodepart{two}
\begin{tikzpicture}[scale=.2]
\node[circle, scale=0.75, fill] (tid0) at (0.75,1.5){};
\node[circle, scale=0.75, fill] (tid1) at (0.75,3){};
\node[circle, scale=0.75, fill] (tid2) at (0.75,4.5){};
\node[circle, scale=0.75, fill, red] (tid3) at (0.75,6){};
\draw[](tid2) -- (tid3);
\draw[](tid1) -- (tid2);
\draw[](tid0) -- (tid1);
\end{tikzpicture}
\nodepart{three}
\footnotesize{4}
\nodepart{four}
\footnotesize{$1$}
};
 & 
\node[draw=black, rectangle split,  rectangle split parts=4] (sn0x188a920){
\footnotesize{0.370113}
\nodepart{two}
\begin{tikzpicture}[scale=.2]
\node[circle, scale=0.75, fill] (tid0) at (1.5,1.5){};
\node[circle, scale=0.75, fill] (tid1) at (0.75,3){};
\node[circle, scale=0.75, fill, red] (tid3) at (0.75,4.5){};
\draw[](tid1) -- (tid3);
\node[circle, scale=0.75, fill, red] (tid2) at (2.25,3){};
\draw[](tid0) -- (tid1);
\draw[](tid0) -- (tid2);
\end{tikzpicture}
\nodepart{three}
\footnotesize{3.25}
\nodepart{four}
\footnotesize{$50\:50$}
};
 & 
\\
};
\end{scope}
\begin{scope}[yshift=\leveltopIIIIIIIII cm]
\matrix (line9) [column sep=1cm] {
\node[draw=black, rectangle split,  rectangle split parts=4] (sn0x1888830){
\footnotesize{0.814943}
\nodepart{two}
\begin{tikzpicture}[scale=.2]
\node[circle, scale=0.75, fill] (tid0) at (0.75,1.5){};
\node[circle, scale=0.75, fill] (tid1) at (0.75,3){};
\node[circle, scale=0.75, fill, red] (tid2) at (0.75,4.5){};
\draw[](tid1) -- (tid2);
\draw[](tid0) -- (tid1);
\end{tikzpicture}
\nodepart{three}
\footnotesize{3}
\nodepart{four}
\footnotesize{$1$}
};
 & 
\node[draw=black, rectangle split,  rectangle split parts=4] (sn0x1889c10){
\footnotesize{0.185057}
\nodepart{two}
\begin{tikzpicture}[scale=.2]
\node[circle, scale=0.75, fill] (tid0) at (1.5,1.5){};
\node[circle, scale=0.75, fill, red] (tid1) at (0.75,3){};
\node[circle, scale=0.75, fill, red] (tid2) at (2.25,3){};
\draw[](tid0) -- (tid1);
\draw[](tid0) -- (tid2);
\end{tikzpicture}
\nodepart{three}
\footnotesize{2.5}
\nodepart{four}
\footnotesize{$1$}
};
 & 
\\
};
\end{scope}
\begin{scope}[yshift=\leveltopIIIIIIIIII cm]
\matrix (line10) [column sep=1cm] {
\node[draw=black, rectangle split,  rectangle split parts=4] (sn0x1887520){
\footnotesize{1}
\nodepart{two}
\begin{tikzpicture}[scale=.2]
\node[circle, scale=0.75, fill] (tid0) at (0.75,1.5){};
\node[circle, scale=0.75, fill, red] (tid1) at (0.75,3){};
\draw[](tid0) -- (tid1);
\end{tikzpicture}
\nodepart{three}
\footnotesize{2}
\nodepart{four}
\footnotesize{$1$}
};
 & 
\\
};
\end{scope}
\begin{scope}[yshift=\leveltopIIIIIIIIIII cm]
\matrix (line11) [column sep=1cm] {
\node[draw=black, rectangle split,  rectangle split parts=4] (sn0x1887450){
\footnotesize{1}
\nodepart{two}
\begin{tikzpicture}[scale=.2]
\node[circle, scale=0.75, fill, red] (tid0) at (0.75,1.5){};
\end{tikzpicture}
\nodepart{three}
\footnotesize{1}
\nodepart{four}
\footnotesize{$$}
};
 & 
\\
};
\end{scope}
\begin{scope}[yshift=\leveltopIIIIIIIIIIII cm]
\matrix (line12) [column sep=1cm] {
\\
};
\end{scope}
\draw (sn0x18908d0.south) -- (sn0x188ec30.north);
\draw (sn0x18908d0.south) -- (sn0x18906a0.north);
\draw (sn0x188ec30.south) -- (sn0x188e220.north);
\draw (sn0x188ec30.south) -- (sn0x188d8a0.north);
\draw (sn0x18906a0.south) -- (sn0x188d8a0.north);
\draw (sn0x18906a0.south) -- (sn0x188f0f0.north);
\draw (sn0x18906a0.south) -- (sn0x188ff90.north);
\draw (sn0x188e220.south) -- (sn0x188c370.north);
\draw (sn0x188e220.south) -- (sn0x188d690.north);
\draw (sn0x188d8a0.south) -- (sn0x188b180.north);
\draw (sn0x188d8a0.south) -- (sn0x188d690.north);
\draw (sn0x188d8a0.south) -- (sn0x188de90.north);
\draw (sn0x188f0f0.south) -- (sn0x188d690.north);
\draw (sn0x188f0f0.south) -- (sn0x188f020.north);
\draw (sn0x188ff90.south) -- (sn0x188de90.north);
\draw (sn0x188ff90.south) -- (sn0x188f020.north);
\draw (sn0x188c370.south) -- (sn0x1888fa0.north);
\draw (sn0x188c370.south) -- (sn0x188ae00.north);
\draw (sn0x188d690.south) -- (sn0x188ae00.north);
\draw (sn0x188d690.south) -- (sn0x188d460.north);
\draw (sn0x188b180.south) -- (sn0x1889a70.north);
\draw (sn0x188b180.south) -- (sn0x188ae00.north);
\draw (sn0x188b180.south) -- (sn0x188b630.north);
\draw (sn0x188de90.south) -- (sn0x188b630.north);
\draw (sn0x188de90.south) -- (sn0x188d460.north);
\draw (sn0x188f020.south) -- (sn0x188d460.north);
\draw (sn0x188f020.south) -- (sn0x188fd60.north);
\draw (sn0x1888fa0.south) -- (sn0x1888e50.north);
\draw (sn0x188ae00.south) -- (sn0x1888e50.north);
\draw (sn0x188ae00.south) -- (sn0x188acf0.north);
\draw (sn0x188d460.south) -- (sn0x188acf0.north);
\draw (sn0x188d460.south) -- (sn0x188c6a0.north);
\draw (sn0x1889a70.south) -- (sn0x1888e50.north);
\draw (sn0x1889a70.south) -- (sn0x1889230.north);
\draw (sn0x188b630.south) -- (sn0x1889230.north);
\draw (sn0x188b630.south) -- (sn0x188acf0.north);
\draw (sn0x188fd60.south) -- (sn0x188c6a0.north);
\draw (sn0x1888e50.south) -- (sn0x1888b50.north);
\draw (sn0x188acf0.south) -- (sn0x1888b50.north);
\draw (sn0x188acf0.south) -- (sn0x188a9f0.north);
\draw (sn0x188c6a0.south) -- (sn0x188a9f0.north);
\draw (sn0x188c6a0.south) -- (sn0x188c5b0.north);
\draw (sn0x1889230.south) -- (sn0x1888b50.north);
\draw (sn0x1888b50.south) -- (sn0x1888900.north);
\draw (sn0x188a9f0.south) -- (sn0x1888900.north);
\draw (sn0x188a9f0.south) -- (sn0x188a920.north);
\draw (sn0x188c5b0.south) -- (sn0x188a920.north);
\draw (sn0x1888900.south) -- (sn0x1888830.north);
\draw (sn0x188a920.south) -- (sn0x1888830.north);
\draw (sn0x188a920.south) -- (sn0x1889c10.north);
\draw (sn0x1888830.south) -- (sn0x1887520.north);
\draw (sn0x1889c10.south) -- (sn0x1887520.north);
\draw (sn0x1887520.south) -- (sn0x1887450.north);
\end{tikzpicture}

%%% Local Variables:
%%% TeX-master: "thesis/thesis.tex"
%%% End: 
\renewcommand{\leveltopI}{-15cm + \leveltop}
\renewcommand{\leveltopII}{-15cm + \leveltopI}
\renewcommand{\leveltopIII}{-15cm + \leveltopII}
\renewcommand{\leveltopIIII}{-15cm + \leveltopIII}
\renewcommand{\leveltopIIIII}{-15cm + \leveltopIIII}
\renewcommand{\leveltopIIIIII}{-15cm + \leveltopIIIII}
\renewcommand{\leveltopIIIIIII}{-15cm + \leveltopIIIIII}
\renewcommand{\leveltopIIIIIIII}{-15cm + \leveltopIIIIIII}
\renewcommand{\leveltopIIIIIIIII}{-15cm + \leveltopIIIIIIII}
\renewcommand{\leveltopIIIIIIIIII}{-15cm + \leveltopIIIIIIIII}
\renewcommand{\leveltopIIIIIIIIIII}{-15cm + \leveltopIIIIIIIIII}
\begin{tikzpicture}[scale=.2, anchor=south]
\begin{scope}[yshift=\leveltopI cm]
\matrix (line1) [column sep=1cm] {
\node[draw=black, rectangle split,  rectangle split parts=4] (sn0x1891fa0){
\footnotesize{1}
\nodepart{two}
\begin{tikzpicture}[scale=.2]
\node[circle, scale=0.75, fill] (tid0) at (3,1.5){};
\node[circle, scale=0.75, fill] (tid1) at (2.25,3){};
\node[circle, scale=0.75, fill] (tid3) at (2.25,4.5){};
\node[circle, scale=0.75, fill] (tid5) at (2.25,6){};
\node[circle, scale=0.75, fill] (tid7) at (1.5,7.5){};
\node[circle, scale=0.75, fill, red] (tid9) at (0.75,9){};
\node[circle, scale=0.75, fill] (tid10) at (2.25,9){};
\draw[](tid7) -- (tid9);
\draw[](tid7) -- (tid10);
\node[circle, scale=0.75, fill, red] (tid8) at (3.75,7.5){};
\draw[](tid5) -- (tid7);
\draw[](tid5) -- (tid8);
\draw[](tid3) -- (tid5);
\draw[](tid1) -- (tid3);
\node[circle, scale=0.75, fill] (tid2) at (5.25,3){};
\node[circle, scale=0.75, fill] (tid4) at (5.25,4.5){};
\node[circle, scale=0.75, fill, red] (tid6) at (5.25,6){};
\draw[](tid4) -- (tid6);
\draw[](tid2) -- (tid4);
\draw[](tid0) -- (tid1);
\draw[](tid0) -- (tid2);
\end{tikzpicture}
\nodepart{three}
\footnotesize{7.03938}
\nodepart{four}
\footnotesize{$33\:33\:33$}
};
 & 
\\
};
\end{scope}
\begin{scope}[yshift=\leveltopII cm]
\matrix (line2) [column sep=1cm] {
\node[draw=black, rectangle split,  rectangle split parts=4] (sn0x188ec30){
\footnotesize{0.333333}
\nodepart{two}
\begin{tikzpicture}[scale=.2]
\node[circle, scale=0.75, fill] (tid0) at (3,1.5){};
\node[circle, scale=0.75, fill] (tid1) at (2.25,3){};
\node[circle, scale=0.75, fill] (tid3) at (2.25,4.5){};
\node[circle, scale=0.75, fill] (tid5) at (2.25,6){};
\node[circle, scale=0.75, fill] (tid6) at (1.5,7.5){};
\node[circle, scale=0.75, fill, red] (tid8) at (0.75,9){};
\node[circle, scale=0.75, fill, red] (tid9) at (2.25,9){};
\draw[](tid6) -- (tid8);
\draw[](tid6) -- (tid9);
\node[circle, scale=0.75, fill, red] (tid7) at (3.75,7.5){};
\draw[](tid5) -- (tid6);
\draw[](tid5) -- (tid7);
\draw[](tid3) -- (tid5);
\draw[](tid1) -- (tid3);
\node[circle, scale=0.75, fill] (tid2) at (5.25,3){};
\node[circle, scale=0.75, fill] (tid4) at (5.25,4.5){};
\draw[](tid2) -- (tid4);
\draw[](tid0) -- (tid1);
\draw[](tid0) -- (tid2);
\end{tikzpicture}
\nodepart{three}
\footnotesize{6.77701}
\nodepart{four}
\footnotesize{$33\:67$}
};
 & 
\node[draw=black, rectangle split,  rectangle split parts=4] (sn0x1891ba0){
\footnotesize{0.333333}
\nodepart{two}
\begin{tikzpicture}[scale=.2]
\node[circle, scale=0.75, fill] (tid0) at (2.25,1.5){};
\node[circle, scale=0.75, fill] (tid1) at (1.5,3){};
\node[circle, scale=0.75, fill] (tid3) at (1.5,4.5){};
\node[circle, scale=0.75, fill] (tid5) at (1.5,6){};
\node[circle, scale=0.75, fill] (tid7) at (1.5,7.5){};
\node[circle, scale=0.75, fill, red] (tid8) at (0.75,9){};
\node[circle, scale=0.75, fill, red] (tid9) at (2.25,9){};
\draw[](tid7) -- (tid8);
\draw[](tid7) -- (tid9);
\draw[](tid5) -- (tid7);
\draw[](tid3) -- (tid5);
\draw[](tid1) -- (tid3);
\node[circle, scale=0.75, fill] (tid2) at (3.75,3){};
\node[circle, scale=0.75, fill] (tid4) at (3.75,4.5){};
\node[circle, scale=0.75, fill, red] (tid6) at (3.75,6){};
\draw[](tid4) -- (tid6);
\draw[](tid2) -- (tid4);
\draw[](tid0) -- (tid1);
\draw[](tid0) -- (tid2);
\end{tikzpicture}
\nodepart{three}
\footnotesize{6.77836}
\nodepart{four}
\footnotesize{$33\:67$}
};
 & 
\node[draw=black, rectangle split,  rectangle split parts=4] (sn0x18906a0){
\footnotesize{0.333333}
\nodepart{two}
\begin{tikzpicture}[scale=.2]
\node[circle, scale=0.75, fill] (tid0) at (2.25,1.5){};
\node[circle, scale=0.75, fill] (tid1) at (1.5,3){};
\node[circle, scale=0.75, fill] (tid3) at (1.5,4.5){};
\node[circle, scale=0.75, fill] (tid5) at (1.5,6){};
\node[circle, scale=0.75, fill] (tid7) at (0.75,7.5){};
\node[circle, scale=0.75, fill, red] (tid9) at (0.75,9){};
\draw[](tid7) -- (tid9);
\node[circle, scale=0.75, fill, red] (tid8) at (2.25,7.5){};
\draw[](tid5) -- (tid7);
\draw[](tid5) -- (tid8);
\draw[](tid3) -- (tid5);
\draw[](tid1) -- (tid3);
\node[circle, scale=0.75, fill] (tid2) at (3.75,3){};
\node[circle, scale=0.75, fill] (tid4) at (3.75,4.5){};
\node[circle, scale=0.75, fill, red] (tid6) at (3.75,6){};
\draw[](tid4) -- (tid6);
\draw[](tid2) -- (tid4);
\draw[](tid0) -- (tid1);
\draw[](tid0) -- (tid2);
\end{tikzpicture}
\nodepart{three}
\footnotesize{6.56279}
\nodepart{four}
\footnotesize{$33\:33\:33$}
};
 & 
\\
};
\end{scope}
\begin{scope}[yshift=\leveltopIII cm]
\matrix (line3) [column sep=1cm] {
\node[draw=black, rectangle split,  rectangle split parts=4] (sn0x188e220){
\footnotesize{0.222222}
\nodepart{two}
\begin{tikzpicture}[scale=.2]
\node[circle, scale=0.75, fill] (tid0) at (2.25,1.5){};
\node[circle, scale=0.75, fill] (tid1) at (1.5,3){};
\node[circle, scale=0.75, fill] (tid3) at (1.5,4.5){};
\node[circle, scale=0.75, fill] (tid5) at (1.5,6){};
\node[circle, scale=0.75, fill] (tid6) at (1.5,7.5){};
\node[circle, scale=0.75, fill, red] (tid7) at (0.75,9){};
\node[circle, scale=0.75, fill, red] (tid8) at (2.25,9){};
\draw[](tid6) -- (tid7);
\draw[](tid6) -- (tid8);
\draw[](tid5) -- (tid6);
\draw[](tid3) -- (tid5);
\draw[](tid1) -- (tid3);
\node[circle, scale=0.75, fill] (tid2) at (3.75,3){};
\node[circle, scale=0.75, fill, red] (tid4) at (3.75,4.5){};
\draw[](tid2) -- (tid4);
\draw[](tid0) -- (tid1);
\draw[](tid0) -- (tid2);
\end{tikzpicture}
\nodepart{three}
\footnotesize{6.60069}
\nodepart{four}
\footnotesize{$33\:67$}
};
 & 
\node[draw=black, rectangle split,  rectangle split parts=4] (sn0x188d8a0){
\footnotesize{0.333333}
\nodepart{two}
\begin{tikzpicture}[scale=.2]
\node[circle, scale=0.75, fill] (tid0) at (2.25,1.5){};
\node[circle, scale=0.75, fill] (tid1) at (1.5,3){};
\node[circle, scale=0.75, fill] (tid3) at (1.5,4.5){};
\node[circle, scale=0.75, fill] (tid5) at (1.5,6){};
\node[circle, scale=0.75, fill] (tid6) at (0.75,7.5){};
\node[circle, scale=0.75, fill, red] (tid8) at (0.75,9){};
\draw[](tid6) -- (tid8);
\node[circle, scale=0.75, fill, red] (tid7) at (2.25,7.5){};
\draw[](tid5) -- (tid6);
\draw[](tid5) -- (tid7);
\draw[](tid3) -- (tid5);
\draw[](tid1) -- (tid3);
\node[circle, scale=0.75, fill] (tid2) at (3.75,3){};
\node[circle, scale=0.75, fill, red] (tid4) at (3.75,4.5){};
\draw[](tid2) -- (tid4);
\draw[](tid0) -- (tid1);
\draw[](tid0) -- (tid2);
\end{tikzpicture}
\nodepart{three}
\footnotesize{6.36516}
\nodepart{four}
\footnotesize{$33\:33\:33$}
};
 & 
\node[draw=black, rectangle split,  rectangle split parts=4] (sn0x188f0f0){
\footnotesize{0.333333}
\nodepart{two}
\begin{tikzpicture}[scale=.2]
\node[circle, scale=0.75, fill] (tid0) at (1.5,1.5){};
\node[circle, scale=0.75, fill] (tid1) at (0.75,3){};
\node[circle, scale=0.75, fill] (tid3) at (0.75,4.5){};
\node[circle, scale=0.75, fill] (tid5) at (0.75,6){};
\node[circle, scale=0.75, fill] (tid7) at (0.75,7.5){};
\node[circle, scale=0.75, fill, red] (tid8) at (0.75,9){};
\draw[](tid7) -- (tid8);
\draw[](tid5) -- (tid7);
\draw[](tid3) -- (tid5);
\draw[](tid1) -- (tid3);
\node[circle, scale=0.75, fill] (tid2) at (2.25,3){};
\node[circle, scale=0.75, fill] (tid4) at (2.25,4.5){};
\node[circle, scale=0.75, fill, red] (tid6) at (2.25,6){};
\draw[](tid4) -- (tid6);
\draw[](tid2) -- (tid4);
\draw[](tid0) -- (tid1);
\draw[](tid0) -- (tid2);
\end{tikzpicture}
\nodepart{three}
\footnotesize{6.36719}
\nodepart{four}
\footnotesize{$50\:50$}
};
 & 
\node[draw=black, rectangle split,  rectangle split parts=4] (sn0x188ff90){
\footnotesize{0.111111}
\nodepart{two}
\begin{tikzpicture}[scale=.2]
\node[circle, scale=0.75, fill] (tid0) at (2.25,1.5){};
\node[circle, scale=0.75, fill] (tid1) at (1.5,3){};
\node[circle, scale=0.75, fill] (tid3) at (1.5,4.5){};
\node[circle, scale=0.75, fill] (tid5) at (1.5,6){};
\node[circle, scale=0.75, fill, red] (tid7) at (0.75,7.5){};
\node[circle, scale=0.75, fill, red] (tid8) at (2.25,7.5){};
\draw[](tid5) -- (tid7);
\draw[](tid5) -- (tid8);
\draw[](tid3) -- (tid5);
\draw[](tid1) -- (tid3);
\node[circle, scale=0.75, fill] (tid2) at (3.75,3){};
\node[circle, scale=0.75, fill] (tid4) at (3.75,4.5){};
\node[circle, scale=0.75, fill, red] (tid6) at (3.75,6){};
\draw[](tid4) -- (tid6);
\draw[](tid2) -- (tid4);
\draw[](tid0) -- (tid1);
\draw[](tid0) -- (tid2);
\end{tikzpicture}
\nodepart{three}
\footnotesize{5.95602}
\nodepart{four}
\footnotesize{$33\:67$}
};
 & 
\\
};
\end{scope}
\begin{scope}[yshift=\leveltopIIII cm]
\matrix (line4) [column sep=1cm] {
\node[draw=black, rectangle split,  rectangle split parts=4] (sn0x188c370){
\footnotesize{0.0740741}
\nodepart{two}
\begin{tikzpicture}[scale=.2]
\node[circle, scale=0.75, fill] (tid0) at (2.25,1.5){};
\node[circle, scale=0.75, fill] (tid1) at (1.5,3){};
\node[circle, scale=0.75, fill] (tid3) at (1.5,4.5){};
\node[circle, scale=0.75, fill] (tid4) at (1.5,6){};
\node[circle, scale=0.75, fill] (tid5) at (1.5,7.5){};
\node[circle, scale=0.75, fill, red] (tid6) at (0.75,9){};
\node[circle, scale=0.75, fill, red] (tid7) at (2.25,9){};
\draw[](tid5) -- (tid6);
\draw[](tid5) -- (tid7);
\draw[](tid4) -- (tid5);
\draw[](tid3) -- (tid4);
\draw[](tid1) -- (tid3);
\node[circle, scale=0.75, fill, red] (tid2) at (3.75,3){};
\draw[](tid0) -- (tid1);
\draw[](tid0) -- (tid2);
\end{tikzpicture}
\nodepart{three}
\footnotesize{6.52083}
\nodepart{four}
\footnotesize{$33\:67$}
};
 & 
\node[draw=black, rectangle split,  rectangle split parts=4] (sn0x188d690){
\footnotesize{0.425926}
\nodepart{two}
\begin{tikzpicture}[scale=.2]
\node[circle, scale=0.75, fill] (tid0) at (1.5,1.5){};
\node[circle, scale=0.75, fill] (tid1) at (0.75,3){};
\node[circle, scale=0.75, fill] (tid3) at (0.75,4.5){};
\node[circle, scale=0.75, fill] (tid5) at (0.75,6){};
\node[circle, scale=0.75, fill] (tid6) at (0.75,7.5){};
\node[circle, scale=0.75, fill, red] (tid7) at (0.75,9){};
\draw[](tid6) -- (tid7);
\draw[](tid5) -- (tid6);
\draw[](tid3) -- (tid5);
\draw[](tid1) -- (tid3);
\node[circle, scale=0.75, fill] (tid2) at (2.25,3){};
\node[circle, scale=0.75, fill, red] (tid4) at (2.25,4.5){};
\draw[](tid2) -- (tid4);
\draw[](tid0) -- (tid1);
\draw[](tid0) -- (tid2);
\end{tikzpicture}
\nodepart{three}
\footnotesize{6.14062}
\nodepart{four}
\footnotesize{$50\:50$}
};
 & 
\node[draw=black, rectangle split,  rectangle split parts=4] (sn0x188b180){
\footnotesize{0.111111}
\nodepart{two}
\begin{tikzpicture}[scale=.2]
\node[circle, scale=0.75, fill] (tid0) at (2.25,1.5){};
\node[circle, scale=0.75, fill] (tid1) at (1.5,3){};
\node[circle, scale=0.75, fill] (tid3) at (1.5,4.5){};
\node[circle, scale=0.75, fill] (tid4) at (1.5,6){};
\node[circle, scale=0.75, fill] (tid5) at (0.75,7.5){};
\node[circle, scale=0.75, fill, red] (tid7) at (0.75,9){};
\draw[](tid5) -- (tid7);
\node[circle, scale=0.75, fill, red] (tid6) at (2.25,7.5){};
\draw[](tid4) -- (tid5);
\draw[](tid4) -- (tid6);
\draw[](tid3) -- (tid4);
\draw[](tid1) -- (tid3);
\node[circle, scale=0.75, fill, red] (tid2) at (3.75,3){};
\draw[](tid0) -- (tid1);
\draw[](tid0) -- (tid2);
\end{tikzpicture}
\nodepart{three}
\footnotesize{6.27431}
\nodepart{four}
\footnotesize{$33\:33\:33$}
};
 & 
\node[draw=black, rectangle split,  rectangle split parts=4] (sn0x188de90){
\footnotesize{0.148148}
\nodepart{two}
\begin{tikzpicture}[scale=.2]
\node[circle, scale=0.75, fill] (tid0) at (2.25,1.5){};
\node[circle, scale=0.75, fill] (tid1) at (1.5,3){};
\node[circle, scale=0.75, fill] (tid3) at (1.5,4.5){};
\node[circle, scale=0.75, fill] (tid5) at (1.5,6){};
\node[circle, scale=0.75, fill, red] (tid6) at (0.75,7.5){};
\node[circle, scale=0.75, fill, red] (tid7) at (2.25,7.5){};
\draw[](tid5) -- (tid6);
\draw[](tid5) -- (tid7);
\draw[](tid3) -- (tid5);
\draw[](tid1) -- (tid3);
\node[circle, scale=0.75, fill] (tid2) at (3.75,3){};
\node[circle, scale=0.75, fill, red] (tid4) at (3.75,4.5){};
\draw[](tid2) -- (tid4);
\draw[](tid0) -- (tid1);
\draw[](tid0) -- (tid2);
\end{tikzpicture}
\nodepart{three}
\footnotesize{5.68056}
\nodepart{four}
\footnotesize{$67\:33$}
};
 & 
\node[draw=black, rectangle split,  rectangle split parts=4] (sn0x188f020){
\footnotesize{0.240741}
\nodepart{two}
\begin{tikzpicture}[scale=.2]
\node[circle, scale=0.75, fill] (tid0) at (1.5,1.5){};
\node[circle, scale=0.75, fill] (tid1) at (0.75,3){};
\node[circle, scale=0.75, fill] (tid3) at (0.75,4.5){};
\node[circle, scale=0.75, fill] (tid5) at (0.75,6){};
\node[circle, scale=0.75, fill, red] (tid7) at (0.75,7.5){};
\draw[](tid5) -- (tid7);
\draw[](tid3) -- (tid5);
\draw[](tid1) -- (tid3);
\node[circle, scale=0.75, fill] (tid2) at (2.25,3){};
\node[circle, scale=0.75, fill] (tid4) at (2.25,4.5){};
\node[circle, scale=0.75, fill, red] (tid6) at (2.25,6){};
\draw[](tid4) -- (tid6);
\draw[](tid2) -- (tid4);
\draw[](tid0) -- (tid1);
\draw[](tid0) -- (tid2);
\end{tikzpicture}
\nodepart{three}
\footnotesize{5.59375}
\nodepart{four}
\footnotesize{$50\:50$}
};
 & 
\\
};
\end{scope}
\begin{scope}[yshift=\leveltopIIIII cm]
\matrix (line5) [column sep=1cm] {
\node[draw=black, rectangle split,  rectangle split parts=4] (sn0x1888fa0){
\footnotesize{0.0246914}
\nodepart{two}
\begin{tikzpicture}[scale=.2]
\node[circle, scale=0.75, fill] (tid0) at (1.5,1.5){};
\node[circle, scale=0.75, fill] (tid1) at (1.5,3){};
\node[circle, scale=0.75, fill] (tid2) at (1.5,4.5){};
\node[circle, scale=0.75, fill] (tid3) at (1.5,6){};
\node[circle, scale=0.75, fill] (tid4) at (1.5,7.5){};
\node[circle, scale=0.75, fill, red] (tid5) at (0.75,9){};
\node[circle, scale=0.75, fill, red] (tid6) at (2.25,9){};
\draw[](tid4) -- (tid5);
\draw[](tid4) -- (tid6);
\draw[](tid3) -- (tid4);
\draw[](tid2) -- (tid3);
\draw[](tid1) -- (tid2);
\draw[](tid0) -- (tid1);
\end{tikzpicture}
\nodepart{three}
\footnotesize{6.5}
\nodepart{four}
\footnotesize{$1$}
};
 & 
\node[draw=black, rectangle split,  rectangle split parts=4] (sn0x188ae00){
\footnotesize{0.299383}
\nodepart{two}
\begin{tikzpicture}[scale=.2]
\node[circle, scale=0.75, fill] (tid0) at (1.5,1.5){};
\node[circle, scale=0.75, fill] (tid1) at (0.75,3){};
\node[circle, scale=0.75, fill] (tid3) at (0.75,4.5){};
\node[circle, scale=0.75, fill] (tid4) at (0.75,6){};
\node[circle, scale=0.75, fill] (tid5) at (0.75,7.5){};
\node[circle, scale=0.75, fill, red] (tid6) at (0.75,9){};
\draw[](tid5) -- (tid6);
\draw[](tid4) -- (tid5);
\draw[](tid3) -- (tid4);
\draw[](tid1) -- (tid3);
\node[circle, scale=0.75, fill, red] (tid2) at (2.25,3){};
\draw[](tid0) -- (tid1);
\draw[](tid0) -- (tid2);
\end{tikzpicture}
\nodepart{three}
\footnotesize{6.03125}
\nodepart{four}
\footnotesize{$50\:50$}
};
 & 
\node[draw=black, rectangle split,  rectangle split parts=4] (sn0x188d460){
\footnotesize{0.432099}
\nodepart{two}
\begin{tikzpicture}[scale=.2]
\node[circle, scale=0.75, fill] (tid0) at (1.5,1.5){};
\node[circle, scale=0.75, fill] (tid1) at (0.75,3){};
\node[circle, scale=0.75, fill] (tid3) at (0.75,4.5){};
\node[circle, scale=0.75, fill] (tid5) at (0.75,6){};
\node[circle, scale=0.75, fill, red] (tid6) at (0.75,7.5){};
\draw[](tid5) -- (tid6);
\draw[](tid3) -- (tid5);
\draw[](tid1) -- (tid3);
\node[circle, scale=0.75, fill] (tid2) at (2.25,3){};
\node[circle, scale=0.75, fill, red] (tid4) at (2.25,4.5){};
\draw[](tid2) -- (tid4);
\draw[](tid0) -- (tid1);
\draw[](tid0) -- (tid2);
\end{tikzpicture}
\nodepart{three}
\footnotesize{5.25}
\nodepart{four}
\footnotesize{$50\:50$}
};
 & 
\node[draw=black, rectangle split,  rectangle split parts=4] (sn0x1889a70){
\footnotesize{0.037037}
\nodepart{two}
\begin{tikzpicture}[scale=.2]
\node[circle, scale=0.75, fill] (tid0) at (1.5,1.5){};
\node[circle, scale=0.75, fill] (tid1) at (1.5,3){};
\node[circle, scale=0.75, fill] (tid2) at (1.5,4.5){};
\node[circle, scale=0.75, fill] (tid3) at (1.5,6){};
\node[circle, scale=0.75, fill] (tid4) at (0.75,7.5){};
\node[circle, scale=0.75, fill, red] (tid6) at (0.75,9){};
\draw[](tid4) -- (tid6);
\node[circle, scale=0.75, fill, red] (tid5) at (2.25,7.5){};
\draw[](tid3) -- (tid4);
\draw[](tid3) -- (tid5);
\draw[](tid2) -- (tid3);
\draw[](tid1) -- (tid2);
\draw[](tid0) -- (tid1);
\end{tikzpicture}
\nodepart{three}
\footnotesize{6.25}
\nodepart{four}
\footnotesize{$50\:50$}
};
 & 
\node[draw=black, rectangle split,  rectangle split parts=4] (sn0x188b630){
\footnotesize{0.0864198}
\nodepart{two}
\begin{tikzpicture}[scale=.2]
\node[circle, scale=0.75, fill] (tid0) at (2.25,1.5){};
\node[circle, scale=0.75, fill] (tid1) at (1.5,3){};
\node[circle, scale=0.75, fill] (tid3) at (1.5,4.5){};
\node[circle, scale=0.75, fill] (tid4) at (1.5,6){};
\node[circle, scale=0.75, fill, red] (tid5) at (0.75,7.5){};
\node[circle, scale=0.75, fill, red] (tid6) at (2.25,7.5){};
\draw[](tid4) -- (tid5);
\draw[](tid4) -- (tid6);
\draw[](tid3) -- (tid4);
\draw[](tid1) -- (tid3);
\node[circle, scale=0.75, fill, red] (tid2) at (3.75,3){};
\draw[](tid0) -- (tid1);
\draw[](tid0) -- (tid2);
\end{tikzpicture}
\nodepart{three}
\footnotesize{5.54167}
\nodepart{four}
\footnotesize{$67\:33$}
};
 & 
\node[draw=black, rectangle split,  rectangle split parts=4] (sn0x188fd60){
\footnotesize{0.12037}
\nodepart{two}
\begin{tikzpicture}[scale=.2]
\node[circle, scale=0.75, fill] (tid0) at (1.5,1.5){};
\node[circle, scale=0.75, fill] (tid1) at (0.75,3){};
\node[circle, scale=0.75, fill] (tid3) at (0.75,4.5){};
\node[circle, scale=0.75, fill, red] (tid5) at (0.75,6){};
\draw[](tid3) -- (tid5);
\draw[](tid1) -- (tid3);
\node[circle, scale=0.75, fill] (tid2) at (2.25,3){};
\node[circle, scale=0.75, fill] (tid4) at (2.25,4.5){};
\node[circle, scale=0.75, fill, red] (tid6) at (2.25,6){};
\draw[](tid4) -- (tid6);
\draw[](tid2) -- (tid4);
\draw[](tid0) -- (tid1);
\draw[](tid0) -- (tid2);
\end{tikzpicture}
\nodepart{three}
\footnotesize{4.9375}
\nodepart{four}
\footnotesize{$1$}
};
 & 
\\
};
\end{scope}
\begin{scope}[yshift=\leveltopIIIIII cm]
\matrix (line6) [column sep=1cm] {
\node[draw=black, rectangle split,  rectangle split parts=4] (sn0x1888e50){
\footnotesize{0.192901}
\nodepart{two}
\begin{tikzpicture}[scale=.2]
\node[circle, scale=0.75, fill] (tid0) at (0.75,1.5){};
\node[circle, scale=0.75, fill] (tid1) at (0.75,3){};
\node[circle, scale=0.75, fill] (tid2) at (0.75,4.5){};
\node[circle, scale=0.75, fill] (tid3) at (0.75,6){};
\node[circle, scale=0.75, fill] (tid4) at (0.75,7.5){};
\node[circle, scale=0.75, fill, red] (tid5) at (0.75,9){};
\draw[](tid4) -- (tid5);
\draw[](tid3) -- (tid4);
\draw[](tid2) -- (tid3);
\draw[](tid1) -- (tid2);
\draw[](tid0) -- (tid1);
\end{tikzpicture}
\nodepart{three}
\footnotesize{6}
\nodepart{four}
\footnotesize{$1$}
};
 & 
\node[draw=black, rectangle split,  rectangle split parts=4] (sn0x188acf0){
\footnotesize{0.423354}
\nodepart{two}
\begin{tikzpicture}[scale=.2]
\node[circle, scale=0.75, fill] (tid0) at (1.5,1.5){};
\node[circle, scale=0.75, fill] (tid1) at (0.75,3){};
\node[circle, scale=0.75, fill] (tid3) at (0.75,4.5){};
\node[circle, scale=0.75, fill] (tid4) at (0.75,6){};
\node[circle, scale=0.75, fill, red] (tid5) at (0.75,7.5){};
\draw[](tid4) -- (tid5);
\draw[](tid3) -- (tid4);
\draw[](tid1) -- (tid3);
\node[circle, scale=0.75, fill, red] (tid2) at (2.25,3){};
\draw[](tid0) -- (tid1);
\draw[](tid0) -- (tid2);
\end{tikzpicture}
\nodepart{three}
\footnotesize{5.0625}
\nodepart{four}
\footnotesize{$50\:50$}
};
 & 
\node[draw=black, rectangle split,  rectangle split parts=4] (sn0x188c6a0){
\footnotesize{0.33642}
\nodepart{two}
\begin{tikzpicture}[scale=.2]
\node[circle, scale=0.75, fill] (tid0) at (1.5,1.5){};
\node[circle, scale=0.75, fill] (tid1) at (0.75,3){};
\node[circle, scale=0.75, fill] (tid3) at (0.75,4.5){};
\node[circle, scale=0.75, fill, red] (tid5) at (0.75,6){};
\draw[](tid3) -- (tid5);
\draw[](tid1) -- (tid3);
\node[circle, scale=0.75, fill] (tid2) at (2.25,3){};
\node[circle, scale=0.75, fill, red] (tid4) at (2.25,4.5){};
\draw[](tid2) -- (tid4);
\draw[](tid0) -- (tid1);
\draw[](tid0) -- (tid2);
\end{tikzpicture}
\nodepart{three}
\footnotesize{4.4375}
\nodepart{four}
\footnotesize{$50\:50$}
};
 & 
\node[draw=black, rectangle split,  rectangle split parts=4] (sn0x1889230){
\footnotesize{0.0473251}
\nodepart{two}
\begin{tikzpicture}[scale=.2]
\node[circle, scale=0.75, fill] (tid0) at (1.5,1.5){};
\node[circle, scale=0.75, fill] (tid1) at (1.5,3){};
\node[circle, scale=0.75, fill] (tid2) at (1.5,4.5){};
\node[circle, scale=0.75, fill] (tid3) at (1.5,6){};
\node[circle, scale=0.75, fill, red] (tid4) at (0.75,7.5){};
\node[circle, scale=0.75, fill, red] (tid5) at (2.25,7.5){};
\draw[](tid3) -- (tid4);
\draw[](tid3) -- (tid5);
\draw[](tid2) -- (tid3);
\draw[](tid1) -- (tid2);
\draw[](tid0) -- (tid1);
\end{tikzpicture}
\nodepart{three}
\footnotesize{5.5}
\nodepart{four}
\footnotesize{$1$}
};
 & 
\\
};
\end{scope}
\begin{scope}[yshift=\leveltopIIIIIII cm]
\matrix (line7) [column sep=1cm] {
\node[draw=black, rectangle split,  rectangle split parts=4] (sn0x1888b50){
\footnotesize{0.451903}
\nodepart{two}
\begin{tikzpicture}[scale=.2]
\node[circle, scale=0.75, fill] (tid0) at (0.75,1.5){};
\node[circle, scale=0.75, fill] (tid1) at (0.75,3){};
\node[circle, scale=0.75, fill] (tid2) at (0.75,4.5){};
\node[circle, scale=0.75, fill] (tid3) at (0.75,6){};
\node[circle, scale=0.75, fill, red] (tid4) at (0.75,7.5){};
\draw[](tid3) -- (tid4);
\draw[](tid2) -- (tid3);
\draw[](tid1) -- (tid2);
\draw[](tid0) -- (tid1);
\end{tikzpicture}
\nodepart{three}
\footnotesize{5}
\nodepart{four}
\footnotesize{$1$}
};
 & 
\node[draw=black, rectangle split,  rectangle split parts=4] (sn0x188a9f0){
\footnotesize{0.379887}
\nodepart{two}
\begin{tikzpicture}[scale=.2]
\node[circle, scale=0.75, fill] (tid0) at (1.5,1.5){};
\node[circle, scale=0.75, fill] (tid1) at (0.75,3){};
\node[circle, scale=0.75, fill] (tid3) at (0.75,4.5){};
\node[circle, scale=0.75, fill, red] (tid4) at (0.75,6){};
\draw[](tid3) -- (tid4);
\draw[](tid1) -- (tid3);
\node[circle, scale=0.75, fill, red] (tid2) at (2.25,3){};
\draw[](tid0) -- (tid1);
\draw[](tid0) -- (tid2);
\end{tikzpicture}
\nodepart{three}
\footnotesize{4.125}
\nodepart{four}
\footnotesize{$50\:50$}
};
 & 
\node[draw=black, rectangle split,  rectangle split parts=4] (sn0x188c5b0){
\footnotesize{0.16821}
\nodepart{two}
\begin{tikzpicture}[scale=.2]
\node[circle, scale=0.75, fill] (tid0) at (1.5,1.5){};
\node[circle, scale=0.75, fill] (tid1) at (0.75,3){};
\node[circle, scale=0.75, fill, red] (tid3) at (0.75,4.5){};
\draw[](tid1) -- (tid3);
\node[circle, scale=0.75, fill] (tid2) at (2.25,3){};
\node[circle, scale=0.75, fill, red] (tid4) at (2.25,4.5){};
\draw[](tid2) -- (tid4);
\draw[](tid0) -- (tid1);
\draw[](tid0) -- (tid2);
\end{tikzpicture}
\nodepart{three}
\footnotesize{3.75}
\nodepart{four}
\footnotesize{$1$}
};
 & 
\\
};
\end{scope}
\begin{scope}[yshift=\leveltopIIIIIIII cm]
\matrix (line8) [column sep=1cm] {
\node[draw=black, rectangle split,  rectangle split parts=4] (sn0x1888900){
\footnotesize{0.641847}
\nodepart{two}
\begin{tikzpicture}[scale=.2]
\node[circle, scale=0.75, fill] (tid0) at (0.75,1.5){};
\node[circle, scale=0.75, fill] (tid1) at (0.75,3){};
\node[circle, scale=0.75, fill] (tid2) at (0.75,4.5){};
\node[circle, scale=0.75, fill, red] (tid3) at (0.75,6){};
\draw[](tid2) -- (tid3);
\draw[](tid1) -- (tid2);
\draw[](tid0) -- (tid1);
\end{tikzpicture}
\nodepart{three}
\footnotesize{4}
\nodepart{four}
\footnotesize{$1$}
};
 & 
\node[draw=black, rectangle split,  rectangle split parts=4] (sn0x188a920){
\footnotesize{0.358153}
\nodepart{two}
\begin{tikzpicture}[scale=.2]
\node[circle, scale=0.75, fill] (tid0) at (1.5,1.5){};
\node[circle, scale=0.75, fill] (tid1) at (0.75,3){};
\node[circle, scale=0.75, fill, red] (tid3) at (0.75,4.5){};
\draw[](tid1) -- (tid3);
\node[circle, scale=0.75, fill, red] (tid2) at (2.25,3){};
\draw[](tid0) -- (tid1);
\draw[](tid0) -- (tid2);
\end{tikzpicture}
\nodepart{three}
\footnotesize{3.25}
\nodepart{four}
\footnotesize{$50\:50$}
};
 & 
\\
};
\end{scope}
\begin{scope}[yshift=\leveltopIIIIIIIII cm]
\matrix (line9) [column sep=1cm] {
\node[draw=black, rectangle split,  rectangle split parts=4] (sn0x1888830){
\footnotesize{0.820923}
\nodepart{two}
\begin{tikzpicture}[scale=.2]
\node[circle, scale=0.75, fill] (tid0) at (0.75,1.5){};
\node[circle, scale=0.75, fill] (tid1) at (0.75,3){};
\node[circle, scale=0.75, fill, red] (tid2) at (0.75,4.5){};
\draw[](tid1) -- (tid2);
\draw[](tid0) -- (tid1);
\end{tikzpicture}
\nodepart{three}
\footnotesize{3}
\nodepart{four}
\footnotesize{$1$}
};
 & 
\node[draw=black, rectangle split,  rectangle split parts=4] (sn0x1889c10){
\footnotesize{0.179077}
\nodepart{two}
\begin{tikzpicture}[scale=.2]
\node[circle, scale=0.75, fill] (tid0) at (1.5,1.5){};
\node[circle, scale=0.75, fill, red] (tid1) at (0.75,3){};
\node[circle, scale=0.75, fill, red] (tid2) at (2.25,3){};
\draw[](tid0) -- (tid1);
\draw[](tid0) -- (tid2);
\end{tikzpicture}
\nodepart{three}
\footnotesize{2.5}
\nodepart{four}
\footnotesize{$1$}
};
 & 
\\
};
\end{scope}
\begin{scope}[yshift=\leveltopIIIIIIIIII cm]
\matrix (line10) [column sep=1cm] {
\node[draw=black, rectangle split,  rectangle split parts=4] (sn0x1887520){
\footnotesize{1}
\nodepart{two}
\begin{tikzpicture}[scale=.2]
\node[circle, scale=0.75, fill] (tid0) at (0.75,1.5){};
\node[circle, scale=0.75, fill, red] (tid1) at (0.75,3){};
\draw[](tid0) -- (tid1);
\end{tikzpicture}
\nodepart{three}
\footnotesize{2}
\nodepart{four}
\footnotesize{$1$}
};
 & 
\\
};
\end{scope}
\begin{scope}[yshift=\leveltopIIIIIIIIIII cm]
\matrix (line11) [column sep=1cm] {
\node[draw=black, rectangle split,  rectangle split parts=4] (sn0x1887450){
\footnotesize{1}
\nodepart{two}
\begin{tikzpicture}[scale=.2]
\node[circle, scale=0.75, fill, red] (tid0) at (0.75,1.5){};
\end{tikzpicture}
\nodepart{three}
\footnotesize{1}
\nodepart{four}
\footnotesize{$$}
};
 & 
\\
};
\end{scope}
\begin{scope}[yshift=\leveltopIIIIIIIIIIII cm]
\matrix (line12) [column sep=1cm] {
\\
};
\end{scope}
\draw (sn0x1891fa0.south) -- (sn0x188ec30.north);
\draw (sn0x1891fa0.south) -- (sn0x1891ba0.north);
\draw (sn0x1891fa0.south) -- (sn0x18906a0.north);
\draw (sn0x188ec30.south) -- (sn0x188e220.north);
\draw (sn0x188ec30.south) -- (sn0x188d8a0.north);
\draw (sn0x1891ba0.south) -- (sn0x188e220.north);
\draw (sn0x1891ba0.south) -- (sn0x188f0f0.north);
\draw (sn0x18906a0.south) -- (sn0x188d8a0.north);
\draw (sn0x18906a0.south) -- (sn0x188f0f0.north);
\draw (sn0x18906a0.south) -- (sn0x188ff90.north);
\draw (sn0x188e220.south) -- (sn0x188c370.north);
\draw (sn0x188e220.south) -- (sn0x188d690.north);
\draw (sn0x188d8a0.south) -- (sn0x188b180.north);
\draw (sn0x188d8a0.south) -- (sn0x188d690.north);
\draw (sn0x188d8a0.south) -- (sn0x188de90.north);
\draw (sn0x188f0f0.south) -- (sn0x188d690.north);
\draw (sn0x188f0f0.south) -- (sn0x188f020.north);
\draw (sn0x188ff90.south) -- (sn0x188de90.north);
\draw (sn0x188ff90.south) -- (sn0x188f020.north);
\draw (sn0x188c370.south) -- (sn0x1888fa0.north);
\draw (sn0x188c370.south) -- (sn0x188ae00.north);
\draw (sn0x188d690.south) -- (sn0x188ae00.north);
\draw (sn0x188d690.south) -- (sn0x188d460.north);
\draw (sn0x188b180.south) -- (sn0x1889a70.north);
\draw (sn0x188b180.south) -- (sn0x188ae00.north);
\draw (sn0x188b180.south) -- (sn0x188b630.north);
\draw (sn0x188de90.south) -- (sn0x188b630.north);
\draw (sn0x188de90.south) -- (sn0x188d460.north);
\draw (sn0x188f020.south) -- (sn0x188d460.north);
\draw (sn0x188f020.south) -- (sn0x188fd60.north);
\draw (sn0x1888fa0.south) -- (sn0x1888e50.north);
\draw (sn0x188ae00.south) -- (sn0x1888e50.north);
\draw (sn0x188ae00.south) -- (sn0x188acf0.north);
\draw (sn0x188d460.south) -- (sn0x188acf0.north);
\draw (sn0x188d460.south) -- (sn0x188c6a0.north);
\draw (sn0x1889a70.south) -- (sn0x1888e50.north);
\draw (sn0x1889a70.south) -- (sn0x1889230.north);
\draw (sn0x188b630.south) -- (sn0x1889230.north);
\draw (sn0x188b630.south) -- (sn0x188acf0.north);
\draw (sn0x188fd60.south) -- (sn0x188c6a0.north);
\draw (sn0x1888e50.south) -- (sn0x1888b50.north);
\draw (sn0x188acf0.south) -- (sn0x1888b50.north);
\draw (sn0x188acf0.south) -- (sn0x188a9f0.north);
\draw (sn0x188c6a0.south) -- (sn0x188a9f0.north);
\draw (sn0x188c6a0.south) -- (sn0x188c5b0.north);
\draw (sn0x1889230.south) -- (sn0x1888b50.north);
\draw (sn0x1888b50.south) -- (sn0x1888900.north);
\draw (sn0x188a9f0.south) -- (sn0x1888900.north);
\draw (sn0x188a9f0.south) -- (sn0x188a920.north);
\draw (sn0x188c5b0.south) -- (sn0x188a920.north);
\draw (sn0x1888900.south) -- (sn0x1888830.north);
\draw (sn0x188a920.south) -- (sn0x1888830.north);
\draw (sn0x188a920.south) -- (sn0x1889c10.north);
\draw (sn0x1888830.south) -- (sn0x1887520.north);
\draw (sn0x1889c10.south) -- (sn0x1887520.north);
\draw (sn0x1887520.south) -- (sn0x1887450.north);
\end{tikzpicture}

%%% Local Variables:
%%% TeX-master: "thesis/thesis.tex"
%%% End: 
\renewcommand{\leveltopI}{-15cm + \leveltop}
\renewcommand{\leveltopII}{-15cm + \leveltopI}
\renewcommand{\leveltopIII}{-15cm + \leveltopII}
\renewcommand{\leveltopIIII}{-15cm + \leveltopIII}
\renewcommand{\leveltopIIIII}{-15cm + \leveltopIIII}
\renewcommand{\leveltopIIIIII}{-15cm + \leveltopIIIII}
\renewcommand{\leveltopIIIIIII}{-15cm + \leveltopIIIIII}
\renewcommand{\leveltopIIIIIIII}{-15cm + \leveltopIIIIIII}
\renewcommand{\leveltopIIIIIIIII}{-15cm + \leveltopIIIIIIII}
\renewcommand{\leveltopIIIIIIIIII}{-15cm + \leveltopIIIIIIIII}
\renewcommand{\leveltopIIIIIIIIIII}{-15cm + \leveltopIIIIIIIIII}
\begin{tikzpicture}[scale=.2, anchor=south]
\begin{scope}[yshift=\leveltopI cm]
\matrix (line1) [column sep=1cm] {
\node[draw=black, rectangle split,  rectangle split parts=4] (sn0x1892300){
\footnotesize{1}
\nodepart{two}
\begin{tikzpicture}[scale=.2]
\node[circle, scale=0.75, fill] (tid0) at (3,1.5){};
\node[circle, scale=0.75, fill] (tid1) at (2.25,3){};
\node[circle, scale=0.75, fill] (tid3) at (2.25,4.5){};
\node[circle, scale=0.75, fill] (tid5) at (2.25,6){};
\node[circle, scale=0.75, fill] (tid7) at (1.5,7.5){};
\node[circle, scale=0.75, fill, red] (tid9) at (0.75,9){};
\node[circle, scale=0.75, fill, red] (tid10) at (2.25,9){};
\draw[](tid7) -- (tid9);
\draw[](tid7) -- (tid10);
\node[circle, scale=0.75, fill, red] (tid8) at (3.75,7.5){};
\draw[](tid5) -- (tid7);
\draw[](tid5) -- (tid8);
\draw[](tid3) -- (tid5);
\draw[](tid1) -- (tid3);
\node[circle, scale=0.75, fill] (tid2) at (5.25,3){};
\node[circle, scale=0.75, fill] (tid4) at (5.25,4.5){};
\node[circle, scale=0.75, fill] (tid6) at (5.25,6){};
\draw[](tid4) -- (tid6);
\draw[](tid2) -- (tid4);
\draw[](tid0) -- (tid1);
\draw[](tid0) -- (tid2);
\end{tikzpicture}
\nodepart{three}
\footnotesize{6.96798}
\nodepart{four}
\footnotesize{$33\:67$}
};
 & 
\\
};
\end{scope}
\begin{scope}[yshift=\leveltopII cm]
\matrix (line2) [column sep=1cm] {
\node[draw=black, rectangle split,  rectangle split parts=4] (sn0x1891ba0){
\footnotesize{0.333333}
\nodepart{two}
\begin{tikzpicture}[scale=.2]
\node[circle, scale=0.75, fill] (tid0) at (2.25,1.5){};
\node[circle, scale=0.75, fill] (tid1) at (1.5,3){};
\node[circle, scale=0.75, fill] (tid3) at (1.5,4.5){};
\node[circle, scale=0.75, fill] (tid5) at (1.5,6){};
\node[circle, scale=0.75, fill] (tid7) at (1.5,7.5){};
\node[circle, scale=0.75, fill, red] (tid8) at (0.75,9){};
\node[circle, scale=0.75, fill, red] (tid9) at (2.25,9){};
\draw[](tid7) -- (tid8);
\draw[](tid7) -- (tid9);
\draw[](tid5) -- (tid7);
\draw[](tid3) -- (tid5);
\draw[](tid1) -- (tid3);
\node[circle, scale=0.75, fill] (tid2) at (3.75,3){};
\node[circle, scale=0.75, fill] (tid4) at (3.75,4.5){};
\node[circle, scale=0.75, fill, red] (tid6) at (3.75,6){};
\draw[](tid4) -- (tid6);
\draw[](tid2) -- (tid4);
\draw[](tid0) -- (tid1);
\draw[](tid0) -- (tid2);
\end{tikzpicture}
\nodepart{three}
\footnotesize{6.77836}
\nodepart{four}
\footnotesize{$33\:67$}
};
 & 
\node[draw=black, rectangle split,  rectangle split parts=4] (sn0x18906a0){
\footnotesize{0.666667}
\nodepart{two}
\begin{tikzpicture}[scale=.2]
\node[circle, scale=0.75, fill] (tid0) at (2.25,1.5){};
\node[circle, scale=0.75, fill] (tid1) at (1.5,3){};
\node[circle, scale=0.75, fill] (tid3) at (1.5,4.5){};
\node[circle, scale=0.75, fill] (tid5) at (1.5,6){};
\node[circle, scale=0.75, fill] (tid7) at (0.75,7.5){};
\node[circle, scale=0.75, fill, red] (tid9) at (0.75,9){};
\draw[](tid7) -- (tid9);
\node[circle, scale=0.75, fill, red] (tid8) at (2.25,7.5){};
\draw[](tid5) -- (tid7);
\draw[](tid5) -- (tid8);
\draw[](tid3) -- (tid5);
\draw[](tid1) -- (tid3);
\node[circle, scale=0.75, fill] (tid2) at (3.75,3){};
\node[circle, scale=0.75, fill] (tid4) at (3.75,4.5){};
\node[circle, scale=0.75, fill, red] (tid6) at (3.75,6){};
\draw[](tid4) -- (tid6);
\draw[](tid2) -- (tid4);
\draw[](tid0) -- (tid1);
\draw[](tid0) -- (tid2);
\end{tikzpicture}
\nodepart{three}
\footnotesize{6.56279}
\nodepart{four}
\footnotesize{$33\:33\:33$}
};
 & 
\\
};
\end{scope}
\begin{scope}[yshift=\leveltopIII cm]
\matrix (line3) [column sep=1cm] {
\node[draw=black, rectangle split,  rectangle split parts=4] (sn0x188e220){
\footnotesize{0.111111}
\nodepart{two}
\begin{tikzpicture}[scale=.2]
\node[circle, scale=0.75, fill] (tid0) at (2.25,1.5){};
\node[circle, scale=0.75, fill] (tid1) at (1.5,3){};
\node[circle, scale=0.75, fill] (tid3) at (1.5,4.5){};
\node[circle, scale=0.75, fill] (tid5) at (1.5,6){};
\node[circle, scale=0.75, fill] (tid6) at (1.5,7.5){};
\node[circle, scale=0.75, fill, red] (tid7) at (0.75,9){};
\node[circle, scale=0.75, fill, red] (tid8) at (2.25,9){};
\draw[](tid6) -- (tid7);
\draw[](tid6) -- (tid8);
\draw[](tid5) -- (tid6);
\draw[](tid3) -- (tid5);
\draw[](tid1) -- (tid3);
\node[circle, scale=0.75, fill] (tid2) at (3.75,3){};
\node[circle, scale=0.75, fill, red] (tid4) at (3.75,4.5){};
\draw[](tid2) -- (tid4);
\draw[](tid0) -- (tid1);
\draw[](tid0) -- (tid2);
\end{tikzpicture}
\nodepart{three}
\footnotesize{6.60069}
\nodepart{four}
\footnotesize{$33\:67$}
};
 & 
\node[draw=black, rectangle split,  rectangle split parts=4] (sn0x188f0f0){
\footnotesize{0.444444}
\nodepart{two}
\begin{tikzpicture}[scale=.2]
\node[circle, scale=0.75, fill] (tid0) at (1.5,1.5){};
\node[circle, scale=0.75, fill] (tid1) at (0.75,3){};
\node[circle, scale=0.75, fill] (tid3) at (0.75,4.5){};
\node[circle, scale=0.75, fill] (tid5) at (0.75,6){};
\node[circle, scale=0.75, fill] (tid7) at (0.75,7.5){};
\node[circle, scale=0.75, fill, red] (tid8) at (0.75,9){};
\draw[](tid7) -- (tid8);
\draw[](tid5) -- (tid7);
\draw[](tid3) -- (tid5);
\draw[](tid1) -- (tid3);
\node[circle, scale=0.75, fill] (tid2) at (2.25,3){};
\node[circle, scale=0.75, fill] (tid4) at (2.25,4.5){};
\node[circle, scale=0.75, fill, red] (tid6) at (2.25,6){};
\draw[](tid4) -- (tid6);
\draw[](tid2) -- (tid4);
\draw[](tid0) -- (tid1);
\draw[](tid0) -- (tid2);
\end{tikzpicture}
\nodepart{three}
\footnotesize{6.36719}
\nodepart{four}
\footnotesize{$50\:50$}
};
 & 
\node[draw=black, rectangle split,  rectangle split parts=4] (sn0x188d8a0){
\footnotesize{0.222222}
\nodepart{two}
\begin{tikzpicture}[scale=.2]
\node[circle, scale=0.75, fill] (tid0) at (2.25,1.5){};
\node[circle, scale=0.75, fill] (tid1) at (1.5,3){};
\node[circle, scale=0.75, fill] (tid3) at (1.5,4.5){};
\node[circle, scale=0.75, fill] (tid5) at (1.5,6){};
\node[circle, scale=0.75, fill] (tid6) at (0.75,7.5){};
\node[circle, scale=0.75, fill, red] (tid8) at (0.75,9){};
\draw[](tid6) -- (tid8);
\node[circle, scale=0.75, fill, red] (tid7) at (2.25,7.5){};
\draw[](tid5) -- (tid6);
\draw[](tid5) -- (tid7);
\draw[](tid3) -- (tid5);
\draw[](tid1) -- (tid3);
\node[circle, scale=0.75, fill] (tid2) at (3.75,3){};
\node[circle, scale=0.75, fill, red] (tid4) at (3.75,4.5){};
\draw[](tid2) -- (tid4);
\draw[](tid0) -- (tid1);
\draw[](tid0) -- (tid2);
\end{tikzpicture}
\nodepart{three}
\footnotesize{6.36516}
\nodepart{four}
\footnotesize{$33\:33\:33$}
};
 & 
\node[draw=black, rectangle split,  rectangle split parts=4] (sn0x188ff90){
\footnotesize{0.222222}
\nodepart{two}
\begin{tikzpicture}[scale=.2]
\node[circle, scale=0.75, fill] (tid0) at (2.25,1.5){};
\node[circle, scale=0.75, fill] (tid1) at (1.5,3){};
\node[circle, scale=0.75, fill] (tid3) at (1.5,4.5){};
\node[circle, scale=0.75, fill] (tid5) at (1.5,6){};
\node[circle, scale=0.75, fill, red] (tid7) at (0.75,7.5){};
\node[circle, scale=0.75, fill, red] (tid8) at (2.25,7.5){};
\draw[](tid5) -- (tid7);
\draw[](tid5) -- (tid8);
\draw[](tid3) -- (tid5);
\draw[](tid1) -- (tid3);
\node[circle, scale=0.75, fill] (tid2) at (3.75,3){};
\node[circle, scale=0.75, fill] (tid4) at (3.75,4.5){};
\node[circle, scale=0.75, fill, red] (tid6) at (3.75,6){};
\draw[](tid4) -- (tid6);
\draw[](tid2) -- (tid4);
\draw[](tid0) -- (tid1);
\draw[](tid0) -- (tid2);
\end{tikzpicture}
\nodepart{three}
\footnotesize{5.95602}
\nodepart{four}
\footnotesize{$67\:33$}
};
 & 
\\
};
\end{scope}
\begin{scope}[yshift=\leveltopIIII cm]
\matrix (line4) [column sep=1cm] {
\node[draw=black, rectangle split,  rectangle split parts=4] (sn0x188c370){
\footnotesize{0.037037}
\nodepart{two}
\begin{tikzpicture}[scale=.2]
\node[circle, scale=0.75, fill] (tid0) at (2.25,1.5){};
\node[circle, scale=0.75, fill] (tid1) at (1.5,3){};
\node[circle, scale=0.75, fill] (tid3) at (1.5,4.5){};
\node[circle, scale=0.75, fill] (tid4) at (1.5,6){};
\node[circle, scale=0.75, fill] (tid5) at (1.5,7.5){};
\node[circle, scale=0.75, fill, red] (tid6) at (0.75,9){};
\node[circle, scale=0.75, fill, red] (tid7) at (2.25,9){};
\draw[](tid5) -- (tid6);
\draw[](tid5) -- (tid7);
\draw[](tid4) -- (tid5);
\draw[](tid3) -- (tid4);
\draw[](tid1) -- (tid3);
\node[circle, scale=0.75, fill, red] (tid2) at (3.75,3){};
\draw[](tid0) -- (tid1);
\draw[](tid0) -- (tid2);
\end{tikzpicture}
\nodepart{three}
\footnotesize{6.52083}
\nodepart{four}
\footnotesize{$33\:67$}
};
 & 
\node[draw=black, rectangle split,  rectangle split parts=4] (sn0x188d690){
\footnotesize{0.37037}
\nodepart{two}
\begin{tikzpicture}[scale=.2]
\node[circle, scale=0.75, fill] (tid0) at (1.5,1.5){};
\node[circle, scale=0.75, fill] (tid1) at (0.75,3){};
\node[circle, scale=0.75, fill] (tid3) at (0.75,4.5){};
\node[circle, scale=0.75, fill] (tid5) at (0.75,6){};
\node[circle, scale=0.75, fill] (tid6) at (0.75,7.5){};
\node[circle, scale=0.75, fill, red] (tid7) at (0.75,9){};
\draw[](tid6) -- (tid7);
\draw[](tid5) -- (tid6);
\draw[](tid3) -- (tid5);
\draw[](tid1) -- (tid3);
\node[circle, scale=0.75, fill] (tid2) at (2.25,3){};
\node[circle, scale=0.75, fill, red] (tid4) at (2.25,4.5){};
\draw[](tid2) -- (tid4);
\draw[](tid0) -- (tid1);
\draw[](tid0) -- (tid2);
\end{tikzpicture}
\nodepart{three}
\footnotesize{6.14062}
\nodepart{four}
\footnotesize{$50\:50$}
};
 & 
\node[draw=black, rectangle split,  rectangle split parts=4] (sn0x188f020){
\footnotesize{0.37037}
\nodepart{two}
\begin{tikzpicture}[scale=.2]
\node[circle, scale=0.75, fill] (tid0) at (1.5,1.5){};
\node[circle, scale=0.75, fill] (tid1) at (0.75,3){};
\node[circle, scale=0.75, fill] (tid3) at (0.75,4.5){};
\node[circle, scale=0.75, fill] (tid5) at (0.75,6){};
\node[circle, scale=0.75, fill, red] (tid7) at (0.75,7.5){};
\draw[](tid5) -- (tid7);
\draw[](tid3) -- (tid5);
\draw[](tid1) -- (tid3);
\node[circle, scale=0.75, fill] (tid2) at (2.25,3){};
\node[circle, scale=0.75, fill] (tid4) at (2.25,4.5){};
\node[circle, scale=0.75, fill, red] (tid6) at (2.25,6){};
\draw[](tid4) -- (tid6);
\draw[](tid2) -- (tid4);
\draw[](tid0) -- (tid1);
\draw[](tid0) -- (tid2);
\end{tikzpicture}
\nodepart{three}
\footnotesize{5.59375}
\nodepart{four}
\footnotesize{$50\:50$}
};
 & 
\node[draw=black, rectangle split,  rectangle split parts=4] (sn0x188b180){
\footnotesize{0.0740741}
\nodepart{two}
\begin{tikzpicture}[scale=.2]
\node[circle, scale=0.75, fill] (tid0) at (2.25,1.5){};
\node[circle, scale=0.75, fill] (tid1) at (1.5,3){};
\node[circle, scale=0.75, fill] (tid3) at (1.5,4.5){};
\node[circle, scale=0.75, fill] (tid4) at (1.5,6){};
\node[circle, scale=0.75, fill] (tid5) at (0.75,7.5){};
\node[circle, scale=0.75, fill, red] (tid7) at (0.75,9){};
\draw[](tid5) -- (tid7);
\node[circle, scale=0.75, fill, red] (tid6) at (2.25,7.5){};
\draw[](tid4) -- (tid5);
\draw[](tid4) -- (tid6);
\draw[](tid3) -- (tid4);
\draw[](tid1) -- (tid3);
\node[circle, scale=0.75, fill, red] (tid2) at (3.75,3){};
\draw[](tid0) -- (tid1);
\draw[](tid0) -- (tid2);
\end{tikzpicture}
\nodepart{three}
\footnotesize{6.27431}
\nodepart{four}
\footnotesize{$33\:33\:33$}
};
 & 
\node[draw=black, rectangle split,  rectangle split parts=4] (sn0x188de90){
\footnotesize{0.148148}
\nodepart{two}
\begin{tikzpicture}[scale=.2]
\node[circle, scale=0.75, fill] (tid0) at (2.25,1.5){};
\node[circle, scale=0.75, fill] (tid1) at (1.5,3){};
\node[circle, scale=0.75, fill] (tid3) at (1.5,4.5){};
\node[circle, scale=0.75, fill] (tid5) at (1.5,6){};
\node[circle, scale=0.75, fill, red] (tid6) at (0.75,7.5){};
\node[circle, scale=0.75, fill, red] (tid7) at (2.25,7.5){};
\draw[](tid5) -- (tid6);
\draw[](tid5) -- (tid7);
\draw[](tid3) -- (tid5);
\draw[](tid1) -- (tid3);
\node[circle, scale=0.75, fill] (tid2) at (3.75,3){};
\node[circle, scale=0.75, fill, red] (tid4) at (3.75,4.5){};
\draw[](tid2) -- (tid4);
\draw[](tid0) -- (tid1);
\draw[](tid0) -- (tid2);
\end{tikzpicture}
\nodepart{three}
\footnotesize{5.68056}
\nodepart{four}
\footnotesize{$67\:33$}
};
 & 
\\
};
\end{scope}
\begin{scope}[yshift=\leveltopIIIII cm]
\matrix (line5) [column sep=1cm] {
\node[draw=black, rectangle split,  rectangle split parts=4] (sn0x1888fa0){
\footnotesize{0.0123457}
\nodepart{two}
\begin{tikzpicture}[scale=.2]
\node[circle, scale=0.75, fill] (tid0) at (1.5,1.5){};
\node[circle, scale=0.75, fill] (tid1) at (1.5,3){};
\node[circle, scale=0.75, fill] (tid2) at (1.5,4.5){};
\node[circle, scale=0.75, fill] (tid3) at (1.5,6){};
\node[circle, scale=0.75, fill] (tid4) at (1.5,7.5){};
\node[circle, scale=0.75, fill, red] (tid5) at (0.75,9){};
\node[circle, scale=0.75, fill, red] (tid6) at (2.25,9){};
\draw[](tid4) -- (tid5);
\draw[](tid4) -- (tid6);
\draw[](tid3) -- (tid4);
\draw[](tid2) -- (tid3);
\draw[](tid1) -- (tid2);
\draw[](tid0) -- (tid1);
\end{tikzpicture}
\nodepart{three}
\footnotesize{6.5}
\nodepart{four}
\footnotesize{$1$}
};
 & 
\node[draw=black, rectangle split,  rectangle split parts=4] (sn0x188ae00){
\footnotesize{0.234568}
\nodepart{two}
\begin{tikzpicture}[scale=.2]
\node[circle, scale=0.75, fill] (tid0) at (1.5,1.5){};
\node[circle, scale=0.75, fill] (tid1) at (0.75,3){};
\node[circle, scale=0.75, fill] (tid3) at (0.75,4.5){};
\node[circle, scale=0.75, fill] (tid4) at (0.75,6){};
\node[circle, scale=0.75, fill] (tid5) at (0.75,7.5){};
\node[circle, scale=0.75, fill, red] (tid6) at (0.75,9){};
\draw[](tid5) -- (tid6);
\draw[](tid4) -- (tid5);
\draw[](tid3) -- (tid4);
\draw[](tid1) -- (tid3);
\node[circle, scale=0.75, fill, red] (tid2) at (2.25,3){};
\draw[](tid0) -- (tid1);
\draw[](tid0) -- (tid2);
\end{tikzpicture}
\nodepart{three}
\footnotesize{6.03125}
\nodepart{four}
\footnotesize{$50\:50$}
};
 & 
\node[draw=black, rectangle split,  rectangle split parts=4] (sn0x188d460){
\footnotesize{0.469136}
\nodepart{two}
\begin{tikzpicture}[scale=.2]
\node[circle, scale=0.75, fill] (tid0) at (1.5,1.5){};
\node[circle, scale=0.75, fill] (tid1) at (0.75,3){};
\node[circle, scale=0.75, fill] (tid3) at (0.75,4.5){};
\node[circle, scale=0.75, fill] (tid5) at (0.75,6){};
\node[circle, scale=0.75, fill, red] (tid6) at (0.75,7.5){};
\draw[](tid5) -- (tid6);
\draw[](tid3) -- (tid5);
\draw[](tid1) -- (tid3);
\node[circle, scale=0.75, fill] (tid2) at (2.25,3){};
\node[circle, scale=0.75, fill, red] (tid4) at (2.25,4.5){};
\draw[](tid2) -- (tid4);
\draw[](tid0) -- (tid1);
\draw[](tid0) -- (tid2);
\end{tikzpicture}
\nodepart{three}
\footnotesize{5.25}
\nodepart{four}
\footnotesize{$50\:50$}
};
 & 
\node[draw=black, rectangle split,  rectangle split parts=4] (sn0x188fd60){
\footnotesize{0.185185}
\nodepart{two}
\begin{tikzpicture}[scale=.2]
\node[circle, scale=0.75, fill] (tid0) at (1.5,1.5){};
\node[circle, scale=0.75, fill] (tid1) at (0.75,3){};
\node[circle, scale=0.75, fill] (tid3) at (0.75,4.5){};
\node[circle, scale=0.75, fill, red] (tid5) at (0.75,6){};
\draw[](tid3) -- (tid5);
\draw[](tid1) -- (tid3);
\node[circle, scale=0.75, fill] (tid2) at (2.25,3){};
\node[circle, scale=0.75, fill] (tid4) at (2.25,4.5){};
\node[circle, scale=0.75, fill, red] (tid6) at (2.25,6){};
\draw[](tid4) -- (tid6);
\draw[](tid2) -- (tid4);
\draw[](tid0) -- (tid1);
\draw[](tid0) -- (tid2);
\end{tikzpicture}
\nodepart{three}
\footnotesize{4.9375}
\nodepart{four}
\footnotesize{$1$}
};
 & 
\node[draw=black, rectangle split,  rectangle split parts=4] (sn0x1889a70){
\footnotesize{0.0246914}
\nodepart{two}
\begin{tikzpicture}[scale=.2]
\node[circle, scale=0.75, fill] (tid0) at (1.5,1.5){};
\node[circle, scale=0.75, fill] (tid1) at (1.5,3){};
\node[circle, scale=0.75, fill] (tid2) at (1.5,4.5){};
\node[circle, scale=0.75, fill] (tid3) at (1.5,6){};
\node[circle, scale=0.75, fill] (tid4) at (0.75,7.5){};
\node[circle, scale=0.75, fill, red] (tid6) at (0.75,9){};
\draw[](tid4) -- (tid6);
\node[circle, scale=0.75, fill, red] (tid5) at (2.25,7.5){};
\draw[](tid3) -- (tid4);
\draw[](tid3) -- (tid5);
\draw[](tid2) -- (tid3);
\draw[](tid1) -- (tid2);
\draw[](tid0) -- (tid1);
\end{tikzpicture}
\nodepart{three}
\footnotesize{6.25}
\nodepart{four}
\footnotesize{$50\:50$}
};
 & 
\node[draw=black, rectangle split,  rectangle split parts=4] (sn0x188b630){
\footnotesize{0.0740741}
\nodepart{two}
\begin{tikzpicture}[scale=.2]
\node[circle, scale=0.75, fill] (tid0) at (2.25,1.5){};
\node[circle, scale=0.75, fill] (tid1) at (1.5,3){};
\node[circle, scale=0.75, fill] (tid3) at (1.5,4.5){};
\node[circle, scale=0.75, fill] (tid4) at (1.5,6){};
\node[circle, scale=0.75, fill, red] (tid5) at (0.75,7.5){};
\node[circle, scale=0.75, fill, red] (tid6) at (2.25,7.5){};
\draw[](tid4) -- (tid5);
\draw[](tid4) -- (tid6);
\draw[](tid3) -- (tid4);
\draw[](tid1) -- (tid3);
\node[circle, scale=0.75, fill, red] (tid2) at (3.75,3){};
\draw[](tid0) -- (tid1);
\draw[](tid0) -- (tid2);
\end{tikzpicture}
\nodepart{three}
\footnotesize{5.54167}
\nodepart{four}
\footnotesize{$67\:33$}
};
 & 
\\
};
\end{scope}
\begin{scope}[yshift=\leveltopIIIIII cm]
\matrix (line6) [column sep=1cm] {
\node[draw=black, rectangle split,  rectangle split parts=4] (sn0x1888e50){
\footnotesize{0.141975}
\nodepart{two}
\begin{tikzpicture}[scale=.2]
\node[circle, scale=0.75, fill] (tid0) at (0.75,1.5){};
\node[circle, scale=0.75, fill] (tid1) at (0.75,3){};
\node[circle, scale=0.75, fill] (tid2) at (0.75,4.5){};
\node[circle, scale=0.75, fill] (tid3) at (0.75,6){};
\node[circle, scale=0.75, fill] (tid4) at (0.75,7.5){};
\node[circle, scale=0.75, fill, red] (tid5) at (0.75,9){};
\draw[](tid4) -- (tid5);
\draw[](tid3) -- (tid4);
\draw[](tid2) -- (tid3);
\draw[](tid1) -- (tid2);
\draw[](tid0) -- (tid1);
\end{tikzpicture}
\nodepart{three}
\footnotesize{6}
\nodepart{four}
\footnotesize{$1$}
};
 & 
\node[draw=black, rectangle split,  rectangle split parts=4] (sn0x188acf0){
\footnotesize{0.401235}
\nodepart{two}
\begin{tikzpicture}[scale=.2]
\node[circle, scale=0.75, fill] (tid0) at (1.5,1.5){};
\node[circle, scale=0.75, fill] (tid1) at (0.75,3){};
\node[circle, scale=0.75, fill] (tid3) at (0.75,4.5){};
\node[circle, scale=0.75, fill] (tid4) at (0.75,6){};
\node[circle, scale=0.75, fill, red] (tid5) at (0.75,7.5){};
\draw[](tid4) -- (tid5);
\draw[](tid3) -- (tid4);
\draw[](tid1) -- (tid3);
\node[circle, scale=0.75, fill, red] (tid2) at (2.25,3){};
\draw[](tid0) -- (tid1);
\draw[](tid0) -- (tid2);
\end{tikzpicture}
\nodepart{three}
\footnotesize{5.0625}
\nodepart{four}
\footnotesize{$50\:50$}
};
 & 
\node[draw=black, rectangle split,  rectangle split parts=4] (sn0x188c6a0){
\footnotesize{0.419753}
\nodepart{two}
\begin{tikzpicture}[scale=.2]
\node[circle, scale=0.75, fill] (tid0) at (1.5,1.5){};
\node[circle, scale=0.75, fill] (tid1) at (0.75,3){};
\node[circle, scale=0.75, fill] (tid3) at (0.75,4.5){};
\node[circle, scale=0.75, fill, red] (tid5) at (0.75,6){};
\draw[](tid3) -- (tid5);
\draw[](tid1) -- (tid3);
\node[circle, scale=0.75, fill] (tid2) at (2.25,3){};
\node[circle, scale=0.75, fill, red] (tid4) at (2.25,4.5){};
\draw[](tid2) -- (tid4);
\draw[](tid0) -- (tid1);
\draw[](tid0) -- (tid2);
\end{tikzpicture}
\nodepart{three}
\footnotesize{4.4375}
\nodepart{four}
\footnotesize{$50\:50$}
};
 & 
\node[draw=black, rectangle split,  rectangle split parts=4] (sn0x1889230){
\footnotesize{0.037037}
\nodepart{two}
\begin{tikzpicture}[scale=.2]
\node[circle, scale=0.75, fill] (tid0) at (1.5,1.5){};
\node[circle, scale=0.75, fill] (tid1) at (1.5,3){};
\node[circle, scale=0.75, fill] (tid2) at (1.5,4.5){};
\node[circle, scale=0.75, fill] (tid3) at (1.5,6){};
\node[circle, scale=0.75, fill, red] (tid4) at (0.75,7.5){};
\node[circle, scale=0.75, fill, red] (tid5) at (2.25,7.5){};
\draw[](tid3) -- (tid4);
\draw[](tid3) -- (tid5);
\draw[](tid2) -- (tid3);
\draw[](tid1) -- (tid2);
\draw[](tid0) -- (tid1);
\end{tikzpicture}
\nodepart{three}
\footnotesize{5.5}
\nodepart{four}
\footnotesize{$1$}
};
 & 
\\
};
\end{scope}
\begin{scope}[yshift=\leveltopIIIIIII cm]
\matrix (line7) [column sep=1cm] {
\node[draw=black, rectangle split,  rectangle split parts=4] (sn0x1888b50){
\footnotesize{0.37963}
\nodepart{two}
\begin{tikzpicture}[scale=.2]
\node[circle, scale=0.75, fill] (tid0) at (0.75,1.5){};
\node[circle, scale=0.75, fill] (tid1) at (0.75,3){};
\node[circle, scale=0.75, fill] (tid2) at (0.75,4.5){};
\node[circle, scale=0.75, fill] (tid3) at (0.75,6){};
\node[circle, scale=0.75, fill, red] (tid4) at (0.75,7.5){};
\draw[](tid3) -- (tid4);
\draw[](tid2) -- (tid3);
\draw[](tid1) -- (tid2);
\draw[](tid0) -- (tid1);
\end{tikzpicture}
\nodepart{three}
\footnotesize{5}
\nodepart{four}
\footnotesize{$1$}
};
 & 
\node[draw=black, rectangle split,  rectangle split parts=4] (sn0x188a9f0){
\footnotesize{0.410494}
\nodepart{two}
\begin{tikzpicture}[scale=.2]
\node[circle, scale=0.75, fill] (tid0) at (1.5,1.5){};
\node[circle, scale=0.75, fill] (tid1) at (0.75,3){};
\node[circle, scale=0.75, fill] (tid3) at (0.75,4.5){};
\node[circle, scale=0.75, fill, red] (tid4) at (0.75,6){};
\draw[](tid3) -- (tid4);
\draw[](tid1) -- (tid3);
\node[circle, scale=0.75, fill, red] (tid2) at (2.25,3){};
\draw[](tid0) -- (tid1);
\draw[](tid0) -- (tid2);
\end{tikzpicture}
\nodepart{three}
\footnotesize{4.125}
\nodepart{four}
\footnotesize{$50\:50$}
};
 & 
\node[draw=black, rectangle split,  rectangle split parts=4] (sn0x188c5b0){
\footnotesize{0.209877}
\nodepart{two}
\begin{tikzpicture}[scale=.2]
\node[circle, scale=0.75, fill] (tid0) at (1.5,1.5){};
\node[circle, scale=0.75, fill] (tid1) at (0.75,3){};
\node[circle, scale=0.75, fill, red] (tid3) at (0.75,4.5){};
\draw[](tid1) -- (tid3);
\node[circle, scale=0.75, fill] (tid2) at (2.25,3){};
\node[circle, scale=0.75, fill, red] (tid4) at (2.25,4.5){};
\draw[](tid2) -- (tid4);
\draw[](tid0) -- (tid1);
\draw[](tid0) -- (tid2);
\end{tikzpicture}
\nodepart{three}
\footnotesize{3.75}
\nodepart{four}
\footnotesize{$1$}
};
 & 
\\
};
\end{scope}
\begin{scope}[yshift=\leveltopIIIIIIII cm]
\matrix (line8) [column sep=1cm] {
\node[draw=black, rectangle split,  rectangle split parts=4] (sn0x1888900){
\footnotesize{0.584877}
\nodepart{two}
\begin{tikzpicture}[scale=.2]
\node[circle, scale=0.75, fill] (tid0) at (0.75,1.5){};
\node[circle, scale=0.75, fill] (tid1) at (0.75,3){};
\node[circle, scale=0.75, fill] (tid2) at (0.75,4.5){};
\node[circle, scale=0.75, fill, red] (tid3) at (0.75,6){};
\draw[](tid2) -- (tid3);
\draw[](tid1) -- (tid2);
\draw[](tid0) -- (tid1);
\end{tikzpicture}
\nodepart{three}
\footnotesize{4}
\nodepart{four}
\footnotesize{$1$}
};
 & 
\node[draw=black, rectangle split,  rectangle split parts=4] (sn0x188a920){
\footnotesize{0.415124}
\nodepart{two}
\begin{tikzpicture}[scale=.2]
\node[circle, scale=0.75, fill] (tid0) at (1.5,1.5){};
\node[circle, scale=0.75, fill] (tid1) at (0.75,3){};
\node[circle, scale=0.75, fill, red] (tid3) at (0.75,4.5){};
\draw[](tid1) -- (tid3);
\node[circle, scale=0.75, fill, red] (tid2) at (2.25,3){};
\draw[](tid0) -- (tid1);
\draw[](tid0) -- (tid2);
\end{tikzpicture}
\nodepart{three}
\footnotesize{3.25}
\nodepart{four}
\footnotesize{$50\:50$}
};
 & 
\\
};
\end{scope}
\begin{scope}[yshift=\leveltopIIIIIIIII cm]
\matrix (line9) [column sep=1cm] {
\node[draw=black, rectangle split,  rectangle split parts=4] (sn0x1888830){
\footnotesize{0.792438}
\nodepart{two}
\begin{tikzpicture}[scale=.2]
\node[circle, scale=0.75, fill] (tid0) at (0.75,1.5){};
\node[circle, scale=0.75, fill] (tid1) at (0.75,3){};
\node[circle, scale=0.75, fill, red] (tid2) at (0.75,4.5){};
\draw[](tid1) -- (tid2);
\draw[](tid0) -- (tid1);
\end{tikzpicture}
\nodepart{three}
\footnotesize{3}
\nodepart{four}
\footnotesize{$1$}
};
 & 
\node[draw=black, rectangle split,  rectangle split parts=4] (sn0x1889c10){
\footnotesize{0.207562}
\nodepart{two}
\begin{tikzpicture}[scale=.2]
\node[circle, scale=0.75, fill] (tid0) at (1.5,1.5){};
\node[circle, scale=0.75, fill, red] (tid1) at (0.75,3){};
\node[circle, scale=0.75, fill, red] (tid2) at (2.25,3){};
\draw[](tid0) -- (tid1);
\draw[](tid0) -- (tid2);
\end{tikzpicture}
\nodepart{three}
\footnotesize{2.5}
\nodepart{four}
\footnotesize{$1$}
};
 & 
\\
};
\end{scope}
\begin{scope}[yshift=\leveltopIIIIIIIIII cm]
\matrix (line10) [column sep=1cm] {
\node[draw=black, rectangle split,  rectangle split parts=4] (sn0x1887520){
\footnotesize{1}
\nodepart{two}
\begin{tikzpicture}[scale=.2]
\node[circle, scale=0.75, fill] (tid0) at (0.75,1.5){};
\node[circle, scale=0.75, fill, red] (tid1) at (0.75,3){};
\draw[](tid0) -- (tid1);
\end{tikzpicture}
\nodepart{three}
\footnotesize{2}
\nodepart{four}
\footnotesize{$1$}
};
 & 
\\
};
\end{scope}
\begin{scope}[yshift=\leveltopIIIIIIIIIII cm]
\matrix (line11) [column sep=1cm] {
\node[draw=black, rectangle split,  rectangle split parts=4] (sn0x1887450){
\footnotesize{1}
\nodepart{two}
\begin{tikzpicture}[scale=.2]
\node[circle, scale=0.75, fill, red] (tid0) at (0.75,1.5){};
\end{tikzpicture}
\nodepart{three}
\footnotesize{1}
\nodepart{four}
\footnotesize{$$}
};
 & 
\\
};
\end{scope}
\begin{scope}[yshift=\leveltopIIIIIIIIIIII cm]
\matrix (line12) [column sep=1cm] {
\\
};
\end{scope}
\draw (sn0x1892300.south) -- (sn0x1891ba0.north);
\draw (sn0x1892300.south) -- (sn0x18906a0.north);
\draw (sn0x1891ba0.south) -- (sn0x188e220.north);
\draw (sn0x1891ba0.south) -- (sn0x188f0f0.north);
\draw (sn0x18906a0.south) -- (sn0x188d8a0.north);
\draw (sn0x18906a0.south) -- (sn0x188f0f0.north);
\draw (sn0x18906a0.south) -- (sn0x188ff90.north);
\draw (sn0x188e220.south) -- (sn0x188c370.north);
\draw (sn0x188e220.south) -- (sn0x188d690.north);
\draw (sn0x188f0f0.south) -- (sn0x188d690.north);
\draw (sn0x188f0f0.south) -- (sn0x188f020.north);
\draw (sn0x188d8a0.south) -- (sn0x188b180.north);
\draw (sn0x188d8a0.south) -- (sn0x188d690.north);
\draw (sn0x188d8a0.south) -- (sn0x188de90.north);
\draw (sn0x188ff90.south) -- (sn0x188de90.north);
\draw (sn0x188ff90.south) -- (sn0x188f020.north);
\draw (sn0x188c370.south) -- (sn0x1888fa0.north);
\draw (sn0x188c370.south) -- (sn0x188ae00.north);
\draw (sn0x188d690.south) -- (sn0x188ae00.north);
\draw (sn0x188d690.south) -- (sn0x188d460.north);
\draw (sn0x188f020.south) -- (sn0x188d460.north);
\draw (sn0x188f020.south) -- (sn0x188fd60.north);
\draw (sn0x188b180.south) -- (sn0x1889a70.north);
\draw (sn0x188b180.south) -- (sn0x188ae00.north);
\draw (sn0x188b180.south) -- (sn0x188b630.north);
\draw (sn0x188de90.south) -- (sn0x188b630.north);
\draw (sn0x188de90.south) -- (sn0x188d460.north);
\draw (sn0x1888fa0.south) -- (sn0x1888e50.north);
\draw (sn0x188ae00.south) -- (sn0x1888e50.north);
\draw (sn0x188ae00.south) -- (sn0x188acf0.north);
\draw (sn0x188d460.south) -- (sn0x188acf0.north);
\draw (sn0x188d460.south) -- (sn0x188c6a0.north);
\draw (sn0x188fd60.south) -- (sn0x188c6a0.north);
\draw (sn0x1889a70.south) -- (sn0x1888e50.north);
\draw (sn0x1889a70.south) -- (sn0x1889230.north);
\draw (sn0x188b630.south) -- (sn0x1889230.north);
\draw (sn0x188b630.south) -- (sn0x188acf0.north);
\draw (sn0x1888e50.south) -- (sn0x1888b50.north);
\draw (sn0x188acf0.south) -- (sn0x1888b50.north);
\draw (sn0x188acf0.south) -- (sn0x188a9f0.north);
\draw (sn0x188c6a0.south) -- (sn0x188a9f0.north);
\draw (sn0x188c6a0.south) -- (sn0x188c5b0.north);
\draw (sn0x1889230.south) -- (sn0x1888b50.north);
\draw (sn0x1888b50.south) -- (sn0x1888900.north);
\draw (sn0x188a9f0.south) -- (sn0x1888900.north);
\draw (sn0x188a9f0.south) -- (sn0x188a920.north);
\draw (sn0x188c5b0.south) -- (sn0x188a920.north);
\draw (sn0x1888900.south) -- (sn0x1888830.north);
\draw (sn0x188a920.south) -- (sn0x1888830.north);
\draw (sn0x188a920.south) -- (sn0x1889c10.north);
\draw (sn0x1888830.south) -- (sn0x1887520.north);
\draw (sn0x1889c10.south) -- (sn0x1887520.north);
\draw (sn0x1887520.south) -- (sn0x1887450.north);
\end{tikzpicture}

%%% Local Variables:
%%% TeX-master: "thesis/thesis.tex"
%%% End: 

  \end{minipage}
}
\frame{
  \begin{minipage}{.25\textwidth}
    \subsection{P3-HLF not optimal  for 0012446788}
    \renewcommand{\leveltopI}{-19cm + \leveltop}
\renewcommand{\leveltopII}{-19cm + \leveltopI}
\renewcommand{\leveltopIII}{-19cm + \leveltopII}
\renewcommand{\leveltopIIII}{-19cm + \leveltopIII}
\renewcommand{\leveltopIIIII}{-19cm + \leveltopIIII}
\renewcommand{\leveltopIIIIII}{-19cm + \leveltopIIIII}
\renewcommand{\leveltopIIIIIII}{-19cm + \leveltopIIIIII}
\renewcommand{\leveltopIIIIIIII}{-19cm + \leveltopIIIIIII}
\renewcommand{\leveltopIIIIIIIII}{-19cm + \leveltopIIIIIIII}
\renewcommand{\leveltopIIIIIIIIII}{-19cm + \leveltopIIIIIIIII}
\renewcommand{\leveltopIIIIIIIIIII}{-19cm + \leveltopIIIIIIIIII}
\begin{tikzpicture}[scale=.2, anchor=south]
\begin{scope}[yshift=\leveltopI cm]
\matrix (line1) [column sep=.5cm] {
\node[draw=black, rectangle split,  rectangle split parts=4] (sn0x983680){
\footnotesize{100}
\nodepart{two}
\begin{tikzpicture}[scale=.2]
\node[circle, scale=0.75, fill] (tid0) at (3,1.5){};
\node[circle, scale=0.75, fill] (tid1) at (2.25,3){};
\node[circle, scale=0.75, fill] (tid3) at (2.25,4.5){};
\node[circle, scale=0.75, fill] (tid5) at (1.5,6){};
\node[circle, scale=0.75, fill] (tid7) at (1.5,7.5){};
\node[circle, scale=0.75, fill] (tid8) at (1.5,9){};
\node[circle, scale=0.75, fill, red] (tid9) at (0.75,10.5){};
\node[circle, scale=0.75, fill, red] (tid10) at (2.25,10.5){};
\draw[](tid8) -- (tid9);
\draw[](tid8) -- (tid10);
\draw[](tid7) -- (tid8);
\draw[](tid5) -- (tid7);
\node[circle, scale=0.75, fill, red] (tid6) at (3.75,6){};
\draw[](tid3) -- (tid5);
\draw[](tid3) -- (tid6);
\draw[](tid1) -- (tid3);
\node[circle, scale=0.75, fill] (tid2) at (5.25,3){};
\node[circle, scale=0.75, fill] (tid4) at (5.25,4.5){};
\draw[](tid2) -- (tid4);
\draw[](tid0) -- (tid1);
\draw[](tid0) -- (tid2);
\end{tikzpicture}
\nodepart{three}
\footnotesize{7.60833}
\nodepart{four}
\footnotesize{$33\:67$}
};
 & 
\\
};
\end{scope}
\begin{scope}[yshift=\leveltopII cm]
\matrix (line2) [column sep=.5cm] {
\node[draw=black, rectangle split,  rectangle split parts=4] (sn0x984240){
\footnotesize{33.3333}
\nodepart{two}
\begin{tikzpicture}[scale=.2]
\node[circle, scale=0.75, fill] (tid0) at (2.25,1.5){};
\node[circle, scale=0.75, fill] (tid1) at (1.5,3){};
\node[circle, scale=0.75, fill] (tid3) at (1.5,4.5){};
\node[circle, scale=0.75, fill] (tid5) at (1.5,6){};
\node[circle, scale=0.75, fill] (tid6) at (1.5,7.5){};
\node[circle, scale=0.75, fill] (tid7) at (1.5,9){};
\node[circle, scale=0.75, fill, red] (tid8) at (0.75,10.5){};
\node[circle, scale=0.75, fill, red] (tid9) at (2.25,10.5){};
\draw[](tid7) -- (tid8);
\draw[](tid7) -- (tid9);
\draw[](tid6) -- (tid7);
\draw[](tid5) -- (tid6);
\draw[](tid3) -- (tid5);
\draw[](tid1) -- (tid3);
\node[circle, scale=0.75, fill] (tid2) at (3.75,3){};
\node[circle, scale=0.75, fill, red] (tid4) at (3.75,4.5){};
\draw[](tid2) -- (tid4);
\draw[](tid0) -- (tid1);
\draw[](tid0) -- (tid2);
\end{tikzpicture}
\nodepart{three}
\footnotesize{7.55556}
\nodepart{four}
\footnotesize{$33\:67$}
};
 & 
\node[draw=black, rectangle split,  rectangle split parts=4] (sn0x984830){
\footnotesize{66.6667}
\nodepart{two}
\begin{tikzpicture}[scale=.2]
\node[circle, scale=0.75, fill] (tid0) at (2.25,1.5){};
\node[circle, scale=0.75, fill] (tid1) at (1.5,3){};
\node[circle, scale=0.75, fill] (tid3) at (1.5,4.5){};
\node[circle, scale=0.75, fill] (tid5) at (0.75,6){};
\node[circle, scale=0.75, fill] (tid7) at (0.75,7.5){};
\node[circle, scale=0.75, fill] (tid8) at (0.75,9){};
\node[circle, scale=0.75, fill, red] (tid9) at (0.75,10.5){};
\draw[](tid8) -- (tid9);
\draw[](tid7) -- (tid8);
\draw[](tid5) -- (tid7);
\node[circle, scale=0.75, fill, red] (tid6) at (2.25,6){};
\draw[](tid3) -- (tid5);
\draw[](tid3) -- (tid6);
\draw[](tid1) -- (tid3);
\node[circle, scale=0.75, fill] (tid2) at (3.75,3){};
\node[circle, scale=0.75, fill, red] (tid4) at (3.75,4.5){};
\draw[](tid2) -- (tid4);
\draw[](tid0) -- (tid1);
\draw[](tid0) -- (tid2);
\end{tikzpicture}
\nodepart{three}
\footnotesize{7.13471}
\nodepart{four}
\footnotesize{$33\:33\:33$}
};
 & 
\\
};
\end{scope}
\begin{scope}[yshift=\leveltopIII cm]
\matrix (line3) [column sep=.5cm] {
\node[draw=black, rectangle split,  rectangle split parts=4] (sn0x984fb0){
\footnotesize{11.1111}
\nodepart{two}
\begin{tikzpicture}[scale=.2]
\node[circle, scale=0.75, fill] (tid0) at (2.25,1.5){};
\node[circle, scale=0.75, fill] (tid1) at (1.5,3){};
\node[circle, scale=0.75, fill] (tid3) at (1.5,4.5){};
\node[circle, scale=0.75, fill] (tid4) at (1.5,6){};
\node[circle, scale=0.75, fill] (tid5) at (1.5,7.5){};
\node[circle, scale=0.75, fill] (tid6) at (1.5,9){};
\node[circle, scale=0.75, fill, red] (tid7) at (0.75,10.5){};
\node[circle, scale=0.75, fill, red] (tid8) at (2.25,10.5){};
\draw[](tid6) -- (tid7);
\draw[](tid6) -- (tid8);
\draw[](tid5) -- (tid6);
\draw[](tid4) -- (tid5);
\draw[](tid3) -- (tid4);
\draw[](tid1) -- (tid3);
\node[circle, scale=0.75, fill, red] (tid2) at (3.75,3){};
\draw[](tid0) -- (tid1);
\draw[](tid0) -- (tid2);
\end{tikzpicture}
\nodepart{three}
\footnotesize{7.51042}
\nodepart{four}
\footnotesize{$33\:67$}
};
 & 
\node[draw=black, rectangle split,  rectangle split parts=4] (sn0x984d10){
\footnotesize{44.4444}
\nodepart{two}
\begin{tikzpicture}[scale=.2]
\node[circle, scale=0.75, fill] (tid0) at (1.5,1.5){};
\node[circle, scale=0.75, fill] (tid1) at (0.75,3){};
\node[circle, scale=0.75, fill] (tid3) at (0.75,4.5){};
\node[circle, scale=0.75, fill] (tid5) at (0.75,6){};
\node[circle, scale=0.75, fill] (tid6) at (0.75,7.5){};
\node[circle, scale=0.75, fill] (tid7) at (0.75,9){};
\node[circle, scale=0.75, fill, red] (tid8) at (0.75,10.5){};
\draw[](tid7) -- (tid8);
\draw[](tid6) -- (tid7);
\draw[](tid5) -- (tid6);
\draw[](tid3) -- (tid5);
\draw[](tid1) -- (tid3);
\node[circle, scale=0.75, fill] (tid2) at (2.25,3){};
\node[circle, scale=0.75, fill, red] (tid4) at (2.25,4.5){};
\draw[](tid2) -- (tid4);
\draw[](tid0) -- (tid1);
\draw[](tid0) -- (tid2);
\end{tikzpicture}
\nodepart{three}
\footnotesize{7.07812}
\nodepart{four}
\footnotesize{$50\:50$}
};
 & 
\node[draw=black, rectangle split,  rectangle split parts=4] (sn0x9897c0){
\footnotesize{22.2222}
\nodepart{two}
\begin{tikzpicture}[scale=.2]
\node[circle, scale=0.75, fill] (tid0) at (2.25,1.5){};
\node[circle, scale=0.75, fill] (tid1) at (1.5,3){};
\node[circle, scale=0.75, fill] (tid3) at (1.5,4.5){};
\node[circle, scale=0.75, fill] (tid4) at (0.75,6){};
\node[circle, scale=0.75, fill] (tid6) at (0.75,7.5){};
\node[circle, scale=0.75, fill] (tid7) at (0.75,9){};
\node[circle, scale=0.75, fill, red] (tid8) at (0.75,10.5){};
\draw[](tid7) -- (tid8);
\draw[](tid6) -- (tid7);
\draw[](tid4) -- (tid6);
\node[circle, scale=0.75, fill, red] (tid5) at (2.25,6){};
\draw[](tid3) -- (tid4);
\draw[](tid3) -- (tid5);
\draw[](tid1) -- (tid3);
\node[circle, scale=0.75, fill, red] (tid2) at (3.75,3){};
\draw[](tid0) -- (tid1);
\draw[](tid0) -- (tid2);
\end{tikzpicture}
\nodepart{three}
\footnotesize{7.07658}
\nodepart{four}
\footnotesize{$33\:33\:33$}
};
 & 
\node[draw=black, rectangle split,  rectangle split parts=4] (sn0x98a0b0){
\footnotesize{22.2222}
\nodepart{two}
\begin{tikzpicture}[scale=.2]
\node[circle, scale=0.75, fill] (tid0) at (2.25,1.5){};
\node[circle, scale=0.75, fill] (tid1) at (1.5,3){};
\node[circle, scale=0.75, fill] (tid3) at (1.5,4.5){};
\node[circle, scale=0.75, fill] (tid5) at (0.75,6){};
\node[circle, scale=0.75, fill] (tid7) at (0.75,7.5){};
\node[circle, scale=0.75, fill, red] (tid8) at (0.75,9){};
\draw[](tid7) -- (tid8);
\draw[](tid5) -- (tid7);
\node[circle, scale=0.75, fill, red] (tid6) at (2.25,6){};
\draw[](tid3) -- (tid5);
\draw[](tid3) -- (tid6);
\draw[](tid1) -- (tid3);
\node[circle, scale=0.75, fill] (tid2) at (3.75,3){};
\node[circle, scale=0.75, fill, red] (tid4) at (3.75,4.5){};
\draw[](tid2) -- (tid4);
\draw[](tid0) -- (tid1);
\draw[](tid0) -- (tid2);
\end{tikzpicture}
\nodepart{three}
\footnotesize{6.24942}
\nodepart{four}
\footnotesize{$33\:33\:33$}
};
 & 
\\
};
\end{scope}
\draw (sn0x983680.south) -- (sn0x984240.north);
\draw (sn0x983680.south) -- (sn0x984830.north);
\draw (sn0x984240.south) -- (sn0x984fb0.north);
\draw (sn0x984240.south) -- (sn0x984d10.north);
\draw (sn0x984830.south) -- (sn0x9897c0.north);
\draw (sn0x984830.south) -- (sn0x984d10.north);
\draw (sn0x984830.south) -- (sn0x98a0b0.north);
\end{tikzpicture}
%%% Local Variables:
%%% TeX-master: "thesis/thesis.tex"
%%% End: 
    \renewcommand{\leveltopI}{-19cm + \leveltop}
\renewcommand{\leveltopII}{-19cm + \leveltopI}
\renewcommand{\leveltopIII}{-19cm + \leveltopII}
\renewcommand{\leveltopIIII}{-19cm + \leveltopIII}
\renewcommand{\leveltopIIIII}{-19cm + \leveltopIIII}
\renewcommand{\leveltopIIIIII}{-19cm + \leveltopIIIII}
\renewcommand{\leveltopIIIIIII}{-19cm + \leveltopIIIIII}
\renewcommand{\leveltopIIIIIIII}{-19cm + \leveltopIIIIIII}
\renewcommand{\leveltopIIIIIIIII}{-19cm + \leveltopIIIIIIII}
\renewcommand{\leveltopIIIIIIIIII}{-19cm + \leveltopIIIIIIIII}
\renewcommand{\leveltopIIIIIIIIIII}{-19cm + \leveltopIIIIIIIIII}
\begin{tikzpicture}[scale=.2, anchor=south]
\begin{scope}[yshift=\leveltopI cm]
\matrix (line1)[column sep=0.5cm] {
\node[draw=black, rectangle split,  rectangle split parts=4] (sn0x18e59c0){
\footnotesize{100}
\nodepart{two}
\begin{tikzpicture}[scale=.2]
\node[circle, scale=0.75, fill] (tid0) at (3,1.5){};
\node[circle, scale=0.75, fill] (tid1) at (2.25,3){};
\node[circle, scale=0.75, fill] (tid3) at (2.25,4.5){};
\node[circle, scale=0.75, fill] (tid5) at (1.5,6){};
\node[circle, scale=0.75, fill] (tid7) at (1.5,7.5){};
\node[circle, scale=0.75, fill] (tid8) at (1.5,9){};
\node[circle, scale=0.75, fill, red] (tid9) at (0.75,10.5){};
\node[circle, scale=0.75, fill, red] (tid10) at (2.25,10.5){};
\draw[](tid8) -- (tid9);
\draw[](tid8) -- (tid10);
\draw[](tid7) -- (tid8);
\draw[](tid5) -- (tid7);
\node[circle, scale=0.75, fill] (tid6) at (3.75,6){};
\draw[](tid3) -- (tid5);
\draw[](tid3) -- (tid6);
\draw[](tid1) -- (tid3);
\node[circle, scale=0.75, fill] (tid2) at (5.25,3){};
\node[circle, scale=0.75, fill, red] (tid4) at (5.25,4.5){};
\draw[](tid2) -- (tid4);
\draw[](tid0) -- (tid1);
\draw[](tid0) -- (tid2);
\end{tikzpicture}
\nodepart{three}
\footnotesize{7.60798}
\nodepart{four}
\footnotesize{$33\:67$}
};
 & 
\\
};
\end{scope}
\begin{scope}[yshift=\leveltopII cm]
\matrix (line2)[column sep=0.5cm] {
\node[draw=black, rectangle split,  rectangle split parts=4] (sn0x18e2cc0){
\footnotesize{33.3333}
\nodepart{two}
\begin{tikzpicture}[scale=.2]
\node[circle, scale=0.75, fill] (tid0) at (3,1.5){};
\node[circle, scale=0.75, fill] (tid1) at (2.25,3){};
\node[circle, scale=0.75, fill] (tid3) at (2.25,4.5){};
\node[circle, scale=0.75, fill] (tid4) at (1.5,6){};
\node[circle, scale=0.75, fill] (tid6) at (1.5,7.5){};
\node[circle, scale=0.75, fill] (tid7) at (1.5,9){};
\node[circle, scale=0.75, fill, red] (tid8) at (0.75,10.5){};
\node[circle, scale=0.75, fill, red] (tid9) at (2.25,10.5){};
\draw[](tid7) -- (tid8);
\draw[](tid7) -- (tid9);
\draw[](tid6) -- (tid7);
\draw[](tid4) -- (tid6);
\node[circle, scale=0.75, fill, red] (tid5) at (3.75,6){};
\draw[](tid3) -- (tid4);
\draw[](tid3) -- (tid5);
\draw[](tid1) -- (tid3);
\node[circle, scale=0.75, fill] (tid2) at (5.25,3){};
\draw[](tid0) -- (tid1);
\draw[](tid0) -- (tid2);
\end{tikzpicture}
\nodepart{three}
\footnotesize{7.55453}
\nodepart{four}
\footnotesize{$33\:67$}
};
 & 
\node[draw=black, rectangle split,  rectangle split parts=4] (sn0x18e5770){
\footnotesize{66.6667}
\nodepart{two}
\begin{tikzpicture}[scale=.2]
\node[circle, scale=0.75, fill] (tid0) at (2.25,1.5){};
\node[circle, scale=0.75, fill] (tid1) at (1.5,3){};
\node[circle, scale=0.75, fill] (tid3) at (1.5,4.5){};
\node[circle, scale=0.75, fill] (tid5) at (0.75,6){};
\node[circle, scale=0.75, fill] (tid7) at (0.75,7.5){};
\node[circle, scale=0.75, fill] (tid8) at (0.75,9){};
\node[circle, scale=0.75, fill, red] (tid9) at (0.75,10.5){};
\draw[](tid8) -- (tid9);
\draw[](tid7) -- (tid8);
\draw[](tid5) -- (tid7);
\node[circle, scale=0.75, fill, red] (tid6) at (2.25,6){};
\draw[](tid3) -- (tid5);
\draw[](tid3) -- (tid6);
\draw[](tid1) -- (tid3);
\node[circle, scale=0.75, fill] (tid2) at (3.75,3){};
\node[circle, scale=0.75, fill, red] (tid4) at (3.75,4.5){};
\draw[](tid2) -- (tid4);
\draw[](tid0) -- (tid1);
\draw[](tid0) -- (tid2);
\end{tikzpicture}
\nodepart{three}
\footnotesize{7.13471}
\nodepart{four}
\footnotesize{$33\:33\:33$}
};
 & 
\\
};
\end{scope}
\begin{scope}[yshift=\leveltopIII cm]
\matrix (line3)[column sep=0.5cm] {
\node[draw=black, rectangle split,  rectangle split parts=4] (sn0x18e21e0){
\footnotesize{11.1111}
\nodepart{two}
\begin{tikzpicture}[scale=.2]
\node[circle, scale=0.75, fill] (tid0) at (2.25,1.5){};
\node[circle, scale=0.75, fill] (tid1) at (1.5,3){};
\node[circle, scale=0.75, fill] (tid3) at (1.5,4.5){};
\node[circle, scale=0.75, fill] (tid4) at (1.5,6){};
\node[circle, scale=0.75, fill] (tid5) at (1.5,7.5){};
\node[circle, scale=0.75, fill] (tid6) at (1.5,9){};
\node[circle, scale=0.75, fill, red] (tid7) at (0.75,10.5){};
\node[circle, scale=0.75, fill, red] (tid8) at (2.25,10.5){};
\draw[](tid6) -- (tid7);
\draw[](tid6) -- (tid8);
\draw[](tid5) -- (tid6);
\draw[](tid4) -- (tid5);
\draw[](tid3) -- (tid4);
\draw[](tid1) -- (tid3);
\node[circle, scale=0.75, fill, red] (tid2) at (3.75,3){};
\draw[](tid0) -- (tid1);
\draw[](tid0) -- (tid2);
\end{tikzpicture}
\nodepart{three}
\footnotesize{7.51042}
\nodepart{four}
\footnotesize{$33\:67$}
};
 & 
\node[draw=black, rectangle split,  rectangle split parts=4] (sn0x18e1e60){
\footnotesize{44.4444}
\nodepart{two}
\begin{tikzpicture}[scale=.2]
\node[circle, scale=0.75, fill] (tid0) at (2.25,1.5){};
\node[circle, scale=0.75, fill] (tid1) at (1.5,3){};
\node[circle, scale=0.75, fill] (tid3) at (1.5,4.5){};
\node[circle, scale=0.75, fill] (tid4) at (0.75,6){};
\node[circle, scale=0.75, fill] (tid6) at (0.75,7.5){};
\node[circle, scale=0.75, fill] (tid7) at (0.75,9){};
\node[circle, scale=0.75, fill, red] (tid8) at (0.75,10.5){};
\draw[](tid7) -- (tid8);
\draw[](tid6) -- (tid7);
\draw[](tid4) -- (tid6);
\node[circle, scale=0.75, fill, red] (tid5) at (2.25,6){};
\draw[](tid3) -- (tid4);
\draw[](tid3) -- (tid5);
\draw[](tid1) -- (tid3);
\node[circle, scale=0.75, fill, red] (tid2) at (3.75,3){};
\draw[](tid0) -- (tid1);
\draw[](tid0) -- (tid2);
\end{tikzpicture}
\nodepart{three}
\footnotesize{7.07658}
\nodepart{four}
\footnotesize{$33\:33\:33$}
};
 & 
\node[draw=black, rectangle split,  rectangle split parts=4] (sn0x18e45f0){
\footnotesize{22.2222}
\nodepart{two}
\begin{tikzpicture}[scale=.2]
\node[circle, scale=0.75, fill] (tid0) at (1.5,1.5){};
\node[circle, scale=0.75, fill] (tid1) at (0.75,3){};
\node[circle, scale=0.75, fill] (tid3) at (0.75,4.5){};
\node[circle, scale=0.75, fill] (tid5) at (0.75,6){};
\node[circle, scale=0.75, fill] (tid6) at (0.75,7.5){};
\node[circle, scale=0.75, fill] (tid7) at (0.75,9){};
\node[circle, scale=0.75, fill, red] (tid8) at (0.75,10.5){};
\draw[](tid7) -- (tid8);
\draw[](tid6) -- (tid7);
\draw[](tid5) -- (tid6);
\draw[](tid3) -- (tid5);
\draw[](tid1) -- (tid3);
\node[circle, scale=0.75, fill] (tid2) at (2.25,3){};
\node[circle, scale=0.75, fill, red] (tid4) at (2.25,4.5){};
\draw[](tid2) -- (tid4);
\draw[](tid0) -- (tid1);
\draw[](tid0) -- (tid2);
\end{tikzpicture}
\nodepart{three}
\footnotesize{7.07812}
\nodepart{four}
\footnotesize{$50\:50$}
};
 & 
\node[draw=black, rectangle split,  rectangle split parts=4] (sn0x18e49e0){
\footnotesize{22.2222}
\nodepart{two}
\begin{tikzpicture}[scale=.2]
\node[circle, scale=0.75, fill] (tid0) at (2.25,1.5){};
\node[circle, scale=0.75, fill] (tid1) at (1.5,3){};
\node[circle, scale=0.75, fill] (tid3) at (1.5,4.5){};
\node[circle, scale=0.75, fill] (tid5) at (0.75,6){};
\node[circle, scale=0.75, fill] (tid7) at (0.75,7.5){};
\node[circle, scale=0.75, fill, red] (tid8) at (0.75,9){};
\draw[](tid7) -- (tid8);
\draw[](tid5) -- (tid7);
\node[circle, scale=0.75, fill, red] (tid6) at (2.25,6){};
\draw[](tid3) -- (tid5);
\draw[](tid3) -- (tid6);
\draw[](tid1) -- (tid3);
\node[circle, scale=0.75, fill] (tid2) at (3.75,3){};
\node[circle, scale=0.75, fill, red] (tid4) at (3.75,4.5){};
\draw[](tid2) -- (tid4);
\draw[](tid0) -- (tid1);
\draw[](tid0) -- (tid2);
\end{tikzpicture}
\nodepart{three}
\footnotesize{6.24942}
\nodepart{four}
\footnotesize{$33\:33\:33$}
};
 & 
\\
};
\end{scope}
\draw (sn0x18e59c0.south) -- (sn0x18e2cc0.north);
\draw (sn0x18e59c0.south) -- (sn0x18e5770.north);
\draw (sn0x18e2cc0.south) -- (sn0x18e21e0.north);
\draw (sn0x18e2cc0.south) -- (sn0x18e1e60.north);
\draw (sn0x18e5770.south) -- (sn0x18e1e60.north);
\draw (sn0x18e5770.south) -- (sn0x18e45f0.north);
\draw (sn0x18e5770.south) -- (sn0x18e49e0.north);
\end{tikzpicture}
%%% Local Variables:
%%% TeX-master: "thesis/thesis.tex"
%%% End: 

  \end{minipage}
}
\frame{
  \begin{minipage}{.25\textwidth}
    \subsection{P3-HLF not optimal for 00123455799}
    \renewcommand{\leveltopI}{-19cm + \leveltop}
\renewcommand{\leveltopII}{-19cm + \leveltopI}
\renewcommand{\leveltopIII}{-19cm + \leveltopII}
\renewcommand{\leveltopIIII}{-19cm + \leveltopIII}
\renewcommand{\leveltopIIIII}{-19cm + \leveltopIIII}
\renewcommand{\leveltopIIIIII}{-19cm + \leveltopIIIII}
\renewcommand{\leveltopIIIIIII}{-19cm + \leveltopIIIIII}
\renewcommand{\leveltopIIIIIIII}{-19cm + \leveltopIIIIIII}
\renewcommand{\leveltopIIIIIIIII}{-19cm + \leveltopIIIIIIII}
\renewcommand{\leveltopIIIIIIIIII}{-19cm + \leveltopIIIIIIIII}
\renewcommand{\leveltopIIIIIIIIIII}{-19cm + \leveltopIIIIIIIIII}
\renewcommand{\leveltopIIIIIIIIIIII}{-19cm + \leveltopIIIIIIIIIII}
\begin{tikzpicture}[scale=.2, anchor=south]
\begin{scope}[yshift=\leveltopI cm]
\matrix (line1)[column sep=0.5cm] {
\node[draw=black, rectangle split,  rectangle split parts=4] (sn0x90bf0b8){
\footnotesize{100}
\nodepart{two}
\begin{tikzpicture}[scale=.2]
\node[circle, scale=0.75, fill] (tid0) at (3,1.5){};
\node[circle, scale=0.75, fill] (tid1) at (2.25,3){};
\node[circle, scale=0.75, fill] (tid3) at (2.25,4.5){};
\node[circle, scale=0.75, fill] (tid5) at (2.25,6){};
\node[circle, scale=0.75, fill] (tid7) at (1.5,7.5){};
\node[circle, scale=0.75, fill] (tid9) at (1.5,9){};
\node[circle, scale=0.75, fill, red] (tid10) at (0.75,10.5){};
\node[circle, scale=0.75, fill, red] (tid11) at (2.25,10.5){};
\draw[](tid9) -- (tid10);
\draw[](tid9) -- (tid11);
\draw[](tid7) -- (tid9);
\node[circle, scale=0.75, fill, red] (tid8) at (3.75,7.5){};
\draw[](tid5) -- (tid7);
\draw[](tid5) -- (tid8);
\draw[](tid3) -- (tid5);
\draw[](tid1) -- (tid3);
\node[circle, scale=0.75, fill] (tid2) at (5.25,3){};
\node[circle, scale=0.75, fill] (tid4) at (5.25,4.5){};
\node[circle, scale=0.75, fill] (tid6) at (5.25,6){};
\draw[](tid4) -- (tid6);
\draw[](tid2) -- (tid4);
\draw[](tid0) -- (tid1);
\draw[](tid0) -- (tid2);
\end{tikzpicture}
\nodepart{three}
\footnotesize{7.77328}
\nodepart{four}
\footnotesize{$33\:67$}
};
 & 
\\
};
\end{scope}
\begin{scope}[yshift=\leveltopII cm]
\matrix (line2)[column sep=0.5cm] {
\node[draw=black, rectangle split,  rectangle split parts=4] (sn0x90be790){
\footnotesize{33.3333}
\nodepart{two}
\begin{tikzpicture}[scale=.2]
\node[circle, scale=0.75, fill] (tid0) at (2.25,1.5){};
\node[circle, scale=0.75, fill] (tid1) at (1.5,3){};
\node[circle, scale=0.75, fill] (tid3) at (1.5,4.5){};
\node[circle, scale=0.75, fill] (tid5) at (1.5,6){};
\node[circle, scale=0.75, fill] (tid7) at (1.5,7.5){};
\node[circle, scale=0.75, fill] (tid8) at (1.5,9){};
\node[circle, scale=0.75, fill, red] (tid9) at (0.75,10.5){};
\node[circle, scale=0.75, fill, red] (tid10) at (2.25,10.5){};
\draw[](tid8) -- (tid9);
\draw[](tid8) -- (tid10);
\draw[](tid7) -- (tid8);
\draw[](tid5) -- (tid7);
\draw[](tid3) -- (tid5);
\draw[](tid1) -- (tid3);
\node[circle, scale=0.75, fill] (tid2) at (3.75,3){};
\node[circle, scale=0.75, fill] (tid4) at (3.75,4.5){};
\node[circle, scale=0.75, fill, red] (tid6) at (3.75,6){};
\draw[](tid4) -- (tid6);
\draw[](tid2) -- (tid4);
\draw[](tid0) -- (tid1);
\draw[](tid0) -- (tid2);
\end{tikzpicture}
\nodepart{three}
\footnotesize{7.66696}
\nodepart{four}
\footnotesize{$33\:67$}
};
 & 
\node[draw=black, rectangle split,  rectangle split parts=4] (sn0x90bd580){
\footnotesize{66.6667}
\nodepart{two}
\begin{tikzpicture}[scale=.2]
\node[circle, scale=0.75, fill] (tid0) at (2.25,1.5){};
\node[circle, scale=0.75, fill] (tid1) at (1.5,3){};
\node[circle, scale=0.75, fill] (tid3) at (1.5,4.5){};
\node[circle, scale=0.75, fill] (tid5) at (1.5,6){};
\node[circle, scale=0.75, fill] (tid7) at (0.75,7.5){};
\node[circle, scale=0.75, fill] (tid9) at (0.75,9){};
\node[circle, scale=0.75, fill, red] (tid10) at (0.75,10.5){};
\draw[](tid9) -- (tid10);
\draw[](tid7) -- (tid9);
\node[circle, scale=0.75, fill, red] (tid8) at (2.25,7.5){};
\draw[](tid5) -- (tid7);
\draw[](tid5) -- (tid8);
\draw[](tid3) -- (tid5);
\draw[](tid1) -- (tid3);
\node[circle, scale=0.75, fill] (tid2) at (3.75,3){};
\node[circle, scale=0.75, fill] (tid4) at (3.75,4.5){};
\node[circle, scale=0.75, fill, red] (tid6) at (3.75,6){};
\draw[](tid4) -- (tid6);
\draw[](tid2) -- (tid4);
\draw[](tid0) -- (tid1);
\draw[](tid0) -- (tid2);
\end{tikzpicture}
\nodepart{three}
\footnotesize{7.32644}
\nodepart{four}
\footnotesize{$33\:33\:33$}
};
 & 
\\
};
\end{scope}
\begin{scope}[yshift=\leveltopIII cm]
\matrix (line3)[column sep=0.5cm] {
\node[draw=black, rectangle split,  rectangle split parts=4] (sn0x90beb08){
\footnotesize{11.1111}
\nodepart{two}
\begin{tikzpicture}[scale=.2]
\node[circle, scale=0.75, fill] (tid0) at (2.25,1.5){};
\node[circle, scale=0.75, fill] (tid1) at (1.5,3){};
\node[circle, scale=0.75, fill] (tid3) at (1.5,4.5){};
\node[circle, scale=0.75, fill] (tid5) at (1.5,6){};
\node[circle, scale=0.75, fill] (tid6) at (1.5,7.5){};
\node[circle, scale=0.75, fill] (tid7) at (1.5,9){};
\node[circle, scale=0.75, fill, red] (tid8) at (0.75,10.5){};
\node[circle, scale=0.75, fill, red] (tid9) at (2.25,10.5){};
\draw[](tid7) -- (tid8);
\draw[](tid7) -- (tid9);
\draw[](tid6) -- (tid7);
\draw[](tid5) -- (tid6);
\draw[](tid3) -- (tid5);
\draw[](tid1) -- (tid3);
\node[circle, scale=0.75, fill] (tid2) at (3.75,3){};
\node[circle, scale=0.75, fill, red] (tid4) at (3.75,4.5){};
\draw[](tid2) -- (tid4);
\draw[](tid0) -- (tid1);
\draw[](tid0) -- (tid2);
\end{tikzpicture}
\nodepart{three}
\footnotesize{7.55556}
\nodepart{four}
\footnotesize{$33\:67$}
};
 & 
\node[draw=black, rectangle split,  rectangle split parts=4] (sn0x90bf898){
\footnotesize{44.4444}
\nodepart{two}
\begin{tikzpicture}[scale=.2]
\node[circle, scale=0.75, fill] (tid0) at (1.5,1.5){};
\node[circle, scale=0.75, fill] (tid1) at (0.75,3){};
\node[circle, scale=0.75, fill] (tid3) at (0.75,4.5){};
\node[circle, scale=0.75, fill] (tid5) at (0.75,6){};
\node[circle, scale=0.75, fill] (tid7) at (0.75,7.5){};
\node[circle, scale=0.75, fill] (tid8) at (0.75,9){};
\node[circle, scale=0.75, fill, red] (tid9) at (0.75,10.5){};
\draw[](tid8) -- (tid9);
\draw[](tid7) -- (tid8);
\draw[](tid5) -- (tid7);
\draw[](tid3) -- (tid5);
\draw[](tid1) -- (tid3);
\node[circle, scale=0.75, fill] (tid2) at (2.25,3){};
\node[circle, scale=0.75, fill] (tid4) at (2.25,4.5){};
\node[circle, scale=0.75, fill, red] (tid6) at (2.25,6){};
\draw[](tid4) -- (tid6);
\draw[](tid2) -- (tid4);
\draw[](tid0) -- (tid1);
\draw[](tid0) -- (tid2);
\end{tikzpicture}
\nodepart{three}
\footnotesize{7.22266}
\nodepart{four}
\footnotesize{$50\:50$}
};
 & 
\node[draw=black, rectangle split,  rectangle split parts=4] (sn0x90c1df8){
\footnotesize{22.2222}
\nodepart{two}
\begin{tikzpicture}[scale=.2]
\node[circle, scale=0.75, fill] (tid0) at (2.25,1.5){};
\node[circle, scale=0.75, fill] (tid1) at (1.5,3){};
\node[circle, scale=0.75, fill] (tid3) at (1.5,4.5){};
\node[circle, scale=0.75, fill] (tid5) at (1.5,6){};
\node[circle, scale=0.75, fill] (tid6) at (0.75,7.5){};
\node[circle, scale=0.75, fill] (tid8) at (0.75,9){};
\node[circle, scale=0.75, fill, red] (tid9) at (0.75,10.5){};
\draw[](tid8) -- (tid9);
\draw[](tid6) -- (tid8);
\node[circle, scale=0.75, fill, red] (tid7) at (2.25,7.5){};
\draw[](tid5) -- (tid6);
\draw[](tid5) -- (tid7);
\draw[](tid3) -- (tid5);
\draw[](tid1) -- (tid3);
\node[circle, scale=0.75, fill] (tid2) at (3.75,3){};
\node[circle, scale=0.75, fill, red] (tid4) at (3.75,4.5){};
\draw[](tid2) -- (tid4);
\draw[](tid0) -- (tid1);
\draw[](tid0) -- (tid2);
\end{tikzpicture}
\nodepart{three}
\footnotesize{7.19387}
\nodepart{four}
\footnotesize{$33\:33\:33$}
};
 & 
\node[draw=black, rectangle split,  rectangle split parts=4] (sn0x90c3bd8){
\footnotesize{22.2222}
\nodepart{two}
\begin{tikzpicture}[scale=.2]
\node[circle, scale=0.75, fill] (tid0) at (2.25,1.5){};
\node[circle, scale=0.75, fill] (tid1) at (1.5,3){};
\node[circle, scale=0.75, fill] (tid3) at (1.5,4.5){};
\node[circle, scale=0.75, fill] (tid5) at (1.5,6){};
\node[circle, scale=0.75, fill] (tid7) at (0.75,7.5){};
\node[circle, scale=0.75, fill, red] (tid9) at (0.75,9){};
\draw[](tid7) -- (tid9);
\node[circle, scale=0.75, fill, red] (tid8) at (2.25,7.5){};
\draw[](tid5) -- (tid7);
\draw[](tid5) -- (tid8);
\draw[](tid3) -- (tid5);
\draw[](tid1) -- (tid3);
\node[circle, scale=0.75, fill] (tid2) at (3.75,3){};
\node[circle, scale=0.75, fill] (tid4) at (3.75,4.5){};
\node[circle, scale=0.75, fill, red] (tid6) at (3.75,6){};
\draw[](tid4) -- (tid6);
\draw[](tid2) -- (tid4);
\draw[](tid0) -- (tid1);
\draw[](tid0) -- (tid2);
\end{tikzpicture}
\nodepart{three}
\footnotesize{6.56279}
\nodepart{four}
\footnotesize{$33\:33\:33$}
};
 & 
\\
};
\end{scope}
\draw (sn0x90bf0b8.south) -- (sn0x90be790.north);
\draw (sn0x90bf0b8.south) -- (sn0x90bd580.north);
\draw (sn0x90be790.south) -- (sn0x90beb08.north);
\draw (sn0x90be790.south) -- (sn0x90bf898.north);
\draw (sn0x90bd580.south) -- (sn0x90c1df8.north);
\draw (sn0x90bd580.south) -- (sn0x90bf898.north);
\draw (sn0x90bd580.south) -- (sn0x90c3bd8.north);
\end{tikzpicture}
%%% Local Variables:
%%% TeX-master: "thesis/thesis.tex"
%%% End: 
    \renewcommand{\leveltopI}{-19cm + \leveltop}
\renewcommand{\leveltopII}{-19cm + \leveltopI}
\renewcommand{\leveltopIII}{-19cm + \leveltopII}
\renewcommand{\leveltopIIII}{-19cm + \leveltopIII}
\renewcommand{\leveltopIIIII}{-19cm + \leveltopIIII}
\renewcommand{\leveltopIIIIII}{-19cm + \leveltopIIIII}
\renewcommand{\leveltopIIIIIII}{-19cm + \leveltopIIIIII}
\renewcommand{\leveltopIIIIIIII}{-19cm + \leveltopIIIIIII}
\renewcommand{\leveltopIIIIIIIII}{-19cm + \leveltopIIIIIIII}
\renewcommand{\leveltopIIIIIIIIII}{-19cm + \leveltopIIIIIIIII}
\renewcommand{\leveltopIIIIIIIIIII}{-19cm + \leveltopIIIIIIIIII}
\renewcommand{\leveltopIIIIIIIIIIII}{-19cm + \leveltopIIIIIIIIIII}
\begin{tikzpicture}[scale=.2, anchor=south]
\begin{scope}[yshift=\leveltopI cm]
\matrix (line1)[column sep=0.5cm] {
\node[draw=black, rectangle split,  rectangle split parts=4] (sn0x9748440){
\footnotesize{100}
\nodepart{two}
\begin{tikzpicture}[scale=.2]
\node[circle, scale=0.75, fill] (tid0) at (3,1.5){};
\node[circle, scale=0.75, fill] (tid1) at (2.25,3){};
\node[circle, scale=0.75, fill] (tid3) at (2.25,4.5){};
\node[circle, scale=0.75, fill] (tid5) at (2.25,6){};
\node[circle, scale=0.75, fill] (tid7) at (1.5,7.5){};
\node[circle, scale=0.75, fill] (tid9) at (1.5,9){};
\node[circle, scale=0.75, fill, red] (tid10) at (0.75,10.5){};
\node[circle, scale=0.75, fill, red] (tid11) at (2.25,10.5){};
\draw[](tid9) -- (tid10);
\draw[](tid9) -- (tid11);
\draw[](tid7) -- (tid9);
\node[circle, scale=0.75, fill] (tid8) at (3.75,7.5){};
\draw[](tid5) -- (tid7);
\draw[](tid5) -- (tid8);
\draw[](tid3) -- (tid5);
\draw[](tid1) -- (tid3);
\node[circle, scale=0.75, fill] (tid2) at (5.25,3){};
\node[circle, scale=0.75, fill] (tid4) at (5.25,4.5){};
\node[circle, scale=0.75, fill, red] (tid6) at (5.25,6){};
\draw[](tid4) -- (tid6);
\draw[](tid2) -- (tid4);
\draw[](tid0) -- (tid1);
\draw[](tid0) -- (tid2);
\end{tikzpicture}
\nodepart{three}
\footnotesize{7.76688}
\nodepart{four}
\footnotesize{$33\:67$}
};
 & 
\\
};
\end{scope}
\begin{scope}[yshift=\leveltopII cm]
\matrix (line2)[column sep=0.5cm] {
\node[draw=black, rectangle split,  rectangle split parts=4] (sn0x97466f8){
\footnotesize{33.3333}
\nodepart{two}
\begin{tikzpicture}[scale=.2]
\node[circle, scale=0.75, fill] (tid0) at (3,1.5){};
\node[circle, scale=0.75, fill] (tid1) at (2.25,3){};
\node[circle, scale=0.75, fill] (tid3) at (2.25,4.5){};
\node[circle, scale=0.75, fill] (tid5) at (2.25,6){};
\node[circle, scale=0.75, fill] (tid6) at (1.5,7.5){};
\node[circle, scale=0.75, fill] (tid8) at (1.5,9){};
\node[circle, scale=0.75, fill, red] (tid9) at (0.75,10.5){};
\node[circle, scale=0.75, fill, red] (tid10) at (2.25,10.5){};
\draw[](tid8) -- (tid9);
\draw[](tid8) -- (tid10);
\draw[](tid6) -- (tid8);
\node[circle, scale=0.75, fill, red] (tid7) at (3.75,7.5){};
\draw[](tid5) -- (tid6);
\draw[](tid5) -- (tid7);
\draw[](tid3) -- (tid5);
\draw[](tid1) -- (tid3);
\node[circle, scale=0.75, fill] (tid2) at (5.25,3){};
\node[circle, scale=0.75, fill] (tid4) at (5.25,4.5){};
\draw[](tid2) -- (tid4);
\draw[](tid0) -- (tid1);
\draw[](tid0) -- (tid2);
\end{tikzpicture}
\nodepart{three}
\footnotesize{7.64776}
\nodepart{four}
\footnotesize{$33\:67$}
};
 & 
\node[draw=black, rectangle split,  rectangle split parts=4] (sn0x9748060){
\footnotesize{66.6667}
\nodepart{two}
\begin{tikzpicture}[scale=.2]
\node[circle, scale=0.75, fill] (tid0) at (2.25,1.5){};
\node[circle, scale=0.75, fill] (tid1) at (1.5,3){};
\node[circle, scale=0.75, fill] (tid3) at (1.5,4.5){};
\node[circle, scale=0.75, fill] (tid5) at (1.5,6){};
\node[circle, scale=0.75, fill] (tid7) at (0.75,7.5){};
\node[circle, scale=0.75, fill] (tid9) at (0.75,9){};
\node[circle, scale=0.75, fill, red] (tid10) at (0.75,10.5){};
\draw[](tid9) -- (tid10);
\draw[](tid7) -- (tid9);
\node[circle, scale=0.75, fill, red] (tid8) at (2.25,7.5){};
\draw[](tid5) -- (tid7);
\draw[](tid5) -- (tid8);
\draw[](tid3) -- (tid5);
\draw[](tid1) -- (tid3);
\node[circle, scale=0.75, fill] (tid2) at (3.75,3){};
\node[circle, scale=0.75, fill] (tid4) at (3.75,4.5){};
\node[circle, scale=0.75, fill, red] (tid6) at (3.75,6){};
\draw[](tid4) -- (tid6);
\draw[](tid2) -- (tid4);
\draw[](tid0) -- (tid1);
\draw[](tid0) -- (tid2);
\end{tikzpicture}
\nodepart{three}
\footnotesize{7.32644}
\nodepart{four}
\footnotesize{$33\:33\:33$}
};
 & 
\\
};
\end{scope}
\begin{scope}[yshift=\leveltopIII cm]
\matrix (line3)[column sep=0.5cm] {
\node[draw=black, rectangle split,  rectangle split parts=4] (sn0x9746d20){
\footnotesize{11.1111}
\nodepart{two}
\begin{tikzpicture}[scale=.2]
\node[circle, scale=0.75, fill] (tid0) at (2.25,1.5){};
\node[circle, scale=0.75, fill] (tid1) at (1.5,3){};
\node[circle, scale=0.75, fill] (tid3) at (1.5,4.5){};
\node[circle, scale=0.75, fill] (tid5) at (1.5,6){};
\node[circle, scale=0.75, fill] (tid6) at (1.5,7.5){};
\node[circle, scale=0.75, fill] (tid7) at (1.5,9){};
\node[circle, scale=0.75, fill, red] (tid8) at (0.75,10.5){};
\node[circle, scale=0.75, fill, red] (tid9) at (2.25,10.5){};
\draw[](tid7) -- (tid8);
\draw[](tid7) -- (tid9);
\draw[](tid6) -- (tid7);
\draw[](tid5) -- (tid6);
\draw[](tid3) -- (tid5);
\draw[](tid1) -- (tid3);
\node[circle, scale=0.75, fill] (tid2) at (3.75,3){};
\node[circle, scale=0.75, fill, red] (tid4) at (3.75,4.5){};
\draw[](tid2) -- (tid4);
\draw[](tid0) -- (tid1);
\draw[](tid0) -- (tid2);
\end{tikzpicture}
\nodepart{three}
\footnotesize{7.55556}
\nodepart{four}
\footnotesize{$33\:67$}
};
 & 
\node[draw=black, rectangle split,  rectangle split parts=4] (sn0x9745bf0){
\footnotesize{44.4444}
\nodepart{two}
\begin{tikzpicture}[scale=.2]
\node[circle, scale=0.75, fill] (tid0) at (2.25,1.5){};
\node[circle, scale=0.75, fill] (tid1) at (1.5,3){};
\node[circle, scale=0.75, fill] (tid3) at (1.5,4.5){};
\node[circle, scale=0.75, fill] (tid5) at (1.5,6){};
\node[circle, scale=0.75, fill] (tid6) at (0.75,7.5){};
\node[circle, scale=0.75, fill] (tid8) at (0.75,9){};
\node[circle, scale=0.75, fill, red] (tid9) at (0.75,10.5){};
\draw[](tid8) -- (tid9);
\draw[](tid6) -- (tid8);
\node[circle, scale=0.75, fill, red] (tid7) at (2.25,7.5){};
\draw[](tid5) -- (tid6);
\draw[](tid5) -- (tid7);
\draw[](tid3) -- (tid5);
\draw[](tid1) -- (tid3);
\node[circle, scale=0.75, fill] (tid2) at (3.75,3){};
\node[circle, scale=0.75, fill, red] (tid4) at (3.75,4.5){};
\draw[](tid2) -- (tid4);
\draw[](tid0) -- (tid1);
\draw[](tid0) -- (tid2);
\end{tikzpicture}
\nodepart{three}
\footnotesize{7.19387}
\nodepart{four}
\footnotesize{$33\:33\:33$}
};
 & 
\node[draw=black, rectangle split,  rectangle split parts=4] (sn0x9747af8){
\footnotesize{22.2222}
\nodepart{two}
\begin{tikzpicture}[scale=.2]
\node[circle, scale=0.75, fill] (tid0) at (1.5,1.5){};
\node[circle, scale=0.75, fill] (tid1) at (0.75,3){};
\node[circle, scale=0.75, fill] (tid3) at (0.75,4.5){};
\node[circle, scale=0.75, fill] (tid5) at (0.75,6){};
\node[circle, scale=0.75, fill] (tid7) at (0.75,7.5){};
\node[circle, scale=0.75, fill] (tid8) at (0.75,9){};
\node[circle, scale=0.75, fill, red] (tid9) at (0.75,10.5){};
\draw[](tid8) -- (tid9);
\draw[](tid7) -- (tid8);
\draw[](tid5) -- (tid7);
\draw[](tid3) -- (tid5);
\draw[](tid1) -- (tid3);
\node[circle, scale=0.75, fill] (tid2) at (2.25,3){};
\node[circle, scale=0.75, fill] (tid4) at (2.25,4.5){};
\node[circle, scale=0.75, fill, red] (tid6) at (2.25,6){};
\draw[](tid4) -- (tid6);
\draw[](tid2) -- (tid4);
\draw[](tid0) -- (tid1);
\draw[](tid0) -- (tid2);
\end{tikzpicture}
\nodepart{three}
\footnotesize{7.22266}
\nodepart{four}
\footnotesize{$50\:50$}
};
 & 
\node[draw=black, rectangle split,  rectangle split parts=4] (sn0x9747bd8){
\footnotesize{22.2222}
\nodepart{two}
\begin{tikzpicture}[scale=.2]
\node[circle, scale=0.75, fill] (tid0) at (2.25,1.5){};
\node[circle, scale=0.75, fill] (tid1) at (1.5,3){};
\node[circle, scale=0.75, fill] (tid3) at (1.5,4.5){};
\node[circle, scale=0.75, fill] (tid5) at (1.5,6){};
\node[circle, scale=0.75, fill] (tid7) at (0.75,7.5){};
\node[circle, scale=0.75, fill, red] (tid9) at (0.75,9){};
\draw[](tid7) -- (tid9);
\node[circle, scale=0.75, fill, red] (tid8) at (2.25,7.5){};
\draw[](tid5) -- (tid7);
\draw[](tid5) -- (tid8);
\draw[](tid3) -- (tid5);
\draw[](tid1) -- (tid3);
\node[circle, scale=0.75, fill] (tid2) at (3.75,3){};
\node[circle, scale=0.75, fill] (tid4) at (3.75,4.5){};
\node[circle, scale=0.75, fill, red] (tid6) at (3.75,6){};
\draw[](tid4) -- (tid6);
\draw[](tid2) -- (tid4);
\draw[](tid0) -- (tid1);
\draw[](tid0) -- (tid2);
\end{tikzpicture}
\nodepart{three}
\footnotesize{6.56279}
\nodepart{four}
\footnotesize{$33\:33\:33$}
};
 & 
\\
};
\end{scope}
\draw (sn0x9748440.south) -- (sn0x97466f8.north);
\draw (sn0x9748440.south) -- (sn0x9748060.north);
\draw (sn0x97466f8.south) -- (sn0x9746d20.north);
\draw (sn0x97466f8.south) -- (sn0x9745bf0.north);
\draw (sn0x9748060.south) -- (sn0x9745bf0.north);
\draw (sn0x9748060.south) -- (sn0x9747af8.north);
\draw (sn0x9748060.south) -- (sn0x9747bd8.north);
\end{tikzpicture}
%%% Local Variables:
%%% TeX-master: "thesis/thesis.tex"
%%% End: 
  \end{minipage}
}

\input{../default_unoptimized}
\input{../default_optimized}

\renewcommand{\leveltopI}{-15cm + \leveltop}
\renewcommand{\leveltopII}{-15cm + \leveltopI}
\renewcommand{\leveltopIII}{-15cm + \leveltopII}
\renewcommand{\leveltopIIII}{-15cm + \leveltopIII}
\renewcommand{\leveltopIIIII}{-15cm + \leveltopIIII}
\renewcommand{\leveltopIIIIII}{-15cm + \leveltopIIIII}
\renewcommand{\leveltopIIIIIII}{-15cm + \leveltopIIIIII}
\renewcommand{\leveltopIIIIIIII}{-15cm + \leveltopIIIIIII}
\renewcommand{\leveltopIIIIIIIII}{-15cm + \leveltopIIIIIIII}
\renewcommand{\leveltopIIIIIIIIII}{-15cm + \leveltopIIIIIIIII}
\renewcommand{\leveltopIIIIIIIIIII}{-15cm + \leveltopIIIIIIIIII}
\begin{tikzpicture}[scale=.2, anchor=south]
\begin{scope}[yshift=\leveltopI cm]
\matrix (line1)[column sep=0.5cm] {
\node[draw=black, rectangle split,  rectangle split parts=4] (sn0xef11d0){
\footnotesize{100}
\nodepart{two}
\begin{tikzpicture}[scale=.2]
\node[circle, scale=0.75, fill] (tid0) at (3,1.5){};
\node[circle, scale=0.75, fill] (tid1) at (3,3){};
\node[circle, scale=0.75, fill] (tid2) at (0.75,4.5){};
\node[circle, scale=0.75, fill] (tid5) at (0.75,6){};
\node[circle, scale=0.75, fill] (tid9) at (0.75,7.5){};
\node[circle, scale=0.75, fill] (tid10) at (0.75,9){};
\draw[](tid9) -- (tid10);
\draw[](tid5) -- (tid9);
\draw[](tid2) -- (tid5);
\node[circle, scale=0.75, fill] (tid3) at (3,4.5){};
\node[circle, scale=0.75, fill, task_scheduled] (tid6) at (2.25,6){};
\node[circle, scale=0.75, fill, task_scheduled] (tid7) at (3.75,6){};
\draw[](tid3) -- (tid6);
\draw[](tid3) -- (tid7);
\node[circle, scale=0.75, fill] (tid4) at (5.25,4.5){};
\node[circle, scale=0.75, fill] (tid8) at (5.25,6){};
\draw[](tid4) -- (tid8);
\draw[](tid1) -- (tid2);
\draw[](tid1) -- (tid3);
\draw[](tid1) -- (tid4);
\draw[](tid0) -- (tid1);
\end{tikzpicture}
\nodepart{three}
\footnotesize{7.5791}
\nodepart{four}
\footnotesize{$50\:50$}
};
 & 
\\
};
\end{scope}
\begin{scope}[yshift=\leveltopII cm]
\matrix (line2)[column sep=0.5cm] {
\node[draw=black, rectangle split,  rectangle split parts=4] (sn0xef2a50){
\footnotesize{50}
\nodepart{two}
\begin{tikzpicture}[scale=.2]
\node[circle, scale=0.75, fill] (tid0) at (2.25,1.5){};
\node[circle, scale=0.75, fill] (tid1) at (2.25,3){};
\node[circle, scale=0.75, fill] (tid2) at (0.75,4.5){};
\node[circle, scale=0.75, fill] (tid5) at (0.75,6){};
\node[circle, scale=0.75, fill] (tid8) at (0.75,7.5){};
\node[circle, scale=0.75, fill] (tid9) at (0.75,9){};
\draw[](tid8) -- (tid9);
\draw[](tid5) -- (tid8);
\draw[](tid2) -- (tid5);
\node[circle, scale=0.75, fill] (tid3) at (2.25,4.5){};
\node[circle, scale=0.75, fill, task_scheduled] (tid6) at (2.25,6){};
\draw[](tid3) -- (tid6);
\node[circle, scale=0.75, fill] (tid4) at (3.75,4.5){};
\node[circle, scale=0.75, fill, task_scheduled] (tid7) at (3.75,6){};
\draw[](tid4) -- (tid7);
\draw[](tid1) -- (tid2);
\draw[](tid1) -- (tid3);
\draw[](tid1) -- (tid4);
\draw[](tid0) -- (tid1);
\end{tikzpicture}
\nodepart{three}
\footnotesize{7.21191}
\nodepart{four}
\footnotesize{$50\:50$}
};
 & 
\node[draw=black, rectangle split,  rectangle split parts=4] (sn0xef2540){
\footnotesize{50}
\nodepart{two}
\begin{tikzpicture}[scale=.2]
\node[circle, scale=0.75, fill] (tid0) at (2.25,1.5){};
\node[circle, scale=0.75, fill] (tid1) at (2.25,3){};
\node[circle, scale=0.75, fill] (tid2) at (0.75,4.5){};
\node[circle, scale=0.75, fill] (tid5) at (0.75,6){};
\node[circle, scale=0.75, fill] (tid8) at (0.75,7.5){};
\node[circle, scale=0.75, fill, task_scheduled] (tid9) at (0.75,9){};
\draw[](tid8) -- (tid9);
\draw[](tid5) -- (tid8);
\draw[](tid2) -- (tid5);
\node[circle, scale=0.75, fill] (tid3) at (2.25,4.5){};
\node[circle, scale=0.75, fill, task_scheduled] (tid6) at (2.25,6){};
\draw[](tid3) -- (tid6);
\node[circle, scale=0.75, fill] (tid4) at (3.75,4.5){};
\node[circle, scale=0.75, fill] (tid7) at (3.75,6){};
\draw[](tid4) -- (tid7);
\draw[](tid1) -- (tid2);
\draw[](tid1) -- (tid3);
\draw[](tid1) -- (tid4);
\draw[](tid0) -- (tid1);
\end{tikzpicture}
\nodepart{three}
\footnotesize{6.94629}
\nodepart{four}
\footnotesize{$25\:25\:25\:25$}
};
 & 
\\
};
\end{scope}
\begin{scope}[yshift=\leveltopIII cm]
\matrix (line3)[column sep=0.5cm] {
\node[draw=black, rectangle split,  rectangle split parts=4] (sn0xef2f20){
\footnotesize{25}
\nodepart{two}
\begin{tikzpicture}[scale=.2]
\node[circle, scale=0.75, fill] (tid0) at (2.25,1.5){};
\node[circle, scale=0.75, fill] (tid1) at (2.25,3){};
\node[circle, scale=0.75, fill] (tid2) at (0.75,4.5){};
\node[circle, scale=0.75, fill] (tid5) at (0.75,6){};
\node[circle, scale=0.75, fill] (tid7) at (0.75,7.5){};
\node[circle, scale=0.75, fill] (tid8) at (0.75,9){};
\draw[](tid7) -- (tid8);
\draw[](tid5) -- (tid7);
\draw[](tid2) -- (tid5);
\node[circle, scale=0.75, fill] (tid3) at (2.25,4.5){};
\node[circle, scale=0.75, fill, task_scheduled] (tid6) at (2.25,6){};
\draw[](tid3) -- (tid6);
\node[circle, scale=0.75, fill, task_scheduled] (tid4) at (3.75,4.5){};
\draw[](tid1) -- (tid2);
\draw[](tid1) -- (tid3);
\draw[](tid1) -- (tid4);
\draw[](tid0) -- (tid1);
\end{tikzpicture}
\nodepart{three}
\footnotesize{6.83789}
\nodepart{four}
\footnotesize{$50\:25\:25$}
};
 & 
\node[draw=black, rectangle split,  rectangle split parts=4] (sn0xef3e70){
\footnotesize{37.5}
\nodepart{two}
\begin{tikzpicture}[scale=.2]
\node[circle, scale=0.75, fill] (tid0) at (2.25,1.5){};
\node[circle, scale=0.75, fill] (tid1) at (2.25,3){};
\node[circle, scale=0.75, fill] (tid2) at (0.75,4.5){};
\node[circle, scale=0.75, fill] (tid5) at (0.75,6){};
\node[circle, scale=0.75, fill] (tid7) at (0.75,7.5){};
\node[circle, scale=0.75, fill, task_scheduled] (tid8) at (0.75,9){};
\draw[](tid7) -- (tid8);
\draw[](tid5) -- (tid7);
\draw[](tid2) -- (tid5);
\node[circle, scale=0.75, fill] (tid3) at (2.25,4.5){};
\node[circle, scale=0.75, fill, task_scheduled] (tid6) at (2.25,6){};
\draw[](tid3) -- (tid6);
\node[circle, scale=0.75, fill] (tid4) at (3.75,4.5){};
\draw[](tid1) -- (tid2);
\draw[](tid1) -- (tid3);
\draw[](tid1) -- (tid4);
\draw[](tid0) -- (tid1);
\end{tikzpicture}
\nodepart{three}
\footnotesize{6.58594}
\nodepart{four}
\footnotesize{$50\:25\:25$}
};
 & 
\node[draw=black, rectangle split,  rectangle split parts=4] (sn0xefa920){
\footnotesize{12.5}
\nodepart{two}
\begin{tikzpicture}[scale=.2]
\node[circle, scale=0.75, fill] (tid0) at (2.25,1.5){};
\node[circle, scale=0.75, fill] (tid1) at (2.25,3){};
\node[circle, scale=0.75, fill] (tid2) at (0.75,4.5){};
\node[circle, scale=0.75, fill] (tid5) at (0.75,6){};
\node[circle, scale=0.75, fill] (tid7) at (0.75,7.5){};
\node[circle, scale=0.75, fill, task_scheduled] (tid8) at (0.75,9){};
\draw[](tid7) -- (tid8);
\draw[](tid5) -- (tid7);
\draw[](tid2) -- (tid5);
\node[circle, scale=0.75, fill] (tid3) at (2.25,4.5){};
\node[circle, scale=0.75, fill] (tid6) at (2.25,6){};
\draw[](tid3) -- (tid6);
\node[circle, scale=0.75, fill, task_scheduled] (tid4) at (3.75,4.5){};
\draw[](tid1) -- (tid2);
\draw[](tid1) -- (tid3);
\draw[](tid1) -- (tid4);
\draw[](tid0) -- (tid1);
\end{tikzpicture}
\nodepart{three}
\footnotesize{6.57617}
\nodepart{four}
\footnotesize{$50\:25\:25$}
};
 & 
\node[draw=black, rectangle split,  rectangle split parts=4] (sn0xefabd0){
\footnotesize{12.5}
\nodepart{two}
\begin{tikzpicture}[scale=.2]
\node[circle, scale=0.75, fill] (tid0) at (2.25,1.5){};
\node[circle, scale=0.75, fill] (tid1) at (2.25,3){};
\node[circle, scale=0.75, fill] (tid2) at (0.75,4.5){};
\node[circle, scale=0.75, fill] (tid5) at (0.75,6){};
\node[circle, scale=0.75, fill] (tid8) at (0.75,7.5){};
\draw[](tid5) -- (tid8);
\draw[](tid2) -- (tid5);
\node[circle, scale=0.75, fill] (tid3) at (2.25,4.5){};
\node[circle, scale=0.75, fill, task_scheduled] (tid6) at (2.25,6){};
\draw[](tid3) -- (tid6);
\node[circle, scale=0.75, fill] (tid4) at (3.75,4.5){};
\node[circle, scale=0.75, fill, task_scheduled] (tid7) at (3.75,6){};
\draw[](tid4) -- (tid7);
\draw[](tid1) -- (tid2);
\draw[](tid1) -- (tid3);
\draw[](tid1) -- (tid4);
\draw[](tid0) -- (tid1);
\end{tikzpicture}
\nodepart{three}
\footnotesize{6.38281}
\nodepart{four}
\footnotesize{$50\:50$}
};
 & 
\node[draw=black, rectangle split,  rectangle split parts=4] (sn0xefb500){
\footnotesize{12.5}
\nodepart{two}
\begin{tikzpicture}[scale=.2]
\node[circle, scale=0.75, fill] (tid0) at (2.25,1.5){};
\node[circle, scale=0.75, fill] (tid1) at (2.25,3){};
\node[circle, scale=0.75, fill] (tid2) at (0.75,4.5){};
\node[circle, scale=0.75, fill] (tid5) at (0.75,6){};
\node[circle, scale=0.75, fill, task_scheduled] (tid8) at (0.75,7.5){};
\draw[](tid5) -- (tid8);
\draw[](tid2) -- (tid5);
\node[circle, scale=0.75, fill] (tid3) at (2.25,4.5){};
\node[circle, scale=0.75, fill, task_scheduled] (tid6) at (2.25,6){};
\draw[](tid3) -- (tid6);
\node[circle, scale=0.75, fill] (tid4) at (3.75,4.5){};
\node[circle, scale=0.75, fill] (tid7) at (3.75,6){};
\draw[](tid4) -- (tid7);
\draw[](tid1) -- (tid2);
\draw[](tid1) -- (tid3);
\draw[](tid1) -- (tid4);
\draw[](tid0) -- (tid1);
\end{tikzpicture}
\nodepart{three}
\footnotesize{6.24023}
\nodepart{four}
\footnotesize{$25\:25\:50$}
};
 & 
\\
};
\end{scope}
\begin{scope}[yshift=\leveltopIIII cm]
\matrix (line4)[column sep=0.5cm] {
\node[draw=black, rectangle split,  rectangle split parts=4] (sn0xef43a0){
\footnotesize{18.75}
\nodepart{two}
\begin{tikzpicture}[scale=.2]
\node[circle, scale=0.75, fill] (tid0) at (1.5,1.5){};
\node[circle, scale=0.75, fill] (tid1) at (1.5,3){};
\node[circle, scale=0.75, fill] (tid2) at (0.75,4.5){};
\node[circle, scale=0.75, fill] (tid4) at (0.75,6){};
\node[circle, scale=0.75, fill] (tid6) at (0.75,7.5){};
\node[circle, scale=0.75, fill, task_scheduled] (tid7) at (0.75,9){};
\draw[](tid6) -- (tid7);
\draw[](tid4) -- (tid6);
\draw[](tid2) -- (tid4);
\node[circle, scale=0.75, fill] (tid3) at (2.25,4.5){};
\node[circle, scale=0.75, fill, task_scheduled] (tid5) at (2.25,6){};
\draw[](tid3) -- (tid5);
\draw[](tid1) -- (tid2);
\draw[](tid1) -- (tid3);
\draw[](tid0) -- (tid1);
\end{tikzpicture}
\nodepart{three}
\footnotesize{6.25}
\nodepart{four}
\footnotesize{$50\:50$}
};
 & 
\node[draw=black, rectangle split,  rectangle split parts=4] (sn0xef4560){
\footnotesize{6.25}
\nodepart{two}
\begin{tikzpicture}[scale=.2]
\node[circle, scale=0.75, fill] (tid0) at (2.25,1.5){};
\node[circle, scale=0.75, fill] (tid1) at (2.25,3){};
\node[circle, scale=0.75, fill] (tid2) at (0.75,4.5){};
\node[circle, scale=0.75, fill] (tid5) at (0.75,6){};
\node[circle, scale=0.75, fill] (tid6) at (0.75,7.5){};
\node[circle, scale=0.75, fill] (tid7) at (0.75,9){};
\draw[](tid6) -- (tid7);
\draw[](tid5) -- (tid6);
\draw[](tid2) -- (tid5);
\node[circle, scale=0.75, fill, task_scheduled] (tid3) at (2.25,4.5){};
\node[circle, scale=0.75, fill, task_scheduled] (tid4) at (3.75,4.5){};
\draw[](tid1) -- (tid2);
\draw[](tid1) -- (tid3);
\draw[](tid1) -- (tid4);
\draw[](tid0) -- (tid1);
\end{tikzpicture}
\nodepart{three}
\footnotesize{6.5625}
\nodepart{four}
\footnotesize{$1$}
};
 & 
\node[draw=black, rectangle split,  rectangle split parts=4] (sn0xef4790){
\footnotesize{25}
\nodepart{two}
\begin{tikzpicture}[scale=.2]
\node[circle, scale=0.75, fill] (tid0) at (2.25,1.5){};
\node[circle, scale=0.75, fill] (tid1) at (2.25,3){};
\node[circle, scale=0.75, fill] (tid2) at (0.75,4.5){};
\node[circle, scale=0.75, fill] (tid5) at (0.75,6){};
\node[circle, scale=0.75, fill] (tid6) at (0.75,7.5){};
\node[circle, scale=0.75, fill, task_scheduled] (tid7) at (0.75,9){};
\draw[](tid6) -- (tid7);
\draw[](tid5) -- (tid6);
\draw[](tid2) -- (tid5);
\node[circle, scale=0.75, fill, task_scheduled] (tid3) at (2.25,4.5){};
\node[circle, scale=0.75, fill] (tid4) at (3.75,4.5){};
\draw[](tid1) -- (tid2);
\draw[](tid1) -- (tid3);
\draw[](tid1) -- (tid4);
\draw[](tid0) -- (tid1);
\end{tikzpicture}
\nodepart{three}
\footnotesize{6.28906}
\nodepart{four}
\footnotesize{$50\:25\:25$}
};
 & 
\node[draw=black, rectangle split,  rectangle split parts=4] (sn0xef8300){
\footnotesize{18.75}
\nodepart{two}
\begin{tikzpicture}[scale=.2]
\node[circle, scale=0.75, fill] (tid0) at (2.25,1.5){};
\node[circle, scale=0.75, fill] (tid1) at (2.25,3){};
\node[circle, scale=0.75, fill] (tid2) at (0.75,4.5){};
\node[circle, scale=0.75, fill] (tid5) at (0.75,6){};
\node[circle, scale=0.75, fill] (tid7) at (0.75,7.5){};
\draw[](tid5) -- (tid7);
\draw[](tid2) -- (tid5);
\node[circle, scale=0.75, fill] (tid3) at (2.25,4.5){};
\node[circle, scale=0.75, fill, task_scheduled] (tid6) at (2.25,6){};
\draw[](tid3) -- (tid6);
\node[circle, scale=0.75, fill, task_scheduled] (tid4) at (3.75,4.5){};
\draw[](tid1) -- (tid2);
\draw[](tid1) -- (tid3);
\draw[](tid1) -- (tid4);
\draw[](tid0) -- (tid1);
\end{tikzpicture}
\nodepart{three}
\footnotesize{5.97656}
\nodepart{four}
\footnotesize{$50\:25\:25$}
};
 & 
\node[draw=black, rectangle split,  rectangle split parts=4] (sn0xef90f0){
\footnotesize{18.75}
\nodepart{two}
\begin{tikzpicture}[scale=.2]
\node[circle, scale=0.75, fill] (tid0) at (2.25,1.5){};
\node[circle, scale=0.75, fill] (tid1) at (2.25,3){};
\node[circle, scale=0.75, fill] (tid2) at (0.75,4.5){};
\node[circle, scale=0.75, fill] (tid5) at (0.75,6){};
\node[circle, scale=0.75, fill, task_scheduled] (tid7) at (0.75,7.5){};
\draw[](tid5) -- (tid7);
\draw[](tid2) -- (tid5);
\node[circle, scale=0.75, fill] (tid3) at (2.25,4.5){};
\node[circle, scale=0.75, fill, task_scheduled] (tid6) at (2.25,6){};
\draw[](tid3) -- (tid6);
\node[circle, scale=0.75, fill] (tid4) at (3.75,4.5){};
\draw[](tid1) -- (tid2);
\draw[](tid1) -- (tid3);
\draw[](tid1) -- (tid4);
\draw[](tid0) -- (tid1);
\end{tikzpicture}
\nodepart{three}
\footnotesize{5.78906}
\nodepart{four}
\footnotesize{$50\:25\:25$}
};
 & 
\node[draw=black, rectangle split,  rectangle split parts=4] (sn0xefbc60){
\footnotesize{6.25}
\nodepart{two}
\begin{tikzpicture}[scale=.2]
\node[circle, scale=0.75, fill] (tid0) at (2.25,1.5){};
\node[circle, scale=0.75, fill] (tid1) at (2.25,3){};
\node[circle, scale=0.75, fill] (tid2) at (0.75,4.5){};
\node[circle, scale=0.75, fill] (tid5) at (0.75,6){};
\node[circle, scale=0.75, fill, task_scheduled] (tid7) at (0.75,7.5){};
\draw[](tid5) -- (tid7);
\draw[](tid2) -- (tid5);
\node[circle, scale=0.75, fill] (tid3) at (2.25,4.5){};
\node[circle, scale=0.75, fill] (tid6) at (2.25,6){};
\draw[](tid3) -- (tid6);
\node[circle, scale=0.75, fill, task_scheduled] (tid4) at (3.75,4.5){};
\draw[](tid1) -- (tid2);
\draw[](tid1) -- (tid3);
\draw[](tid1) -- (tid4);
\draw[](tid0) -- (tid1);
\end{tikzpicture}
\nodepart{three}
\footnotesize{5.82812}
\nodepart{four}
\footnotesize{$50\:50$}
};
 & 
\node[draw=black, rectangle split,  rectangle split parts=4] (sn0xefc0b0){
\footnotesize{6.25}
\nodepart{two}
\begin{tikzpicture}[scale=.2]
\node[circle, scale=0.75, fill] (tid0) at (2.25,1.5){};
\node[circle, scale=0.75, fill] (tid1) at (2.25,3){};
\node[circle, scale=0.75, fill] (tid2) at (0.75,4.5){};
\node[circle, scale=0.75, fill, task_scheduled] (tid5) at (0.75,6){};
\draw[](tid2) -- (tid5);
\node[circle, scale=0.75, fill] (tid3) at (2.25,4.5){};
\node[circle, scale=0.75, fill, task_scheduled] (tid6) at (2.25,6){};
\draw[](tid3) -- (tid6);
\node[circle, scale=0.75, fill] (tid4) at (3.75,4.5){};
\node[circle, scale=0.75, fill] (tid7) at (3.75,6){};
\draw[](tid4) -- (tid7);
\draw[](tid1) -- (tid2);
\draw[](tid1) -- (tid3);
\draw[](tid1) -- (tid4);
\draw[](tid0) -- (tid1);
\end{tikzpicture}
\nodepart{three}
\footnotesize{5.67188}
\nodepart{four}
\footnotesize{$50\:50$}
};
 & 
\\
};
\end{scope}
\begin{scope}[yshift=\leveltopIIIII cm]
\matrix (line5)[column sep=0.5cm] {
\node[draw=black, rectangle split,  rectangle split parts=4] (sn0xef48f0){
\footnotesize{28.125}
\nodepart{two}
\begin{tikzpicture}[scale=.2]
\node[circle, scale=0.75, fill] (tid0) at (1.5,1.5){};
\node[circle, scale=0.75, fill] (tid1) at (1.5,3){};
\node[circle, scale=0.75, fill] (tid2) at (0.75,4.5){};
\node[circle, scale=0.75, fill] (tid4) at (0.75,6){};
\node[circle, scale=0.75, fill] (tid5) at (0.75,7.5){};
\node[circle, scale=0.75, fill, task_scheduled] (tid6) at (0.75,9){};
\draw[](tid5) -- (tid6);
\draw[](tid4) -- (tid5);
\draw[](tid2) -- (tid4);
\node[circle, scale=0.75, fill, task_scheduled] (tid3) at (2.25,4.5){};
\draw[](tid1) -- (tid2);
\draw[](tid1) -- (tid3);
\draw[](tid0) -- (tid1);
\end{tikzpicture}
\nodepart{three}
\footnotesize{6.0625}
\nodepart{four}
\footnotesize{$50\:50$}
};
 & 
\node[draw=black, rectangle split,  rectangle split parts=4] (sn0xef5110){
\footnotesize{21.875}
\nodepart{two}
\begin{tikzpicture}[scale=.2]
\node[circle, scale=0.75, fill] (tid0) at (1.5,1.5){};
\node[circle, scale=0.75, fill] (tid1) at (1.5,3){};
\node[circle, scale=0.75, fill] (tid2) at (0.75,4.5){};
\node[circle, scale=0.75, fill] (tid4) at (0.75,6){};
\node[circle, scale=0.75, fill, task_scheduled] (tid6) at (0.75,7.5){};
\draw[](tid4) -- (tid6);
\draw[](tid2) -- (tid4);
\node[circle, scale=0.75, fill] (tid3) at (2.25,4.5){};
\node[circle, scale=0.75, fill, task_scheduled] (tid5) at (2.25,6){};
\draw[](tid3) -- (tid5);
\draw[](tid1) -- (tid2);
\draw[](tid1) -- (tid3);
\draw[](tid0) -- (tid1);
\end{tikzpicture}
\nodepart{three}
\footnotesize{5.4375}
\nodepart{four}
\footnotesize{$50\:50$}
};
 & 
\node[draw=black, rectangle split,  rectangle split parts=4] (sn0xef7880){
\footnotesize{10.9375}
\nodepart{two}
\begin{tikzpicture}[scale=.2]
\node[circle, scale=0.75, fill] (tid0) at (2.25,1.5){};
\node[circle, scale=0.75, fill] (tid1) at (2.25,3){};
\node[circle, scale=0.75, fill] (tid2) at (0.75,4.5){};
\node[circle, scale=0.75, fill] (tid5) at (0.75,6){};
\node[circle, scale=0.75, fill] (tid6) at (0.75,7.5){};
\draw[](tid5) -- (tid6);
\draw[](tid2) -- (tid5);
\node[circle, scale=0.75, fill, task_scheduled] (tid3) at (2.25,4.5){};
\node[circle, scale=0.75, fill, task_scheduled] (tid4) at (3.75,4.5){};
\draw[](tid1) -- (tid2);
\draw[](tid1) -- (tid3);
\draw[](tid1) -- (tid4);
\draw[](tid0) -- (tid1);
\end{tikzpicture}
\nodepart{three}
\footnotesize{5.625}
\nodepart{four}
\footnotesize{$1$}
};
 & 
\node[draw=black, rectangle split,  rectangle split parts=4] (sn0xef7c10){
\footnotesize{20.3125}
\nodepart{two}
\begin{tikzpicture}[scale=.2]
\node[circle, scale=0.75, fill] (tid0) at (2.25,1.5){};
\node[circle, scale=0.75, fill] (tid1) at (2.25,3){};
\node[circle, scale=0.75, fill] (tid2) at (0.75,4.5){};
\node[circle, scale=0.75, fill] (tid5) at (0.75,6){};
\node[circle, scale=0.75, fill, task_scheduled] (tid6) at (0.75,7.5){};
\draw[](tid5) -- (tid6);
\draw[](tid2) -- (tid5);
\node[circle, scale=0.75, fill, task_scheduled] (tid3) at (2.25,4.5){};
\node[circle, scale=0.75, fill] (tid4) at (3.75,4.5){};
\draw[](tid1) -- (tid2);
\draw[](tid1) -- (tid3);
\draw[](tid1) -- (tid4);
\draw[](tid0) -- (tid1);
\end{tikzpicture}
\nodepart{three}
\footnotesize{5.40625}
\nodepart{four}
\footnotesize{$50\:25\:25$}
};
 & 
\node[draw=black, rectangle split,  rectangle split parts=4] (sn0xef9e10){
\footnotesize{10.9375}
\nodepart{two}
\begin{tikzpicture}[scale=.2]
\node[circle, scale=0.75, fill] (tid0) at (2.25,1.5){};
\node[circle, scale=0.75, fill] (tid1) at (2.25,3){};
\node[circle, scale=0.75, fill] (tid2) at (0.75,4.5){};
\node[circle, scale=0.75, fill, task_scheduled] (tid5) at (0.75,6){};
\draw[](tid2) -- (tid5);
\node[circle, scale=0.75, fill] (tid3) at (2.25,4.5){};
\node[circle, scale=0.75, fill] (tid6) at (2.25,6){};
\draw[](tid3) -- (tid6);
\node[circle, scale=0.75, fill, task_scheduled] (tid4) at (3.75,4.5){};
\draw[](tid1) -- (tid2);
\draw[](tid1) -- (tid3);
\draw[](tid1) -- (tid4);
\draw[](tid0) -- (tid1);
\end{tikzpicture}
\nodepart{three}
\footnotesize{5.21875}
\nodepart{four}
\footnotesize{$50\:25\:25$}
};
 & 
\node[draw=black, rectangle split,  rectangle split parts=4] (sn0xefa190){
\footnotesize{7.8125}
\nodepart{two}
\begin{tikzpicture}[scale=.2]
\node[circle, scale=0.75, fill] (tid0) at (2.25,1.5){};
\node[circle, scale=0.75, fill] (tid1) at (2.25,3){};
\node[circle, scale=0.75, fill] (tid2) at (0.75,4.5){};
\node[circle, scale=0.75, fill, task_scheduled] (tid5) at (0.75,6){};
\draw[](tid2) -- (tid5);
\node[circle, scale=0.75, fill] (tid3) at (2.25,4.5){};
\node[circle, scale=0.75, fill, task_scheduled] (tid6) at (2.25,6){};
\draw[](tid3) -- (tid6);
\node[circle, scale=0.75, fill] (tid4) at (3.75,4.5){};
\draw[](tid1) -- (tid2);
\draw[](tid1) -- (tid3);
\draw[](tid1) -- (tid4);
\draw[](tid0) -- (tid1);
\end{tikzpicture}
\nodepart{three}
\footnotesize{5.125}
\nodepart{four}
\footnotesize{$1$}
};
 & 
\\
};
\end{scope}
\begin{scope}[yshift=\leveltopIIIIII cm]
\matrix (line6)[column sep=0.5cm] {
\node[draw=black, rectangle split,  rectangle split parts=4] (sn0xef5310){
\footnotesize{14.0625}
\nodepart{two}
\begin{tikzpicture}[scale=.2]
\node[circle, scale=0.75, fill] (tid0) at (0.75,1.5){};
\node[circle, scale=0.75, fill] (tid1) at (0.75,3){};
\node[circle, scale=0.75, fill] (tid2) at (0.75,4.5){};
\node[circle, scale=0.75, fill] (tid3) at (0.75,6){};
\node[circle, scale=0.75, fill] (tid4) at (0.75,7.5){};
\node[circle, scale=0.75, fill, task_scheduled] (tid5) at (0.75,9){};
\draw[](tid4) -- (tid5);
\draw[](tid3) -- (tid4);
\draw[](tid2) -- (tid3);
\draw[](tid1) -- (tid2);
\draw[](tid0) -- (tid1);
\end{tikzpicture}
\nodepart{three}
\footnotesize{6}
\nodepart{four}
\footnotesize{$1$}
};
 & 
\node[draw=black, rectangle split,  rectangle split parts=4] (sn0xef5750){
\footnotesize{46.0938}
\nodepart{two}
\begin{tikzpicture}[scale=.2]
\node[circle, scale=0.75, fill] (tid0) at (1.5,1.5){};
\node[circle, scale=0.75, fill] (tid1) at (1.5,3){};
\node[circle, scale=0.75, fill] (tid2) at (0.75,4.5){};
\node[circle, scale=0.75, fill] (tid4) at (0.75,6){};
\node[circle, scale=0.75, fill, task_scheduled] (tid5) at (0.75,7.5){};
\draw[](tid4) -- (tid5);
\draw[](tid2) -- (tid4);
\node[circle, scale=0.75, fill, task_scheduled] (tid3) at (2.25,4.5){};
\draw[](tid1) -- (tid2);
\draw[](tid1) -- (tid3);
\draw[](tid0) -- (tid1);
\end{tikzpicture}
\nodepart{three}
\footnotesize{5.125}
\nodepart{four}
\footnotesize{$50\:50$}
};
 & 
\node[draw=black, rectangle split,  rectangle split parts=4] (sn0xef7040){
\footnotesize{16.4062}
\nodepart{two}
\begin{tikzpicture}[scale=.2]
\node[circle, scale=0.75, fill] (tid0) at (1.5,1.5){};
\node[circle, scale=0.75, fill] (tid1) at (1.5,3){};
\node[circle, scale=0.75, fill] (tid2) at (0.75,4.5){};
\node[circle, scale=0.75, fill, task_scheduled] (tid4) at (0.75,6){};
\draw[](tid2) -- (tid4);
\node[circle, scale=0.75, fill] (tid3) at (2.25,4.5){};
\node[circle, scale=0.75, fill, task_scheduled] (tid5) at (2.25,6){};
\draw[](tid3) -- (tid5);
\draw[](tid1) -- (tid2);
\draw[](tid1) -- (tid3);
\draw[](tid0) -- (tid1);
\end{tikzpicture}
\nodepart{three}
\footnotesize{4.75}
\nodepart{four}
\footnotesize{$1$}
};
 & 
\node[draw=black, rectangle split,  rectangle split parts=4] (sn0xef7d10){
\footnotesize{7.8125}
\nodepart{two}
\begin{tikzpicture}[scale=.2]
\node[circle, scale=0.75, fill] (tid0) at (2.25,1.5){};
\node[circle, scale=0.75, fill] (tid1) at (2.25,3){};
\node[circle, scale=0.75, fill] (tid2) at (0.75,4.5){};
\node[circle, scale=0.75, fill] (tid5) at (0.75,6){};
\draw[](tid2) -- (tid5);
\node[circle, scale=0.75, fill, task_scheduled] (tid3) at (2.25,4.5){};
\node[circle, scale=0.75, fill, task_scheduled] (tid4) at (3.75,4.5){};
\draw[](tid1) -- (tid2);
\draw[](tid1) -- (tid3);
\draw[](tid1) -- (tid4);
\draw[](tid0) -- (tid1);
\end{tikzpicture}
\nodepart{three}
\footnotesize{4.75}
\nodepart{four}
\footnotesize{$1$}
};
 & 
\node[draw=black, rectangle split,  rectangle split parts=4] (sn0xef8590){
\footnotesize{15.625}
\nodepart{two}
\begin{tikzpicture}[scale=.2]
\node[circle, scale=0.75, fill] (tid0) at (2.25,1.5){};
\node[circle, scale=0.75, fill] (tid1) at (2.25,3){};
\node[circle, scale=0.75, fill] (tid2) at (0.75,4.5){};
\node[circle, scale=0.75, fill, task_scheduled] (tid5) at (0.75,6){};
\draw[](tid2) -- (tid5);
\node[circle, scale=0.75, fill, task_scheduled] (tid3) at (2.25,4.5){};
\node[circle, scale=0.75, fill] (tid4) at (3.75,4.5){};
\draw[](tid1) -- (tid2);
\draw[](tid1) -- (tid3);
\draw[](tid1) -- (tid4);
\draw[](tid0) -- (tid1);
\end{tikzpicture}
\nodepart{three}
\footnotesize{4.625}
\nodepart{four}
\footnotesize{$50\:50$}
};
 & 
\\
};
\end{scope}
\begin{scope}[yshift=\leveltopIIIIIII cm]
\matrix (line7)[column sep=0.5cm] {
\node[draw=black, rectangle split,  rectangle split parts=4] (sn0xef5850){
\footnotesize{37.1094}
\nodepart{two}
\begin{tikzpicture}[scale=.2]
\node[circle, scale=0.75, fill] (tid0) at (0.75,1.5){};
\node[circle, scale=0.75, fill] (tid1) at (0.75,3){};
\node[circle, scale=0.75, fill] (tid2) at (0.75,4.5){};
\node[circle, scale=0.75, fill] (tid3) at (0.75,6){};
\node[circle, scale=0.75, fill, task_scheduled] (tid4) at (0.75,7.5){};
\draw[](tid3) -- (tid4);
\draw[](tid2) -- (tid3);
\draw[](tid1) -- (tid2);
\draw[](tid0) -- (tid1);
\end{tikzpicture}
\nodepart{three}
\footnotesize{5}
\nodepart{four}
\footnotesize{$1$}
};
 & 
\node[draw=black, rectangle split,  rectangle split parts=4] (sn0xef64d0){
\footnotesize{55.0781}
\nodepart{two}
\begin{tikzpicture}[scale=.2]
\node[circle, scale=0.75, fill] (tid0) at (1.5,1.5){};
\node[circle, scale=0.75, fill] (tid1) at (1.5,3){};
\node[circle, scale=0.75, fill] (tid2) at (0.75,4.5){};
\node[circle, scale=0.75, fill, task_scheduled] (tid4) at (0.75,6){};
\draw[](tid2) -- (tid4);
\node[circle, scale=0.75, fill, task_scheduled] (tid3) at (2.25,4.5){};
\draw[](tid1) -- (tid2);
\draw[](tid1) -- (tid3);
\draw[](tid0) -- (tid1);
\end{tikzpicture}
\nodepart{three}
\footnotesize{4.25}
\nodepart{four}
\footnotesize{$50\:50$}
};
 & 
\node[draw=black, rectangle split,  rectangle split parts=4] (sn0xef8750){
\footnotesize{7.8125}
\nodepart{two}
\begin{tikzpicture}[scale=.2]
\node[circle, scale=0.75, fill] (tid0) at (2.25,1.5){};
\node[circle, scale=0.75, fill] (tid1) at (2.25,3){};
\node[circle, scale=0.75, fill, task_scheduled] (tid2) at (0.75,4.5){};
\node[circle, scale=0.75, fill, task_scheduled] (tid3) at (2.25,4.5){};
\node[circle, scale=0.75, fill] (tid4) at (3.75,4.5){};
\draw[](tid1) -- (tid2);
\draw[](tid1) -- (tid3);
\draw[](tid1) -- (tid4);
\draw[](tid0) -- (tid1);
\end{tikzpicture}
\nodepart{three}
\footnotesize{4}
\nodepart{four}
\footnotesize{$1$}
};
 & 
\\
};
\end{scope}
\begin{scope}[yshift=\leveltopIIIIIIII cm]
\matrix (line8)[column sep=0.5cm] {
\node[draw=black, rectangle split,  rectangle split parts=4] (sn0xef5c90){
\footnotesize{64.6484}
\nodepart{two}
\begin{tikzpicture}[scale=.2]
\node[circle, scale=0.75, fill] (tid0) at (0.75,1.5){};
\node[circle, scale=0.75, fill] (tid1) at (0.75,3){};
\node[circle, scale=0.75, fill] (tid2) at (0.75,4.5){};
\node[circle, scale=0.75, fill, task_scheduled] (tid3) at (0.75,6){};
\draw[](tid2) -- (tid3);
\draw[](tid1) -- (tid2);
\draw[](tid0) -- (tid1);
\end{tikzpicture}
\nodepart{three}
\footnotesize{4}
\nodepart{four}
\footnotesize{$1$}
};
 & 
\node[draw=black, rectangle split,  rectangle split parts=4] (sn0xef6940){
\footnotesize{35.3516}
\nodepart{two}
\begin{tikzpicture}[scale=.2]
\node[circle, scale=0.75, fill] (tid0) at (1.5,1.5){};
\node[circle, scale=0.75, fill] (tid1) at (1.5,3){};
\node[circle, scale=0.75, fill, task_scheduled] (tid2) at (0.75,4.5){};
\node[circle, scale=0.75, fill, task_scheduled] (tid3) at (2.25,4.5){};
\draw[](tid1) -- (tid2);
\draw[](tid1) -- (tid3);
\draw[](tid0) -- (tid1);
\end{tikzpicture}
\nodepart{three}
\footnotesize{3.5}
\nodepart{four}
\footnotesize{$1$}
};
 & 
\\
};
\end{scope}
\begin{scope}[yshift=\leveltopIIIIIIIII cm]
\matrix (line9)[column sep=0.5cm] {
\node[draw=black, rectangle split,  rectangle split parts=4] (sn0xef5950){
\footnotesize{100}
\nodepart{two}
\begin{tikzpicture}[scale=.2]
\node[circle, scale=0.75, fill] (tid0) at (0.75,1.5){};
\node[circle, scale=0.75, fill] (tid1) at (0.75,3){};
\node[circle, scale=0.75, fill, task_scheduled] (tid2) at (0.75,4.5){};
\draw[](tid1) -- (tid2);
\draw[](tid0) -- (tid1);
\end{tikzpicture}
\nodepart{three}
\footnotesize{3}
\nodepart{four}
\footnotesize{$1$}
};
 & 
\\
};
\end{scope}
\begin{scope}[yshift=\leveltopIIIIIIIIII cm]
\matrix (line10)[column sep=0.5cm] {
\node[draw=black, rectangle split,  rectangle split parts=4] (sn0xef5b30){
\footnotesize{100}
\nodepart{two}
\begin{tikzpicture}[scale=.2]
\node[circle, scale=0.75, fill] (tid0) at (0.75,1.5){};
\node[circle, scale=0.75, fill, task_scheduled] (tid1) at (0.75,3){};
\draw[](tid0) -- (tid1);
\end{tikzpicture}
\nodepart{three}
\footnotesize{2}
\nodepart{four}
\footnotesize{$1$}
};
 & 
\\
};
\end{scope}
\draw (sn0xef11d0.south) -- (sn0xef2a50.north);
\draw (sn0xef11d0.south) -- (sn0xef2540.north);
\draw (sn0xef2a50.south) -- (sn0xef2f20.north);
\draw (sn0xef2a50.south) -- (sn0xef3e70.north);
\draw (sn0xef2540.south) -- (sn0xefa920.north);
\draw (sn0xef2540.south) -- (sn0xef3e70.north);
\draw (sn0xef2540.south) -- (sn0xefabd0.north);
\draw (sn0xef2540.south) -- (sn0xefb500.north);
\draw (sn0xef2f20.south) -- (sn0xef43a0.north);
\draw (sn0xef2f20.south) -- (sn0xef4560.north);
\draw (sn0xef2f20.south) -- (sn0xef4790.north);
\draw (sn0xef3e70.south) -- (sn0xef4790.north);
\draw (sn0xef3e70.south) -- (sn0xef8300.north);
\draw (sn0xef3e70.south) -- (sn0xef90f0.north);
\draw (sn0xefa920.south) -- (sn0xef43a0.north);
\draw (sn0xefa920.south) -- (sn0xef8300.north);
\draw (sn0xefa920.south) -- (sn0xefbc60.north);
\draw (sn0xefabd0.south) -- (sn0xef8300.north);
\draw (sn0xefabd0.south) -- (sn0xef90f0.north);
\draw (sn0xefb500.south) -- (sn0xefbc60.north);
\draw (sn0xefb500.south) -- (sn0xef90f0.north);
\draw (sn0xefb500.south) -- (sn0xefc0b0.north);
\draw (sn0xef43a0.south) -- (sn0xef48f0.north);
\draw (sn0xef43a0.south) -- (sn0xef5110.north);
\draw (sn0xef4560.south) -- (sn0xef48f0.north);
\draw (sn0xef4790.south) -- (sn0xef48f0.north);
\draw (sn0xef4790.south) -- (sn0xef7880.north);
\draw (sn0xef4790.south) -- (sn0xef7c10.north);
\draw (sn0xef8300.south) -- (sn0xef5110.north);
\draw (sn0xef8300.south) -- (sn0xef7880.north);
\draw (sn0xef8300.south) -- (sn0xef7c10.north);
\draw (sn0xef90f0.south) -- (sn0xef7c10.north);
\draw (sn0xef90f0.south) -- (sn0xef9e10.north);
\draw (sn0xef90f0.south) -- (sn0xefa190.north);
\draw (sn0xefbc60.south) -- (sn0xef5110.north);
\draw (sn0xefbc60.south) -- (sn0xef9e10.north);
\draw (sn0xefc0b0.south) -- (sn0xef9e10.north);
\draw (sn0xefc0b0.south) -- (sn0xefa190.north);
\draw (sn0xef48f0.south) -- (sn0xef5310.north);
\draw (sn0xef48f0.south) -- (sn0xef5750.north);
\draw (sn0xef5110.south) -- (sn0xef5750.north);
\draw (sn0xef5110.south) -- (sn0xef7040.north);
\draw (sn0xef7880.south) -- (sn0xef5750.north);
\draw (sn0xef7c10.south) -- (sn0xef5750.north);
\draw (sn0xef7c10.south) -- (sn0xef7d10.north);
\draw (sn0xef7c10.south) -- (sn0xef8590.north);
\draw (sn0xef9e10.south) -- (sn0xef7040.north);
\draw (sn0xef9e10.south) -- (sn0xef7d10.north);
\draw (sn0xef9e10.south) -- (sn0xef8590.north);
\draw (sn0xefa190.south) -- (sn0xef8590.north);
\draw (sn0xef5310.south) -- (sn0xef5850.north);
\draw (sn0xef5750.south) -- (sn0xef5850.north);
\draw (sn0xef5750.south) -- (sn0xef64d0.north);
\draw (sn0xef7040.south) -- (sn0xef64d0.north);
\draw (sn0xef7d10.south) -- (sn0xef64d0.north);
\draw (sn0xef8590.south) -- (sn0xef64d0.north);
\draw (sn0xef8590.south) -- (sn0xef8750.north);
\draw (sn0xef5850.south) -- (sn0xef5c90.north);
\draw (sn0xef64d0.south) -- (sn0xef5c90.north);
\draw (sn0xef64d0.south) -- (sn0xef6940.north);
\draw (sn0xef8750.south) -- (sn0xef6940.north);
\draw (sn0xef5c90.south) -- (sn0xef5950.north);
\draw (sn0xef6940.south) -- (sn0xef5950.north);
\draw (sn0xef5950.south) -- (sn0xef5b30.north);
\end{tikzpicture}

%%% Local Variables:
%%% TeX-master: "thesis/thesis.tex"
%%% End: 
\renewcommand{\leveltopI}{-15cm + \leveltop}
\renewcommand{\leveltopII}{-15cm + \leveltopI}
\renewcommand{\leveltopIII}{-15cm + \leveltopII}
\renewcommand{\leveltopIIII}{-15cm + \leveltopIII}
\renewcommand{\leveltopIIIII}{-15cm + \leveltopIIII}
\renewcommand{\leveltopIIIIII}{-15cm + \leveltopIIIII}
\renewcommand{\leveltopIIIIIII}{-15cm + \leveltopIIIIII}
\renewcommand{\leveltopIIIIIIII}{-15cm + \leveltopIIIIIII}
\renewcommand{\leveltopIIIIIIIII}{-15cm + \leveltopIIIIIIII}
\renewcommand{\leveltopIIIIIIIIII}{-15cm + \leveltopIIIIIIIII}
\renewcommand{\leveltopIIIIIIIIIII}{-15cm + \leveltopIIIIIIIIII}
\begin{tikzpicture}[scale=.2, anchor=south]
\begin{scope}[yshift=\leveltopI cm]
\matrix (line1)[column sep=0.5cm] {
\node[draw=black, rectangle split,  rectangle split parts=4] (sn0xef1700){
\footnotesize{100}
\nodepart{two}
\begin{tikzpicture}[scale=.2]
\node[circle, scale=0.75, fill] (tid0) at (3,1.5){};
\node[circle, scale=0.75, fill] (tid1) at (3,3){};
\node[circle, scale=0.75, fill] (tid2) at (0.75,4.5){};
\node[circle, scale=0.75, fill] (tid5) at (0.75,6){};
\node[circle, scale=0.75, fill] (tid9) at (0.75,7.5){};
\node[circle, scale=0.75, fill] (tid10) at (0.75,9){};
\draw[](tid9) -- (tid10);
\draw[](tid5) -- (tid9);
\draw[](tid2) -- (tid5);
\node[circle, scale=0.75, fill] (tid3) at (3,4.5){};
\node[circle, scale=0.75, fill, task_scheduled] (tid6) at (2.25,6){};
\node[circle, scale=0.75, fill] (tid7) at (3.75,6){};
\draw[](tid3) -- (tid6);
\draw[](tid3) -- (tid7);
\node[circle, scale=0.75, fill] (tid4) at (5.25,4.5){};
\node[circle, scale=0.75, fill, task_scheduled] (tid8) at (5.25,6){};
\draw[](tid4) -- (tid8);
\draw[](tid1) -- (tid2);
\draw[](tid1) -- (tid3);
\draw[](tid1) -- (tid4);
\draw[](tid0) -- (tid1);
\end{tikzpicture}
\nodepart{three}
\footnotesize{7.59942}
\nodepart{four}
\footnotesize{$17\:17\:17\:25\:25$}
};
 & 
\\
};
\end{scope}
\begin{scope}[yshift=\leveltopII cm]
\matrix (line2)[column sep=0.5cm] {
\node[draw=black, rectangle split,  rectangle split parts=4] (sn0xefc570){
\footnotesize{16.6667}
\nodepart{two}
\begin{tikzpicture}[scale=.2]
\node[circle, scale=0.75, fill] (tid0) at (3,1.5){};
\node[circle, scale=0.75, fill] (tid1) at (3,3){};
\node[circle, scale=0.75, fill] (tid2) at (0.75,4.5){};
\node[circle, scale=0.75, fill] (tid5) at (0.75,6){};
\node[circle, scale=0.75, fill] (tid8) at (0.75,7.5){};
\node[circle, scale=0.75, fill] (tid9) at (0.75,9){};
\draw[](tid8) -- (tid9);
\draw[](tid5) -- (tid8);
\draw[](tid2) -- (tid5);
\node[circle, scale=0.75, fill] (tid3) at (3,4.5){};
\node[circle, scale=0.75, fill, task_scheduled] (tid6) at (2.25,6){};
\node[circle, scale=0.75, fill] (tid7) at (3.75,6){};
\draw[](tid3) -- (tid6);
\draw[](tid3) -- (tid7);
\node[circle, scale=0.75, fill, task_scheduled] (tid4) at (5.25,4.5){};
\draw[](tid1) -- (tid2);
\draw[](tid1) -- (tid3);
\draw[](tid1) -- (tid4);
\draw[](tid0) -- (tid1);
\end{tikzpicture}
\nodepart{three}
\footnotesize{7.18408}
\nodepart{four}
\footnotesize{$25\:25\:25\:25$}
};
 & 
\node[draw=black, rectangle split,  rectangle split parts=4] (sn0xefc970){
\footnotesize{16.6667}
\nodepart{two}
\begin{tikzpicture}[scale=.2]
\node[circle, scale=0.75, fill] (tid0) at (3,1.5){};
\node[circle, scale=0.75, fill] (tid1) at (3,3){};
\node[circle, scale=0.75, fill] (tid2) at (0.75,4.5){};
\node[circle, scale=0.75, fill] (tid5) at (0.75,6){};
\node[circle, scale=0.75, fill] (tid8) at (0.75,7.5){};
\node[circle, scale=0.75, fill] (tid9) at (0.75,9){};
\draw[](tid8) -- (tid9);
\draw[](tid5) -- (tid8);
\draw[](tid2) -- (tid5);
\node[circle, scale=0.75, fill] (tid3) at (3,4.5){};
\node[circle, scale=0.75, fill, task_scheduled] (tid6) at (2.25,6){};
\node[circle, scale=0.75, fill, task_scheduled] (tid7) at (3.75,6){};
\draw[](tid3) -- (tid6);
\draw[](tid3) -- (tid7);
\node[circle, scale=0.75, fill] (tid4) at (5.25,4.5){};
\draw[](tid1) -- (tid2);
\draw[](tid1) -- (tid3);
\draw[](tid1) -- (tid4);
\draw[](tid0) -- (tid1);
\end{tikzpicture}
\nodepart{three}
\footnotesize{7.21191}
\nodepart{four}
\footnotesize{$50\:50$}
};
 & 
\node[draw=black, rectangle split,  rectangle split parts=4] (sn0xefcc90){
\footnotesize{16.6667}
\nodepart{two}
\begin{tikzpicture}[scale=.2]
\node[circle, scale=0.75, fill] (tid0) at (3,1.5){};
\node[circle, scale=0.75, fill] (tid1) at (3,3){};
\node[circle, scale=0.75, fill] (tid2) at (0.75,4.5){};
\node[circle, scale=0.75, fill] (tid5) at (0.75,6){};
\node[circle, scale=0.75, fill] (tid8) at (0.75,7.5){};
\node[circle, scale=0.75, fill, task_scheduled] (tid9) at (0.75,9){};
\draw[](tid8) -- (tid9);
\draw[](tid5) -- (tid8);
\draw[](tid2) -- (tid5);
\node[circle, scale=0.75, fill] (tid3) at (3,4.5){};
\node[circle, scale=0.75, fill, task_scheduled] (tid6) at (2.25,6){};
\node[circle, scale=0.75, fill] (tid7) at (3.75,6){};
\draw[](tid3) -- (tid6);
\draw[](tid3) -- (tid7);
\node[circle, scale=0.75, fill] (tid4) at (5.25,4.5){};
\draw[](tid1) -- (tid2);
\draw[](tid1) -- (tid3);
\draw[](tid1) -- (tid4);
\draw[](tid0) -- (tid1);
\end{tikzpicture}
\nodepart{three}
\footnotesize{6.96323}
\nodepart{four}
\footnotesize{$25\:25\:17\:17\:17$}
};
 & 
\node[draw=black, rectangle split,  rectangle split parts=4] (sn0xef2a50){
\footnotesize{25}
\nodepart{two}
\begin{tikzpicture}[scale=.2]
\node[circle, scale=0.75, fill] (tid0) at (2.25,1.5){};
\node[circle, scale=0.75, fill] (tid1) at (2.25,3){};
\node[circle, scale=0.75, fill] (tid2) at (0.75,4.5){};
\node[circle, scale=0.75, fill] (tid5) at (0.75,6){};
\node[circle, scale=0.75, fill] (tid8) at (0.75,7.5){};
\node[circle, scale=0.75, fill] (tid9) at (0.75,9){};
\draw[](tid8) -- (tid9);
\draw[](tid5) -- (tid8);
\draw[](tid2) -- (tid5);
\node[circle, scale=0.75, fill] (tid3) at (2.25,4.5){};
\node[circle, scale=0.75, fill, task_scheduled] (tid6) at (2.25,6){};
\draw[](tid3) -- (tid6);
\node[circle, scale=0.75, fill] (tid4) at (3.75,4.5){};
\node[circle, scale=0.75, fill, task_scheduled] (tid7) at (3.75,6){};
\draw[](tid4) -- (tid7);
\draw[](tid1) -- (tid2);
\draw[](tid1) -- (tid3);
\draw[](tid1) -- (tid4);
\draw[](tid0) -- (tid1);
\end{tikzpicture}
\nodepart{three}
\footnotesize{7.21191}
\nodepart{four}
\footnotesize{$50\:50$}
};
 & 
\node[draw=black, rectangle split,  rectangle split parts=4] (sn0xef2540){
\footnotesize{25}
\nodepart{two}
\begin{tikzpicture}[scale=.2]
\node[circle, scale=0.75, fill] (tid0) at (2.25,1.5){};
\node[circle, scale=0.75, fill] (tid1) at (2.25,3){};
\node[circle, scale=0.75, fill] (tid2) at (0.75,4.5){};
\node[circle, scale=0.75, fill] (tid5) at (0.75,6){};
\node[circle, scale=0.75, fill] (tid8) at (0.75,7.5){};
\node[circle, scale=0.75, fill, task_scheduled] (tid9) at (0.75,9){};
\draw[](tid8) -- (tid9);
\draw[](tid5) -- (tid8);
\draw[](tid2) -- (tid5);
\node[circle, scale=0.75, fill] (tid3) at (2.25,4.5){};
\node[circle, scale=0.75, fill, task_scheduled] (tid6) at (2.25,6){};
\draw[](tid3) -- (tid6);
\node[circle, scale=0.75, fill] (tid4) at (3.75,4.5){};
\node[circle, scale=0.75, fill] (tid7) at (3.75,6){};
\draw[](tid4) -- (tid7);
\draw[](tid1) -- (tid2);
\draw[](tid1) -- (tid3);
\draw[](tid1) -- (tid4);
\draw[](tid0) -- (tid1);
\end{tikzpicture}
\nodepart{three}
\footnotesize{6.94629}
\nodepart{four}
\footnotesize{$25\:25\:25\:25$}
};
 & 
\\
};
\end{scope}
\begin{scope}[yshift=\leveltopIII cm]
\matrix (line3)[column sep=0.5cm] {
\node[draw=black, rectangle split,  rectangle split parts=4] (sn0xefd2a0){
\footnotesize{4.16667}
\nodepart{two}
\begin{tikzpicture}[scale=.2]
\node[circle, scale=0.75, fill] (tid0) at (2.25,1.5){};
\node[circle, scale=0.75, fill] (tid1) at (2.25,3){};
\node[circle, scale=0.75, fill] (tid2) at (0.75,4.5){};
\node[circle, scale=0.75, fill] (tid4) at (0.75,6){};
\node[circle, scale=0.75, fill] (tid7) at (0.75,7.5){};
\node[circle, scale=0.75, fill] (tid8) at (0.75,9){};
\draw[](tid7) -- (tid8);
\draw[](tid4) -- (tid7);
\draw[](tid2) -- (tid4);
\node[circle, scale=0.75, fill] (tid3) at (3,4.5){};
\node[circle, scale=0.75, fill, task_scheduled] (tid5) at (2.25,6){};
\node[circle, scale=0.75, fill, task_scheduled] (tid6) at (3.75,6){};
\draw[](tid3) -- (tid5);
\draw[](tid3) -- (tid6);
\draw[](tid1) -- (tid2);
\draw[](tid1) -- (tid3);
\draw[](tid0) -- (tid1);
\end{tikzpicture}
\nodepart{three}
\footnotesize{6.75}
\nodepart{four}
\footnotesize{$1$}
};
 & 
\node[draw=black, rectangle split,  rectangle split parts=4] (sn0xefe2f0){
\footnotesize{4.16667}
\nodepart{two}
\begin{tikzpicture}[scale=.2]
\node[circle, scale=0.75, fill] (tid0) at (2.25,1.5){};
\node[circle, scale=0.75, fill] (tid1) at (2.25,3){};
\node[circle, scale=0.75, fill] (tid2) at (0.75,4.5){};
\node[circle, scale=0.75, fill] (tid4) at (0.75,6){};
\node[circle, scale=0.75, fill] (tid7) at (0.75,7.5){};
\node[circle, scale=0.75, fill, task_scheduled] (tid8) at (0.75,9){};
\draw[](tid7) -- (tid8);
\draw[](tid4) -- (tid7);
\draw[](tid2) -- (tid4);
\node[circle, scale=0.75, fill] (tid3) at (3,4.5){};
\node[circle, scale=0.75, fill, task_scheduled] (tid5) at (2.25,6){};
\node[circle, scale=0.75, fill] (tid6) at (3.75,6){};
\draw[](tid3) -- (tid5);
\draw[](tid3) -- (tid6);
\draw[](tid1) -- (tid2);
\draw[](tid1) -- (tid3);
\draw[](tid0) -- (tid1);
\end{tikzpicture}
\nodepart{three}
\footnotesize{6.57227}
\nodepart{four}
\footnotesize{$50\:25\:25$}
};
 & 
\node[draw=black, rectangle split,  rectangle split parts=4] (sn0xef2f20){
\footnotesize{25}
\nodepart{two}
\begin{tikzpicture}[scale=.2]
\node[circle, scale=0.75, fill] (tid0) at (2.25,1.5){};
\node[circle, scale=0.75, fill] (tid1) at (2.25,3){};
\node[circle, scale=0.75, fill] (tid2) at (0.75,4.5){};
\node[circle, scale=0.75, fill] (tid5) at (0.75,6){};
\node[circle, scale=0.75, fill] (tid7) at (0.75,7.5){};
\node[circle, scale=0.75, fill] (tid8) at (0.75,9){};
\draw[](tid7) -- (tid8);
\draw[](tid5) -- (tid7);
\draw[](tid2) -- (tid5);
\node[circle, scale=0.75, fill] (tid3) at (2.25,4.5){};
\node[circle, scale=0.75, fill, task_scheduled] (tid6) at (2.25,6){};
\draw[](tid3) -- (tid6);
\node[circle, scale=0.75, fill, task_scheduled] (tid4) at (3.75,4.5){};
\draw[](tid1) -- (tid2);
\draw[](tid1) -- (tid3);
\draw[](tid1) -- (tid4);
\draw[](tid0) -- (tid1);
\end{tikzpicture}
\nodepart{three}
\footnotesize{6.83789}
\nodepart{four}
\footnotesize{$50\:25\:25$}
};
 & 
\node[draw=black, rectangle split,  rectangle split parts=4] (sn0xef3e70){
\footnotesize{31.25}
\nodepart{two}
\begin{tikzpicture}[scale=.2]
\node[circle, scale=0.75, fill] (tid0) at (2.25,1.5){};
\node[circle, scale=0.75, fill] (tid1) at (2.25,3){};
\node[circle, scale=0.75, fill] (tid2) at (0.75,4.5){};
\node[circle, scale=0.75, fill] (tid5) at (0.75,6){};
\node[circle, scale=0.75, fill] (tid7) at (0.75,7.5){};
\node[circle, scale=0.75, fill, task_scheduled] (tid8) at (0.75,9){};
\draw[](tid7) -- (tid8);
\draw[](tid5) -- (tid7);
\draw[](tid2) -- (tid5);
\node[circle, scale=0.75, fill] (tid3) at (2.25,4.5){};
\node[circle, scale=0.75, fill, task_scheduled] (tid6) at (2.25,6){};
\draw[](tid3) -- (tid6);
\node[circle, scale=0.75, fill] (tid4) at (3.75,4.5){};
\draw[](tid1) -- (tid2);
\draw[](tid1) -- (tid3);
\draw[](tid1) -- (tid4);
\draw[](tid0) -- (tid1);
\end{tikzpicture}
\nodepart{three}
\footnotesize{6.58594}
\nodepart{four}
\footnotesize{$50\:25\:25$}
};
 & 
\node[draw=black, rectangle split,  rectangle split parts=4] (sn0xefa920){
\footnotesize{14.5833}
\nodepart{two}
\begin{tikzpicture}[scale=.2]
\node[circle, scale=0.75, fill] (tid0) at (2.25,1.5){};
\node[circle, scale=0.75, fill] (tid1) at (2.25,3){};
\node[circle, scale=0.75, fill] (tid2) at (0.75,4.5){};
\node[circle, scale=0.75, fill] (tid5) at (0.75,6){};
\node[circle, scale=0.75, fill] (tid7) at (0.75,7.5){};
\node[circle, scale=0.75, fill, task_scheduled] (tid8) at (0.75,9){};
\draw[](tid7) -- (tid8);
\draw[](tid5) -- (tid7);
\draw[](tid2) -- (tid5);
\node[circle, scale=0.75, fill] (tid3) at (2.25,4.5){};
\node[circle, scale=0.75, fill] (tid6) at (2.25,6){};
\draw[](tid3) -- (tid6);
\node[circle, scale=0.75, fill, task_scheduled] (tid4) at (3.75,4.5){};
\draw[](tid1) -- (tid2);
\draw[](tid1) -- (tid3);
\draw[](tid1) -- (tid4);
\draw[](tid0) -- (tid1);
\end{tikzpicture}
\nodepart{three}
\footnotesize{6.57617}
\nodepart{four}
\footnotesize{$50\:25\:25$}
};
 & 
\node[draw=black, rectangle split,  rectangle split parts=4] (sn0xefabd0){
\footnotesize{6.25}
\nodepart{two}
\begin{tikzpicture}[scale=.2]
\node[circle, scale=0.75, fill] (tid0) at (2.25,1.5){};
\node[circle, scale=0.75, fill] (tid1) at (2.25,3){};
\node[circle, scale=0.75, fill] (tid2) at (0.75,4.5){};
\node[circle, scale=0.75, fill] (tid5) at (0.75,6){};
\node[circle, scale=0.75, fill] (tid8) at (0.75,7.5){};
\draw[](tid5) -- (tid8);
\draw[](tid2) -- (tid5);
\node[circle, scale=0.75, fill] (tid3) at (2.25,4.5){};
\node[circle, scale=0.75, fill, task_scheduled] (tid6) at (2.25,6){};
\draw[](tid3) -- (tid6);
\node[circle, scale=0.75, fill] (tid4) at (3.75,4.5){};
\node[circle, scale=0.75, fill, task_scheduled] (tid7) at (3.75,6){};
\draw[](tid4) -- (tid7);
\draw[](tid1) -- (tid2);
\draw[](tid1) -- (tid3);
\draw[](tid1) -- (tid4);
\draw[](tid0) -- (tid1);
\end{tikzpicture}
\nodepart{three}
\footnotesize{6.38281}
\nodepart{four}
\footnotesize{$50\:50$}
};
 & 
\node[draw=black, rectangle split,  rectangle split parts=4] (sn0xefb500){
\footnotesize{6.25}
\nodepart{two}
\begin{tikzpicture}[scale=.2]
\node[circle, scale=0.75, fill] (tid0) at (2.25,1.5){};
\node[circle, scale=0.75, fill] (tid1) at (2.25,3){};
\node[circle, scale=0.75, fill] (tid2) at (0.75,4.5){};
\node[circle, scale=0.75, fill] (tid5) at (0.75,6){};
\node[circle, scale=0.75, fill, task_scheduled] (tid8) at (0.75,7.5){};
\draw[](tid5) -- (tid8);
\draw[](tid2) -- (tid5);
\node[circle, scale=0.75, fill] (tid3) at (2.25,4.5){};
\node[circle, scale=0.75, fill, task_scheduled] (tid6) at (2.25,6){};
\draw[](tid3) -- (tid6);
\node[circle, scale=0.75, fill] (tid4) at (3.75,4.5){};
\node[circle, scale=0.75, fill] (tid7) at (3.75,6){};
\draw[](tid4) -- (tid7);
\draw[](tid1) -- (tid2);
\draw[](tid1) -- (tid3);
\draw[](tid1) -- (tid4);
\draw[](tid0) -- (tid1);
\end{tikzpicture}
\nodepart{three}
\footnotesize{6.24023}
\nodepart{four}
\footnotesize{$25\:25\:50$}
};
 & 
\node[draw=black, rectangle split,  rectangle split parts=4] (sn0xeffff0){
\footnotesize{2.77778}
\nodepart{two}
\begin{tikzpicture}[scale=.2]
\node[circle, scale=0.75, fill] (tid0) at (3,1.5){};
\node[circle, scale=0.75, fill] (tid1) at (3,3){};
\node[circle, scale=0.75, fill] (tid2) at (1.5,4.5){};
\node[circle, scale=0.75, fill, task_scheduled] (tid5) at (0.75,6){};
\node[circle, scale=0.75, fill] (tid6) at (2.25,6){};
\draw[](tid2) -- (tid5);
\draw[](tid2) -- (tid6);
\node[circle, scale=0.75, fill] (tid3) at (3.75,4.5){};
\node[circle, scale=0.75, fill] (tid7) at (3.75,6){};
\node[circle, scale=0.75, fill] (tid8) at (3.75,7.5){};
\draw[](tid7) -- (tid8);
\draw[](tid3) -- (tid7);
\node[circle, scale=0.75, fill, task_scheduled] (tid4) at (5.25,4.5){};
\draw[](tid1) -- (tid2);
\draw[](tid1) -- (tid3);
\draw[](tid1) -- (tid4);
\draw[](tid0) -- (tid1);
\end{tikzpicture}
\nodepart{three}
\footnotesize{6.39844}
\nodepart{four}
\footnotesize{$25\:25\:25\:25$}
};
 & 
\node[draw=black, rectangle split,  rectangle split parts=4] (sn0xf00d20){
\footnotesize{2.77778}
\nodepart{two}
\begin{tikzpicture}[scale=.2]
\node[circle, scale=0.75, fill] (tid0) at (3,1.5){};
\node[circle, scale=0.75, fill] (tid1) at (3,3){};
\node[circle, scale=0.75, fill] (tid2) at (1.5,4.5){};
\node[circle, scale=0.75, fill, task_scheduled] (tid5) at (0.75,6){};
\node[circle, scale=0.75, fill, task_scheduled] (tid6) at (2.25,6){};
\draw[](tid2) -- (tid5);
\draw[](tid2) -- (tid6);
\node[circle, scale=0.75, fill] (tid3) at (3.75,4.5){};
\node[circle, scale=0.75, fill] (tid7) at (3.75,6){};
\node[circle, scale=0.75, fill] (tid8) at (3.75,7.5){};
\draw[](tid7) -- (tid8);
\draw[](tid3) -- (tid7);
\node[circle, scale=0.75, fill] (tid4) at (5.25,4.5){};
\draw[](tid1) -- (tid2);
\draw[](tid1) -- (tid3);
\draw[](tid1) -- (tid4);
\draw[](tid0) -- (tid1);
\end{tikzpicture}
\nodepart{three}
\footnotesize{6.38281}
\nodepart{four}
\footnotesize{$50\:50$}
};
 & 
\node[draw=black, rectangle split,  rectangle split parts=4] (sn0xf01470){
\footnotesize{2.77778}
\nodepart{two}
\begin{tikzpicture}[scale=.2]
\node[circle, scale=0.75, fill] (tid0) at (3,1.5){};
\node[circle, scale=0.75, fill] (tid1) at (3,3){};
\node[circle, scale=0.75, fill] (tid2) at (1.5,4.5){};
\node[circle, scale=0.75, fill, task_scheduled] (tid5) at (0.75,6){};
\node[circle, scale=0.75, fill] (tid6) at (2.25,6){};
\draw[](tid2) -- (tid5);
\draw[](tid2) -- (tid6);
\node[circle, scale=0.75, fill] (tid3) at (3.75,4.5){};
\node[circle, scale=0.75, fill] (tid7) at (3.75,6){};
\node[circle, scale=0.75, fill, task_scheduled] (tid8) at (3.75,7.5){};
\draw[](tid7) -- (tid8);
\draw[](tid3) -- (tid7);
\node[circle, scale=0.75, fill] (tid4) at (5.25,4.5){};
\draw[](tid1) -- (tid2);
\draw[](tid1) -- (tid3);
\draw[](tid1) -- (tid4);
\draw[](tid0) -- (tid1);
\end{tikzpicture}
\nodepart{three}
\footnotesize{6.25499}
\nodepart{four}
\footnotesize{$25\:25\:17\:17\:17$}
};
 & 
\\
};
\end{scope}
\begin{scope}[yshift=\leveltopIIII cm]
\matrix (line4)[column sep=0.5cm] {
\node[draw=black, rectangle split,  rectangle split parts=4] (sn0xef43a0){
\footnotesize{26.0417}
\nodepart{two}
\begin{tikzpicture}[scale=.2]
\node[circle, scale=0.75, fill] (tid0) at (1.5,1.5){};
\node[circle, scale=0.75, fill] (tid1) at (1.5,3){};
\node[circle, scale=0.75, fill] (tid2) at (0.75,4.5){};
\node[circle, scale=0.75, fill] (tid4) at (0.75,6){};
\node[circle, scale=0.75, fill] (tid6) at (0.75,7.5){};
\node[circle, scale=0.75, fill, task_scheduled] (tid7) at (0.75,9){};
\draw[](tid6) -- (tid7);
\draw[](tid4) -- (tid6);
\draw[](tid2) -- (tid4);
\node[circle, scale=0.75, fill] (tid3) at (2.25,4.5){};
\node[circle, scale=0.75, fill, task_scheduled] (tid5) at (2.25,6){};
\draw[](tid3) -- (tid5);
\draw[](tid1) -- (tid2);
\draw[](tid1) -- (tid3);
\draw[](tid0) -- (tid1);
\end{tikzpicture}
\nodepart{three}
\footnotesize{6.25}
\nodepart{four}
\footnotesize{$50\:50$}
};
 & 
\node[draw=black, rectangle split,  rectangle split parts=4] (sn0xef4560){
\footnotesize{6.25}
\nodepart{two}
\begin{tikzpicture}[scale=.2]
\node[circle, scale=0.75, fill] (tid0) at (2.25,1.5){};
\node[circle, scale=0.75, fill] (tid1) at (2.25,3){};
\node[circle, scale=0.75, fill] (tid2) at (0.75,4.5){};
\node[circle, scale=0.75, fill] (tid5) at (0.75,6){};
\node[circle, scale=0.75, fill] (tid6) at (0.75,7.5){};
\node[circle, scale=0.75, fill] (tid7) at (0.75,9){};
\draw[](tid6) -- (tid7);
\draw[](tid5) -- (tid6);
\draw[](tid2) -- (tid5);
\node[circle, scale=0.75, fill, task_scheduled] (tid3) at (2.25,4.5){};
\node[circle, scale=0.75, fill, task_scheduled] (tid4) at (3.75,4.5){};
\draw[](tid1) -- (tid2);
\draw[](tid1) -- (tid3);
\draw[](tid1) -- (tid4);
\draw[](tid0) -- (tid1);
\end{tikzpicture}
\nodepart{three}
\footnotesize{6.5625}
\nodepart{four}
\footnotesize{$1$}
};
 & 
\node[draw=black, rectangle split,  rectangle split parts=4] (sn0xef4790){
\footnotesize{21.875}
\nodepart{two}
\begin{tikzpicture}[scale=.2]
\node[circle, scale=0.75, fill] (tid0) at (2.25,1.5){};
\node[circle, scale=0.75, fill] (tid1) at (2.25,3){};
\node[circle, scale=0.75, fill] (tid2) at (0.75,4.5){};
\node[circle, scale=0.75, fill] (tid5) at (0.75,6){};
\node[circle, scale=0.75, fill] (tid6) at (0.75,7.5){};
\node[circle, scale=0.75, fill, task_scheduled] (tid7) at (0.75,9){};
\draw[](tid6) -- (tid7);
\draw[](tid5) -- (tid6);
\draw[](tid2) -- (tid5);
\node[circle, scale=0.75, fill, task_scheduled] (tid3) at (2.25,4.5){};
\node[circle, scale=0.75, fill] (tid4) at (3.75,4.5){};
\draw[](tid1) -- (tid2);
\draw[](tid1) -- (tid3);
\draw[](tid1) -- (tid4);
\draw[](tid0) -- (tid1);
\end{tikzpicture}
\nodepart{three}
\footnotesize{6.28906}
\nodepart{four}
\footnotesize{$50\:25\:25$}
};
 & 
\node[draw=black, rectangle split,  rectangle split parts=4] (sn0xef8300){
\footnotesize{16.6667}
\nodepart{two}
\begin{tikzpicture}[scale=.2]
\node[circle, scale=0.75, fill] (tid0) at (2.25,1.5){};
\node[circle, scale=0.75, fill] (tid1) at (2.25,3){};
\node[circle, scale=0.75, fill] (tid2) at (0.75,4.5){};
\node[circle, scale=0.75, fill] (tid5) at (0.75,6){};
\node[circle, scale=0.75, fill] (tid7) at (0.75,7.5){};
\draw[](tid5) -- (tid7);
\draw[](tid2) -- (tid5);
\node[circle, scale=0.75, fill] (tid3) at (2.25,4.5){};
\node[circle, scale=0.75, fill, task_scheduled] (tid6) at (2.25,6){};
\draw[](tid3) -- (tid6);
\node[circle, scale=0.75, fill, task_scheduled] (tid4) at (3.75,4.5){};
\draw[](tid1) -- (tid2);
\draw[](tid1) -- (tid3);
\draw[](tid1) -- (tid4);
\draw[](tid0) -- (tid1);
\end{tikzpicture}
\nodepart{three}
\footnotesize{5.97656}
\nodepart{four}
\footnotesize{$50\:25\:25$}
};
 & 
\node[draw=black, rectangle split,  rectangle split parts=4] (sn0xef90f0){
\footnotesize{14.5833}
\nodepart{two}
\begin{tikzpicture}[scale=.2]
\node[circle, scale=0.75, fill] (tid0) at (2.25,1.5){};
\node[circle, scale=0.75, fill] (tid1) at (2.25,3){};
\node[circle, scale=0.75, fill] (tid2) at (0.75,4.5){};
\node[circle, scale=0.75, fill] (tid5) at (0.75,6){};
\node[circle, scale=0.75, fill, task_scheduled] (tid7) at (0.75,7.5){};
\draw[](tid5) -- (tid7);
\draw[](tid2) -- (tid5);
\node[circle, scale=0.75, fill] (tid3) at (2.25,4.5){};
\node[circle, scale=0.75, fill, task_scheduled] (tid6) at (2.25,6){};
\draw[](tid3) -- (tid6);
\node[circle, scale=0.75, fill] (tid4) at (3.75,4.5){};
\draw[](tid1) -- (tid2);
\draw[](tid1) -- (tid3);
\draw[](tid1) -- (tid4);
\draw[](tid0) -- (tid1);
\end{tikzpicture}
\nodepart{three}
\footnotesize{5.78906}
\nodepart{four}
\footnotesize{$50\:25\:25$}
};
 & 
\node[draw=black, rectangle split,  rectangle split parts=4] (sn0xefbc60){
\footnotesize{6.59722}
\nodepart{two}
\begin{tikzpicture}[scale=.2]
\node[circle, scale=0.75, fill] (tid0) at (2.25,1.5){};
\node[circle, scale=0.75, fill] (tid1) at (2.25,3){};
\node[circle, scale=0.75, fill] (tid2) at (0.75,4.5){};
\node[circle, scale=0.75, fill] (tid5) at (0.75,6){};
\node[circle, scale=0.75, fill, task_scheduled] (tid7) at (0.75,7.5){};
\draw[](tid5) -- (tid7);
\draw[](tid2) -- (tid5);
\node[circle, scale=0.75, fill] (tid3) at (2.25,4.5){};
\node[circle, scale=0.75, fill] (tid6) at (2.25,6){};
\draw[](tid3) -- (tid6);
\node[circle, scale=0.75, fill, task_scheduled] (tid4) at (3.75,4.5){};
\draw[](tid1) -- (tid2);
\draw[](tid1) -- (tid3);
\draw[](tid1) -- (tid4);
\draw[](tid0) -- (tid1);
\end{tikzpicture}
\nodepart{three}
\footnotesize{5.82812}
\nodepart{four}
\footnotesize{$50\:50$}
};
 & 
\node[draw=black, rectangle split,  rectangle split parts=4] (sn0xefe720){
\footnotesize{1.73611}
\nodepart{two}
\begin{tikzpicture}[scale=.2]
\node[circle, scale=0.75, fill] (tid0) at (2.25,1.5){};
\node[circle, scale=0.75, fill] (tid1) at (2.25,3){};
\node[circle, scale=0.75, fill] (tid2) at (1.5,4.5){};
\node[circle, scale=0.75, fill, task_scheduled] (tid4) at (0.75,6){};
\node[circle, scale=0.75, fill, task_scheduled] (tid5) at (2.25,6){};
\draw[](tid2) -- (tid4);
\draw[](tid2) -- (tid5);
\node[circle, scale=0.75, fill] (tid3) at (3.75,4.5){};
\node[circle, scale=0.75, fill] (tid6) at (3.75,6){};
\node[circle, scale=0.75, fill] (tid7) at (3.75,7.5){};
\draw[](tid6) -- (tid7);
\draw[](tid3) -- (tid6);
\draw[](tid1) -- (tid2);
\draw[](tid1) -- (tid3);
\draw[](tid0) -- (tid1);
\end{tikzpicture}
\nodepart{three}
\footnotesize{5.9375}
\nodepart{four}
\footnotesize{$1$}
};
 & 
\node[draw=black, rectangle split,  rectangle split parts=4] (sn0xefea20){
\footnotesize{1.73611}
\nodepart{two}
\begin{tikzpicture}[scale=.2]
\node[circle, scale=0.75, fill] (tid0) at (2.25,1.5){};
\node[circle, scale=0.75, fill] (tid1) at (2.25,3){};
\node[circle, scale=0.75, fill] (tid2) at (1.5,4.5){};
\node[circle, scale=0.75, fill, task_scheduled] (tid4) at (0.75,6){};
\node[circle, scale=0.75, fill] (tid5) at (2.25,6){};
\draw[](tid2) -- (tid4);
\draw[](tid2) -- (tid5);
\node[circle, scale=0.75, fill] (tid3) at (3.75,4.5){};
\node[circle, scale=0.75, fill] (tid6) at (3.75,6){};
\node[circle, scale=0.75, fill, task_scheduled] (tid7) at (3.75,7.5){};
\draw[](tid6) -- (tid7);
\draw[](tid3) -- (tid6);
\draw[](tid1) -- (tid2);
\draw[](tid1) -- (tid3);
\draw[](tid0) -- (tid1);
\end{tikzpicture}
\nodepart{three}
\footnotesize{5.85156}
\nodepart{four}
\footnotesize{$50\:25\:25$}
};
 & 
\node[draw=black, rectangle split,  rectangle split parts=4] (sn0xf01940){
\footnotesize{0.462963}
\nodepart{two}
\begin{tikzpicture}[scale=.2]
\node[circle, scale=0.75, fill] (tid0) at (3,1.5){};
\node[circle, scale=0.75, fill] (tid1) at (3,3){};
\node[circle, scale=0.75, fill] (tid2) at (1.5,4.5){};
\node[circle, scale=0.75, fill, task_scheduled] (tid5) at (0.75,6){};
\node[circle, scale=0.75, fill] (tid6) at (2.25,6){};
\draw[](tid2) -- (tid5);
\draw[](tid2) -- (tid6);
\node[circle, scale=0.75, fill] (tid3) at (3.75,4.5){};
\node[circle, scale=0.75, fill] (tid7) at (3.75,6){};
\draw[](tid3) -- (tid7);
\node[circle, scale=0.75, fill, task_scheduled] (tid4) at (5.25,4.5){};
\draw[](tid1) -- (tid2);
\draw[](tid1) -- (tid3);
\draw[](tid1) -- (tid4);
\draw[](tid0) -- (tid1);
\end{tikzpicture}
\nodepart{three}
\footnotesize{5.74219}
\nodepart{four}
\footnotesize{$25\:25\:50$}
};
 & 
\node[draw=black, rectangle split,  rectangle split parts=4] (sn0xf01e70){
\footnotesize{0.462963}
\nodepart{two}
\begin{tikzpicture}[scale=.2]
\node[circle, scale=0.75, fill] (tid0) at (3,1.5){};
\node[circle, scale=0.75, fill] (tid1) at (3,3){};
\node[circle, scale=0.75, fill] (tid2) at (1.5,4.5){};
\node[circle, scale=0.75, fill, task_scheduled] (tid5) at (0.75,6){};
\node[circle, scale=0.75, fill, task_scheduled] (tid6) at (2.25,6){};
\draw[](tid2) -- (tid5);
\draw[](tid2) -- (tid6);
\node[circle, scale=0.75, fill] (tid3) at (3.75,4.5){};
\node[circle, scale=0.75, fill] (tid7) at (3.75,6){};
\draw[](tid3) -- (tid7);
\node[circle, scale=0.75, fill] (tid4) at (5.25,4.5){};
\draw[](tid1) -- (tid2);
\draw[](tid1) -- (tid3);
\draw[](tid1) -- (tid4);
\draw[](tid0) -- (tid1);
\end{tikzpicture}
\nodepart{three}
\footnotesize{5.67188}
\nodepart{four}
\footnotesize{$50\:50$}
};
 & 
\node[draw=black, rectangle split,  rectangle split parts=4] (sn0xf02290){
\footnotesize{0.462963}
\nodepart{two}
\begin{tikzpicture}[scale=.2]
\node[circle, scale=0.75, fill] (tid0) at (3,1.5){};
\node[circle, scale=0.75, fill] (tid1) at (3,3){};
\node[circle, scale=0.75, fill] (tid2) at (1.5,4.5){};
\node[circle, scale=0.75, fill, task_scheduled] (tid5) at (0.75,6){};
\node[circle, scale=0.75, fill] (tid6) at (2.25,6){};
\draw[](tid2) -- (tid5);
\draw[](tid2) -- (tid6);
\node[circle, scale=0.75, fill] (tid3) at (3.75,4.5){};
\node[circle, scale=0.75, fill, task_scheduled] (tid7) at (3.75,6){};
\draw[](tid3) -- (tid7);
\node[circle, scale=0.75, fill] (tid4) at (5.25,4.5){};
\draw[](tid1) -- (tid2);
\draw[](tid1) -- (tid3);
\draw[](tid1) -- (tid4);
\draw[](tid0) -- (tid1);
\end{tikzpicture}
\nodepart{three}
\footnotesize{5.6901}
\nodepart{four}
\footnotesize{$33\:17\:25\:25$}
};
 & 
\node[draw=black, rectangle split,  rectangle split parts=4] (sn0xefc0b0){
\footnotesize{3.125}
\nodepart{two}
\begin{tikzpicture}[scale=.2]
\node[circle, scale=0.75, fill] (tid0) at (2.25,1.5){};
\node[circle, scale=0.75, fill] (tid1) at (2.25,3){};
\node[circle, scale=0.75, fill] (tid2) at (0.75,4.5){};
\node[circle, scale=0.75, fill, task_scheduled] (tid5) at (0.75,6){};
\draw[](tid2) -- (tid5);
\node[circle, scale=0.75, fill] (tid3) at (2.25,4.5){};
\node[circle, scale=0.75, fill, task_scheduled] (tid6) at (2.25,6){};
\draw[](tid3) -- (tid6);
\node[circle, scale=0.75, fill] (tid4) at (3.75,4.5){};
\node[circle, scale=0.75, fill] (tid7) at (3.75,6){};
\draw[](tid4) -- (tid7);
\draw[](tid1) -- (tid2);
\draw[](tid1) -- (tid3);
\draw[](tid1) -- (tid4);
\draw[](tid0) -- (tid1);
\end{tikzpicture}
\nodepart{three}
\footnotesize{5.67188}
\nodepart{four}
\footnotesize{$50\:50$}
};
 & 
\\
};
\end{scope}
\begin{scope}[yshift=\leveltopIIIII cm]
\matrix (line5)[column sep=0.5cm] {
\node[draw=black, rectangle split,  rectangle split parts=4] (sn0xef48f0){
\footnotesize{30.2083}
\nodepart{two}
\begin{tikzpicture}[scale=.2]
\node[circle, scale=0.75, fill] (tid0) at (1.5,1.5){};
\node[circle, scale=0.75, fill] (tid1) at (1.5,3){};
\node[circle, scale=0.75, fill] (tid2) at (0.75,4.5){};
\node[circle, scale=0.75, fill] (tid4) at (0.75,6){};
\node[circle, scale=0.75, fill] (tid5) at (0.75,7.5){};
\node[circle, scale=0.75, fill, task_scheduled] (tid6) at (0.75,9){};
\draw[](tid5) -- (tid6);
\draw[](tid4) -- (tid5);
\draw[](tid2) -- (tid4);
\node[circle, scale=0.75, fill, task_scheduled] (tid3) at (2.25,4.5){};
\draw[](tid1) -- (tid2);
\draw[](tid1) -- (tid3);
\draw[](tid0) -- (tid1);
\end{tikzpicture}
\nodepart{three}
\footnotesize{6.0625}
\nodepart{four}
\footnotesize{$50\:50$}
};
 & 
\node[draw=black, rectangle split,  rectangle split parts=4] (sn0xef5110){
\footnotesize{27.2569}
\nodepart{two}
\begin{tikzpicture}[scale=.2]
\node[circle, scale=0.75, fill] (tid0) at (1.5,1.5){};
\node[circle, scale=0.75, fill] (tid1) at (1.5,3){};
\node[circle, scale=0.75, fill] (tid2) at (0.75,4.5){};
\node[circle, scale=0.75, fill] (tid4) at (0.75,6){};
\node[circle, scale=0.75, fill, task_scheduled] (tid6) at (0.75,7.5){};
\draw[](tid4) -- (tid6);
\draw[](tid2) -- (tid4);
\node[circle, scale=0.75, fill] (tid3) at (2.25,4.5){};
\node[circle, scale=0.75, fill, task_scheduled] (tid5) at (2.25,6){};
\draw[](tid3) -- (tid5);
\draw[](tid1) -- (tid2);
\draw[](tid1) -- (tid3);
\draw[](tid0) -- (tid1);
\end{tikzpicture}
\nodepart{three}
\footnotesize{5.4375}
\nodepart{four}
\footnotesize{$50\:50$}
};
 & 
\node[draw=black, rectangle split,  rectangle split parts=4] (sn0xef7880){
\footnotesize{9.63542}
\nodepart{two}
\begin{tikzpicture}[scale=.2]
\node[circle, scale=0.75, fill] (tid0) at (2.25,1.5){};
\node[circle, scale=0.75, fill] (tid1) at (2.25,3){};
\node[circle, scale=0.75, fill] (tid2) at (0.75,4.5){};
\node[circle, scale=0.75, fill] (tid5) at (0.75,6){};
\node[circle, scale=0.75, fill] (tid6) at (0.75,7.5){};
\draw[](tid5) -- (tid6);
\draw[](tid2) -- (tid5);
\node[circle, scale=0.75, fill, task_scheduled] (tid3) at (2.25,4.5){};
\node[circle, scale=0.75, fill, task_scheduled] (tid4) at (3.75,4.5){};
\draw[](tid1) -- (tid2);
\draw[](tid1) -- (tid3);
\draw[](tid1) -- (tid4);
\draw[](tid0) -- (tid1);
\end{tikzpicture}
\nodepart{three}
\footnotesize{5.625}
\nodepart{four}
\footnotesize{$1$}
};
 & 
\node[draw=black, rectangle split,  rectangle split parts=4] (sn0xef7c10){
\footnotesize{16.9271}
\nodepart{two}
\begin{tikzpicture}[scale=.2]
\node[circle, scale=0.75, fill] (tid0) at (2.25,1.5){};
\node[circle, scale=0.75, fill] (tid1) at (2.25,3){};
\node[circle, scale=0.75, fill] (tid2) at (0.75,4.5){};
\node[circle, scale=0.75, fill] (tid5) at (0.75,6){};
\node[circle, scale=0.75, fill, task_scheduled] (tid6) at (0.75,7.5){};
\draw[](tid5) -- (tid6);
\draw[](tid2) -- (tid5);
\node[circle, scale=0.75, fill, task_scheduled] (tid3) at (2.25,4.5){};
\node[circle, scale=0.75, fill] (tid4) at (3.75,4.5){};
\draw[](tid1) -- (tid2);
\draw[](tid1) -- (tid3);
\draw[](tid1) -- (tid4);
\draw[](tid0) -- (tid1);
\end{tikzpicture}
\nodepart{three}
\footnotesize{5.40625}
\nodepart{four}
\footnotesize{$50\:25\:25$}
};
 & 
\node[draw=black, rectangle split,  rectangle split parts=4] (sn0xefeea0){
\footnotesize{0.549769}
\nodepart{two}
\begin{tikzpicture}[scale=.2]
\node[circle, scale=0.75, fill] (tid0) at (2.25,1.5){};
\node[circle, scale=0.75, fill] (tid1) at (2.25,3){};
\node[circle, scale=0.75, fill] (tid2) at (1.5,4.5){};
\node[circle, scale=0.75, fill, task_scheduled] (tid4) at (0.75,6){};
\node[circle, scale=0.75, fill, task_scheduled] (tid5) at (2.25,6){};
\draw[](tid2) -- (tid4);
\draw[](tid2) -- (tid5);
\node[circle, scale=0.75, fill] (tid3) at (3.75,4.5){};
\node[circle, scale=0.75, fill] (tid6) at (3.75,6){};
\draw[](tid3) -- (tid6);
\draw[](tid1) -- (tid2);
\draw[](tid1) -- (tid3);
\draw[](tid0) -- (tid1);
\end{tikzpicture}
\nodepart{three}
\footnotesize{5.25}
\nodepart{four}
\footnotesize{$1$}
};
 & 
\node[draw=black, rectangle split,  rectangle split parts=4] (sn0xeff210){
\footnotesize{0.549769}
\nodepart{two}
\begin{tikzpicture}[scale=.2]
\node[circle, scale=0.75, fill] (tid0) at (2.25,1.5){};
\node[circle, scale=0.75, fill] (tid1) at (2.25,3){};
\node[circle, scale=0.75, fill] (tid2) at (1.5,4.5){};
\node[circle, scale=0.75, fill, task_scheduled] (tid4) at (0.75,6){};
\node[circle, scale=0.75, fill] (tid5) at (2.25,6){};
\draw[](tid2) -- (tid4);
\draw[](tid2) -- (tid5);
\node[circle, scale=0.75, fill] (tid3) at (3.75,4.5){};
\node[circle, scale=0.75, fill, task_scheduled] (tid6) at (3.75,6){};
\draw[](tid3) -- (tid6);
\draw[](tid1) -- (tid2);
\draw[](tid1) -- (tid3);
\draw[](tid0) -- (tid1);
\end{tikzpicture}
\nodepart{three}
\footnotesize{5.28125}
\nodepart{four}
\footnotesize{$25\:25\:50$}
};
 & 
\node[draw=black, rectangle split,  rectangle split parts=4] (sn0xf026b0){
\footnotesize{0.154321}
\nodepart{two}
\begin{tikzpicture}[scale=.2]
\node[circle, scale=0.75, fill] (tid0) at (3,1.5){};
\node[circle, scale=0.75, fill] (tid1) at (3,3){};
\node[circle, scale=0.75, fill] (tid2) at (1.5,4.5){};
\node[circle, scale=0.75, fill, task_scheduled] (tid5) at (0.75,6){};
\node[circle, scale=0.75, fill] (tid6) at (2.25,6){};
\draw[](tid2) -- (tid5);
\draw[](tid2) -- (tid6);
\node[circle, scale=0.75, fill, task_scheduled] (tid3) at (3.75,4.5){};
\node[circle, scale=0.75, fill] (tid4) at (5.25,4.5){};
\draw[](tid1) -- (tid2);
\draw[](tid1) -- (tid3);
\draw[](tid1) -- (tid4);
\draw[](tid0) -- (tid1);
\end{tikzpicture}
\nodepart{three}
\footnotesize{5.25}
\nodepart{four}
\footnotesize{$25\:25\:25\:25$}
};
 & 
\node[draw=black, rectangle split,  rectangle split parts=4] (sn0xf02bb0){
\footnotesize{0.0771605}
\nodepart{two}
\begin{tikzpicture}[scale=.2]
\node[circle, scale=0.75, fill] (tid0) at (3,1.5){};
\node[circle, scale=0.75, fill] (tid1) at (3,3){};
\node[circle, scale=0.75, fill] (tid2) at (1.5,4.5){};
\node[circle, scale=0.75, fill, task_scheduled] (tid5) at (0.75,6){};
\node[circle, scale=0.75, fill, task_scheduled] (tid6) at (2.25,6){};
\draw[](tid2) -- (tid5);
\draw[](tid2) -- (tid6);
\node[circle, scale=0.75, fill] (tid3) at (3.75,4.5){};
\node[circle, scale=0.75, fill] (tid4) at (5.25,4.5){};
\draw[](tid1) -- (tid2);
\draw[](tid1) -- (tid3);
\draw[](tid1) -- (tid4);
\draw[](tid0) -- (tid1);
\end{tikzpicture}
\nodepart{three}
\footnotesize{5.125}
\nodepart{four}
\footnotesize{$1$}
};
 & 
\node[draw=black, rectangle split,  rectangle split parts=4] (sn0xef9e10){
\footnotesize{9.08565}
\nodepart{two}
\begin{tikzpicture}[scale=.2]
\node[circle, scale=0.75, fill] (tid0) at (2.25,1.5){};
\node[circle, scale=0.75, fill] (tid1) at (2.25,3){};
\node[circle, scale=0.75, fill] (tid2) at (0.75,4.5){};
\node[circle, scale=0.75, fill, task_scheduled] (tid5) at (0.75,6){};
\draw[](tid2) -- (tid5);
\node[circle, scale=0.75, fill] (tid3) at (2.25,4.5){};
\node[circle, scale=0.75, fill] (tid6) at (2.25,6){};
\draw[](tid3) -- (tid6);
\node[circle, scale=0.75, fill, task_scheduled] (tid4) at (3.75,4.5){};
\draw[](tid1) -- (tid2);
\draw[](tid1) -- (tid3);
\draw[](tid1) -- (tid4);
\draw[](tid0) -- (tid1);
\end{tikzpicture}
\nodepart{three}
\footnotesize{5.21875}
\nodepart{four}
\footnotesize{$50\:25\:25$}
};
 & 
\node[draw=black, rectangle split,  rectangle split parts=4] (sn0xefa190){
\footnotesize{5.55556}
\nodepart{two}
\begin{tikzpicture}[scale=.2]
\node[circle, scale=0.75, fill] (tid0) at (2.25,1.5){};
\node[circle, scale=0.75, fill] (tid1) at (2.25,3){};
\node[circle, scale=0.75, fill] (tid2) at (0.75,4.5){};
\node[circle, scale=0.75, fill, task_scheduled] (tid5) at (0.75,6){};
\draw[](tid2) -- (tid5);
\node[circle, scale=0.75, fill] (tid3) at (2.25,4.5){};
\node[circle, scale=0.75, fill, task_scheduled] (tid6) at (2.25,6){};
\draw[](tid3) -- (tid6);
\node[circle, scale=0.75, fill] (tid4) at (3.75,4.5){};
\draw[](tid1) -- (tid2);
\draw[](tid1) -- (tid3);
\draw[](tid1) -- (tid4);
\draw[](tid0) -- (tid1);
\end{tikzpicture}
\nodepart{three}
\footnotesize{5.125}
\nodepart{four}
\footnotesize{$1$}
};
 & 
\\
};
\end{scope}
\begin{scope}[yshift=\leveltopIIIIII cm]
\matrix (line6)[column sep=0.5cm] {
\node[draw=black, rectangle split,  rectangle split parts=4] (sn0xef5310){
\footnotesize{15.1042}
\nodepart{two}
\begin{tikzpicture}[scale=.2]
\node[circle, scale=0.75, fill] (tid0) at (0.75,1.5){};
\node[circle, scale=0.75, fill] (tid1) at (0.75,3){};
\node[circle, scale=0.75, fill] (tid2) at (0.75,4.5){};
\node[circle, scale=0.75, fill] (tid3) at (0.75,6){};
\node[circle, scale=0.75, fill] (tid4) at (0.75,7.5){};
\node[circle, scale=0.75, fill, task_scheduled] (tid5) at (0.75,9){};
\draw[](tid4) -- (tid5);
\draw[](tid3) -- (tid4);
\draw[](tid2) -- (tid3);
\draw[](tid1) -- (tid2);
\draw[](tid0) -- (tid1);
\end{tikzpicture}
\nodepart{three}
\footnotesize{6}
\nodepart{four}
\footnotesize{$1$}
};
 & 
\node[draw=black, rectangle split,  rectangle split parts=4] (sn0xef5750){
\footnotesize{46.8316}
\nodepart{two}
\begin{tikzpicture}[scale=.2]
\node[circle, scale=0.75, fill] (tid0) at (1.5,1.5){};
\node[circle, scale=0.75, fill] (tid1) at (1.5,3){};
\node[circle, scale=0.75, fill] (tid2) at (0.75,4.5){};
\node[circle, scale=0.75, fill] (tid4) at (0.75,6){};
\node[circle, scale=0.75, fill, task_scheduled] (tid5) at (0.75,7.5){};
\draw[](tid4) -- (tid5);
\draw[](tid2) -- (tid4);
\node[circle, scale=0.75, fill, task_scheduled] (tid3) at (2.25,4.5){};
\draw[](tid1) -- (tid2);
\draw[](tid1) -- (tid3);
\draw[](tid0) -- (tid1);
\end{tikzpicture}
\nodepart{three}
\footnotesize{5.125}
\nodepart{four}
\footnotesize{$50\:50$}
};
 & 
\node[draw=black, rectangle split,  rectangle split parts=4] (sn0xeff630){
\footnotesize{0.176022}
\nodepart{two}
\begin{tikzpicture}[scale=.2]
\node[circle, scale=0.75, fill] (tid0) at (2.25,1.5){};
\node[circle, scale=0.75, fill] (tid1) at (2.25,3){};
\node[circle, scale=0.75, fill] (tid2) at (1.5,4.5){};
\node[circle, scale=0.75, fill, task_scheduled] (tid4) at (0.75,6){};
\node[circle, scale=0.75, fill] (tid5) at (2.25,6){};
\draw[](tid2) -- (tid4);
\draw[](tid2) -- (tid5);
\node[circle, scale=0.75, fill, task_scheduled] (tid3) at (3.75,4.5){};
\draw[](tid1) -- (tid2);
\draw[](tid1) -- (tid3);
\draw[](tid0) -- (tid1);
\end{tikzpicture}
\nodepart{three}
\footnotesize{4.875}
\nodepart{four}
\footnotesize{$50\:50$}
};
 & 
\node[draw=black, rectangle split,  rectangle split parts=4] (sn0xeff8f0){
\footnotesize{0.176022}
\nodepart{two}
\begin{tikzpicture}[scale=.2]
\node[circle, scale=0.75, fill] (tid0) at (2.25,1.5){};
\node[circle, scale=0.75, fill] (tid1) at (2.25,3){};
\node[circle, scale=0.75, fill] (tid2) at (1.5,4.5){};
\node[circle, scale=0.75, fill, task_scheduled] (tid4) at (0.75,6){};
\node[circle, scale=0.75, fill, task_scheduled] (tid5) at (2.25,6){};
\draw[](tid2) -- (tid4);
\draw[](tid2) -- (tid5);
\node[circle, scale=0.75, fill] (tid3) at (3.75,4.5){};
\draw[](tid1) -- (tid2);
\draw[](tid1) -- (tid3);
\draw[](tid0) -- (tid1);
\end{tikzpicture}
\nodepart{three}
\footnotesize{4.75}
\nodepart{four}
\footnotesize{$1$}
};
 & 
\node[draw=black, rectangle split,  rectangle split parts=4] (sn0xef7040){
\footnotesize{18.9959}
\nodepart{two}
\begin{tikzpicture}[scale=.2]
\node[circle, scale=0.75, fill] (tid0) at (1.5,1.5){};
\node[circle, scale=0.75, fill] (tid1) at (1.5,3){};
\node[circle, scale=0.75, fill] (tid2) at (0.75,4.5){};
\node[circle, scale=0.75, fill, task_scheduled] (tid4) at (0.75,6){};
\draw[](tid2) -- (tid4);
\node[circle, scale=0.75, fill] (tid3) at (2.25,4.5){};
\node[circle, scale=0.75, fill, task_scheduled] (tid5) at (2.25,6){};
\draw[](tid3) -- (tid5);
\draw[](tid1) -- (tid2);
\draw[](tid1) -- (tid3);
\draw[](tid0) -- (tid1);
\end{tikzpicture}
\nodepart{three}
\footnotesize{4.75}
\nodepart{four}
\footnotesize{$1$}
};
 & 
\node[draw=black, rectangle split,  rectangle split parts=4] (sn0xef7d10){
\footnotesize{6.54176}
\nodepart{two}
\begin{tikzpicture}[scale=.2]
\node[circle, scale=0.75, fill] (tid0) at (2.25,1.5){};
\node[circle, scale=0.75, fill] (tid1) at (2.25,3){};
\node[circle, scale=0.75, fill] (tid2) at (0.75,4.5){};
\node[circle, scale=0.75, fill] (tid5) at (0.75,6){};
\draw[](tid2) -- (tid5);
\node[circle, scale=0.75, fill, task_scheduled] (tid3) at (2.25,4.5){};
\node[circle, scale=0.75, fill, task_scheduled] (tid4) at (3.75,4.5){};
\draw[](tid1) -- (tid2);
\draw[](tid1) -- (tid3);
\draw[](tid1) -- (tid4);
\draw[](tid0) -- (tid1);
\end{tikzpicture}
\nodepart{three}
\footnotesize{4.75}
\nodepart{four}
\footnotesize{$1$}
};
 & 
\node[draw=black, rectangle split,  rectangle split parts=4] (sn0xef8590){
\footnotesize{12.1745}
\nodepart{two}
\begin{tikzpicture}[scale=.2]
\node[circle, scale=0.75, fill] (tid0) at (2.25,1.5){};
\node[circle, scale=0.75, fill] (tid1) at (2.25,3){};
\node[circle, scale=0.75, fill] (tid2) at (0.75,4.5){};
\node[circle, scale=0.75, fill, task_scheduled] (tid5) at (0.75,6){};
\draw[](tid2) -- (tid5);
\node[circle, scale=0.75, fill, task_scheduled] (tid3) at (2.25,4.5){};
\node[circle, scale=0.75, fill] (tid4) at (3.75,4.5){};
\draw[](tid1) -- (tid2);
\draw[](tid1) -- (tid3);
\draw[](tid1) -- (tid4);
\draw[](tid0) -- (tid1);
\end{tikzpicture}
\nodepart{three}
\footnotesize{4.625}
\nodepart{four}
\footnotesize{$50\:50$}
};
 & 
\\
};
\end{scope}
\begin{scope}[yshift=\leveltopIIIIIII cm]
\matrix (line7)[column sep=0.5cm] {
\node[draw=black, rectangle split,  rectangle split parts=4] (sn0xef5850){
\footnotesize{38.52}
\nodepart{two}
\begin{tikzpicture}[scale=.2]
\node[circle, scale=0.75, fill] (tid0) at (0.75,1.5){};
\node[circle, scale=0.75, fill] (tid1) at (0.75,3){};
\node[circle, scale=0.75, fill] (tid2) at (0.75,4.5){};
\node[circle, scale=0.75, fill] (tid3) at (0.75,6){};
\node[circle, scale=0.75, fill, task_scheduled] (tid4) at (0.75,7.5){};
\draw[](tid3) -- (tid4);
\draw[](tid2) -- (tid3);
\draw[](tid1) -- (tid2);
\draw[](tid0) -- (tid1);
\end{tikzpicture}
\nodepart{three}
\footnotesize{5}
\nodepart{four}
\footnotesize{$1$}
};
 & 
\node[draw=black, rectangle split,  rectangle split parts=4] (sn0xeffc80){
\footnotesize{0.0880112}
\nodepart{two}
\begin{tikzpicture}[scale=.2]
\node[circle, scale=0.75, fill] (tid0) at (1.5,1.5){};
\node[circle, scale=0.75, fill] (tid1) at (1.5,3){};
\node[circle, scale=0.75, fill] (tid2) at (1.5,4.5){};
\node[circle, scale=0.75, fill, task_scheduled] (tid3) at (0.75,6){};
\node[circle, scale=0.75, fill, task_scheduled] (tid4) at (2.25,6){};
\draw[](tid2) -- (tid3);
\draw[](tid2) -- (tid4);
\draw[](tid1) -- (tid2);
\draw[](tid0) -- (tid1);
\end{tikzpicture}
\nodepart{three}
\footnotesize{4.5}
\nodepart{four}
\footnotesize{$1$}
};
 & 
\node[draw=black, rectangle split,  rectangle split parts=4] (sn0xef64d0){
\footnotesize{55.3048}
\nodepart{two}
\begin{tikzpicture}[scale=.2]
\node[circle, scale=0.75, fill] (tid0) at (1.5,1.5){};
\node[circle, scale=0.75, fill] (tid1) at (1.5,3){};
\node[circle, scale=0.75, fill] (tid2) at (0.75,4.5){};
\node[circle, scale=0.75, fill, task_scheduled] (tid4) at (0.75,6){};
\draw[](tid2) -- (tid4);
\node[circle, scale=0.75, fill, task_scheduled] (tid3) at (2.25,4.5){};
\draw[](tid1) -- (tid2);
\draw[](tid1) -- (tid3);
\draw[](tid0) -- (tid1);
\end{tikzpicture}
\nodepart{three}
\footnotesize{4.25}
\nodepart{four}
\footnotesize{$50\:50$}
};
 & 
\node[draw=black, rectangle split,  rectangle split parts=4] (sn0xef8750){
\footnotesize{6.08724}
\nodepart{two}
\begin{tikzpicture}[scale=.2]
\node[circle, scale=0.75, fill] (tid0) at (2.25,1.5){};
\node[circle, scale=0.75, fill] (tid1) at (2.25,3){};
\node[circle, scale=0.75, fill, task_scheduled] (tid2) at (0.75,4.5){};
\node[circle, scale=0.75, fill, task_scheduled] (tid3) at (2.25,4.5){};
\node[circle, scale=0.75, fill] (tid4) at (3.75,4.5){};
\draw[](tid1) -- (tid2);
\draw[](tid1) -- (tid3);
\draw[](tid1) -- (tid4);
\draw[](tid0) -- (tid1);
\end{tikzpicture}
\nodepart{three}
\footnotesize{4}
\nodepart{four}
\footnotesize{$1$}
};
 & 
\\
};
\end{scope}
\begin{scope}[yshift=\leveltopIIIIIIII cm]
\matrix (line8)[column sep=0.5cm] {
\node[draw=black, rectangle split,  rectangle split parts=4] (sn0xef5c90){
\footnotesize{66.2604}
\nodepart{two}
\begin{tikzpicture}[scale=.2]
\node[circle, scale=0.75, fill] (tid0) at (0.75,1.5){};
\node[circle, scale=0.75, fill] (tid1) at (0.75,3){};
\node[circle, scale=0.75, fill] (tid2) at (0.75,4.5){};
\node[circle, scale=0.75, fill, task_scheduled] (tid3) at (0.75,6){};
\draw[](tid2) -- (tid3);
\draw[](tid1) -- (tid2);
\draw[](tid0) -- (tid1);
\end{tikzpicture}
\nodepart{three}
\footnotesize{4}
\nodepart{four}
\footnotesize{$1$}
};
 & 
\node[draw=black, rectangle split,  rectangle split parts=4] (sn0xef6940){
\footnotesize{33.7396}
\nodepart{two}
\begin{tikzpicture}[scale=.2]
\node[circle, scale=0.75, fill] (tid0) at (1.5,1.5){};
\node[circle, scale=0.75, fill] (tid1) at (1.5,3){};
\node[circle, scale=0.75, fill, task_scheduled] (tid2) at (0.75,4.5){};
\node[circle, scale=0.75, fill, task_scheduled] (tid3) at (2.25,4.5){};
\draw[](tid1) -- (tid2);
\draw[](tid1) -- (tid3);
\draw[](tid0) -- (tid1);
\end{tikzpicture}
\nodepart{three}
\footnotesize{3.5}
\nodepart{four}
\footnotesize{$1$}
};
 & 
\\
};
\end{scope}
\begin{scope}[yshift=\leveltopIIIIIIIII cm]
\matrix (line9)[column sep=0.5cm] {
\node[draw=black, rectangle split,  rectangle split parts=4] (sn0xef5950){
\footnotesize{100}
\nodepart{two}
\begin{tikzpicture}[scale=.2]
\node[circle, scale=0.75, fill] (tid0) at (0.75,1.5){};
\node[circle, scale=0.75, fill] (tid1) at (0.75,3){};
\node[circle, scale=0.75, fill, task_scheduled] (tid2) at (0.75,4.5){};
\draw[](tid1) -- (tid2);
\draw[](tid0) -- (tid1);
\end{tikzpicture}
\nodepart{three}
\footnotesize{3}
\nodepart{four}
\footnotesize{$1$}
};
 & 
\\
};
\end{scope}
\begin{scope}[yshift=\leveltopIIIIIIIIII cm]
\matrix (line10)[column sep=0.5cm] {
\node[draw=black, rectangle split,  rectangle split parts=4] (sn0xef5b30){
\footnotesize{100}
\nodepart{two}
\begin{tikzpicture}[scale=.2]
\node[circle, scale=0.75, fill] (tid0) at (0.75,1.5){};
\node[circle, scale=0.75, fill, task_scheduled] (tid1) at (0.75,3){};
\draw[](tid0) -- (tid1);
\end{tikzpicture}
\nodepart{three}
\footnotesize{2}
\nodepart{four}
\footnotesize{$1$}
};
 & 
\\
};
\end{scope}
\draw (sn0xef1700.south) -- (sn0xef2a50.north);
\draw (sn0xef1700.south) -- (sn0xef2540.north);
\draw (sn0xef1700.south) -- (sn0xefc570.north);
\draw (sn0xef1700.south) -- (sn0xefc970.north);
\draw (sn0xef1700.south) -- (sn0xefcc90.north);
\draw (sn0xefc570.south) -- (sn0xefd2a0.north);
\draw (sn0xefc570.south) -- (sn0xefe2f0.north);
\draw (sn0xefc570.south) -- (sn0xef2f20.north);
\draw (sn0xefc570.south) -- (sn0xefa920.north);
\draw (sn0xefc970.south) -- (sn0xef2f20.north);
\draw (sn0xefc970.south) -- (sn0xef3e70.north);
\draw (sn0xefcc90.south) -- (sn0xefa920.north);
\draw (sn0xefcc90.south) -- (sn0xef3e70.north);
\draw (sn0xefcc90.south) -- (sn0xeffff0.north);
\draw (sn0xefcc90.south) -- (sn0xf00d20.north);
\draw (sn0xefcc90.south) -- (sn0xf01470.north);
\draw (sn0xef2a50.south) -- (sn0xef2f20.north);
\draw (sn0xef2a50.south) -- (sn0xef3e70.north);
\draw (sn0xef2540.south) -- (sn0xefa920.north);
\draw (sn0xef2540.south) -- (sn0xef3e70.north);
\draw (sn0xef2540.south) -- (sn0xefabd0.north);
\draw (sn0xef2540.south) -- (sn0xefb500.north);
\draw (sn0xefd2a0.south) -- (sn0xef43a0.north);
\draw (sn0xefe2f0.south) -- (sn0xef43a0.north);
\draw (sn0xefe2f0.south) -- (sn0xefe720.north);
\draw (sn0xefe2f0.south) -- (sn0xefea20.north);
\draw (sn0xef2f20.south) -- (sn0xef43a0.north);
\draw (sn0xef2f20.south) -- (sn0xef4560.north);
\draw (sn0xef2f20.south) -- (sn0xef4790.north);
\draw (sn0xef3e70.south) -- (sn0xef4790.north);
\draw (sn0xef3e70.south) -- (sn0xef8300.north);
\draw (sn0xef3e70.south) -- (sn0xef90f0.north);
\draw (sn0xefa920.south) -- (sn0xef43a0.north);
\draw (sn0xefa920.south) -- (sn0xef8300.north);
\draw (sn0xefa920.south) -- (sn0xefbc60.north);
\draw (sn0xefabd0.south) -- (sn0xef8300.north);
\draw (sn0xefabd0.south) -- (sn0xef90f0.north);
\draw (sn0xefb500.south) -- (sn0xefbc60.north);
\draw (sn0xefb500.south) -- (sn0xef90f0.north);
\draw (sn0xefb500.south) -- (sn0xefc0b0.north);
\draw (sn0xeffff0.south) -- (sn0xefe720.north);
\draw (sn0xeffff0.south) -- (sn0xefea20.north);
\draw (sn0xeffff0.south) -- (sn0xef8300.north);
\draw (sn0xeffff0.south) -- (sn0xefbc60.north);
\draw (sn0xf00d20.south) -- (sn0xef8300.north);
\draw (sn0xf00d20.south) -- (sn0xef90f0.north);
\draw (sn0xf01470.south) -- (sn0xefbc60.north);
\draw (sn0xf01470.south) -- (sn0xef90f0.north);
\draw (sn0xf01470.south) -- (sn0xf01940.north);
\draw (sn0xf01470.south) -- (sn0xf01e70.north);
\draw (sn0xf01470.south) -- (sn0xf02290.north);
\draw (sn0xef43a0.south) -- (sn0xef48f0.north);
\draw (sn0xef43a0.south) -- (sn0xef5110.north);
\draw (sn0xef4560.south) -- (sn0xef48f0.north);
\draw (sn0xef4790.south) -- (sn0xef48f0.north);
\draw (sn0xef4790.south) -- (sn0xef7880.north);
\draw (sn0xef4790.south) -- (sn0xef7c10.north);
\draw (sn0xef8300.south) -- (sn0xef5110.north);
\draw (sn0xef8300.south) -- (sn0xef7880.north);
\draw (sn0xef8300.south) -- (sn0xef7c10.north);
\draw (sn0xef90f0.south) -- (sn0xef7c10.north);
\draw (sn0xef90f0.south) -- (sn0xef9e10.north);
\draw (sn0xef90f0.south) -- (sn0xefa190.north);
\draw (sn0xefbc60.south) -- (sn0xef5110.north);
\draw (sn0xefbc60.south) -- (sn0xef9e10.north);
\draw (sn0xefe720.south) -- (sn0xef5110.north);
\draw (sn0xefea20.south) -- (sn0xef5110.north);
\draw (sn0xefea20.south) -- (sn0xefeea0.north);
\draw (sn0xefea20.south) -- (sn0xeff210.north);
\draw (sn0xf01940.south) -- (sn0xefeea0.north);
\draw (sn0xf01940.south) -- (sn0xeff210.north);
\draw (sn0xf01940.south) -- (sn0xef9e10.north);
\draw (sn0xf01e70.south) -- (sn0xef9e10.north);
\draw (sn0xf01e70.south) -- (sn0xefa190.north);
\draw (sn0xf02290.south) -- (sn0xef9e10.north);
\draw (sn0xf02290.south) -- (sn0xefa190.north);
\draw (sn0xf02290.south) -- (sn0xf026b0.north);
\draw (sn0xf02290.south) -- (sn0xf02bb0.north);
\draw (sn0xefc0b0.south) -- (sn0xef9e10.north);
\draw (sn0xefc0b0.south) -- (sn0xefa190.north);
\draw (sn0xef48f0.south) -- (sn0xef5310.north);
\draw (sn0xef48f0.south) -- (sn0xef5750.north);
\draw (sn0xef5110.south) -- (sn0xef5750.north);
\draw (sn0xef5110.south) -- (sn0xef7040.north);
\draw (sn0xef7880.south) -- (sn0xef5750.north);
\draw (sn0xef7c10.south) -- (sn0xef5750.north);
\draw (sn0xef7c10.south) -- (sn0xef7d10.north);
\draw (sn0xef7c10.south) -- (sn0xef8590.north);
\draw (sn0xefeea0.south) -- (sn0xef7040.north);
\draw (sn0xeff210.south) -- (sn0xef7040.north);
\draw (sn0xeff210.south) -- (sn0xeff630.north);
\draw (sn0xeff210.south) -- (sn0xeff8f0.north);
\draw (sn0xf026b0.south) -- (sn0xeff630.north);
\draw (sn0xf026b0.south) -- (sn0xeff8f0.north);
\draw (sn0xf026b0.south) -- (sn0xef7d10.north);
\draw (sn0xf026b0.south) -- (sn0xef8590.north);
\draw (sn0xf02bb0.south) -- (sn0xef8590.north);
\draw (sn0xef9e10.south) -- (sn0xef7040.north);
\draw (sn0xef9e10.south) -- (sn0xef7d10.north);
\draw (sn0xef9e10.south) -- (sn0xef8590.north);
\draw (sn0xefa190.south) -- (sn0xef8590.north);
\draw (sn0xef5310.south) -- (sn0xef5850.north);
\draw (sn0xef5750.south) -- (sn0xef5850.north);
\draw (sn0xef5750.south) -- (sn0xef64d0.north);
\draw (sn0xeff630.south) -- (sn0xeffc80.north);
\draw (sn0xeff630.south) -- (sn0xef64d0.north);
\draw (sn0xeff8f0.south) -- (sn0xef64d0.north);
\draw (sn0xef7040.south) -- (sn0xef64d0.north);
\draw (sn0xef7d10.south) -- (sn0xef64d0.north);
\draw (sn0xef8590.south) -- (sn0xef64d0.north);
\draw (sn0xef8590.south) -- (sn0xef8750.north);
\draw (sn0xef5850.south) -- (sn0xef5c90.north);
\draw (sn0xeffc80.south) -- (sn0xef5c90.north);
\draw (sn0xef64d0.south) -- (sn0xef5c90.north);
\draw (sn0xef64d0.south) -- (sn0xef6940.north);
\draw (sn0xef8750.south) -- (sn0xef6940.north);
\draw (sn0xef5c90.south) -- (sn0xef5950.north);
\draw (sn0xef6940.south) -- (sn0xef5950.north);
\draw (sn0xef5950.south) -- (sn0xef5b30.north);
\end{tikzpicture}

%%% Local Variables:
%%% TeX-master: "thesis/thesis.tex"
%%% End: 
\renewcommand{\leveltopI}{-15cm + \leveltop}
\renewcommand{\leveltopII}{-15cm + \leveltopI}
\renewcommand{\leveltopIII}{-15cm + \leveltopII}
\renewcommand{\leveltopIIII}{-15cm + \leveltopIII}
\renewcommand{\leveltopIIIII}{-15cm + \leveltopIIII}
\renewcommand{\leveltopIIIIII}{-15cm + \leveltopIIIII}
\renewcommand{\leveltopIIIIIII}{-15cm + \leveltopIIIIII}
\renewcommand{\leveltopIIIIIIII}{-15cm + \leveltopIIIIIII}
\renewcommand{\leveltopIIIIIIIII}{-15cm + \leveltopIIIIIIII}
\renewcommand{\leveltopIIIIIIIIII}{-15cm + \leveltopIIIIIIIII}
\renewcommand{\leveltopIIIIIIIIIII}{-15cm + \leveltopIIIIIIIIII}
\begin{tikzpicture}[scale=.2, anchor=south]
\begin{scope}[yshift=\leveltopI cm]
\matrix (line1)[column sep=0.5cm] {
\node[draw=black, rectangle split,  rectangle split parts=4] (sn0xef1ca0){
\footnotesize{100}
\nodepart{two}
\begin{tikzpicture}[scale=.2]
\node[circle, scale=0.75, fill] (tid0) at (3,1.5){};
\node[circle, scale=0.75, fill] (tid1) at (3,3){};
\node[circle, scale=0.75, fill] (tid2) at (0.75,4.5){};
\node[circle, scale=0.75, fill] (tid5) at (0.75,6){};
\node[circle, scale=0.75, fill] (tid9) at (0.75,7.5){};
\node[circle, scale=0.75, fill, task_scheduled] (tid10) at (0.75,9){};
\draw[](tid9) -- (tid10);
\draw[](tid5) -- (tid9);
\draw[](tid2) -- (tid5);
\node[circle, scale=0.75, fill] (tid3) at (3,4.5){};
\node[circle, scale=0.75, fill, task_scheduled] (tid6) at (2.25,6){};
\node[circle, scale=0.75, fill] (tid7) at (3.75,6){};
\draw[](tid3) -- (tid6);
\draw[](tid3) -- (tid7);
\node[circle, scale=0.75, fill] (tid4) at (5.25,4.5){};
\node[circle, scale=0.75, fill] (tid8) at (5.25,6){};
\draw[](tid4) -- (tid8);
\draw[](tid1) -- (tid2);
\draw[](tid1) -- (tid3);
\draw[](tid1) -- (tid4);
\draw[](tid0) -- (tid1);
\end{tikzpicture}
\nodepart{three}
\footnotesize{7.36497}
\nodepart{four}
\footnotesize{$50\:17\:17\:17$}
};
 & 
\\
};
\end{scope}
\begin{scope}[yshift=\leveltopII cm]
\matrix (line2)[column sep=0.5cm] {
\node[draw=black, rectangle split,  rectangle split parts=4] (sn0xef2540){
\footnotesize{50}
\nodepart{two}
\begin{tikzpicture}[scale=.2]
\node[circle, scale=0.75, fill] (tid0) at (2.25,1.5){};
\node[circle, scale=0.75, fill] (tid1) at (2.25,3){};
\node[circle, scale=0.75, fill] (tid2) at (0.75,4.5){};
\node[circle, scale=0.75, fill] (tid5) at (0.75,6){};
\node[circle, scale=0.75, fill] (tid8) at (0.75,7.5){};
\node[circle, scale=0.75, fill, task_scheduled] (tid9) at (0.75,9){};
\draw[](tid8) -- (tid9);
\draw[](tid5) -- (tid8);
\draw[](tid2) -- (tid5);
\node[circle, scale=0.75, fill] (tid3) at (2.25,4.5){};
\node[circle, scale=0.75, fill, task_scheduled] (tid6) at (2.25,6){};
\draw[](tid3) -- (tid6);
\node[circle, scale=0.75, fill] (tid4) at (3.75,4.5){};
\node[circle, scale=0.75, fill] (tid7) at (3.75,6){};
\draw[](tid4) -- (tid7);
\draw[](tid1) -- (tid2);
\draw[](tid1) -- (tid3);
\draw[](tid1) -- (tid4);
\draw[](tid0) -- (tid1);
\end{tikzpicture}
\nodepart{three}
\footnotesize{6.94629}
\nodepart{four}
\footnotesize{$25\:25\:25\:25$}
};
 & 
\node[draw=black, rectangle split,  rectangle split parts=4] (sn0xf02cb0){
\footnotesize{16.6667}
\nodepart{two}
\begin{tikzpicture}[scale=.2]
\node[circle, scale=0.75, fill] (tid0) at (3,1.5){};
\node[circle, scale=0.75, fill] (tid1) at (3,3){};
\node[circle, scale=0.75, fill] (tid2) at (1.5,4.5){};
\node[circle, scale=0.75, fill, task_scheduled] (tid5) at (0.75,6){};
\node[circle, scale=0.75, fill, task_scheduled] (tid6) at (2.25,6){};
\draw[](tid2) -- (tid5);
\draw[](tid2) -- (tid6);
\node[circle, scale=0.75, fill] (tid3) at (3.75,4.5){};
\node[circle, scale=0.75, fill] (tid7) at (3.75,6){};
\node[circle, scale=0.75, fill] (tid9) at (3.75,7.5){};
\draw[](tid7) -- (tid9);
\draw[](tid3) -- (tid7);
\node[circle, scale=0.75, fill] (tid4) at (5.25,4.5){};
\node[circle, scale=0.75, fill] (tid8) at (5.25,6){};
\draw[](tid4) -- (tid8);
\draw[](tid1) -- (tid2);
\draw[](tid1) -- (tid3);
\draw[](tid1) -- (tid4);
\draw[](tid0) -- (tid1);
\end{tikzpicture}
\nodepart{three}
\footnotesize{6.81152}
\nodepart{four}
\footnotesize{$50\:50$}
};
 & 
\node[draw=black, rectangle split,  rectangle split parts=4] (sn0xf03560){
\footnotesize{16.6667}
\nodepart{two}
\begin{tikzpicture}[scale=.2]
\node[circle, scale=0.75, fill] (tid0) at (3,1.5){};
\node[circle, scale=0.75, fill] (tid1) at (3,3){};
\node[circle, scale=0.75, fill] (tid2) at (1.5,4.5){};
\node[circle, scale=0.75, fill, task_scheduled] (tid5) at (0.75,6){};
\node[circle, scale=0.75, fill] (tid6) at (2.25,6){};
\draw[](tid2) -- (tid5);
\draw[](tid2) -- (tid6);
\node[circle, scale=0.75, fill] (tid3) at (3.75,4.5){};
\node[circle, scale=0.75, fill] (tid7) at (3.75,6){};
\node[circle, scale=0.75, fill] (tid9) at (3.75,7.5){};
\draw[](tid7) -- (tid9);
\draw[](tid3) -- (tid7);
\node[circle, scale=0.75, fill] (tid4) at (5.25,4.5){};
\node[circle, scale=0.75, fill, task_scheduled] (tid8) at (5.25,6){};
\draw[](tid4) -- (tid8);
\draw[](tid1) -- (tid2);
\draw[](tid1) -- (tid3);
\draw[](tid1) -- (tid4);
\draw[](tid0) -- (tid1);
\end{tikzpicture}
\nodepart{three}
\footnotesize{6.82847}
\nodepart{four}
\footnotesize{$25\:25\:17\:17\:17$}
};
 & 
\node[draw=black, rectangle split,  rectangle split parts=4] (sn0xf038a0){
\footnotesize{16.6667}
\nodepart{two}
\begin{tikzpicture}[scale=.2]
\node[circle, scale=0.75, fill] (tid0) at (3,1.5){};
\node[circle, scale=0.75, fill] (tid1) at (3,3){};
\node[circle, scale=0.75, fill] (tid2) at (1.5,4.5){};
\node[circle, scale=0.75, fill, task_scheduled] (tid5) at (0.75,6){};
\node[circle, scale=0.75, fill] (tid6) at (2.25,6){};
\draw[](tid2) -- (tid5);
\draw[](tid2) -- (tid6);
\node[circle, scale=0.75, fill] (tid3) at (3.75,4.5){};
\node[circle, scale=0.75, fill] (tid7) at (3.75,6){};
\node[circle, scale=0.75, fill, task_scheduled] (tid9) at (3.75,7.5){};
\draw[](tid7) -- (tid9);
\draw[](tid3) -- (tid7);
\node[circle, scale=0.75, fill] (tid4) at (5.25,4.5){};
\node[circle, scale=0.75, fill] (tid8) at (5.25,6){};
\draw[](tid4) -- (tid8);
\draw[](tid1) -- (tid2);
\draw[](tid1) -- (tid3);
\draw[](tid1) -- (tid4);
\draw[](tid0) -- (tid1);
\end{tikzpicture}
\nodepart{three}
\footnotesize{6.71097}
\nodepart{four}
\footnotesize{$50\:17\:33$}
};
 & 
\\
};
\end{scope}
\begin{scope}[yshift=\leveltopIII cm]
\matrix (line3)[column sep=0.5cm] {
\node[draw=black, rectangle split,  rectangle split parts=4] (sn0xefa920){
\footnotesize{12.5}
\nodepart{two}
\begin{tikzpicture}[scale=.2]
\node[circle, scale=0.75, fill] (tid0) at (2.25,1.5){};
\node[circle, scale=0.75, fill] (tid1) at (2.25,3){};
\node[circle, scale=0.75, fill] (tid2) at (0.75,4.5){};
\node[circle, scale=0.75, fill] (tid5) at (0.75,6){};
\node[circle, scale=0.75, fill] (tid7) at (0.75,7.5){};
\node[circle, scale=0.75, fill, task_scheduled] (tid8) at (0.75,9){};
\draw[](tid7) -- (tid8);
\draw[](tid5) -- (tid7);
\draw[](tid2) -- (tid5);
\node[circle, scale=0.75, fill] (tid3) at (2.25,4.5){};
\node[circle, scale=0.75, fill] (tid6) at (2.25,6){};
\draw[](tid3) -- (tid6);
\node[circle, scale=0.75, fill, task_scheduled] (tid4) at (3.75,4.5){};
\draw[](tid1) -- (tid2);
\draw[](tid1) -- (tid3);
\draw[](tid1) -- (tid4);
\draw[](tid0) -- (tid1);
\end{tikzpicture}
\nodepart{three}
\footnotesize{6.57617}
\nodepart{four}
\footnotesize{$50\:25\:25$}
};
 & 
\node[draw=black, rectangle split,  rectangle split parts=4] (sn0xef3e70){
\footnotesize{12.5}
\nodepart{two}
\begin{tikzpicture}[scale=.2]
\node[circle, scale=0.75, fill] (tid0) at (2.25,1.5){};
\node[circle, scale=0.75, fill] (tid1) at (2.25,3){};
\node[circle, scale=0.75, fill] (tid2) at (0.75,4.5){};
\node[circle, scale=0.75, fill] (tid5) at (0.75,6){};
\node[circle, scale=0.75, fill] (tid7) at (0.75,7.5){};
\node[circle, scale=0.75, fill, task_scheduled] (tid8) at (0.75,9){};
\draw[](tid7) -- (tid8);
\draw[](tid5) -- (tid7);
\draw[](tid2) -- (tid5);
\node[circle, scale=0.75, fill] (tid3) at (2.25,4.5){};
\node[circle, scale=0.75, fill, task_scheduled] (tid6) at (2.25,6){};
\draw[](tid3) -- (tid6);
\node[circle, scale=0.75, fill] (tid4) at (3.75,4.5){};
\draw[](tid1) -- (tid2);
\draw[](tid1) -- (tid3);
\draw[](tid1) -- (tid4);
\draw[](tid0) -- (tid1);
\end{tikzpicture}
\nodepart{three}
\footnotesize{6.58594}
\nodepart{four}
\footnotesize{$50\:25\:25$}
};
 & 
\node[draw=black, rectangle split,  rectangle split parts=4] (sn0xefabd0){
\footnotesize{25}
\nodepart{two}
\begin{tikzpicture}[scale=.2]
\node[circle, scale=0.75, fill] (tid0) at (2.25,1.5){};
\node[circle, scale=0.75, fill] (tid1) at (2.25,3){};
\node[circle, scale=0.75, fill] (tid2) at (0.75,4.5){};
\node[circle, scale=0.75, fill] (tid5) at (0.75,6){};
\node[circle, scale=0.75, fill] (tid8) at (0.75,7.5){};
\draw[](tid5) -- (tid8);
\draw[](tid2) -- (tid5);
\node[circle, scale=0.75, fill] (tid3) at (2.25,4.5){};
\node[circle, scale=0.75, fill, task_scheduled] (tid6) at (2.25,6){};
\draw[](tid3) -- (tid6);
\node[circle, scale=0.75, fill] (tid4) at (3.75,4.5){};
\node[circle, scale=0.75, fill, task_scheduled] (tid7) at (3.75,6){};
\draw[](tid4) -- (tid7);
\draw[](tid1) -- (tid2);
\draw[](tid1) -- (tid3);
\draw[](tid1) -- (tid4);
\draw[](tid0) -- (tid1);
\end{tikzpicture}
\nodepart{three}
\footnotesize{6.38281}
\nodepart{four}
\footnotesize{$50\:50$}
};
 & 
\node[draw=black, rectangle split,  rectangle split parts=4] (sn0xefb500){
\footnotesize{33.3333}
\nodepart{two}
\begin{tikzpicture}[scale=.2]
\node[circle, scale=0.75, fill] (tid0) at (2.25,1.5){};
\node[circle, scale=0.75, fill] (tid1) at (2.25,3){};
\node[circle, scale=0.75, fill] (tid2) at (0.75,4.5){};
\node[circle, scale=0.75, fill] (tid5) at (0.75,6){};
\node[circle, scale=0.75, fill, task_scheduled] (tid8) at (0.75,7.5){};
\draw[](tid5) -- (tid8);
\draw[](tid2) -- (tid5);
\node[circle, scale=0.75, fill] (tid3) at (2.25,4.5){};
\node[circle, scale=0.75, fill, task_scheduled] (tid6) at (2.25,6){};
\draw[](tid3) -- (tid6);
\node[circle, scale=0.75, fill] (tid4) at (3.75,4.5){};
\node[circle, scale=0.75, fill] (tid7) at (3.75,6){};
\draw[](tid4) -- (tid7);
\draw[](tid1) -- (tid2);
\draw[](tid1) -- (tid3);
\draw[](tid1) -- (tid4);
\draw[](tid0) -- (tid1);
\end{tikzpicture}
\nodepart{three}
\footnotesize{6.24023}
\nodepart{four}
\footnotesize{$25\:25\:50$}
};
 & 
\node[draw=black, rectangle split,  rectangle split parts=4] (sn0xeffff0){
\footnotesize{2.77778}
\nodepart{two}
\begin{tikzpicture}[scale=.2]
\node[circle, scale=0.75, fill] (tid0) at (3,1.5){};
\node[circle, scale=0.75, fill] (tid1) at (3,3){};
\node[circle, scale=0.75, fill] (tid2) at (1.5,4.5){};
\node[circle, scale=0.75, fill, task_scheduled] (tid5) at (0.75,6){};
\node[circle, scale=0.75, fill] (tid6) at (2.25,6){};
\draw[](tid2) -- (tid5);
\draw[](tid2) -- (tid6);
\node[circle, scale=0.75, fill] (tid3) at (3.75,4.5){};
\node[circle, scale=0.75, fill] (tid7) at (3.75,6){};
\node[circle, scale=0.75, fill] (tid8) at (3.75,7.5){};
\draw[](tid7) -- (tid8);
\draw[](tid3) -- (tid7);
\node[circle, scale=0.75, fill, task_scheduled] (tid4) at (5.25,4.5){};
\draw[](tid1) -- (tid2);
\draw[](tid1) -- (tid3);
\draw[](tid1) -- (tid4);
\draw[](tid0) -- (tid1);
\end{tikzpicture}
\nodepart{three}
\footnotesize{6.39844}
\nodepart{four}
\footnotesize{$25\:25\:25\:25$}
};
 & 
\node[draw=black, rectangle split,  rectangle split parts=4] (sn0xf00d20){
\footnotesize{2.77778}
\nodepart{two}
\begin{tikzpicture}[scale=.2]
\node[circle, scale=0.75, fill] (tid0) at (3,1.5){};
\node[circle, scale=0.75, fill] (tid1) at (3,3){};
\node[circle, scale=0.75, fill] (tid2) at (1.5,4.5){};
\node[circle, scale=0.75, fill, task_scheduled] (tid5) at (0.75,6){};
\node[circle, scale=0.75, fill, task_scheduled] (tid6) at (2.25,6){};
\draw[](tid2) -- (tid5);
\draw[](tid2) -- (tid6);
\node[circle, scale=0.75, fill] (tid3) at (3.75,4.5){};
\node[circle, scale=0.75, fill] (tid7) at (3.75,6){};
\node[circle, scale=0.75, fill] (tid8) at (3.75,7.5){};
\draw[](tid7) -- (tid8);
\draw[](tid3) -- (tid7);
\node[circle, scale=0.75, fill] (tid4) at (5.25,4.5){};
\draw[](tid1) -- (tid2);
\draw[](tid1) -- (tid3);
\draw[](tid1) -- (tid4);
\draw[](tid0) -- (tid1);
\end{tikzpicture}
\nodepart{three}
\footnotesize{6.38281}
\nodepart{four}
\footnotesize{$50\:50$}
};
 & 
\node[draw=black, rectangle split,  rectangle split parts=4] (sn0xf01470){
\footnotesize{2.77778}
\nodepart{two}
\begin{tikzpicture}[scale=.2]
\node[circle, scale=0.75, fill] (tid0) at (3,1.5){};
\node[circle, scale=0.75, fill] (tid1) at (3,3){};
\node[circle, scale=0.75, fill] (tid2) at (1.5,4.5){};
\node[circle, scale=0.75, fill, task_scheduled] (tid5) at (0.75,6){};
\node[circle, scale=0.75, fill] (tid6) at (2.25,6){};
\draw[](tid2) -- (tid5);
\draw[](tid2) -- (tid6);
\node[circle, scale=0.75, fill] (tid3) at (3.75,4.5){};
\node[circle, scale=0.75, fill] (tid7) at (3.75,6){};
\node[circle, scale=0.75, fill, task_scheduled] (tid8) at (3.75,7.5){};
\draw[](tid7) -- (tid8);
\draw[](tid3) -- (tid7);
\node[circle, scale=0.75, fill] (tid4) at (5.25,4.5){};
\draw[](tid1) -- (tid2);
\draw[](tid1) -- (tid3);
\draw[](tid1) -- (tid4);
\draw[](tid0) -- (tid1);
\end{tikzpicture}
\nodepart{three}
\footnotesize{6.25499}
\nodepart{four}
\footnotesize{$25\:25\:17\:17\:17$}
};
 & 
\node[draw=black, rectangle split,  rectangle split parts=4] (sn0xf04590){
\footnotesize{2.77778}
\nodepart{two}
\begin{tikzpicture}[scale=.2]
\node[circle, scale=0.75, fill] (tid0) at (3,1.5){};
\node[circle, scale=0.75, fill] (tid1) at (3,3){};
\node[circle, scale=0.75, fill] (tid2) at (1.5,4.5){};
\node[circle, scale=0.75, fill, task_scheduled] (tid5) at (0.75,6){};
\node[circle, scale=0.75, fill, task_scheduled] (tid6) at (2.25,6){};
\draw[](tid2) -- (tid5);
\draw[](tid2) -- (tid6);
\node[circle, scale=0.75, fill] (tid3) at (3.75,4.5){};
\node[circle, scale=0.75, fill] (tid7) at (3.75,6){};
\draw[](tid3) -- (tid7);
\node[circle, scale=0.75, fill] (tid4) at (5.25,4.5){};
\node[circle, scale=0.75, fill] (tid8) at (5.25,6){};
\draw[](tid4) -- (tid8);
\draw[](tid1) -- (tid2);
\draw[](tid1) -- (tid3);
\draw[](tid1) -- (tid4);
\draw[](tid0) -- (tid1);
\end{tikzpicture}
\nodepart{three}
\footnotesize{6.17188}
\nodepart{four}
\footnotesize{$1$}
};
 & 
\node[draw=black, rectangle split,  rectangle split parts=4] (sn0xf04f90){
\footnotesize{5.55556}
\nodepart{two}
\begin{tikzpicture}[scale=.2]
\node[circle, scale=0.75, fill] (tid0) at (3,1.5){};
\node[circle, scale=0.75, fill] (tid1) at (3,3){};
\node[circle, scale=0.75, fill] (tid2) at (1.5,4.5){};
\node[circle, scale=0.75, fill, task_scheduled] (tid5) at (0.75,6){};
\node[circle, scale=0.75, fill] (tid6) at (2.25,6){};
\draw[](tid2) -- (tid5);
\draw[](tid2) -- (tid6);
\node[circle, scale=0.75, fill] (tid3) at (3.75,4.5){};
\node[circle, scale=0.75, fill, task_scheduled] (tid7) at (3.75,6){};
\draw[](tid3) -- (tid7);
\node[circle, scale=0.75, fill] (tid4) at (5.25,4.5){};
\node[circle, scale=0.75, fill] (tid8) at (5.25,6){};
\draw[](tid4) -- (tid8);
\draw[](tid1) -- (tid2);
\draw[](tid1) -- (tid3);
\draw[](tid1) -- (tid4);
\draw[](tid0) -- (tid1);
\end{tikzpicture}
\nodepart{three}
\footnotesize{6.18663}
\nodepart{four}
\footnotesize{$17\:17\:17\:50$}
};
 & 
\\
};
\end{scope}
\begin{scope}[yshift=\leveltopIIII cm]
\matrix (line4)[column sep=0.5cm] {
\node[draw=black, rectangle split,  rectangle split parts=4] (sn0xef43a0){
\footnotesize{6.25}
\nodepart{two}
\begin{tikzpicture}[scale=.2]
\node[circle, scale=0.75, fill] (tid0) at (1.5,1.5){};
\node[circle, scale=0.75, fill] (tid1) at (1.5,3){};
\node[circle, scale=0.75, fill] (tid2) at (0.75,4.5){};
\node[circle, scale=0.75, fill] (tid4) at (0.75,6){};
\node[circle, scale=0.75, fill] (tid6) at (0.75,7.5){};
\node[circle, scale=0.75, fill, task_scheduled] (tid7) at (0.75,9){};
\draw[](tid6) -- (tid7);
\draw[](tid4) -- (tid6);
\draw[](tid2) -- (tid4);
\node[circle, scale=0.75, fill] (tid3) at (2.25,4.5){};
\node[circle, scale=0.75, fill, task_scheduled] (tid5) at (2.25,6){};
\draw[](tid3) -- (tid5);
\draw[](tid1) -- (tid2);
\draw[](tid1) -- (tid3);
\draw[](tid0) -- (tid1);
\end{tikzpicture}
\nodepart{three}
\footnotesize{6.25}
\nodepart{four}
\footnotesize{$50\:50$}
};
 & 
\node[draw=black, rectangle split,  rectangle split parts=4] (sn0xef4790){
\footnotesize{6.25}
\nodepart{two}
\begin{tikzpicture}[scale=.2]
\node[circle, scale=0.75, fill] (tid0) at (2.25,1.5){};
\node[circle, scale=0.75, fill] (tid1) at (2.25,3){};
\node[circle, scale=0.75, fill] (tid2) at (0.75,4.5){};
\node[circle, scale=0.75, fill] (tid5) at (0.75,6){};
\node[circle, scale=0.75, fill] (tid6) at (0.75,7.5){};
\node[circle, scale=0.75, fill, task_scheduled] (tid7) at (0.75,9){};
\draw[](tid6) -- (tid7);
\draw[](tid5) -- (tid6);
\draw[](tid2) -- (tid5);
\node[circle, scale=0.75, fill, task_scheduled] (tid3) at (2.25,4.5){};
\node[circle, scale=0.75, fill] (tid4) at (3.75,4.5){};
\draw[](tid1) -- (tid2);
\draw[](tid1) -- (tid3);
\draw[](tid1) -- (tid4);
\draw[](tid0) -- (tid1);
\end{tikzpicture}
\nodepart{three}
\footnotesize{6.28906}
\nodepart{four}
\footnotesize{$50\:25\:25$}
};
 & 
\node[draw=black, rectangle split,  rectangle split parts=4] (sn0xef8300){
\footnotesize{20.8333}
\nodepart{two}
\begin{tikzpicture}[scale=.2]
\node[circle, scale=0.75, fill] (tid0) at (2.25,1.5){};
\node[circle, scale=0.75, fill] (tid1) at (2.25,3){};
\node[circle, scale=0.75, fill] (tid2) at (0.75,4.5){};
\node[circle, scale=0.75, fill] (tid5) at (0.75,6){};
\node[circle, scale=0.75, fill] (tid7) at (0.75,7.5){};
\draw[](tid5) -- (tid7);
\draw[](tid2) -- (tid5);
\node[circle, scale=0.75, fill] (tid3) at (2.25,4.5){};
\node[circle, scale=0.75, fill, task_scheduled] (tid6) at (2.25,6){};
\draw[](tid3) -- (tid6);
\node[circle, scale=0.75, fill, task_scheduled] (tid4) at (3.75,4.5){};
\draw[](tid1) -- (tid2);
\draw[](tid1) -- (tid3);
\draw[](tid1) -- (tid4);
\draw[](tid0) -- (tid1);
\end{tikzpicture}
\nodepart{three}
\footnotesize{5.97656}
\nodepart{four}
\footnotesize{$50\:25\:25$}
};
 & 
\node[draw=black, rectangle split,  rectangle split parts=4] (sn0xefbc60){
\footnotesize{12.8472}
\nodepart{two}
\begin{tikzpicture}[scale=.2]
\node[circle, scale=0.75, fill] (tid0) at (2.25,1.5){};
\node[circle, scale=0.75, fill] (tid1) at (2.25,3){};
\node[circle, scale=0.75, fill] (tid2) at (0.75,4.5){};
\node[circle, scale=0.75, fill] (tid5) at (0.75,6){};
\node[circle, scale=0.75, fill, task_scheduled] (tid7) at (0.75,7.5){};
\draw[](tid5) -- (tid7);
\draw[](tid2) -- (tid5);
\node[circle, scale=0.75, fill] (tid3) at (2.25,4.5){};
\node[circle, scale=0.75, fill] (tid6) at (2.25,6){};
\draw[](tid3) -- (tid6);
\node[circle, scale=0.75, fill, task_scheduled] (tid4) at (3.75,4.5){};
\draw[](tid1) -- (tid2);
\draw[](tid1) -- (tid3);
\draw[](tid1) -- (tid4);
\draw[](tid0) -- (tid1);
\end{tikzpicture}
\nodepart{three}
\footnotesize{5.82812}
\nodepart{four}
\footnotesize{$50\:50$}
};
 & 
\node[draw=black, rectangle split,  rectangle split parts=4] (sn0xef90f0){
\footnotesize{26.0417}
\nodepart{two}
\begin{tikzpicture}[scale=.2]
\node[circle, scale=0.75, fill] (tid0) at (2.25,1.5){};
\node[circle, scale=0.75, fill] (tid1) at (2.25,3){};
\node[circle, scale=0.75, fill] (tid2) at (0.75,4.5){};
\node[circle, scale=0.75, fill] (tid5) at (0.75,6){};
\node[circle, scale=0.75, fill, task_scheduled] (tid7) at (0.75,7.5){};
\draw[](tid5) -- (tid7);
\draw[](tid2) -- (tid5);
\node[circle, scale=0.75, fill] (tid3) at (2.25,4.5){};
\node[circle, scale=0.75, fill, task_scheduled] (tid6) at (2.25,6){};
\draw[](tid3) -- (tid6);
\node[circle, scale=0.75, fill] (tid4) at (3.75,4.5){};
\draw[](tid1) -- (tid2);
\draw[](tid1) -- (tid3);
\draw[](tid1) -- (tid4);
\draw[](tid0) -- (tid1);
\end{tikzpicture}
\nodepart{three}
\footnotesize{5.78906}
\nodepart{four}
\footnotesize{$50\:25\:25$}
};
 & 
\node[draw=black, rectangle split,  rectangle split parts=4] (sn0xefe720){
\footnotesize{0.694444}
\nodepart{two}
\begin{tikzpicture}[scale=.2]
\node[circle, scale=0.75, fill] (tid0) at (2.25,1.5){};
\node[circle, scale=0.75, fill] (tid1) at (2.25,3){};
\node[circle, scale=0.75, fill] (tid2) at (1.5,4.5){};
\node[circle, scale=0.75, fill, task_scheduled] (tid4) at (0.75,6){};
\node[circle, scale=0.75, fill, task_scheduled] (tid5) at (2.25,6){};
\draw[](tid2) -- (tid4);
\draw[](tid2) -- (tid5);
\node[circle, scale=0.75, fill] (tid3) at (3.75,4.5){};
\node[circle, scale=0.75, fill] (tid6) at (3.75,6){};
\node[circle, scale=0.75, fill] (tid7) at (3.75,7.5){};
\draw[](tid6) -- (tid7);
\draw[](tid3) -- (tid6);
\draw[](tid1) -- (tid2);
\draw[](tid1) -- (tid3);
\draw[](tid0) -- (tid1);
\end{tikzpicture}
\nodepart{three}
\footnotesize{5.9375}
\nodepart{four}
\footnotesize{$1$}
};
 & 
\node[draw=black, rectangle split,  rectangle split parts=4] (sn0xefea20){
\footnotesize{0.694444}
\nodepart{two}
\begin{tikzpicture}[scale=.2]
\node[circle, scale=0.75, fill] (tid0) at (2.25,1.5){};
\node[circle, scale=0.75, fill] (tid1) at (2.25,3){};
\node[circle, scale=0.75, fill] (tid2) at (1.5,4.5){};
\node[circle, scale=0.75, fill, task_scheduled] (tid4) at (0.75,6){};
\node[circle, scale=0.75, fill] (tid5) at (2.25,6){};
\draw[](tid2) -- (tid4);
\draw[](tid2) -- (tid5);
\node[circle, scale=0.75, fill] (tid3) at (3.75,4.5){};
\node[circle, scale=0.75, fill] (tid6) at (3.75,6){};
\node[circle, scale=0.75, fill, task_scheduled] (tid7) at (3.75,7.5){};
\draw[](tid6) -- (tid7);
\draw[](tid3) -- (tid6);
\draw[](tid1) -- (tid2);
\draw[](tid1) -- (tid3);
\draw[](tid0) -- (tid1);
\end{tikzpicture}
\nodepart{three}
\footnotesize{5.85156}
\nodepart{four}
\footnotesize{$50\:25\:25$}
};
 & 
\node[draw=black, rectangle split,  rectangle split parts=4] (sn0xf01940){
\footnotesize{1.38889}
\nodepart{two}
\begin{tikzpicture}[scale=.2]
\node[circle, scale=0.75, fill] (tid0) at (3,1.5){};
\node[circle, scale=0.75, fill] (tid1) at (3,3){};
\node[circle, scale=0.75, fill] (tid2) at (1.5,4.5){};
\node[circle, scale=0.75, fill, task_scheduled] (tid5) at (0.75,6){};
\node[circle, scale=0.75, fill] (tid6) at (2.25,6){};
\draw[](tid2) -- (tid5);
\draw[](tid2) -- (tid6);
\node[circle, scale=0.75, fill] (tid3) at (3.75,4.5){};
\node[circle, scale=0.75, fill] (tid7) at (3.75,6){};
\draw[](tid3) -- (tid7);
\node[circle, scale=0.75, fill, task_scheduled] (tid4) at (5.25,4.5){};
\draw[](tid1) -- (tid2);
\draw[](tid1) -- (tid3);
\draw[](tid1) -- (tid4);
\draw[](tid0) -- (tid1);
\end{tikzpicture}
\nodepart{three}
\footnotesize{5.74219}
\nodepart{four}
\footnotesize{$25\:25\:50$}
};
 & 
\node[draw=black, rectangle split,  rectangle split parts=4] (sn0xf01e70){
\footnotesize{1.38889}
\nodepart{two}
\begin{tikzpicture}[scale=.2]
\node[circle, scale=0.75, fill] (tid0) at (3,1.5){};
\node[circle, scale=0.75, fill] (tid1) at (3,3){};
\node[circle, scale=0.75, fill] (tid2) at (1.5,4.5){};
\node[circle, scale=0.75, fill, task_scheduled] (tid5) at (0.75,6){};
\node[circle, scale=0.75, fill, task_scheduled] (tid6) at (2.25,6){};
\draw[](tid2) -- (tid5);
\draw[](tid2) -- (tid6);
\node[circle, scale=0.75, fill] (tid3) at (3.75,4.5){};
\node[circle, scale=0.75, fill] (tid7) at (3.75,6){};
\draw[](tid3) -- (tid7);
\node[circle, scale=0.75, fill] (tid4) at (5.25,4.5){};
\draw[](tid1) -- (tid2);
\draw[](tid1) -- (tid3);
\draw[](tid1) -- (tid4);
\draw[](tid0) -- (tid1);
\end{tikzpicture}
\nodepart{three}
\footnotesize{5.67188}
\nodepart{four}
\footnotesize{$50\:50$}
};
 & 
\node[draw=black, rectangle split,  rectangle split parts=4] (sn0xf02290){
\footnotesize{1.38889}
\nodepart{two}
\begin{tikzpicture}[scale=.2]
\node[circle, scale=0.75, fill] (tid0) at (3,1.5){};
\node[circle, scale=0.75, fill] (tid1) at (3,3){};
\node[circle, scale=0.75, fill] (tid2) at (1.5,4.5){};
\node[circle, scale=0.75, fill, task_scheduled] (tid5) at (0.75,6){};
\node[circle, scale=0.75, fill] (tid6) at (2.25,6){};
\draw[](tid2) -- (tid5);
\draw[](tid2) -- (tid6);
\node[circle, scale=0.75, fill] (tid3) at (3.75,4.5){};
\node[circle, scale=0.75, fill, task_scheduled] (tid7) at (3.75,6){};
\draw[](tid3) -- (tid7);
\node[circle, scale=0.75, fill] (tid4) at (5.25,4.5){};
\draw[](tid1) -- (tid2);
\draw[](tid1) -- (tid3);
\draw[](tid1) -- (tid4);
\draw[](tid0) -- (tid1);
\end{tikzpicture}
\nodepart{three}
\footnotesize{5.6901}
\nodepart{four}
\footnotesize{$33\:17\:25\:25$}
};
 & 
\node[draw=black, rectangle split,  rectangle split parts=4] (sn0xefc0b0){
\footnotesize{22.2222}
\nodepart{two}
\begin{tikzpicture}[scale=.2]
\node[circle, scale=0.75, fill] (tid0) at (2.25,1.5){};
\node[circle, scale=0.75, fill] (tid1) at (2.25,3){};
\node[circle, scale=0.75, fill] (tid2) at (0.75,4.5){};
\node[circle, scale=0.75, fill, task_scheduled] (tid5) at (0.75,6){};
\draw[](tid2) -- (tid5);
\node[circle, scale=0.75, fill] (tid3) at (2.25,4.5){};
\node[circle, scale=0.75, fill, task_scheduled] (tid6) at (2.25,6){};
\draw[](tid3) -- (tid6);
\node[circle, scale=0.75, fill] (tid4) at (3.75,4.5){};
\node[circle, scale=0.75, fill] (tid7) at (3.75,6){};
\draw[](tid4) -- (tid7);
\draw[](tid1) -- (tid2);
\draw[](tid1) -- (tid3);
\draw[](tid1) -- (tid4);
\draw[](tid0) -- (tid1);
\end{tikzpicture}
\nodepart{three}
\footnotesize{5.67188}
\nodepart{four}
\footnotesize{$50\:50$}
};
 & 
\\
};
\end{scope}
\begin{scope}[yshift=\leveltopIIIII cm]
\matrix (line5)[column sep=0.5cm] {
\node[draw=black, rectangle split,  rectangle split parts=4] (sn0xef48f0){
\footnotesize{6.25}
\nodepart{two}
\begin{tikzpicture}[scale=.2]
\node[circle, scale=0.75, fill] (tid0) at (1.5,1.5){};
\node[circle, scale=0.75, fill] (tid1) at (1.5,3){};
\node[circle, scale=0.75, fill] (tid2) at (0.75,4.5){};
\node[circle, scale=0.75, fill] (tid4) at (0.75,6){};
\node[circle, scale=0.75, fill] (tid5) at (0.75,7.5){};
\node[circle, scale=0.75, fill, task_scheduled] (tid6) at (0.75,9){};
\draw[](tid5) -- (tid6);
\draw[](tid4) -- (tid5);
\draw[](tid2) -- (tid4);
\node[circle, scale=0.75, fill, task_scheduled] (tid3) at (2.25,4.5){};
\draw[](tid1) -- (tid2);
\draw[](tid1) -- (tid3);
\draw[](tid0) -- (tid1);
\end{tikzpicture}
\nodepart{three}
\footnotesize{6.0625}
\nodepart{four}
\footnotesize{$50\:50$}
};
 & 
\node[draw=black, rectangle split,  rectangle split parts=4] (sn0xef5110){
\footnotesize{21.0069}
\nodepart{two}
\begin{tikzpicture}[scale=.2]
\node[circle, scale=0.75, fill] (tid0) at (1.5,1.5){};
\node[circle, scale=0.75, fill] (tid1) at (1.5,3){};
\node[circle, scale=0.75, fill] (tid2) at (0.75,4.5){};
\node[circle, scale=0.75, fill] (tid4) at (0.75,6){};
\node[circle, scale=0.75, fill, task_scheduled] (tid6) at (0.75,7.5){};
\draw[](tid4) -- (tid6);
\draw[](tid2) -- (tid4);
\node[circle, scale=0.75, fill] (tid3) at (2.25,4.5){};
\node[circle, scale=0.75, fill, task_scheduled] (tid5) at (2.25,6){};
\draw[](tid3) -- (tid5);
\draw[](tid1) -- (tid2);
\draw[](tid1) -- (tid3);
\draw[](tid0) -- (tid1);
\end{tikzpicture}
\nodepart{three}
\footnotesize{5.4375}
\nodepart{four}
\footnotesize{$50\:50$}
};
 & 
\node[draw=black, rectangle split,  rectangle split parts=4] (sn0xef7880){
\footnotesize{6.77083}
\nodepart{two}
\begin{tikzpicture}[scale=.2]
\node[circle, scale=0.75, fill] (tid0) at (2.25,1.5){};
\node[circle, scale=0.75, fill] (tid1) at (2.25,3){};
\node[circle, scale=0.75, fill] (tid2) at (0.75,4.5){};
\node[circle, scale=0.75, fill] (tid5) at (0.75,6){};
\node[circle, scale=0.75, fill] (tid6) at (0.75,7.5){};
\draw[](tid5) -- (tid6);
\draw[](tid2) -- (tid5);
\node[circle, scale=0.75, fill, task_scheduled] (tid3) at (2.25,4.5){};
\node[circle, scale=0.75, fill, task_scheduled] (tid4) at (3.75,4.5){};
\draw[](tid1) -- (tid2);
\draw[](tid1) -- (tid3);
\draw[](tid1) -- (tid4);
\draw[](tid0) -- (tid1);
\end{tikzpicture}
\nodepart{three}
\footnotesize{5.625}
\nodepart{four}
\footnotesize{$1$}
};
 & 
\node[draw=black, rectangle split,  rectangle split parts=4] (sn0xef7c10){
\footnotesize{19.7917}
\nodepart{two}
\begin{tikzpicture}[scale=.2]
\node[circle, scale=0.75, fill] (tid0) at (2.25,1.5){};
\node[circle, scale=0.75, fill] (tid1) at (2.25,3){};
\node[circle, scale=0.75, fill] (tid2) at (0.75,4.5){};
\node[circle, scale=0.75, fill] (tid5) at (0.75,6){};
\node[circle, scale=0.75, fill, task_scheduled] (tid6) at (0.75,7.5){};
\draw[](tid5) -- (tid6);
\draw[](tid2) -- (tid5);
\node[circle, scale=0.75, fill, task_scheduled] (tid3) at (2.25,4.5){};
\node[circle, scale=0.75, fill] (tid4) at (3.75,4.5){};
\draw[](tid1) -- (tid2);
\draw[](tid1) -- (tid3);
\draw[](tid1) -- (tid4);
\draw[](tid0) -- (tid1);
\end{tikzpicture}
\nodepart{three}
\footnotesize{5.40625}
\nodepart{four}
\footnotesize{$50\:25\:25$}
};
 & 
\node[draw=black, rectangle split,  rectangle split parts=4] (sn0xefeea0){
\footnotesize{0.520833}
\nodepart{two}
\begin{tikzpicture}[scale=.2]
\node[circle, scale=0.75, fill] (tid0) at (2.25,1.5){};
\node[circle, scale=0.75, fill] (tid1) at (2.25,3){};
\node[circle, scale=0.75, fill] (tid2) at (1.5,4.5){};
\node[circle, scale=0.75, fill, task_scheduled] (tid4) at (0.75,6){};
\node[circle, scale=0.75, fill, task_scheduled] (tid5) at (2.25,6){};
\draw[](tid2) -- (tid4);
\draw[](tid2) -- (tid5);
\node[circle, scale=0.75, fill] (tid3) at (3.75,4.5){};
\node[circle, scale=0.75, fill] (tid6) at (3.75,6){};
\draw[](tid3) -- (tid6);
\draw[](tid1) -- (tid2);
\draw[](tid1) -- (tid3);
\draw[](tid0) -- (tid1);
\end{tikzpicture}
\nodepart{three}
\footnotesize{5.25}
\nodepart{four}
\footnotesize{$1$}
};
 & 
\node[draw=black, rectangle split,  rectangle split parts=4] (sn0xeff210){
\footnotesize{0.520833}
\nodepart{two}
\begin{tikzpicture}[scale=.2]
\node[circle, scale=0.75, fill] (tid0) at (2.25,1.5){};
\node[circle, scale=0.75, fill] (tid1) at (2.25,3){};
\node[circle, scale=0.75, fill] (tid2) at (1.5,4.5){};
\node[circle, scale=0.75, fill, task_scheduled] (tid4) at (0.75,6){};
\node[circle, scale=0.75, fill] (tid5) at (2.25,6){};
\draw[](tid2) -- (tid4);
\draw[](tid2) -- (tid5);
\node[circle, scale=0.75, fill] (tid3) at (3.75,4.5){};
\node[circle, scale=0.75, fill, task_scheduled] (tid6) at (3.75,6){};
\draw[](tid3) -- (tid6);
\draw[](tid1) -- (tid2);
\draw[](tid1) -- (tid3);
\draw[](tid0) -- (tid1);
\end{tikzpicture}
\nodepart{three}
\footnotesize{5.28125}
\nodepart{four}
\footnotesize{$25\:25\:50$}
};
 & 
\node[draw=black, rectangle split,  rectangle split parts=4] (sn0xf026b0){
\footnotesize{0.462963}
\nodepart{two}
\begin{tikzpicture}[scale=.2]
\node[circle, scale=0.75, fill] (tid0) at (3,1.5){};
\node[circle, scale=0.75, fill] (tid1) at (3,3){};
\node[circle, scale=0.75, fill] (tid2) at (1.5,4.5){};
\node[circle, scale=0.75, fill, task_scheduled] (tid5) at (0.75,6){};
\node[circle, scale=0.75, fill] (tid6) at (2.25,6){};
\draw[](tid2) -- (tid5);
\draw[](tid2) -- (tid6);
\node[circle, scale=0.75, fill, task_scheduled] (tid3) at (3.75,4.5){};
\node[circle, scale=0.75, fill] (tid4) at (5.25,4.5){};
\draw[](tid1) -- (tid2);
\draw[](tid1) -- (tid3);
\draw[](tid1) -- (tid4);
\draw[](tid0) -- (tid1);
\end{tikzpicture}
\nodepart{three}
\footnotesize{5.25}
\nodepart{four}
\footnotesize{$25\:25\:25\:25$}
};
 & 
\node[draw=black, rectangle split,  rectangle split parts=4] (sn0xf02bb0){
\footnotesize{0.231482}
\nodepart{two}
\begin{tikzpicture}[scale=.2]
\node[circle, scale=0.75, fill] (tid0) at (3,1.5){};
\node[circle, scale=0.75, fill] (tid1) at (3,3){};
\node[circle, scale=0.75, fill] (tid2) at (1.5,4.5){};
\node[circle, scale=0.75, fill, task_scheduled] (tid5) at (0.75,6){};
\node[circle, scale=0.75, fill, task_scheduled] (tid6) at (2.25,6){};
\draw[](tid2) -- (tid5);
\draw[](tid2) -- (tid6);
\node[circle, scale=0.75, fill] (tid3) at (3.75,4.5){};
\node[circle, scale=0.75, fill] (tid4) at (5.25,4.5){};
\draw[](tid1) -- (tid2);
\draw[](tid1) -- (tid3);
\draw[](tid1) -- (tid4);
\draw[](tid0) -- (tid1);
\end{tikzpicture}
\nodepart{three}
\footnotesize{5.125}
\nodepart{four}
\footnotesize{$1$}
};
 & 
\node[draw=black, rectangle split,  rectangle split parts=4] (sn0xef9e10){
\footnotesize{25.7812}
\nodepart{two}
\begin{tikzpicture}[scale=.2]
\node[circle, scale=0.75, fill] (tid0) at (2.25,1.5){};
\node[circle, scale=0.75, fill] (tid1) at (2.25,3){};
\node[circle, scale=0.75, fill] (tid2) at (0.75,4.5){};
\node[circle, scale=0.75, fill, task_scheduled] (tid5) at (0.75,6){};
\draw[](tid2) -- (tid5);
\node[circle, scale=0.75, fill] (tid3) at (2.25,4.5){};
\node[circle, scale=0.75, fill] (tid6) at (2.25,6){};
\draw[](tid3) -- (tid6);
\node[circle, scale=0.75, fill, task_scheduled] (tid4) at (3.75,4.5){};
\draw[](tid1) -- (tid2);
\draw[](tid1) -- (tid3);
\draw[](tid1) -- (tid4);
\draw[](tid0) -- (tid1);
\end{tikzpicture}
\nodepart{three}
\footnotesize{5.21875}
\nodepart{four}
\footnotesize{$50\:25\:25$}
};
 & 
\node[draw=black, rectangle split,  rectangle split parts=4] (sn0xefa190){
\footnotesize{18.6632}
\nodepart{two}
\begin{tikzpicture}[scale=.2]
\node[circle, scale=0.75, fill] (tid0) at (2.25,1.5){};
\node[circle, scale=0.75, fill] (tid1) at (2.25,3){};
\node[circle, scale=0.75, fill] (tid2) at (0.75,4.5){};
\node[circle, scale=0.75, fill, task_scheduled] (tid5) at (0.75,6){};
\draw[](tid2) -- (tid5);
\node[circle, scale=0.75, fill] (tid3) at (2.25,4.5){};
\node[circle, scale=0.75, fill, task_scheduled] (tid6) at (2.25,6){};
\draw[](tid3) -- (tid6);
\node[circle, scale=0.75, fill] (tid4) at (3.75,4.5){};
\draw[](tid1) -- (tid2);
\draw[](tid1) -- (tid3);
\draw[](tid1) -- (tid4);
\draw[](tid0) -- (tid1);
\end{tikzpicture}
\nodepart{three}
\footnotesize{5.125}
\nodepart{four}
\footnotesize{$1$}
};
 & 
\\
};
\end{scope}
\begin{scope}[yshift=\leveltopIIIIII cm]
\matrix (line6)[column sep=0.5cm] {
\node[draw=black, rectangle split,  rectangle split parts=4] (sn0xef5310){
\footnotesize{3.125}
\nodepart{two}
\begin{tikzpicture}[scale=.2]
\node[circle, scale=0.75, fill] (tid0) at (0.75,1.5){};
\node[circle, scale=0.75, fill] (tid1) at (0.75,3){};
\node[circle, scale=0.75, fill] (tid2) at (0.75,4.5){};
\node[circle, scale=0.75, fill] (tid3) at (0.75,6){};
\node[circle, scale=0.75, fill] (tid4) at (0.75,7.5){};
\node[circle, scale=0.75, fill, task_scheduled] (tid5) at (0.75,9){};
\draw[](tid4) -- (tid5);
\draw[](tid3) -- (tid4);
\draw[](tid2) -- (tid3);
\draw[](tid1) -- (tid2);
\draw[](tid0) -- (tid1);
\end{tikzpicture}
\nodepart{three}
\footnotesize{6}
\nodepart{four}
\footnotesize{$1$}
};
 & 
\node[draw=black, rectangle split,  rectangle split parts=4] (sn0xef5750){
\footnotesize{30.2951}
\nodepart{two}
\begin{tikzpicture}[scale=.2]
\node[circle, scale=0.75, fill] (tid0) at (1.5,1.5){};
\node[circle, scale=0.75, fill] (tid1) at (1.5,3){};
\node[circle, scale=0.75, fill] (tid2) at (0.75,4.5){};
\node[circle, scale=0.75, fill] (tid4) at (0.75,6){};
\node[circle, scale=0.75, fill, task_scheduled] (tid5) at (0.75,7.5){};
\draw[](tid4) -- (tid5);
\draw[](tid2) -- (tid4);
\node[circle, scale=0.75, fill, task_scheduled] (tid3) at (2.25,4.5){};
\draw[](tid1) -- (tid2);
\draw[](tid1) -- (tid3);
\draw[](tid0) -- (tid1);
\end{tikzpicture}
\nodepart{three}
\footnotesize{5.125}
\nodepart{four}
\footnotesize{$50\:50$}
};
 & 
\node[draw=black, rectangle split,  rectangle split parts=4] (sn0xeff630){
\footnotesize{0.245949}
\nodepart{two}
\begin{tikzpicture}[scale=.2]
\node[circle, scale=0.75, fill] (tid0) at (2.25,1.5){};
\node[circle, scale=0.75, fill] (tid1) at (2.25,3){};
\node[circle, scale=0.75, fill] (tid2) at (1.5,4.5){};
\node[circle, scale=0.75, fill, task_scheduled] (tid4) at (0.75,6){};
\node[circle, scale=0.75, fill] (tid5) at (2.25,6){};
\draw[](tid2) -- (tid4);
\draw[](tid2) -- (tid5);
\node[circle, scale=0.75, fill, task_scheduled] (tid3) at (3.75,4.5){};
\draw[](tid1) -- (tid2);
\draw[](tid1) -- (tid3);
\draw[](tid0) -- (tid1);
\end{tikzpicture}
\nodepart{three}
\footnotesize{4.875}
\nodepart{four}
\footnotesize{$50\:50$}
};
 & 
\node[draw=black, rectangle split,  rectangle split parts=4] (sn0xeff8f0){
\footnotesize{0.245949}
\nodepart{two}
\begin{tikzpicture}[scale=.2]
\node[circle, scale=0.75, fill] (tid0) at (2.25,1.5){};
\node[circle, scale=0.75, fill] (tid1) at (2.25,3){};
\node[circle, scale=0.75, fill] (tid2) at (1.5,4.5){};
\node[circle, scale=0.75, fill, task_scheduled] (tid4) at (0.75,6){};
\node[circle, scale=0.75, fill, task_scheduled] (tid5) at (2.25,6){};
\draw[](tid2) -- (tid4);
\draw[](tid2) -- (tid5);
\node[circle, scale=0.75, fill] (tid3) at (3.75,4.5){};
\draw[](tid1) -- (tid2);
\draw[](tid1) -- (tid3);
\draw[](tid0) -- (tid1);
\end{tikzpicture}
\nodepart{three}
\footnotesize{4.75}
\nodepart{four}
\footnotesize{$1$}
};
 & 
\node[draw=black, rectangle split,  rectangle split parts=4] (sn0xef7040){
\footnotesize{24.1753}
\nodepart{two}
\begin{tikzpicture}[scale=.2]
\node[circle, scale=0.75, fill] (tid0) at (1.5,1.5){};
\node[circle, scale=0.75, fill] (tid1) at (1.5,3){};
\node[circle, scale=0.75, fill] (tid2) at (0.75,4.5){};
\node[circle, scale=0.75, fill, task_scheduled] (tid4) at (0.75,6){};
\draw[](tid2) -- (tid4);
\node[circle, scale=0.75, fill] (tid3) at (2.25,4.5){};
\node[circle, scale=0.75, fill, task_scheduled] (tid5) at (2.25,6){};
\draw[](tid3) -- (tid5);
\draw[](tid1) -- (tid2);
\draw[](tid1) -- (tid3);
\draw[](tid0) -- (tid1);
\end{tikzpicture}
\nodepart{three}
\footnotesize{4.75}
\nodepart{four}
\footnotesize{$1$}
};
 & 
\node[draw=black, rectangle split,  rectangle split parts=4] (sn0xef7d10){
\footnotesize{11.509}
\nodepart{two}
\begin{tikzpicture}[scale=.2]
\node[circle, scale=0.75, fill] (tid0) at (2.25,1.5){};
\node[circle, scale=0.75, fill] (tid1) at (2.25,3){};
\node[circle, scale=0.75, fill] (tid2) at (0.75,4.5){};
\node[circle, scale=0.75, fill] (tid5) at (0.75,6){};
\draw[](tid2) -- (tid5);
\node[circle, scale=0.75, fill, task_scheduled] (tid3) at (2.25,4.5){};
\node[circle, scale=0.75, fill, task_scheduled] (tid4) at (3.75,4.5){};
\draw[](tid1) -- (tid2);
\draw[](tid1) -- (tid3);
\draw[](tid1) -- (tid4);
\draw[](tid0) -- (tid1);
\end{tikzpicture}
\nodepart{three}
\footnotesize{4.75}
\nodepart{four}
\footnotesize{$1$}
};
 & 
\node[draw=black, rectangle split,  rectangle split parts=4] (sn0xef8590){
\footnotesize{30.4036}
\nodepart{two}
\begin{tikzpicture}[scale=.2]
\node[circle, scale=0.75, fill] (tid0) at (2.25,1.5){};
\node[circle, scale=0.75, fill] (tid1) at (2.25,3){};
\node[circle, scale=0.75, fill] (tid2) at (0.75,4.5){};
\node[circle, scale=0.75, fill, task_scheduled] (tid5) at (0.75,6){};
\draw[](tid2) -- (tid5);
\node[circle, scale=0.75, fill, task_scheduled] (tid3) at (2.25,4.5){};
\node[circle, scale=0.75, fill] (tid4) at (3.75,4.5){};
\draw[](tid1) -- (tid2);
\draw[](tid1) -- (tid3);
\draw[](tid1) -- (tid4);
\draw[](tid0) -- (tid1);
\end{tikzpicture}
\nodepart{three}
\footnotesize{4.625}
\nodepart{four}
\footnotesize{$50\:50$}
};
 & 
\\
};
\end{scope}
\begin{scope}[yshift=\leveltopIIIIIII cm]
\matrix (line7)[column sep=0.5cm] {
\node[draw=black, rectangle split,  rectangle split parts=4] (sn0xef5850){
\footnotesize{18.2726}
\nodepart{two}
\begin{tikzpicture}[scale=.2]
\node[circle, scale=0.75, fill] (tid0) at (0.75,1.5){};
\node[circle, scale=0.75, fill] (tid1) at (0.75,3){};
\node[circle, scale=0.75, fill] (tid2) at (0.75,4.5){};
\node[circle, scale=0.75, fill] (tid3) at (0.75,6){};
\node[circle, scale=0.75, fill, task_scheduled] (tid4) at (0.75,7.5){};
\draw[](tid3) -- (tid4);
\draw[](tid2) -- (tid3);
\draw[](tid1) -- (tid2);
\draw[](tid0) -- (tid1);
\end{tikzpicture}
\nodepart{three}
\footnotesize{5}
\nodepart{four}
\footnotesize{$1$}
};
 & 
\node[draw=black, rectangle split,  rectangle split parts=4] (sn0xeffc80){
\footnotesize{0.122975}
\nodepart{two}
\begin{tikzpicture}[scale=.2]
\node[circle, scale=0.75, fill] (tid0) at (1.5,1.5){};
\node[circle, scale=0.75, fill] (tid1) at (1.5,3){};
\node[circle, scale=0.75, fill] (tid2) at (1.5,4.5){};
\node[circle, scale=0.75, fill, task_scheduled] (tid3) at (0.75,6){};
\node[circle, scale=0.75, fill, task_scheduled] (tid4) at (2.25,6){};
\draw[](tid2) -- (tid3);
\draw[](tid2) -- (tid4);
\draw[](tid1) -- (tid2);
\draw[](tid0) -- (tid1);
\end{tikzpicture}
\nodepart{three}
\footnotesize{4.5}
\nodepart{four}
\footnotesize{$1$}
};
 & 
\node[draw=black, rectangle split,  rectangle split parts=4] (sn0xef64d0){
\footnotesize{66.4026}
\nodepart{two}
\begin{tikzpicture}[scale=.2]
\node[circle, scale=0.75, fill] (tid0) at (1.5,1.5){};
\node[circle, scale=0.75, fill] (tid1) at (1.5,3){};
\node[circle, scale=0.75, fill] (tid2) at (0.75,4.5){};
\node[circle, scale=0.75, fill, task_scheduled] (tid4) at (0.75,6){};
\draw[](tid2) -- (tid4);
\node[circle, scale=0.75, fill, task_scheduled] (tid3) at (2.25,4.5){};
\draw[](tid1) -- (tid2);
\draw[](tid1) -- (tid3);
\draw[](tid0) -- (tid1);
\end{tikzpicture}
\nodepart{three}
\footnotesize{4.25}
\nodepart{four}
\footnotesize{$50\:50$}
};
 & 
\node[draw=black, rectangle split,  rectangle split parts=4] (sn0xef8750){
\footnotesize{15.2018}
\nodepart{two}
\begin{tikzpicture}[scale=.2]
\node[circle, scale=0.75, fill] (tid0) at (2.25,1.5){};
\node[circle, scale=0.75, fill] (tid1) at (2.25,3){};
\node[circle, scale=0.75, fill, task_scheduled] (tid2) at (0.75,4.5){};
\node[circle, scale=0.75, fill, task_scheduled] (tid3) at (2.25,4.5){};
\node[circle, scale=0.75, fill] (tid4) at (3.75,4.5){};
\draw[](tid1) -- (tid2);
\draw[](tid1) -- (tid3);
\draw[](tid1) -- (tid4);
\draw[](tid0) -- (tid1);
\end{tikzpicture}
\nodepart{three}
\footnotesize{4}
\nodepart{four}
\footnotesize{$1$}
};
 & 
\\
};
\end{scope}
\begin{scope}[yshift=\leveltopIIIIIIII cm]
\matrix (line8)[column sep=0.5cm] {
\node[draw=black, rectangle split,  rectangle split parts=4] (sn0xef5c90){
\footnotesize{51.5969}
\nodepart{two}
\begin{tikzpicture}[scale=.2]
\node[circle, scale=0.75, fill] (tid0) at (0.75,1.5){};
\node[circle, scale=0.75, fill] (tid1) at (0.75,3){};
\node[circle, scale=0.75, fill] (tid2) at (0.75,4.5){};
\node[circle, scale=0.75, fill, task_scheduled] (tid3) at (0.75,6){};
\draw[](tid2) -- (tid3);
\draw[](tid1) -- (tid2);
\draw[](tid0) -- (tid1);
\end{tikzpicture}
\nodepart{three}
\footnotesize{4}
\nodepart{four}
\footnotesize{$1$}
};
 & 
\node[draw=black, rectangle split,  rectangle split parts=4] (sn0xef6940){
\footnotesize{48.4031}
\nodepart{two}
\begin{tikzpicture}[scale=.2]
\node[circle, scale=0.75, fill] (tid0) at (1.5,1.5){};
\node[circle, scale=0.75, fill] (tid1) at (1.5,3){};
\node[circle, scale=0.75, fill, task_scheduled] (tid2) at (0.75,4.5){};
\node[circle, scale=0.75, fill, task_scheduled] (tid3) at (2.25,4.5){};
\draw[](tid1) -- (tid2);
\draw[](tid1) -- (tid3);
\draw[](tid0) -- (tid1);
\end{tikzpicture}
\nodepart{three}
\footnotesize{3.5}
\nodepart{four}
\footnotesize{$1$}
};
 & 
\\
};
\end{scope}
\begin{scope}[yshift=\leveltopIIIIIIIII cm]
\matrix (line9)[column sep=0.5cm] {
\node[draw=black, rectangle split,  rectangle split parts=4] (sn0xef5950){
\footnotesize{100}
\nodepart{two}
\begin{tikzpicture}[scale=.2]
\node[circle, scale=0.75, fill] (tid0) at (0.75,1.5){};
\node[circle, scale=0.75, fill] (tid1) at (0.75,3){};
\node[circle, scale=0.75, fill, task_scheduled] (tid2) at (0.75,4.5){};
\draw[](tid1) -- (tid2);
\draw[](tid0) -- (tid1);
\end{tikzpicture}
\nodepart{three}
\footnotesize{3}
\nodepart{four}
\footnotesize{$1$}
};
 & 
\\
};
\end{scope}
\begin{scope}[yshift=\leveltopIIIIIIIIII cm]
\matrix (line10)[column sep=0.5cm] {
\node[draw=black, rectangle split,  rectangle split parts=4] (sn0xef5b30){
\footnotesize{100}
\nodepart{two}
\begin{tikzpicture}[scale=.2]
\node[circle, scale=0.75, fill] (tid0) at (0.75,1.5){};
\node[circle, scale=0.75, fill, task_scheduled] (tid1) at (0.75,3){};
\draw[](tid0) -- (tid1);
\end{tikzpicture}
\nodepart{three}
\footnotesize{2}
\nodepart{four}
\footnotesize{$1$}
};
 & 
\\
};
\end{scope}
\draw (sn0xef1ca0.south) -- (sn0xef2540.north);
\draw (sn0xef1ca0.south) -- (sn0xf02cb0.north);
\draw (sn0xef1ca0.south) -- (sn0xf03560.north);
\draw (sn0xef1ca0.south) -- (sn0xf038a0.north);
\draw (sn0xef2540.south) -- (sn0xefa920.north);
\draw (sn0xef2540.south) -- (sn0xef3e70.north);
\draw (sn0xef2540.south) -- (sn0xefabd0.north);
\draw (sn0xef2540.south) -- (sn0xefb500.north);
\draw (sn0xf02cb0.south) -- (sn0xefabd0.north);
\draw (sn0xf02cb0.south) -- (sn0xefb500.north);
\draw (sn0xf03560.south) -- (sn0xefabd0.north);
\draw (sn0xf03560.south) -- (sn0xefb500.north);
\draw (sn0xf03560.south) -- (sn0xeffff0.north);
\draw (sn0xf03560.south) -- (sn0xf00d20.north);
\draw (sn0xf03560.south) -- (sn0xf01470.north);
\draw (sn0xf038a0.south) -- (sn0xefb500.north);
\draw (sn0xf038a0.south) -- (sn0xf04590.north);
\draw (sn0xf038a0.south) -- (sn0xf04f90.north);
\draw (sn0xefa920.south) -- (sn0xef43a0.north);
\draw (sn0xefa920.south) -- (sn0xef8300.north);
\draw (sn0xefa920.south) -- (sn0xefbc60.north);
\draw (sn0xef3e70.south) -- (sn0xef4790.north);
\draw (sn0xef3e70.south) -- (sn0xef8300.north);
\draw (sn0xef3e70.south) -- (sn0xef90f0.north);
\draw (sn0xefabd0.south) -- (sn0xef8300.north);
\draw (sn0xefabd0.south) -- (sn0xef90f0.north);
\draw (sn0xefb500.south) -- (sn0xefbc60.north);
\draw (sn0xefb500.south) -- (sn0xef90f0.north);
\draw (sn0xefb500.south) -- (sn0xefc0b0.north);
\draw (sn0xeffff0.south) -- (sn0xefe720.north);
\draw (sn0xeffff0.south) -- (sn0xefea20.north);
\draw (sn0xeffff0.south) -- (sn0xef8300.north);
\draw (sn0xeffff0.south) -- (sn0xefbc60.north);
\draw (sn0xf00d20.south) -- (sn0xef8300.north);
\draw (sn0xf00d20.south) -- (sn0xef90f0.north);
\draw (sn0xf01470.south) -- (sn0xefbc60.north);
\draw (sn0xf01470.south) -- (sn0xef90f0.north);
\draw (sn0xf01470.south) -- (sn0xf01940.north);
\draw (sn0xf01470.south) -- (sn0xf01e70.north);
\draw (sn0xf01470.south) -- (sn0xf02290.north);
\draw (sn0xf04590.south) -- (sn0xefc0b0.north);
\draw (sn0xf04f90.south) -- (sn0xefc0b0.north);
\draw (sn0xf04f90.south) -- (sn0xf01940.north);
\draw (sn0xf04f90.south) -- (sn0xf01e70.north);
\draw (sn0xf04f90.south) -- (sn0xf02290.north);
\draw (sn0xef43a0.south) -- (sn0xef48f0.north);
\draw (sn0xef43a0.south) -- (sn0xef5110.north);
\draw (sn0xef4790.south) -- (sn0xef48f0.north);
\draw (sn0xef4790.south) -- (sn0xef7880.north);
\draw (sn0xef4790.south) -- (sn0xef7c10.north);
\draw (sn0xef8300.south) -- (sn0xef5110.north);
\draw (sn0xef8300.south) -- (sn0xef7880.north);
\draw (sn0xef8300.south) -- (sn0xef7c10.north);
\draw (sn0xefbc60.south) -- (sn0xef5110.north);
\draw (sn0xefbc60.south) -- (sn0xef9e10.north);
\draw (sn0xef90f0.south) -- (sn0xef7c10.north);
\draw (sn0xef90f0.south) -- (sn0xef9e10.north);
\draw (sn0xef90f0.south) -- (sn0xefa190.north);
\draw (sn0xefe720.south) -- (sn0xef5110.north);
\draw (sn0xefea20.south) -- (sn0xef5110.north);
\draw (sn0xefea20.south) -- (sn0xefeea0.north);
\draw (sn0xefea20.south) -- (sn0xeff210.north);
\draw (sn0xf01940.south) -- (sn0xefeea0.north);
\draw (sn0xf01940.south) -- (sn0xeff210.north);
\draw (sn0xf01940.south) -- (sn0xef9e10.north);
\draw (sn0xf01e70.south) -- (sn0xef9e10.north);
\draw (sn0xf01e70.south) -- (sn0xefa190.north);
\draw (sn0xf02290.south) -- (sn0xef9e10.north);
\draw (sn0xf02290.south) -- (sn0xefa190.north);
\draw (sn0xf02290.south) -- (sn0xf026b0.north);
\draw (sn0xf02290.south) -- (sn0xf02bb0.north);
\draw (sn0xefc0b0.south) -- (sn0xef9e10.north);
\draw (sn0xefc0b0.south) -- (sn0xefa190.north);
\draw (sn0xef48f0.south) -- (sn0xef5310.north);
\draw (sn0xef48f0.south) -- (sn0xef5750.north);
\draw (sn0xef5110.south) -- (sn0xef5750.north);
\draw (sn0xef5110.south) -- (sn0xef7040.north);
\draw (sn0xef7880.south) -- (sn0xef5750.north);
\draw (sn0xef7c10.south) -- (sn0xef5750.north);
\draw (sn0xef7c10.south) -- (sn0xef7d10.north);
\draw (sn0xef7c10.south) -- (sn0xef8590.north);
\draw (sn0xefeea0.south) -- (sn0xef7040.north);
\draw (sn0xeff210.south) -- (sn0xef7040.north);
\draw (sn0xeff210.south) -- (sn0xeff630.north);
\draw (sn0xeff210.south) -- (sn0xeff8f0.north);
\draw (sn0xf026b0.south) -- (sn0xeff630.north);
\draw (sn0xf026b0.south) -- (sn0xeff8f0.north);
\draw (sn0xf026b0.south) -- (sn0xef7d10.north);
\draw (sn0xf026b0.south) -- (sn0xef8590.north);
\draw (sn0xf02bb0.south) -- (sn0xef8590.north);
\draw (sn0xef9e10.south) -- (sn0xef7040.north);
\draw (sn0xef9e10.south) -- (sn0xef7d10.north);
\draw (sn0xef9e10.south) -- (sn0xef8590.north);
\draw (sn0xefa190.south) -- (sn0xef8590.north);
\draw (sn0xef5310.south) -- (sn0xef5850.north);
\draw (sn0xef5750.south) -- (sn0xef5850.north);
\draw (sn0xef5750.south) -- (sn0xef64d0.north);
\draw (sn0xeff630.south) -- (sn0xeffc80.north);
\draw (sn0xeff630.south) -- (sn0xef64d0.north);
\draw (sn0xeff8f0.south) -- (sn0xef64d0.north);
\draw (sn0xef7040.south) -- (sn0xef64d0.north);
\draw (sn0xef7d10.south) -- (sn0xef64d0.north);
\draw (sn0xef8590.south) -- (sn0xef64d0.north);
\draw (sn0xef8590.south) -- (sn0xef8750.north);
\draw (sn0xef5850.south) -- (sn0xef5c90.north);
\draw (sn0xeffc80.south) -- (sn0xef5c90.north);
\draw (sn0xef64d0.south) -- (sn0xef5c90.north);
\draw (sn0xef64d0.south) -- (sn0xef6940.north);
\draw (sn0xef8750.south) -- (sn0xef6940.north);
\draw (sn0xef5c90.south) -- (sn0xef5950.north);
\draw (sn0xef6940.south) -- (sn0xef5950.north);
\draw (sn0xef5950.south) -- (sn0xef5b30.north);
\end{tikzpicture}

%%% Local Variables:
%%% TeX-master: "thesis/thesis.tex"
%%% End: 
\renewcommand{\leveltopI}{-15cm + \leveltop}
\renewcommand{\leveltopII}{-15cm + \leveltopI}
\renewcommand{\leveltopIII}{-15cm + \leveltopII}
\renewcommand{\leveltopIIII}{-15cm + \leveltopIII}
\renewcommand{\leveltopIIIII}{-15cm + \leveltopIIII}
\renewcommand{\leveltopIIIIII}{-15cm + \leveltopIIIII}
\renewcommand{\leveltopIIIIIII}{-15cm + \leveltopIIIIII}
\renewcommand{\leveltopIIIIIIII}{-15cm + \leveltopIIIIIII}
\renewcommand{\leveltopIIIIIIIII}{-15cm + \leveltopIIIIIIII}
\renewcommand{\leveltopIIIIIIIIII}{-15cm + \leveltopIIIIIIIII}
\renewcommand{\leveltopIIIIIIIIIII}{-15cm + \leveltopIIIIIIIIII}
\begin{tikzpicture}[scale=.2, anchor=south]
\begin{scope}[yshift=\leveltopI cm]
\matrix (line1)[column sep=0.5cm] {
\node[draw=black, rectangle split,  rectangle split parts=4] (sn0xef2440){
\footnotesize{100}
\nodepart{two}
\begin{tikzpicture}[scale=.2]
\node[circle, scale=0.75, fill] (tid0) at (3,1.5){};
\node[circle, scale=0.75, fill] (tid1) at (3,3){};
\node[circle, scale=0.75, fill] (tid2) at (0.75,4.5){};
\node[circle, scale=0.75, fill] (tid5) at (0.75,6){};
\node[circle, scale=0.75, fill] (tid9) at (0.75,7.5){};
\node[circle, scale=0.75, fill, task_scheduled] (tid10) at (0.75,9){};
\draw[](tid9) -- (tid10);
\draw[](tid5) -- (tid9);
\draw[](tid2) -- (tid5);
\node[circle, scale=0.75, fill] (tid3) at (3,4.5){};
\node[circle, scale=0.75, fill] (tid6) at (2.25,6){};
\node[circle, scale=0.75, fill] (tid7) at (3.75,6){};
\draw[](tid3) -- (tid6);
\draw[](tid3) -- (tid7);
\node[circle, scale=0.75, fill] (tid4) at (5.25,4.5){};
\node[circle, scale=0.75, fill, task_scheduled] (tid8) at (5.25,6){};
\draw[](tid4) -- (tid8);
\draw[](tid1) -- (tid2);
\draw[](tid1) -- (tid3);
\draw[](tid1) -- (tid4);
\draw[](tid0) -- (tid1);
\end{tikzpicture}
\nodepart{three}
\footnotesize{7.38134}
\nodepart{four}
\footnotesize{$17\:33\:33\:17$}
};
 & 
\\
};
\end{scope}
\begin{scope}[yshift=\leveltopII cm]
\matrix (line2)[column sep=0.5cm] {
\node[draw=black, rectangle split,  rectangle split parts=4] (sn0xf051d0){
\footnotesize{16.6667}
\nodepart{two}
\begin{tikzpicture}[scale=.2]
\node[circle, scale=0.75, fill] (tid0) at (3,1.5){};
\node[circle, scale=0.75, fill] (tid1) at (3,3){};
\node[circle, scale=0.75, fill] (tid2) at (0.75,4.5){};
\node[circle, scale=0.75, fill] (tid5) at (0.75,6){};
\node[circle, scale=0.75, fill] (tid8) at (0.75,7.5){};
\node[circle, scale=0.75, fill, task_scheduled] (tid9) at (0.75,9){};
\draw[](tid8) -- (tid9);
\draw[](tid5) -- (tid8);
\draw[](tid2) -- (tid5);
\node[circle, scale=0.75, fill] (tid3) at (3,4.5){};
\node[circle, scale=0.75, fill] (tid6) at (2.25,6){};
\node[circle, scale=0.75, fill] (tid7) at (3.75,6){};
\draw[](tid3) -- (tid6);
\draw[](tid3) -- (tid7);
\node[circle, scale=0.75, fill, task_scheduled] (tid4) at (5.25,4.5){};
\draw[](tid1) -- (tid2);
\draw[](tid1) -- (tid3);
\draw[](tid1) -- (tid4);
\draw[](tid0) -- (tid1);
\end{tikzpicture}
\nodepart{three}
\footnotesize{6.96965}
\nodepart{four}
\footnotesize{$50\:33\:17$}
};
 & 
\node[draw=black, rectangle split,  rectangle split parts=4] (sn0xefcc90){
\footnotesize{33.3333}
\nodepart{two}
\begin{tikzpicture}[scale=.2]
\node[circle, scale=0.75, fill] (tid0) at (3,1.5){};
\node[circle, scale=0.75, fill] (tid1) at (3,3){};
\node[circle, scale=0.75, fill] (tid2) at (0.75,4.5){};
\node[circle, scale=0.75, fill] (tid5) at (0.75,6){};
\node[circle, scale=0.75, fill] (tid8) at (0.75,7.5){};
\node[circle, scale=0.75, fill, task_scheduled] (tid9) at (0.75,9){};
\draw[](tid8) -- (tid9);
\draw[](tid5) -- (tid8);
\draw[](tid2) -- (tid5);
\node[circle, scale=0.75, fill] (tid3) at (3,4.5){};
\node[circle, scale=0.75, fill, task_scheduled] (tid6) at (2.25,6){};
\node[circle, scale=0.75, fill] (tid7) at (3.75,6){};
\draw[](tid3) -- (tid6);
\draw[](tid3) -- (tid7);
\node[circle, scale=0.75, fill] (tid4) at (5.25,4.5){};
\draw[](tid1) -- (tid2);
\draw[](tid1) -- (tid3);
\draw[](tid1) -- (tid4);
\draw[](tid0) -- (tid1);
\end{tikzpicture}
\nodepart{three}
\footnotesize{6.96323}
\nodepart{four}
\footnotesize{$25\:25\:17\:17\:17$}
};
 & 
\node[draw=black, rectangle split,  rectangle split parts=4] (sn0xf03560){
\footnotesize{33.3333}
\nodepart{two}
\begin{tikzpicture}[scale=.2]
\node[circle, scale=0.75, fill] (tid0) at (3,1.5){};
\node[circle, scale=0.75, fill] (tid1) at (3,3){};
\node[circle, scale=0.75, fill] (tid2) at (1.5,4.5){};
\node[circle, scale=0.75, fill, task_scheduled] (tid5) at (0.75,6){};
\node[circle, scale=0.75, fill] (tid6) at (2.25,6){};
\draw[](tid2) -- (tid5);
\draw[](tid2) -- (tid6);
\node[circle, scale=0.75, fill] (tid3) at (3.75,4.5){};
\node[circle, scale=0.75, fill] (tid7) at (3.75,6){};
\node[circle, scale=0.75, fill] (tid9) at (3.75,7.5){};
\draw[](tid7) -- (tid9);
\draw[](tid3) -- (tid7);
\node[circle, scale=0.75, fill] (tid4) at (5.25,4.5){};
\node[circle, scale=0.75, fill, task_scheduled] (tid8) at (5.25,6){};
\draw[](tid4) -- (tid8);
\draw[](tid1) -- (tid2);
\draw[](tid1) -- (tid3);
\draw[](tid1) -- (tid4);
\draw[](tid0) -- (tid1);
\end{tikzpicture}
\nodepart{three}
\footnotesize{6.82847}
\nodepart{four}
\footnotesize{$25\:25\:17\:17\:17$}
};
 & 
\node[draw=black, rectangle split,  rectangle split parts=4] (sn0xf05e10){
\footnotesize{16.6667}
\nodepart{two}
\begin{tikzpicture}[scale=.2]
\node[circle, scale=0.75, fill] (tid0) at (3,1.5){};
\node[circle, scale=0.75, fill] (tid1) at (3,3){};
\node[circle, scale=0.75, fill] (tid2) at (1.5,4.5){};
\node[circle, scale=0.75, fill] (tid5) at (0.75,6){};
\node[circle, scale=0.75, fill] (tid6) at (2.25,6){};
\draw[](tid2) -- (tid5);
\draw[](tid2) -- (tid6);
\node[circle, scale=0.75, fill] (tid3) at (3.75,4.5){};
\node[circle, scale=0.75, fill] (tid7) at (3.75,6){};
\node[circle, scale=0.75, fill, task_scheduled] (tid9) at (3.75,7.5){};
\draw[](tid7) -- (tid9);
\draw[](tid3) -- (tid7);
\node[circle, scale=0.75, fill] (tid4) at (5.25,4.5){};
\node[circle, scale=0.75, fill, task_scheduled] (tid8) at (5.25,6){};
\draw[](tid4) -- (tid8);
\draw[](tid1) -- (tid2);
\draw[](tid1) -- (tid3);
\draw[](tid1) -- (tid4);
\draw[](tid0) -- (tid1);
\end{tikzpicture}
\nodepart{three}
\footnotesize{6.73495}
\nodepart{four}
\footnotesize{$17\:33\:33\:17$}
};
 & 
\\
};
\end{scope}
\begin{scope}[yshift=\leveltopIII cm]
\matrix (line3)[column sep=0.5cm] {
\node[draw=black, rectangle split,  rectangle split parts=4] (sn0xefe2f0){
\footnotesize{8.33333}
\nodepart{two}
\begin{tikzpicture}[scale=.2]
\node[circle, scale=0.75, fill] (tid0) at (2.25,1.5){};
\node[circle, scale=0.75, fill] (tid1) at (2.25,3){};
\node[circle, scale=0.75, fill] (tid2) at (0.75,4.5){};
\node[circle, scale=0.75, fill] (tid4) at (0.75,6){};
\node[circle, scale=0.75, fill] (tid7) at (0.75,7.5){};
\node[circle, scale=0.75, fill, task_scheduled] (tid8) at (0.75,9){};
\draw[](tid7) -- (tid8);
\draw[](tid4) -- (tid7);
\draw[](tid2) -- (tid4);
\node[circle, scale=0.75, fill] (tid3) at (3,4.5){};
\node[circle, scale=0.75, fill, task_scheduled] (tid5) at (2.25,6){};
\node[circle, scale=0.75, fill] (tid6) at (3.75,6){};
\draw[](tid3) -- (tid5);
\draw[](tid3) -- (tid6);
\draw[](tid1) -- (tid2);
\draw[](tid1) -- (tid3);
\draw[](tid0) -- (tid1);
\end{tikzpicture}
\nodepart{three}
\footnotesize{6.57227}
\nodepart{four}
\footnotesize{$50\:25\:25$}
};
 & 
\node[draw=black, rectangle split,  rectangle split parts=4] (sn0xefa920){
\footnotesize{8.33333}
\nodepart{two}
\begin{tikzpicture}[scale=.2]
\node[circle, scale=0.75, fill] (tid0) at (2.25,1.5){};
\node[circle, scale=0.75, fill] (tid1) at (2.25,3){};
\node[circle, scale=0.75, fill] (tid2) at (0.75,4.5){};
\node[circle, scale=0.75, fill] (tid5) at (0.75,6){};
\node[circle, scale=0.75, fill] (tid7) at (0.75,7.5){};
\node[circle, scale=0.75, fill, task_scheduled] (tid8) at (0.75,9){};
\draw[](tid7) -- (tid8);
\draw[](tid5) -- (tid7);
\draw[](tid2) -- (tid5);
\node[circle, scale=0.75, fill] (tid3) at (2.25,4.5){};
\node[circle, scale=0.75, fill] (tid6) at (2.25,6){};
\draw[](tid3) -- (tid6);
\node[circle, scale=0.75, fill, task_scheduled] (tid4) at (3.75,4.5){};
\draw[](tid1) -- (tid2);
\draw[](tid1) -- (tid3);
\draw[](tid1) -- (tid4);
\draw[](tid0) -- (tid1);
\end{tikzpicture}
\nodepart{three}
\footnotesize{6.57617}
\nodepart{four}
\footnotesize{$50\:25\:25$}
};
 & 
\node[draw=black, rectangle split,  rectangle split parts=4] (sn0xef3e70){
\footnotesize{8.33333}
\nodepart{two}
\begin{tikzpicture}[scale=.2]
\node[circle, scale=0.75, fill] (tid0) at (2.25,1.5){};
\node[circle, scale=0.75, fill] (tid1) at (2.25,3){};
\node[circle, scale=0.75, fill] (tid2) at (0.75,4.5){};
\node[circle, scale=0.75, fill] (tid5) at (0.75,6){};
\node[circle, scale=0.75, fill] (tid7) at (0.75,7.5){};
\node[circle, scale=0.75, fill, task_scheduled] (tid8) at (0.75,9){};
\draw[](tid7) -- (tid8);
\draw[](tid5) -- (tid7);
\draw[](tid2) -- (tid5);
\node[circle, scale=0.75, fill] (tid3) at (2.25,4.5){};
\node[circle, scale=0.75, fill, task_scheduled] (tid6) at (2.25,6){};
\draw[](tid3) -- (tid6);
\node[circle, scale=0.75, fill] (tid4) at (3.75,4.5){};
\draw[](tid1) -- (tid2);
\draw[](tid1) -- (tid3);
\draw[](tid1) -- (tid4);
\draw[](tid0) -- (tid1);
\end{tikzpicture}
\nodepart{three}
\footnotesize{6.58594}
\nodepart{four}
\footnotesize{$50\:25\:25$}
};
 & 
\node[draw=black, rectangle split,  rectangle split parts=4] (sn0xefabd0){
\footnotesize{8.33333}
\nodepart{two}
\begin{tikzpicture}[scale=.2]
\node[circle, scale=0.75, fill] (tid0) at (2.25,1.5){};
\node[circle, scale=0.75, fill] (tid1) at (2.25,3){};
\node[circle, scale=0.75, fill] (tid2) at (0.75,4.5){};
\node[circle, scale=0.75, fill] (tid5) at (0.75,6){};
\node[circle, scale=0.75, fill] (tid8) at (0.75,7.5){};
\draw[](tid5) -- (tid8);
\draw[](tid2) -- (tid5);
\node[circle, scale=0.75, fill] (tid3) at (2.25,4.5){};
\node[circle, scale=0.75, fill, task_scheduled] (tid6) at (2.25,6){};
\draw[](tid3) -- (tid6);
\node[circle, scale=0.75, fill] (tid4) at (3.75,4.5){};
\node[circle, scale=0.75, fill, task_scheduled] (tid7) at (3.75,6){};
\draw[](tid4) -- (tid7);
\draw[](tid1) -- (tid2);
\draw[](tid1) -- (tid3);
\draw[](tid1) -- (tid4);
\draw[](tid0) -- (tid1);
\end{tikzpicture}
\nodepart{three}
\footnotesize{6.38281}
\nodepart{four}
\footnotesize{$50\:50$}
};
 & 
\node[draw=black, rectangle split,  rectangle split parts=4] (sn0xefb500){
\footnotesize{8.33333}
\nodepart{two}
\begin{tikzpicture}[scale=.2]
\node[circle, scale=0.75, fill] (tid0) at (2.25,1.5){};
\node[circle, scale=0.75, fill] (tid1) at (2.25,3){};
\node[circle, scale=0.75, fill] (tid2) at (0.75,4.5){};
\node[circle, scale=0.75, fill] (tid5) at (0.75,6){};
\node[circle, scale=0.75, fill, task_scheduled] (tid8) at (0.75,7.5){};
\draw[](tid5) -- (tid8);
\draw[](tid2) -- (tid5);
\node[circle, scale=0.75, fill] (tid3) at (2.25,4.5){};
\node[circle, scale=0.75, fill, task_scheduled] (tid6) at (2.25,6){};
\draw[](tid3) -- (tid6);
\node[circle, scale=0.75, fill] (tid4) at (3.75,4.5){};
\node[circle, scale=0.75, fill] (tid7) at (3.75,6){};
\draw[](tid4) -- (tid7);
\draw[](tid1) -- (tid2);
\draw[](tid1) -- (tid3);
\draw[](tid1) -- (tid4);
\draw[](tid0) -- (tid1);
\end{tikzpicture}
\nodepart{three}
\footnotesize{6.24023}
\nodepart{four}
\footnotesize{$25\:25\:50$}
};
 & 
\node[draw=black, rectangle split,  rectangle split parts=4] (sn0xeffff0){
\footnotesize{16.6667}
\nodepart{two}
\begin{tikzpicture}[scale=.2]
\node[circle, scale=0.75, fill] (tid0) at (3,1.5){};
\node[circle, scale=0.75, fill] (tid1) at (3,3){};
\node[circle, scale=0.75, fill] (tid2) at (1.5,4.5){};
\node[circle, scale=0.75, fill, task_scheduled] (tid5) at (0.75,6){};
\node[circle, scale=0.75, fill] (tid6) at (2.25,6){};
\draw[](tid2) -- (tid5);
\draw[](tid2) -- (tid6);
\node[circle, scale=0.75, fill] (tid3) at (3.75,4.5){};
\node[circle, scale=0.75, fill] (tid7) at (3.75,6){};
\node[circle, scale=0.75, fill] (tid8) at (3.75,7.5){};
\draw[](tid7) -- (tid8);
\draw[](tid3) -- (tid7);
\node[circle, scale=0.75, fill, task_scheduled] (tid4) at (5.25,4.5){};
\draw[](tid1) -- (tid2);
\draw[](tid1) -- (tid3);
\draw[](tid1) -- (tid4);
\draw[](tid0) -- (tid1);
\end{tikzpicture}
\nodepart{three}
\footnotesize{6.39844}
\nodepart{four}
\footnotesize{$25\:25\:25\:25$}
};
 & 
\node[draw=black, rectangle split,  rectangle split parts=4] (sn0xf05830){
\footnotesize{5.55556}
\nodepart{two}
\begin{tikzpicture}[scale=.2]
\node[circle, scale=0.75, fill] (tid0) at (3,1.5){};
\node[circle, scale=0.75, fill] (tid1) at (3,3){};
\node[circle, scale=0.75, fill] (tid2) at (1.5,4.5){};
\node[circle, scale=0.75, fill] (tid5) at (0.75,6){};
\node[circle, scale=0.75, fill] (tid6) at (2.25,6){};
\draw[](tid2) -- (tid5);
\draw[](tid2) -- (tid6);
\node[circle, scale=0.75, fill] (tid3) at (3.75,4.5){};
\node[circle, scale=0.75, fill] (tid7) at (3.75,6){};
\node[circle, scale=0.75, fill, task_scheduled] (tid8) at (3.75,7.5){};
\draw[](tid7) -- (tid8);
\draw[](tid3) -- (tid7);
\node[circle, scale=0.75, fill, task_scheduled] (tid4) at (5.25,4.5){};
\draw[](tid1) -- (tid2);
\draw[](tid1) -- (tid3);
\draw[](tid1) -- (tid4);
\draw[](tid0) -- (tid1);
\end{tikzpicture}
\nodepart{three}
\footnotesize{6.30425}
\nodepart{four}
\footnotesize{$50\:33\:17$}
};
 & 
\node[draw=black, rectangle split,  rectangle split parts=4] (sn0xf00d20){
\footnotesize{11.1111}
\nodepart{two}
\begin{tikzpicture}[scale=.2]
\node[circle, scale=0.75, fill] (tid0) at (3,1.5){};
\node[circle, scale=0.75, fill] (tid1) at (3,3){};
\node[circle, scale=0.75, fill] (tid2) at (1.5,4.5){};
\node[circle, scale=0.75, fill, task_scheduled] (tid5) at (0.75,6){};
\node[circle, scale=0.75, fill, task_scheduled] (tid6) at (2.25,6){};
\draw[](tid2) -- (tid5);
\draw[](tid2) -- (tid6);
\node[circle, scale=0.75, fill] (tid3) at (3.75,4.5){};
\node[circle, scale=0.75, fill] (tid7) at (3.75,6){};
\node[circle, scale=0.75, fill] (tid8) at (3.75,7.5){};
\draw[](tid7) -- (tid8);
\draw[](tid3) -- (tid7);
\node[circle, scale=0.75, fill] (tid4) at (5.25,4.5){};
\draw[](tid1) -- (tid2);
\draw[](tid1) -- (tid3);
\draw[](tid1) -- (tid4);
\draw[](tid0) -- (tid1);
\end{tikzpicture}
\nodepart{three}
\footnotesize{6.38281}
\nodepart{four}
\footnotesize{$50\:50$}
};
 & 
\node[draw=black, rectangle split,  rectangle split parts=4] (sn0xf01470){
\footnotesize{16.6667}
\nodepart{two}
\begin{tikzpicture}[scale=.2]
\node[circle, scale=0.75, fill] (tid0) at (3,1.5){};
\node[circle, scale=0.75, fill] (tid1) at (3,3){};
\node[circle, scale=0.75, fill] (tid2) at (1.5,4.5){};
\node[circle, scale=0.75, fill, task_scheduled] (tid5) at (0.75,6){};
\node[circle, scale=0.75, fill] (tid6) at (2.25,6){};
\draw[](tid2) -- (tid5);
\draw[](tid2) -- (tid6);
\node[circle, scale=0.75, fill] (tid3) at (3.75,4.5){};
\node[circle, scale=0.75, fill] (tid7) at (3.75,6){};
\node[circle, scale=0.75, fill, task_scheduled] (tid8) at (3.75,7.5){};
\draw[](tid7) -- (tid8);
\draw[](tid3) -- (tid7);
\node[circle, scale=0.75, fill] (tid4) at (5.25,4.5){};
\draw[](tid1) -- (tid2);
\draw[](tid1) -- (tid3);
\draw[](tid1) -- (tid4);
\draw[](tid0) -- (tid1);
\end{tikzpicture}
\nodepart{three}
\footnotesize{6.25499}
\nodepart{four}
\footnotesize{$25\:25\:17\:17\:17$}
};
 & 
\node[draw=black, rectangle split,  rectangle split parts=4] (sn0xf04f90){
\footnotesize{5.55556}
\nodepart{two}
\begin{tikzpicture}[scale=.2]
\node[circle, scale=0.75, fill] (tid0) at (3,1.5){};
\node[circle, scale=0.75, fill] (tid1) at (3,3){};
\node[circle, scale=0.75, fill] (tid2) at (1.5,4.5){};
\node[circle, scale=0.75, fill, task_scheduled] (tid5) at (0.75,6){};
\node[circle, scale=0.75, fill] (tid6) at (2.25,6){};
\draw[](tid2) -- (tid5);
\draw[](tid2) -- (tid6);
\node[circle, scale=0.75, fill] (tid3) at (3.75,4.5){};
\node[circle, scale=0.75, fill, task_scheduled] (tid7) at (3.75,6){};
\draw[](tid3) -- (tid7);
\node[circle, scale=0.75, fill] (tid4) at (5.25,4.5){};
\node[circle, scale=0.75, fill] (tid8) at (5.25,6){};
\draw[](tid4) -- (tid8);
\draw[](tid1) -- (tid2);
\draw[](tid1) -- (tid3);
\draw[](tid1) -- (tid4);
\draw[](tid0) -- (tid1);
\end{tikzpicture}
\nodepart{three}
\footnotesize{6.18663}
\nodepart{four}
\footnotesize{$17\:17\:17\:50$}
};
 & 
\node[draw=black, rectangle split,  rectangle split parts=4] (sn0xf06500){
\footnotesize{2.77778}
\nodepart{two}
\begin{tikzpicture}[scale=.2]
\node[circle, scale=0.75, fill] (tid0) at (3,1.5){};
\node[circle, scale=0.75, fill] (tid1) at (3,3){};
\node[circle, scale=0.75, fill] (tid2) at (1.5,4.5){};
\node[circle, scale=0.75, fill] (tid5) at (0.75,6){};
\node[circle, scale=0.75, fill] (tid6) at (2.25,6){};
\draw[](tid2) -- (tid5);
\draw[](tid2) -- (tid6);
\node[circle, scale=0.75, fill] (tid3) at (3.75,4.5){};
\node[circle, scale=0.75, fill, task_scheduled] (tid7) at (3.75,6){};
\draw[](tid3) -- (tid7);
\node[circle, scale=0.75, fill] (tid4) at (5.25,4.5){};
\node[circle, scale=0.75, fill, task_scheduled] (tid8) at (5.25,6){};
\draw[](tid4) -- (tid8);
\draw[](tid1) -- (tid2);
\draw[](tid1) -- (tid3);
\draw[](tid1) -- (tid4);
\draw[](tid0) -- (tid1);
\end{tikzpicture}
\nodepart{three}
\footnotesize{6.22222}
\nodepart{four}
\footnotesize{$33\:67$}
};
 & 
\\
};
\end{scope}
\begin{scope}[yshift=\leveltopIIII cm]
\matrix (line4)[column sep=0.5cm] {
\node[draw=black, rectangle split,  rectangle split parts=4] (sn0xef43a0){
\footnotesize{8.33333}
\nodepart{two}
\begin{tikzpicture}[scale=.2]
\node[circle, scale=0.75, fill] (tid0) at (1.5,1.5){};
\node[circle, scale=0.75, fill] (tid1) at (1.5,3){};
\node[circle, scale=0.75, fill] (tid2) at (0.75,4.5){};
\node[circle, scale=0.75, fill] (tid4) at (0.75,6){};
\node[circle, scale=0.75, fill] (tid6) at (0.75,7.5){};
\node[circle, scale=0.75, fill, task_scheduled] (tid7) at (0.75,9){};
\draw[](tid6) -- (tid7);
\draw[](tid4) -- (tid6);
\draw[](tid2) -- (tid4);
\node[circle, scale=0.75, fill] (tid3) at (2.25,4.5){};
\node[circle, scale=0.75, fill, task_scheduled] (tid5) at (2.25,6){};
\draw[](tid3) -- (tid5);
\draw[](tid1) -- (tid2);
\draw[](tid1) -- (tid3);
\draw[](tid0) -- (tid1);
\end{tikzpicture}
\nodepart{three}
\footnotesize{6.25}
\nodepart{four}
\footnotesize{$50\:50$}
};
 & 
\node[draw=black, rectangle split,  rectangle split parts=4] (sn0xef4790){
\footnotesize{4.16667}
\nodepart{two}
\begin{tikzpicture}[scale=.2]
\node[circle, scale=0.75, fill] (tid0) at (2.25,1.5){};
\node[circle, scale=0.75, fill] (tid1) at (2.25,3){};
\node[circle, scale=0.75, fill] (tid2) at (0.75,4.5){};
\node[circle, scale=0.75, fill] (tid5) at (0.75,6){};
\node[circle, scale=0.75, fill] (tid6) at (0.75,7.5){};
\node[circle, scale=0.75, fill, task_scheduled] (tid7) at (0.75,9){};
\draw[](tid6) -- (tid7);
\draw[](tid5) -- (tid6);
\draw[](tid2) -- (tid5);
\node[circle, scale=0.75, fill, task_scheduled] (tid3) at (2.25,4.5){};
\node[circle, scale=0.75, fill] (tid4) at (3.75,4.5){};
\draw[](tid1) -- (tid2);
\draw[](tid1) -- (tid3);
\draw[](tid1) -- (tid4);
\draw[](tid0) -- (tid1);
\end{tikzpicture}
\nodepart{three}
\footnotesize{6.28906}
\nodepart{four}
\footnotesize{$50\:25\:25$}
};
 & 
\node[draw=black, rectangle split,  rectangle split parts=4] (sn0xef8300){
\footnotesize{18.0556}
\nodepart{two}
\begin{tikzpicture}[scale=.2]
\node[circle, scale=0.75, fill] (tid0) at (2.25,1.5){};
\node[circle, scale=0.75, fill] (tid1) at (2.25,3){};
\node[circle, scale=0.75, fill] (tid2) at (0.75,4.5){};
\node[circle, scale=0.75, fill] (tid5) at (0.75,6){};
\node[circle, scale=0.75, fill] (tid7) at (0.75,7.5){};
\draw[](tid5) -- (tid7);
\draw[](tid2) -- (tid5);
\node[circle, scale=0.75, fill] (tid3) at (2.25,4.5){};
\node[circle, scale=0.75, fill, task_scheduled] (tid6) at (2.25,6){};
\draw[](tid3) -- (tid6);
\node[circle, scale=0.75, fill, task_scheduled] (tid4) at (3.75,4.5){};
\draw[](tid1) -- (tid2);
\draw[](tid1) -- (tid3);
\draw[](tid1) -- (tid4);
\draw[](tid0) -- (tid1);
\end{tikzpicture}
\nodepart{three}
\footnotesize{5.97656}
\nodepart{four}
\footnotesize{$50\:25\:25$}
};
 & 
\node[draw=black, rectangle split,  rectangle split parts=4] (sn0xefbc60){
\footnotesize{12.5}
\nodepart{two}
\begin{tikzpicture}[scale=.2]
\node[circle, scale=0.75, fill] (tid0) at (2.25,1.5){};
\node[circle, scale=0.75, fill] (tid1) at (2.25,3){};
\node[circle, scale=0.75, fill] (tid2) at (0.75,4.5){};
\node[circle, scale=0.75, fill] (tid5) at (0.75,6){};
\node[circle, scale=0.75, fill, task_scheduled] (tid7) at (0.75,7.5){};
\draw[](tid5) -- (tid7);
\draw[](tid2) -- (tid5);
\node[circle, scale=0.75, fill] (tid3) at (2.25,4.5){};
\node[circle, scale=0.75, fill] (tid6) at (2.25,6){};
\draw[](tid3) -- (tid6);
\node[circle, scale=0.75, fill, task_scheduled] (tid4) at (3.75,4.5){};
\draw[](tid1) -- (tid2);
\draw[](tid1) -- (tid3);
\draw[](tid1) -- (tid4);
\draw[](tid0) -- (tid1);
\end{tikzpicture}
\nodepart{three}
\footnotesize{5.82812}
\nodepart{four}
\footnotesize{$50\:50$}
};
 & 
\node[draw=black, rectangle split,  rectangle split parts=4] (sn0xef90f0){
\footnotesize{18.0556}
\nodepart{two}
\begin{tikzpicture}[scale=.2]
\node[circle, scale=0.75, fill] (tid0) at (2.25,1.5){};
\node[circle, scale=0.75, fill] (tid1) at (2.25,3){};
\node[circle, scale=0.75, fill] (tid2) at (0.75,4.5){};
\node[circle, scale=0.75, fill] (tid5) at (0.75,6){};
\node[circle, scale=0.75, fill, task_scheduled] (tid7) at (0.75,7.5){};
\draw[](tid5) -- (tid7);
\draw[](tid2) -- (tid5);
\node[circle, scale=0.75, fill] (tid3) at (2.25,4.5){};
\node[circle, scale=0.75, fill, task_scheduled] (tid6) at (2.25,6){};
\draw[](tid3) -- (tid6);
\node[circle, scale=0.75, fill] (tid4) at (3.75,4.5){};
\draw[](tid1) -- (tid2);
\draw[](tid1) -- (tid3);
\draw[](tid1) -- (tid4);
\draw[](tid0) -- (tid1);
\end{tikzpicture}
\nodepart{three}
\footnotesize{5.78906}
\nodepart{four}
\footnotesize{$50\:25\:25$}
};
 & 
\node[draw=black, rectangle split,  rectangle split parts=4] (sn0xefe720){
\footnotesize{6.25}
\nodepart{two}
\begin{tikzpicture}[scale=.2]
\node[circle, scale=0.75, fill] (tid0) at (2.25,1.5){};
\node[circle, scale=0.75, fill] (tid1) at (2.25,3){};
\node[circle, scale=0.75, fill] (tid2) at (1.5,4.5){};
\node[circle, scale=0.75, fill, task_scheduled] (tid4) at (0.75,6){};
\node[circle, scale=0.75, fill, task_scheduled] (tid5) at (2.25,6){};
\draw[](tid2) -- (tid4);
\draw[](tid2) -- (tid5);
\node[circle, scale=0.75, fill] (tid3) at (3.75,4.5){};
\node[circle, scale=0.75, fill] (tid6) at (3.75,6){};
\node[circle, scale=0.75, fill] (tid7) at (3.75,7.5){};
\draw[](tid6) -- (tid7);
\draw[](tid3) -- (tid6);
\draw[](tid1) -- (tid2);
\draw[](tid1) -- (tid3);
\draw[](tid0) -- (tid1);
\end{tikzpicture}
\nodepart{three}
\footnotesize{5.9375}
\nodepart{four}
\footnotesize{$1$}
};
 & 
\node[draw=black, rectangle split,  rectangle split parts=4] (sn0xefea20){
\footnotesize{9.02778}
\nodepart{two}
\begin{tikzpicture}[scale=.2]
\node[circle, scale=0.75, fill] (tid0) at (2.25,1.5){};
\node[circle, scale=0.75, fill] (tid1) at (2.25,3){};
\node[circle, scale=0.75, fill] (tid2) at (1.5,4.5){};
\node[circle, scale=0.75, fill, task_scheduled] (tid4) at (0.75,6){};
\node[circle, scale=0.75, fill] (tid5) at (2.25,6){};
\draw[](tid2) -- (tid4);
\draw[](tid2) -- (tid5);
\node[circle, scale=0.75, fill] (tid3) at (3.75,4.5){};
\node[circle, scale=0.75, fill] (tid6) at (3.75,6){};
\node[circle, scale=0.75, fill, task_scheduled] (tid7) at (3.75,7.5){};
\draw[](tid6) -- (tid7);
\draw[](tid3) -- (tid6);
\draw[](tid1) -- (tid2);
\draw[](tid1) -- (tid3);
\draw[](tid0) -- (tid1);
\end{tikzpicture}
\nodepart{three}
\footnotesize{5.85156}
\nodepart{four}
\footnotesize{$50\:25\:25$}
};
 & 
\node[draw=black, rectangle split,  rectangle split parts=4] (sn0xf01940){
\footnotesize{5.55556}
\nodepart{two}
\begin{tikzpicture}[scale=.2]
\node[circle, scale=0.75, fill] (tid0) at (3,1.5){};
\node[circle, scale=0.75, fill] (tid1) at (3,3){};
\node[circle, scale=0.75, fill] (tid2) at (1.5,4.5){};
\node[circle, scale=0.75, fill, task_scheduled] (tid5) at (0.75,6){};
\node[circle, scale=0.75, fill] (tid6) at (2.25,6){};
\draw[](tid2) -- (tid5);
\draw[](tid2) -- (tid6);
\node[circle, scale=0.75, fill] (tid3) at (3.75,4.5){};
\node[circle, scale=0.75, fill] (tid7) at (3.75,6){};
\draw[](tid3) -- (tid7);
\node[circle, scale=0.75, fill, task_scheduled] (tid4) at (5.25,4.5){};
\draw[](tid1) -- (tid2);
\draw[](tid1) -- (tid3);
\draw[](tid1) -- (tid4);
\draw[](tid0) -- (tid1);
\end{tikzpicture}
\nodepart{three}
\footnotesize{5.74219}
\nodepart{four}
\footnotesize{$25\:25\:50$}
};
 & 
\node[draw=black, rectangle split,  rectangle split parts=4] (sn0xf069c0){
\footnotesize{1.85185}
\nodepart{two}
\begin{tikzpicture}[scale=.2]
\node[circle, scale=0.75, fill] (tid0) at (3,1.5){};
\node[circle, scale=0.75, fill] (tid1) at (3,3){};
\node[circle, scale=0.75, fill] (tid2) at (1.5,4.5){};
\node[circle, scale=0.75, fill] (tid5) at (0.75,6){};
\node[circle, scale=0.75, fill] (tid6) at (2.25,6){};
\draw[](tid2) -- (tid5);
\draw[](tid2) -- (tid6);
\node[circle, scale=0.75, fill] (tid3) at (3.75,4.5){};
\node[circle, scale=0.75, fill, task_scheduled] (tid7) at (3.75,6){};
\draw[](tid3) -- (tid7);
\node[circle, scale=0.75, fill, task_scheduled] (tid4) at (5.25,4.5){};
\draw[](tid1) -- (tid2);
\draw[](tid1) -- (tid3);
\draw[](tid1) -- (tid4);
\draw[](tid0) -- (tid1);
\end{tikzpicture}
\nodepart{three}
\footnotesize{5.78646}
\nodepart{four}
\footnotesize{$50\:17\:33$}
};
 & 
\node[draw=black, rectangle split,  rectangle split parts=4] (sn0xf01e70){
\footnotesize{3.7037}
\nodepart{two}
\begin{tikzpicture}[scale=.2]
\node[circle, scale=0.75, fill] (tid0) at (3,1.5){};
\node[circle, scale=0.75, fill] (tid1) at (3,3){};
\node[circle, scale=0.75, fill] (tid2) at (1.5,4.5){};
\node[circle, scale=0.75, fill, task_scheduled] (tid5) at (0.75,6){};
\node[circle, scale=0.75, fill, task_scheduled] (tid6) at (2.25,6){};
\draw[](tid2) -- (tid5);
\draw[](tid2) -- (tid6);
\node[circle, scale=0.75, fill] (tid3) at (3.75,4.5){};
\node[circle, scale=0.75, fill] (tid7) at (3.75,6){};
\draw[](tid3) -- (tid7);
\node[circle, scale=0.75, fill] (tid4) at (5.25,4.5){};
\draw[](tid1) -- (tid2);
\draw[](tid1) -- (tid3);
\draw[](tid1) -- (tid4);
\draw[](tid0) -- (tid1);
\end{tikzpicture}
\nodepart{three}
\footnotesize{5.67188}
\nodepart{four}
\footnotesize{$50\:50$}
};
 & 
\node[draw=black, rectangle split,  rectangle split parts=4] (sn0xf02290){
\footnotesize{5.55556}
\nodepart{two}
\begin{tikzpicture}[scale=.2]
\node[circle, scale=0.75, fill] (tid0) at (3,1.5){};
\node[circle, scale=0.75, fill] (tid1) at (3,3){};
\node[circle, scale=0.75, fill] (tid2) at (1.5,4.5){};
\node[circle, scale=0.75, fill, task_scheduled] (tid5) at (0.75,6){};
\node[circle, scale=0.75, fill] (tid6) at (2.25,6){};
\draw[](tid2) -- (tid5);
\draw[](tid2) -- (tid6);
\node[circle, scale=0.75, fill] (tid3) at (3.75,4.5){};
\node[circle, scale=0.75, fill, task_scheduled] (tid7) at (3.75,6){};
\draw[](tid3) -- (tid7);
\node[circle, scale=0.75, fill] (tid4) at (5.25,4.5){};
\draw[](tid1) -- (tid2);
\draw[](tid1) -- (tid3);
\draw[](tid1) -- (tid4);
\draw[](tid0) -- (tid1);
\end{tikzpicture}
\nodepart{three}
\footnotesize{5.6901}
\nodepart{four}
\footnotesize{$33\:17\:25\:25$}
};
 & 
\node[draw=black, rectangle split,  rectangle split parts=4] (sn0xefc0b0){
\footnotesize{6.94444}
\nodepart{two}
\begin{tikzpicture}[scale=.2]
\node[circle, scale=0.75, fill] (tid0) at (2.25,1.5){};
\node[circle, scale=0.75, fill] (tid1) at (2.25,3){};
\node[circle, scale=0.75, fill] (tid2) at (0.75,4.5){};
\node[circle, scale=0.75, fill, task_scheduled] (tid5) at (0.75,6){};
\draw[](tid2) -- (tid5);
\node[circle, scale=0.75, fill] (tid3) at (2.25,4.5){};
\node[circle, scale=0.75, fill, task_scheduled] (tid6) at (2.25,6){};
\draw[](tid3) -- (tid6);
\node[circle, scale=0.75, fill] (tid4) at (3.75,4.5){};
\node[circle, scale=0.75, fill] (tid7) at (3.75,6){};
\draw[](tid4) -- (tid7);
\draw[](tid1) -- (tid2);
\draw[](tid1) -- (tid3);
\draw[](tid1) -- (tid4);
\draw[](tid0) -- (tid1);
\end{tikzpicture}
\nodepart{three}
\footnotesize{5.67188}
\nodepart{four}
\footnotesize{$50\:50$}
};
 & 
\\
};
\end{scope}
\begin{scope}[yshift=\leveltopIIIII cm]
\matrix (line5)[column sep=0.5cm] {
\node[draw=black, rectangle split,  rectangle split parts=4] (sn0xef48f0){
\footnotesize{6.25}
\nodepart{two}
\begin{tikzpicture}[scale=.2]
\node[circle, scale=0.75, fill] (tid0) at (1.5,1.5){};
\node[circle, scale=0.75, fill] (tid1) at (1.5,3){};
\node[circle, scale=0.75, fill] (tid2) at (0.75,4.5){};
\node[circle, scale=0.75, fill] (tid4) at (0.75,6){};
\node[circle, scale=0.75, fill] (tid5) at (0.75,7.5){};
\node[circle, scale=0.75, fill, task_scheduled] (tid6) at (0.75,9){};
\draw[](tid5) -- (tid6);
\draw[](tid4) -- (tid5);
\draw[](tid2) -- (tid4);
\node[circle, scale=0.75, fill, task_scheduled] (tid3) at (2.25,4.5){};
\draw[](tid1) -- (tid2);
\draw[](tid1) -- (tid3);
\draw[](tid0) -- (tid1);
\end{tikzpicture}
\nodepart{three}
\footnotesize{6.0625}
\nodepart{four}
\footnotesize{$50\:50$}
};
 & 
\node[draw=black, rectangle split,  rectangle split parts=4] (sn0xef5110){
\footnotesize{30.2083}
\nodepart{two}
\begin{tikzpicture}[scale=.2]
\node[circle, scale=0.75, fill] (tid0) at (1.5,1.5){};
\node[circle, scale=0.75, fill] (tid1) at (1.5,3){};
\node[circle, scale=0.75, fill] (tid2) at (0.75,4.5){};
\node[circle, scale=0.75, fill] (tid4) at (0.75,6){};
\node[circle, scale=0.75, fill, task_scheduled] (tid6) at (0.75,7.5){};
\draw[](tid4) -- (tid6);
\draw[](tid2) -- (tid4);
\node[circle, scale=0.75, fill] (tid3) at (2.25,4.5){};
\node[circle, scale=0.75, fill, task_scheduled] (tid5) at (2.25,6){};
\draw[](tid3) -- (tid5);
\draw[](tid1) -- (tid2);
\draw[](tid1) -- (tid3);
\draw[](tid0) -- (tid1);
\end{tikzpicture}
\nodepart{three}
\footnotesize{5.4375}
\nodepart{four}
\footnotesize{$50\:50$}
};
 & 
\node[draw=black, rectangle split,  rectangle split parts=4] (sn0xef7880){
\footnotesize{5.55556}
\nodepart{two}
\begin{tikzpicture}[scale=.2]
\node[circle, scale=0.75, fill] (tid0) at (2.25,1.5){};
\node[circle, scale=0.75, fill] (tid1) at (2.25,3){};
\node[circle, scale=0.75, fill] (tid2) at (0.75,4.5){};
\node[circle, scale=0.75, fill] (tid5) at (0.75,6){};
\node[circle, scale=0.75, fill] (tid6) at (0.75,7.5){};
\draw[](tid5) -- (tid6);
\draw[](tid2) -- (tid5);
\node[circle, scale=0.75, fill, task_scheduled] (tid3) at (2.25,4.5){};
\node[circle, scale=0.75, fill, task_scheduled] (tid4) at (3.75,4.5){};
\draw[](tid1) -- (tid2);
\draw[](tid1) -- (tid3);
\draw[](tid1) -- (tid4);
\draw[](tid0) -- (tid1);
\end{tikzpicture}
\nodepart{three}
\footnotesize{5.625}
\nodepart{four}
\footnotesize{$1$}
};
 & 
\node[draw=black, rectangle split,  rectangle split parts=4] (sn0xef7c10){
\footnotesize{14.5833}
\nodepart{two}
\begin{tikzpicture}[scale=.2]
\node[circle, scale=0.75, fill] (tid0) at (2.25,1.5){};
\node[circle, scale=0.75, fill] (tid1) at (2.25,3){};
\node[circle, scale=0.75, fill] (tid2) at (0.75,4.5){};
\node[circle, scale=0.75, fill] (tid5) at (0.75,6){};
\node[circle, scale=0.75, fill, task_scheduled] (tid6) at (0.75,7.5){};
\draw[](tid5) -- (tid6);
\draw[](tid2) -- (tid5);
\node[circle, scale=0.75, fill, task_scheduled] (tid3) at (2.25,4.5){};
\node[circle, scale=0.75, fill] (tid4) at (3.75,4.5){};
\draw[](tid1) -- (tid2);
\draw[](tid1) -- (tid3);
\draw[](tid1) -- (tid4);
\draw[](tid0) -- (tid1);
\end{tikzpicture}
\nodepart{three}
\footnotesize{5.40625}
\nodepart{four}
\footnotesize{$50\:25\:25$}
};
 & 
\node[draw=black, rectangle split,  rectangle split parts=4] (sn0xefeea0){
\footnotesize{3.64583}
\nodepart{two}
\begin{tikzpicture}[scale=.2]
\node[circle, scale=0.75, fill] (tid0) at (2.25,1.5){};
\node[circle, scale=0.75, fill] (tid1) at (2.25,3){};
\node[circle, scale=0.75, fill] (tid2) at (1.5,4.5){};
\node[circle, scale=0.75, fill, task_scheduled] (tid4) at (0.75,6){};
\node[circle, scale=0.75, fill, task_scheduled] (tid5) at (2.25,6){};
\draw[](tid2) -- (tid4);
\draw[](tid2) -- (tid5);
\node[circle, scale=0.75, fill] (tid3) at (3.75,4.5){};
\node[circle, scale=0.75, fill] (tid6) at (3.75,6){};
\draw[](tid3) -- (tid6);
\draw[](tid1) -- (tid2);
\draw[](tid1) -- (tid3);
\draw[](tid0) -- (tid1);
\end{tikzpicture}
\nodepart{three}
\footnotesize{5.25}
\nodepart{four}
\footnotesize{$1$}
};
 & 
\node[draw=black, rectangle split,  rectangle split parts=4] (sn0xeff210){
\footnotesize{4.57176}
\nodepart{two}
\begin{tikzpicture}[scale=.2]
\node[circle, scale=0.75, fill] (tid0) at (2.25,1.5){};
\node[circle, scale=0.75, fill] (tid1) at (2.25,3){};
\node[circle, scale=0.75, fill] (tid2) at (1.5,4.5){};
\node[circle, scale=0.75, fill, task_scheduled] (tid4) at (0.75,6){};
\node[circle, scale=0.75, fill] (tid5) at (2.25,6){};
\draw[](tid2) -- (tid4);
\draw[](tid2) -- (tid5);
\node[circle, scale=0.75, fill] (tid3) at (3.75,4.5){};
\node[circle, scale=0.75, fill, task_scheduled] (tid6) at (3.75,6){};
\draw[](tid3) -- (tid6);
\draw[](tid1) -- (tid2);
\draw[](tid1) -- (tid3);
\draw[](tid0) -- (tid1);
\end{tikzpicture}
\nodepart{three}
\footnotesize{5.28125}
\nodepart{four}
\footnotesize{$25\:25\:50$}
};
 & 
\node[draw=black, rectangle split,  rectangle split parts=4] (sn0xf06c40){
\footnotesize{0.308642}
\nodepart{two}
\begin{tikzpicture}[scale=.2]
\node[circle, scale=0.75, fill] (tid0) at (3,1.5){};
\node[circle, scale=0.75, fill] (tid1) at (3,3){};
\node[circle, scale=0.75, fill] (tid2) at (1.5,4.5){};
\node[circle, scale=0.75, fill] (tid5) at (0.75,6){};
\node[circle, scale=0.75, fill] (tid6) at (2.25,6){};
\draw[](tid2) -- (tid5);
\draw[](tid2) -- (tid6);
\node[circle, scale=0.75, fill, task_scheduled] (tid3) at (3.75,4.5){};
\node[circle, scale=0.75, fill, task_scheduled] (tid4) at (5.25,4.5){};
\draw[](tid1) -- (tid2);
\draw[](tid1) -- (tid3);
\draw[](tid1) -- (tid4);
\draw[](tid0) -- (tid1);
\end{tikzpicture}
\nodepart{three}
\footnotesize{5.375}
\nodepart{four}
\footnotesize{$1$}
};
 & 
\node[draw=black, rectangle split,  rectangle split parts=4] (sn0xf026b0){
\footnotesize{2.46914}
\nodepart{two}
\begin{tikzpicture}[scale=.2]
\node[circle, scale=0.75, fill] (tid0) at (3,1.5){};
\node[circle, scale=0.75, fill] (tid1) at (3,3){};
\node[circle, scale=0.75, fill] (tid2) at (1.5,4.5){};
\node[circle, scale=0.75, fill, task_scheduled] (tid5) at (0.75,6){};
\node[circle, scale=0.75, fill] (tid6) at (2.25,6){};
\draw[](tid2) -- (tid5);
\draw[](tid2) -- (tid6);
\node[circle, scale=0.75, fill, task_scheduled] (tid3) at (3.75,4.5){};
\node[circle, scale=0.75, fill] (tid4) at (5.25,4.5){};
\draw[](tid1) -- (tid2);
\draw[](tid1) -- (tid3);
\draw[](tid1) -- (tid4);
\draw[](tid0) -- (tid1);
\end{tikzpicture}
\nodepart{three}
\footnotesize{5.25}
\nodepart{four}
\footnotesize{$25\:25\:25\:25$}
};
 & 
\node[draw=black, rectangle split,  rectangle split parts=4] (sn0xf02bb0){
\footnotesize{0.925926}
\nodepart{two}
\begin{tikzpicture}[scale=.2]
\node[circle, scale=0.75, fill] (tid0) at (3,1.5){};
\node[circle, scale=0.75, fill] (tid1) at (3,3){};
\node[circle, scale=0.75, fill] (tid2) at (1.5,4.5){};
\node[circle, scale=0.75, fill, task_scheduled] (tid5) at (0.75,6){};
\node[circle, scale=0.75, fill, task_scheduled] (tid6) at (2.25,6){};
\draw[](tid2) -- (tid5);
\draw[](tid2) -- (tid6);
\node[circle, scale=0.75, fill] (tid3) at (3.75,4.5){};
\node[circle, scale=0.75, fill] (tid4) at (5.25,4.5){};
\draw[](tid1) -- (tid2);
\draw[](tid1) -- (tid3);
\draw[](tid1) -- (tid4);
\draw[](tid0) -- (tid1);
\end{tikzpicture}
\nodepart{three}
\footnotesize{5.125}
\nodepart{four}
\footnotesize{$1$}
};
 & 
\node[draw=black, rectangle split,  rectangle split parts=4] (sn0xef9e10){
\footnotesize{20.2546}
\nodepart{two}
\begin{tikzpicture}[scale=.2]
\node[circle, scale=0.75, fill] (tid0) at (2.25,1.5){};
\node[circle, scale=0.75, fill] (tid1) at (2.25,3){};
\node[circle, scale=0.75, fill] (tid2) at (0.75,4.5){};
\node[circle, scale=0.75, fill, task_scheduled] (tid5) at (0.75,6){};
\draw[](tid2) -- (tid5);
\node[circle, scale=0.75, fill] (tid3) at (2.25,4.5){};
\node[circle, scale=0.75, fill] (tid6) at (2.25,6){};
\draw[](tid3) -- (tid6);
\node[circle, scale=0.75, fill, task_scheduled] (tid4) at (3.75,4.5){};
\draw[](tid1) -- (tid2);
\draw[](tid1) -- (tid3);
\draw[](tid1) -- (tid4);
\draw[](tid0) -- (tid1);
\end{tikzpicture}
\nodepart{three}
\footnotesize{5.21875}
\nodepart{four}
\footnotesize{$50\:25\:25$}
};
 & 
\node[draw=black, rectangle split,  rectangle split parts=4] (sn0xefa190){
\footnotesize{11.2269}
\nodepart{two}
\begin{tikzpicture}[scale=.2]
\node[circle, scale=0.75, fill] (tid0) at (2.25,1.5){};
\node[circle, scale=0.75, fill] (tid1) at (2.25,3){};
\node[circle, scale=0.75, fill] (tid2) at (0.75,4.5){};
\node[circle, scale=0.75, fill, task_scheduled] (tid5) at (0.75,6){};
\draw[](tid2) -- (tid5);
\node[circle, scale=0.75, fill] (tid3) at (2.25,4.5){};
\node[circle, scale=0.75, fill, task_scheduled] (tid6) at (2.25,6){};
\draw[](tid3) -- (tid6);
\node[circle, scale=0.75, fill] (tid4) at (3.75,4.5){};
\draw[](tid1) -- (tid2);
\draw[](tid1) -- (tid3);
\draw[](tid1) -- (tid4);
\draw[](tid0) -- (tid1);
\end{tikzpicture}
\nodepart{three}
\footnotesize{5.125}
\nodepart{four}
\footnotesize{$1$}
};
 & 
\\
};
\end{scope}
\begin{scope}[yshift=\leveltopIIIIII cm]
\matrix (line6)[column sep=0.5cm] {
\node[draw=black, rectangle split,  rectangle split parts=4] (sn0xef5310){
\footnotesize{3.125}
\nodepart{two}
\begin{tikzpicture}[scale=.2]
\node[circle, scale=0.75, fill] (tid0) at (0.75,1.5){};
\node[circle, scale=0.75, fill] (tid1) at (0.75,3){};
\node[circle, scale=0.75, fill] (tid2) at (0.75,4.5){};
\node[circle, scale=0.75, fill] (tid3) at (0.75,6){};
\node[circle, scale=0.75, fill] (tid4) at (0.75,7.5){};
\node[circle, scale=0.75, fill, task_scheduled] (tid5) at (0.75,9){};
\draw[](tid4) -- (tid5);
\draw[](tid3) -- (tid4);
\draw[](tid2) -- (tid3);
\draw[](tid1) -- (tid2);
\draw[](tid0) -- (tid1);
\end{tikzpicture}
\nodepart{three}
\footnotesize{6}
\nodepart{four}
\footnotesize{$1$}
};
 & 
\node[draw=black, rectangle split,  rectangle split parts=4] (sn0xef5750){
\footnotesize{31.0764}
\nodepart{two}
\begin{tikzpicture}[scale=.2]
\node[circle, scale=0.75, fill] (tid0) at (1.5,1.5){};
\node[circle, scale=0.75, fill] (tid1) at (1.5,3){};
\node[circle, scale=0.75, fill] (tid2) at (0.75,4.5){};
\node[circle, scale=0.75, fill] (tid4) at (0.75,6){};
\node[circle, scale=0.75, fill, task_scheduled] (tid5) at (0.75,7.5){};
\draw[](tid4) -- (tid5);
\draw[](tid2) -- (tid4);
\node[circle, scale=0.75, fill, task_scheduled] (tid3) at (2.25,4.5){};
\draw[](tid1) -- (tid2);
\draw[](tid1) -- (tid3);
\draw[](tid0) -- (tid1);
\end{tikzpicture}
\nodepart{three}
\footnotesize{5.125}
\nodepart{four}
\footnotesize{$50\:50$}
};
 & 
\node[draw=black, rectangle split,  rectangle split parts=4] (sn0xeff630){
\footnotesize{2.06887}
\nodepart{two}
\begin{tikzpicture}[scale=.2]
\node[circle, scale=0.75, fill] (tid0) at (2.25,1.5){};
\node[circle, scale=0.75, fill] (tid1) at (2.25,3){};
\node[circle, scale=0.75, fill] (tid2) at (1.5,4.5){};
\node[circle, scale=0.75, fill, task_scheduled] (tid4) at (0.75,6){};
\node[circle, scale=0.75, fill] (tid5) at (2.25,6){};
\draw[](tid2) -- (tid4);
\draw[](tid2) -- (tid5);
\node[circle, scale=0.75, fill, task_scheduled] (tid3) at (3.75,4.5){};
\draw[](tid1) -- (tid2);
\draw[](tid1) -- (tid3);
\draw[](tid0) -- (tid1);
\end{tikzpicture}
\nodepart{three}
\footnotesize{4.875}
\nodepart{four}
\footnotesize{$50\:50$}
};
 & 
\node[draw=black, rectangle split,  rectangle split parts=4] (sn0xeff8f0){
\footnotesize{1.76022}
\nodepart{two}
\begin{tikzpicture}[scale=.2]
\node[circle, scale=0.75, fill] (tid0) at (2.25,1.5){};
\node[circle, scale=0.75, fill] (tid1) at (2.25,3){};
\node[circle, scale=0.75, fill] (tid2) at (1.5,4.5){};
\node[circle, scale=0.75, fill, task_scheduled] (tid4) at (0.75,6){};
\node[circle, scale=0.75, fill, task_scheduled] (tid5) at (2.25,6){};
\draw[](tid2) -- (tid4);
\draw[](tid2) -- (tid5);
\node[circle, scale=0.75, fill] (tid3) at (3.75,4.5){};
\draw[](tid1) -- (tid2);
\draw[](tid1) -- (tid3);
\draw[](tid0) -- (tid1);
\end{tikzpicture}
\nodepart{three}
\footnotesize{4.75}
\nodepart{four}
\footnotesize{$1$}
};
 & 
\node[draw=black, rectangle split,  rectangle split parts=4] (sn0xef7040){
\footnotesize{31.1632}
\nodepart{two}
\begin{tikzpicture}[scale=.2]
\node[circle, scale=0.75, fill] (tid0) at (1.5,1.5){};
\node[circle, scale=0.75, fill] (tid1) at (1.5,3){};
\node[circle, scale=0.75, fill] (tid2) at (0.75,4.5){};
\node[circle, scale=0.75, fill, task_scheduled] (tid4) at (0.75,6){};
\draw[](tid2) -- (tid4);
\node[circle, scale=0.75, fill] (tid3) at (2.25,4.5){};
\node[circle, scale=0.75, fill, task_scheduled] (tid5) at (2.25,6){};
\draw[](tid3) -- (tid5);
\draw[](tid1) -- (tid2);
\draw[](tid1) -- (tid3);
\draw[](tid0) -- (tid1);
\end{tikzpicture}
\nodepart{three}
\footnotesize{4.75}
\nodepart{four}
\footnotesize{$1$}
};
 & 
\node[draw=black, rectangle split,  rectangle split parts=4] (sn0xef7d10){
\footnotesize{9.32678}
\nodepart{two}
\begin{tikzpicture}[scale=.2]
\node[circle, scale=0.75, fill] (tid0) at (2.25,1.5){};
\node[circle, scale=0.75, fill] (tid1) at (2.25,3){};
\node[circle, scale=0.75, fill] (tid2) at (0.75,4.5){};
\node[circle, scale=0.75, fill] (tid5) at (0.75,6){};
\draw[](tid2) -- (tid5);
\node[circle, scale=0.75, fill, task_scheduled] (tid3) at (2.25,4.5){};
\node[circle, scale=0.75, fill, task_scheduled] (tid4) at (3.75,4.5){};
\draw[](tid1) -- (tid2);
\draw[](tid1) -- (tid3);
\draw[](tid1) -- (tid4);
\draw[](tid0) -- (tid1);
\end{tikzpicture}
\nodepart{three}
\footnotesize{4.75}
\nodepart{four}
\footnotesize{$1$}
};
 & 
\node[draw=black, rectangle split,  rectangle split parts=4] (sn0xef8590){
\footnotesize{21.4796}
\nodepart{two}
\begin{tikzpicture}[scale=.2]
\node[circle, scale=0.75, fill] (tid0) at (2.25,1.5){};
\node[circle, scale=0.75, fill] (tid1) at (2.25,3){};
\node[circle, scale=0.75, fill] (tid2) at (0.75,4.5){};
\node[circle, scale=0.75, fill, task_scheduled] (tid5) at (0.75,6){};
\draw[](tid2) -- (tid5);
\node[circle, scale=0.75, fill, task_scheduled] (tid3) at (2.25,4.5){};
\node[circle, scale=0.75, fill] (tid4) at (3.75,4.5){};
\draw[](tid1) -- (tid2);
\draw[](tid1) -- (tid3);
\draw[](tid1) -- (tid4);
\draw[](tid0) -- (tid1);
\end{tikzpicture}
\nodepart{three}
\footnotesize{4.625}
\nodepart{four}
\footnotesize{$50\:50$}
};
 & 
\\
};
\end{scope}
\begin{scope}[yshift=\leveltopIIIIIII cm]
\matrix (line7)[column sep=0.5cm] {
\node[draw=black, rectangle split,  rectangle split parts=4] (sn0xef5850){
\footnotesize{18.6632}
\nodepart{two}
\begin{tikzpicture}[scale=.2]
\node[circle, scale=0.75, fill] (tid0) at (0.75,1.5){};
\node[circle, scale=0.75, fill] (tid1) at (0.75,3){};
\node[circle, scale=0.75, fill] (tid2) at (0.75,4.5){};
\node[circle, scale=0.75, fill] (tid3) at (0.75,6){};
\node[circle, scale=0.75, fill, task_scheduled] (tid4) at (0.75,7.5){};
\draw[](tid3) -- (tid4);
\draw[](tid2) -- (tid3);
\draw[](tid1) -- (tid2);
\draw[](tid0) -- (tid1);
\end{tikzpicture}
\nodepart{three}
\footnotesize{5}
\nodepart{four}
\footnotesize{$1$}
};
 & 
\node[draw=black, rectangle split,  rectangle split parts=4] (sn0xeffc80){
\footnotesize{1.03443}
\nodepart{two}
\begin{tikzpicture}[scale=.2]
\node[circle, scale=0.75, fill] (tid0) at (1.5,1.5){};
\node[circle, scale=0.75, fill] (tid1) at (1.5,3){};
\node[circle, scale=0.75, fill] (tid2) at (1.5,4.5){};
\node[circle, scale=0.75, fill, task_scheduled] (tid3) at (0.75,6){};
\node[circle, scale=0.75, fill, task_scheduled] (tid4) at (2.25,6){};
\draw[](tid2) -- (tid3);
\draw[](tid2) -- (tid4);
\draw[](tid1) -- (tid2);
\draw[](tid0) -- (tid1);
\end{tikzpicture}
\nodepart{three}
\footnotesize{4.5}
\nodepart{four}
\footnotesize{$1$}
};
 & 
\node[draw=black, rectangle split,  rectangle split parts=4] (sn0xef64d0){
\footnotesize{69.5626}
\nodepart{two}
\begin{tikzpicture}[scale=.2]
\node[circle, scale=0.75, fill] (tid0) at (1.5,1.5){};
\node[circle, scale=0.75, fill] (tid1) at (1.5,3){};
\node[circle, scale=0.75, fill] (tid2) at (0.75,4.5){};
\node[circle, scale=0.75, fill, task_scheduled] (tid4) at (0.75,6){};
\draw[](tid2) -- (tid4);
\node[circle, scale=0.75, fill, task_scheduled] (tid3) at (2.25,4.5){};
\draw[](tid1) -- (tid2);
\draw[](tid1) -- (tid3);
\draw[](tid0) -- (tid1);
\end{tikzpicture}
\nodepart{three}
\footnotesize{4.25}
\nodepart{four}
\footnotesize{$50\:50$}
};
 & 
\node[draw=black, rectangle split,  rectangle split parts=4] (sn0xef8750){
\footnotesize{10.7398}
\nodepart{two}
\begin{tikzpicture}[scale=.2]
\node[circle, scale=0.75, fill] (tid0) at (2.25,1.5){};
\node[circle, scale=0.75, fill] (tid1) at (2.25,3){};
\node[circle, scale=0.75, fill, task_scheduled] (tid2) at (0.75,4.5){};
\node[circle, scale=0.75, fill, task_scheduled] (tid3) at (2.25,4.5){};
\node[circle, scale=0.75, fill] (tid4) at (3.75,4.5){};
\draw[](tid1) -- (tid2);
\draw[](tid1) -- (tid3);
\draw[](tid1) -- (tid4);
\draw[](tid0) -- (tid1);
\end{tikzpicture}
\nodepart{three}
\footnotesize{4}
\nodepart{four}
\footnotesize{$1$}
};
 & 
\\
};
\end{scope}
\begin{scope}[yshift=\leveltopIIIIIIII cm]
\matrix (line8)[column sep=0.5cm] {
\node[draw=black, rectangle split,  rectangle split parts=4] (sn0xef5c90){
\footnotesize{54.4789}
\nodepart{two}
\begin{tikzpicture}[scale=.2]
\node[circle, scale=0.75, fill] (tid0) at (0.75,1.5){};
\node[circle, scale=0.75, fill] (tid1) at (0.75,3){};
\node[circle, scale=0.75, fill] (tid2) at (0.75,4.5){};
\node[circle, scale=0.75, fill, task_scheduled] (tid3) at (0.75,6){};
\draw[](tid2) -- (tid3);
\draw[](tid1) -- (tid2);
\draw[](tid0) -- (tid1);
\end{tikzpicture}
\nodepart{three}
\footnotesize{4}
\nodepart{four}
\footnotesize{$1$}
};
 & 
\node[draw=black, rectangle split,  rectangle split parts=4] (sn0xef6940){
\footnotesize{45.5211}
\nodepart{two}
\begin{tikzpicture}[scale=.2]
\node[circle, scale=0.75, fill] (tid0) at (1.5,1.5){};
\node[circle, scale=0.75, fill] (tid1) at (1.5,3){};
\node[circle, scale=0.75, fill, task_scheduled] (tid2) at (0.75,4.5){};
\node[circle, scale=0.75, fill, task_scheduled] (tid3) at (2.25,4.5){};
\draw[](tid1) -- (tid2);
\draw[](tid1) -- (tid3);
\draw[](tid0) -- (tid1);
\end{tikzpicture}
\nodepart{three}
\footnotesize{3.5}
\nodepart{four}
\footnotesize{$1$}
};
 & 
\\
};
\end{scope}
\begin{scope}[yshift=\leveltopIIIIIIIII cm]
\matrix (line9)[column sep=0.5cm] {
\node[draw=black, rectangle split,  rectangle split parts=4] (sn0xef5950){
\footnotesize{100}
\nodepart{two}
\begin{tikzpicture}[scale=.2]
\node[circle, scale=0.75, fill] (tid0) at (0.75,1.5){};
\node[circle, scale=0.75, fill] (tid1) at (0.75,3){};
\node[circle, scale=0.75, fill, task_scheduled] (tid2) at (0.75,4.5){};
\draw[](tid1) -- (tid2);
\draw[](tid0) -- (tid1);
\end{tikzpicture}
\nodepart{three}
\footnotesize{3}
\nodepart{four}
\footnotesize{$1$}
};
 & 
\\
};
\end{scope}
\begin{scope}[yshift=\leveltopIIIIIIIIII cm]
\matrix (line10)[column sep=0.5cm] {
\node[draw=black, rectangle split,  rectangle split parts=4] (sn0xef5b30){
\footnotesize{100}
\nodepart{two}
\begin{tikzpicture}[scale=.2]
\node[circle, scale=0.75, fill] (tid0) at (0.75,1.5){};
\node[circle, scale=0.75, fill, task_scheduled] (tid1) at (0.75,3){};
\draw[](tid0) -- (tid1);
\end{tikzpicture}
\nodepart{three}
\footnotesize{2}
\nodepart{four}
\footnotesize{$1$}
};
 & 
\\
};
\end{scope}
\draw (sn0xef2440.south) -- (sn0xf051d0.north);
\draw (sn0xef2440.south) -- (sn0xefcc90.north);
\draw (sn0xef2440.south) -- (sn0xf03560.north);
\draw (sn0xef2440.south) -- (sn0xf05e10.north);
\draw (sn0xf051d0.south) -- (sn0xefe2f0.north);
\draw (sn0xf051d0.south) -- (sn0xeffff0.north);
\draw (sn0xf051d0.south) -- (sn0xf05830.north);
\draw (sn0xefcc90.south) -- (sn0xefa920.north);
\draw (sn0xefcc90.south) -- (sn0xef3e70.north);
\draw (sn0xefcc90.south) -- (sn0xeffff0.north);
\draw (sn0xefcc90.south) -- (sn0xf00d20.north);
\draw (sn0xefcc90.south) -- (sn0xf01470.north);
\draw (sn0xf03560.south) -- (sn0xefabd0.north);
\draw (sn0xf03560.south) -- (sn0xefb500.north);
\draw (sn0xf03560.south) -- (sn0xeffff0.north);
\draw (sn0xf03560.south) -- (sn0xf00d20.north);
\draw (sn0xf03560.south) -- (sn0xf01470.north);
\draw (sn0xf05e10.south) -- (sn0xf05830.north);
\draw (sn0xf05e10.south) -- (sn0xf01470.north);
\draw (sn0xf05e10.south) -- (sn0xf04f90.north);
\draw (sn0xf05e10.south) -- (sn0xf06500.north);
\draw (sn0xefe2f0.south) -- (sn0xef43a0.north);
\draw (sn0xefe2f0.south) -- (sn0xefe720.north);
\draw (sn0xefe2f0.south) -- (sn0xefea20.north);
\draw (sn0xefa920.south) -- (sn0xef43a0.north);
\draw (sn0xefa920.south) -- (sn0xef8300.north);
\draw (sn0xefa920.south) -- (sn0xefbc60.north);
\draw (sn0xef3e70.south) -- (sn0xef4790.north);
\draw (sn0xef3e70.south) -- (sn0xef8300.north);
\draw (sn0xef3e70.south) -- (sn0xef90f0.north);
\draw (sn0xefabd0.south) -- (sn0xef8300.north);
\draw (sn0xefabd0.south) -- (sn0xef90f0.north);
\draw (sn0xefb500.south) -- (sn0xefbc60.north);
\draw (sn0xefb500.south) -- (sn0xef90f0.north);
\draw (sn0xefb500.south) -- (sn0xefc0b0.north);
\draw (sn0xeffff0.south) -- (sn0xefe720.north);
\draw (sn0xeffff0.south) -- (sn0xefea20.north);
\draw (sn0xeffff0.south) -- (sn0xef8300.north);
\draw (sn0xeffff0.south) -- (sn0xefbc60.north);
\draw (sn0xf05830.south) -- (sn0xefea20.north);
\draw (sn0xf05830.south) -- (sn0xf01940.north);
\draw (sn0xf05830.south) -- (sn0xf069c0.north);
\draw (sn0xf00d20.south) -- (sn0xef8300.north);
\draw (sn0xf00d20.south) -- (sn0xef90f0.north);
\draw (sn0xf01470.south) -- (sn0xefbc60.north);
\draw (sn0xf01470.south) -- (sn0xef90f0.north);
\draw (sn0xf01470.south) -- (sn0xf01940.north);
\draw (sn0xf01470.south) -- (sn0xf01e70.north);
\draw (sn0xf01470.south) -- (sn0xf02290.north);
\draw (sn0xf04f90.south) -- (sn0xefc0b0.north);
\draw (sn0xf04f90.south) -- (sn0xf01940.north);
\draw (sn0xf04f90.south) -- (sn0xf01e70.north);
\draw (sn0xf04f90.south) -- (sn0xf02290.north);
\draw (sn0xf06500.south) -- (sn0xf069c0.north);
\draw (sn0xf06500.south) -- (sn0xf02290.north);
\draw (sn0xef43a0.south) -- (sn0xef48f0.north);
\draw (sn0xef43a0.south) -- (sn0xef5110.north);
\draw (sn0xef4790.south) -- (sn0xef48f0.north);
\draw (sn0xef4790.south) -- (sn0xef7880.north);
\draw (sn0xef4790.south) -- (sn0xef7c10.north);
\draw (sn0xef8300.south) -- (sn0xef5110.north);
\draw (sn0xef8300.south) -- (sn0xef7880.north);
\draw (sn0xef8300.south) -- (sn0xef7c10.north);
\draw (sn0xefbc60.south) -- (sn0xef5110.north);
\draw (sn0xefbc60.south) -- (sn0xef9e10.north);
\draw (sn0xef90f0.south) -- (sn0xef7c10.north);
\draw (sn0xef90f0.south) -- (sn0xef9e10.north);
\draw (sn0xef90f0.south) -- (sn0xefa190.north);
\draw (sn0xefe720.south) -- (sn0xef5110.north);
\draw (sn0xefea20.south) -- (sn0xef5110.north);
\draw (sn0xefea20.south) -- (sn0xefeea0.north);
\draw (sn0xefea20.south) -- (sn0xeff210.north);
\draw (sn0xf01940.south) -- (sn0xefeea0.north);
\draw (sn0xf01940.south) -- (sn0xeff210.north);
\draw (sn0xf01940.south) -- (sn0xef9e10.north);
\draw (sn0xf069c0.south) -- (sn0xeff210.north);
\draw (sn0xf069c0.south) -- (sn0xf06c40.north);
\draw (sn0xf069c0.south) -- (sn0xf026b0.north);
\draw (sn0xf01e70.south) -- (sn0xef9e10.north);
\draw (sn0xf01e70.south) -- (sn0xefa190.north);
\draw (sn0xf02290.south) -- (sn0xef9e10.north);
\draw (sn0xf02290.south) -- (sn0xefa190.north);
\draw (sn0xf02290.south) -- (sn0xf026b0.north);
\draw (sn0xf02290.south) -- (sn0xf02bb0.north);
\draw (sn0xefc0b0.south) -- (sn0xef9e10.north);
\draw (sn0xefc0b0.south) -- (sn0xefa190.north);
\draw (sn0xef48f0.south) -- (sn0xef5310.north);
\draw (sn0xef48f0.south) -- (sn0xef5750.north);
\draw (sn0xef5110.south) -- (sn0xef5750.north);
\draw (sn0xef5110.south) -- (sn0xef7040.north);
\draw (sn0xef7880.south) -- (sn0xef5750.north);
\draw (sn0xef7c10.south) -- (sn0xef5750.north);
\draw (sn0xef7c10.south) -- (sn0xef7d10.north);
\draw (sn0xef7c10.south) -- (sn0xef8590.north);
\draw (sn0xefeea0.south) -- (sn0xef7040.north);
\draw (sn0xeff210.south) -- (sn0xef7040.north);
\draw (sn0xeff210.south) -- (sn0xeff630.north);
\draw (sn0xeff210.south) -- (sn0xeff8f0.north);
\draw (sn0xf06c40.south) -- (sn0xeff630.north);
\draw (sn0xf026b0.south) -- (sn0xeff630.north);
\draw (sn0xf026b0.south) -- (sn0xeff8f0.north);
\draw (sn0xf026b0.south) -- (sn0xef7d10.north);
\draw (sn0xf026b0.south) -- (sn0xef8590.north);
\draw (sn0xf02bb0.south) -- (sn0xef8590.north);
\draw (sn0xef9e10.south) -- (sn0xef7040.north);
\draw (sn0xef9e10.south) -- (sn0xef7d10.north);
\draw (sn0xef9e10.south) -- (sn0xef8590.north);
\draw (sn0xefa190.south) -- (sn0xef8590.north);
\draw (sn0xef5310.south) -- (sn0xef5850.north);
\draw (sn0xef5750.south) -- (sn0xef5850.north);
\draw (sn0xef5750.south) -- (sn0xef64d0.north);
\draw (sn0xeff630.south) -- (sn0xeffc80.north);
\draw (sn0xeff630.south) -- (sn0xef64d0.north);
\draw (sn0xeff8f0.south) -- (sn0xef64d0.north);
\draw (sn0xef7040.south) -- (sn0xef64d0.north);
\draw (sn0xef7d10.south) -- (sn0xef64d0.north);
\draw (sn0xef8590.south) -- (sn0xef64d0.north);
\draw (sn0xef8590.south) -- (sn0xef8750.north);
\draw (sn0xef5850.south) -- (sn0xef5c90.north);
\draw (sn0xeffc80.south) -- (sn0xef5c90.north);
\draw (sn0xef64d0.south) -- (sn0xef5c90.north);
\draw (sn0xef64d0.south) -- (sn0xef6940.north);
\draw (sn0xef8750.south) -- (sn0xef6940.north);
\draw (sn0xef5c90.south) -- (sn0xef5950.north);
\draw (sn0xef6940.south) -- (sn0xef5950.north);
\draw (sn0xef5950.south) -- (sn0xef5b30.north);
\end{tikzpicture}

%%% Local Variables:
%%% TeX-master: "thesis/thesis.tex"
%%% End: 


% \renewcommand{\leveltopI}{-15cm + \leveltop}
\renewcommand{\leveltopII}{-15cm + \leveltopI}
\renewcommand{\leveltopIII}{-15cm + \leveltopII}
\renewcommand{\leveltopIIII}{-15cm + \leveltopIII}
\renewcommand{\leveltopIIIII}{-15cm + \leveltopIIII}
\renewcommand{\leveltopIIIIII}{-15cm + \leveltopIIIII}
\renewcommand{\leveltopIIIIIII}{-15cm + \leveltopIIIIII}
\renewcommand{\leveltopIIIIIIII}{-15cm + \leveltopIIIIIII}
\renewcommand{\leveltopIIIIIIIII}{-15cm + \leveltopIIIIIIII}
\renewcommand{\leveltopIIIIIIIIII}{-15cm + \leveltopIIIIIIIII}
\begin{tikzpicture}[scale=.2, anchor=south]
\begin{scope}[yshift=\leveltopI cm]
\matrix (line1) [column sep=1cm] {
\node[draw=black, rectangle split,  rectangle split parts=3] (sn0x9c04b0){
\begin{tikzpicture}[scale=.2]
\node[circle, scale=0.75, fill] (tid0) at (4.5,1.5){};
\node[circle, scale=0.75, fill] (tid1) at (2.25,3){};
\node[circle, scale=0.75, fill, red] (tid4) at (0.75,4.5){};
\node[circle, scale=0.75, fill, red] (tid5) at (2.25,4.5){};
\node[circle, scale=0.75, fill, red] (tid6) at (3.75,4.5){};
\draw[](tid1) -- (tid4);
\draw[](tid1) -- (tid5);
\draw[](tid1) -- (tid6);
\node[circle, scale=0.75, fill] (tid2) at (6,3){};
\node[circle, scale=0.75, fill] (tid7) at (5.25,4.5){};
\node[circle, scale=0.75, fill] (tid8) at (6.75,4.5){};
\draw[](tid2) -- (tid7);
\draw[](tid2) -- (tid8);
\node[circle, scale=0.75, fill] (tid3) at (8.25,3){};
\node[circle, scale=0.75, fill] (tid9) at (8.25,4.5){};
\draw[](tid3) -- (tid9);
\draw[](tid0) -- (tid1);
\draw[](tid0) -- (tid2);
\draw[](tid0) -- (tid3);
\end{tikzpicture}
\nodepart{two}
\footnotesize{5.20782}
\nodepart{three}
\footnotesize{$67\:33$}
};
 & 
\\
};
\end{scope}
\begin{scope}[yshift=\leveltopII cm]
\matrix (line2) [column sep=1cm] {
\node[draw=black, rectangle split,  rectangle split parts=3] (sn0x9c1f80){
\begin{tikzpicture}[scale=.2]
\node[circle, scale=0.75, fill] (tid0) at (3.75,1.5){};
\node[circle, scale=0.75, fill] (tid1) at (1.5,3){};
\node[circle, scale=0.75, fill, red] (tid4) at (0.75,4.5){};
\node[circle, scale=0.75, fill, red] (tid5) at (2.25,4.5){};
\draw[](tid1) -- (tid4);
\draw[](tid1) -- (tid5);
\node[circle, scale=0.75, fill] (tid2) at (4.5,3){};
\node[circle, scale=0.75, fill, red] (tid6) at (3.75,4.5){};
\node[circle, scale=0.75, fill] (tid7) at (5.25,4.5){};
\draw[](tid2) -- (tid6);
\draw[](tid2) -- (tid7);
\node[circle, scale=0.75, fill] (tid3) at (6.75,3){};
\node[circle, scale=0.75, fill] (tid8) at (6.75,4.5){};
\draw[](tid3) -- (tid8);
\draw[](tid0) -- (tid1);
\draw[](tid0) -- (tid2);
\draw[](tid0) -- (tid3);
\end{tikzpicture}
\nodepart{two}
\footnotesize{4.87551}
\nodepart{three}
\footnotesize{$67\:33$}
};
 & 
\node[draw=black, rectangle split,  rectangle split parts=3] (sn0x9c2480){
\begin{tikzpicture}[scale=.2]
\node[circle, scale=0.75, fill] (tid0) at (3.75,1.5){};
\node[circle, scale=0.75, fill] (tid1) at (1.5,3){};
\node[circle, scale=0.75, fill, red] (tid4) at (0.75,4.5){};
\node[circle, scale=0.75, fill, red] (tid5) at (2.25,4.5){};
\draw[](tid1) -- (tid4);
\draw[](tid1) -- (tid5);
\node[circle, scale=0.75, fill] (tid2) at (4.5,3){};
\node[circle, scale=0.75, fill] (tid6) at (3.75,4.5){};
\node[circle, scale=0.75, fill] (tid7) at (5.25,4.5){};
\draw[](tid2) -- (tid6);
\draw[](tid2) -- (tid7);
\node[circle, scale=0.75, fill] (tid3) at (6.75,3){};
\node[circle, scale=0.75, fill, red] (tid8) at (6.75,4.5){};
\draw[](tid3) -- (tid8);
\draw[](tid0) -- (tid1);
\draw[](tid0) -- (tid2);
\draw[](tid0) -- (tid3);
\end{tikzpicture}
\nodepart{two}
\footnotesize{4.87243}
\nodepart{three}
\footnotesize{$67\:33$}
};
 & 
\\
};
\end{scope}
\begin{scope}[yshift=\leveltopIII cm]
\matrix (line3) [column sep=1cm] {
\node[draw=black, rectangle split,  rectangle split parts=3] (sn0x9c1e40){
\begin{tikzpicture}[scale=.2]
\node[circle, scale=0.75, fill] (tid0) at (3,1.5){};
\node[circle, scale=0.75, fill] (tid1) at (1.5,3){};
\node[circle, scale=0.75, fill, red] (tid4) at (0.75,4.5){};
\node[circle, scale=0.75, fill, red] (tid5) at (2.25,4.5){};
\draw[](tid1) -- (tid4);
\draw[](tid1) -- (tid5);
\node[circle, scale=0.75, fill] (tid2) at (3.75,3){};
\node[circle, scale=0.75, fill, red] (tid6) at (3.75,4.5){};
\draw[](tid2) -- (tid6);
\node[circle, scale=0.75, fill] (tid3) at (5.25,3){};
\node[circle, scale=0.75, fill] (tid7) at (5.25,4.5){};
\draw[](tid3) -- (tid7);
\draw[](tid0) -- (tid1);
\draw[](tid0) -- (tid2);
\draw[](tid0) -- (tid3);
\end{tikzpicture}
\nodepart{two}
\footnotesize{4.54321}
\nodepart{three}
\footnotesize{$67\:33$}
};
 & 
\node[draw=black, rectangle split,  rectangle split parts=3] (sn0x9c2ec0){
\begin{tikzpicture}[scale=.2]
\node[circle, scale=0.75, fill] (tid0) at (3,1.5){};
\node[circle, scale=0.75, fill] (tid1) at (1.5,3){};
\node[circle, scale=0.75, fill, red] (tid4) at (0.75,4.5){};
\node[circle, scale=0.75, fill] (tid5) at (2.25,4.5){};
\draw[](tid1) -- (tid4);
\draw[](tid1) -- (tid5);
\node[circle, scale=0.75, fill] (tid2) at (3.75,3){};
\node[circle, scale=0.75, fill, red] (tid6) at (3.75,4.5){};
\draw[](tid2) -- (tid6);
\node[circle, scale=0.75, fill] (tid3) at (5.25,3){};
\node[circle, scale=0.75, fill, red] (tid7) at (5.25,4.5){};
\draw[](tid3) -- (tid7);
\draw[](tid0) -- (tid1);
\draw[](tid0) -- (tid2);
\draw[](tid0) -- (tid3);
\end{tikzpicture}
\nodepart{two}
\footnotesize{4.54012}
\nodepart{three}
\footnotesize{$33\:67$}
};
 & 
\node[draw=black, rectangle split,  rectangle split parts=3] (sn0x9c68f0){
\begin{tikzpicture}[scale=.2]
\node[circle, scale=0.75, fill] (tid0) at (3.75,1.5){};
\node[circle, scale=0.75, fill] (tid1) at (1.5,3){};
\node[circle, scale=0.75, fill, red] (tid4) at (0.75,4.5){};
\node[circle, scale=0.75, fill, red] (tid5) at (2.25,4.5){};
\draw[](tid1) -- (tid4);
\draw[](tid1) -- (tid5);
\node[circle, scale=0.75, fill] (tid2) at (4.5,3){};
\node[circle, scale=0.75, fill, red] (tid6) at (3.75,4.5){};
\node[circle, scale=0.75, fill] (tid7) at (5.25,4.5){};
\draw[](tid2) -- (tid6);
\draw[](tid2) -- (tid7);
\node[circle, scale=0.75, fill] (tid3) at (6.75,3){};
\draw[](tid0) -- (tid1);
\draw[](tid0) -- (tid2);
\draw[](tid0) -- (tid3);
\end{tikzpicture}
\nodepart{two}
\footnotesize{4.53704}
\nodepart{three}
\footnotesize{$1$}
};
 & 
\\
};
\end{scope}
\begin{scope}[yshift=\leveltopIIII cm]
\matrix (line4) [column sep=1cm] {
\node[draw=black, rectangle split,  rectangle split parts=3] (sn0x9c3190){
\begin{tikzpicture}[scale=.2]
\node[circle, scale=0.75, fill] (tid0) at (2.25,1.5){};
\node[circle, scale=0.75, fill] (tid1) at (0.75,3){};
\node[circle, scale=0.75, fill, red] (tid4) at (0.75,4.5){};
\draw[](tid1) -- (tid4);
\node[circle, scale=0.75, fill] (tid2) at (2.25,3){};
\node[circle, scale=0.75, fill, red] (tid5) at (2.25,4.5){};
\draw[](tid2) -- (tid5);
\node[circle, scale=0.75, fill] (tid3) at (3.75,3){};
\node[circle, scale=0.75, fill, red] (tid6) at (3.75,4.5){};
\draw[](tid3) -- (tid6);
\draw[](tid0) -- (tid1);
\draw[](tid0) -- (tid2);
\draw[](tid0) -- (tid3);
\end{tikzpicture}
\nodepart{two}
\footnotesize{4.21296}
\nodepart{three}
\footnotesize{$1$}
};
 & 
\node[draw=black, rectangle split,  rectangle split parts=3] (sn0x9c3bb0){
\begin{tikzpicture}[scale=.2]
\node[circle, scale=0.75, fill] (tid0) at (3,1.5){};
\node[circle, scale=0.75, fill] (tid1) at (1.5,3){};
\node[circle, scale=0.75, fill, red] (tid4) at (0.75,4.5){};
\node[circle, scale=0.75, fill, red] (tid5) at (2.25,4.5){};
\draw[](tid1) -- (tid4);
\draw[](tid1) -- (tid5);
\node[circle, scale=0.75, fill] (tid2) at (3.75,3){};
\node[circle, scale=0.75, fill, red] (tid6) at (3.75,4.5){};
\draw[](tid2) -- (tid6);
\node[circle, scale=0.75, fill] (tid3) at (5.25,3){};
\draw[](tid0) -- (tid1);
\draw[](tid0) -- (tid2);
\draw[](tid0) -- (tid3);
\end{tikzpicture}
\nodepart{two}
\footnotesize{4.2037}
\nodepart{three}
\footnotesize{$67\:33$}
};
 & 
\\
};
\end{scope}
\begin{scope}[yshift=\leveltopIIIII cm]
\matrix (line5) [column sep=1cm] {
\node[draw=black, rectangle split,  rectangle split parts=3] (sn0x9c2bc0){
\begin{tikzpicture}[scale=.2]
\node[circle, scale=0.75, fill] (tid0) at (2.25,1.5){};
\node[circle, scale=0.75, fill] (tid1) at (0.75,3){};
\node[circle, scale=0.75, fill, red] (tid4) at (0.75,4.5){};
\draw[](tid1) -- (tid4);
\node[circle, scale=0.75, fill] (tid2) at (2.25,3){};
\node[circle, scale=0.75, fill, red] (tid5) at (2.25,4.5){};
\draw[](tid2) -- (tid5);
\node[circle, scale=0.75, fill, red] (tid3) at (3.75,3){};
\draw[](tid0) -- (tid1);
\draw[](tid0) -- (tid2);
\draw[](tid0) -- (tid3);
\end{tikzpicture}
\nodepart{two}
\footnotesize{3.87963}
\nodepart{three}
\footnotesize{$33\:67$}
};
 & 
\node[draw=black, rectangle split,  rectangle split parts=3] (sn0x9c48e0){
\begin{tikzpicture}[scale=.2]
\node[circle, scale=0.75, fill] (tid0) at (3,1.5){};
\node[circle, scale=0.75, fill] (tid1) at (1.5,3){};
\node[circle, scale=0.75, fill, red] (tid4) at (0.75,4.5){};
\node[circle, scale=0.75, fill, red] (tid5) at (2.25,4.5){};
\draw[](tid1) -- (tid4);
\draw[](tid1) -- (tid5);
\node[circle, scale=0.75, fill, red] (tid2) at (3.75,3){};
\node[circle, scale=0.75, fill] (tid3) at (5.25,3){};
\draw[](tid0) -- (tid1);
\draw[](tid0) -- (tid2);
\draw[](tid0) -- (tid3);
\end{tikzpicture}
\nodepart{two}
\footnotesize{3.85185}
\nodepart{three}
\footnotesize{$67\:33$}
};
 & 
\\
};
\end{scope}
\begin{scope}[yshift=\leveltopIIIIII cm]
\matrix (line6) [column sep=1cm] {
\node[draw=black, rectangle split,  rectangle split parts=3] (sn0x9c4130){
\begin{tikzpicture}[scale=.2]
\node[circle, scale=0.75, fill] (tid0) at (1.5,1.5){};
\node[circle, scale=0.75, fill] (tid1) at (0.75,3){};
\node[circle, scale=0.75, fill, red] (tid3) at (0.75,4.5){};
\draw[](tid1) -- (tid3);
\node[circle, scale=0.75, fill] (tid2) at (2.25,3){};
\node[circle, scale=0.75, fill, red] (tid4) at (2.25,4.5){};
\draw[](tid2) -- (tid4);
\draw[](tid0) -- (tid1);
\draw[](tid0) -- (tid2);
\end{tikzpicture}
\nodepart{two}
\footnotesize{3.75}
\nodepart{three}
\footnotesize{$1$}
};
 & 
\node[draw=black, rectangle split,  rectangle split parts=3] (sn0x9c36c0){
\begin{tikzpicture}[scale=.2]
\node[circle, scale=0.75, fill] (tid0) at (2.25,1.5){};
\node[circle, scale=0.75, fill] (tid1) at (0.75,3){};
\node[circle, scale=0.75, fill, red] (tid4) at (0.75,4.5){};
\draw[](tid1) -- (tid4);
\node[circle, scale=0.75, fill, red] (tid2) at (2.25,3){};
\node[circle, scale=0.75, fill, red] (tid3) at (3.75,3){};
\draw[](tid0) -- (tid1);
\draw[](tid0) -- (tid2);
\draw[](tid0) -- (tid3);
\end{tikzpicture}
\nodepart{two}
\footnotesize{3.44444}
\nodepart{three}
\footnotesize{$67\:33$}
};
 & 
\node[draw=black, rectangle split,  rectangle split parts=3] (sn0x9c5ae0){
\begin{tikzpicture}[scale=.2]
\node[circle, scale=0.75, fill] (tid0) at (2.25,1.5){};
\node[circle, scale=0.75, fill] (tid1) at (1.5,3){};
\node[circle, scale=0.75, fill, red] (tid3) at (0.75,4.5){};
\node[circle, scale=0.75, fill, red] (tid4) at (2.25,4.5){};
\draw[](tid1) -- (tid3);
\draw[](tid1) -- (tid4);
\node[circle, scale=0.75, fill, red] (tid2) at (3.75,3){};
\draw[](tid0) -- (tid1);
\draw[](tid0) -- (tid2);
\end{tikzpicture}
\nodepart{two}
\footnotesize{3.66667}
\nodepart{three}
\footnotesize{$67\:33$}
};
 & 
\\
};
\end{scope}
\begin{scope}[yshift=\leveltopIIIIIII cm]
\matrix (line7) [column sep=1cm] {
\node[draw=black, rectangle split,  rectangle split parts=3] (sn0x9c4350){
\begin{tikzpicture}[scale=.2]
\node[circle, scale=0.75, fill] (tid0) at (1.5,1.5){};
\node[circle, scale=0.75, fill] (tid1) at (0.75,3){};
\node[circle, scale=0.75, fill, red] (tid3) at (0.75,4.5){};
\draw[](tid1) -- (tid3);
\node[circle, scale=0.75, fill, red] (tid2) at (2.25,3){};
\draw[](tid0) -- (tid1);
\draw[](tid0) -- (tid2);
\end{tikzpicture}
\nodepart{two}
\footnotesize{3.25}
\nodepart{three}
\footnotesize{$50\:50$}
};
 & 
\node[draw=black, rectangle split,  rectangle split parts=3] (sn0x9c4af0){
\begin{tikzpicture}[scale=.2]
\node[circle, scale=0.75, fill] (tid0) at (2.25,1.5){};
\node[circle, scale=0.75, fill, red] (tid1) at (0.75,3){};
\node[circle, scale=0.75, fill, red] (tid2) at (2.25,3){};
\node[circle, scale=0.75, fill, red] (tid3) at (3.75,3){};
\draw[](tid0) -- (tid1);
\draw[](tid0) -- (tid2);
\draw[](tid0) -- (tid3);
\end{tikzpicture}
\nodepart{two}
\footnotesize{2.83333}
\nodepart{three}
\footnotesize{$1$}
};
 & 
\node[draw=black, rectangle split,  rectangle split parts=3] (sn0x9c57b0){
\begin{tikzpicture}[scale=.2]
\node[circle, scale=0.75, fill] (tid0) at (1.5,1.5){};
\node[circle, scale=0.75, fill] (tid1) at (1.5,3){};
\node[circle, scale=0.75, fill, red] (tid2) at (0.75,4.5){};
\node[circle, scale=0.75, fill, red] (tid3) at (2.25,4.5){};
\draw[](tid1) -- (tid2);
\draw[](tid1) -- (tid3);
\draw[](tid0) -- (tid1);
\end{tikzpicture}
\nodepart{two}
\footnotesize{3.5}
\nodepart{three}
\footnotesize{$1$}
};
 & 
\\
};
\end{scope}
\begin{scope}[yshift=\leveltopIIIIIIII cm]
\matrix (line8) [column sep=1cm] {
\node[draw=black, rectangle split,  rectangle split parts=3] (sn0x9c4420){
\begin{tikzpicture}[scale=.2]
\node[circle, scale=0.75, fill] (tid0) at (0.75,1.5){};
\node[circle, scale=0.75, fill] (tid1) at (0.75,3){};
\node[circle, scale=0.75, fill, red] (tid2) at (0.75,4.5){};
\draw[](tid1) -- (tid2);
\draw[](tid0) -- (tid1);
\end{tikzpicture}
\nodepart{two}
\footnotesize{3}
\nodepart{three}
\footnotesize{$1$}
};
 & 
\node[draw=black, rectangle split,  rectangle split parts=3] (sn0x9c45d0){
\begin{tikzpicture}[scale=.2]
\node[circle, scale=0.75, fill] (tid0) at (1.5,1.5){};
\node[circle, scale=0.75, fill, red] (tid1) at (0.75,3){};
\node[circle, scale=0.75, fill, red] (tid2) at (2.25,3){};
\draw[](tid0) -- (tid1);
\draw[](tid0) -- (tid2);
\end{tikzpicture}
\nodepart{two}
\footnotesize{2.5}
\nodepart{three}
\footnotesize{$1$}
};
 & 
\\
};
\end{scope}
\begin{scope}[yshift=\leveltopIIIIIIIII cm]
\matrix (line9) [column sep=1cm] {
\node[draw=black, rectangle split,  rectangle split parts=3] (sn0x9c46e0){
\begin{tikzpicture}[scale=.2]
\node[circle, scale=0.75, fill] (tid0) at (0.75,1.5){};
\node[circle, scale=0.75, fill, red] (tid1) at (0.75,3){};
\draw[](tid0) -- (tid1);
\end{tikzpicture}
\nodepart{two}
\footnotesize{2}
\nodepart{three}
\footnotesize{$1$}
};
 & 
\\
};
\end{scope}
\begin{scope}[yshift=\leveltopIIIIIIIIII cm]
\matrix (line10) [column sep=1cm] {
\node[draw=black, rectangle split,  rectangle split parts=3] (sn0x9c47b0){
\begin{tikzpicture}[scale=.2]
\node[circle, scale=0.75, fill, red] (tid0) at (0.75,1.5){};
\end{tikzpicture}
\nodepart{two}
\footnotesize{1}
\nodepart{three}
\footnotesize{$$}
};
 & 
\\
};
\end{scope}
\begin{scope}[yshift=\leveltopIIIIIIIIIII cm]
\matrix (line11) [column sep=1cm] {
\\
};
\end{scope}
\draw (sn0x9c04b0.south) -- (sn0x9c1f80.north);
\draw (sn0x9c04b0.south) -- (sn0x9c2480.north);
\draw (sn0x9c1f80.south) -- (sn0x9c1e40.north);
\draw (sn0x9c1f80.south) -- (sn0x9c2ec0.north);
\draw (sn0x9c2480.south) -- (sn0x9c2ec0.north);
\draw (sn0x9c2480.south) -- (sn0x9c68f0.north);
\draw (sn0x9c1e40.south) -- (sn0x9c3190.north);
\draw (sn0x9c1e40.south) -- (sn0x9c3bb0.north);
\draw (sn0x9c2ec0.south) -- (sn0x9c3190.north);
\draw (sn0x9c2ec0.south) -- (sn0x9c3bb0.north);
\draw (sn0x9c68f0.south) -- (sn0x9c3bb0.north);
\draw (sn0x9c3190.south) -- (sn0x9c2bc0.north);
\draw (sn0x9c3bb0.south) -- (sn0x9c2bc0.north);
\draw (sn0x9c3bb0.south) -- (sn0x9c48e0.north);
\draw (sn0x9c2bc0.south) -- (sn0x9c4130.north);
\draw (sn0x9c2bc0.south) -- (sn0x9c36c0.north);
\draw (sn0x9c48e0.south) -- (sn0x9c5ae0.north);
\draw (sn0x9c48e0.south) -- (sn0x9c36c0.north);
\draw (sn0x9c4130.south) -- (sn0x9c4350.north);
\draw (sn0x9c36c0.south) -- (sn0x9c4350.north);
\draw (sn0x9c36c0.south) -- (sn0x9c4af0.north);
\draw (sn0x9c5ae0.south) -- (sn0x9c57b0.north);
\draw (sn0x9c5ae0.south) -- (sn0x9c4350.north);
\draw (sn0x9c4350.south) -- (sn0x9c4420.north);
\draw (sn0x9c4350.south) -- (sn0x9c45d0.north);
\draw (sn0x9c4af0.south) -- (sn0x9c45d0.north);
\draw (sn0x9c57b0.south) -- (sn0x9c4420.north);
\draw (sn0x9c4420.south) -- (sn0x9c46e0.north);
\draw (sn0x9c45d0.south) -- (sn0x9c46e0.north);
\draw (sn0x9c46e0.south) -- (sn0x9c47b0.north);
\end{tikzpicture}

%%% Local Variables:
%%% TeX-master: "thesis/thesis.tex"
%%% End: 
\renewcommand{\leveltopI}{-15cm + \leveltop}
\renewcommand{\leveltopII}{-15cm + \leveltopI}
\renewcommand{\leveltopIII}{-15cm + \leveltopII}
\renewcommand{\leveltopIIII}{-15cm + \leveltopIII}
\renewcommand{\leveltopIIIII}{-15cm + \leveltopIIII}
\renewcommand{\leveltopIIIIII}{-15cm + \leveltopIIIII}
\renewcommand{\leveltopIIIIIII}{-15cm + \leveltopIIIIII}
\renewcommand{\leveltopIIIIIIII}{-15cm + \leveltopIIIIIII}
\renewcommand{\leveltopIIIIIIIII}{-15cm + \leveltopIIIIIIII}
\renewcommand{\leveltopIIIIIIIIII}{-15cm + \leveltopIIIIIIIII}
\begin{tikzpicture}[scale=.2, anchor=south]
\begin{scope}[yshift=\leveltopI cm]
\matrix (line1) [column sep=1cm] {
\node[draw=black, rectangle split,  rectangle split parts=3] (sn0x9c0580){
\begin{tikzpicture}[scale=.2]
\node[circle, scale=0.75, fill] (tid0) at (4.5,1.5){};
\node[circle, scale=0.75, fill] (tid1) at (2.25,3){};
\node[circle, scale=0.75, fill, red] (tid4) at (0.75,4.5){};
\node[circle, scale=0.75, fill, red] (tid5) at (2.25,4.5){};
\node[circle, scale=0.75, fill] (tid6) at (3.75,4.5){};
\draw[](tid1) -- (tid4);
\draw[](tid1) -- (tid5);
\draw[](tid1) -- (tid6);
\node[circle, scale=0.75, fill] (tid2) at (6,3){};
\node[circle, scale=0.75, fill, red] (tid7) at (5.25,4.5){};
\node[circle, scale=0.75, fill] (tid8) at (6.75,4.5){};
\draw[](tid2) -- (tid7);
\draw[](tid2) -- (tid8);
\node[circle, scale=0.75, fill] (tid3) at (8.25,3){};
\node[circle, scale=0.75, fill] (tid9) at (8.25,4.5){};
\draw[](tid3) -- (tid9);
\draw[](tid0) -- (tid1);
\draw[](tid0) -- (tid2);
\draw[](tid0) -- (tid3);
\end{tikzpicture}
\nodepart{two}
\footnotesize{5.20782}
\nodepart{three}
\footnotesize{$44\:22\:11\:22$}
};
 & 
\\
};
\end{scope}
\begin{scope}[yshift=\leveltopII cm]
\matrix (line2) [column sep=1cm] {
\node[draw=black, rectangle split,  rectangle split parts=3] (sn0x9c1f80){
\begin{tikzpicture}[scale=.2]
\node[circle, scale=0.75, fill] (tid0) at (3.75,1.5){};
\node[circle, scale=0.75, fill] (tid1) at (1.5,3){};
\node[circle, scale=0.75, fill, red] (tid4) at (0.75,4.5){};
\node[circle, scale=0.75, fill, red] (tid5) at (2.25,4.5){};
\draw[](tid1) -- (tid4);
\draw[](tid1) -- (tid5);
\node[circle, scale=0.75, fill] (tid2) at (4.5,3){};
\node[circle, scale=0.75, fill, red] (tid6) at (3.75,4.5){};
\node[circle, scale=0.75, fill] (tid7) at (5.25,4.5){};
\draw[](tid2) -- (tid6);
\draw[](tid2) -- (tid7);
\node[circle, scale=0.75, fill] (tid3) at (6.75,3){};
\node[circle, scale=0.75, fill] (tid8) at (6.75,4.5){};
\draw[](tid3) -- (tid8);
\draw[](tid0) -- (tid1);
\draw[](tid0) -- (tid2);
\draw[](tid0) -- (tid3);
\end{tikzpicture}
\nodepart{two}
\footnotesize{4.87551}
\nodepart{three}
\footnotesize{$67\:33$}
};
 & 
\node[draw=black, rectangle split,  rectangle split parts=3] (sn0x9c6cb0){
\begin{tikzpicture}[scale=.2]
\node[circle, scale=0.75, fill] (tid0) at (3.75,1.5){};
\node[circle, scale=0.75, fill] (tid1) at (1.5,3){};
\node[circle, scale=0.75, fill, red] (tid4) at (0.75,4.5){};
\node[circle, scale=0.75, fill] (tid5) at (2.25,4.5){};
\draw[](tid1) -- (tid4);
\draw[](tid1) -- (tid5);
\node[circle, scale=0.75, fill] (tid2) at (4.5,3){};
\node[circle, scale=0.75, fill, red] (tid6) at (3.75,4.5){};
\node[circle, scale=0.75, fill] (tid7) at (5.25,4.5){};
\draw[](tid2) -- (tid6);
\draw[](tid2) -- (tid7);
\node[circle, scale=0.75, fill] (tid3) at (6.75,3){};
\node[circle, scale=0.75, fill, red] (tid8) at (6.75,4.5){};
\draw[](tid3) -- (tid8);
\draw[](tid0) -- (tid1);
\draw[](tid0) -- (tid2);
\draw[](tid0) -- (tid3);
\end{tikzpicture}
\nodepart{two}
\footnotesize{4.87346}
\nodepart{three}
\footnotesize{$33\:33\:33$}
};
 & 
\node[draw=black, rectangle split,  rectangle split parts=3] (sn0x9c7450){
\begin{tikzpicture}[scale=.2]
\node[circle, scale=0.75, fill] (tid0) at (3.75,1.5){};
\node[circle, scale=0.75, fill] (tid1) at (2.25,3){};
\node[circle, scale=0.75, fill, red] (tid4) at (0.75,4.5){};
\node[circle, scale=0.75, fill, red] (tid5) at (2.25,4.5){};
\node[circle, scale=0.75, fill, red] (tid6) at (3.75,4.5){};
\draw[](tid1) -- (tid4);
\draw[](tid1) -- (tid5);
\draw[](tid1) -- (tid6);
\node[circle, scale=0.75, fill] (tid2) at (5.25,3){};
\node[circle, scale=0.75, fill] (tid7) at (5.25,4.5){};
\draw[](tid2) -- (tid7);
\node[circle, scale=0.75, fill] (tid3) at (6.75,3){};
\node[circle, scale=0.75, fill] (tid8) at (6.75,4.5){};
\draw[](tid3) -- (tid8);
\draw[](tid0) -- (tid1);
\draw[](tid0) -- (tid2);
\draw[](tid0) -- (tid3);
\end{tikzpicture}
\nodepart{two}
\footnotesize{4.87654}
\nodepart{three}
\footnotesize{$1$}
};
 & 
\node[draw=black, rectangle split,  rectangle split parts=3] (sn0x9c7520){
\begin{tikzpicture}[scale=.2]
\node[circle, scale=0.75, fill] (tid0) at (3.75,1.5){};
\node[circle, scale=0.75, fill] (tid1) at (2.25,3){};
\node[circle, scale=0.75, fill, red] (tid4) at (0.75,4.5){};
\node[circle, scale=0.75, fill, red] (tid5) at (2.25,4.5){};
\node[circle, scale=0.75, fill] (tid6) at (3.75,4.5){};
\draw[](tid1) -- (tid4);
\draw[](tid1) -- (tid5);
\draw[](tid1) -- (tid6);
\node[circle, scale=0.75, fill] (tid2) at (5.25,3){};
\node[circle, scale=0.75, fill, red] (tid7) at (5.25,4.5){};
\draw[](tid2) -- (tid7);
\node[circle, scale=0.75, fill] (tid3) at (6.75,3){};
\node[circle, scale=0.75, fill] (tid8) at (6.75,4.5){};
\draw[](tid3) -- (tid8);
\draw[](tid0) -- (tid1);
\draw[](tid0) -- (tid2);
\draw[](tid0) -- (tid3);
\end{tikzpicture}
\nodepart{two}
\footnotesize{4.87243}
\nodepart{three}
\footnotesize{$33\:33\:17\:17$}
};
 & 
\\
};
\end{scope}
\begin{scope}[yshift=\leveltopIII cm]
\matrix (line3) [column sep=1cm] {
\node[draw=black, rectangle split,  rectangle split parts=3] (sn0x9c1e40){
\begin{tikzpicture}[scale=.2]
\node[circle, scale=0.75, fill] (tid0) at (3,1.5){};
\node[circle, scale=0.75, fill] (tid1) at (1.5,3){};
\node[circle, scale=0.75, fill, red] (tid4) at (0.75,4.5){};
\node[circle, scale=0.75, fill, red] (tid5) at (2.25,4.5){};
\draw[](tid1) -- (tid4);
\draw[](tid1) -- (tid5);
\node[circle, scale=0.75, fill] (tid2) at (3.75,3){};
\node[circle, scale=0.75, fill, red] (tid6) at (3.75,4.5){};
\draw[](tid2) -- (tid6);
\node[circle, scale=0.75, fill] (tid3) at (5.25,3){};
\node[circle, scale=0.75, fill] (tid7) at (5.25,4.5){};
\draw[](tid3) -- (tid7);
\draw[](tid0) -- (tid1);
\draw[](tid0) -- (tid2);
\draw[](tid0) -- (tid3);
\end{tikzpicture}
\nodepart{two}
\footnotesize{4.54321}
\nodepart{three}
\footnotesize{$67\:33$}
};
 & 
\node[draw=black, rectangle split,  rectangle split parts=3] (sn0x9c2ec0){
\begin{tikzpicture}[scale=.2]
\node[circle, scale=0.75, fill] (tid0) at (3,1.5){};
\node[circle, scale=0.75, fill] (tid1) at (1.5,3){};
\node[circle, scale=0.75, fill, red] (tid4) at (0.75,4.5){};
\node[circle, scale=0.75, fill] (tid5) at (2.25,4.5){};
\draw[](tid1) -- (tid4);
\draw[](tid1) -- (tid5);
\node[circle, scale=0.75, fill] (tid2) at (3.75,3){};
\node[circle, scale=0.75, fill, red] (tid6) at (3.75,4.5){};
\draw[](tid2) -- (tid6);
\node[circle, scale=0.75, fill] (tid3) at (5.25,3){};
\node[circle, scale=0.75, fill, red] (tid7) at (5.25,4.5){};
\draw[](tid3) -- (tid7);
\draw[](tid0) -- (tid1);
\draw[](tid0) -- (tid2);
\draw[](tid0) -- (tid3);
\end{tikzpicture}
\nodepart{two}
\footnotesize{4.54012}
\nodepart{three}
\footnotesize{$33\:67$}
};
 & 
\node[draw=black, rectangle split,  rectangle split parts=3] (sn0x9c68f0){
\begin{tikzpicture}[scale=.2]
\node[circle, scale=0.75, fill] (tid0) at (3.75,1.5){};
\node[circle, scale=0.75, fill] (tid1) at (1.5,3){};
\node[circle, scale=0.75, fill, red] (tid4) at (0.75,4.5){};
\node[circle, scale=0.75, fill, red] (tid5) at (2.25,4.5){};
\draw[](tid1) -- (tid4);
\draw[](tid1) -- (tid5);
\node[circle, scale=0.75, fill] (tid2) at (4.5,3){};
\node[circle, scale=0.75, fill, red] (tid6) at (3.75,4.5){};
\node[circle, scale=0.75, fill] (tid7) at (5.25,4.5){};
\draw[](tid2) -- (tid6);
\draw[](tid2) -- (tid7);
\node[circle, scale=0.75, fill] (tid3) at (6.75,3){};
\draw[](tid0) -- (tid1);
\draw[](tid0) -- (tid2);
\draw[](tid0) -- (tid3);
\end{tikzpicture}
\nodepart{two}
\footnotesize{4.53704}
\nodepart{three}
\footnotesize{$1$}
};
 & 
\node[draw=black, rectangle split,  rectangle split parts=3] (sn0x9c7a50){
\begin{tikzpicture}[scale=.2]
\node[circle, scale=0.75, fill] (tid0) at (3.75,1.5){};
\node[circle, scale=0.75, fill] (tid1) at (2.25,3){};
\node[circle, scale=0.75, fill, red] (tid4) at (0.75,4.5){};
\node[circle, scale=0.75, fill, red] (tid5) at (2.25,4.5){};
\node[circle, scale=0.75, fill, red] (tid6) at (3.75,4.5){};
\draw[](tid1) -- (tid4);
\draw[](tid1) -- (tid5);
\draw[](tid1) -- (tid6);
\node[circle, scale=0.75, fill] (tid2) at (5.25,3){};
\node[circle, scale=0.75, fill] (tid7) at (5.25,4.5){};
\draw[](tid2) -- (tid7);
\node[circle, scale=0.75, fill] (tid3) at (6.75,3){};
\draw[](tid0) -- (tid1);
\draw[](tid0) -- (tid2);
\draw[](tid0) -- (tid3);
\end{tikzpicture}
\nodepart{two}
\footnotesize{4.53704}
\nodepart{three}
\footnotesize{$1$}
};
 & 
\node[draw=black, rectangle split,  rectangle split parts=3] (sn0x9c7b50){
\begin{tikzpicture}[scale=.2]
\node[circle, scale=0.75, fill] (tid0) at (3.75,1.5){};
\node[circle, scale=0.75, fill] (tid1) at (2.25,3){};
\node[circle, scale=0.75, fill, red] (tid4) at (0.75,4.5){};
\node[circle, scale=0.75, fill, red] (tid5) at (2.25,4.5){};
\node[circle, scale=0.75, fill] (tid6) at (3.75,4.5){};
\draw[](tid1) -- (tid4);
\draw[](tid1) -- (tid5);
\draw[](tid1) -- (tid6);
\node[circle, scale=0.75, fill] (tid2) at (5.25,3){};
\node[circle, scale=0.75, fill, red] (tid7) at (5.25,4.5){};
\draw[](tid2) -- (tid7);
\node[circle, scale=0.75, fill] (tid3) at (6.75,3){};
\draw[](tid0) -- (tid1);
\draw[](tid0) -- (tid2);
\draw[](tid0) -- (tid3);
\end{tikzpicture}
\nodepart{two}
\footnotesize{4.53086}
\nodepart{three}
\footnotesize{$67\:33$}
};
 & 
\\
};
\end{scope}
\begin{scope}[yshift=\leveltopIIII cm]
\matrix (line4) [column sep=1cm] {
\node[draw=black, rectangle split,  rectangle split parts=3] (sn0x9c3190){
\begin{tikzpicture}[scale=.2]
\node[circle, scale=0.75, fill] (tid0) at (2.25,1.5){};
\node[circle, scale=0.75, fill] (tid1) at (0.75,3){};
\node[circle, scale=0.75, fill, red] (tid4) at (0.75,4.5){};
\draw[](tid1) -- (tid4);
\node[circle, scale=0.75, fill] (tid2) at (2.25,3){};
\node[circle, scale=0.75, fill, red] (tid5) at (2.25,4.5){};
\draw[](tid2) -- (tid5);
\node[circle, scale=0.75, fill] (tid3) at (3.75,3){};
\node[circle, scale=0.75, fill, red] (tid6) at (3.75,4.5){};
\draw[](tid3) -- (tid6);
\draw[](tid0) -- (tid1);
\draw[](tid0) -- (tid2);
\draw[](tid0) -- (tid3);
\end{tikzpicture}
\nodepart{two}
\footnotesize{4.21296}
\nodepart{three}
\footnotesize{$1$}
};
 & 
\node[draw=black, rectangle split,  rectangle split parts=3] (sn0x9c3bb0){
\begin{tikzpicture}[scale=.2]
\node[circle, scale=0.75, fill] (tid0) at (3,1.5){};
\node[circle, scale=0.75, fill] (tid1) at (1.5,3){};
\node[circle, scale=0.75, fill, red] (tid4) at (0.75,4.5){};
\node[circle, scale=0.75, fill, red] (tid5) at (2.25,4.5){};
\draw[](tid1) -- (tid4);
\draw[](tid1) -- (tid5);
\node[circle, scale=0.75, fill] (tid2) at (3.75,3){};
\node[circle, scale=0.75, fill, red] (tid6) at (3.75,4.5){};
\draw[](tid2) -- (tid6);
\node[circle, scale=0.75, fill] (tid3) at (5.25,3){};
\draw[](tid0) -- (tid1);
\draw[](tid0) -- (tid2);
\draw[](tid0) -- (tid3);
\end{tikzpicture}
\nodepart{two}
\footnotesize{4.2037}
\nodepart{three}
\footnotesize{$67\:33$}
};
 & 
\node[draw=black, rectangle split,  rectangle split parts=3] (sn0x9c8420){
\begin{tikzpicture}[scale=.2]
\node[circle, scale=0.75, fill] (tid0) at (3.75,1.5){};
\node[circle, scale=0.75, fill] (tid1) at (2.25,3){};
\node[circle, scale=0.75, fill, red] (tid4) at (0.75,4.5){};
\node[circle, scale=0.75, fill, red] (tid5) at (2.25,4.5){};
\node[circle, scale=0.75, fill, red] (tid6) at (3.75,4.5){};
\draw[](tid1) -- (tid4);
\draw[](tid1) -- (tid5);
\draw[](tid1) -- (tid6);
\node[circle, scale=0.75, fill] (tid2) at (5.25,3){};
\node[circle, scale=0.75, fill] (tid3) at (6.75,3){};
\draw[](tid0) -- (tid1);
\draw[](tid0) -- (tid2);
\draw[](tid0) -- (tid3);
\end{tikzpicture}
\nodepart{two}
\footnotesize{4.18519}
\nodepart{three}
\footnotesize{$1$}
};
 & 
\\
};
\end{scope}
\begin{scope}[yshift=\leveltopIIIII cm]
\matrix (line5) [column sep=1cm] {
\node[draw=black, rectangle split,  rectangle split parts=3] (sn0x9c2bc0){
\begin{tikzpicture}[scale=.2]
\node[circle, scale=0.75, fill] (tid0) at (2.25,1.5){};
\node[circle, scale=0.75, fill] (tid1) at (0.75,3){};
\node[circle, scale=0.75, fill, red] (tid4) at (0.75,4.5){};
\draw[](tid1) -- (tid4);
\node[circle, scale=0.75, fill] (tid2) at (2.25,3){};
\node[circle, scale=0.75, fill, red] (tid5) at (2.25,4.5){};
\draw[](tid2) -- (tid5);
\node[circle, scale=0.75, fill, red] (tid3) at (3.75,3){};
\draw[](tid0) -- (tid1);
\draw[](tid0) -- (tid2);
\draw[](tid0) -- (tid3);
\end{tikzpicture}
\nodepart{two}
\footnotesize{3.87963}
\nodepart{three}
\footnotesize{$33\:67$}
};
 & 
\node[draw=black, rectangle split,  rectangle split parts=3] (sn0x9c48e0){
\begin{tikzpicture}[scale=.2]
\node[circle, scale=0.75, fill] (tid0) at (3,1.5){};
\node[circle, scale=0.75, fill] (tid1) at (1.5,3){};
\node[circle, scale=0.75, fill, red] (tid4) at (0.75,4.5){};
\node[circle, scale=0.75, fill, red] (tid5) at (2.25,4.5){};
\draw[](tid1) -- (tid4);
\draw[](tid1) -- (tid5);
\node[circle, scale=0.75, fill, red] (tid2) at (3.75,3){};
\node[circle, scale=0.75, fill] (tid3) at (5.25,3){};
\draw[](tid0) -- (tid1);
\draw[](tid0) -- (tid2);
\draw[](tid0) -- (tid3);
\end{tikzpicture}
\nodepart{two}
\footnotesize{3.85185}
\nodepart{three}
\footnotesize{$67\:33$}
};
 & 
\\
};
\end{scope}
\begin{scope}[yshift=\leveltopIIIIII cm]
\matrix (line6) [column sep=1cm] {
\node[draw=black, rectangle split,  rectangle split parts=3] (sn0x9c4130){
\begin{tikzpicture}[scale=.2]
\node[circle, scale=0.75, fill] (tid0) at (1.5,1.5){};
\node[circle, scale=0.75, fill] (tid1) at (0.75,3){};
\node[circle, scale=0.75, fill, red] (tid3) at (0.75,4.5){};
\draw[](tid1) -- (tid3);
\node[circle, scale=0.75, fill] (tid2) at (2.25,3){};
\node[circle, scale=0.75, fill, red] (tid4) at (2.25,4.5){};
\draw[](tid2) -- (tid4);
\draw[](tid0) -- (tid1);
\draw[](tid0) -- (tid2);
\end{tikzpicture}
\nodepart{two}
\footnotesize{3.75}
\nodepart{three}
\footnotesize{$1$}
};
 & 
\node[draw=black, rectangle split,  rectangle split parts=3] (sn0x9c36c0){
\begin{tikzpicture}[scale=.2]
\node[circle, scale=0.75, fill] (tid0) at (2.25,1.5){};
\node[circle, scale=0.75, fill] (tid1) at (0.75,3){};
\node[circle, scale=0.75, fill, red] (tid4) at (0.75,4.5){};
\draw[](tid1) -- (tid4);
\node[circle, scale=0.75, fill, red] (tid2) at (2.25,3){};
\node[circle, scale=0.75, fill, red] (tid3) at (3.75,3){};
\draw[](tid0) -- (tid1);
\draw[](tid0) -- (tid2);
\draw[](tid0) -- (tid3);
\end{tikzpicture}
\nodepart{two}
\footnotesize{3.44444}
\nodepart{three}
\footnotesize{$67\:33$}
};
 & 
\node[draw=black, rectangle split,  rectangle split parts=3] (sn0x9c5ae0){
\begin{tikzpicture}[scale=.2]
\node[circle, scale=0.75, fill] (tid0) at (2.25,1.5){};
\node[circle, scale=0.75, fill] (tid1) at (1.5,3){};
\node[circle, scale=0.75, fill, red] (tid3) at (0.75,4.5){};
\node[circle, scale=0.75, fill, red] (tid4) at (2.25,4.5){};
\draw[](tid1) -- (tid3);
\draw[](tid1) -- (tid4);
\node[circle, scale=0.75, fill, red] (tid2) at (3.75,3){};
\draw[](tid0) -- (tid1);
\draw[](tid0) -- (tid2);
\end{tikzpicture}
\nodepart{two}
\footnotesize{3.66667}
\nodepart{three}
\footnotesize{$67\:33$}
};
 & 
\\
};
\end{scope}
\begin{scope}[yshift=\leveltopIIIIIII cm]
\matrix (line7) [column sep=1cm] {
\node[draw=black, rectangle split,  rectangle split parts=3] (sn0x9c4350){
\begin{tikzpicture}[scale=.2]
\node[circle, scale=0.75, fill] (tid0) at (1.5,1.5){};
\node[circle, scale=0.75, fill] (tid1) at (0.75,3){};
\node[circle, scale=0.75, fill, red] (tid3) at (0.75,4.5){};
\draw[](tid1) -- (tid3);
\node[circle, scale=0.75, fill, red] (tid2) at (2.25,3){};
\draw[](tid0) -- (tid1);
\draw[](tid0) -- (tid2);
\end{tikzpicture}
\nodepart{two}
\footnotesize{3.25}
\nodepart{three}
\footnotesize{$50\:50$}
};
 & 
\node[draw=black, rectangle split,  rectangle split parts=3] (sn0x9c4af0){
\begin{tikzpicture}[scale=.2]
\node[circle, scale=0.75, fill] (tid0) at (2.25,1.5){};
\node[circle, scale=0.75, fill, red] (tid1) at (0.75,3){};
\node[circle, scale=0.75, fill, red] (tid2) at (2.25,3){};
\node[circle, scale=0.75, fill, red] (tid3) at (3.75,3){};
\draw[](tid0) -- (tid1);
\draw[](tid0) -- (tid2);
\draw[](tid0) -- (tid3);
\end{tikzpicture}
\nodepart{two}
\footnotesize{2.83333}
\nodepart{three}
\footnotesize{$1$}
};
 & 
\node[draw=black, rectangle split,  rectangle split parts=3] (sn0x9c57b0){
\begin{tikzpicture}[scale=.2]
\node[circle, scale=0.75, fill] (tid0) at (1.5,1.5){};
\node[circle, scale=0.75, fill] (tid1) at (1.5,3){};
\node[circle, scale=0.75, fill, red] (tid2) at (0.75,4.5){};
\node[circle, scale=0.75, fill, red] (tid3) at (2.25,4.5){};
\draw[](tid1) -- (tid2);
\draw[](tid1) -- (tid3);
\draw[](tid0) -- (tid1);
\end{tikzpicture}
\nodepart{two}
\footnotesize{3.5}
\nodepart{three}
\footnotesize{$1$}
};
 & 
\\
};
\end{scope}
\begin{scope}[yshift=\leveltopIIIIIIII cm]
\matrix (line8) [column sep=1cm] {
\node[draw=black, rectangle split,  rectangle split parts=3] (sn0x9c4420){
\begin{tikzpicture}[scale=.2]
\node[circle, scale=0.75, fill] (tid0) at (0.75,1.5){};
\node[circle, scale=0.75, fill] (tid1) at (0.75,3){};
\node[circle, scale=0.75, fill, red] (tid2) at (0.75,4.5){};
\draw[](tid1) -- (tid2);
\draw[](tid0) -- (tid1);
\end{tikzpicture}
\nodepart{two}
\footnotesize{3}
\nodepart{three}
\footnotesize{$1$}
};
 & 
\node[draw=black, rectangle split,  rectangle split parts=3] (sn0x9c45d0){
\begin{tikzpicture}[scale=.2]
\node[circle, scale=0.75, fill] (tid0) at (1.5,1.5){};
\node[circle, scale=0.75, fill, red] (tid1) at (0.75,3){};
\node[circle, scale=0.75, fill, red] (tid2) at (2.25,3){};
\draw[](tid0) -- (tid1);
\draw[](tid0) -- (tid2);
\end{tikzpicture}
\nodepart{two}
\footnotesize{2.5}
\nodepart{three}
\footnotesize{$1$}
};
 & 
\\
};
\end{scope}
\begin{scope}[yshift=\leveltopIIIIIIIII cm]
\matrix (line9) [column sep=1cm] {
\node[draw=black, rectangle split,  rectangle split parts=3] (sn0x9c46e0){
\begin{tikzpicture}[scale=.2]
\node[circle, scale=0.75, fill] (tid0) at (0.75,1.5){};
\node[circle, scale=0.75, fill, red] (tid1) at (0.75,3){};
\draw[](tid0) -- (tid1);
\end{tikzpicture}
\nodepart{two}
\footnotesize{2}
\nodepart{three}
\footnotesize{$1$}
};
 & 
\\
};
\end{scope}
\begin{scope}[yshift=\leveltopIIIIIIIIII cm]
\matrix (line10) [column sep=1cm] {
\node[draw=black, rectangle split,  rectangle split parts=3] (sn0x9c47b0){
\begin{tikzpicture}[scale=.2]
\node[circle, scale=0.75, fill, red] (tid0) at (0.75,1.5){};
\end{tikzpicture}
\nodepart{two}
\footnotesize{1}
\nodepart{three}
\footnotesize{$$}
};
 & 
\\
};
\end{scope}
\begin{scope}[yshift=\leveltopIIIIIIIIIII cm]
\matrix (line11) [column sep=1cm] {
\\
};
\end{scope}
\draw (sn0x9c0580.south) -- (sn0x9c1f80.north);
\draw (sn0x9c0580.south) -- (sn0x9c6cb0.north);
\draw (sn0x9c0580.south) -- (sn0x9c7450.north);
\draw (sn0x9c0580.south) -- (sn0x9c7520.north);
\draw (sn0x9c1f80.south) -- (sn0x9c1e40.north);
\draw (sn0x9c1f80.south) -- (sn0x9c2ec0.north);
\draw (sn0x9c6cb0.south) -- (sn0x9c2ec0.north);
\draw (sn0x9c6cb0.south) -- (sn0x9c1e40.north);
\draw (sn0x9c6cb0.south) -- (sn0x9c68f0.north);
\draw (sn0x9c7450.south) -- (sn0x9c1e40.north);
\draw (sn0x9c7520.south) -- (sn0x9c1e40.north);
\draw (sn0x9c7520.south) -- (sn0x9c2ec0.north);
\draw (sn0x9c7520.south) -- (sn0x9c7a50.north);
\draw (sn0x9c7520.south) -- (sn0x9c7b50.north);
\draw (sn0x9c1e40.south) -- (sn0x9c3190.north);
\draw (sn0x9c1e40.south) -- (sn0x9c3bb0.north);
\draw (sn0x9c2ec0.south) -- (sn0x9c3190.north);
\draw (sn0x9c2ec0.south) -- (sn0x9c3bb0.north);
\draw (sn0x9c68f0.south) -- (sn0x9c3bb0.north);
\draw (sn0x9c7a50.south) -- (sn0x9c3bb0.north);
\draw (sn0x9c7b50.south) -- (sn0x9c3bb0.north);
\draw (sn0x9c7b50.south) -- (sn0x9c8420.north);
\draw (sn0x9c3190.south) -- (sn0x9c2bc0.north);
\draw (sn0x9c3bb0.south) -- (sn0x9c2bc0.north);
\draw (sn0x9c3bb0.south) -- (sn0x9c48e0.north);
\draw (sn0x9c8420.south) -- (sn0x9c48e0.north);
\draw (sn0x9c2bc0.south) -- (sn0x9c4130.north);
\draw (sn0x9c2bc0.south) -- (sn0x9c36c0.north);
\draw (sn0x9c48e0.south) -- (sn0x9c5ae0.north);
\draw (sn0x9c48e0.south) -- (sn0x9c36c0.north);
\draw (sn0x9c4130.south) -- (sn0x9c4350.north);
\draw (sn0x9c36c0.south) -- (sn0x9c4350.north);
\draw (sn0x9c36c0.south) -- (sn0x9c4af0.north);
\draw (sn0x9c5ae0.south) -- (sn0x9c57b0.north);
\draw (sn0x9c5ae0.south) -- (sn0x9c4350.north);
\draw (sn0x9c4350.south) -- (sn0x9c4420.north);
\draw (sn0x9c4350.south) -- (sn0x9c45d0.north);
\draw (sn0x9c4af0.south) -- (sn0x9c45d0.north);
\draw (sn0x9c57b0.south) -- (sn0x9c4420.north);
\draw (sn0x9c4420.south) -- (sn0x9c46e0.north);
\draw (sn0x9c45d0.south) -- (sn0x9c46e0.north);
\draw (sn0x9c46e0.south) -- (sn0x9c47b0.north);
\end{tikzpicture}

%%% Local Variables:
%%% TeX-master: "thesis/thesis.tex"
%%% End: 
\renewcommand{\leveltopI}{-15cm + \leveltop}
\renewcommand{\leveltopII}{-15cm + \leveltopI}
\renewcommand{\leveltopIII}{-15cm + \leveltopII}
\renewcommand{\leveltopIIII}{-15cm + \leveltopIII}
\renewcommand{\leveltopIIIII}{-15cm + \leveltopIIII}
\renewcommand{\leveltopIIIIII}{-15cm + \leveltopIIIII}
\renewcommand{\leveltopIIIIIII}{-15cm + \leveltopIIIIII}
\renewcommand{\leveltopIIIIIIII}{-15cm + \leveltopIIIIIII}
\renewcommand{\leveltopIIIIIIIII}{-15cm + \leveltopIIIIIIII}
\renewcommand{\leveltopIIIIIIIIII}{-15cm + \leveltopIIIIIIIII}
\begin{tikzpicture}[scale=.2, anchor=south]
\begin{scope}[yshift=\leveltopI cm]
\matrix (line1) [column sep=1cm] {
\node[draw=black, rectangle split,  rectangle split parts=3] (sn0x9c0e30){
\begin{tikzpicture}[scale=.2]
\node[circle, scale=0.75, fill] (tid0) at (4.5,1.5){};
\node[circle, scale=0.75, fill] (tid1) at (2.25,3){};
\node[circle, scale=0.75, fill, red] (tid4) at (0.75,4.5){};
\node[circle, scale=0.75, fill] (tid5) at (2.25,4.5){};
\node[circle, scale=0.75, fill] (tid6) at (3.75,4.5){};
\draw[](tid1) -- (tid4);
\draw[](tid1) -- (tid5);
\draw[](tid1) -- (tid6);
\node[circle, scale=0.75, fill] (tid2) at (6,3){};
\node[circle, scale=0.75, fill, red] (tid7) at (5.25,4.5){};
\node[circle, scale=0.75, fill, red] (tid8) at (6.75,4.5){};
\draw[](tid2) -- (tid7);
\draw[](tid2) -- (tid8);
\node[circle, scale=0.75, fill] (tid3) at (8.25,3){};
\node[circle, scale=0.75, fill] (tid9) at (8.25,4.5){};
\draw[](tid3) -- (tid9);
\draw[](tid0) -- (tid1);
\draw[](tid0) -- (tid2);
\draw[](tid0) -- (tid3);
\end{tikzpicture}
\nodepart{two}
\footnotesize{5.2053}
\nodepart{three}
\footnotesize{$22\:11\:44\:22$}
};
 & 
\\
};
\end{scope}
\begin{scope}[yshift=\leveltopII cm]
\matrix (line2) [column sep=1cm] {
\node[draw=black, rectangle split,  rectangle split parts=3] (sn0x9c1f80){
\begin{tikzpicture}[scale=.2]
\node[circle, scale=0.75, fill] (tid0) at (3.75,1.5){};
\node[circle, scale=0.75, fill] (tid1) at (1.5,3){};
\node[circle, scale=0.75, fill, red] (tid4) at (0.75,4.5){};
\node[circle, scale=0.75, fill, red] (tid5) at (2.25,4.5){};
\draw[](tid1) -- (tid4);
\draw[](tid1) -- (tid5);
\node[circle, scale=0.75, fill] (tid2) at (4.5,3){};
\node[circle, scale=0.75, fill, red] (tid6) at (3.75,4.5){};
\node[circle, scale=0.75, fill] (tid7) at (5.25,4.5){};
\draw[](tid2) -- (tid6);
\draw[](tid2) -- (tid7);
\node[circle, scale=0.75, fill] (tid3) at (6.75,3){};
\node[circle, scale=0.75, fill] (tid8) at (6.75,4.5){};
\draw[](tid3) -- (tid8);
\draw[](tid0) -- (tid1);
\draw[](tid0) -- (tid2);
\draw[](tid0) -- (tid3);
\end{tikzpicture}
\nodepart{two}
\footnotesize{4.87551}
\nodepart{three}
\footnotesize{$67\:33$}
};
 & 
\node[draw=black, rectangle split,  rectangle split parts=3] (sn0x9c2480){
\begin{tikzpicture}[scale=.2]
\node[circle, scale=0.75, fill] (tid0) at (3.75,1.5){};
\node[circle, scale=0.75, fill] (tid1) at (1.5,3){};
\node[circle, scale=0.75, fill, red] (tid4) at (0.75,4.5){};
\node[circle, scale=0.75, fill, red] (tid5) at (2.25,4.5){};
\draw[](tid1) -- (tid4);
\draw[](tid1) -- (tid5);
\node[circle, scale=0.75, fill] (tid2) at (4.5,3){};
\node[circle, scale=0.75, fill] (tid6) at (3.75,4.5){};
\node[circle, scale=0.75, fill] (tid7) at (5.25,4.5){};
\draw[](tid2) -- (tid6);
\draw[](tid2) -- (tid7);
\node[circle, scale=0.75, fill] (tid3) at (6.75,3){};
\node[circle, scale=0.75, fill, red] (tid8) at (6.75,4.5){};
\draw[](tid3) -- (tid8);
\draw[](tid0) -- (tid1);
\draw[](tid0) -- (tid2);
\draw[](tid0) -- (tid3);
\end{tikzpicture}
\nodepart{two}
\footnotesize{4.87243}
\nodepart{three}
\footnotesize{$67\:33$}
};
 & 
\node[draw=black, rectangle split,  rectangle split parts=3] (sn0x9c7520){
\begin{tikzpicture}[scale=.2]
\node[circle, scale=0.75, fill] (tid0) at (3.75,1.5){};
\node[circle, scale=0.75, fill] (tid1) at (2.25,3){};
\node[circle, scale=0.75, fill, red] (tid4) at (0.75,4.5){};
\node[circle, scale=0.75, fill, red] (tid5) at (2.25,4.5){};
\node[circle, scale=0.75, fill] (tid6) at (3.75,4.5){};
\draw[](tid1) -- (tid4);
\draw[](tid1) -- (tid5);
\draw[](tid1) -- (tid6);
\node[circle, scale=0.75, fill] (tid2) at (5.25,3){};
\node[circle, scale=0.75, fill, red] (tid7) at (5.25,4.5){};
\draw[](tid2) -- (tid7);
\node[circle, scale=0.75, fill] (tid3) at (6.75,3){};
\node[circle, scale=0.75, fill] (tid8) at (6.75,4.5){};
\draw[](tid3) -- (tid8);
\draw[](tid0) -- (tid1);
\draw[](tid0) -- (tid2);
\draw[](tid0) -- (tid3);
\end{tikzpicture}
\nodepart{two}
\footnotesize{4.87243}
\nodepart{three}
\footnotesize{$33\:33\:17\:17$}
};
 & 
\node[draw=black, rectangle split,  rectangle split parts=3] (sn0x9c8f30){
\begin{tikzpicture}[scale=.2]
\node[circle, scale=0.75, fill] (tid0) at (3.75,1.5){};
\node[circle, scale=0.75, fill] (tid1) at (2.25,3){};
\node[circle, scale=0.75, fill, red] (tid4) at (0.75,4.5){};
\node[circle, scale=0.75, fill] (tid5) at (2.25,4.5){};
\node[circle, scale=0.75, fill] (tid6) at (3.75,4.5){};
\draw[](tid1) -- (tid4);
\draw[](tid1) -- (tid5);
\draw[](tid1) -- (tid6);
\node[circle, scale=0.75, fill] (tid2) at (5.25,3){};
\node[circle, scale=0.75, fill, red] (tid7) at (5.25,4.5){};
\draw[](tid2) -- (tid7);
\node[circle, scale=0.75, fill] (tid3) at (6.75,3){};
\node[circle, scale=0.75, fill, red] (tid8) at (6.75,4.5){};
\draw[](tid3) -- (tid8);
\draw[](tid0) -- (tid1);
\draw[](tid0) -- (tid2);
\draw[](tid0) -- (tid3);
\end{tikzpicture}
\nodepart{two}
\footnotesize{4.86728}
\nodepart{three}
\footnotesize{$33\:67$}
};
 & 
\\
};
\end{scope}
\begin{scope}[yshift=\leveltopIII cm]
\matrix (line3) [column sep=1cm] {
\node[draw=black, rectangle split,  rectangle split parts=3] (sn0x9c1e40){
\begin{tikzpicture}[scale=.2]
\node[circle, scale=0.75, fill] (tid0) at (3,1.5){};
\node[circle, scale=0.75, fill] (tid1) at (1.5,3){};
\node[circle, scale=0.75, fill, red] (tid4) at (0.75,4.5){};
\node[circle, scale=0.75, fill, red] (tid5) at (2.25,4.5){};
\draw[](tid1) -- (tid4);
\draw[](tid1) -- (tid5);
\node[circle, scale=0.75, fill] (tid2) at (3.75,3){};
\node[circle, scale=0.75, fill, red] (tid6) at (3.75,4.5){};
\draw[](tid2) -- (tid6);
\node[circle, scale=0.75, fill] (tid3) at (5.25,3){};
\node[circle, scale=0.75, fill] (tid7) at (5.25,4.5){};
\draw[](tid3) -- (tid7);
\draw[](tid0) -- (tid1);
\draw[](tid0) -- (tid2);
\draw[](tid0) -- (tid3);
\end{tikzpicture}
\nodepart{two}
\footnotesize{4.54321}
\nodepart{three}
\footnotesize{$67\:33$}
};
 & 
\node[draw=black, rectangle split,  rectangle split parts=3] (sn0x9c2ec0){
\begin{tikzpicture}[scale=.2]
\node[circle, scale=0.75, fill] (tid0) at (3,1.5){};
\node[circle, scale=0.75, fill] (tid1) at (1.5,3){};
\node[circle, scale=0.75, fill, red] (tid4) at (0.75,4.5){};
\node[circle, scale=0.75, fill] (tid5) at (2.25,4.5){};
\draw[](tid1) -- (tid4);
\draw[](tid1) -- (tid5);
\node[circle, scale=0.75, fill] (tid2) at (3.75,3){};
\node[circle, scale=0.75, fill, red] (tid6) at (3.75,4.5){};
\draw[](tid2) -- (tid6);
\node[circle, scale=0.75, fill] (tid3) at (5.25,3){};
\node[circle, scale=0.75, fill, red] (tid7) at (5.25,4.5){};
\draw[](tid3) -- (tid7);
\draw[](tid0) -- (tid1);
\draw[](tid0) -- (tid2);
\draw[](tid0) -- (tid3);
\end{tikzpicture}
\nodepart{two}
\footnotesize{4.54012}
\nodepart{three}
\footnotesize{$33\:67$}
};
 & 
\node[draw=black, rectangle split,  rectangle split parts=3] (sn0x9c68f0){
\begin{tikzpicture}[scale=.2]
\node[circle, scale=0.75, fill] (tid0) at (3.75,1.5){};
\node[circle, scale=0.75, fill] (tid1) at (1.5,3){};
\node[circle, scale=0.75, fill, red] (tid4) at (0.75,4.5){};
\node[circle, scale=0.75, fill, red] (tid5) at (2.25,4.5){};
\draw[](tid1) -- (tid4);
\draw[](tid1) -- (tid5);
\node[circle, scale=0.75, fill] (tid2) at (4.5,3){};
\node[circle, scale=0.75, fill, red] (tid6) at (3.75,4.5){};
\node[circle, scale=0.75, fill] (tid7) at (5.25,4.5){};
\draw[](tid2) -- (tid6);
\draw[](tid2) -- (tid7);
\node[circle, scale=0.75, fill] (tid3) at (6.75,3){};
\draw[](tid0) -- (tid1);
\draw[](tid0) -- (tid2);
\draw[](tid0) -- (tid3);
\end{tikzpicture}
\nodepart{two}
\footnotesize{4.53704}
\nodepart{three}
\footnotesize{$1$}
};
 & 
\node[draw=black, rectangle split,  rectangle split parts=3] (sn0x9c7a50){
\begin{tikzpicture}[scale=.2]
\node[circle, scale=0.75, fill] (tid0) at (3.75,1.5){};
\node[circle, scale=0.75, fill] (tid1) at (2.25,3){};
\node[circle, scale=0.75, fill, red] (tid4) at (0.75,4.5){};
\node[circle, scale=0.75, fill, red] (tid5) at (2.25,4.5){};
\node[circle, scale=0.75, fill, red] (tid6) at (3.75,4.5){};
\draw[](tid1) -- (tid4);
\draw[](tid1) -- (tid5);
\draw[](tid1) -- (tid6);
\node[circle, scale=0.75, fill] (tid2) at (5.25,3){};
\node[circle, scale=0.75, fill] (tid7) at (5.25,4.5){};
\draw[](tid2) -- (tid7);
\node[circle, scale=0.75, fill] (tid3) at (6.75,3){};
\draw[](tid0) -- (tid1);
\draw[](tid0) -- (tid2);
\draw[](tid0) -- (tid3);
\end{tikzpicture}
\nodepart{two}
\footnotesize{4.53704}
\nodepart{three}
\footnotesize{$1$}
};
 & 
\node[draw=black, rectangle split,  rectangle split parts=3] (sn0x9c7b50){
\begin{tikzpicture}[scale=.2]
\node[circle, scale=0.75, fill] (tid0) at (3.75,1.5){};
\node[circle, scale=0.75, fill] (tid1) at (2.25,3){};
\node[circle, scale=0.75, fill, red] (tid4) at (0.75,4.5){};
\node[circle, scale=0.75, fill, red] (tid5) at (2.25,4.5){};
\node[circle, scale=0.75, fill] (tid6) at (3.75,4.5){};
\draw[](tid1) -- (tid4);
\draw[](tid1) -- (tid5);
\draw[](tid1) -- (tid6);
\node[circle, scale=0.75, fill] (tid2) at (5.25,3){};
\node[circle, scale=0.75, fill, red] (tid7) at (5.25,4.5){};
\draw[](tid2) -- (tid7);
\node[circle, scale=0.75, fill] (tid3) at (6.75,3){};
\draw[](tid0) -- (tid1);
\draw[](tid0) -- (tid2);
\draw[](tid0) -- (tid3);
\end{tikzpicture}
\nodepart{two}
\footnotesize{4.53086}
\nodepart{three}
\footnotesize{$67\:33$}
};
 & 
\\
};
\end{scope}
\begin{scope}[yshift=\leveltopIIII cm]
\matrix (line4) [column sep=1cm] {
\node[draw=black, rectangle split,  rectangle split parts=3] (sn0x9c3190){
\begin{tikzpicture}[scale=.2]
\node[circle, scale=0.75, fill] (tid0) at (2.25,1.5){};
\node[circle, scale=0.75, fill] (tid1) at (0.75,3){};
\node[circle, scale=0.75, fill, red] (tid4) at (0.75,4.5){};
\draw[](tid1) -- (tid4);
\node[circle, scale=0.75, fill] (tid2) at (2.25,3){};
\node[circle, scale=0.75, fill, red] (tid5) at (2.25,4.5){};
\draw[](tid2) -- (tid5);
\node[circle, scale=0.75, fill] (tid3) at (3.75,3){};
\node[circle, scale=0.75, fill, red] (tid6) at (3.75,4.5){};
\draw[](tid3) -- (tid6);
\draw[](tid0) -- (tid1);
\draw[](tid0) -- (tid2);
\draw[](tid0) -- (tid3);
\end{tikzpicture}
\nodepart{two}
\footnotesize{4.21296}
\nodepart{three}
\footnotesize{$1$}
};
 & 
\node[draw=black, rectangle split,  rectangle split parts=3] (sn0x9c3bb0){
\begin{tikzpicture}[scale=.2]
\node[circle, scale=0.75, fill] (tid0) at (3,1.5){};
\node[circle, scale=0.75, fill] (tid1) at (1.5,3){};
\node[circle, scale=0.75, fill, red] (tid4) at (0.75,4.5){};
\node[circle, scale=0.75, fill, red] (tid5) at (2.25,4.5){};
\draw[](tid1) -- (tid4);
\draw[](tid1) -- (tid5);
\node[circle, scale=0.75, fill] (tid2) at (3.75,3){};
\node[circle, scale=0.75, fill, red] (tid6) at (3.75,4.5){};
\draw[](tid2) -- (tid6);
\node[circle, scale=0.75, fill] (tid3) at (5.25,3){};
\draw[](tid0) -- (tid1);
\draw[](tid0) -- (tid2);
\draw[](tid0) -- (tid3);
\end{tikzpicture}
\nodepart{two}
\footnotesize{4.2037}
\nodepart{three}
\footnotesize{$67\:33$}
};
 & 
\node[draw=black, rectangle split,  rectangle split parts=3] (sn0x9c8420){
\begin{tikzpicture}[scale=.2]
\node[circle, scale=0.75, fill] (tid0) at (3.75,1.5){};
\node[circle, scale=0.75, fill] (tid1) at (2.25,3){};
\node[circle, scale=0.75, fill, red] (tid4) at (0.75,4.5){};
\node[circle, scale=0.75, fill, red] (tid5) at (2.25,4.5){};
\node[circle, scale=0.75, fill, red] (tid6) at (3.75,4.5){};
\draw[](tid1) -- (tid4);
\draw[](tid1) -- (tid5);
\draw[](tid1) -- (tid6);
\node[circle, scale=0.75, fill] (tid2) at (5.25,3){};
\node[circle, scale=0.75, fill] (tid3) at (6.75,3){};
\draw[](tid0) -- (tid1);
\draw[](tid0) -- (tid2);
\draw[](tid0) -- (tid3);
\end{tikzpicture}
\nodepart{two}
\footnotesize{4.18519}
\nodepart{three}
\footnotesize{$1$}
};
 & 
\\
};
\end{scope}
\begin{scope}[yshift=\leveltopIIIII cm]
\matrix (line5) [column sep=1cm] {
\node[draw=black, rectangle split,  rectangle split parts=3] (sn0x9c2bc0){
\begin{tikzpicture}[scale=.2]
\node[circle, scale=0.75, fill] (tid0) at (2.25,1.5){};
\node[circle, scale=0.75, fill] (tid1) at (0.75,3){};
\node[circle, scale=0.75, fill, red] (tid4) at (0.75,4.5){};
\draw[](tid1) -- (tid4);
\node[circle, scale=0.75, fill] (tid2) at (2.25,3){};
\node[circle, scale=0.75, fill, red] (tid5) at (2.25,4.5){};
\draw[](tid2) -- (tid5);
\node[circle, scale=0.75, fill, red] (tid3) at (3.75,3){};
\draw[](tid0) -- (tid1);
\draw[](tid0) -- (tid2);
\draw[](tid0) -- (tid3);
\end{tikzpicture}
\nodepart{two}
\footnotesize{3.87963}
\nodepart{three}
\footnotesize{$33\:67$}
};
 & 
\node[draw=black, rectangle split,  rectangle split parts=3] (sn0x9c48e0){
\begin{tikzpicture}[scale=.2]
\node[circle, scale=0.75, fill] (tid0) at (3,1.5){};
\node[circle, scale=0.75, fill] (tid1) at (1.5,3){};
\node[circle, scale=0.75, fill, red] (tid4) at (0.75,4.5){};
\node[circle, scale=0.75, fill, red] (tid5) at (2.25,4.5){};
\draw[](tid1) -- (tid4);
\draw[](tid1) -- (tid5);
\node[circle, scale=0.75, fill, red] (tid2) at (3.75,3){};
\node[circle, scale=0.75, fill] (tid3) at (5.25,3){};
\draw[](tid0) -- (tid1);
\draw[](tid0) -- (tid2);
\draw[](tid0) -- (tid3);
\end{tikzpicture}
\nodepart{two}
\footnotesize{3.85185}
\nodepart{three}
\footnotesize{$67\:33$}
};
 & 
\\
};
\end{scope}
\begin{scope}[yshift=\leveltopIIIIII cm]
\matrix (line6) [column sep=1cm] {
\node[draw=black, rectangle split,  rectangle split parts=3] (sn0x9c4130){
\begin{tikzpicture}[scale=.2]
\node[circle, scale=0.75, fill] (tid0) at (1.5,1.5){};
\node[circle, scale=0.75, fill] (tid1) at (0.75,3){};
\node[circle, scale=0.75, fill, red] (tid3) at (0.75,4.5){};
\draw[](tid1) -- (tid3);
\node[circle, scale=0.75, fill] (tid2) at (2.25,3){};
\node[circle, scale=0.75, fill, red] (tid4) at (2.25,4.5){};
\draw[](tid2) -- (tid4);
\draw[](tid0) -- (tid1);
\draw[](tid0) -- (tid2);
\end{tikzpicture}
\nodepart{two}
\footnotesize{3.75}
\nodepart{three}
\footnotesize{$1$}
};
 & 
\node[draw=black, rectangle split,  rectangle split parts=3] (sn0x9c36c0){
\begin{tikzpicture}[scale=.2]
\node[circle, scale=0.75, fill] (tid0) at (2.25,1.5){};
\node[circle, scale=0.75, fill] (tid1) at (0.75,3){};
\node[circle, scale=0.75, fill, red] (tid4) at (0.75,4.5){};
\draw[](tid1) -- (tid4);
\node[circle, scale=0.75, fill, red] (tid2) at (2.25,3){};
\node[circle, scale=0.75, fill, red] (tid3) at (3.75,3){};
\draw[](tid0) -- (tid1);
\draw[](tid0) -- (tid2);
\draw[](tid0) -- (tid3);
\end{tikzpicture}
\nodepart{two}
\footnotesize{3.44444}
\nodepart{three}
\footnotesize{$67\:33$}
};
 & 
\node[draw=black, rectangle split,  rectangle split parts=3] (sn0x9c5ae0){
\begin{tikzpicture}[scale=.2]
\node[circle, scale=0.75, fill] (tid0) at (2.25,1.5){};
\node[circle, scale=0.75, fill] (tid1) at (1.5,3){};
\node[circle, scale=0.75, fill, red] (tid3) at (0.75,4.5){};
\node[circle, scale=0.75, fill, red] (tid4) at (2.25,4.5){};
\draw[](tid1) -- (tid3);
\draw[](tid1) -- (tid4);
\node[circle, scale=0.75, fill, red] (tid2) at (3.75,3){};
\draw[](tid0) -- (tid1);
\draw[](tid0) -- (tid2);
\end{tikzpicture}
\nodepart{two}
\footnotesize{3.66667}
\nodepart{three}
\footnotesize{$67\:33$}
};
 & 
\\
};
\end{scope}
\begin{scope}[yshift=\leveltopIIIIIII cm]
\matrix (line7) [column sep=1cm] {
\node[draw=black, rectangle split,  rectangle split parts=3] (sn0x9c4350){
\begin{tikzpicture}[scale=.2]
\node[circle, scale=0.75, fill] (tid0) at (1.5,1.5){};
\node[circle, scale=0.75, fill] (tid1) at (0.75,3){};
\node[circle, scale=0.75, fill, red] (tid3) at (0.75,4.5){};
\draw[](tid1) -- (tid3);
\node[circle, scale=0.75, fill, red] (tid2) at (2.25,3){};
\draw[](tid0) -- (tid1);
\draw[](tid0) -- (tid2);
\end{tikzpicture}
\nodepart{two}
\footnotesize{3.25}
\nodepart{three}
\footnotesize{$50\:50$}
};
 & 
\node[draw=black, rectangle split,  rectangle split parts=3] (sn0x9c4af0){
\begin{tikzpicture}[scale=.2]
\node[circle, scale=0.75, fill] (tid0) at (2.25,1.5){};
\node[circle, scale=0.75, fill, red] (tid1) at (0.75,3){};
\node[circle, scale=0.75, fill, red] (tid2) at (2.25,3){};
\node[circle, scale=0.75, fill, red] (tid3) at (3.75,3){};
\draw[](tid0) -- (tid1);
\draw[](tid0) -- (tid2);
\draw[](tid0) -- (tid3);
\end{tikzpicture}
\nodepart{two}
\footnotesize{2.83333}
\nodepart{three}
\footnotesize{$1$}
};
 & 
\node[draw=black, rectangle split,  rectangle split parts=3] (sn0x9c57b0){
\begin{tikzpicture}[scale=.2]
\node[circle, scale=0.75, fill] (tid0) at (1.5,1.5){};
\node[circle, scale=0.75, fill] (tid1) at (1.5,3){};
\node[circle, scale=0.75, fill, red] (tid2) at (0.75,4.5){};
\node[circle, scale=0.75, fill, red] (tid3) at (2.25,4.5){};
\draw[](tid1) -- (tid2);
\draw[](tid1) -- (tid3);
\draw[](tid0) -- (tid1);
\end{tikzpicture}
\nodepart{two}
\footnotesize{3.5}
\nodepart{three}
\footnotesize{$1$}
};
 & 
\\
};
\end{scope}
\begin{scope}[yshift=\leveltopIIIIIIII cm]
\matrix (line8) [column sep=1cm] {
\node[draw=black, rectangle split,  rectangle split parts=3] (sn0x9c4420){
\begin{tikzpicture}[scale=.2]
\node[circle, scale=0.75, fill] (tid0) at (0.75,1.5){};
\node[circle, scale=0.75, fill] (tid1) at (0.75,3){};
\node[circle, scale=0.75, fill, red] (tid2) at (0.75,4.5){};
\draw[](tid1) -- (tid2);
\draw[](tid0) -- (tid1);
\end{tikzpicture}
\nodepart{two}
\footnotesize{3}
\nodepart{three}
\footnotesize{$1$}
};
 & 
\node[draw=black, rectangle split,  rectangle split parts=3] (sn0x9c45d0){
\begin{tikzpicture}[scale=.2]
\node[circle, scale=0.75, fill] (tid0) at (1.5,1.5){};
\node[circle, scale=0.75, fill, red] (tid1) at (0.75,3){};
\node[circle, scale=0.75, fill, red] (tid2) at (2.25,3){};
\draw[](tid0) -- (tid1);
\draw[](tid0) -- (tid2);
\end{tikzpicture}
\nodepart{two}
\footnotesize{2.5}
\nodepart{three}
\footnotesize{$1$}
};
 & 
\\
};
\end{scope}
\begin{scope}[yshift=\leveltopIIIIIIIII cm]
\matrix (line9) [column sep=1cm] {
\node[draw=black, rectangle split,  rectangle split parts=3] (sn0x9c46e0){
\begin{tikzpicture}[scale=.2]
\node[circle, scale=0.75, fill] (tid0) at (0.75,1.5){};
\node[circle, scale=0.75, fill, red] (tid1) at (0.75,3){};
\draw[](tid0) -- (tid1);
\end{tikzpicture}
\nodepart{two}
\footnotesize{2}
\nodepart{three}
\footnotesize{$1$}
};
 & 
\\
};
\end{scope}
\begin{scope}[yshift=\leveltopIIIIIIIIII cm]
\matrix (line10) [column sep=1cm] {
\node[draw=black, rectangle split,  rectangle split parts=3] (sn0x9c47b0){
\begin{tikzpicture}[scale=.2]
\node[circle, scale=0.75, fill, red] (tid0) at (0.75,1.5){};
\end{tikzpicture}
\nodepart{two}
\footnotesize{1}
\nodepart{three}
\footnotesize{$$}
};
 & 
\\
};
\end{scope}
\begin{scope}[yshift=\leveltopIIIIIIIIIII cm]
\matrix (line11) [column sep=1cm] {
\\
};
\end{scope}
\draw (sn0x9c0e30.south) -- (sn0x9c1f80.north);
\draw (sn0x9c0e30.south) -- (sn0x9c2480.north);
\draw (sn0x9c0e30.south) -- (sn0x9c7520.north);
\draw (sn0x9c0e30.south) -- (sn0x9c8f30.north);
\draw (sn0x9c1f80.south) -- (sn0x9c1e40.north);
\draw (sn0x9c1f80.south) -- (sn0x9c2ec0.north);
\draw (sn0x9c2480.south) -- (sn0x9c2ec0.north);
\draw (sn0x9c2480.south) -- (sn0x9c68f0.north);
\draw (sn0x9c7520.south) -- (sn0x9c1e40.north);
\draw (sn0x9c7520.south) -- (sn0x9c2ec0.north);
\draw (sn0x9c7520.south) -- (sn0x9c7a50.north);
\draw (sn0x9c7520.south) -- (sn0x9c7b50.north);
\draw (sn0x9c8f30.south) -- (sn0x9c2ec0.north);
\draw (sn0x9c8f30.south) -- (sn0x9c7b50.north);
\draw (sn0x9c1e40.south) -- (sn0x9c3190.north);
\draw (sn0x9c1e40.south) -- (sn0x9c3bb0.north);
\draw (sn0x9c2ec0.south) -- (sn0x9c3190.north);
\draw (sn0x9c2ec0.south) -- (sn0x9c3bb0.north);
\draw (sn0x9c68f0.south) -- (sn0x9c3bb0.north);
\draw (sn0x9c7a50.south) -- (sn0x9c3bb0.north);
\draw (sn0x9c7b50.south) -- (sn0x9c3bb0.north);
\draw (sn0x9c7b50.south) -- (sn0x9c8420.north);
\draw (sn0x9c3190.south) -- (sn0x9c2bc0.north);
\draw (sn0x9c3bb0.south) -- (sn0x9c2bc0.north);
\draw (sn0x9c3bb0.south) -- (sn0x9c48e0.north);
\draw (sn0x9c8420.south) -- (sn0x9c48e0.north);
\draw (sn0x9c2bc0.south) -- (sn0x9c4130.north);
\draw (sn0x9c2bc0.south) -- (sn0x9c36c0.north);
\draw (sn0x9c48e0.south) -- (sn0x9c5ae0.north);
\draw (sn0x9c48e0.south) -- (sn0x9c36c0.north);
\draw (sn0x9c4130.south) -- (sn0x9c4350.north);
\draw (sn0x9c36c0.south) -- (sn0x9c4350.north);
\draw (sn0x9c36c0.south) -- (sn0x9c4af0.north);
\draw (sn0x9c5ae0.south) -- (sn0x9c57b0.north);
\draw (sn0x9c5ae0.south) -- (sn0x9c4350.north);
\draw (sn0x9c4350.south) -- (sn0x9c4420.north);
\draw (sn0x9c4350.south) -- (sn0x9c45d0.north);
\draw (sn0x9c4af0.south) -- (sn0x9c45d0.north);
\draw (sn0x9c57b0.south) -- (sn0x9c4420.north);
\draw (sn0x9c4420.south) -- (sn0x9c46e0.north);
\draw (sn0x9c45d0.south) -- (sn0x9c46e0.north);
\draw (sn0x9c46e0.south) -- (sn0x9c47b0.north);
\end{tikzpicture}

%%% Local Variables:
%%% TeX-master: "thesis/thesis.tex"
%%% End: 
\renewcommand{\leveltopI}{-15cm + \leveltop}
\renewcommand{\leveltopII}{-15cm + \leveltopI}
\renewcommand{\leveltopIII}{-15cm + \leveltopII}
\renewcommand{\leveltopIIII}{-15cm + \leveltopIII}
\renewcommand{\leveltopIIIII}{-15cm + \leveltopIIII}
\renewcommand{\leveltopIIIIII}{-15cm + \leveltopIIIII}
\renewcommand{\leveltopIIIIIII}{-15cm + \leveltopIIIIII}
\renewcommand{\leveltopIIIIIIII}{-15cm + \leveltopIIIIIII}
\renewcommand{\leveltopIIIIIIIII}{-15cm + \leveltopIIIIIIII}
\renewcommand{\leveltopIIIIIIIIII}{-15cm + \leveltopIIIIIIIII}
\begin{tikzpicture}[scale=.2, anchor=south]
\begin{scope}[yshift=\leveltopI cm]
\matrix (line1) [column sep=1cm] {
\node[draw=black, rectangle split,  rectangle split parts=3] (sn0x9c1170){
\begin{tikzpicture}[scale=.2]
\node[circle, scale=0.75, fill] (tid0) at (4.5,1.5){};
\node[circle, scale=0.75, fill] (tid1) at (2.25,3){};
\node[circle, scale=0.75, fill, red] (tid4) at (0.75,4.5){};
\node[circle, scale=0.75, fill, red] (tid5) at (2.25,4.5){};
\node[circle, scale=0.75, fill] (tid6) at (3.75,4.5){};
\draw[](tid1) -- (tid4);
\draw[](tid1) -- (tid5);
\draw[](tid1) -- (tid6);
\node[circle, scale=0.75, fill] (tid2) at (6,3){};
\node[circle, scale=0.75, fill] (tid7) at (5.25,4.5){};
\node[circle, scale=0.75, fill] (tid8) at (6.75,4.5){};
\draw[](tid2) -- (tid7);
\draw[](tid2) -- (tid8);
\node[circle, scale=0.75, fill] (tid3) at (8.25,3){};
\node[circle, scale=0.75, fill, red] (tid9) at (8.25,4.5){};
\draw[](tid3) -- (tid9);
\draw[](tid0) -- (tid1);
\draw[](tid0) -- (tid2);
\draw[](tid0) -- (tid3);
\end{tikzpicture}
\nodepart{two}
\footnotesize{5.20531}
\nodepart{three}
\footnotesize{$22\:44\:11\:22$}
};
 & 
\\
};
\end{scope}
\begin{scope}[yshift=\leveltopII cm]
\matrix (line2) [column sep=1cm] {
\node[draw=black, rectangle split,  rectangle split parts=3] (sn0x9c2480){
\begin{tikzpicture}[scale=.2]
\node[circle, scale=0.75, fill] (tid0) at (3.75,1.5){};
\node[circle, scale=0.75, fill] (tid1) at (1.5,3){};
\node[circle, scale=0.75, fill, red] (tid4) at (0.75,4.5){};
\node[circle, scale=0.75, fill, red] (tid5) at (2.25,4.5){};
\draw[](tid1) -- (tid4);
\draw[](tid1) -- (tid5);
\node[circle, scale=0.75, fill] (tid2) at (4.5,3){};
\node[circle, scale=0.75, fill] (tid6) at (3.75,4.5){};
\node[circle, scale=0.75, fill] (tid7) at (5.25,4.5){};
\draw[](tid2) -- (tid6);
\draw[](tid2) -- (tid7);
\node[circle, scale=0.75, fill] (tid3) at (6.75,3){};
\node[circle, scale=0.75, fill, red] (tid8) at (6.75,4.5){};
\draw[](tid3) -- (tid8);
\draw[](tid0) -- (tid1);
\draw[](tid0) -- (tid2);
\draw[](tid0) -- (tid3);
\end{tikzpicture}
\nodepart{two}
\footnotesize{4.87243}
\nodepart{three}
\footnotesize{$67\:33$}
};
 & 
\node[draw=black, rectangle split,  rectangle split parts=3] (sn0x9c6cb0){
\begin{tikzpicture}[scale=.2]
\node[circle, scale=0.75, fill] (tid0) at (3.75,1.5){};
\node[circle, scale=0.75, fill] (tid1) at (1.5,3){};
\node[circle, scale=0.75, fill, red] (tid4) at (0.75,4.5){};
\node[circle, scale=0.75, fill] (tid5) at (2.25,4.5){};
\draw[](tid1) -- (tid4);
\draw[](tid1) -- (tid5);
\node[circle, scale=0.75, fill] (tid2) at (4.5,3){};
\node[circle, scale=0.75, fill, red] (tid6) at (3.75,4.5){};
\node[circle, scale=0.75, fill] (tid7) at (5.25,4.5){};
\draw[](tid2) -- (tid6);
\draw[](tid2) -- (tid7);
\node[circle, scale=0.75, fill] (tid3) at (6.75,3){};
\node[circle, scale=0.75, fill, red] (tid8) at (6.75,4.5){};
\draw[](tid3) -- (tid8);
\draw[](tid0) -- (tid1);
\draw[](tid0) -- (tid2);
\draw[](tid0) -- (tid3);
\end{tikzpicture}
\nodepart{two}
\footnotesize{4.87346}
\nodepart{three}
\footnotesize{$33\:33\:33$}
};
 & 
\node[draw=black, rectangle split,  rectangle split parts=3] (sn0x9c91d0){
\begin{tikzpicture}[scale=.2]
\node[circle, scale=0.75, fill] (tid0) at (4.5,1.5){};
\node[circle, scale=0.75, fill] (tid1) at (2.25,3){};
\node[circle, scale=0.75, fill, red] (tid4) at (0.75,4.5){};
\node[circle, scale=0.75, fill, red] (tid5) at (2.25,4.5){};
\node[circle, scale=0.75, fill, red] (tid6) at (3.75,4.5){};
\draw[](tid1) -- (tid4);
\draw[](tid1) -- (tid5);
\draw[](tid1) -- (tid6);
\node[circle, scale=0.75, fill] (tid2) at (6,3){};
\node[circle, scale=0.75, fill] (tid7) at (5.25,4.5){};
\node[circle, scale=0.75, fill] (tid8) at (6.75,4.5){};
\draw[](tid2) -- (tid7);
\draw[](tid2) -- (tid8);
\node[circle, scale=0.75, fill] (tid3) at (8.25,3){};
\draw[](tid0) -- (tid1);
\draw[](tid0) -- (tid2);
\draw[](tid0) -- (tid3);
\end{tikzpicture}
\nodepart{two}
\footnotesize{4.87037}
\nodepart{three}
\footnotesize{$1$}
};
 & 
\node[draw=black, rectangle split,  rectangle split parts=3] (sn0x9c9450){
\begin{tikzpicture}[scale=.2]
\node[circle, scale=0.75, fill] (tid0) at (4.5,1.5){};
\node[circle, scale=0.75, fill] (tid1) at (2.25,3){};
\node[circle, scale=0.75, fill, red] (tid4) at (0.75,4.5){};
\node[circle, scale=0.75, fill, red] (tid5) at (2.25,4.5){};
\node[circle, scale=0.75, fill] (tid6) at (3.75,4.5){};
\draw[](tid1) -- (tid4);
\draw[](tid1) -- (tid5);
\draw[](tid1) -- (tid6);
\node[circle, scale=0.75, fill] (tid2) at (6,3){};
\node[circle, scale=0.75, fill, red] (tid7) at (5.25,4.5){};
\node[circle, scale=0.75, fill] (tid8) at (6.75,4.5){};
\draw[](tid2) -- (tid7);
\draw[](tid2) -- (tid8);
\node[circle, scale=0.75, fill] (tid3) at (8.25,3){};
\draw[](tid0) -- (tid1);
\draw[](tid0) -- (tid2);
\draw[](tid0) -- (tid3);
\end{tikzpicture}
\nodepart{two}
\footnotesize{4.86934}
\nodepart{three}
\footnotesize{$67\:17\:17$}
};
 & 
\\
};
\end{scope}
\begin{scope}[yshift=\leveltopIII cm]
\matrix (line3) [column sep=1cm] {
\node[draw=black, rectangle split,  rectangle split parts=3] (sn0x9c2ec0){
\begin{tikzpicture}[scale=.2]
\node[circle, scale=0.75, fill] (tid0) at (3,1.5){};
\node[circle, scale=0.75, fill] (tid1) at (1.5,3){};
\node[circle, scale=0.75, fill, red] (tid4) at (0.75,4.5){};
\node[circle, scale=0.75, fill] (tid5) at (2.25,4.5){};
\draw[](tid1) -- (tid4);
\draw[](tid1) -- (tid5);
\node[circle, scale=0.75, fill] (tid2) at (3.75,3){};
\node[circle, scale=0.75, fill, red] (tid6) at (3.75,4.5){};
\draw[](tid2) -- (tid6);
\node[circle, scale=0.75, fill] (tid3) at (5.25,3){};
\node[circle, scale=0.75, fill, red] (tid7) at (5.25,4.5){};
\draw[](tid3) -- (tid7);
\draw[](tid0) -- (tid1);
\draw[](tid0) -- (tid2);
\draw[](tid0) -- (tid3);
\end{tikzpicture}
\nodepart{two}
\footnotesize{4.54012}
\nodepart{three}
\footnotesize{$33\:67$}
};
 & 
\node[draw=black, rectangle split,  rectangle split parts=3] (sn0x9c68f0){
\begin{tikzpicture}[scale=.2]
\node[circle, scale=0.75, fill] (tid0) at (3.75,1.5){};
\node[circle, scale=0.75, fill] (tid1) at (1.5,3){};
\node[circle, scale=0.75, fill, red] (tid4) at (0.75,4.5){};
\node[circle, scale=0.75, fill, red] (tid5) at (2.25,4.5){};
\draw[](tid1) -- (tid4);
\draw[](tid1) -- (tid5);
\node[circle, scale=0.75, fill] (tid2) at (4.5,3){};
\node[circle, scale=0.75, fill, red] (tid6) at (3.75,4.5){};
\node[circle, scale=0.75, fill] (tid7) at (5.25,4.5){};
\draw[](tid2) -- (tid6);
\draw[](tid2) -- (tid7);
\node[circle, scale=0.75, fill] (tid3) at (6.75,3){};
\draw[](tid0) -- (tid1);
\draw[](tid0) -- (tid2);
\draw[](tid0) -- (tid3);
\end{tikzpicture}
\nodepart{two}
\footnotesize{4.53704}
\nodepart{three}
\footnotesize{$1$}
};
 & 
\node[draw=black, rectangle split,  rectangle split parts=3] (sn0x9c1e40){
\begin{tikzpicture}[scale=.2]
\node[circle, scale=0.75, fill] (tid0) at (3,1.5){};
\node[circle, scale=0.75, fill] (tid1) at (1.5,3){};
\node[circle, scale=0.75, fill, red] (tid4) at (0.75,4.5){};
\node[circle, scale=0.75, fill, red] (tid5) at (2.25,4.5){};
\draw[](tid1) -- (tid4);
\draw[](tid1) -- (tid5);
\node[circle, scale=0.75, fill] (tid2) at (3.75,3){};
\node[circle, scale=0.75, fill, red] (tid6) at (3.75,4.5){};
\draw[](tid2) -- (tid6);
\node[circle, scale=0.75, fill] (tid3) at (5.25,3){};
\node[circle, scale=0.75, fill] (tid7) at (5.25,4.5){};
\draw[](tid3) -- (tid7);
\draw[](tid0) -- (tid1);
\draw[](tid0) -- (tid2);
\draw[](tid0) -- (tid3);
\end{tikzpicture}
\nodepart{two}
\footnotesize{4.54321}
\nodepart{three}
\footnotesize{$67\:33$}
};
 & 
\node[draw=black, rectangle split,  rectangle split parts=3] (sn0x9c7a50){
\begin{tikzpicture}[scale=.2]
\node[circle, scale=0.75, fill] (tid0) at (3.75,1.5){};
\node[circle, scale=0.75, fill] (tid1) at (2.25,3){};
\node[circle, scale=0.75, fill, red] (tid4) at (0.75,4.5){};
\node[circle, scale=0.75, fill, red] (tid5) at (2.25,4.5){};
\node[circle, scale=0.75, fill, red] (tid6) at (3.75,4.5){};
\draw[](tid1) -- (tid4);
\draw[](tid1) -- (tid5);
\draw[](tid1) -- (tid6);
\node[circle, scale=0.75, fill] (tid2) at (5.25,3){};
\node[circle, scale=0.75, fill] (tid7) at (5.25,4.5){};
\draw[](tid2) -- (tid7);
\node[circle, scale=0.75, fill] (tid3) at (6.75,3){};
\draw[](tid0) -- (tid1);
\draw[](tid0) -- (tid2);
\draw[](tid0) -- (tid3);
\end{tikzpicture}
\nodepart{two}
\footnotesize{4.53704}
\nodepart{three}
\footnotesize{$1$}
};
 & 
\node[draw=black, rectangle split,  rectangle split parts=3] (sn0x9c7b50){
\begin{tikzpicture}[scale=.2]
\node[circle, scale=0.75, fill] (tid0) at (3.75,1.5){};
\node[circle, scale=0.75, fill] (tid1) at (2.25,3){};
\node[circle, scale=0.75, fill, red] (tid4) at (0.75,4.5){};
\node[circle, scale=0.75, fill, red] (tid5) at (2.25,4.5){};
\node[circle, scale=0.75, fill] (tid6) at (3.75,4.5){};
\draw[](tid1) -- (tid4);
\draw[](tid1) -- (tid5);
\draw[](tid1) -- (tid6);
\node[circle, scale=0.75, fill] (tid2) at (5.25,3){};
\node[circle, scale=0.75, fill, red] (tid7) at (5.25,4.5){};
\draw[](tid2) -- (tid7);
\node[circle, scale=0.75, fill] (tid3) at (6.75,3){};
\draw[](tid0) -- (tid1);
\draw[](tid0) -- (tid2);
\draw[](tid0) -- (tid3);
\end{tikzpicture}
\nodepart{two}
\footnotesize{4.53086}
\nodepart{three}
\footnotesize{$67\:33$}
};
 & 
\\
};
\end{scope}
\begin{scope}[yshift=\leveltopIIII cm]
\matrix (line4) [column sep=1cm] {
\node[draw=black, rectangle split,  rectangle split parts=3] (sn0x9c3190){
\begin{tikzpicture}[scale=.2]
\node[circle, scale=0.75, fill] (tid0) at (2.25,1.5){};
\node[circle, scale=0.75, fill] (tid1) at (0.75,3){};
\node[circle, scale=0.75, fill, red] (tid4) at (0.75,4.5){};
\draw[](tid1) -- (tid4);
\node[circle, scale=0.75, fill] (tid2) at (2.25,3){};
\node[circle, scale=0.75, fill, red] (tid5) at (2.25,4.5){};
\draw[](tid2) -- (tid5);
\node[circle, scale=0.75, fill] (tid3) at (3.75,3){};
\node[circle, scale=0.75, fill, red] (tid6) at (3.75,4.5){};
\draw[](tid3) -- (tid6);
\draw[](tid0) -- (tid1);
\draw[](tid0) -- (tid2);
\draw[](tid0) -- (tid3);
\end{tikzpicture}
\nodepart{two}
\footnotesize{4.21296}
\nodepart{three}
\footnotesize{$1$}
};
 & 
\node[draw=black, rectangle split,  rectangle split parts=3] (sn0x9c3bb0){
\begin{tikzpicture}[scale=.2]
\node[circle, scale=0.75, fill] (tid0) at (3,1.5){};
\node[circle, scale=0.75, fill] (tid1) at (1.5,3){};
\node[circle, scale=0.75, fill, red] (tid4) at (0.75,4.5){};
\node[circle, scale=0.75, fill, red] (tid5) at (2.25,4.5){};
\draw[](tid1) -- (tid4);
\draw[](tid1) -- (tid5);
\node[circle, scale=0.75, fill] (tid2) at (3.75,3){};
\node[circle, scale=0.75, fill, red] (tid6) at (3.75,4.5){};
\draw[](tid2) -- (tid6);
\node[circle, scale=0.75, fill] (tid3) at (5.25,3){};
\draw[](tid0) -- (tid1);
\draw[](tid0) -- (tid2);
\draw[](tid0) -- (tid3);
\end{tikzpicture}
\nodepart{two}
\footnotesize{4.2037}
\nodepart{three}
\footnotesize{$67\:33$}
};
 & 
\node[draw=black, rectangle split,  rectangle split parts=3] (sn0x9c8420){
\begin{tikzpicture}[scale=.2]
\node[circle, scale=0.75, fill] (tid0) at (3.75,1.5){};
\node[circle, scale=0.75, fill] (tid1) at (2.25,3){};
\node[circle, scale=0.75, fill, red] (tid4) at (0.75,4.5){};
\node[circle, scale=0.75, fill, red] (tid5) at (2.25,4.5){};
\node[circle, scale=0.75, fill, red] (tid6) at (3.75,4.5){};
\draw[](tid1) -- (tid4);
\draw[](tid1) -- (tid5);
\draw[](tid1) -- (tid6);
\node[circle, scale=0.75, fill] (tid2) at (5.25,3){};
\node[circle, scale=0.75, fill] (tid3) at (6.75,3){};
\draw[](tid0) -- (tid1);
\draw[](tid0) -- (tid2);
\draw[](tid0) -- (tid3);
\end{tikzpicture}
\nodepart{two}
\footnotesize{4.18519}
\nodepart{three}
\footnotesize{$1$}
};
 & 
\\
};
\end{scope}
\begin{scope}[yshift=\leveltopIIIII cm]
\matrix (line5) [column sep=1cm] {
\node[draw=black, rectangle split,  rectangle split parts=3] (sn0x9c2bc0){
\begin{tikzpicture}[scale=.2]
\node[circle, scale=0.75, fill] (tid0) at (2.25,1.5){};
\node[circle, scale=0.75, fill] (tid1) at (0.75,3){};
\node[circle, scale=0.75, fill, red] (tid4) at (0.75,4.5){};
\draw[](tid1) -- (tid4);
\node[circle, scale=0.75, fill] (tid2) at (2.25,3){};
\node[circle, scale=0.75, fill, red] (tid5) at (2.25,4.5){};
\draw[](tid2) -- (tid5);
\node[circle, scale=0.75, fill, red] (tid3) at (3.75,3){};
\draw[](tid0) -- (tid1);
\draw[](tid0) -- (tid2);
\draw[](tid0) -- (tid3);
\end{tikzpicture}
\nodepart{two}
\footnotesize{3.87963}
\nodepart{three}
\footnotesize{$33\:67$}
};
 & 
\node[draw=black, rectangle split,  rectangle split parts=3] (sn0x9c48e0){
\begin{tikzpicture}[scale=.2]
\node[circle, scale=0.75, fill] (tid0) at (3,1.5){};
\node[circle, scale=0.75, fill] (tid1) at (1.5,3){};
\node[circle, scale=0.75, fill, red] (tid4) at (0.75,4.5){};
\node[circle, scale=0.75, fill, red] (tid5) at (2.25,4.5){};
\draw[](tid1) -- (tid4);
\draw[](tid1) -- (tid5);
\node[circle, scale=0.75, fill, red] (tid2) at (3.75,3){};
\node[circle, scale=0.75, fill] (tid3) at (5.25,3){};
\draw[](tid0) -- (tid1);
\draw[](tid0) -- (tid2);
\draw[](tid0) -- (tid3);
\end{tikzpicture}
\nodepart{two}
\footnotesize{3.85185}
\nodepart{three}
\footnotesize{$67\:33$}
};
 & 
\\
};
\end{scope}
\begin{scope}[yshift=\leveltopIIIIII cm]
\matrix (line6) [column sep=1cm] {
\node[draw=black, rectangle split,  rectangle split parts=3] (sn0x9c4130){
\begin{tikzpicture}[scale=.2]
\node[circle, scale=0.75, fill] (tid0) at (1.5,1.5){};
\node[circle, scale=0.75, fill] (tid1) at (0.75,3){};
\node[circle, scale=0.75, fill, red] (tid3) at (0.75,4.5){};
\draw[](tid1) -- (tid3);
\node[circle, scale=0.75, fill] (tid2) at (2.25,3){};
\node[circle, scale=0.75, fill, red] (tid4) at (2.25,4.5){};
\draw[](tid2) -- (tid4);
\draw[](tid0) -- (tid1);
\draw[](tid0) -- (tid2);
\end{tikzpicture}
\nodepart{two}
\footnotesize{3.75}
\nodepart{three}
\footnotesize{$1$}
};
 & 
\node[draw=black, rectangle split,  rectangle split parts=3] (sn0x9c36c0){
\begin{tikzpicture}[scale=.2]
\node[circle, scale=0.75, fill] (tid0) at (2.25,1.5){};
\node[circle, scale=0.75, fill] (tid1) at (0.75,3){};
\node[circle, scale=0.75, fill, red] (tid4) at (0.75,4.5){};
\draw[](tid1) -- (tid4);
\node[circle, scale=0.75, fill, red] (tid2) at (2.25,3){};
\node[circle, scale=0.75, fill, red] (tid3) at (3.75,3){};
\draw[](tid0) -- (tid1);
\draw[](tid0) -- (tid2);
\draw[](tid0) -- (tid3);
\end{tikzpicture}
\nodepart{two}
\footnotesize{3.44444}
\nodepart{three}
\footnotesize{$67\:33$}
};
 & 
\node[draw=black, rectangle split,  rectangle split parts=3] (sn0x9c5ae0){
\begin{tikzpicture}[scale=.2]
\node[circle, scale=0.75, fill] (tid0) at (2.25,1.5){};
\node[circle, scale=0.75, fill] (tid1) at (1.5,3){};
\node[circle, scale=0.75, fill, red] (tid3) at (0.75,4.5){};
\node[circle, scale=0.75, fill, red] (tid4) at (2.25,4.5){};
\draw[](tid1) -- (tid3);
\draw[](tid1) -- (tid4);
\node[circle, scale=0.75, fill, red] (tid2) at (3.75,3){};
\draw[](tid0) -- (tid1);
\draw[](tid0) -- (tid2);
\end{tikzpicture}
\nodepart{two}
\footnotesize{3.66667}
\nodepart{three}
\footnotesize{$67\:33$}
};
 & 
\\
};
\end{scope}
\begin{scope}[yshift=\leveltopIIIIIII cm]
\matrix (line7) [column sep=1cm] {
\node[draw=black, rectangle split,  rectangle split parts=3] (sn0x9c4350){
\begin{tikzpicture}[scale=.2]
\node[circle, scale=0.75, fill] (tid0) at (1.5,1.5){};
\node[circle, scale=0.75, fill] (tid1) at (0.75,3){};
\node[circle, scale=0.75, fill, red] (tid3) at (0.75,4.5){};
\draw[](tid1) -- (tid3);
\node[circle, scale=0.75, fill, red] (tid2) at (2.25,3){};
\draw[](tid0) -- (tid1);
\draw[](tid0) -- (tid2);
\end{tikzpicture}
\nodepart{two}
\footnotesize{3.25}
\nodepart{three}
\footnotesize{$50\:50$}
};
 & 
\node[draw=black, rectangle split,  rectangle split parts=3] (sn0x9c4af0){
\begin{tikzpicture}[scale=.2]
\node[circle, scale=0.75, fill] (tid0) at (2.25,1.5){};
\node[circle, scale=0.75, fill, red] (tid1) at (0.75,3){};
\node[circle, scale=0.75, fill, red] (tid2) at (2.25,3){};
\node[circle, scale=0.75, fill, red] (tid3) at (3.75,3){};
\draw[](tid0) -- (tid1);
\draw[](tid0) -- (tid2);
\draw[](tid0) -- (tid3);
\end{tikzpicture}
\nodepart{two}
\footnotesize{2.83333}
\nodepart{three}
\footnotesize{$1$}
};
 & 
\node[draw=black, rectangle split,  rectangle split parts=3] (sn0x9c57b0){
\begin{tikzpicture}[scale=.2]
\node[circle, scale=0.75, fill] (tid0) at (1.5,1.5){};
\node[circle, scale=0.75, fill] (tid1) at (1.5,3){};
\node[circle, scale=0.75, fill, red] (tid2) at (0.75,4.5){};
\node[circle, scale=0.75, fill, red] (tid3) at (2.25,4.5){};
\draw[](tid1) -- (tid2);
\draw[](tid1) -- (tid3);
\draw[](tid0) -- (tid1);
\end{tikzpicture}
\nodepart{two}
\footnotesize{3.5}
\nodepart{three}
\footnotesize{$1$}
};
 & 
\\
};
\end{scope}
\begin{scope}[yshift=\leveltopIIIIIIII cm]
\matrix (line8) [column sep=1cm] {
\node[draw=black, rectangle split,  rectangle split parts=3] (sn0x9c4420){
\begin{tikzpicture}[scale=.2]
\node[circle, scale=0.75, fill] (tid0) at (0.75,1.5){};
\node[circle, scale=0.75, fill] (tid1) at (0.75,3){};
\node[circle, scale=0.75, fill, red] (tid2) at (0.75,4.5){};
\draw[](tid1) -- (tid2);
\draw[](tid0) -- (tid1);
\end{tikzpicture}
\nodepart{two}
\footnotesize{3}
\nodepart{three}
\footnotesize{$1$}
};
 & 
\node[draw=black, rectangle split,  rectangle split parts=3] (sn0x9c45d0){
\begin{tikzpicture}[scale=.2]
\node[circle, scale=0.75, fill] (tid0) at (1.5,1.5){};
\node[circle, scale=0.75, fill, red] (tid1) at (0.75,3){};
\node[circle, scale=0.75, fill, red] (tid2) at (2.25,3){};
\draw[](tid0) -- (tid1);
\draw[](tid0) -- (tid2);
\end{tikzpicture}
\nodepart{two}
\footnotesize{2.5}
\nodepart{three}
\footnotesize{$1$}
};
 & 
\\
};
\end{scope}
\begin{scope}[yshift=\leveltopIIIIIIIII cm]
\matrix (line9) [column sep=1cm] {
\node[draw=black, rectangle split,  rectangle split parts=3] (sn0x9c46e0){
\begin{tikzpicture}[scale=.2]
\node[circle, scale=0.75, fill] (tid0) at (0.75,1.5){};
\node[circle, scale=0.75, fill, red] (tid1) at (0.75,3){};
\draw[](tid0) -- (tid1);
\end{tikzpicture}
\nodepart{two}
\footnotesize{2}
\nodepart{three}
\footnotesize{$1$}
};
 & 
\\
};
\end{scope}
\begin{scope}[yshift=\leveltopIIIIIIIIII cm]
\matrix (line10) [column sep=1cm] {
\node[draw=black, rectangle split,  rectangle split parts=3] (sn0x9c47b0){
\begin{tikzpicture}[scale=.2]
\node[circle, scale=0.75, fill, red] (tid0) at (0.75,1.5){};
\end{tikzpicture}
\nodepart{two}
\footnotesize{1}
\nodepart{three}
\footnotesize{$$}
};
 & 
\\
};
\end{scope}
\begin{scope}[yshift=\leveltopIIIIIIIIIII cm]
\matrix (line11) [column sep=1cm] {
\\
};
\end{scope}
\draw (sn0x9c1170.south) -- (sn0x9c2480.north);
\draw (sn0x9c1170.south) -- (sn0x9c6cb0.north);
\draw (sn0x9c1170.south) -- (sn0x9c91d0.north);
\draw (sn0x9c1170.south) -- (sn0x9c9450.north);
\draw (sn0x9c2480.south) -- (sn0x9c2ec0.north);
\draw (sn0x9c2480.south) -- (sn0x9c68f0.north);
\draw (sn0x9c6cb0.south) -- (sn0x9c2ec0.north);
\draw (sn0x9c6cb0.south) -- (sn0x9c1e40.north);
\draw (sn0x9c6cb0.south) -- (sn0x9c68f0.north);
\draw (sn0x9c91d0.south) -- (sn0x9c68f0.north);
\draw (sn0x9c9450.south) -- (sn0x9c68f0.north);
\draw (sn0x9c9450.south) -- (sn0x9c7a50.north);
\draw (sn0x9c9450.south) -- (sn0x9c7b50.north);
\draw (sn0x9c2ec0.south) -- (sn0x9c3190.north);
\draw (sn0x9c2ec0.south) -- (sn0x9c3bb0.north);
\draw (sn0x9c68f0.south) -- (sn0x9c3bb0.north);
\draw (sn0x9c1e40.south) -- (sn0x9c3190.north);
\draw (sn0x9c1e40.south) -- (sn0x9c3bb0.north);
\draw (sn0x9c7a50.south) -- (sn0x9c3bb0.north);
\draw (sn0x9c7b50.south) -- (sn0x9c3bb0.north);
\draw (sn0x9c7b50.south) -- (sn0x9c8420.north);
\draw (sn0x9c3190.south) -- (sn0x9c2bc0.north);
\draw (sn0x9c3bb0.south) -- (sn0x9c2bc0.north);
\draw (sn0x9c3bb0.south) -- (sn0x9c48e0.north);
\draw (sn0x9c8420.south) -- (sn0x9c48e0.north);
\draw (sn0x9c2bc0.south) -- (sn0x9c4130.north);
\draw (sn0x9c2bc0.south) -- (sn0x9c36c0.north);
\draw (sn0x9c48e0.south) -- (sn0x9c5ae0.north);
\draw (sn0x9c48e0.south) -- (sn0x9c36c0.north);
\draw (sn0x9c4130.south) -- (sn0x9c4350.north);
\draw (sn0x9c36c0.south) -- (sn0x9c4350.north);
\draw (sn0x9c36c0.south) -- (sn0x9c4af0.north);
\draw (sn0x9c5ae0.south) -- (sn0x9c57b0.north);
\draw (sn0x9c5ae0.south) -- (sn0x9c4350.north);
\draw (sn0x9c4350.south) -- (sn0x9c4420.north);
\draw (sn0x9c4350.south) -- (sn0x9c45d0.north);
\draw (sn0x9c4af0.south) -- (sn0x9c45d0.north);
\draw (sn0x9c57b0.south) -- (sn0x9c4420.north);
\draw (sn0x9c4420.south) -- (sn0x9c46e0.north);
\draw (sn0x9c45d0.south) -- (sn0x9c46e0.north);
\draw (sn0x9c46e0.south) -- (sn0x9c47b0.north);
\end{tikzpicture}

%%% Local Variables:
%%% TeX-master: "thesis/thesis.tex"
%%% End: 
\renewcommand{\leveltopI}{-15cm + \leveltop}
\renewcommand{\leveltopII}{-15cm + \leveltopI}
\renewcommand{\leveltopIII}{-15cm + \leveltopII}
\renewcommand{\leveltopIIII}{-15cm + \leveltopIII}
\renewcommand{\leveltopIIIII}{-15cm + \leveltopIIII}
\renewcommand{\leveltopIIIIII}{-15cm + \leveltopIIIII}
\renewcommand{\leveltopIIIIIII}{-15cm + \leveltopIIIIII}
\renewcommand{\leveltopIIIIIIII}{-15cm + \leveltopIIIIIII}
\renewcommand{\leveltopIIIIIIIII}{-15cm + \leveltopIIIIIIII}
\renewcommand{\leveltopIIIIIIIIII}{-15cm + \leveltopIIIIIIIII}
\begin{tikzpicture}[scale=.2, anchor=south]
\begin{scope}[yshift=\leveltopI cm]
\matrix (line1) [column sep=1cm] {
\node[draw=black, rectangle split,  rectangle split parts=3] (sn0x9c14f0){
\begin{tikzpicture}[scale=.2]
\node[circle, scale=0.75, fill] (tid0) at (4.5,1.5){};
\node[circle, scale=0.75, fill] (tid1) at (2.25,3){};
\node[circle, scale=0.75, fill, red] (tid4) at (0.75,4.5){};
\node[circle, scale=0.75, fill] (tid5) at (2.25,4.5){};
\node[circle, scale=0.75, fill] (tid6) at (3.75,4.5){};
\draw[](tid1) -- (tid4);
\draw[](tid1) -- (tid5);
\draw[](tid1) -- (tid6);
\node[circle, scale=0.75, fill] (tid2) at (6,3){};
\node[circle, scale=0.75, fill, red] (tid7) at (5.25,4.5){};
\node[circle, scale=0.75, fill] (tid8) at (6.75,4.5){};
\draw[](tid2) -- (tid7);
\draw[](tid2) -- (tid8);
\node[circle, scale=0.75, fill] (tid3) at (8.25,3){};
\node[circle, scale=0.75, fill, red] (tid9) at (8.25,4.5){};
\draw[](tid3) -- (tid9);
\draw[](tid0) -- (tid1);
\draw[](tid0) -- (tid2);
\draw[](tid0) -- (tid3);
\end{tikzpicture}
\nodepart{two}
\footnotesize{5.20405}
\nodepart{three}
\footnotesize{$22\:11\:22\:11\:22\:11$}
};
 & 
\\
};
\end{scope}
\begin{scope}[yshift=\leveltopII cm]
\matrix (line2) [column sep=1cm] {
\node[draw=black, rectangle split,  rectangle split parts=3] (sn0x9c6cb0){
\begin{tikzpicture}[scale=.2]
\node[circle, scale=0.75, fill] (tid0) at (3.75,1.5){};
\node[circle, scale=0.75, fill] (tid1) at (1.5,3){};
\node[circle, scale=0.75, fill, red] (tid4) at (0.75,4.5){};
\node[circle, scale=0.75, fill] (tid5) at (2.25,4.5){};
\draw[](tid1) -- (tid4);
\draw[](tid1) -- (tid5);
\node[circle, scale=0.75, fill] (tid2) at (4.5,3){};
\node[circle, scale=0.75, fill, red] (tid6) at (3.75,4.5){};
\node[circle, scale=0.75, fill] (tid7) at (5.25,4.5){};
\draw[](tid2) -- (tid6);
\draw[](tid2) -- (tid7);
\node[circle, scale=0.75, fill] (tid3) at (6.75,3){};
\node[circle, scale=0.75, fill, red] (tid8) at (6.75,4.5){};
\draw[](tid3) -- (tid8);
\draw[](tid0) -- (tid1);
\draw[](tid0) -- (tid2);
\draw[](tid0) -- (tid3);
\end{tikzpicture}
\nodepart{two}
\footnotesize{4.87346}
\nodepart{three}
\footnotesize{$33\:33\:33$}
};
 & 
\node[draw=black, rectangle split,  rectangle split parts=3] (sn0x9c2480){
\begin{tikzpicture}[scale=.2]
\node[circle, scale=0.75, fill] (tid0) at (3.75,1.5){};
\node[circle, scale=0.75, fill] (tid1) at (1.5,3){};
\node[circle, scale=0.75, fill, red] (tid4) at (0.75,4.5){};
\node[circle, scale=0.75, fill, red] (tid5) at (2.25,4.5){};
\draw[](tid1) -- (tid4);
\draw[](tid1) -- (tid5);
\node[circle, scale=0.75, fill] (tid2) at (4.5,3){};
\node[circle, scale=0.75, fill] (tid6) at (3.75,4.5){};
\node[circle, scale=0.75, fill] (tid7) at (5.25,4.5){};
\draw[](tid2) -- (tid6);
\draw[](tid2) -- (tid7);
\node[circle, scale=0.75, fill] (tid3) at (6.75,3){};
\node[circle, scale=0.75, fill, red] (tid8) at (6.75,4.5){};
\draw[](tid3) -- (tid8);
\draw[](tid0) -- (tid1);
\draw[](tid0) -- (tid2);
\draw[](tid0) -- (tid3);
\end{tikzpicture}
\nodepart{two}
\footnotesize{4.87243}
\nodepart{three}
\footnotesize{$67\:33$}
};
 & 
\node[draw=black, rectangle split,  rectangle split parts=3] (sn0x9c7520){
\begin{tikzpicture}[scale=.2]
\node[circle, scale=0.75, fill] (tid0) at (3.75,1.5){};
\node[circle, scale=0.75, fill] (tid1) at (2.25,3){};
\node[circle, scale=0.75, fill, red] (tid4) at (0.75,4.5){};
\node[circle, scale=0.75, fill, red] (tid5) at (2.25,4.5){};
\node[circle, scale=0.75, fill] (tid6) at (3.75,4.5){};
\draw[](tid1) -- (tid4);
\draw[](tid1) -- (tid5);
\draw[](tid1) -- (tid6);
\node[circle, scale=0.75, fill] (tid2) at (5.25,3){};
\node[circle, scale=0.75, fill, red] (tid7) at (5.25,4.5){};
\draw[](tid2) -- (tid7);
\node[circle, scale=0.75, fill] (tid3) at (6.75,3){};
\node[circle, scale=0.75, fill] (tid8) at (6.75,4.5){};
\draw[](tid3) -- (tid8);
\draw[](tid0) -- (tid1);
\draw[](tid0) -- (tid2);
\draw[](tid0) -- (tid3);
\end{tikzpicture}
\nodepart{two}
\footnotesize{4.87243}
\nodepart{three}
\footnotesize{$33\:33\:17\:17$}
};
 & 
\node[draw=black, rectangle split,  rectangle split parts=3] (sn0x9c8f30){
\begin{tikzpicture}[scale=.2]
\node[circle, scale=0.75, fill] (tid0) at (3.75,1.5){};
\node[circle, scale=0.75, fill] (tid1) at (2.25,3){};
\node[circle, scale=0.75, fill, red] (tid4) at (0.75,4.5){};
\node[circle, scale=0.75, fill] (tid5) at (2.25,4.5){};
\node[circle, scale=0.75, fill] (tid6) at (3.75,4.5){};
\draw[](tid1) -- (tid4);
\draw[](tid1) -- (tid5);
\draw[](tid1) -- (tid6);
\node[circle, scale=0.75, fill] (tid2) at (5.25,3){};
\node[circle, scale=0.75, fill, red] (tid7) at (5.25,4.5){};
\draw[](tid2) -- (tid7);
\node[circle, scale=0.75, fill] (tid3) at (6.75,3){};
\node[circle, scale=0.75, fill, red] (tid8) at (6.75,4.5){};
\draw[](tid3) -- (tid8);
\draw[](tid0) -- (tid1);
\draw[](tid0) -- (tid2);
\draw[](tid0) -- (tid3);
\end{tikzpicture}
\nodepart{two}
\footnotesize{4.86728}
\nodepart{three}
\footnotesize{$33\:67$}
};
 & 
\node[draw=black, rectangle split,  rectangle split parts=3] (sn0x9c9450){
\begin{tikzpicture}[scale=.2]
\node[circle, scale=0.75, fill] (tid0) at (4.5,1.5){};
\node[circle, scale=0.75, fill] (tid1) at (2.25,3){};
\node[circle, scale=0.75, fill, red] (tid4) at (0.75,4.5){};
\node[circle, scale=0.75, fill, red] (tid5) at (2.25,4.5){};
\node[circle, scale=0.75, fill] (tid6) at (3.75,4.5){};
\draw[](tid1) -- (tid4);
\draw[](tid1) -- (tid5);
\draw[](tid1) -- (tid6);
\node[circle, scale=0.75, fill] (tid2) at (6,3){};
\node[circle, scale=0.75, fill, red] (tid7) at (5.25,4.5){};
\node[circle, scale=0.75, fill] (tid8) at (6.75,4.5){};
\draw[](tid2) -- (tid7);
\draw[](tid2) -- (tid8);
\node[circle, scale=0.75, fill] (tid3) at (8.25,3){};
\draw[](tid0) -- (tid1);
\draw[](tid0) -- (tid2);
\draw[](tid0) -- (tid3);
\end{tikzpicture}
\nodepart{two}
\footnotesize{4.86934}
\nodepart{three}
\footnotesize{$67\:17\:17$}
};
 & 
\node[draw=black, rectangle split,  rectangle split parts=3] (sn0x9c9a10){
\begin{tikzpicture}[scale=.2]
\node[circle, scale=0.75, fill] (tid0) at (4.5,1.5){};
\node[circle, scale=0.75, fill] (tid1) at (2.25,3){};
\node[circle, scale=0.75, fill, red] (tid4) at (0.75,4.5){};
\node[circle, scale=0.75, fill] (tid5) at (2.25,4.5){};
\node[circle, scale=0.75, fill] (tid6) at (3.75,4.5){};
\draw[](tid1) -- (tid4);
\draw[](tid1) -- (tid5);
\draw[](tid1) -- (tid6);
\node[circle, scale=0.75, fill] (tid2) at (6,3){};
\node[circle, scale=0.75, fill, red] (tid7) at (5.25,4.5){};
\node[circle, scale=0.75, fill, red] (tid8) at (6.75,4.5){};
\draw[](tid2) -- (tid7);
\draw[](tid2) -- (tid8);
\node[circle, scale=0.75, fill] (tid3) at (8.25,3){};
\draw[](tid0) -- (tid1);
\draw[](tid0) -- (tid2);
\draw[](tid0) -- (tid3);
\end{tikzpicture}
\nodepart{two}
\footnotesize{4.86626}
\nodepart{three}
\footnotesize{$33\:67$}
};
 & 
\\
};
\end{scope}
\begin{scope}[yshift=\leveltopIII cm]
\matrix (line3) [column sep=1cm] {
\node[draw=black, rectangle split,  rectangle split parts=3] (sn0x9c2ec0){
\begin{tikzpicture}[scale=.2]
\node[circle, scale=0.75, fill] (tid0) at (3,1.5){};
\node[circle, scale=0.75, fill] (tid1) at (1.5,3){};
\node[circle, scale=0.75, fill, red] (tid4) at (0.75,4.5){};
\node[circle, scale=0.75, fill] (tid5) at (2.25,4.5){};
\draw[](tid1) -- (tid4);
\draw[](tid1) -- (tid5);
\node[circle, scale=0.75, fill] (tid2) at (3.75,3){};
\node[circle, scale=0.75, fill, red] (tid6) at (3.75,4.5){};
\draw[](tid2) -- (tid6);
\node[circle, scale=0.75, fill] (tid3) at (5.25,3){};
\node[circle, scale=0.75, fill, red] (tid7) at (5.25,4.5){};
\draw[](tid3) -- (tid7);
\draw[](tid0) -- (tid1);
\draw[](tid0) -- (tid2);
\draw[](tid0) -- (tid3);
\end{tikzpicture}
\nodepart{two}
\footnotesize{4.54012}
\nodepart{three}
\footnotesize{$33\:67$}
};
 & 
\node[draw=black, rectangle split,  rectangle split parts=3] (sn0x9c1e40){
\begin{tikzpicture}[scale=.2]
\node[circle, scale=0.75, fill] (tid0) at (3,1.5){};
\node[circle, scale=0.75, fill] (tid1) at (1.5,3){};
\node[circle, scale=0.75, fill, red] (tid4) at (0.75,4.5){};
\node[circle, scale=0.75, fill, red] (tid5) at (2.25,4.5){};
\draw[](tid1) -- (tid4);
\draw[](tid1) -- (tid5);
\node[circle, scale=0.75, fill] (tid2) at (3.75,3){};
\node[circle, scale=0.75, fill, red] (tid6) at (3.75,4.5){};
\draw[](tid2) -- (tid6);
\node[circle, scale=0.75, fill] (tid3) at (5.25,3){};
\node[circle, scale=0.75, fill] (tid7) at (5.25,4.5){};
\draw[](tid3) -- (tid7);
\draw[](tid0) -- (tid1);
\draw[](tid0) -- (tid2);
\draw[](tid0) -- (tid3);
\end{tikzpicture}
\nodepart{two}
\footnotesize{4.54321}
\nodepart{three}
\footnotesize{$67\:33$}
};
 & 
\node[draw=black, rectangle split,  rectangle split parts=3] (sn0x9c68f0){
\begin{tikzpicture}[scale=.2]
\node[circle, scale=0.75, fill] (tid0) at (3.75,1.5){};
\node[circle, scale=0.75, fill] (tid1) at (1.5,3){};
\node[circle, scale=0.75, fill, red] (tid4) at (0.75,4.5){};
\node[circle, scale=0.75, fill, red] (tid5) at (2.25,4.5){};
\draw[](tid1) -- (tid4);
\draw[](tid1) -- (tid5);
\node[circle, scale=0.75, fill] (tid2) at (4.5,3){};
\node[circle, scale=0.75, fill, red] (tid6) at (3.75,4.5){};
\node[circle, scale=0.75, fill] (tid7) at (5.25,4.5){};
\draw[](tid2) -- (tid6);
\draw[](tid2) -- (tid7);
\node[circle, scale=0.75, fill] (tid3) at (6.75,3){};
\draw[](tid0) -- (tid1);
\draw[](tid0) -- (tid2);
\draw[](tid0) -- (tid3);
\end{tikzpicture}
\nodepart{two}
\footnotesize{4.53704}
\nodepart{three}
\footnotesize{$1$}
};
 & 
\node[draw=black, rectangle split,  rectangle split parts=3] (sn0x9c7a50){
\begin{tikzpicture}[scale=.2]
\node[circle, scale=0.75, fill] (tid0) at (3.75,1.5){};
\node[circle, scale=0.75, fill] (tid1) at (2.25,3){};
\node[circle, scale=0.75, fill, red] (tid4) at (0.75,4.5){};
\node[circle, scale=0.75, fill, red] (tid5) at (2.25,4.5){};
\node[circle, scale=0.75, fill, red] (tid6) at (3.75,4.5){};
\draw[](tid1) -- (tid4);
\draw[](tid1) -- (tid5);
\draw[](tid1) -- (tid6);
\node[circle, scale=0.75, fill] (tid2) at (5.25,3){};
\node[circle, scale=0.75, fill] (tid7) at (5.25,4.5){};
\draw[](tid2) -- (tid7);
\node[circle, scale=0.75, fill] (tid3) at (6.75,3){};
\draw[](tid0) -- (tid1);
\draw[](tid0) -- (tid2);
\draw[](tid0) -- (tid3);
\end{tikzpicture}
\nodepart{two}
\footnotesize{4.53704}
\nodepart{three}
\footnotesize{$1$}
};
 & 
\node[draw=black, rectangle split,  rectangle split parts=3] (sn0x9c7b50){
\begin{tikzpicture}[scale=.2]
\node[circle, scale=0.75, fill] (tid0) at (3.75,1.5){};
\node[circle, scale=0.75, fill] (tid1) at (2.25,3){};
\node[circle, scale=0.75, fill, red] (tid4) at (0.75,4.5){};
\node[circle, scale=0.75, fill, red] (tid5) at (2.25,4.5){};
\node[circle, scale=0.75, fill] (tid6) at (3.75,4.5){};
\draw[](tid1) -- (tid4);
\draw[](tid1) -- (tid5);
\draw[](tid1) -- (tid6);
\node[circle, scale=0.75, fill] (tid2) at (5.25,3){};
\node[circle, scale=0.75, fill, red] (tid7) at (5.25,4.5){};
\draw[](tid2) -- (tid7);
\node[circle, scale=0.75, fill] (tid3) at (6.75,3){};
\draw[](tid0) -- (tid1);
\draw[](tid0) -- (tid2);
\draw[](tid0) -- (tid3);
\end{tikzpicture}
\nodepart{two}
\footnotesize{4.53086}
\nodepart{three}
\footnotesize{$67\:33$}
};
 & 
\\
};
\end{scope}
\begin{scope}[yshift=\leveltopIIII cm]
\matrix (line4) [column sep=1cm] {
\node[draw=black, rectangle split,  rectangle split parts=3] (sn0x9c3190){
\begin{tikzpicture}[scale=.2]
\node[circle, scale=0.75, fill] (tid0) at (2.25,1.5){};
\node[circle, scale=0.75, fill] (tid1) at (0.75,3){};
\node[circle, scale=0.75, fill, red] (tid4) at (0.75,4.5){};
\draw[](tid1) -- (tid4);
\node[circle, scale=0.75, fill] (tid2) at (2.25,3){};
\node[circle, scale=0.75, fill, red] (tid5) at (2.25,4.5){};
\draw[](tid2) -- (tid5);
\node[circle, scale=0.75, fill] (tid3) at (3.75,3){};
\node[circle, scale=0.75, fill, red] (tid6) at (3.75,4.5){};
\draw[](tid3) -- (tid6);
\draw[](tid0) -- (tid1);
\draw[](tid0) -- (tid2);
\draw[](tid0) -- (tid3);
\end{tikzpicture}
\nodepart{two}
\footnotesize{4.21296}
\nodepart{three}
\footnotesize{$1$}
};
 & 
\node[draw=black, rectangle split,  rectangle split parts=3] (sn0x9c3bb0){
\begin{tikzpicture}[scale=.2]
\node[circle, scale=0.75, fill] (tid0) at (3,1.5){};
\node[circle, scale=0.75, fill] (tid1) at (1.5,3){};
\node[circle, scale=0.75, fill, red] (tid4) at (0.75,4.5){};
\node[circle, scale=0.75, fill, red] (tid5) at (2.25,4.5){};
\draw[](tid1) -- (tid4);
\draw[](tid1) -- (tid5);
\node[circle, scale=0.75, fill] (tid2) at (3.75,3){};
\node[circle, scale=0.75, fill, red] (tid6) at (3.75,4.5){};
\draw[](tid2) -- (tid6);
\node[circle, scale=0.75, fill] (tid3) at (5.25,3){};
\draw[](tid0) -- (tid1);
\draw[](tid0) -- (tid2);
\draw[](tid0) -- (tid3);
\end{tikzpicture}
\nodepart{two}
\footnotesize{4.2037}
\nodepart{three}
\footnotesize{$67\:33$}
};
 & 
\node[draw=black, rectangle split,  rectangle split parts=3] (sn0x9c8420){
\begin{tikzpicture}[scale=.2]
\node[circle, scale=0.75, fill] (tid0) at (3.75,1.5){};
\node[circle, scale=0.75, fill] (tid1) at (2.25,3){};
\node[circle, scale=0.75, fill, red] (tid4) at (0.75,4.5){};
\node[circle, scale=0.75, fill, red] (tid5) at (2.25,4.5){};
\node[circle, scale=0.75, fill, red] (tid6) at (3.75,4.5){};
\draw[](tid1) -- (tid4);
\draw[](tid1) -- (tid5);
\draw[](tid1) -- (tid6);
\node[circle, scale=0.75, fill] (tid2) at (5.25,3){};
\node[circle, scale=0.75, fill] (tid3) at (6.75,3){};
\draw[](tid0) -- (tid1);
\draw[](tid0) -- (tid2);
\draw[](tid0) -- (tid3);
\end{tikzpicture}
\nodepart{two}
\footnotesize{4.18519}
\nodepart{three}
\footnotesize{$1$}
};
 & 
\\
};
\end{scope}
\begin{scope}[yshift=\leveltopIIIII cm]
\matrix (line5) [column sep=1cm] {
\node[draw=black, rectangle split,  rectangle split parts=3] (sn0x9c2bc0){
\begin{tikzpicture}[scale=.2]
\node[circle, scale=0.75, fill] (tid0) at (2.25,1.5){};
\node[circle, scale=0.75, fill] (tid1) at (0.75,3){};
\node[circle, scale=0.75, fill, red] (tid4) at (0.75,4.5){};
\draw[](tid1) -- (tid4);
\node[circle, scale=0.75, fill] (tid2) at (2.25,3){};
\node[circle, scale=0.75, fill, red] (tid5) at (2.25,4.5){};
\draw[](tid2) -- (tid5);
\node[circle, scale=0.75, fill, red] (tid3) at (3.75,3){};
\draw[](tid0) -- (tid1);
\draw[](tid0) -- (tid2);
\draw[](tid0) -- (tid3);
\end{tikzpicture}
\nodepart{two}
\footnotesize{3.87963}
\nodepart{three}
\footnotesize{$33\:67$}
};
 & 
\node[draw=black, rectangle split,  rectangle split parts=3] (sn0x9c48e0){
\begin{tikzpicture}[scale=.2]
\node[circle, scale=0.75, fill] (tid0) at (3,1.5){};
\node[circle, scale=0.75, fill] (tid1) at (1.5,3){};
\node[circle, scale=0.75, fill, red] (tid4) at (0.75,4.5){};
\node[circle, scale=0.75, fill, red] (tid5) at (2.25,4.5){};
\draw[](tid1) -- (tid4);
\draw[](tid1) -- (tid5);
\node[circle, scale=0.75, fill, red] (tid2) at (3.75,3){};
\node[circle, scale=0.75, fill] (tid3) at (5.25,3){};
\draw[](tid0) -- (tid1);
\draw[](tid0) -- (tid2);
\draw[](tid0) -- (tid3);
\end{tikzpicture}
\nodepart{two}
\footnotesize{3.85185}
\nodepart{three}
\footnotesize{$67\:33$}
};
 & 
\\
};
\end{scope}
\begin{scope}[yshift=\leveltopIIIIII cm]
\matrix (line6) [column sep=1cm] {
\node[draw=black, rectangle split,  rectangle split parts=3] (sn0x9c4130){
\begin{tikzpicture}[scale=.2]
\node[circle, scale=0.75, fill] (tid0) at (1.5,1.5){};
\node[circle, scale=0.75, fill] (tid1) at (0.75,3){};
\node[circle, scale=0.75, fill, red] (tid3) at (0.75,4.5){};
\draw[](tid1) -- (tid3);
\node[circle, scale=0.75, fill] (tid2) at (2.25,3){};
\node[circle, scale=0.75, fill, red] (tid4) at (2.25,4.5){};
\draw[](tid2) -- (tid4);
\draw[](tid0) -- (tid1);
\draw[](tid0) -- (tid2);
\end{tikzpicture}
\nodepart{two}
\footnotesize{3.75}
\nodepart{three}
\footnotesize{$1$}
};
 & 
\node[draw=black, rectangle split,  rectangle split parts=3] (sn0x9c36c0){
\begin{tikzpicture}[scale=.2]
\node[circle, scale=0.75, fill] (tid0) at (2.25,1.5){};
\node[circle, scale=0.75, fill] (tid1) at (0.75,3){};
\node[circle, scale=0.75, fill, red] (tid4) at (0.75,4.5){};
\draw[](tid1) -- (tid4);
\node[circle, scale=0.75, fill, red] (tid2) at (2.25,3){};
\node[circle, scale=0.75, fill, red] (tid3) at (3.75,3){};
\draw[](tid0) -- (tid1);
\draw[](tid0) -- (tid2);
\draw[](tid0) -- (tid3);
\end{tikzpicture}
\nodepart{two}
\footnotesize{3.44444}
\nodepart{three}
\footnotesize{$67\:33$}
};
 & 
\node[draw=black, rectangle split,  rectangle split parts=3] (sn0x9c5ae0){
\begin{tikzpicture}[scale=.2]
\node[circle, scale=0.75, fill] (tid0) at (2.25,1.5){};
\node[circle, scale=0.75, fill] (tid1) at (1.5,3){};
\node[circle, scale=0.75, fill, red] (tid3) at (0.75,4.5){};
\node[circle, scale=0.75, fill, red] (tid4) at (2.25,4.5){};
\draw[](tid1) -- (tid3);
\draw[](tid1) -- (tid4);
\node[circle, scale=0.75, fill, red] (tid2) at (3.75,3){};
\draw[](tid0) -- (tid1);
\draw[](tid0) -- (tid2);
\end{tikzpicture}
\nodepart{two}
\footnotesize{3.66667}
\nodepart{three}
\footnotesize{$67\:33$}
};
 & 
\\
};
\end{scope}
\begin{scope}[yshift=\leveltopIIIIIII cm]
\matrix (line7) [column sep=1cm] {
\node[draw=black, rectangle split,  rectangle split parts=3] (sn0x9c4350){
\begin{tikzpicture}[scale=.2]
\node[circle, scale=0.75, fill] (tid0) at (1.5,1.5){};
\node[circle, scale=0.75, fill] (tid1) at (0.75,3){};
\node[circle, scale=0.75, fill, red] (tid3) at (0.75,4.5){};
\draw[](tid1) -- (tid3);
\node[circle, scale=0.75, fill, red] (tid2) at (2.25,3){};
\draw[](tid0) -- (tid1);
\draw[](tid0) -- (tid2);
\end{tikzpicture}
\nodepart{two}
\footnotesize{3.25}
\nodepart{three}
\footnotesize{$50\:50$}
};
 & 
\node[draw=black, rectangle split,  rectangle split parts=3] (sn0x9c4af0){
\begin{tikzpicture}[scale=.2]
\node[circle, scale=0.75, fill] (tid0) at (2.25,1.5){};
\node[circle, scale=0.75, fill, red] (tid1) at (0.75,3){};
\node[circle, scale=0.75, fill, red] (tid2) at (2.25,3){};
\node[circle, scale=0.75, fill, red] (tid3) at (3.75,3){};
\draw[](tid0) -- (tid1);
\draw[](tid0) -- (tid2);
\draw[](tid0) -- (tid3);
\end{tikzpicture}
\nodepart{two}
\footnotesize{2.83333}
\nodepart{three}
\footnotesize{$1$}
};
 & 
\node[draw=black, rectangle split,  rectangle split parts=3] (sn0x9c57b0){
\begin{tikzpicture}[scale=.2]
\node[circle, scale=0.75, fill] (tid0) at (1.5,1.5){};
\node[circle, scale=0.75, fill] (tid1) at (1.5,3){};
\node[circle, scale=0.75, fill, red] (tid2) at (0.75,4.5){};
\node[circle, scale=0.75, fill, red] (tid3) at (2.25,4.5){};
\draw[](tid1) -- (tid2);
\draw[](tid1) -- (tid3);
\draw[](tid0) -- (tid1);
\end{tikzpicture}
\nodepart{two}
\footnotesize{3.5}
\nodepart{three}
\footnotesize{$1$}
};
 & 
\\
};
\end{scope}
\begin{scope}[yshift=\leveltopIIIIIIII cm]
\matrix (line8) [column sep=1cm] {
\node[draw=black, rectangle split,  rectangle split parts=3] (sn0x9c4420){
\begin{tikzpicture}[scale=.2]
\node[circle, scale=0.75, fill] (tid0) at (0.75,1.5){};
\node[circle, scale=0.75, fill] (tid1) at (0.75,3){};
\node[circle, scale=0.75, fill, red] (tid2) at (0.75,4.5){};
\draw[](tid1) -- (tid2);
\draw[](tid0) -- (tid1);
\end{tikzpicture}
\nodepart{two}
\footnotesize{3}
\nodepart{three}
\footnotesize{$1$}
};
 & 
\node[draw=black, rectangle split,  rectangle split parts=3] (sn0x9c45d0){
\begin{tikzpicture}[scale=.2]
\node[circle, scale=0.75, fill] (tid0) at (1.5,1.5){};
\node[circle, scale=0.75, fill, red] (tid1) at (0.75,3){};
\node[circle, scale=0.75, fill, red] (tid2) at (2.25,3){};
\draw[](tid0) -- (tid1);
\draw[](tid0) -- (tid2);
\end{tikzpicture}
\nodepart{two}
\footnotesize{2.5}
\nodepart{three}
\footnotesize{$1$}
};
 & 
\\
};
\end{scope}
\begin{scope}[yshift=\leveltopIIIIIIIII cm]
\matrix (line9) [column sep=1cm] {
\node[draw=black, rectangle split,  rectangle split parts=3] (sn0x9c46e0){
\begin{tikzpicture}[scale=.2]
\node[circle, scale=0.75, fill] (tid0) at (0.75,1.5){};
\node[circle, scale=0.75, fill, red] (tid1) at (0.75,3){};
\draw[](tid0) -- (tid1);
\end{tikzpicture}
\nodepart{two}
\footnotesize{2}
\nodepart{three}
\footnotesize{$1$}
};
 & 
\\
};
\end{scope}
\begin{scope}[yshift=\leveltopIIIIIIIIII cm]
\matrix (line10) [column sep=1cm] {
\node[draw=black, rectangle split,  rectangle split parts=3] (sn0x9c47b0){
\begin{tikzpicture}[scale=.2]
\node[circle, scale=0.75, fill, red] (tid0) at (0.75,1.5){};
\end{tikzpicture}
\nodepart{two}
\footnotesize{1}
\nodepart{three}
\footnotesize{$$}
};
 & 
\\
};
\end{scope}
\begin{scope}[yshift=\leveltopIIIIIIIIIII cm]
\matrix (line11) [column sep=1cm] {
\\
};
\end{scope}
\draw (sn0x9c14f0.south) -- (sn0x9c6cb0.north);
\draw (sn0x9c14f0.south) -- (sn0x9c2480.north);
\draw (sn0x9c14f0.south) -- (sn0x9c7520.north);
\draw (sn0x9c14f0.south) -- (sn0x9c8f30.north);
\draw (sn0x9c14f0.south) -- (sn0x9c9450.north);
\draw (sn0x9c14f0.south) -- (sn0x9c9a10.north);
\draw (sn0x9c6cb0.south) -- (sn0x9c2ec0.north);
\draw (sn0x9c6cb0.south) -- (sn0x9c1e40.north);
\draw (sn0x9c6cb0.south) -- (sn0x9c68f0.north);
\draw (sn0x9c2480.south) -- (sn0x9c2ec0.north);
\draw (sn0x9c2480.south) -- (sn0x9c68f0.north);
\draw (sn0x9c7520.south) -- (sn0x9c1e40.north);
\draw (sn0x9c7520.south) -- (sn0x9c2ec0.north);
\draw (sn0x9c7520.south) -- (sn0x9c7a50.north);
\draw (sn0x9c7520.south) -- (sn0x9c7b50.north);
\draw (sn0x9c8f30.south) -- (sn0x9c2ec0.north);
\draw (sn0x9c8f30.south) -- (sn0x9c7b50.north);
\draw (sn0x9c9450.south) -- (sn0x9c68f0.north);
\draw (sn0x9c9450.south) -- (sn0x9c7a50.north);
\draw (sn0x9c9450.south) -- (sn0x9c7b50.north);
\draw (sn0x9c9a10.south) -- (sn0x9c68f0.north);
\draw (sn0x9c9a10.south) -- (sn0x9c7b50.north);
\draw (sn0x9c2ec0.south) -- (sn0x9c3190.north);
\draw (sn0x9c2ec0.south) -- (sn0x9c3bb0.north);
\draw (sn0x9c1e40.south) -- (sn0x9c3190.north);
\draw (sn0x9c1e40.south) -- (sn0x9c3bb0.north);
\draw (sn0x9c68f0.south) -- (sn0x9c3bb0.north);
\draw (sn0x9c7a50.south) -- (sn0x9c3bb0.north);
\draw (sn0x9c7b50.south) -- (sn0x9c3bb0.north);
\draw (sn0x9c7b50.south) -- (sn0x9c8420.north);
\draw (sn0x9c3190.south) -- (sn0x9c2bc0.north);
\draw (sn0x9c3bb0.south) -- (sn0x9c2bc0.north);
\draw (sn0x9c3bb0.south) -- (sn0x9c48e0.north);
\draw (sn0x9c8420.south) -- (sn0x9c48e0.north);
\draw (sn0x9c2bc0.south) -- (sn0x9c4130.north);
\draw (sn0x9c2bc0.south) -- (sn0x9c36c0.north);
\draw (sn0x9c48e0.south) -- (sn0x9c5ae0.north);
\draw (sn0x9c48e0.south) -- (sn0x9c36c0.north);
\draw (sn0x9c4130.south) -- (sn0x9c4350.north);
\draw (sn0x9c36c0.south) -- (sn0x9c4350.north);
\draw (sn0x9c36c0.south) -- (sn0x9c4af0.north);
\draw (sn0x9c5ae0.south) -- (sn0x9c57b0.north);
\draw (sn0x9c5ae0.south) -- (sn0x9c4350.north);
\draw (sn0x9c4350.south) -- (sn0x9c4420.north);
\draw (sn0x9c4350.south) -- (sn0x9c45d0.north);
\draw (sn0x9c4af0.south) -- (sn0x9c45d0.north);
\draw (sn0x9c57b0.south) -- (sn0x9c4420.north);
\draw (sn0x9c4420.south) -- (sn0x9c46e0.north);
\draw (sn0x9c45d0.south) -- (sn0x9c46e0.north);
\draw (sn0x9c46e0.south) -- (sn0x9c47b0.north);
\end{tikzpicture}

%%% Local Variables:
%%% TeX-master: "thesis/thesis.tex"
%%% End: 
\renewcommand{\leveltopI}{-15cm + \leveltop}
\renewcommand{\leveltopII}{-15cm + \leveltopI}
\renewcommand{\leveltopIII}{-15cm + \leveltopII}
\renewcommand{\leveltopIIII}{-15cm + \leveltopIII}
\renewcommand{\leveltopIIIII}{-15cm + \leveltopIIII}
\renewcommand{\leveltopIIIIII}{-15cm + \leveltopIIIII}
\renewcommand{\leveltopIIIIIII}{-15cm + \leveltopIIIIII}
\renewcommand{\leveltopIIIIIIII}{-15cm + \leveltopIIIIIII}
\renewcommand{\leveltopIIIIIIIII}{-15cm + \leveltopIIIIIIII}
\renewcommand{\leveltopIIIIIIIIII}{-15cm + \leveltopIIIIIIIII}
\begin{tikzpicture}[scale=.2, anchor=south]
\begin{scope}[yshift=\leveltopI cm]
\matrix (line1) [column sep=1cm] {
\node[draw=black, rectangle split,  rectangle split parts=3] (sn0x9c27e0){
\begin{tikzpicture}[scale=.2]
\node[circle, scale=0.75, fill] (tid0) at (4.5,1.5){};
\node[circle, scale=0.75, fill] (tid1) at (2.25,3){};
\node[circle, scale=0.75, fill] (tid4) at (0.75,4.5){};
\node[circle, scale=0.75, fill] (tid5) at (2.25,4.5){};
\node[circle, scale=0.75, fill] (tid6) at (3.75,4.5){};
\draw[](tid1) -- (tid4);
\draw[](tid1) -- (tid5);
\draw[](tid1) -- (tid6);
\node[circle, scale=0.75, fill] (tid2) at (6,3){};
\node[circle, scale=0.75, fill, red] (tid7) at (5.25,4.5){};
\node[circle, scale=0.75, fill, red] (tid8) at (6.75,4.5){};
\draw[](tid2) -- (tid7);
\draw[](tid2) -- (tid8);
\node[circle, scale=0.75, fill] (tid3) at (8.25,3){};
\node[circle, scale=0.75, fill, red] (tid9) at (8.25,4.5){};
\draw[](tid3) -- (tid9);
\draw[](tid0) -- (tid1);
\draw[](tid0) -- (tid2);
\draw[](tid0) -- (tid3);
\end{tikzpicture}
\nodepart{two}
\footnotesize{5.20028}
\nodepart{three}
\footnotesize{$67\:33$}
};
 & 
\\
};
\end{scope}
\begin{scope}[yshift=\leveltopII cm]
\matrix (line2) [column sep=1cm] {
\node[draw=black, rectangle split,  rectangle split parts=3] (sn0x9c8f30){
\begin{tikzpicture}[scale=.2]
\node[circle, scale=0.75, fill] (tid0) at (3.75,1.5){};
\node[circle, scale=0.75, fill] (tid1) at (2.25,3){};
\node[circle, scale=0.75, fill, red] (tid4) at (0.75,4.5){};
\node[circle, scale=0.75, fill] (tid5) at (2.25,4.5){};
\node[circle, scale=0.75, fill] (tid6) at (3.75,4.5){};
\draw[](tid1) -- (tid4);
\draw[](tid1) -- (tid5);
\draw[](tid1) -- (tid6);
\node[circle, scale=0.75, fill] (tid2) at (5.25,3){};
\node[circle, scale=0.75, fill, red] (tid7) at (5.25,4.5){};
\draw[](tid2) -- (tid7);
\node[circle, scale=0.75, fill] (tid3) at (6.75,3){};
\node[circle, scale=0.75, fill, red] (tid8) at (6.75,4.5){};
\draw[](tid3) -- (tid8);
\draw[](tid0) -- (tid1);
\draw[](tid0) -- (tid2);
\draw[](tid0) -- (tid3);
\end{tikzpicture}
\nodepart{two}
\footnotesize{4.86728}
\nodepart{three}
\footnotesize{$33\:67$}
};
 & 
\node[draw=black, rectangle split,  rectangle split parts=3] (sn0x9c9a10){
\begin{tikzpicture}[scale=.2]
\node[circle, scale=0.75, fill] (tid0) at (4.5,1.5){};
\node[circle, scale=0.75, fill] (tid1) at (2.25,3){};
\node[circle, scale=0.75, fill, red] (tid4) at (0.75,4.5){};
\node[circle, scale=0.75, fill] (tid5) at (2.25,4.5){};
\node[circle, scale=0.75, fill] (tid6) at (3.75,4.5){};
\draw[](tid1) -- (tid4);
\draw[](tid1) -- (tid5);
\draw[](tid1) -- (tid6);
\node[circle, scale=0.75, fill] (tid2) at (6,3){};
\node[circle, scale=0.75, fill, red] (tid7) at (5.25,4.5){};
\node[circle, scale=0.75, fill, red] (tid8) at (6.75,4.5){};
\draw[](tid2) -- (tid7);
\draw[](tid2) -- (tid8);
\node[circle, scale=0.75, fill] (tid3) at (8.25,3){};
\draw[](tid0) -- (tid1);
\draw[](tid0) -- (tid2);
\draw[](tid0) -- (tid3);
\end{tikzpicture}
\nodepart{two}
\footnotesize{4.86626}
\nodepart{three}
\footnotesize{$67\:33$}
};
 & 
\\
};
\end{scope}
\begin{scope}[yshift=\leveltopIII cm]
\matrix (line3) [column sep=1cm] {
\node[draw=black, rectangle split,  rectangle split parts=3] (sn0x9c2ec0){
\begin{tikzpicture}[scale=.2]
\node[circle, scale=0.75, fill] (tid0) at (3,1.5){};
\node[circle, scale=0.75, fill] (tid1) at (1.5,3){};
\node[circle, scale=0.75, fill, red] (tid4) at (0.75,4.5){};
\node[circle, scale=0.75, fill] (tid5) at (2.25,4.5){};
\draw[](tid1) -- (tid4);
\draw[](tid1) -- (tid5);
\node[circle, scale=0.75, fill] (tid2) at (3.75,3){};
\node[circle, scale=0.75, fill, red] (tid6) at (3.75,4.5){};
\draw[](tid2) -- (tid6);
\node[circle, scale=0.75, fill] (tid3) at (5.25,3){};
\node[circle, scale=0.75, fill, red] (tid7) at (5.25,4.5){};
\draw[](tid3) -- (tid7);
\draw[](tid0) -- (tid1);
\draw[](tid0) -- (tid2);
\draw[](tid0) -- (tid3);
\end{tikzpicture}
\nodepart{two}
\footnotesize{4.54012}
\nodepart{three}
\footnotesize{$33\:67$}
};
 & 
\node[draw=black, rectangle split,  rectangle split parts=3] (sn0x9c7b50){
\begin{tikzpicture}[scale=.2]
\node[circle, scale=0.75, fill] (tid0) at (3.75,1.5){};
\node[circle, scale=0.75, fill] (tid1) at (2.25,3){};
\node[circle, scale=0.75, fill, red] (tid4) at (0.75,4.5){};
\node[circle, scale=0.75, fill, red] (tid5) at (2.25,4.5){};
\node[circle, scale=0.75, fill] (tid6) at (3.75,4.5){};
\draw[](tid1) -- (tid4);
\draw[](tid1) -- (tid5);
\draw[](tid1) -- (tid6);
\node[circle, scale=0.75, fill] (tid2) at (5.25,3){};
\node[circle, scale=0.75, fill, red] (tid7) at (5.25,4.5){};
\draw[](tid2) -- (tid7);
\node[circle, scale=0.75, fill] (tid3) at (6.75,3){};
\draw[](tid0) -- (tid1);
\draw[](tid0) -- (tid2);
\draw[](tid0) -- (tid3);
\end{tikzpicture}
\nodepart{two}
\footnotesize{4.53086}
\nodepart{three}
\footnotesize{$67\:33$}
};
 & 
\node[draw=black, rectangle split,  rectangle split parts=3] (sn0x9c68f0){
\begin{tikzpicture}[scale=.2]
\node[circle, scale=0.75, fill] (tid0) at (3.75,1.5){};
\node[circle, scale=0.75, fill] (tid1) at (1.5,3){};
\node[circle, scale=0.75, fill, red] (tid4) at (0.75,4.5){};
\node[circle, scale=0.75, fill, red] (tid5) at (2.25,4.5){};
\draw[](tid1) -- (tid4);
\draw[](tid1) -- (tid5);
\node[circle, scale=0.75, fill] (tid2) at (4.5,3){};
\node[circle, scale=0.75, fill, red] (tid6) at (3.75,4.5){};
\node[circle, scale=0.75, fill] (tid7) at (5.25,4.5){};
\draw[](tid2) -- (tid6);
\draw[](tid2) -- (tid7);
\node[circle, scale=0.75, fill] (tid3) at (6.75,3){};
\draw[](tid0) -- (tid1);
\draw[](tid0) -- (tid2);
\draw[](tid0) -- (tid3);
\end{tikzpicture}
\nodepart{two}
\footnotesize{4.53704}
\nodepart{three}
\footnotesize{$1$}
};
 & 
\\
};
\end{scope}
\begin{scope}[yshift=\leveltopIIII cm]
\matrix (line4) [column sep=1cm] {
\node[draw=black, rectangle split,  rectangle split parts=3] (sn0x9c3190){
\begin{tikzpicture}[scale=.2]
\node[circle, scale=0.75, fill] (tid0) at (2.25,1.5){};
\node[circle, scale=0.75, fill] (tid1) at (0.75,3){};
\node[circle, scale=0.75, fill, red] (tid4) at (0.75,4.5){};
\draw[](tid1) -- (tid4);
\node[circle, scale=0.75, fill] (tid2) at (2.25,3){};
\node[circle, scale=0.75, fill, red] (tid5) at (2.25,4.5){};
\draw[](tid2) -- (tid5);
\node[circle, scale=0.75, fill] (tid3) at (3.75,3){};
\node[circle, scale=0.75, fill, red] (tid6) at (3.75,4.5){};
\draw[](tid3) -- (tid6);
\draw[](tid0) -- (tid1);
\draw[](tid0) -- (tid2);
\draw[](tid0) -- (tid3);
\end{tikzpicture}
\nodepart{two}
\footnotesize{4.21296}
\nodepart{three}
\footnotesize{$1$}
};
 & 
\node[draw=black, rectangle split,  rectangle split parts=3] (sn0x9c3bb0){
\begin{tikzpicture}[scale=.2]
\node[circle, scale=0.75, fill] (tid0) at (3,1.5){};
\node[circle, scale=0.75, fill] (tid1) at (1.5,3){};
\node[circle, scale=0.75, fill, red] (tid4) at (0.75,4.5){};
\node[circle, scale=0.75, fill, red] (tid5) at (2.25,4.5){};
\draw[](tid1) -- (tid4);
\draw[](tid1) -- (tid5);
\node[circle, scale=0.75, fill] (tid2) at (3.75,3){};
\node[circle, scale=0.75, fill, red] (tid6) at (3.75,4.5){};
\draw[](tid2) -- (tid6);
\node[circle, scale=0.75, fill] (tid3) at (5.25,3){};
\draw[](tid0) -- (tid1);
\draw[](tid0) -- (tid2);
\draw[](tid0) -- (tid3);
\end{tikzpicture}
\nodepart{two}
\footnotesize{4.2037}
\nodepart{three}
\footnotesize{$67\:33$}
};
 & 
\node[draw=black, rectangle split,  rectangle split parts=3] (sn0x9c8420){
\begin{tikzpicture}[scale=.2]
\node[circle, scale=0.75, fill] (tid0) at (3.75,1.5){};
\node[circle, scale=0.75, fill] (tid1) at (2.25,3){};
\node[circle, scale=0.75, fill, red] (tid4) at (0.75,4.5){};
\node[circle, scale=0.75, fill, red] (tid5) at (2.25,4.5){};
\node[circle, scale=0.75, fill, red] (tid6) at (3.75,4.5){};
\draw[](tid1) -- (tid4);
\draw[](tid1) -- (tid5);
\draw[](tid1) -- (tid6);
\node[circle, scale=0.75, fill] (tid2) at (5.25,3){};
\node[circle, scale=0.75, fill] (tid3) at (6.75,3){};
\draw[](tid0) -- (tid1);
\draw[](tid0) -- (tid2);
\draw[](tid0) -- (tid3);
\end{tikzpicture}
\nodepart{two}
\footnotesize{4.18519}
\nodepart{three}
\footnotesize{$1$}
};
 & 
\\
};
\end{scope}
\begin{scope}[yshift=\leveltopIIIII cm]
\matrix (line5) [column sep=1cm] {
\node[draw=black, rectangle split,  rectangle split parts=3] (sn0x9c2bc0){
\begin{tikzpicture}[scale=.2]
\node[circle, scale=0.75, fill] (tid0) at (2.25,1.5){};
\node[circle, scale=0.75, fill] (tid1) at (0.75,3){};
\node[circle, scale=0.75, fill, red] (tid4) at (0.75,4.5){};
\draw[](tid1) -- (tid4);
\node[circle, scale=0.75, fill] (tid2) at (2.25,3){};
\node[circle, scale=0.75, fill, red] (tid5) at (2.25,4.5){};
\draw[](tid2) -- (tid5);
\node[circle, scale=0.75, fill, red] (tid3) at (3.75,3){};
\draw[](tid0) -- (tid1);
\draw[](tid0) -- (tid2);
\draw[](tid0) -- (tid3);
\end{tikzpicture}
\nodepart{two}
\footnotesize{3.87963}
\nodepart{three}
\footnotesize{$33\:67$}
};
 & 
\node[draw=black, rectangle split,  rectangle split parts=3] (sn0x9c48e0){
\begin{tikzpicture}[scale=.2]
\node[circle, scale=0.75, fill] (tid0) at (3,1.5){};
\node[circle, scale=0.75, fill] (tid1) at (1.5,3){};
\node[circle, scale=0.75, fill, red] (tid4) at (0.75,4.5){};
\node[circle, scale=0.75, fill, red] (tid5) at (2.25,4.5){};
\draw[](tid1) -- (tid4);
\draw[](tid1) -- (tid5);
\node[circle, scale=0.75, fill, red] (tid2) at (3.75,3){};
\node[circle, scale=0.75, fill] (tid3) at (5.25,3){};
\draw[](tid0) -- (tid1);
\draw[](tid0) -- (tid2);
\draw[](tid0) -- (tid3);
\end{tikzpicture}
\nodepart{two}
\footnotesize{3.85185}
\nodepart{three}
\footnotesize{$67\:33$}
};
 & 
\\
};
\end{scope}
\begin{scope}[yshift=\leveltopIIIIII cm]
\matrix (line6) [column sep=1cm] {
\node[draw=black, rectangle split,  rectangle split parts=3] (sn0x9c4130){
\begin{tikzpicture}[scale=.2]
\node[circle, scale=0.75, fill] (tid0) at (1.5,1.5){};
\node[circle, scale=0.75, fill] (tid1) at (0.75,3){};
\node[circle, scale=0.75, fill, red] (tid3) at (0.75,4.5){};
\draw[](tid1) -- (tid3);
\node[circle, scale=0.75, fill] (tid2) at (2.25,3){};
\node[circle, scale=0.75, fill, red] (tid4) at (2.25,4.5){};
\draw[](tid2) -- (tid4);
\draw[](tid0) -- (tid1);
\draw[](tid0) -- (tid2);
\end{tikzpicture}
\nodepart{two}
\footnotesize{3.75}
\nodepart{three}
\footnotesize{$1$}
};
 & 
\node[draw=black, rectangle split,  rectangle split parts=3] (sn0x9c36c0){
\begin{tikzpicture}[scale=.2]
\node[circle, scale=0.75, fill] (tid0) at (2.25,1.5){};
\node[circle, scale=0.75, fill] (tid1) at (0.75,3){};
\node[circle, scale=0.75, fill, red] (tid4) at (0.75,4.5){};
\draw[](tid1) -- (tid4);
\node[circle, scale=0.75, fill, red] (tid2) at (2.25,3){};
\node[circle, scale=0.75, fill, red] (tid3) at (3.75,3){};
\draw[](tid0) -- (tid1);
\draw[](tid0) -- (tid2);
\draw[](tid0) -- (tid3);
\end{tikzpicture}
\nodepart{two}
\footnotesize{3.44444}
\nodepart{three}
\footnotesize{$67\:33$}
};
 & 
\node[draw=black, rectangle split,  rectangle split parts=3] (sn0x9c5ae0){
\begin{tikzpicture}[scale=.2]
\node[circle, scale=0.75, fill] (tid0) at (2.25,1.5){};
\node[circle, scale=0.75, fill] (tid1) at (1.5,3){};
\node[circle, scale=0.75, fill, red] (tid3) at (0.75,4.5){};
\node[circle, scale=0.75, fill, red] (tid4) at (2.25,4.5){};
\draw[](tid1) -- (tid3);
\draw[](tid1) -- (tid4);
\node[circle, scale=0.75, fill, red] (tid2) at (3.75,3){};
\draw[](tid0) -- (tid1);
\draw[](tid0) -- (tid2);
\end{tikzpicture}
\nodepart{two}
\footnotesize{3.66667}
\nodepart{three}
\footnotesize{$67\:33$}
};
 & 
\\
};
\end{scope}
\begin{scope}[yshift=\leveltopIIIIIII cm]
\matrix (line7) [column sep=1cm] {
\node[draw=black, rectangle split,  rectangle split parts=3] (sn0x9c4350){
\begin{tikzpicture}[scale=.2]
\node[circle, scale=0.75, fill] (tid0) at (1.5,1.5){};
\node[circle, scale=0.75, fill] (tid1) at (0.75,3){};
\node[circle, scale=0.75, fill, red] (tid3) at (0.75,4.5){};
\draw[](tid1) -- (tid3);
\node[circle, scale=0.75, fill, red] (tid2) at (2.25,3){};
\draw[](tid0) -- (tid1);
\draw[](tid0) -- (tid2);
\end{tikzpicture}
\nodepart{two}
\footnotesize{3.25}
\nodepart{three}
\footnotesize{$50\:50$}
};
 & 
\node[draw=black, rectangle split,  rectangle split parts=3] (sn0x9c4af0){
\begin{tikzpicture}[scale=.2]
\node[circle, scale=0.75, fill] (tid0) at (2.25,1.5){};
\node[circle, scale=0.75, fill, red] (tid1) at (0.75,3){};
\node[circle, scale=0.75, fill, red] (tid2) at (2.25,3){};
\node[circle, scale=0.75, fill, red] (tid3) at (3.75,3){};
\draw[](tid0) -- (tid1);
\draw[](tid0) -- (tid2);
\draw[](tid0) -- (tid3);
\end{tikzpicture}
\nodepart{two}
\footnotesize{2.83333}
\nodepart{three}
\footnotesize{$1$}
};
 & 
\node[draw=black, rectangle split,  rectangle split parts=3] (sn0x9c57b0){
\begin{tikzpicture}[scale=.2]
\node[circle, scale=0.75, fill] (tid0) at (1.5,1.5){};
\node[circle, scale=0.75, fill] (tid1) at (1.5,3){};
\node[circle, scale=0.75, fill, red] (tid2) at (0.75,4.5){};
\node[circle, scale=0.75, fill, red] (tid3) at (2.25,4.5){};
\draw[](tid1) -- (tid2);
\draw[](tid1) -- (tid3);
\draw[](tid0) -- (tid1);
\end{tikzpicture}
\nodepart{two}
\footnotesize{3.5}
\nodepart{three}
\footnotesize{$1$}
};
 & 
\\
};
\end{scope}
\begin{scope}[yshift=\leveltopIIIIIIII cm]
\matrix (line8) [column sep=1cm] {
\node[draw=black, rectangle split,  rectangle split parts=3] (sn0x9c4420){
\begin{tikzpicture}[scale=.2]
\node[circle, scale=0.75, fill] (tid0) at (0.75,1.5){};
\node[circle, scale=0.75, fill] (tid1) at (0.75,3){};
\node[circle, scale=0.75, fill, red] (tid2) at (0.75,4.5){};
\draw[](tid1) -- (tid2);
\draw[](tid0) -- (tid1);
\end{tikzpicture}
\nodepart{two}
\footnotesize{3}
\nodepart{three}
\footnotesize{$1$}
};
 & 
\node[draw=black, rectangle split,  rectangle split parts=3] (sn0x9c45d0){
\begin{tikzpicture}[scale=.2]
\node[circle, scale=0.75, fill] (tid0) at (1.5,1.5){};
\node[circle, scale=0.75, fill, red] (tid1) at (0.75,3){};
\node[circle, scale=0.75, fill, red] (tid2) at (2.25,3){};
\draw[](tid0) -- (tid1);
\draw[](tid0) -- (tid2);
\end{tikzpicture}
\nodepart{two}
\footnotesize{2.5}
\nodepart{three}
\footnotesize{$1$}
};
 & 
\\
};
\end{scope}
\begin{scope}[yshift=\leveltopIIIIIIIII cm]
\matrix (line9) [column sep=1cm] {
\node[draw=black, rectangle split,  rectangle split parts=3] (sn0x9c46e0){
\begin{tikzpicture}[scale=.2]
\node[circle, scale=0.75, fill] (tid0) at (0.75,1.5){};
\node[circle, scale=0.75, fill, red] (tid1) at (0.75,3){};
\draw[](tid0) -- (tid1);
\end{tikzpicture}
\nodepart{two}
\footnotesize{2}
\nodepart{three}
\footnotesize{$1$}
};
 & 
\\
};
\end{scope}
\begin{scope}[yshift=\leveltopIIIIIIIIII cm]
\matrix (line10) [column sep=1cm] {
\node[draw=black, rectangle split,  rectangle split parts=3] (sn0x9c47b0){
\begin{tikzpicture}[scale=.2]
\node[circle, scale=0.75, fill, red] (tid0) at (0.75,1.5){};
\end{tikzpicture}
\nodepart{two}
\footnotesize{1}
\nodepart{three}
\footnotesize{$$}
};
 & 
\\
};
\end{scope}
\begin{scope}[yshift=\leveltopIIIIIIIIIII cm]
\matrix (line11) [column sep=1cm] {
\\
};
\end{scope}
\draw (sn0x9c27e0.south) -- (sn0x9c8f30.north);
\draw (sn0x9c27e0.south) -- (sn0x9c9a10.north);
\draw (sn0x9c8f30.south) -- (sn0x9c2ec0.north);
\draw (sn0x9c8f30.south) -- (sn0x9c7b50.north);
\draw (sn0x9c9a10.south) -- (sn0x9c68f0.north);
\draw (sn0x9c9a10.south) -- (sn0x9c7b50.north);
\draw (sn0x9c2ec0.south) -- (sn0x9c3190.north);
\draw (sn0x9c2ec0.south) -- (sn0x9c3bb0.north);
\draw (sn0x9c7b50.south) -- (sn0x9c3bb0.north);
\draw (sn0x9c7b50.south) -- (sn0x9c8420.north);
\draw (sn0x9c68f0.south) -- (sn0x9c3bb0.north);
\draw (sn0x9c3190.south) -- (sn0x9c2bc0.north);
\draw (sn0x9c3bb0.south) -- (sn0x9c2bc0.north);
\draw (sn0x9c3bb0.south) -- (sn0x9c48e0.north);
\draw (sn0x9c8420.south) -- (sn0x9c48e0.north);
\draw (sn0x9c2bc0.south) -- (sn0x9c4130.north);
\draw (sn0x9c2bc0.south) -- (sn0x9c36c0.north);
\draw (sn0x9c48e0.south) -- (sn0x9c5ae0.north);
\draw (sn0x9c48e0.south) -- (sn0x9c36c0.north);
\draw (sn0x9c4130.south) -- (sn0x9c4350.north);
\draw (sn0x9c36c0.south) -- (sn0x9c4350.north);
\draw (sn0x9c36c0.south) -- (sn0x9c4af0.north);
\draw (sn0x9c5ae0.south) -- (sn0x9c57b0.north);
\draw (sn0x9c5ae0.south) -- (sn0x9c4350.north);
\draw (sn0x9c4350.south) -- (sn0x9c4420.north);
\draw (sn0x9c4350.south) -- (sn0x9c45d0.north);
\draw (sn0x9c4af0.south) -- (sn0x9c45d0.north);
\draw (sn0x9c57b0.south) -- (sn0x9c4420.north);
\draw (sn0x9c4420.south) -- (sn0x9c46e0.north);
\draw (sn0x9c45d0.south) -- (sn0x9c46e0.north);
\draw (sn0x9c46e0.south) -- (sn0x9c47b0.north);
\end{tikzpicture}

%%% Local Variables:
%%% TeX-master: "thesis/thesis.tex"
%%% End: 

% \renewcommand{\leveltopI}{-15cm + \leveltop}
\renewcommand{\leveltopII}{-15cm + \leveltopI}
\renewcommand{\leveltopIII}{-15cm + \leveltopII}
\renewcommand{\leveltopIIII}{-15cm + \leveltopIII}
\renewcommand{\leveltopIIIII}{-15cm + \leveltopIIII}
\renewcommand{\leveltopIIIIII}{-15cm + \leveltopIIIII}
\renewcommand{\leveltopIIIIIII}{-15cm + \leveltopIIIIII}
\renewcommand{\leveltopIIIIIIII}{-15cm + \leveltopIIIIIII}
\renewcommand{\leveltopIIIIIIIII}{-15cm + \leveltopIIIIIIII}
\renewcommand{\leveltopIIIIIIIIII}{-15cm + \leveltopIIIIIIIII}
\begin{tikzpicture}[scale=.2, anchor=south]
\begin{scope}[yshift=\leveltopI cm]
\matrix (line1) [column sep=1cm] {
\node[draw=black, rectangle split,  rectangle split parts=3] (sn0xd9fa30){
\begin{tikzpicture}[scale=.2]
\node[circle, scale=0.75, fill] (tid0) at (4.5,1.5){};
\node[circle, scale=0.75, fill] (tid1) at (2.25,3){};
\node[circle, scale=0.75, fill] (tid4) at (0.75,4.5){};
\node[circle, scale=0.75, fill] (tid5) at (2.25,4.5){};
\node[circle, scale=0.75, fill] (tid6) at (3.75,4.5){};
\draw[](tid1) -- (tid4);
\draw[](tid1) -- (tid5);
\draw[](tid1) -- (tid6);
\node[circle, scale=0.75, fill] (tid2) at (6,3){};
\node[circle, scale=0.75, fill, red] (tid7) at (5.25,4.5){};
\node[circle, scale=0.75, fill, red] (tid8) at (6.75,4.5){};
\draw[](tid2) -- (tid7);
\draw[](tid2) -- (tid8);
\node[circle, scale=0.75, fill] (tid3) at (8.25,3){};
\node[circle, scale=0.75, fill, red] (tid9) at (8.25,4.5){};
\draw[](tid3) -- (tid9);
\draw[](tid0) -- (tid1);
\draw[](tid0) -- (tid2);
\draw[](tid0) -- (tid3);
\end{tikzpicture}
\nodepart{two}
\footnotesize{5.20028}
\nodepart{three}
\footnotesize{$33\:67$}
};
 & 
\\
};
\end{scope}
\begin{scope}[yshift=\leveltopII cm]
\matrix (line2) [column sep=1cm] {
\node[draw=black, rectangle split,  rectangle split parts=3] (sn0xd9fb90){
\begin{tikzpicture}[scale=.2]
\node[circle, scale=0.75, fill] (tid0) at (4.5,1.5){};
\node[circle, scale=0.75, fill] (tid1) at (2.25,3){};
\node[circle, scale=0.75, fill, red] (tid4) at (0.75,4.5){};
\node[circle, scale=0.75, fill] (tid5) at (2.25,4.5){};
\node[circle, scale=0.75, fill] (tid6) at (3.75,4.5){};
\draw[](tid1) -- (tid4);
\draw[](tid1) -- (tid5);
\draw[](tid1) -- (tid6);
\node[circle, scale=0.75, fill] (tid2) at (6,3){};
\node[circle, scale=0.75, fill, red] (tid7) at (5.25,4.5){};
\node[circle, scale=0.75, fill, red] (tid8) at (6.75,4.5){};
\draw[](tid2) -- (tid7);
\draw[](tid2) -- (tid8);
\node[circle, scale=0.75, fill] (tid3) at (8.25,3){};
\draw[](tid0) -- (tid1);
\draw[](tid0) -- (tid2);
\draw[](tid0) -- (tid3);
\end{tikzpicture}
\nodepart{two}
\footnotesize{4.86626}
\nodepart{three}
\footnotesize{$67\:33$}
};
 & 
\node[draw=black, rectangle split,  rectangle split parts=3] (sn0xd9d000){
\begin{tikzpicture}[scale=.2]
\node[circle, scale=0.75, fill] (tid0) at (3.75,1.5){};
\node[circle, scale=0.75, fill] (tid1) at (2.25,3){};
\node[circle, scale=0.75, fill, red] (tid4) at (0.75,4.5){};
\node[circle, scale=0.75, fill] (tid5) at (2.25,4.5){};
\node[circle, scale=0.75, fill] (tid6) at (3.75,4.5){};
\draw[](tid1) -- (tid4);
\draw[](tid1) -- (tid5);
\draw[](tid1) -- (tid6);
\node[circle, scale=0.75, fill] (tid2) at (5.25,3){};
\node[circle, scale=0.75, fill, red] (tid7) at (5.25,4.5){};
\draw[](tid2) -- (tid7);
\node[circle, scale=0.75, fill] (tid3) at (6.75,3){};
\node[circle, scale=0.75, fill, red] (tid8) at (6.75,4.5){};
\draw[](tid3) -- (tid8);
\draw[](tid0) -- (tid1);
\draw[](tid0) -- (tid2);
\draw[](tid0) -- (tid3);
\end{tikzpicture}
\nodepart{two}
\footnotesize{4.86728}
\nodepart{three}
\footnotesize{$67\:33$}
};
 & 
\\
};
\end{scope}
\begin{scope}[yshift=\leveltopIII cm]
\matrix (line3) [column sep=1cm] {
\node[draw=black, rectangle split,  rectangle split parts=3] (sn0xd9cf30){
\begin{tikzpicture}[scale=.2]
\node[circle, scale=0.75, fill] (tid0) at (3.75,1.5){};
\node[circle, scale=0.75, fill] (tid1) at (2.25,3){};
\node[circle, scale=0.75, fill, red] (tid4) at (0.75,4.5){};
\node[circle, scale=0.75, fill, red] (tid5) at (2.25,4.5){};
\node[circle, scale=0.75, fill] (tid6) at (3.75,4.5){};
\draw[](tid1) -- (tid4);
\draw[](tid1) -- (tid5);
\draw[](tid1) -- (tid6);
\node[circle, scale=0.75, fill] (tid2) at (5.25,3){};
\node[circle, scale=0.75, fill, red] (tid7) at (5.25,4.5){};
\draw[](tid2) -- (tid7);
\node[circle, scale=0.75, fill] (tid3) at (6.75,3){};
\draw[](tid0) -- (tid1);
\draw[](tid0) -- (tid2);
\draw[](tid0) -- (tid3);
\end{tikzpicture}
\nodepart{two}
\footnotesize{4.53086}
\nodepart{three}
\footnotesize{$33\:67$}
};
 & 
\node[draw=black, rectangle split,  rectangle split parts=3] (sn0xd9f7a0){
\begin{tikzpicture}[scale=.2]
\node[circle, scale=0.75, fill] (tid0) at (3.75,1.5){};
\node[circle, scale=0.75, fill] (tid1) at (1.5,3){};
\node[circle, scale=0.75, fill, red] (tid4) at (0.75,4.5){};
\node[circle, scale=0.75, fill, red] (tid5) at (2.25,4.5){};
\draw[](tid1) -- (tid4);
\draw[](tid1) -- (tid5);
\node[circle, scale=0.75, fill] (tid2) at (4.5,3){};
\node[circle, scale=0.75, fill, red] (tid6) at (3.75,4.5){};
\node[circle, scale=0.75, fill] (tid7) at (5.25,4.5){};
\draw[](tid2) -- (tid6);
\draw[](tid2) -- (tid7);
\node[circle, scale=0.75, fill] (tid3) at (6.75,3){};
\draw[](tid0) -- (tid1);
\draw[](tid0) -- (tid2);
\draw[](tid0) -- (tid3);
\end{tikzpicture}
\nodepart{two}
\footnotesize{4.53704}
\nodepart{three}
\footnotesize{$1$}
};
 & 
\node[draw=black, rectangle split,  rectangle split parts=3] (sn0xd9a9c0){
\begin{tikzpicture}[scale=.2]
\node[circle, scale=0.75, fill] (tid0) at (3,1.5){};
\node[circle, scale=0.75, fill] (tid1) at (1.5,3){};
\node[circle, scale=0.75, fill, red] (tid4) at (0.75,4.5){};
\node[circle, scale=0.75, fill] (tid5) at (2.25,4.5){};
\draw[](tid1) -- (tid4);
\draw[](tid1) -- (tid5);
\node[circle, scale=0.75, fill] (tid2) at (3.75,3){};
\node[circle, scale=0.75, fill, red] (tid6) at (3.75,4.5){};
\draw[](tid2) -- (tid6);
\node[circle, scale=0.75, fill] (tid3) at (5.25,3){};
\node[circle, scale=0.75, fill, red] (tid7) at (5.25,4.5){};
\draw[](tid3) -- (tid7);
\draw[](tid0) -- (tid1);
\draw[](tid0) -- (tid2);
\draw[](tid0) -- (tid3);
\end{tikzpicture}
\nodepart{two}
\footnotesize{4.54012}
\nodepart{three}
\footnotesize{$67\:33$}
};
 & 
\\
};
\end{scope}
\begin{scope}[yshift=\leveltopIIII cm]
\matrix (line4) [column sep=1cm] {
\node[draw=black, rectangle split,  rectangle split parts=3] (sn0xd9d6c0){
\begin{tikzpicture}[scale=.2]
\node[circle, scale=0.75, fill] (tid0) at (3.75,1.5){};
\node[circle, scale=0.75, fill] (tid1) at (2.25,3){};
\node[circle, scale=0.75, fill, red] (tid4) at (0.75,4.5){};
\node[circle, scale=0.75, fill, red] (tid5) at (2.25,4.5){};
\node[circle, scale=0.75, fill, red] (tid6) at (3.75,4.5){};
\draw[](tid1) -- (tid4);
\draw[](tid1) -- (tid5);
\draw[](tid1) -- (tid6);
\node[circle, scale=0.75, fill] (tid2) at (5.25,3){};
\node[circle, scale=0.75, fill] (tid3) at (6.75,3){};
\draw[](tid0) -- (tid1);
\draw[](tid0) -- (tid2);
\draw[](tid0) -- (tid3);
\end{tikzpicture}
\nodepart{two}
\footnotesize{4.18519}
\nodepart{three}
\footnotesize{$1$}
};
 & 
\node[draw=black, rectangle split,  rectangle split parts=3] (sn0xd9ae50){
\begin{tikzpicture}[scale=.2]
\node[circle, scale=0.75, fill] (tid0) at (3,1.5){};
\node[circle, scale=0.75, fill] (tid1) at (1.5,3){};
\node[circle, scale=0.75, fill, red] (tid4) at (0.75,4.5){};
\node[circle, scale=0.75, fill, red] (tid5) at (2.25,4.5){};
\draw[](tid1) -- (tid4);
\draw[](tid1) -- (tid5);
\node[circle, scale=0.75, fill] (tid2) at (3.75,3){};
\node[circle, scale=0.75, fill, red] (tid6) at (3.75,4.5){};
\draw[](tid2) -- (tid6);
\node[circle, scale=0.75, fill] (tid3) at (5.25,3){};
\draw[](tid0) -- (tid1);
\draw[](tid0) -- (tid2);
\draw[](tid0) -- (tid3);
\end{tikzpicture}
\nodepart{two}
\footnotesize{4.2037}
\nodepart{three}
\footnotesize{$33\:67$}
};
 & 
\node[draw=black, rectangle split,  rectangle split parts=3] (sn0xd98db0){
\begin{tikzpicture}[scale=.2]
\node[circle, scale=0.75, fill] (tid0) at (2.25,1.5){};
\node[circle, scale=0.75, fill] (tid1) at (0.75,3){};
\node[circle, scale=0.75, fill, red] (tid4) at (0.75,4.5){};
\draw[](tid1) -- (tid4);
\node[circle, scale=0.75, fill] (tid2) at (2.25,3){};
\node[circle, scale=0.75, fill, red] (tid5) at (2.25,4.5){};
\draw[](tid2) -- (tid5);
\node[circle, scale=0.75, fill] (tid3) at (3.75,3){};
\node[circle, scale=0.75, fill, red] (tid6) at (3.75,4.5){};
\draw[](tid3) -- (tid6);
\draw[](tid0) -- (tid1);
\draw[](tid0) -- (tid2);
\draw[](tid0) -- (tid3);
\end{tikzpicture}
\nodepart{two}
\footnotesize{4.21296}
\nodepart{three}
\footnotesize{$1$}
};
 & 
\\
};
\end{scope}
\begin{scope}[yshift=\leveltopIIIII cm]
\matrix (line5) [column sep=1cm] {
\node[draw=black, rectangle split,  rectangle split parts=3] (sn0xd9a340){
\begin{tikzpicture}[scale=.2]
\node[circle, scale=0.75, fill] (tid0) at (3,1.5){};
\node[circle, scale=0.75, fill] (tid1) at (1.5,3){};
\node[circle, scale=0.75, fill, red] (tid4) at (0.75,4.5){};
\node[circle, scale=0.75, fill, red] (tid5) at (2.25,4.5){};
\draw[](tid1) -- (tid4);
\draw[](tid1) -- (tid5);
\node[circle, scale=0.75, fill, red] (tid2) at (3.75,3){};
\node[circle, scale=0.75, fill] (tid3) at (5.25,3){};
\draw[](tid0) -- (tid1);
\draw[](tid0) -- (tid2);
\draw[](tid0) -- (tid3);
\end{tikzpicture}
\nodepart{two}
\footnotesize{3.85185}
\nodepart{three}
\footnotesize{$33\:67$}
};
 & 
\node[draw=black, rectangle split,  rectangle split parts=3] (sn0xd98be0){
\begin{tikzpicture}[scale=.2]
\node[circle, scale=0.75, fill] (tid0) at (2.25,1.5){};
\node[circle, scale=0.75, fill] (tid1) at (0.75,3){};
\node[circle, scale=0.75, fill, red] (tid4) at (0.75,4.5){};
\draw[](tid1) -- (tid4);
\node[circle, scale=0.75, fill] (tid2) at (2.25,3){};
\node[circle, scale=0.75, fill, red] (tid5) at (2.25,4.5){};
\draw[](tid2) -- (tid5);
\node[circle, scale=0.75, fill, red] (tid3) at (3.75,3){};
\draw[](tid0) -- (tid1);
\draw[](tid0) -- (tid2);
\draw[](tid0) -- (tid3);
\end{tikzpicture}
\nodepart{two}
\footnotesize{3.87963}
\nodepart{three}
\footnotesize{$67\:33$}
};
 & 
\\
};
\end{scope}
\begin{scope}[yshift=\leveltopIIIIII cm]
\matrix (line6) [column sep=1cm] {
\node[draw=black, rectangle split,  rectangle split parts=3] (sn0xd99950){
\begin{tikzpicture}[scale=.2]
\node[circle, scale=0.75, fill] (tid0) at (2.25,1.5){};
\node[circle, scale=0.75, fill] (tid1) at (1.5,3){};
\node[circle, scale=0.75, fill, red] (tid3) at (0.75,4.5){};
\node[circle, scale=0.75, fill, red] (tid4) at (2.25,4.5){};
\draw[](tid1) -- (tid3);
\draw[](tid1) -- (tid4);
\node[circle, scale=0.75, fill, red] (tid2) at (3.75,3){};
\draw[](tid0) -- (tid1);
\draw[](tid0) -- (tid2);
\end{tikzpicture}
\nodepart{two}
\footnotesize{3.66667}
\nodepart{three}
\footnotesize{$33\:67$}
};
 & 
\node[draw=black, rectangle split,  rectangle split parts=3] (sn0xd98b10){
\begin{tikzpicture}[scale=.2]
\node[circle, scale=0.75, fill] (tid0) at (2.25,1.5){};
\node[circle, scale=0.75, fill] (tid1) at (0.75,3){};
\node[circle, scale=0.75, fill, red] (tid4) at (0.75,4.5){};
\draw[](tid1) -- (tid4);
\node[circle, scale=0.75, fill, red] (tid2) at (2.25,3){};
\node[circle, scale=0.75, fill, red] (tid3) at (3.75,3){};
\draw[](tid0) -- (tid1);
\draw[](tid0) -- (tid2);
\draw[](tid0) -- (tid3);
\end{tikzpicture}
\nodepart{two}
\footnotesize{3.44444}
\nodepart{three}
\footnotesize{$67\:33$}
};
 & 
\node[draw=black, rectangle split,  rectangle split parts=3] (sn0xd982f0){
\begin{tikzpicture}[scale=.2]
\node[circle, scale=0.75, fill] (tid0) at (1.5,1.5){};
\node[circle, scale=0.75, fill] (tid1) at (0.75,3){};
\node[circle, scale=0.75, fill, red] (tid3) at (0.75,4.5){};
\draw[](tid1) -- (tid3);
\node[circle, scale=0.75, fill] (tid2) at (2.25,3){};
\node[circle, scale=0.75, fill, red] (tid4) at (2.25,4.5){};
\draw[](tid2) -- (tid4);
\draw[](tid0) -- (tid1);
\draw[](tid0) -- (tid2);
\end{tikzpicture}
\nodepart{two}
\footnotesize{3.75}
\nodepart{three}
\footnotesize{$1$}
};
 & 
\\
};
\end{scope}
\begin{scope}[yshift=\leveltopIIIIIII cm]
\matrix (line7) [column sep=1cm] {
\node[draw=black, rectangle split,  rectangle split parts=3] (sn0xd99040){
\begin{tikzpicture}[scale=.2]
\node[circle, scale=0.75, fill] (tid0) at (1.5,1.5){};
\node[circle, scale=0.75, fill] (tid1) at (1.5,3){};
\node[circle, scale=0.75, fill, red] (tid2) at (0.75,4.5){};
\node[circle, scale=0.75, fill, red] (tid3) at (2.25,4.5){};
\draw[](tid1) -- (tid2);
\draw[](tid1) -- (tid3);
\draw[](tid0) -- (tid1);
\end{tikzpicture}
\nodepart{two}
\footnotesize{3.5}
\nodepart{three}
\footnotesize{$1$}
};
 & 
\node[draw=black, rectangle split,  rectangle split parts=3] (sn0xd981e0){
\begin{tikzpicture}[scale=.2]
\node[circle, scale=0.75, fill] (tid0) at (1.5,1.5){};
\node[circle, scale=0.75, fill] (tid1) at (0.75,3){};
\node[circle, scale=0.75, fill, red] (tid3) at (0.75,4.5){};
\draw[](tid1) -- (tid3);
\node[circle, scale=0.75, fill, red] (tid2) at (2.25,3){};
\draw[](tid0) -- (tid1);
\draw[](tid0) -- (tid2);
\end{tikzpicture}
\nodepart{two}
\footnotesize{3.25}
\nodepart{three}
\footnotesize{$50\:50$}
};
 & 
\node[draw=black, rectangle split,  rectangle split parts=3] (sn0xd985f0){
\begin{tikzpicture}[scale=.2]
\node[circle, scale=0.75, fill] (tid0) at (2.25,1.5){};
\node[circle, scale=0.75, fill, red] (tid1) at (0.75,3){};
\node[circle, scale=0.75, fill, red] (tid2) at (2.25,3){};
\node[circle, scale=0.75, fill, red] (tid3) at (3.75,3){};
\draw[](tid0) -- (tid1);
\draw[](tid0) -- (tid2);
\draw[](tid0) -- (tid3);
\end{tikzpicture}
\nodepart{two}
\footnotesize{2.83333}
\nodepart{three}
\footnotesize{$1$}
};
 & 
\\
};
\end{scope}
\begin{scope}[yshift=\leveltopIIIIIIII cm]
\matrix (line8) [column sep=1cm] {
\node[draw=black, rectangle split,  rectangle split parts=3] (sn0xd97cb0){
\begin{tikzpicture}[scale=.2]
\node[circle, scale=0.75, fill] (tid0) at (0.75,1.5){};
\node[circle, scale=0.75, fill] (tid1) at (0.75,3){};
\node[circle, scale=0.75, fill, red] (tid2) at (0.75,4.5){};
\draw[](tid1) -- (tid2);
\draw[](tid0) -- (tid1);
\end{tikzpicture}
\nodepart{two}
\footnotesize{3}
\nodepart{three}
\footnotesize{$1$}
};
 & 
\node[draw=black, rectangle split,  rectangle split parts=3] (sn0xd97ee0){
\begin{tikzpicture}[scale=.2]
\node[circle, scale=0.75, fill] (tid0) at (1.5,1.5){};
\node[circle, scale=0.75, fill, red] (tid1) at (0.75,3){};
\node[circle, scale=0.75, fill, red] (tid2) at (2.25,3){};
\draw[](tid0) -- (tid1);
\draw[](tid0) -- (tid2);
\end{tikzpicture}
\nodepart{two}
\footnotesize{2.5}
\nodepart{three}
\footnotesize{$1$}
};
 & 
\\
};
\end{scope}
\begin{scope}[yshift=\leveltopIIIIIIIII cm]
\matrix (line9) [column sep=1cm] {
\node[draw=black, rectangle split,  rectangle split parts=3] (sn0xd88140){
\begin{tikzpicture}[scale=.2]
\node[circle, scale=0.75, fill] (tid0) at (0.75,1.5){};
\node[circle, scale=0.75, fill, red] (tid1) at (0.75,3){};
\draw[](tid0) -- (tid1);
\end{tikzpicture}
\nodepart{two}
\footnotesize{2}
\nodepart{three}
\footnotesize{$1$}
};
 & 
\\
};
\end{scope}
\begin{scope}[yshift=\leveltopIIIIIIIIII cm]
\matrix (line10) [column sep=1cm] {
\node[draw=black, rectangle split,  rectangle split parts=3] (sn0xd88070){
\begin{tikzpicture}[scale=.2]
\node[circle, scale=0.75, fill, red] (tid0) at (0.75,1.5){};
\end{tikzpicture}
\nodepart{two}
\footnotesize{1}
\nodepart{three}
\footnotesize{$$}
};
 & 
\\
};
\end{scope}
\begin{scope}[yshift=\leveltopIIIIIIIIIII cm]
\matrix (line11) [column sep=1cm] {
\\
};
\end{scope}
\draw (sn0xd9fa30.south) -- (sn0xd9fb90.north);
\draw (sn0xd9fa30.south) -- (sn0xd9d000.north);
\draw (sn0xd9fb90.south) -- (sn0xd9cf30.north);
\draw (sn0xd9fb90.south) -- (sn0xd9f7a0.north);
\draw (sn0xd9d000.south) -- (sn0xd9cf30.north);
\draw (sn0xd9d000.south) -- (sn0xd9a9c0.north);
\draw (sn0xd9cf30.south) -- (sn0xd9d6c0.north);
\draw (sn0xd9cf30.south) -- (sn0xd9ae50.north);
\draw (sn0xd9f7a0.south) -- (sn0xd9ae50.north);
\draw (sn0xd9a9c0.south) -- (sn0xd9ae50.north);
\draw (sn0xd9a9c0.south) -- (sn0xd98db0.north);
\draw (sn0xd9d6c0.south) -- (sn0xd9a340.north);
\draw (sn0xd9ae50.south) -- (sn0xd9a340.north);
\draw (sn0xd9ae50.south) -- (sn0xd98be0.north);
\draw (sn0xd98db0.south) -- (sn0xd98be0.north);
\draw (sn0xd9a340.south) -- (sn0xd99950.north);
\draw (sn0xd9a340.south) -- (sn0xd98b10.north);
\draw (sn0xd98be0.south) -- (sn0xd982f0.north);
\draw (sn0xd98be0.south) -- (sn0xd98b10.north);
\draw (sn0xd99950.south) -- (sn0xd99040.north);
\draw (sn0xd99950.south) -- (sn0xd981e0.north);
\draw (sn0xd98b10.south) -- (sn0xd981e0.north);
\draw (sn0xd98b10.south) -- (sn0xd985f0.north);
\draw (sn0xd982f0.south) -- (sn0xd981e0.north);
\draw (sn0xd99040.south) -- (sn0xd97cb0.north);
\draw (sn0xd981e0.south) -- (sn0xd97cb0.north);
\draw (sn0xd981e0.south) -- (sn0xd97ee0.north);
\draw (sn0xd985f0.south) -- (sn0xd97ee0.north);
\draw (sn0xd97cb0.south) -- (sn0xd88140.north);
\draw (sn0xd97ee0.south) -- (sn0xd88140.north);
\draw (sn0xd88140.south) -- (sn0xd88070.north);
\end{tikzpicture}

%%% Local Variables:
%%% TeX-master: "thesis/thesis.tex"
%%% End: 
\renewcommand{\leveltopI}{-15cm + \leveltop}
\renewcommand{\leveltopII}{-15cm + \leveltopI}
\renewcommand{\leveltopIII}{-15cm + \leveltopII}
\renewcommand{\leveltopIIII}{-15cm + \leveltopIII}
\renewcommand{\leveltopIIIII}{-15cm + \leveltopIIII}
\renewcommand{\leveltopIIIIII}{-15cm + \leveltopIIIII}
\renewcommand{\leveltopIIIIIII}{-15cm + \leveltopIIIIII}
\renewcommand{\leveltopIIIIIIII}{-15cm + \leveltopIIIIIII}
\renewcommand{\leveltopIIIIIIIII}{-15cm + \leveltopIIIIIIII}
\renewcommand{\leveltopIIIIIIIIII}{-15cm + \leveltopIIIIIIIII}
\begin{tikzpicture}[scale=.2, anchor=south]
\begin{scope}[yshift=\leveltopI cm]
\matrix (line1) [column sep=1cm] {
\node[draw=black, rectangle split,  rectangle split parts=3] (sn0xda2a50){
\begin{tikzpicture}[scale=.2]
\node[circle, scale=0.75, fill] (tid0) at (4.5,1.5){};
\node[circle, scale=0.75, fill] (tid1) at (2.25,3){};
\node[circle, scale=0.75, fill, red] (tid4) at (0.75,4.5){};
\node[circle, scale=0.75, fill, red] (tid5) at (2.25,4.5){};
\node[circle, scale=0.75, fill] (tid6) at (3.75,4.5){};
\draw[](tid1) -- (tid4);
\draw[](tid1) -- (tid5);
\draw[](tid1) -- (tid6);
\node[circle, scale=0.75, fill] (tid2) at (6,3){};
\node[circle, scale=0.75, fill, red] (tid7) at (5.25,4.5){};
\node[circle, scale=0.75, fill] (tid8) at (6.75,4.5){};
\draw[](tid2) -- (tid7);
\draw[](tid2) -- (tid8);
\node[circle, scale=0.75, fill] (tid3) at (8.25,3){};
\node[circle, scale=0.75, fill] (tid9) at (8.25,4.5){};
\draw[](tid3) -- (tid9);
\draw[](tid0) -- (tid1);
\draw[](tid0) -- (tid2);
\draw[](tid0) -- (tid3);
\end{tikzpicture}
\nodepart{two}
\footnotesize{5.20508}
\nodepart{three}
\footnotesize{$33\:67$}
};
 & 
\\
};
\end{scope}
\begin{scope}[yshift=\leveltopII cm]
\matrix (line2) [column sep=1cm] {
\node[draw=black, rectangle split,  rectangle split parts=3] (sn0xda2090){
\begin{tikzpicture}[scale=.2]
\node[circle, scale=0.75, fill] (tid0) at (3.75,1.5){};
\node[circle, scale=0.75, fill] (tid1) at (2.25,3){};
\node[circle, scale=0.75, fill, red] (tid4) at (0.75,4.5){};
\node[circle, scale=0.75, fill, red] (tid5) at (2.25,4.5){};
\node[circle, scale=0.75, fill] (tid6) at (3.75,4.5){};
\draw[](tid1) -- (tid4);
\draw[](tid1) -- (tid5);
\draw[](tid1) -- (tid6);
\node[circle, scale=0.75, fill] (tid2) at (5.25,3){};
\node[circle, scale=0.75, fill, red] (tid7) at (5.25,4.5){};
\draw[](tid2) -- (tid7);
\node[circle, scale=0.75, fill] (tid3) at (6.75,3){};
\node[circle, scale=0.75, fill] (tid8) at (6.75,4.5){};
\draw[](tid3) -- (tid8);
\draw[](tid0) -- (tid1);
\draw[](tid0) -- (tid2);
\draw[](tid0) -- (tid3);
\end{tikzpicture}
\nodepart{two}
\footnotesize{4.87037}
\nodepart{three}
\footnotesize{$33\:67$}
};
 & 
\node[draw=black, rectangle split,  rectangle split parts=3] (sn0xda0d60){
\begin{tikzpicture}[scale=.2]
\node[circle, scale=0.75, fill] (tid0) at (3.75,1.5){};
\node[circle, scale=0.75, fill] (tid1) at (1.5,3){};
\node[circle, scale=0.75, fill, red] (tid4) at (0.75,4.5){};
\node[circle, scale=0.75, fill] (tid5) at (2.25,4.5){};
\draw[](tid1) -- (tid4);
\draw[](tid1) -- (tid5);
\node[circle, scale=0.75, fill] (tid2) at (4.5,3){};
\node[circle, scale=0.75, fill, red] (tid6) at (3.75,4.5){};
\node[circle, scale=0.75, fill] (tid7) at (5.25,4.5){};
\draw[](tid2) -- (tid6);
\draw[](tid2) -- (tid7);
\node[circle, scale=0.75, fill] (tid3) at (6.75,3){};
\node[circle, scale=0.75, fill, red] (tid8) at (6.75,4.5){};
\draw[](tid3) -- (tid8);
\draw[](tid0) -- (tid1);
\draw[](tid0) -- (tid2);
\draw[](tid0) -- (tid3);
\end{tikzpicture}
\nodepart{two}
\footnotesize{4.87243}
\nodepart{three}
\footnotesize{$67\:33$}
};
 & 
\\
};
\end{scope}
\begin{scope}[yshift=\leveltopIII cm]
\matrix (line3) [column sep=1cm] {
\node[draw=black, rectangle split,  rectangle split parts=3] (sn0xd9cf30){
\begin{tikzpicture}[scale=.2]
\node[circle, scale=0.75, fill] (tid0) at (3.75,1.5){};
\node[circle, scale=0.75, fill] (tid1) at (2.25,3){};
\node[circle, scale=0.75, fill, red] (tid4) at (0.75,4.5){};
\node[circle, scale=0.75, fill, red] (tid5) at (2.25,4.5){};
\node[circle, scale=0.75, fill] (tid6) at (3.75,4.5){};
\draw[](tid1) -- (tid4);
\draw[](tid1) -- (tid5);
\draw[](tid1) -- (tid6);
\node[circle, scale=0.75, fill] (tid2) at (5.25,3){};
\node[circle, scale=0.75, fill, red] (tid7) at (5.25,4.5){};
\draw[](tid2) -- (tid7);
\node[circle, scale=0.75, fill] (tid3) at (6.75,3){};
\draw[](tid0) -- (tid1);
\draw[](tid0) -- (tid2);
\draw[](tid0) -- (tid3);
\end{tikzpicture}
\nodepart{two}
\footnotesize{4.53086}
\nodepart{three}
\footnotesize{$33\:67$}
};
 & 
\node[draw=black, rectangle split,  rectangle split parts=3] (sn0xd9a9c0){
\begin{tikzpicture}[scale=.2]
\node[circle, scale=0.75, fill] (tid0) at (3,1.5){};
\node[circle, scale=0.75, fill] (tid1) at (1.5,3){};
\node[circle, scale=0.75, fill, red] (tid4) at (0.75,4.5){};
\node[circle, scale=0.75, fill] (tid5) at (2.25,4.5){};
\draw[](tid1) -- (tid4);
\draw[](tid1) -- (tid5);
\node[circle, scale=0.75, fill] (tid2) at (3.75,3){};
\node[circle, scale=0.75, fill, red] (tid6) at (3.75,4.5){};
\draw[](tid2) -- (tid6);
\node[circle, scale=0.75, fill] (tid3) at (5.25,3){};
\node[circle, scale=0.75, fill, red] (tid7) at (5.25,4.5){};
\draw[](tid3) -- (tid7);
\draw[](tid0) -- (tid1);
\draw[](tid0) -- (tid2);
\draw[](tid0) -- (tid3);
\end{tikzpicture}
\nodepart{two}
\footnotesize{4.54012}
\nodepart{three}
\footnotesize{$67\:33$}
};
 & 
\node[draw=black, rectangle split,  rectangle split parts=3] (sn0xd9f7a0){
\begin{tikzpicture}[scale=.2]
\node[circle, scale=0.75, fill] (tid0) at (3.75,1.5){};
\node[circle, scale=0.75, fill] (tid1) at (1.5,3){};
\node[circle, scale=0.75, fill, red] (tid4) at (0.75,4.5){};
\node[circle, scale=0.75, fill, red] (tid5) at (2.25,4.5){};
\draw[](tid1) -- (tid4);
\draw[](tid1) -- (tid5);
\node[circle, scale=0.75, fill] (tid2) at (4.5,3){};
\node[circle, scale=0.75, fill, red] (tid6) at (3.75,4.5){};
\node[circle, scale=0.75, fill] (tid7) at (5.25,4.5){};
\draw[](tid2) -- (tid6);
\draw[](tid2) -- (tid7);
\node[circle, scale=0.75, fill] (tid3) at (6.75,3){};
\draw[](tid0) -- (tid1);
\draw[](tid0) -- (tid2);
\draw[](tid0) -- (tid3);
\end{tikzpicture}
\nodepart{two}
\footnotesize{4.53704}
\nodepart{three}
\footnotesize{$1$}
};
 & 
\\
};
\end{scope}
\begin{scope}[yshift=\leveltopIIII cm]
\matrix (line4) [column sep=1cm] {
\node[draw=black, rectangle split,  rectangle split parts=3] (sn0xd9d6c0){
\begin{tikzpicture}[scale=.2]
\node[circle, scale=0.75, fill] (tid0) at (3.75,1.5){};
\node[circle, scale=0.75, fill] (tid1) at (2.25,3){};
\node[circle, scale=0.75, fill, red] (tid4) at (0.75,4.5){};
\node[circle, scale=0.75, fill, red] (tid5) at (2.25,4.5){};
\node[circle, scale=0.75, fill, red] (tid6) at (3.75,4.5){};
\draw[](tid1) -- (tid4);
\draw[](tid1) -- (tid5);
\draw[](tid1) -- (tid6);
\node[circle, scale=0.75, fill] (tid2) at (5.25,3){};
\node[circle, scale=0.75, fill] (tid3) at (6.75,3){};
\draw[](tid0) -- (tid1);
\draw[](tid0) -- (tid2);
\draw[](tid0) -- (tid3);
\end{tikzpicture}
\nodepart{two}
\footnotesize{4.18519}
\nodepart{three}
\footnotesize{$1$}
};
 & 
\node[draw=black, rectangle split,  rectangle split parts=3] (sn0xd9ae50){
\begin{tikzpicture}[scale=.2]
\node[circle, scale=0.75, fill] (tid0) at (3,1.5){};
\node[circle, scale=0.75, fill] (tid1) at (1.5,3){};
\node[circle, scale=0.75, fill, red] (tid4) at (0.75,4.5){};
\node[circle, scale=0.75, fill, red] (tid5) at (2.25,4.5){};
\draw[](tid1) -- (tid4);
\draw[](tid1) -- (tid5);
\node[circle, scale=0.75, fill] (tid2) at (3.75,3){};
\node[circle, scale=0.75, fill, red] (tid6) at (3.75,4.5){};
\draw[](tid2) -- (tid6);
\node[circle, scale=0.75, fill] (tid3) at (5.25,3){};
\draw[](tid0) -- (tid1);
\draw[](tid0) -- (tid2);
\draw[](tid0) -- (tid3);
\end{tikzpicture}
\nodepart{two}
\footnotesize{4.2037}
\nodepart{three}
\footnotesize{$33\:67$}
};
 & 
\node[draw=black, rectangle split,  rectangle split parts=3] (sn0xd98db0){
\begin{tikzpicture}[scale=.2]
\node[circle, scale=0.75, fill] (tid0) at (2.25,1.5){};
\node[circle, scale=0.75, fill] (tid1) at (0.75,3){};
\node[circle, scale=0.75, fill, red] (tid4) at (0.75,4.5){};
\draw[](tid1) -- (tid4);
\node[circle, scale=0.75, fill] (tid2) at (2.25,3){};
\node[circle, scale=0.75, fill, red] (tid5) at (2.25,4.5){};
\draw[](tid2) -- (tid5);
\node[circle, scale=0.75, fill] (tid3) at (3.75,3){};
\node[circle, scale=0.75, fill, red] (tid6) at (3.75,4.5){};
\draw[](tid3) -- (tid6);
\draw[](tid0) -- (tid1);
\draw[](tid0) -- (tid2);
\draw[](tid0) -- (tid3);
\end{tikzpicture}
\nodepart{two}
\footnotesize{4.21296}
\nodepart{three}
\footnotesize{$1$}
};
 & 
\\
};
\end{scope}
\begin{scope}[yshift=\leveltopIIIII cm]
\matrix (line5) [column sep=1cm] {
\node[draw=black, rectangle split,  rectangle split parts=3] (sn0xd9a340){
\begin{tikzpicture}[scale=.2]
\node[circle, scale=0.75, fill] (tid0) at (3,1.5){};
\node[circle, scale=0.75, fill] (tid1) at (1.5,3){};
\node[circle, scale=0.75, fill, red] (tid4) at (0.75,4.5){};
\node[circle, scale=0.75, fill, red] (tid5) at (2.25,4.5){};
\draw[](tid1) -- (tid4);
\draw[](tid1) -- (tid5);
\node[circle, scale=0.75, fill, red] (tid2) at (3.75,3){};
\node[circle, scale=0.75, fill] (tid3) at (5.25,3){};
\draw[](tid0) -- (tid1);
\draw[](tid0) -- (tid2);
\draw[](tid0) -- (tid3);
\end{tikzpicture}
\nodepart{two}
\footnotesize{3.85185}
\nodepart{three}
\footnotesize{$33\:67$}
};
 & 
\node[draw=black, rectangle split,  rectangle split parts=3] (sn0xd98be0){
\begin{tikzpicture}[scale=.2]
\node[circle, scale=0.75, fill] (tid0) at (2.25,1.5){};
\node[circle, scale=0.75, fill] (tid1) at (0.75,3){};
\node[circle, scale=0.75, fill, red] (tid4) at (0.75,4.5){};
\draw[](tid1) -- (tid4);
\node[circle, scale=0.75, fill] (tid2) at (2.25,3){};
\node[circle, scale=0.75, fill, red] (tid5) at (2.25,4.5){};
\draw[](tid2) -- (tid5);
\node[circle, scale=0.75, fill, red] (tid3) at (3.75,3){};
\draw[](tid0) -- (tid1);
\draw[](tid0) -- (tid2);
\draw[](tid0) -- (tid3);
\end{tikzpicture}
\nodepart{two}
\footnotesize{3.87963}
\nodepart{three}
\footnotesize{$67\:33$}
};
 & 
\\
};
\end{scope}
\begin{scope}[yshift=\leveltopIIIIII cm]
\matrix (line6) [column sep=1cm] {
\node[draw=black, rectangle split,  rectangle split parts=3] (sn0xd99950){
\begin{tikzpicture}[scale=.2]
\node[circle, scale=0.75, fill] (tid0) at (2.25,1.5){};
\node[circle, scale=0.75, fill] (tid1) at (1.5,3){};
\node[circle, scale=0.75, fill, red] (tid3) at (0.75,4.5){};
\node[circle, scale=0.75, fill, red] (tid4) at (2.25,4.5){};
\draw[](tid1) -- (tid3);
\draw[](tid1) -- (tid4);
\node[circle, scale=0.75, fill, red] (tid2) at (3.75,3){};
\draw[](tid0) -- (tid1);
\draw[](tid0) -- (tid2);
\end{tikzpicture}
\nodepart{two}
\footnotesize{3.66667}
\nodepart{three}
\footnotesize{$33\:67$}
};
 & 
\node[draw=black, rectangle split,  rectangle split parts=3] (sn0xd98b10){
\begin{tikzpicture}[scale=.2]
\node[circle, scale=0.75, fill] (tid0) at (2.25,1.5){};
\node[circle, scale=0.75, fill] (tid1) at (0.75,3){};
\node[circle, scale=0.75, fill, red] (tid4) at (0.75,4.5){};
\draw[](tid1) -- (tid4);
\node[circle, scale=0.75, fill, red] (tid2) at (2.25,3){};
\node[circle, scale=0.75, fill, red] (tid3) at (3.75,3){};
\draw[](tid0) -- (tid1);
\draw[](tid0) -- (tid2);
\draw[](tid0) -- (tid3);
\end{tikzpicture}
\nodepart{two}
\footnotesize{3.44444}
\nodepart{three}
\footnotesize{$67\:33$}
};
 & 
\node[draw=black, rectangle split,  rectangle split parts=3] (sn0xd982f0){
\begin{tikzpicture}[scale=.2]
\node[circle, scale=0.75, fill] (tid0) at (1.5,1.5){};
\node[circle, scale=0.75, fill] (tid1) at (0.75,3){};
\node[circle, scale=0.75, fill, red] (tid3) at (0.75,4.5){};
\draw[](tid1) -- (tid3);
\node[circle, scale=0.75, fill] (tid2) at (2.25,3){};
\node[circle, scale=0.75, fill, red] (tid4) at (2.25,4.5){};
\draw[](tid2) -- (tid4);
\draw[](tid0) -- (tid1);
\draw[](tid0) -- (tid2);
\end{tikzpicture}
\nodepart{two}
\footnotesize{3.75}
\nodepart{three}
\footnotesize{$1$}
};
 & 
\\
};
\end{scope}
\begin{scope}[yshift=\leveltopIIIIIII cm]
\matrix (line7) [column sep=1cm] {
\node[draw=black, rectangle split,  rectangle split parts=3] (sn0xd99040){
\begin{tikzpicture}[scale=.2]
\node[circle, scale=0.75, fill] (tid0) at (1.5,1.5){};
\node[circle, scale=0.75, fill] (tid1) at (1.5,3){};
\node[circle, scale=0.75, fill, red] (tid2) at (0.75,4.5){};
\node[circle, scale=0.75, fill, red] (tid3) at (2.25,4.5){};
\draw[](tid1) -- (tid2);
\draw[](tid1) -- (tid3);
\draw[](tid0) -- (tid1);
\end{tikzpicture}
\nodepart{two}
\footnotesize{3.5}
\nodepart{three}
\footnotesize{$1$}
};
 & 
\node[draw=black, rectangle split,  rectangle split parts=3] (sn0xd981e0){
\begin{tikzpicture}[scale=.2]
\node[circle, scale=0.75, fill] (tid0) at (1.5,1.5){};
\node[circle, scale=0.75, fill] (tid1) at (0.75,3){};
\node[circle, scale=0.75, fill, red] (tid3) at (0.75,4.5){};
\draw[](tid1) -- (tid3);
\node[circle, scale=0.75, fill, red] (tid2) at (2.25,3){};
\draw[](tid0) -- (tid1);
\draw[](tid0) -- (tid2);
\end{tikzpicture}
\nodepart{two}
\footnotesize{3.25}
\nodepart{three}
\footnotesize{$50\:50$}
};
 & 
\node[draw=black, rectangle split,  rectangle split parts=3] (sn0xd985f0){
\begin{tikzpicture}[scale=.2]
\node[circle, scale=0.75, fill] (tid0) at (2.25,1.5){};
\node[circle, scale=0.75, fill, red] (tid1) at (0.75,3){};
\node[circle, scale=0.75, fill, red] (tid2) at (2.25,3){};
\node[circle, scale=0.75, fill, red] (tid3) at (3.75,3){};
\draw[](tid0) -- (tid1);
\draw[](tid0) -- (tid2);
\draw[](tid0) -- (tid3);
\end{tikzpicture}
\nodepart{two}
\footnotesize{2.83333}
\nodepart{three}
\footnotesize{$1$}
};
 & 
\\
};
\end{scope}
\begin{scope}[yshift=\leveltopIIIIIIII cm]
\matrix (line8) [column sep=1cm] {
\node[draw=black, rectangle split,  rectangle split parts=3] (sn0xd97cb0){
\begin{tikzpicture}[scale=.2]
\node[circle, scale=0.75, fill] (tid0) at (0.75,1.5){};
\node[circle, scale=0.75, fill] (tid1) at (0.75,3){};
\node[circle, scale=0.75, fill, red] (tid2) at (0.75,4.5){};
\draw[](tid1) -- (tid2);
\draw[](tid0) -- (tid1);
\end{tikzpicture}
\nodepart{two}
\footnotesize{3}
\nodepart{three}
\footnotesize{$1$}
};
 & 
\node[draw=black, rectangle split,  rectangle split parts=3] (sn0xd97ee0){
\begin{tikzpicture}[scale=.2]
\node[circle, scale=0.75, fill] (tid0) at (1.5,1.5){};
\node[circle, scale=0.75, fill, red] (tid1) at (0.75,3){};
\node[circle, scale=0.75, fill, red] (tid2) at (2.25,3){};
\draw[](tid0) -- (tid1);
\draw[](tid0) -- (tid2);
\end{tikzpicture}
\nodepart{two}
\footnotesize{2.5}
\nodepart{three}
\footnotesize{$1$}
};
 & 
\\
};
\end{scope}
\begin{scope}[yshift=\leveltopIIIIIIIII cm]
\matrix (line9) [column sep=1cm] {
\node[draw=black, rectangle split,  rectangle split parts=3] (sn0xd88140){
\begin{tikzpicture}[scale=.2]
\node[circle, scale=0.75, fill] (tid0) at (0.75,1.5){};
\node[circle, scale=0.75, fill, red] (tid1) at (0.75,3){};
\draw[](tid0) -- (tid1);
\end{tikzpicture}
\nodepart{two}
\footnotesize{2}
\nodepart{three}
\footnotesize{$1$}
};
 & 
\\
};
\end{scope}
\begin{scope}[yshift=\leveltopIIIIIIIIII cm]
\matrix (line10) [column sep=1cm] {
\node[draw=black, rectangle split,  rectangle split parts=3] (sn0xd88070){
\begin{tikzpicture}[scale=.2]
\node[circle, scale=0.75, fill, red] (tid0) at (0.75,1.5){};
\end{tikzpicture}
\nodepart{two}
\footnotesize{1}
\nodepart{three}
\footnotesize{$$}
};
 & 
\\
};
\end{scope}
\begin{scope}[yshift=\leveltopIIIIIIIIIII cm]
\matrix (line11) [column sep=1cm] {
\\
};
\end{scope}
\draw (sn0xda2a50.south) -- (sn0xda2090.north);
\draw (sn0xda2a50.south) -- (sn0xda0d60.north);
\draw (sn0xda2090.south) -- (sn0xd9cf30.north);
\draw (sn0xda2090.south) -- (sn0xd9a9c0.north);
\draw (sn0xda0d60.south) -- (sn0xd9f7a0.north);
\draw (sn0xda0d60.south) -- (sn0xd9a9c0.north);
\draw (sn0xd9cf30.south) -- (sn0xd9d6c0.north);
\draw (sn0xd9cf30.south) -- (sn0xd9ae50.north);
\draw (sn0xd9a9c0.south) -- (sn0xd9ae50.north);
\draw (sn0xd9a9c0.south) -- (sn0xd98db0.north);
\draw (sn0xd9f7a0.south) -- (sn0xd9ae50.north);
\draw (sn0xd9d6c0.south) -- (sn0xd9a340.north);
\draw (sn0xd9ae50.south) -- (sn0xd9a340.north);
\draw (sn0xd9ae50.south) -- (sn0xd98be0.north);
\draw (sn0xd98db0.south) -- (sn0xd98be0.north);
\draw (sn0xd9a340.south) -- (sn0xd99950.north);
\draw (sn0xd9a340.south) -- (sn0xd98b10.north);
\draw (sn0xd98be0.south) -- (sn0xd982f0.north);
\draw (sn0xd98be0.south) -- (sn0xd98b10.north);
\draw (sn0xd99950.south) -- (sn0xd99040.north);
\draw (sn0xd99950.south) -- (sn0xd981e0.north);
\draw (sn0xd98b10.south) -- (sn0xd981e0.north);
\draw (sn0xd98b10.south) -- (sn0xd985f0.north);
\draw (sn0xd982f0.south) -- (sn0xd981e0.north);
\draw (sn0xd99040.south) -- (sn0xd97cb0.north);
\draw (sn0xd981e0.south) -- (sn0xd97cb0.north);
\draw (sn0xd981e0.south) -- (sn0xd97ee0.north);
\draw (sn0xd985f0.south) -- (sn0xd97ee0.north);
\draw (sn0xd97cb0.south) -- (sn0xd88140.north);
\draw (sn0xd97ee0.south) -- (sn0xd88140.north);
\draw (sn0xd88140.south) -- (sn0xd88070.north);
\end{tikzpicture}

%%% Local Variables:
%%% TeX-master: "thesis/thesis.tex"
%%% End: 
\renewcommand{\leveltopI}{-15cm + \leveltop}
\renewcommand{\leveltopII}{-15cm + \leveltopI}
\renewcommand{\leveltopIII}{-15cm + \leveltopII}
\renewcommand{\leveltopIIII}{-15cm + \leveltopIII}
\renewcommand{\leveltopIIIII}{-15cm + \leveltopIIII}
\renewcommand{\leveltopIIIIII}{-15cm + \leveltopIIIII}
\renewcommand{\leveltopIIIIIII}{-15cm + \leveltopIIIIII}
\renewcommand{\leveltopIIIIIIII}{-15cm + \leveltopIIIIIII}
\renewcommand{\leveltopIIIIIIIII}{-15cm + \leveltopIIIIIIII}
\renewcommand{\leveltopIIIIIIIIII}{-15cm + \leveltopIIIIIIIII}
\begin{tikzpicture}[scale=.2, anchor=south]
\begin{scope}[yshift=\leveltopI cm]
\matrix (line1) [column sep=1cm] {
\node[draw=black, rectangle split,  rectangle split parts=3] (sn0xda3200){
\begin{tikzpicture}[scale=.2]
\node[circle, scale=0.75, fill] (tid0) at (4.5,1.5){};
\node[circle, scale=0.75, fill] (tid1) at (2.25,3){};
\node[circle, scale=0.75, fill, red] (tid4) at (0.75,4.5){};
\node[circle, scale=0.75, fill] (tid5) at (2.25,4.5){};
\node[circle, scale=0.75, fill] (tid6) at (3.75,4.5){};
\draw[](tid1) -- (tid4);
\draw[](tid1) -- (tid5);
\draw[](tid1) -- (tid6);
\node[circle, scale=0.75, fill] (tid2) at (6,3){};
\node[circle, scale=0.75, fill, red] (tid7) at (5.25,4.5){};
\node[circle, scale=0.75, fill, red] (tid8) at (6.75,4.5){};
\draw[](tid2) -- (tid7);
\draw[](tid2) -- (tid8);
\node[circle, scale=0.75, fill] (tid3) at (8.25,3){};
\node[circle, scale=0.75, fill] (tid9) at (8.25,4.5){};
\draw[](tid3) -- (tid9);
\draw[](tid0) -- (tid1);
\draw[](tid0) -- (tid2);
\draw[](tid0) -- (tid3);
\end{tikzpicture}
\nodepart{two}
\footnotesize{5.20233}
\nodepart{three}
\footnotesize{$67\:33$}
};
 & 
\\
};
\end{scope}
\begin{scope}[yshift=\leveltopII cm]
\matrix (line2) [column sep=1cm] {
\node[draw=black, rectangle split,  rectangle split parts=3] (sn0xd9d000){
\begin{tikzpicture}[scale=.2]
\node[circle, scale=0.75, fill] (tid0) at (3.75,1.5){};
\node[circle, scale=0.75, fill] (tid1) at (2.25,3){};
\node[circle, scale=0.75, fill, red] (tid4) at (0.75,4.5){};
\node[circle, scale=0.75, fill] (tid5) at (2.25,4.5){};
\node[circle, scale=0.75, fill] (tid6) at (3.75,4.5){};
\draw[](tid1) -- (tid4);
\draw[](tid1) -- (tid5);
\draw[](tid1) -- (tid6);
\node[circle, scale=0.75, fill] (tid2) at (5.25,3){};
\node[circle, scale=0.75, fill, red] (tid7) at (5.25,4.5){};
\draw[](tid2) -- (tid7);
\node[circle, scale=0.75, fill] (tid3) at (6.75,3){};
\node[circle, scale=0.75, fill, red] (tid8) at (6.75,4.5){};
\draw[](tid3) -- (tid8);
\draw[](tid0) -- (tid1);
\draw[](tid0) -- (tid2);
\draw[](tid0) -- (tid3);
\end{tikzpicture}
\nodepart{two}
\footnotesize{4.86728}
\nodepart{three}
\footnotesize{$67\:33$}
};
 & 
\node[draw=black, rectangle split,  rectangle split parts=3] (sn0xda2f30){
\begin{tikzpicture}[scale=.2]
\node[circle, scale=0.75, fill] (tid0) at (3.75,1.5){};
\node[circle, scale=0.75, fill] (tid1) at (1.5,3){};
\node[circle, scale=0.75, fill, red] (tid4) at (0.75,4.5){};
\node[circle, scale=0.75, fill, red] (tid5) at (2.25,4.5){};
\draw[](tid1) -- (tid4);
\draw[](tid1) -- (tid5);
\node[circle, scale=0.75, fill] (tid2) at (4.5,3){};
\node[circle, scale=0.75, fill] (tid6) at (3.75,4.5){};
\node[circle, scale=0.75, fill] (tid7) at (5.25,4.5){};
\draw[](tid2) -- (tid6);
\draw[](tid2) -- (tid7);
\node[circle, scale=0.75, fill] (tid3) at (6.75,3){};
\node[circle, scale=0.75, fill, red] (tid8) at (6.75,4.5){};
\draw[](tid3) -- (tid8);
\draw[](tid0) -- (tid1);
\draw[](tid0) -- (tid2);
\draw[](tid0) -- (tid3);
\end{tikzpicture}
\nodepart{two}
\footnotesize{4.87243}
\nodepart{three}
\footnotesize{$67\:33$}
};
 & 
\\
};
\end{scope}
\begin{scope}[yshift=\leveltopIII cm]
\matrix (line3) [column sep=1cm] {
\node[draw=black, rectangle split,  rectangle split parts=3] (sn0xd9cf30){
\begin{tikzpicture}[scale=.2]
\node[circle, scale=0.75, fill] (tid0) at (3.75,1.5){};
\node[circle, scale=0.75, fill] (tid1) at (2.25,3){};
\node[circle, scale=0.75, fill, red] (tid4) at (0.75,4.5){};
\node[circle, scale=0.75, fill, red] (tid5) at (2.25,4.5){};
\node[circle, scale=0.75, fill] (tid6) at (3.75,4.5){};
\draw[](tid1) -- (tid4);
\draw[](tid1) -- (tid5);
\draw[](tid1) -- (tid6);
\node[circle, scale=0.75, fill] (tid2) at (5.25,3){};
\node[circle, scale=0.75, fill, red] (tid7) at (5.25,4.5){};
\draw[](tid2) -- (tid7);
\node[circle, scale=0.75, fill] (tid3) at (6.75,3){};
\draw[](tid0) -- (tid1);
\draw[](tid0) -- (tid2);
\draw[](tid0) -- (tid3);
\end{tikzpicture}
\nodepart{two}
\footnotesize{4.53086}
\nodepart{three}
\footnotesize{$33\:67$}
};
 & 
\node[draw=black, rectangle split,  rectangle split parts=3] (sn0xd9a9c0){
\begin{tikzpicture}[scale=.2]
\node[circle, scale=0.75, fill] (tid0) at (3,1.5){};
\node[circle, scale=0.75, fill] (tid1) at (1.5,3){};
\node[circle, scale=0.75, fill, red] (tid4) at (0.75,4.5){};
\node[circle, scale=0.75, fill] (tid5) at (2.25,4.5){};
\draw[](tid1) -- (tid4);
\draw[](tid1) -- (tid5);
\node[circle, scale=0.75, fill] (tid2) at (3.75,3){};
\node[circle, scale=0.75, fill, red] (tid6) at (3.75,4.5){};
\draw[](tid2) -- (tid6);
\node[circle, scale=0.75, fill] (tid3) at (5.25,3){};
\node[circle, scale=0.75, fill, red] (tid7) at (5.25,4.5){};
\draw[](tid3) -- (tid7);
\draw[](tid0) -- (tid1);
\draw[](tid0) -- (tid2);
\draw[](tid0) -- (tid3);
\end{tikzpicture}
\nodepart{two}
\footnotesize{4.54012}
\nodepart{three}
\footnotesize{$67\:33$}
};
 & 
\node[draw=black, rectangle split,  rectangle split parts=3] (sn0xd9f7a0){
\begin{tikzpicture}[scale=.2]
\node[circle, scale=0.75, fill] (tid0) at (3.75,1.5){};
\node[circle, scale=0.75, fill] (tid1) at (1.5,3){};
\node[circle, scale=0.75, fill, red] (tid4) at (0.75,4.5){};
\node[circle, scale=0.75, fill, red] (tid5) at (2.25,4.5){};
\draw[](tid1) -- (tid4);
\draw[](tid1) -- (tid5);
\node[circle, scale=0.75, fill] (tid2) at (4.5,3){};
\node[circle, scale=0.75, fill, red] (tid6) at (3.75,4.5){};
\node[circle, scale=0.75, fill] (tid7) at (5.25,4.5){};
\draw[](tid2) -- (tid6);
\draw[](tid2) -- (tid7);
\node[circle, scale=0.75, fill] (tid3) at (6.75,3){};
\draw[](tid0) -- (tid1);
\draw[](tid0) -- (tid2);
\draw[](tid0) -- (tid3);
\end{tikzpicture}
\nodepart{two}
\footnotesize{4.53704}
\nodepart{three}
\footnotesize{$1$}
};
 & 
\\
};
\end{scope}
\begin{scope}[yshift=\leveltopIIII cm]
\matrix (line4) [column sep=1cm] {
\node[draw=black, rectangle split,  rectangle split parts=3] (sn0xd9d6c0){
\begin{tikzpicture}[scale=.2]
\node[circle, scale=0.75, fill] (tid0) at (3.75,1.5){};
\node[circle, scale=0.75, fill] (tid1) at (2.25,3){};
\node[circle, scale=0.75, fill, red] (tid4) at (0.75,4.5){};
\node[circle, scale=0.75, fill, red] (tid5) at (2.25,4.5){};
\node[circle, scale=0.75, fill, red] (tid6) at (3.75,4.5){};
\draw[](tid1) -- (tid4);
\draw[](tid1) -- (tid5);
\draw[](tid1) -- (tid6);
\node[circle, scale=0.75, fill] (tid2) at (5.25,3){};
\node[circle, scale=0.75, fill] (tid3) at (6.75,3){};
\draw[](tid0) -- (tid1);
\draw[](tid0) -- (tid2);
\draw[](tid0) -- (tid3);
\end{tikzpicture}
\nodepart{two}
\footnotesize{4.18519}
\nodepart{three}
\footnotesize{$1$}
};
 & 
\node[draw=black, rectangle split,  rectangle split parts=3] (sn0xd9ae50){
\begin{tikzpicture}[scale=.2]
\node[circle, scale=0.75, fill] (tid0) at (3,1.5){};
\node[circle, scale=0.75, fill] (tid1) at (1.5,3){};
\node[circle, scale=0.75, fill, red] (tid4) at (0.75,4.5){};
\node[circle, scale=0.75, fill, red] (tid5) at (2.25,4.5){};
\draw[](tid1) -- (tid4);
\draw[](tid1) -- (tid5);
\node[circle, scale=0.75, fill] (tid2) at (3.75,3){};
\node[circle, scale=0.75, fill, red] (tid6) at (3.75,4.5){};
\draw[](tid2) -- (tid6);
\node[circle, scale=0.75, fill] (tid3) at (5.25,3){};
\draw[](tid0) -- (tid1);
\draw[](tid0) -- (tid2);
\draw[](tid0) -- (tid3);
\end{tikzpicture}
\nodepart{two}
\footnotesize{4.2037}
\nodepart{three}
\footnotesize{$33\:67$}
};
 & 
\node[draw=black, rectangle split,  rectangle split parts=3] (sn0xd98db0){
\begin{tikzpicture}[scale=.2]
\node[circle, scale=0.75, fill] (tid0) at (2.25,1.5){};
\node[circle, scale=0.75, fill] (tid1) at (0.75,3){};
\node[circle, scale=0.75, fill, red] (tid4) at (0.75,4.5){};
\draw[](tid1) -- (tid4);
\node[circle, scale=0.75, fill] (tid2) at (2.25,3){};
\node[circle, scale=0.75, fill, red] (tid5) at (2.25,4.5){};
\draw[](tid2) -- (tid5);
\node[circle, scale=0.75, fill] (tid3) at (3.75,3){};
\node[circle, scale=0.75, fill, red] (tid6) at (3.75,4.5){};
\draw[](tid3) -- (tid6);
\draw[](tid0) -- (tid1);
\draw[](tid0) -- (tid2);
\draw[](tid0) -- (tid3);
\end{tikzpicture}
\nodepart{two}
\footnotesize{4.21296}
\nodepart{three}
\footnotesize{$1$}
};
 & 
\\
};
\end{scope}
\begin{scope}[yshift=\leveltopIIIII cm]
\matrix (line5) [column sep=1cm] {
\node[draw=black, rectangle split,  rectangle split parts=3] (sn0xd9a340){
\begin{tikzpicture}[scale=.2]
\node[circle, scale=0.75, fill] (tid0) at (3,1.5){};
\node[circle, scale=0.75, fill] (tid1) at (1.5,3){};
\node[circle, scale=0.75, fill, red] (tid4) at (0.75,4.5){};
\node[circle, scale=0.75, fill, red] (tid5) at (2.25,4.5){};
\draw[](tid1) -- (tid4);
\draw[](tid1) -- (tid5);
\node[circle, scale=0.75, fill, red] (tid2) at (3.75,3){};
\node[circle, scale=0.75, fill] (tid3) at (5.25,3){};
\draw[](tid0) -- (tid1);
\draw[](tid0) -- (tid2);
\draw[](tid0) -- (tid3);
\end{tikzpicture}
\nodepart{two}
\footnotesize{3.85185}
\nodepart{three}
\footnotesize{$33\:67$}
};
 & 
\node[draw=black, rectangle split,  rectangle split parts=3] (sn0xd98be0){
\begin{tikzpicture}[scale=.2]
\node[circle, scale=0.75, fill] (tid0) at (2.25,1.5){};
\node[circle, scale=0.75, fill] (tid1) at (0.75,3){};
\node[circle, scale=0.75, fill, red] (tid4) at (0.75,4.5){};
\draw[](tid1) -- (tid4);
\node[circle, scale=0.75, fill] (tid2) at (2.25,3){};
\node[circle, scale=0.75, fill, red] (tid5) at (2.25,4.5){};
\draw[](tid2) -- (tid5);
\node[circle, scale=0.75, fill, red] (tid3) at (3.75,3){};
\draw[](tid0) -- (tid1);
\draw[](tid0) -- (tid2);
\draw[](tid0) -- (tid3);
\end{tikzpicture}
\nodepart{two}
\footnotesize{3.87963}
\nodepart{three}
\footnotesize{$67\:33$}
};
 & 
\\
};
\end{scope}
\begin{scope}[yshift=\leveltopIIIIII cm]
\matrix (line6) [column sep=1cm] {
\node[draw=black, rectangle split,  rectangle split parts=3] (sn0xd99950){
\begin{tikzpicture}[scale=.2]
\node[circle, scale=0.75, fill] (tid0) at (2.25,1.5){};
\node[circle, scale=0.75, fill] (tid1) at (1.5,3){};
\node[circle, scale=0.75, fill, red] (tid3) at (0.75,4.5){};
\node[circle, scale=0.75, fill, red] (tid4) at (2.25,4.5){};
\draw[](tid1) -- (tid3);
\draw[](tid1) -- (tid4);
\node[circle, scale=0.75, fill, red] (tid2) at (3.75,3){};
\draw[](tid0) -- (tid1);
\draw[](tid0) -- (tid2);
\end{tikzpicture}
\nodepart{two}
\footnotesize{3.66667}
\nodepart{three}
\footnotesize{$33\:67$}
};
 & 
\node[draw=black, rectangle split,  rectangle split parts=3] (sn0xd98b10){
\begin{tikzpicture}[scale=.2]
\node[circle, scale=0.75, fill] (tid0) at (2.25,1.5){};
\node[circle, scale=0.75, fill] (tid1) at (0.75,3){};
\node[circle, scale=0.75, fill, red] (tid4) at (0.75,4.5){};
\draw[](tid1) -- (tid4);
\node[circle, scale=0.75, fill, red] (tid2) at (2.25,3){};
\node[circle, scale=0.75, fill, red] (tid3) at (3.75,3){};
\draw[](tid0) -- (tid1);
\draw[](tid0) -- (tid2);
\draw[](tid0) -- (tid3);
\end{tikzpicture}
\nodepart{two}
\footnotesize{3.44444}
\nodepart{three}
\footnotesize{$67\:33$}
};
 & 
\node[draw=black, rectangle split,  rectangle split parts=3] (sn0xd982f0){
\begin{tikzpicture}[scale=.2]
\node[circle, scale=0.75, fill] (tid0) at (1.5,1.5){};
\node[circle, scale=0.75, fill] (tid1) at (0.75,3){};
\node[circle, scale=0.75, fill, red] (tid3) at (0.75,4.5){};
\draw[](tid1) -- (tid3);
\node[circle, scale=0.75, fill] (tid2) at (2.25,3){};
\node[circle, scale=0.75, fill, red] (tid4) at (2.25,4.5){};
\draw[](tid2) -- (tid4);
\draw[](tid0) -- (tid1);
\draw[](tid0) -- (tid2);
\end{tikzpicture}
\nodepart{two}
\footnotesize{3.75}
\nodepart{three}
\footnotesize{$1$}
};
 & 
\\
};
\end{scope}
\begin{scope}[yshift=\leveltopIIIIIII cm]
\matrix (line7) [column sep=1cm] {
\node[draw=black, rectangle split,  rectangle split parts=3] (sn0xd99040){
\begin{tikzpicture}[scale=.2]
\node[circle, scale=0.75, fill] (tid0) at (1.5,1.5){};
\node[circle, scale=0.75, fill] (tid1) at (1.5,3){};
\node[circle, scale=0.75, fill, red] (tid2) at (0.75,4.5){};
\node[circle, scale=0.75, fill, red] (tid3) at (2.25,4.5){};
\draw[](tid1) -- (tid2);
\draw[](tid1) -- (tid3);
\draw[](tid0) -- (tid1);
\end{tikzpicture}
\nodepart{two}
\footnotesize{3.5}
\nodepart{three}
\footnotesize{$1$}
};
 & 
\node[draw=black, rectangle split,  rectangle split parts=3] (sn0xd981e0){
\begin{tikzpicture}[scale=.2]
\node[circle, scale=0.75, fill] (tid0) at (1.5,1.5){};
\node[circle, scale=0.75, fill] (tid1) at (0.75,3){};
\node[circle, scale=0.75, fill, red] (tid3) at (0.75,4.5){};
\draw[](tid1) -- (tid3);
\node[circle, scale=0.75, fill, red] (tid2) at (2.25,3){};
\draw[](tid0) -- (tid1);
\draw[](tid0) -- (tid2);
\end{tikzpicture}
\nodepart{two}
\footnotesize{3.25}
\nodepart{three}
\footnotesize{$50\:50$}
};
 & 
\node[draw=black, rectangle split,  rectangle split parts=3] (sn0xd985f0){
\begin{tikzpicture}[scale=.2]
\node[circle, scale=0.75, fill] (tid0) at (2.25,1.5){};
\node[circle, scale=0.75, fill, red] (tid1) at (0.75,3){};
\node[circle, scale=0.75, fill, red] (tid2) at (2.25,3){};
\node[circle, scale=0.75, fill, red] (tid3) at (3.75,3){};
\draw[](tid0) -- (tid1);
\draw[](tid0) -- (tid2);
\draw[](tid0) -- (tid3);
\end{tikzpicture}
\nodepart{two}
\footnotesize{2.83333}
\nodepart{three}
\footnotesize{$1$}
};
 & 
\\
};
\end{scope}
\begin{scope}[yshift=\leveltopIIIIIIII cm]
\matrix (line8) [column sep=1cm] {
\node[draw=black, rectangle split,  rectangle split parts=3] (sn0xd97cb0){
\begin{tikzpicture}[scale=.2]
\node[circle, scale=0.75, fill] (tid0) at (0.75,1.5){};
\node[circle, scale=0.75, fill] (tid1) at (0.75,3){};
\node[circle, scale=0.75, fill, red] (tid2) at (0.75,4.5){};
\draw[](tid1) -- (tid2);
\draw[](tid0) -- (tid1);
\end{tikzpicture}
\nodepart{two}
\footnotesize{3}
\nodepart{three}
\footnotesize{$1$}
};
 & 
\node[draw=black, rectangle split,  rectangle split parts=3] (sn0xd97ee0){
\begin{tikzpicture}[scale=.2]
\node[circle, scale=0.75, fill] (tid0) at (1.5,1.5){};
\node[circle, scale=0.75, fill, red] (tid1) at (0.75,3){};
\node[circle, scale=0.75, fill, red] (tid2) at (2.25,3){};
\draw[](tid0) -- (tid1);
\draw[](tid0) -- (tid2);
\end{tikzpicture}
\nodepart{two}
\footnotesize{2.5}
\nodepart{three}
\footnotesize{$1$}
};
 & 
\\
};
\end{scope}
\begin{scope}[yshift=\leveltopIIIIIIIII cm]
\matrix (line9) [column sep=1cm] {
\node[draw=black, rectangle split,  rectangle split parts=3] (sn0xd88140){
\begin{tikzpicture}[scale=.2]
\node[circle, scale=0.75, fill] (tid0) at (0.75,1.5){};
\node[circle, scale=0.75, fill, red] (tid1) at (0.75,3){};
\draw[](tid0) -- (tid1);
\end{tikzpicture}
\nodepart{two}
\footnotesize{2}
\nodepart{three}
\footnotesize{$1$}
};
 & 
\\
};
\end{scope}
\begin{scope}[yshift=\leveltopIIIIIIIIII cm]
\matrix (line10) [column sep=1cm] {
\node[draw=black, rectangle split,  rectangle split parts=3] (sn0xd88070){
\begin{tikzpicture}[scale=.2]
\node[circle, scale=0.75, fill, red] (tid0) at (0.75,1.5){};
\end{tikzpicture}
\nodepart{two}
\footnotesize{1}
\nodepart{three}
\footnotesize{$$}
};
 & 
\\
};
\end{scope}
\begin{scope}[yshift=\leveltopIIIIIIIIIII cm]
\matrix (line11) [column sep=1cm] {
\\
};
\end{scope}
\draw (sn0xda3200.south) -- (sn0xd9d000.north);
\draw (sn0xda3200.south) -- (sn0xda2f30.north);
\draw (sn0xd9d000.south) -- (sn0xd9cf30.north);
\draw (sn0xd9d000.south) -- (sn0xd9a9c0.north);
\draw (sn0xda2f30.south) -- (sn0xd9f7a0.north);
\draw (sn0xda2f30.south) -- (sn0xd9a9c0.north);
\draw (sn0xd9cf30.south) -- (sn0xd9d6c0.north);
\draw (sn0xd9cf30.south) -- (sn0xd9ae50.north);
\draw (sn0xd9a9c0.south) -- (sn0xd9ae50.north);
\draw (sn0xd9a9c0.south) -- (sn0xd98db0.north);
\draw (sn0xd9f7a0.south) -- (sn0xd9ae50.north);
\draw (sn0xd9d6c0.south) -- (sn0xd9a340.north);
\draw (sn0xd9ae50.south) -- (sn0xd9a340.north);
\draw (sn0xd9ae50.south) -- (sn0xd98be0.north);
\draw (sn0xd98db0.south) -- (sn0xd98be0.north);
\draw (sn0xd9a340.south) -- (sn0xd99950.north);
\draw (sn0xd9a340.south) -- (sn0xd98b10.north);
\draw (sn0xd98be0.south) -- (sn0xd982f0.north);
\draw (sn0xd98be0.south) -- (sn0xd98b10.north);
\draw (sn0xd99950.south) -- (sn0xd99040.north);
\draw (sn0xd99950.south) -- (sn0xd981e0.north);
\draw (sn0xd98b10.south) -- (sn0xd981e0.north);
\draw (sn0xd98b10.south) -- (sn0xd985f0.north);
\draw (sn0xd982f0.south) -- (sn0xd981e0.north);
\draw (sn0xd99040.south) -- (sn0xd97cb0.north);
\draw (sn0xd981e0.south) -- (sn0xd97cb0.north);
\draw (sn0xd981e0.south) -- (sn0xd97ee0.north);
\draw (sn0xd985f0.south) -- (sn0xd97ee0.north);
\draw (sn0xd97cb0.south) -- (sn0xd88140.north);
\draw (sn0xd97ee0.south) -- (sn0xd88140.north);
\draw (sn0xd88140.south) -- (sn0xd88070.north);
\end{tikzpicture}

%%% Local Variables:
%%% TeX-master: "thesis/thesis.tex"
%%% End: 
\renewcommand{\leveltopI}{-15cm + \leveltop}
\renewcommand{\leveltopII}{-15cm + \leveltopI}
\renewcommand{\leveltopIII}{-15cm + \leveltopII}
\renewcommand{\leveltopIIII}{-15cm + \leveltopIII}
\renewcommand{\leveltopIIIII}{-15cm + \leveltopIIII}
\renewcommand{\leveltopIIIIII}{-15cm + \leveltopIIIII}
\renewcommand{\leveltopIIIIIII}{-15cm + \leveltopIIIIII}
\renewcommand{\leveltopIIIIIIII}{-15cm + \leveltopIIIIIII}
\renewcommand{\leveltopIIIIIIIII}{-15cm + \leveltopIIIIIIII}
\renewcommand{\leveltopIIIIIIIIII}{-15cm + \leveltopIIIIIIIII}
\begin{tikzpicture}[scale=.2, anchor=south]
\begin{scope}[yshift=\leveltopI cm]
\matrix (line1) [column sep=1cm] {
\node[draw=black, rectangle split,  rectangle split parts=3] (sn0xda3ca0){
\begin{tikzpicture}[scale=.2]
\node[circle, scale=0.75, fill] (tid0) at (4.5,1.5){};
\node[circle, scale=0.75, fill] (tid1) at (2.25,3){};
\node[circle, scale=0.75, fill, red] (tid4) at (0.75,4.5){};
\node[circle, scale=0.75, fill, red] (tid5) at (2.25,4.5){};
\node[circle, scale=0.75, fill, red] (tid6) at (3.75,4.5){};
\draw[](tid1) -- (tid4);
\draw[](tid1) -- (tid5);
\draw[](tid1) -- (tid6);
\node[circle, scale=0.75, fill] (tid2) at (6,3){};
\node[circle, scale=0.75, fill] (tid7) at (5.25,4.5){};
\node[circle, scale=0.75, fill] (tid8) at (6.75,4.5){};
\draw[](tid2) -- (tid7);
\draw[](tid2) -- (tid8);
\node[circle, scale=0.75, fill] (tid3) at (8.25,3){};
\node[circle, scale=0.75, fill] (tid9) at (8.25,4.5){};
\draw[](tid3) -- (tid9);
\draw[](tid0) -- (tid1);
\draw[](tid0) -- (tid2);
\draw[](tid0) -- (tid3);
\end{tikzpicture}
\nodepart{two}
\footnotesize{5.20576}
\nodepart{three}
\footnotesize{$1$}
};
 & 
\\
};
\end{scope}
\begin{scope}[yshift=\leveltopII cm]
\matrix (line2) [column sep=1cm] {
\node[draw=black, rectangle split,  rectangle split parts=3] (sn0xda2f30){
\begin{tikzpicture}[scale=.2]
\node[circle, scale=0.75, fill] (tid0) at (3.75,1.5){};
\node[circle, scale=0.75, fill] (tid1) at (1.5,3){};
\node[circle, scale=0.75, fill, red] (tid4) at (0.75,4.5){};
\node[circle, scale=0.75, fill, red] (tid5) at (2.25,4.5){};
\draw[](tid1) -- (tid4);
\draw[](tid1) -- (tid5);
\node[circle, scale=0.75, fill] (tid2) at (4.5,3){};
\node[circle, scale=0.75, fill] (tid6) at (3.75,4.5){};
\node[circle, scale=0.75, fill] (tid7) at (5.25,4.5){};
\draw[](tid2) -- (tid6);
\draw[](tid2) -- (tid7);
\node[circle, scale=0.75, fill] (tid3) at (6.75,3){};
\node[circle, scale=0.75, fill, red] (tid8) at (6.75,4.5){};
\draw[](tid3) -- (tid8);
\draw[](tid0) -- (tid1);
\draw[](tid0) -- (tid2);
\draw[](tid0) -- (tid3);
\end{tikzpicture}
\nodepart{two}
\footnotesize{4.87243}
\nodepart{three}
\footnotesize{$33\:67$}
};
 & 
\\
};
\end{scope}
\begin{scope}[yshift=\leveltopIII cm]
\matrix (line3) [column sep=1cm] {
\node[draw=black, rectangle split,  rectangle split parts=3] (sn0xd9f7a0){
\begin{tikzpicture}[scale=.2]
\node[circle, scale=0.75, fill] (tid0) at (3.75,1.5){};
\node[circle, scale=0.75, fill] (tid1) at (1.5,3){};
\node[circle, scale=0.75, fill, red] (tid4) at (0.75,4.5){};
\node[circle, scale=0.75, fill, red] (tid5) at (2.25,4.5){};
\draw[](tid1) -- (tid4);
\draw[](tid1) -- (tid5);
\node[circle, scale=0.75, fill] (tid2) at (4.5,3){};
\node[circle, scale=0.75, fill, red] (tid6) at (3.75,4.5){};
\node[circle, scale=0.75, fill] (tid7) at (5.25,4.5){};
\draw[](tid2) -- (tid6);
\draw[](tid2) -- (tid7);
\node[circle, scale=0.75, fill] (tid3) at (6.75,3){};
\draw[](tid0) -- (tid1);
\draw[](tid0) -- (tid2);
\draw[](tid0) -- (tid3);
\end{tikzpicture}
\nodepart{two}
\footnotesize{4.53704}
\nodepart{three}
\footnotesize{$1$}
};
 & 
\node[draw=black, rectangle split,  rectangle split parts=3] (sn0xd9a9c0){
\begin{tikzpicture}[scale=.2]
\node[circle, scale=0.75, fill] (tid0) at (3,1.5){};
\node[circle, scale=0.75, fill] (tid1) at (1.5,3){};
\node[circle, scale=0.75, fill, red] (tid4) at (0.75,4.5){};
\node[circle, scale=0.75, fill] (tid5) at (2.25,4.5){};
\draw[](tid1) -- (tid4);
\draw[](tid1) -- (tid5);
\node[circle, scale=0.75, fill] (tid2) at (3.75,3){};
\node[circle, scale=0.75, fill, red] (tid6) at (3.75,4.5){};
\draw[](tid2) -- (tid6);
\node[circle, scale=0.75, fill] (tid3) at (5.25,3){};
\node[circle, scale=0.75, fill, red] (tid7) at (5.25,4.5){};
\draw[](tid3) -- (tid7);
\draw[](tid0) -- (tid1);
\draw[](tid0) -- (tid2);
\draw[](tid0) -- (tid3);
\end{tikzpicture}
\nodepart{two}
\footnotesize{4.54012}
\nodepart{three}
\footnotesize{$67\:33$}
};
 & 
\\
};
\end{scope}
\begin{scope}[yshift=\leveltopIIII cm]
\matrix (line4) [column sep=1cm] {
\node[draw=black, rectangle split,  rectangle split parts=3] (sn0xd9ae50){
\begin{tikzpicture}[scale=.2]
\node[circle, scale=0.75, fill] (tid0) at (3,1.5){};
\node[circle, scale=0.75, fill] (tid1) at (1.5,3){};
\node[circle, scale=0.75, fill, red] (tid4) at (0.75,4.5){};
\node[circle, scale=0.75, fill, red] (tid5) at (2.25,4.5){};
\draw[](tid1) -- (tid4);
\draw[](tid1) -- (tid5);
\node[circle, scale=0.75, fill] (tid2) at (3.75,3){};
\node[circle, scale=0.75, fill, red] (tid6) at (3.75,4.5){};
\draw[](tid2) -- (tid6);
\node[circle, scale=0.75, fill] (tid3) at (5.25,3){};
\draw[](tid0) -- (tid1);
\draw[](tid0) -- (tid2);
\draw[](tid0) -- (tid3);
\end{tikzpicture}
\nodepart{two}
\footnotesize{4.2037}
\nodepart{three}
\footnotesize{$33\:67$}
};
 & 
\node[draw=black, rectangle split,  rectangle split parts=3] (sn0xd98db0){
\begin{tikzpicture}[scale=.2]
\node[circle, scale=0.75, fill] (tid0) at (2.25,1.5){};
\node[circle, scale=0.75, fill] (tid1) at (0.75,3){};
\node[circle, scale=0.75, fill, red] (tid4) at (0.75,4.5){};
\draw[](tid1) -- (tid4);
\node[circle, scale=0.75, fill] (tid2) at (2.25,3){};
\node[circle, scale=0.75, fill, red] (tid5) at (2.25,4.5){};
\draw[](tid2) -- (tid5);
\node[circle, scale=0.75, fill] (tid3) at (3.75,3){};
\node[circle, scale=0.75, fill, red] (tid6) at (3.75,4.5){};
\draw[](tid3) -- (tid6);
\draw[](tid0) -- (tid1);
\draw[](tid0) -- (tid2);
\draw[](tid0) -- (tid3);
\end{tikzpicture}
\nodepart{two}
\footnotesize{4.21296}
\nodepart{three}
\footnotesize{$1$}
};
 & 
\\
};
\end{scope}
\begin{scope}[yshift=\leveltopIIIII cm]
\matrix (line5) [column sep=1cm] {
\node[draw=black, rectangle split,  rectangle split parts=3] (sn0xd9a340){
\begin{tikzpicture}[scale=.2]
\node[circle, scale=0.75, fill] (tid0) at (3,1.5){};
\node[circle, scale=0.75, fill] (tid1) at (1.5,3){};
\node[circle, scale=0.75, fill, red] (tid4) at (0.75,4.5){};
\node[circle, scale=0.75, fill, red] (tid5) at (2.25,4.5){};
\draw[](tid1) -- (tid4);
\draw[](tid1) -- (tid5);
\node[circle, scale=0.75, fill, red] (tid2) at (3.75,3){};
\node[circle, scale=0.75, fill] (tid3) at (5.25,3){};
\draw[](tid0) -- (tid1);
\draw[](tid0) -- (tid2);
\draw[](tid0) -- (tid3);
\end{tikzpicture}
\nodepart{two}
\footnotesize{3.85185}
\nodepart{three}
\footnotesize{$33\:67$}
};
 & 
\node[draw=black, rectangle split,  rectangle split parts=3] (sn0xd98be0){
\begin{tikzpicture}[scale=.2]
\node[circle, scale=0.75, fill] (tid0) at (2.25,1.5){};
\node[circle, scale=0.75, fill] (tid1) at (0.75,3){};
\node[circle, scale=0.75, fill, red] (tid4) at (0.75,4.5){};
\draw[](tid1) -- (tid4);
\node[circle, scale=0.75, fill] (tid2) at (2.25,3){};
\node[circle, scale=0.75, fill, red] (tid5) at (2.25,4.5){};
\draw[](tid2) -- (tid5);
\node[circle, scale=0.75, fill, red] (tid3) at (3.75,3){};
\draw[](tid0) -- (tid1);
\draw[](tid0) -- (tid2);
\draw[](tid0) -- (tid3);
\end{tikzpicture}
\nodepart{two}
\footnotesize{3.87963}
\nodepart{three}
\footnotesize{$67\:33$}
};
 & 
\\
};
\end{scope}
\begin{scope}[yshift=\leveltopIIIIII cm]
\matrix (line6) [column sep=1cm] {
\node[draw=black, rectangle split,  rectangle split parts=3] (sn0xd99950){
\begin{tikzpicture}[scale=.2]
\node[circle, scale=0.75, fill] (tid0) at (2.25,1.5){};
\node[circle, scale=0.75, fill] (tid1) at (1.5,3){};
\node[circle, scale=0.75, fill, red] (tid3) at (0.75,4.5){};
\node[circle, scale=0.75, fill, red] (tid4) at (2.25,4.5){};
\draw[](tid1) -- (tid3);
\draw[](tid1) -- (tid4);
\node[circle, scale=0.75, fill, red] (tid2) at (3.75,3){};
\draw[](tid0) -- (tid1);
\draw[](tid0) -- (tid2);
\end{tikzpicture}
\nodepart{two}
\footnotesize{3.66667}
\nodepart{three}
\footnotesize{$33\:67$}
};
 & 
\node[draw=black, rectangle split,  rectangle split parts=3] (sn0xd98b10){
\begin{tikzpicture}[scale=.2]
\node[circle, scale=0.75, fill] (tid0) at (2.25,1.5){};
\node[circle, scale=0.75, fill] (tid1) at (0.75,3){};
\node[circle, scale=0.75, fill, red] (tid4) at (0.75,4.5){};
\draw[](tid1) -- (tid4);
\node[circle, scale=0.75, fill, red] (tid2) at (2.25,3){};
\node[circle, scale=0.75, fill, red] (tid3) at (3.75,3){};
\draw[](tid0) -- (tid1);
\draw[](tid0) -- (tid2);
\draw[](tid0) -- (tid3);
\end{tikzpicture}
\nodepart{two}
\footnotesize{3.44444}
\nodepart{three}
\footnotesize{$67\:33$}
};
 & 
\node[draw=black, rectangle split,  rectangle split parts=3] (sn0xd982f0){
\begin{tikzpicture}[scale=.2]
\node[circle, scale=0.75, fill] (tid0) at (1.5,1.5){};
\node[circle, scale=0.75, fill] (tid1) at (0.75,3){};
\node[circle, scale=0.75, fill, red] (tid3) at (0.75,4.5){};
\draw[](tid1) -- (tid3);
\node[circle, scale=0.75, fill] (tid2) at (2.25,3){};
\node[circle, scale=0.75, fill, red] (tid4) at (2.25,4.5){};
\draw[](tid2) -- (tid4);
\draw[](tid0) -- (tid1);
\draw[](tid0) -- (tid2);
\end{tikzpicture}
\nodepart{two}
\footnotesize{3.75}
\nodepart{three}
\footnotesize{$1$}
};
 & 
\\
};
\end{scope}
\begin{scope}[yshift=\leveltopIIIIIII cm]
\matrix (line7) [column sep=1cm] {
\node[draw=black, rectangle split,  rectangle split parts=3] (sn0xd99040){
\begin{tikzpicture}[scale=.2]
\node[circle, scale=0.75, fill] (tid0) at (1.5,1.5){};
\node[circle, scale=0.75, fill] (tid1) at (1.5,3){};
\node[circle, scale=0.75, fill, red] (tid2) at (0.75,4.5){};
\node[circle, scale=0.75, fill, red] (tid3) at (2.25,4.5){};
\draw[](tid1) -- (tid2);
\draw[](tid1) -- (tid3);
\draw[](tid0) -- (tid1);
\end{tikzpicture}
\nodepart{two}
\footnotesize{3.5}
\nodepart{three}
\footnotesize{$1$}
};
 & 
\node[draw=black, rectangle split,  rectangle split parts=3] (sn0xd981e0){
\begin{tikzpicture}[scale=.2]
\node[circle, scale=0.75, fill] (tid0) at (1.5,1.5){};
\node[circle, scale=0.75, fill] (tid1) at (0.75,3){};
\node[circle, scale=0.75, fill, red] (tid3) at (0.75,4.5){};
\draw[](tid1) -- (tid3);
\node[circle, scale=0.75, fill, red] (tid2) at (2.25,3){};
\draw[](tid0) -- (tid1);
\draw[](tid0) -- (tid2);
\end{tikzpicture}
\nodepart{two}
\footnotesize{3.25}
\nodepart{three}
\footnotesize{$50\:50$}
};
 & 
\node[draw=black, rectangle split,  rectangle split parts=3] (sn0xd985f0){
\begin{tikzpicture}[scale=.2]
\node[circle, scale=0.75, fill] (tid0) at (2.25,1.5){};
\node[circle, scale=0.75, fill, red] (tid1) at (0.75,3){};
\node[circle, scale=0.75, fill, red] (tid2) at (2.25,3){};
\node[circle, scale=0.75, fill, red] (tid3) at (3.75,3){};
\draw[](tid0) -- (tid1);
\draw[](tid0) -- (tid2);
\draw[](tid0) -- (tid3);
\end{tikzpicture}
\nodepart{two}
\footnotesize{2.83333}
\nodepart{three}
\footnotesize{$1$}
};
 & 
\\
};
\end{scope}
\begin{scope}[yshift=\leveltopIIIIIIII cm]
\matrix (line8) [column sep=1cm] {
\node[draw=black, rectangle split,  rectangle split parts=3] (sn0xd97cb0){
\begin{tikzpicture}[scale=.2]
\node[circle, scale=0.75, fill] (tid0) at (0.75,1.5){};
\node[circle, scale=0.75, fill] (tid1) at (0.75,3){};
\node[circle, scale=0.75, fill, red] (tid2) at (0.75,4.5){};
\draw[](tid1) -- (tid2);
\draw[](tid0) -- (tid1);
\end{tikzpicture}
\nodepart{two}
\footnotesize{3}
\nodepart{three}
\footnotesize{$1$}
};
 & 
\node[draw=black, rectangle split,  rectangle split parts=3] (sn0xd97ee0){
\begin{tikzpicture}[scale=.2]
\node[circle, scale=0.75, fill] (tid0) at (1.5,1.5){};
\node[circle, scale=0.75, fill, red] (tid1) at (0.75,3){};
\node[circle, scale=0.75, fill, red] (tid2) at (2.25,3){};
\draw[](tid0) -- (tid1);
\draw[](tid0) -- (tid2);
\end{tikzpicture}
\nodepart{two}
\footnotesize{2.5}
\nodepart{three}
\footnotesize{$1$}
};
 & 
\\
};
\end{scope}
\begin{scope}[yshift=\leveltopIIIIIIIII cm]
\matrix (line9) [column sep=1cm] {
\node[draw=black, rectangle split,  rectangle split parts=3] (sn0xd88140){
\begin{tikzpicture}[scale=.2]
\node[circle, scale=0.75, fill] (tid0) at (0.75,1.5){};
\node[circle, scale=0.75, fill, red] (tid1) at (0.75,3){};
\draw[](tid0) -- (tid1);
\end{tikzpicture}
\nodepart{two}
\footnotesize{2}
\nodepart{three}
\footnotesize{$1$}
};
 & 
\\
};
\end{scope}
\begin{scope}[yshift=\leveltopIIIIIIIIII cm]
\matrix (line10) [column sep=1cm] {
\node[draw=black, rectangle split,  rectangle split parts=3] (sn0xd88070){
\begin{tikzpicture}[scale=.2]
\node[circle, scale=0.75, fill, red] (tid0) at (0.75,1.5){};
\end{tikzpicture}
\nodepart{two}
\footnotesize{1}
\nodepart{three}
\footnotesize{$$}
};
 & 
\\
};
\end{scope}
\begin{scope}[yshift=\leveltopIIIIIIIIIII cm]
\matrix (line11) [column sep=1cm] {
\\
};
\end{scope}
\draw (sn0xda3ca0.south) -- (sn0xda2f30.north);
\draw (sn0xda2f30.south) -- (sn0xd9f7a0.north);
\draw (sn0xda2f30.south) -- (sn0xd9a9c0.north);
\draw (sn0xd9f7a0.south) -- (sn0xd9ae50.north);
\draw (sn0xd9a9c0.south) -- (sn0xd9ae50.north);
\draw (sn0xd9a9c0.south) -- (sn0xd98db0.north);
\draw (sn0xd9ae50.south) -- (sn0xd9a340.north);
\draw (sn0xd9ae50.south) -- (sn0xd98be0.north);
\draw (sn0xd98db0.south) -- (sn0xd98be0.north);
\draw (sn0xd9a340.south) -- (sn0xd99950.north);
\draw (sn0xd9a340.south) -- (sn0xd98b10.north);
\draw (sn0xd98be0.south) -- (sn0xd982f0.north);
\draw (sn0xd98be0.south) -- (sn0xd98b10.north);
\draw (sn0xd99950.south) -- (sn0xd99040.north);
\draw (sn0xd99950.south) -- (sn0xd981e0.north);
\draw (sn0xd98b10.south) -- (sn0xd981e0.north);
\draw (sn0xd98b10.south) -- (sn0xd985f0.north);
\draw (sn0xd982f0.south) -- (sn0xd981e0.north);
\draw (sn0xd99040.south) -- (sn0xd97cb0.north);
\draw (sn0xd981e0.south) -- (sn0xd97cb0.north);
\draw (sn0xd981e0.south) -- (sn0xd97ee0.north);
\draw (sn0xd985f0.south) -- (sn0xd97ee0.north);
\draw (sn0xd97cb0.south) -- (sn0xd88140.north);
\draw (sn0xd97ee0.south) -- (sn0xd88140.north);
\draw (sn0xd88140.south) -- (sn0xd88070.north);
\end{tikzpicture}

%%% Local Variables:
%%% TeX-master: "thesis/thesis.tex"
%%% End: 
\renewcommand{\leveltopI}{-15cm + \leveltop}
\renewcommand{\leveltopII}{-15cm + \leveltopI}
\renewcommand{\leveltopIII}{-15cm + \leveltopII}
\renewcommand{\leveltopIIII}{-15cm + \leveltopIII}
\renewcommand{\leveltopIIIII}{-15cm + \leveltopIIII}
\renewcommand{\leveltopIIIIII}{-15cm + \leveltopIIIII}
\renewcommand{\leveltopIIIIIII}{-15cm + \leveltopIIIIII}
\renewcommand{\leveltopIIIIIIII}{-15cm + \leveltopIIIIIII}
\renewcommand{\leveltopIIIIIIIII}{-15cm + \leveltopIIIIIIII}
\renewcommand{\leveltopIIIIIIIIII}{-15cm + \leveltopIIIIIIIII}
\begin{tikzpicture}[scale=.2, anchor=south]
\begin{scope}[yshift=\leveltopI cm]
\matrix (line1) [column sep=1cm] {
\node[draw=black, rectangle split,  rectangle split parts=3] (sn0xda5120){
\begin{tikzpicture}[scale=.2]
\node[circle, scale=0.75, fill] (tid0) at (4.5,1.5){};
\node[circle, scale=0.75, fill] (tid1) at (2.25,3){};
\node[circle, scale=0.75, fill, red] (tid4) at (0.75,4.5){};
\node[circle, scale=0.75, fill, red] (tid5) at (2.25,4.5){};
\node[circle, scale=0.75, fill] (tid6) at (3.75,4.5){};
\draw[](tid1) -- (tid4);
\draw[](tid1) -- (tid5);
\draw[](tid1) -- (tid6);
\node[circle, scale=0.75, fill] (tid2) at (6,3){};
\node[circle, scale=0.75, fill] (tid7) at (5.25,4.5){};
\node[circle, scale=0.75, fill] (tid8) at (6.75,4.5){};
\draw[](tid2) -- (tid7);
\draw[](tid2) -- (tid8);
\node[circle, scale=0.75, fill] (tid3) at (8.25,3){};
\node[circle, scale=0.75, fill, red] (tid9) at (8.25,4.5){};
\draw[](tid3) -- (tid9);
\draw[](tid0) -- (tid1);
\draw[](tid0) -- (tid2);
\draw[](tid0) -- (tid3);
\end{tikzpicture}
\nodepart{two}
\footnotesize{5.20439}
\nodepart{three}
\footnotesize{$33\:22\:44$}
};
 & 
\\
};
\end{scope}
\begin{scope}[yshift=\leveltopII cm]
\matrix (line2) [column sep=1cm] {
\node[draw=black, rectangle split,  rectangle split parts=3] (sn0xda6020){
\begin{tikzpicture}[scale=.2]
\node[circle, scale=0.75, fill] (tid0) at (4.5,1.5){};
\node[circle, scale=0.75, fill] (tid1) at (2.25,3){};
\node[circle, scale=0.75, fill, red] (tid4) at (0.75,4.5){};
\node[circle, scale=0.75, fill, red] (tid5) at (2.25,4.5){};
\node[circle, scale=0.75, fill] (tid6) at (3.75,4.5){};
\draw[](tid1) -- (tid4);
\draw[](tid1) -- (tid5);
\draw[](tid1) -- (tid6);
\node[circle, scale=0.75, fill] (tid2) at (6,3){};
\node[circle, scale=0.75, fill, red] (tid7) at (5.25,4.5){};
\node[circle, scale=0.75, fill] (tid8) at (6.75,4.5){};
\draw[](tid2) -- (tid7);
\draw[](tid2) -- (tid8);
\node[circle, scale=0.75, fill] (tid3) at (8.25,3){};
\draw[](tid0) -- (tid1);
\draw[](tid0) -- (tid2);
\draw[](tid0) -- (tid3);
\end{tikzpicture}
\nodepart{two}
\footnotesize{4.86831}
\nodepart{three}
\footnotesize{$33\:67$}
};
 & 
\node[draw=black, rectangle split,  rectangle split parts=3] (sn0xda2f30){
\begin{tikzpicture}[scale=.2]
\node[circle, scale=0.75, fill] (tid0) at (3.75,1.5){};
\node[circle, scale=0.75, fill] (tid1) at (1.5,3){};
\node[circle, scale=0.75, fill, red] (tid4) at (0.75,4.5){};
\node[circle, scale=0.75, fill, red] (tid5) at (2.25,4.5){};
\draw[](tid1) -- (tid4);
\draw[](tid1) -- (tid5);
\node[circle, scale=0.75, fill] (tid2) at (4.5,3){};
\node[circle, scale=0.75, fill] (tid6) at (3.75,4.5){};
\node[circle, scale=0.75, fill] (tid7) at (5.25,4.5){};
\draw[](tid2) -- (tid6);
\draw[](tid2) -- (tid7);
\node[circle, scale=0.75, fill] (tid3) at (6.75,3){};
\node[circle, scale=0.75, fill, red] (tid8) at (6.75,4.5){};
\draw[](tid3) -- (tid8);
\draw[](tid0) -- (tid1);
\draw[](tid0) -- (tid2);
\draw[](tid0) -- (tid3);
\end{tikzpicture}
\nodepart{two}
\footnotesize{4.87243}
\nodepart{three}
\footnotesize{$33\:67$}
};
 & 
\node[draw=black, rectangle split,  rectangle split parts=3] (sn0xda0d60){
\begin{tikzpicture}[scale=.2]
\node[circle, scale=0.75, fill] (tid0) at (3.75,1.5){};
\node[circle, scale=0.75, fill] (tid1) at (1.5,3){};
\node[circle, scale=0.75, fill, red] (tid4) at (0.75,4.5){};
\node[circle, scale=0.75, fill] (tid5) at (2.25,4.5){};
\draw[](tid1) -- (tid4);
\draw[](tid1) -- (tid5);
\node[circle, scale=0.75, fill] (tid2) at (4.5,3){};
\node[circle, scale=0.75, fill, red] (tid6) at (3.75,4.5){};
\node[circle, scale=0.75, fill] (tid7) at (5.25,4.5){};
\draw[](tid2) -- (tid6);
\draw[](tid2) -- (tid7);
\node[circle, scale=0.75, fill] (tid3) at (6.75,3){};
\node[circle, scale=0.75, fill, red] (tid8) at (6.75,4.5){};
\draw[](tid3) -- (tid8);
\draw[](tid0) -- (tid1);
\draw[](tid0) -- (tid2);
\draw[](tid0) -- (tid3);
\end{tikzpicture}
\nodepart{two}
\footnotesize{4.87243}
\nodepart{three}
\footnotesize{$33\:67$}
};
 & 
\\
};
\end{scope}
\begin{scope}[yshift=\leveltopIII cm]
\matrix (line3) [column sep=1cm] {
\node[draw=black, rectangle split,  rectangle split parts=3] (sn0xd9cf30){
\begin{tikzpicture}[scale=.2]
\node[circle, scale=0.75, fill] (tid0) at (3.75,1.5){};
\node[circle, scale=0.75, fill] (tid1) at (2.25,3){};
\node[circle, scale=0.75, fill, red] (tid4) at (0.75,4.5){};
\node[circle, scale=0.75, fill, red] (tid5) at (2.25,4.5){};
\node[circle, scale=0.75, fill] (tid6) at (3.75,4.5){};
\draw[](tid1) -- (tid4);
\draw[](tid1) -- (tid5);
\draw[](tid1) -- (tid6);
\node[circle, scale=0.75, fill] (tid2) at (5.25,3){};
\node[circle, scale=0.75, fill, red] (tid7) at (5.25,4.5){};
\draw[](tid2) -- (tid7);
\node[circle, scale=0.75, fill] (tid3) at (6.75,3){};
\draw[](tid0) -- (tid1);
\draw[](tid0) -- (tid2);
\draw[](tid0) -- (tid3);
\end{tikzpicture}
\nodepart{two}
\footnotesize{4.53086}
\nodepart{three}
\footnotesize{$33\:67$}
};
 & 
\node[draw=black, rectangle split,  rectangle split parts=3] (sn0xd9f7a0){
\begin{tikzpicture}[scale=.2]
\node[circle, scale=0.75, fill] (tid0) at (3.75,1.5){};
\node[circle, scale=0.75, fill] (tid1) at (1.5,3){};
\node[circle, scale=0.75, fill, red] (tid4) at (0.75,4.5){};
\node[circle, scale=0.75, fill, red] (tid5) at (2.25,4.5){};
\draw[](tid1) -- (tid4);
\draw[](tid1) -- (tid5);
\node[circle, scale=0.75, fill] (tid2) at (4.5,3){};
\node[circle, scale=0.75, fill, red] (tid6) at (3.75,4.5){};
\node[circle, scale=0.75, fill] (tid7) at (5.25,4.5){};
\draw[](tid2) -- (tid6);
\draw[](tid2) -- (tid7);
\node[circle, scale=0.75, fill] (tid3) at (6.75,3){};
\draw[](tid0) -- (tid1);
\draw[](tid0) -- (tid2);
\draw[](tid0) -- (tid3);
\end{tikzpicture}
\nodepart{two}
\footnotesize{4.53704}
\nodepart{three}
\footnotesize{$1$}
};
 & 
\node[draw=black, rectangle split,  rectangle split parts=3] (sn0xd9a9c0){
\begin{tikzpicture}[scale=.2]
\node[circle, scale=0.75, fill] (tid0) at (3,1.5){};
\node[circle, scale=0.75, fill] (tid1) at (1.5,3){};
\node[circle, scale=0.75, fill, red] (tid4) at (0.75,4.5){};
\node[circle, scale=0.75, fill] (tid5) at (2.25,4.5){};
\draw[](tid1) -- (tid4);
\draw[](tid1) -- (tid5);
\node[circle, scale=0.75, fill] (tid2) at (3.75,3){};
\node[circle, scale=0.75, fill, red] (tid6) at (3.75,4.5){};
\draw[](tid2) -- (tid6);
\node[circle, scale=0.75, fill] (tid3) at (5.25,3){};
\node[circle, scale=0.75, fill, red] (tid7) at (5.25,4.5){};
\draw[](tid3) -- (tid7);
\draw[](tid0) -- (tid1);
\draw[](tid0) -- (tid2);
\draw[](tid0) -- (tid3);
\end{tikzpicture}
\nodepart{two}
\footnotesize{4.54012}
\nodepart{three}
\footnotesize{$67\:33$}
};
 & 
\\
};
\end{scope}
\begin{scope}[yshift=\leveltopIIII cm]
\matrix (line4) [column sep=1cm] {
\node[draw=black, rectangle split,  rectangle split parts=3] (sn0xd9d6c0){
\begin{tikzpicture}[scale=.2]
\node[circle, scale=0.75, fill] (tid0) at (3.75,1.5){};
\node[circle, scale=0.75, fill] (tid1) at (2.25,3){};
\node[circle, scale=0.75, fill, red] (tid4) at (0.75,4.5){};
\node[circle, scale=0.75, fill, red] (tid5) at (2.25,4.5){};
\node[circle, scale=0.75, fill, red] (tid6) at (3.75,4.5){};
\draw[](tid1) -- (tid4);
\draw[](tid1) -- (tid5);
\draw[](tid1) -- (tid6);
\node[circle, scale=0.75, fill] (tid2) at (5.25,3){};
\node[circle, scale=0.75, fill] (tid3) at (6.75,3){};
\draw[](tid0) -- (tid1);
\draw[](tid0) -- (tid2);
\draw[](tid0) -- (tid3);
\end{tikzpicture}
\nodepart{two}
\footnotesize{4.18519}
\nodepart{three}
\footnotesize{$1$}
};
 & 
\node[draw=black, rectangle split,  rectangle split parts=3] (sn0xd9ae50){
\begin{tikzpicture}[scale=.2]
\node[circle, scale=0.75, fill] (tid0) at (3,1.5){};
\node[circle, scale=0.75, fill] (tid1) at (1.5,3){};
\node[circle, scale=0.75, fill, red] (tid4) at (0.75,4.5){};
\node[circle, scale=0.75, fill, red] (tid5) at (2.25,4.5){};
\draw[](tid1) -- (tid4);
\draw[](tid1) -- (tid5);
\node[circle, scale=0.75, fill] (tid2) at (3.75,3){};
\node[circle, scale=0.75, fill, red] (tid6) at (3.75,4.5){};
\draw[](tid2) -- (tid6);
\node[circle, scale=0.75, fill] (tid3) at (5.25,3){};
\draw[](tid0) -- (tid1);
\draw[](tid0) -- (tid2);
\draw[](tid0) -- (tid3);
\end{tikzpicture}
\nodepart{two}
\footnotesize{4.2037}
\nodepart{three}
\footnotesize{$33\:67$}
};
 & 
\node[draw=black, rectangle split,  rectangle split parts=3] (sn0xd98db0){
\begin{tikzpicture}[scale=.2]
\node[circle, scale=0.75, fill] (tid0) at (2.25,1.5){};
\node[circle, scale=0.75, fill] (tid1) at (0.75,3){};
\node[circle, scale=0.75, fill, red] (tid4) at (0.75,4.5){};
\draw[](tid1) -- (tid4);
\node[circle, scale=0.75, fill] (tid2) at (2.25,3){};
\node[circle, scale=0.75, fill, red] (tid5) at (2.25,4.5){};
\draw[](tid2) -- (tid5);
\node[circle, scale=0.75, fill] (tid3) at (3.75,3){};
\node[circle, scale=0.75, fill, red] (tid6) at (3.75,4.5){};
\draw[](tid3) -- (tid6);
\draw[](tid0) -- (tid1);
\draw[](tid0) -- (tid2);
\draw[](tid0) -- (tid3);
\end{tikzpicture}
\nodepart{two}
\footnotesize{4.21296}
\nodepart{three}
\footnotesize{$1$}
};
 & 
\\
};
\end{scope}
\begin{scope}[yshift=\leveltopIIIII cm]
\matrix (line5) [column sep=1cm] {
\node[draw=black, rectangle split,  rectangle split parts=3] (sn0xd9a340){
\begin{tikzpicture}[scale=.2]
\node[circle, scale=0.75, fill] (tid0) at (3,1.5){};
\node[circle, scale=0.75, fill] (tid1) at (1.5,3){};
\node[circle, scale=0.75, fill, red] (tid4) at (0.75,4.5){};
\node[circle, scale=0.75, fill, red] (tid5) at (2.25,4.5){};
\draw[](tid1) -- (tid4);
\draw[](tid1) -- (tid5);
\node[circle, scale=0.75, fill, red] (tid2) at (3.75,3){};
\node[circle, scale=0.75, fill] (tid3) at (5.25,3){};
\draw[](tid0) -- (tid1);
\draw[](tid0) -- (tid2);
\draw[](tid0) -- (tid3);
\end{tikzpicture}
\nodepart{two}
\footnotesize{3.85185}
\nodepart{three}
\footnotesize{$33\:67$}
};
 & 
\node[draw=black, rectangle split,  rectangle split parts=3] (sn0xd98be0){
\begin{tikzpicture}[scale=.2]
\node[circle, scale=0.75, fill] (tid0) at (2.25,1.5){};
\node[circle, scale=0.75, fill] (tid1) at (0.75,3){};
\node[circle, scale=0.75, fill, red] (tid4) at (0.75,4.5){};
\draw[](tid1) -- (tid4);
\node[circle, scale=0.75, fill] (tid2) at (2.25,3){};
\node[circle, scale=0.75, fill, red] (tid5) at (2.25,4.5){};
\draw[](tid2) -- (tid5);
\node[circle, scale=0.75, fill, red] (tid3) at (3.75,3){};
\draw[](tid0) -- (tid1);
\draw[](tid0) -- (tid2);
\draw[](tid0) -- (tid3);
\end{tikzpicture}
\nodepart{two}
\footnotesize{3.87963}
\nodepart{three}
\footnotesize{$67\:33$}
};
 & 
\\
};
\end{scope}
\begin{scope}[yshift=\leveltopIIIIII cm]
\matrix (line6) [column sep=1cm] {
\node[draw=black, rectangle split,  rectangle split parts=3] (sn0xd99950){
\begin{tikzpicture}[scale=.2]
\node[circle, scale=0.75, fill] (tid0) at (2.25,1.5){};
\node[circle, scale=0.75, fill] (tid1) at (1.5,3){};
\node[circle, scale=0.75, fill, red] (tid3) at (0.75,4.5){};
\node[circle, scale=0.75, fill, red] (tid4) at (2.25,4.5){};
\draw[](tid1) -- (tid3);
\draw[](tid1) -- (tid4);
\node[circle, scale=0.75, fill, red] (tid2) at (3.75,3){};
\draw[](tid0) -- (tid1);
\draw[](tid0) -- (tid2);
\end{tikzpicture}
\nodepart{two}
\footnotesize{3.66667}
\nodepart{three}
\footnotesize{$33\:67$}
};
 & 
\node[draw=black, rectangle split,  rectangle split parts=3] (sn0xd98b10){
\begin{tikzpicture}[scale=.2]
\node[circle, scale=0.75, fill] (tid0) at (2.25,1.5){};
\node[circle, scale=0.75, fill] (tid1) at (0.75,3){};
\node[circle, scale=0.75, fill, red] (tid4) at (0.75,4.5){};
\draw[](tid1) -- (tid4);
\node[circle, scale=0.75, fill, red] (tid2) at (2.25,3){};
\node[circle, scale=0.75, fill, red] (tid3) at (3.75,3){};
\draw[](tid0) -- (tid1);
\draw[](tid0) -- (tid2);
\draw[](tid0) -- (tid3);
\end{tikzpicture}
\nodepart{two}
\footnotesize{3.44444}
\nodepart{three}
\footnotesize{$67\:33$}
};
 & 
\node[draw=black, rectangle split,  rectangle split parts=3] (sn0xd982f0){
\begin{tikzpicture}[scale=.2]
\node[circle, scale=0.75, fill] (tid0) at (1.5,1.5){};
\node[circle, scale=0.75, fill] (tid1) at (0.75,3){};
\node[circle, scale=0.75, fill, red] (tid3) at (0.75,4.5){};
\draw[](tid1) -- (tid3);
\node[circle, scale=0.75, fill] (tid2) at (2.25,3){};
\node[circle, scale=0.75, fill, red] (tid4) at (2.25,4.5){};
\draw[](tid2) -- (tid4);
\draw[](tid0) -- (tid1);
\draw[](tid0) -- (tid2);
\end{tikzpicture}
\nodepart{two}
\footnotesize{3.75}
\nodepart{three}
\footnotesize{$1$}
};
 & 
\\
};
\end{scope}
\begin{scope}[yshift=\leveltopIIIIIII cm]
\matrix (line7) [column sep=1cm] {
\node[draw=black, rectangle split,  rectangle split parts=3] (sn0xd99040){
\begin{tikzpicture}[scale=.2]
\node[circle, scale=0.75, fill] (tid0) at (1.5,1.5){};
\node[circle, scale=0.75, fill] (tid1) at (1.5,3){};
\node[circle, scale=0.75, fill, red] (tid2) at (0.75,4.5){};
\node[circle, scale=0.75, fill, red] (tid3) at (2.25,4.5){};
\draw[](tid1) -- (tid2);
\draw[](tid1) -- (tid3);
\draw[](tid0) -- (tid1);
\end{tikzpicture}
\nodepart{two}
\footnotesize{3.5}
\nodepart{three}
\footnotesize{$1$}
};
 & 
\node[draw=black, rectangle split,  rectangle split parts=3] (sn0xd981e0){
\begin{tikzpicture}[scale=.2]
\node[circle, scale=0.75, fill] (tid0) at (1.5,1.5){};
\node[circle, scale=0.75, fill] (tid1) at (0.75,3){};
\node[circle, scale=0.75, fill, red] (tid3) at (0.75,4.5){};
\draw[](tid1) -- (tid3);
\node[circle, scale=0.75, fill, red] (tid2) at (2.25,3){};
\draw[](tid0) -- (tid1);
\draw[](tid0) -- (tid2);
\end{tikzpicture}
\nodepart{two}
\footnotesize{3.25}
\nodepart{three}
\footnotesize{$50\:50$}
};
 & 
\node[draw=black, rectangle split,  rectangle split parts=3] (sn0xd985f0){
\begin{tikzpicture}[scale=.2]
\node[circle, scale=0.75, fill] (tid0) at (2.25,1.5){};
\node[circle, scale=0.75, fill, red] (tid1) at (0.75,3){};
\node[circle, scale=0.75, fill, red] (tid2) at (2.25,3){};
\node[circle, scale=0.75, fill, red] (tid3) at (3.75,3){};
\draw[](tid0) -- (tid1);
\draw[](tid0) -- (tid2);
\draw[](tid0) -- (tid3);
\end{tikzpicture}
\nodepart{two}
\footnotesize{2.83333}
\nodepart{three}
\footnotesize{$1$}
};
 & 
\\
};
\end{scope}
\begin{scope}[yshift=\leveltopIIIIIIII cm]
\matrix (line8) [column sep=1cm] {
\node[draw=black, rectangle split,  rectangle split parts=3] (sn0xd97cb0){
\begin{tikzpicture}[scale=.2]
\node[circle, scale=0.75, fill] (tid0) at (0.75,1.5){};
\node[circle, scale=0.75, fill] (tid1) at (0.75,3){};
\node[circle, scale=0.75, fill, red] (tid2) at (0.75,4.5){};
\draw[](tid1) -- (tid2);
\draw[](tid0) -- (tid1);
\end{tikzpicture}
\nodepart{two}
\footnotesize{3}
\nodepart{three}
\footnotesize{$1$}
};
 & 
\node[draw=black, rectangle split,  rectangle split parts=3] (sn0xd97ee0){
\begin{tikzpicture}[scale=.2]
\node[circle, scale=0.75, fill] (tid0) at (1.5,1.5){};
\node[circle, scale=0.75, fill, red] (tid1) at (0.75,3){};
\node[circle, scale=0.75, fill, red] (tid2) at (2.25,3){};
\draw[](tid0) -- (tid1);
\draw[](tid0) -- (tid2);
\end{tikzpicture}
\nodepart{two}
\footnotesize{2.5}
\nodepart{three}
\footnotesize{$1$}
};
 & 
\\
};
\end{scope}
\begin{scope}[yshift=\leveltopIIIIIIIII cm]
\matrix (line9) [column sep=1cm] {
\node[draw=black, rectangle split,  rectangle split parts=3] (sn0xd88140){
\begin{tikzpicture}[scale=.2]
\node[circle, scale=0.75, fill] (tid0) at (0.75,1.5){};
\node[circle, scale=0.75, fill, red] (tid1) at (0.75,3){};
\draw[](tid0) -- (tid1);
\end{tikzpicture}
\nodepart{two}
\footnotesize{2}
\nodepart{three}
\footnotesize{$1$}
};
 & 
\\
};
\end{scope}
\begin{scope}[yshift=\leveltopIIIIIIIIII cm]
\matrix (line10) [column sep=1cm] {
\node[draw=black, rectangle split,  rectangle split parts=3] (sn0xd88070){
\begin{tikzpicture}[scale=.2]
\node[circle, scale=0.75, fill, red] (tid0) at (0.75,1.5){};
\end{tikzpicture}
\nodepart{two}
\footnotesize{1}
\nodepart{three}
\footnotesize{$$}
};
 & 
\\
};
\end{scope}
\begin{scope}[yshift=\leveltopIIIIIIIIIII cm]
\matrix (line11) [column sep=1cm] {
\\
};
\end{scope}
\draw (sn0xda5120.south) -- (sn0xda6020.north);
\draw (sn0xda5120.south) -- (sn0xda2f30.north);
\draw (sn0xda5120.south) -- (sn0xda0d60.north);
\draw (sn0xda6020.south) -- (sn0xd9cf30.north);
\draw (sn0xda6020.south) -- (sn0xd9f7a0.north);
\draw (sn0xda2f30.south) -- (sn0xd9f7a0.north);
\draw (sn0xda2f30.south) -- (sn0xd9a9c0.north);
\draw (sn0xda0d60.south) -- (sn0xd9f7a0.north);
\draw (sn0xda0d60.south) -- (sn0xd9a9c0.north);
\draw (sn0xd9cf30.south) -- (sn0xd9d6c0.north);
\draw (sn0xd9cf30.south) -- (sn0xd9ae50.north);
\draw (sn0xd9f7a0.south) -- (sn0xd9ae50.north);
\draw (sn0xd9a9c0.south) -- (sn0xd9ae50.north);
\draw (sn0xd9a9c0.south) -- (sn0xd98db0.north);
\draw (sn0xd9d6c0.south) -- (sn0xd9a340.north);
\draw (sn0xd9ae50.south) -- (sn0xd9a340.north);
\draw (sn0xd9ae50.south) -- (sn0xd98be0.north);
\draw (sn0xd98db0.south) -- (sn0xd98be0.north);
\draw (sn0xd9a340.south) -- (sn0xd99950.north);
\draw (sn0xd9a340.south) -- (sn0xd98b10.north);
\draw (sn0xd98be0.south) -- (sn0xd982f0.north);
\draw (sn0xd98be0.south) -- (sn0xd98b10.north);
\draw (sn0xd99950.south) -- (sn0xd99040.north);
\draw (sn0xd99950.south) -- (sn0xd981e0.north);
\draw (sn0xd98b10.south) -- (sn0xd981e0.north);
\draw (sn0xd98b10.south) -- (sn0xd985f0.north);
\draw (sn0xd982f0.south) -- (sn0xd981e0.north);
\draw (sn0xd99040.south) -- (sn0xd97cb0.north);
\draw (sn0xd981e0.south) -- (sn0xd97cb0.north);
\draw (sn0xd981e0.south) -- (sn0xd97ee0.north);
\draw (sn0xd985f0.south) -- (sn0xd97ee0.north);
\draw (sn0xd97cb0.south) -- (sn0xd88140.north);
\draw (sn0xd97ee0.south) -- (sn0xd88140.north);
\draw (sn0xd88140.south) -- (sn0xd88070.north);
\end{tikzpicture}

%%% Local Variables:
%%% TeX-master: "thesis/thesis.tex"
%%% End: 
\renewcommand{\leveltopI}{-15cm + \leveltop}
\renewcommand{\leveltopII}{-15cm + \leveltopI}
\renewcommand{\leveltopIII}{-15cm + \leveltopII}
\renewcommand{\leveltopIIII}{-15cm + \leveltopIII}
\renewcommand{\leveltopIIIII}{-15cm + \leveltopIIII}
\renewcommand{\leveltopIIIIII}{-15cm + \leveltopIIIII}
\renewcommand{\leveltopIIIIIII}{-15cm + \leveltopIIIIII}
\renewcommand{\leveltopIIIIIIII}{-15cm + \leveltopIIIIIII}
\renewcommand{\leveltopIIIIIIIII}{-15cm + \leveltopIIIIIIII}
\renewcommand{\leveltopIIIIIIIIII}{-15cm + \leveltopIIIIIIIII}
\begin{tikzpicture}[scale=.2, anchor=south]
\begin{scope}[yshift=\leveltopI cm]
\matrix (line1) [column sep=1cm] {
\node[draw=black, rectangle split,  rectangle split parts=3] (sn0xda6f40){
\begin{tikzpicture}[scale=.2]
\node[circle, scale=0.75, fill] (tid0) at (4.5,1.5){};
\node[circle, scale=0.75, fill] (tid1) at (2.25,3){};
\node[circle, scale=0.75, fill, red] (tid4) at (0.75,4.5){};
\node[circle, scale=0.75, fill] (tid5) at (2.25,4.5){};
\node[circle, scale=0.75, fill] (tid6) at (3.75,4.5){};
\draw[](tid1) -- (tid4);
\draw[](tid1) -- (tid5);
\draw[](tid1) -- (tid6);
\node[circle, scale=0.75, fill] (tid2) at (6,3){};
\node[circle, scale=0.75, fill, red] (tid7) at (5.25,4.5){};
\node[circle, scale=0.75, fill] (tid8) at (6.75,4.5){};
\draw[](tid2) -- (tid7);
\draw[](tid2) -- (tid8);
\node[circle, scale=0.75, fill] (tid3) at (8.25,3){};
\node[circle, scale=0.75, fill, red] (tid9) at (8.25,4.5){};
\draw[](tid3) -- (tid9);
\draw[](tid0) -- (tid1);
\draw[](tid0) -- (tid2);
\draw[](tid0) -- (tid3);
\end{tikzpicture}
\nodepart{two}
\footnotesize{5.20199}
\nodepart{three}
\footnotesize{$33\:33\:22\:11$}
};
 & 
\\
};
\end{scope}
\begin{scope}[yshift=\leveltopII cm]
\matrix (line2) [column sep=1cm] {
\node[draw=black, rectangle split,  rectangle split parts=3] (sn0xd9fb90){
\begin{tikzpicture}[scale=.2]
\node[circle, scale=0.75, fill] (tid0) at (4.5,1.5){};
\node[circle, scale=0.75, fill] (tid1) at (2.25,3){};
\node[circle, scale=0.75, fill, red] (tid4) at (0.75,4.5){};
\node[circle, scale=0.75, fill] (tid5) at (2.25,4.5){};
\node[circle, scale=0.75, fill] (tid6) at (3.75,4.5){};
\draw[](tid1) -- (tid4);
\draw[](tid1) -- (tid5);
\draw[](tid1) -- (tid6);
\node[circle, scale=0.75, fill] (tid2) at (6,3){};
\node[circle, scale=0.75, fill, red] (tid7) at (5.25,4.5){};
\node[circle, scale=0.75, fill, red] (tid8) at (6.75,4.5){};
\draw[](tid2) -- (tid7);
\draw[](tid2) -- (tid8);
\node[circle, scale=0.75, fill] (tid3) at (8.25,3){};
\draw[](tid0) -- (tid1);
\draw[](tid0) -- (tid2);
\draw[](tid0) -- (tid3);
\end{tikzpicture}
\nodepart{two}
\footnotesize{4.86626}
\nodepart{three}
\footnotesize{$67\:33$}
};
 & 
\node[draw=black, rectangle split,  rectangle split parts=3] (sn0xd9d000){
\begin{tikzpicture}[scale=.2]
\node[circle, scale=0.75, fill] (tid0) at (3.75,1.5){};
\node[circle, scale=0.75, fill] (tid1) at (2.25,3){};
\node[circle, scale=0.75, fill, red] (tid4) at (0.75,4.5){};
\node[circle, scale=0.75, fill] (tid5) at (2.25,4.5){};
\node[circle, scale=0.75, fill] (tid6) at (3.75,4.5){};
\draw[](tid1) -- (tid4);
\draw[](tid1) -- (tid5);
\draw[](tid1) -- (tid6);
\node[circle, scale=0.75, fill] (tid2) at (5.25,3){};
\node[circle, scale=0.75, fill, red] (tid7) at (5.25,4.5){};
\draw[](tid2) -- (tid7);
\node[circle, scale=0.75, fill] (tid3) at (6.75,3){};
\node[circle, scale=0.75, fill, red] (tid8) at (6.75,4.5){};
\draw[](tid3) -- (tid8);
\draw[](tid0) -- (tid1);
\draw[](tid0) -- (tid2);
\draw[](tid0) -- (tid3);
\end{tikzpicture}
\nodepart{two}
\footnotesize{4.86728}
\nodepart{three}
\footnotesize{$67\:33$}
};
 & 
\node[draw=black, rectangle split,  rectangle split parts=3] (sn0xda0d60){
\begin{tikzpicture}[scale=.2]
\node[circle, scale=0.75, fill] (tid0) at (3.75,1.5){};
\node[circle, scale=0.75, fill] (tid1) at (1.5,3){};
\node[circle, scale=0.75, fill, red] (tid4) at (0.75,4.5){};
\node[circle, scale=0.75, fill] (tid5) at (2.25,4.5){};
\draw[](tid1) -- (tid4);
\draw[](tid1) -- (tid5);
\node[circle, scale=0.75, fill] (tid2) at (4.5,3){};
\node[circle, scale=0.75, fill, red] (tid6) at (3.75,4.5){};
\node[circle, scale=0.75, fill] (tid7) at (5.25,4.5){};
\draw[](tid2) -- (tid6);
\draw[](tid2) -- (tid7);
\node[circle, scale=0.75, fill] (tid3) at (6.75,3){};
\node[circle, scale=0.75, fill, red] (tid8) at (6.75,4.5){};
\draw[](tid3) -- (tid8);
\draw[](tid0) -- (tid1);
\draw[](tid0) -- (tid2);
\draw[](tid0) -- (tid3);
\end{tikzpicture}
\nodepart{two}
\footnotesize{4.87243}
\nodepart{three}
\footnotesize{$33\:67$}
};
 & 
\node[draw=black, rectangle split,  rectangle split parts=3] (sn0xda2f30){
\begin{tikzpicture}[scale=.2]
\node[circle, scale=0.75, fill] (tid0) at (3.75,1.5){};
\node[circle, scale=0.75, fill] (tid1) at (1.5,3){};
\node[circle, scale=0.75, fill, red] (tid4) at (0.75,4.5){};
\node[circle, scale=0.75, fill, red] (tid5) at (2.25,4.5){};
\draw[](tid1) -- (tid4);
\draw[](tid1) -- (tid5);
\node[circle, scale=0.75, fill] (tid2) at (4.5,3){};
\node[circle, scale=0.75, fill] (tid6) at (3.75,4.5){};
\node[circle, scale=0.75, fill] (tid7) at (5.25,4.5){};
\draw[](tid2) -- (tid6);
\draw[](tid2) -- (tid7);
\node[circle, scale=0.75, fill] (tid3) at (6.75,3){};
\node[circle, scale=0.75, fill, red] (tid8) at (6.75,4.5){};
\draw[](tid3) -- (tid8);
\draw[](tid0) -- (tid1);
\draw[](tid0) -- (tid2);
\draw[](tid0) -- (tid3);
\end{tikzpicture}
\nodepart{two}
\footnotesize{4.87243}
\nodepart{three}
\footnotesize{$33\:67$}
};
 & 
\\
};
\end{scope}
\begin{scope}[yshift=\leveltopIII cm]
\matrix (line3) [column sep=1cm] {
\node[draw=black, rectangle split,  rectangle split parts=3] (sn0xd9cf30){
\begin{tikzpicture}[scale=.2]
\node[circle, scale=0.75, fill] (tid0) at (3.75,1.5){};
\node[circle, scale=0.75, fill] (tid1) at (2.25,3){};
\node[circle, scale=0.75, fill, red] (tid4) at (0.75,4.5){};
\node[circle, scale=0.75, fill, red] (tid5) at (2.25,4.5){};
\node[circle, scale=0.75, fill] (tid6) at (3.75,4.5){};
\draw[](tid1) -- (tid4);
\draw[](tid1) -- (tid5);
\draw[](tid1) -- (tid6);
\node[circle, scale=0.75, fill] (tid2) at (5.25,3){};
\node[circle, scale=0.75, fill, red] (tid7) at (5.25,4.5){};
\draw[](tid2) -- (tid7);
\node[circle, scale=0.75, fill] (tid3) at (6.75,3){};
\draw[](tid0) -- (tid1);
\draw[](tid0) -- (tid2);
\draw[](tid0) -- (tid3);
\end{tikzpicture}
\nodepart{two}
\footnotesize{4.53086}
\nodepart{three}
\footnotesize{$33\:67$}
};
 & 
\node[draw=black, rectangle split,  rectangle split parts=3] (sn0xd9f7a0){
\begin{tikzpicture}[scale=.2]
\node[circle, scale=0.75, fill] (tid0) at (3.75,1.5){};
\node[circle, scale=0.75, fill] (tid1) at (1.5,3){};
\node[circle, scale=0.75, fill, red] (tid4) at (0.75,4.5){};
\node[circle, scale=0.75, fill, red] (tid5) at (2.25,4.5){};
\draw[](tid1) -- (tid4);
\draw[](tid1) -- (tid5);
\node[circle, scale=0.75, fill] (tid2) at (4.5,3){};
\node[circle, scale=0.75, fill, red] (tid6) at (3.75,4.5){};
\node[circle, scale=0.75, fill] (tid7) at (5.25,4.5){};
\draw[](tid2) -- (tid6);
\draw[](tid2) -- (tid7);
\node[circle, scale=0.75, fill] (tid3) at (6.75,3){};
\draw[](tid0) -- (tid1);
\draw[](tid0) -- (tid2);
\draw[](tid0) -- (tid3);
\end{tikzpicture}
\nodepart{two}
\footnotesize{4.53704}
\nodepart{three}
\footnotesize{$1$}
};
 & 
\node[draw=black, rectangle split,  rectangle split parts=3] (sn0xd9a9c0){
\begin{tikzpicture}[scale=.2]
\node[circle, scale=0.75, fill] (tid0) at (3,1.5){};
\node[circle, scale=0.75, fill] (tid1) at (1.5,3){};
\node[circle, scale=0.75, fill, red] (tid4) at (0.75,4.5){};
\node[circle, scale=0.75, fill] (tid5) at (2.25,4.5){};
\draw[](tid1) -- (tid4);
\draw[](tid1) -- (tid5);
\node[circle, scale=0.75, fill] (tid2) at (3.75,3){};
\node[circle, scale=0.75, fill, red] (tid6) at (3.75,4.5){};
\draw[](tid2) -- (tid6);
\node[circle, scale=0.75, fill] (tid3) at (5.25,3){};
\node[circle, scale=0.75, fill, red] (tid7) at (5.25,4.5){};
\draw[](tid3) -- (tid7);
\draw[](tid0) -- (tid1);
\draw[](tid0) -- (tid2);
\draw[](tid0) -- (tid3);
\end{tikzpicture}
\nodepart{two}
\footnotesize{4.54012}
\nodepart{three}
\footnotesize{$67\:33$}
};
 & 
\\
};
\end{scope}
\begin{scope}[yshift=\leveltopIIII cm]
\matrix (line4) [column sep=1cm] {
\node[draw=black, rectangle split,  rectangle split parts=3] (sn0xd9d6c0){
\begin{tikzpicture}[scale=.2]
\node[circle, scale=0.75, fill] (tid0) at (3.75,1.5){};
\node[circle, scale=0.75, fill] (tid1) at (2.25,3){};
\node[circle, scale=0.75, fill, red] (tid4) at (0.75,4.5){};
\node[circle, scale=0.75, fill, red] (tid5) at (2.25,4.5){};
\node[circle, scale=0.75, fill, red] (tid6) at (3.75,4.5){};
\draw[](tid1) -- (tid4);
\draw[](tid1) -- (tid5);
\draw[](tid1) -- (tid6);
\node[circle, scale=0.75, fill] (tid2) at (5.25,3){};
\node[circle, scale=0.75, fill] (tid3) at (6.75,3){};
\draw[](tid0) -- (tid1);
\draw[](tid0) -- (tid2);
\draw[](tid0) -- (tid3);
\end{tikzpicture}
\nodepart{two}
\footnotesize{4.18519}
\nodepart{three}
\footnotesize{$1$}
};
 & 
\node[draw=black, rectangle split,  rectangle split parts=3] (sn0xd9ae50){
\begin{tikzpicture}[scale=.2]
\node[circle, scale=0.75, fill] (tid0) at (3,1.5){};
\node[circle, scale=0.75, fill] (tid1) at (1.5,3){};
\node[circle, scale=0.75, fill, red] (tid4) at (0.75,4.5){};
\node[circle, scale=0.75, fill, red] (tid5) at (2.25,4.5){};
\draw[](tid1) -- (tid4);
\draw[](tid1) -- (tid5);
\node[circle, scale=0.75, fill] (tid2) at (3.75,3){};
\node[circle, scale=0.75, fill, red] (tid6) at (3.75,4.5){};
\draw[](tid2) -- (tid6);
\node[circle, scale=0.75, fill] (tid3) at (5.25,3){};
\draw[](tid0) -- (tid1);
\draw[](tid0) -- (tid2);
\draw[](tid0) -- (tid3);
\end{tikzpicture}
\nodepart{two}
\footnotesize{4.2037}
\nodepart{three}
\footnotesize{$33\:67$}
};
 & 
\node[draw=black, rectangle split,  rectangle split parts=3] (sn0xd98db0){
\begin{tikzpicture}[scale=.2]
\node[circle, scale=0.75, fill] (tid0) at (2.25,1.5){};
\node[circle, scale=0.75, fill] (tid1) at (0.75,3){};
\node[circle, scale=0.75, fill, red] (tid4) at (0.75,4.5){};
\draw[](tid1) -- (tid4);
\node[circle, scale=0.75, fill] (tid2) at (2.25,3){};
\node[circle, scale=0.75, fill, red] (tid5) at (2.25,4.5){};
\draw[](tid2) -- (tid5);
\node[circle, scale=0.75, fill] (tid3) at (3.75,3){};
\node[circle, scale=0.75, fill, red] (tid6) at (3.75,4.5){};
\draw[](tid3) -- (tid6);
\draw[](tid0) -- (tid1);
\draw[](tid0) -- (tid2);
\draw[](tid0) -- (tid3);
\end{tikzpicture}
\nodepart{two}
\footnotesize{4.21296}
\nodepart{three}
\footnotesize{$1$}
};
 & 
\\
};
\end{scope}
\begin{scope}[yshift=\leveltopIIIII cm]
\matrix (line5) [column sep=1cm] {
\node[draw=black, rectangle split,  rectangle split parts=3] (sn0xd9a340){
\begin{tikzpicture}[scale=.2]
\node[circle, scale=0.75, fill] (tid0) at (3,1.5){};
\node[circle, scale=0.75, fill] (tid1) at (1.5,3){};
\node[circle, scale=0.75, fill, red] (tid4) at (0.75,4.5){};
\node[circle, scale=0.75, fill, red] (tid5) at (2.25,4.5){};
\draw[](tid1) -- (tid4);
\draw[](tid1) -- (tid5);
\node[circle, scale=0.75, fill, red] (tid2) at (3.75,3){};
\node[circle, scale=0.75, fill] (tid3) at (5.25,3){};
\draw[](tid0) -- (tid1);
\draw[](tid0) -- (tid2);
\draw[](tid0) -- (tid3);
\end{tikzpicture}
\nodepart{two}
\footnotesize{3.85185}
\nodepart{three}
\footnotesize{$33\:67$}
};
 & 
\node[draw=black, rectangle split,  rectangle split parts=3] (sn0xd98be0){
\begin{tikzpicture}[scale=.2]
\node[circle, scale=0.75, fill] (tid0) at (2.25,1.5){};
\node[circle, scale=0.75, fill] (tid1) at (0.75,3){};
\node[circle, scale=0.75, fill, red] (tid4) at (0.75,4.5){};
\draw[](tid1) -- (tid4);
\node[circle, scale=0.75, fill] (tid2) at (2.25,3){};
\node[circle, scale=0.75, fill, red] (tid5) at (2.25,4.5){};
\draw[](tid2) -- (tid5);
\node[circle, scale=0.75, fill, red] (tid3) at (3.75,3){};
\draw[](tid0) -- (tid1);
\draw[](tid0) -- (tid2);
\draw[](tid0) -- (tid3);
\end{tikzpicture}
\nodepart{two}
\footnotesize{3.87963}
\nodepart{three}
\footnotesize{$67\:33$}
};
 & 
\\
};
\end{scope}
\begin{scope}[yshift=\leveltopIIIIII cm]
\matrix (line6) [column sep=1cm] {
\node[draw=black, rectangle split,  rectangle split parts=3] (sn0xd99950){
\begin{tikzpicture}[scale=.2]
\node[circle, scale=0.75, fill] (tid0) at (2.25,1.5){};
\node[circle, scale=0.75, fill] (tid1) at (1.5,3){};
\node[circle, scale=0.75, fill, red] (tid3) at (0.75,4.5){};
\node[circle, scale=0.75, fill, red] (tid4) at (2.25,4.5){};
\draw[](tid1) -- (tid3);
\draw[](tid1) -- (tid4);
\node[circle, scale=0.75, fill, red] (tid2) at (3.75,3){};
\draw[](tid0) -- (tid1);
\draw[](tid0) -- (tid2);
\end{tikzpicture}
\nodepart{two}
\footnotesize{3.66667}
\nodepart{three}
\footnotesize{$33\:67$}
};
 & 
\node[draw=black, rectangle split,  rectangle split parts=3] (sn0xd98b10){
\begin{tikzpicture}[scale=.2]
\node[circle, scale=0.75, fill] (tid0) at (2.25,1.5){};
\node[circle, scale=0.75, fill] (tid1) at (0.75,3){};
\node[circle, scale=0.75, fill, red] (tid4) at (0.75,4.5){};
\draw[](tid1) -- (tid4);
\node[circle, scale=0.75, fill, red] (tid2) at (2.25,3){};
\node[circle, scale=0.75, fill, red] (tid3) at (3.75,3){};
\draw[](tid0) -- (tid1);
\draw[](tid0) -- (tid2);
\draw[](tid0) -- (tid3);
\end{tikzpicture}
\nodepart{two}
\footnotesize{3.44444}
\nodepart{three}
\footnotesize{$67\:33$}
};
 & 
\node[draw=black, rectangle split,  rectangle split parts=3] (sn0xd982f0){
\begin{tikzpicture}[scale=.2]
\node[circle, scale=0.75, fill] (tid0) at (1.5,1.5){};
\node[circle, scale=0.75, fill] (tid1) at (0.75,3){};
\node[circle, scale=0.75, fill, red] (tid3) at (0.75,4.5){};
\draw[](tid1) -- (tid3);
\node[circle, scale=0.75, fill] (tid2) at (2.25,3){};
\node[circle, scale=0.75, fill, red] (tid4) at (2.25,4.5){};
\draw[](tid2) -- (tid4);
\draw[](tid0) -- (tid1);
\draw[](tid0) -- (tid2);
\end{tikzpicture}
\nodepart{two}
\footnotesize{3.75}
\nodepart{three}
\footnotesize{$1$}
};
 & 
\\
};
\end{scope}
\begin{scope}[yshift=\leveltopIIIIIII cm]
\matrix (line7) [column sep=1cm] {
\node[draw=black, rectangle split,  rectangle split parts=3] (sn0xd99040){
\begin{tikzpicture}[scale=.2]
\node[circle, scale=0.75, fill] (tid0) at (1.5,1.5){};
\node[circle, scale=0.75, fill] (tid1) at (1.5,3){};
\node[circle, scale=0.75, fill, red] (tid2) at (0.75,4.5){};
\node[circle, scale=0.75, fill, red] (tid3) at (2.25,4.5){};
\draw[](tid1) -- (tid2);
\draw[](tid1) -- (tid3);
\draw[](tid0) -- (tid1);
\end{tikzpicture}
\nodepart{two}
\footnotesize{3.5}
\nodepart{three}
\footnotesize{$1$}
};
 & 
\node[draw=black, rectangle split,  rectangle split parts=3] (sn0xd981e0){
\begin{tikzpicture}[scale=.2]
\node[circle, scale=0.75, fill] (tid0) at (1.5,1.5){};
\node[circle, scale=0.75, fill] (tid1) at (0.75,3){};
\node[circle, scale=0.75, fill, red] (tid3) at (0.75,4.5){};
\draw[](tid1) -- (tid3);
\node[circle, scale=0.75, fill, red] (tid2) at (2.25,3){};
\draw[](tid0) -- (tid1);
\draw[](tid0) -- (tid2);
\end{tikzpicture}
\nodepart{two}
\footnotesize{3.25}
\nodepart{three}
\footnotesize{$50\:50$}
};
 & 
\node[draw=black, rectangle split,  rectangle split parts=3] (sn0xd985f0){
\begin{tikzpicture}[scale=.2]
\node[circle, scale=0.75, fill] (tid0) at (2.25,1.5){};
\node[circle, scale=0.75, fill, red] (tid1) at (0.75,3){};
\node[circle, scale=0.75, fill, red] (tid2) at (2.25,3){};
\node[circle, scale=0.75, fill, red] (tid3) at (3.75,3){};
\draw[](tid0) -- (tid1);
\draw[](tid0) -- (tid2);
\draw[](tid0) -- (tid3);
\end{tikzpicture}
\nodepart{two}
\footnotesize{2.83333}
\nodepart{three}
\footnotesize{$1$}
};
 & 
\\
};
\end{scope}
\begin{scope}[yshift=\leveltopIIIIIIII cm]
\matrix (line8) [column sep=1cm] {
\node[draw=black, rectangle split,  rectangle split parts=3] (sn0xd97cb0){
\begin{tikzpicture}[scale=.2]
\node[circle, scale=0.75, fill] (tid0) at (0.75,1.5){};
\node[circle, scale=0.75, fill] (tid1) at (0.75,3){};
\node[circle, scale=0.75, fill, red] (tid2) at (0.75,4.5){};
\draw[](tid1) -- (tid2);
\draw[](tid0) -- (tid1);
\end{tikzpicture}
\nodepart{two}
\footnotesize{3}
\nodepart{three}
\footnotesize{$1$}
};
 & 
\node[draw=black, rectangle split,  rectangle split parts=3] (sn0xd97ee0){
\begin{tikzpicture}[scale=.2]
\node[circle, scale=0.75, fill] (tid0) at (1.5,1.5){};
\node[circle, scale=0.75, fill, red] (tid1) at (0.75,3){};
\node[circle, scale=0.75, fill, red] (tid2) at (2.25,3){};
\draw[](tid0) -- (tid1);
\draw[](tid0) -- (tid2);
\end{tikzpicture}
\nodepart{two}
\footnotesize{2.5}
\nodepart{three}
\footnotesize{$1$}
};
 & 
\\
};
\end{scope}
\begin{scope}[yshift=\leveltopIIIIIIIII cm]
\matrix (line9) [column sep=1cm] {
\node[draw=black, rectangle split,  rectangle split parts=3] (sn0xd88140){
\begin{tikzpicture}[scale=.2]
\node[circle, scale=0.75, fill] (tid0) at (0.75,1.5){};
\node[circle, scale=0.75, fill, red] (tid1) at (0.75,3){};
\draw[](tid0) -- (tid1);
\end{tikzpicture}
\nodepart{two}
\footnotesize{2}
\nodepart{three}
\footnotesize{$1$}
};
 & 
\\
};
\end{scope}
\begin{scope}[yshift=\leveltopIIIIIIIIII cm]
\matrix (line10) [column sep=1cm] {
\node[draw=black, rectangle split,  rectangle split parts=3] (sn0xd88070){
\begin{tikzpicture}[scale=.2]
\node[circle, scale=0.75, fill, red] (tid0) at (0.75,1.5){};
\end{tikzpicture}
\nodepart{two}
\footnotesize{1}
\nodepart{three}
\footnotesize{$$}
};
 & 
\\
};
\end{scope}
\begin{scope}[yshift=\leveltopIIIIIIIIIII cm]
\matrix (line11) [column sep=1cm] {
\\
};
\end{scope}
\draw (sn0xda6f40.south) -- (sn0xd9fb90.north);
\draw (sn0xda6f40.south) -- (sn0xd9d000.north);
\draw (sn0xda6f40.south) -- (sn0xda0d60.north);
\draw (sn0xda6f40.south) -- (sn0xda2f30.north);
\draw (sn0xd9fb90.south) -- (sn0xd9cf30.north);
\draw (sn0xd9fb90.south) -- (sn0xd9f7a0.north);
\draw (sn0xd9d000.south) -- (sn0xd9cf30.north);
\draw (sn0xd9d000.south) -- (sn0xd9a9c0.north);
\draw (sn0xda0d60.south) -- (sn0xd9f7a0.north);
\draw (sn0xda0d60.south) -- (sn0xd9a9c0.north);
\draw (sn0xda2f30.south) -- (sn0xd9f7a0.north);
\draw (sn0xda2f30.south) -- (sn0xd9a9c0.north);
\draw (sn0xd9cf30.south) -- (sn0xd9d6c0.north);
\draw (sn0xd9cf30.south) -- (sn0xd9ae50.north);
\draw (sn0xd9f7a0.south) -- (sn0xd9ae50.north);
\draw (sn0xd9a9c0.south) -- (sn0xd9ae50.north);
\draw (sn0xd9a9c0.south) -- (sn0xd98db0.north);
\draw (sn0xd9d6c0.south) -- (sn0xd9a340.north);
\draw (sn0xd9ae50.south) -- (sn0xd9a340.north);
\draw (sn0xd9ae50.south) -- (sn0xd98be0.north);
\draw (sn0xd98db0.south) -- (sn0xd98be0.north);
\draw (sn0xd9a340.south) -- (sn0xd99950.north);
\draw (sn0xd9a340.south) -- (sn0xd98b10.north);
\draw (sn0xd98be0.south) -- (sn0xd982f0.north);
\draw (sn0xd98be0.south) -- (sn0xd98b10.north);
\draw (sn0xd99950.south) -- (sn0xd99040.north);
\draw (sn0xd99950.south) -- (sn0xd981e0.north);
\draw (sn0xd98b10.south) -- (sn0xd981e0.north);
\draw (sn0xd98b10.south) -- (sn0xd985f0.north);
\draw (sn0xd982f0.south) -- (sn0xd981e0.north);
\draw (sn0xd99040.south) -- (sn0xd97cb0.north);
\draw (sn0xd981e0.south) -- (sn0xd97cb0.north);
\draw (sn0xd981e0.south) -- (sn0xd97ee0.north);
\draw (sn0xd985f0.south) -- (sn0xd97ee0.north);
\draw (sn0xd97cb0.south) -- (sn0xd88140.north);
\draw (sn0xd97ee0.south) -- (sn0xd88140.north);
\draw (sn0xd88140.south) -- (sn0xd88070.north);
\end{tikzpicture}

%%% Local Variables:
%%% TeX-master: "thesis/thesis.tex"
%%% End: 


\end{document}
