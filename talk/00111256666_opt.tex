\renewcommand{\leveltopI}{-15cm + \leveltop}
\renewcommand{\leveltopII}{-15cm + \leveltopI}
\renewcommand{\leveltopIII}{-15cm + \leveltopII}
\renewcommand{\leveltopIIII}{-15cm + \leveltopIII}
\renewcommand{\leveltopIIIII}{-15cm + \leveltopIIII}
\renewcommand{\leveltopIIIIII}{-15cm + \leveltopIIIII}
\renewcommand{\leveltopIIIIIII}{-15cm + \leveltopIIIIII}
\renewcommand{\leveltopIIIIIIII}{-15cm + \leveltopIIIIIII}
\renewcommand{\leveltopIIIIIIIII}{-15cm + \leveltopIIIIIIII}
\renewcommand{\leveltopIIIIIIIIII}{-15cm + \leveltopIIIIIIIII}
\renewcommand{\leveltopIIIIIIIIIII}{-15cm + \leveltopIIIIIIIIII}
\renewcommand{\leveltopIIIIIIIIIIII}{-15cm + \leveltopIIIIIIIIIII}
\begin{tikzpicture}[scale=.2, anchor=south]
\begin{scope}[yshift=\leveltopI cm]
\matrix (line1)[column sep=1cm] {
\node[draw=black, rectangle split,  rectangle split parts=2] (sn0x9dbbc98){
\begin{tikzpicture}[scale=.2]
\node[circle, scale=0.75, fill] (tid0) at (5.25,1.5){};
\node[circle, scale=0.75, fill] (tid1) at (2.25,3){};
\node[circle, scale=0.75, fill] (tid3) at (0.75,4.5){};
\node[circle, scale=0.75, fill] (tid4) at (2.25,4.5){};
\node[circle, scale=0.75, fill] (tid5) at (3.75,4.5){};
\node[circle, scale=0.75, fill] (tid7) at (3.75,6){};
\draw[](tid5) -- (tid7);
\draw[](tid1) -- (tid3);
\draw[](tid1) -- (tid4);
\draw[](tid1) -- (tid5);
\node[circle, scale=0.75, fill] (tid2) at (7.5,3){};
\node[circle, scale=0.75, fill] (tid6) at (7.5,4.5){};
\node[circle, scale=0.75, fill, task_scheduled] (tid8) at (5.25,6){};
\node[circle, scale=0.75, fill, task_scheduled] (tid9) at (6.75,6){};
\node[circle, scale=0.75, fill, task_scheduled] (tid10) at (8.25,6){};
\node[circle, scale=0.75, fill] (tid11) at (9.75,6){};
\draw[](tid6) -- (tid8);
\draw[](tid6) -- (tid9);
\draw[](tid6) -- (tid10);
\draw[](tid6) -- (tid11);
\draw[](tid2) -- (tid6);
\draw[](tid0) -- (tid1);
\draw[](tid0) -- (tid2);
\end{tikzpicture}
\nodepart{two}
\footnotesize{6.26749}
};
\\
};
\end{scope}
\end{tikzpicture}
%%% Local Variables:
%%% TeX-master: "thesis/thesis.tex"
%%% End: 
